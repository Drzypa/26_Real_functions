\documentclass[12pt]{article}
\usepackage{pmmeta}
\pmcanonicalname{IllustrationOfIntegrationTechniques}
\pmcreated{2013-03-22 17:50:16}
\pmmodified{2013-03-22 17:50:16}
\pmowner{Wkbj79}{1863}
\pmmodifier{Wkbj79}{1863}
\pmtitle{illustration of integration techniques}
\pmrecord{13}{40309}
\pmprivacy{1}
\pmauthor{Wkbj79}{1863}
\pmtype{Example}
\pmcomment{trigger rebuild}
\pmclassification{msc}{26A36}

\usepackage{amssymb}
\usepackage{amsmath}
\usepackage{amsfonts}
\usepackage{pstricks}
\usepackage{psfrag}
\usepackage{graphicx}
\usepackage{amsthm}
%%\usepackage{xypic}

\begin{document}
\PMlinkescapeword{formula}
\PMlinkescapeword{integral}
\PMlinkescapeword{integrals}

The following integral is an example that illustrates many integration techniques.

\textbf{Problem.} Determine the antiderivative of $\sqrt{\tan x}$.

\emph{\PMlinkescapetext{Solution}.}  We start with \PMlinkname{substitution}{IntegrationBySubstitution}:

\begin{align*}
u & =\sqrt{\tan x} \\
u^2 & =\tan x \\
2u \, du & =\sec^2 x \, dx
\end{align*}
Using the Pythagorean identity $\tan^2 x+1=\sec^2 x$, we obtain:
\begin{align*}
2u \, du & =(\tan^2 x+1) \, dx \\
2u \, du & =(u^4+1) \, dx \\
\frac{2u}{u^4+1} \, du & =dx
\end{align*}

Thus,
\begin{align*}
\int\sqrt{\tan x} \, dx & =\int u\,\frac{2u}{u^4+1} \, du \\
& =\int\frac{2u^2}{(u^2-u\sqrt{2}+1)(u^2+u\sqrt{2}+1)} \, du.
\end{align*}

For this last integral, we use the method of \PMlinkname{partial fractions}{ALectureOnThePartialFractionDecompositionMethod}:
\begin{align*}
\frac{2u^2}{(u^2-u\sqrt{2}+1)(u^2+u\sqrt{2}+1)} & =\frac{A+Bu}{u^2-u\sqrt{2}+1}+\frac{C+Du}{u^2+u\sqrt{2}+1} \\ \\
2u^2 & =(A+Bu)(u^2+u\sqrt{2}+1)+(C+Du)(u^2-u\sqrt{2}+1) \\
& =(B\!+\!D)u^3+(A\!+\!C\!+\!(B\!-\!D)\sqrt{2})u^2+(B\!+\!D\!+\!(A\!-\!C)\sqrt{2})u+A\!+\!C
\end{align*}
From this, we obtain the following system of equations:

\[
\left\{
\begin{array}{cccc}
              &   & B+D           & =0 \\
A+C           & + & (B-D)\sqrt{2} & =2 \\
(A-C)\sqrt{2} & + & B+D           & =0 \\
A+C           &   &               & =0
\end{array}
\right.
\]

This can be \PMlinkescapetext{separated} into two smaller systems of equations:

\[
\left\{
\begin{array}{rcrc}
A         & + & C         & =0 \\
A\sqrt{2} & - & C\sqrt{2} & =0
\end{array}
\right.
\]

\[
\left\{
\begin{array}{rcrc}
B         & + & D         & =0 \\
B\sqrt{2} & - & D\sqrt{2} & =2
\end{array}
\right.
\]

It is clear that the first system yields $A=C=0$, and it can easily be verified that $B=\frac{1}{\sqrt{2}}$ and $D=\frac{-1}{\sqrt{2}}$.  Therefore,
\begin{align*}
\int\sqrt{\tan x} \, dx & =\frac{1}{\sqrt{2}}\int\frac{u}{u^2-u\sqrt{2}+1} \, du-\frac{1}{\sqrt{2}}\int\frac{u}{u^2+u\sqrt{2}+1} \, du \\
& =\frac{1}{\sqrt{2}}\int\frac{u}{u^2-u\sqrt{2}+\frac{1}{2}+\frac{1}{2}} \, du-\frac{1}{\sqrt{2}}\int\frac{u}{u^2+u\sqrt{2}+\frac{1}{2}+\frac{1}{2}} \, du \\
& =\frac{1}{\sqrt{2}}\int\frac{u}{(u-\frac{1}{\sqrt{2}})^2+\frac{1}{2}} \, du-\frac{1}{\sqrt{2}}\int\frac{u}{(u+\frac{1}{\sqrt{2}})^2+\frac{1}{2}} \, du.
\end{align*}

Now we make the following substitutions:

\begin{center}
$\begin{array}{rclcrcl}
v  & = & u-\frac{1}{\sqrt{2}} & \bigskip & w  & = & u+\frac{1}{\sqrt{2}} \\
dv & = & du                   &          & dw & = & du
\end{array}$
\end{center}

Note that we have $v+\frac{1}{\sqrt{2}}=u=w-\frac{1}{\sqrt{2}}$.  Therefore,
\begin{align*}
\int\sqrt{\tan x} \, dx & =\frac{1}{\sqrt{2}}\int\frac{v+\frac{1}{\sqrt{2}}}{v^2+\frac{1}{2}} \, dv-\frac{1}{\sqrt{2}}\int\frac{w-\frac{1}{\sqrt{2}}}{w^2+\frac{1}{2}} \, dw \\
& =\frac{1}{\sqrt{2}}\int\frac{v}{v^2+\frac{1}{2}} \, dv-\frac{1}{2}\int\frac{dv}{v^2+\frac{1}{2}}-\frac{1}{\sqrt{2}}\int\frac{w}{w^2+\frac{1}{2}} \, dw+\frac{1}{2}\int\frac{dw}{w^2+\frac{1}{2}}.
\end{align*}

For the first and third integrals in the last expression, note that the numerator is a \PMlinkescapetext{constant multiple} of the derivative of the denominator.  For these, we use the formula
\[
\int\frac{kf'(x)}{f(x)} \, dx=k\ln|f(x)|.
\]
For the second and fourth integrals in the last expression, we use the formula
\[
\int\frac{dx}{x^2+a^2}=\frac{1}{a}\arctan\left(\frac{x}{a}\right)
\]
with $a=\frac{1}{\sqrt{2}}$.  Hence,
\begin{align*}
\int\sqrt{\tan x} \, dx & =\frac{1}{2\sqrt{2}}\ln\left(v^2+\frac{1}{2}\right)+\frac{1}{\sqrt{2}}\arctan(v\sqrt{2}) -\frac{1}{2\sqrt{2}}\ln\left(w^2+\frac{1}{2}\right)+\frac{1}{\sqrt{2}}\arctan(w\sqrt{2})+K \\
& =\frac{1}{2\sqrt{2}}\ln\left(\frac{v^2+\frac{1}{2}}{w^2+\frac{1}{2}}\right) +\frac{1}{\sqrt{2}}(\arctan(v\sqrt{2})+\arctan(w\sqrt{2}))+K \\
& =\frac{1}{2\sqrt{2}}\ln\left(\frac{(u-\frac{1}{\sqrt{2}})^2+\frac{1}{2}}{(u+\frac{1}{\sqrt{2}})^2+\frac{1}{2}}\right) +\frac{1}{\sqrt{2}}\left(\!\arctan\left[\!\left(\!u-\frac{1}{\sqrt{2}}\!\right)\!\sqrt{2}\right]\! +\!\arctan\left[\!\left(\!u+\frac{1}{\sqrt{2}}\!\right)\!\sqrt{2}\right]\!\right)\!+\!K \\
& =\frac{1}{2\sqrt{2}}\ln\left(\frac{u^2-u\sqrt{2}+1}{u^2+u\sqrt{2}+1}\right) +\frac{1}{\sqrt{2}}[\arctan(u\sqrt{2}-1)+\arctan(u\sqrt{2}+1)]+K \\
& =\frac{1}{2\sqrt{2}}\ln\left(\frac{\tan x-\sqrt{2\tan x}+1}{\tan x+\sqrt{2\tan x}+1}\right) +\frac{1}{\sqrt{2}}[\arctan(\sqrt{2\tan x}-1)+\arctan(\sqrt{2\tan x}+1)]+K.
\end{align*}

(We use $K$ for the constant of integration to avoid confusion with $C$ from the system of equations.)
%%%%%
%%%%%
\end{document}
