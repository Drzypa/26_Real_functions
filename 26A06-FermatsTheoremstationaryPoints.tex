\documentclass[12pt]{article}
\usepackage{pmmeta}
\pmcanonicalname{FermatsTheoremstationaryPoints}
\pmcreated{2013-03-22 13:45:05}
\pmmodified{2013-03-22 13:45:05}
\pmowner{paolini}{1187}
\pmmodifier{paolini}{1187}
\pmtitle{Fermat's theorem (stationary points)}
\pmrecord{7}{34450}
\pmprivacy{1}
\pmauthor{paolini}{1187}
\pmtype{Theorem}
\pmcomment{trigger rebuild}
\pmclassification{msc}{26A06}
\pmrelated{ProofOfLeastAndReatestValueOfFunction}
\pmrelated{LeastAndGreatestValueOfFunction}

% this is the default PlanetMath preamble.  as your knowledge
% of TeX increases, you will probably want to edit this, but
% it should be fine as is for beginners.

% almost certainly you want these
\usepackage{amssymb}
\usepackage{amsmath}
\usepackage{amsfonts}

% used for TeXing text within eps files
%\usepackage{psfrag}
% need this for including graphics (\includegraphics)
%\usepackage{graphicx}
% for neatly defining theorems and propositions
%\usepackage{amsthm}
% making logically defined graphics
%%%\usepackage{xypic}

% there are many more packages, add them here as you need them

% define commands here
\begin{document}
Let $f\colon (a,b)\to \mathbb R$ be a continuous function and suppose that 
$x_0\in (a,b)$ is a local extremum of $f$. If $f$ is differentiable in $x_0$ then $f'(x_0)=0$.

Moreover if $f$ has a local maximum at $a$ and $f$ is differentiable at $a$ (the right derivative exists) 
then $f'(a)\le 0$; if $f$ has a local minimum at $a$ then $f'(a)\ge 0$. 
If $f$ is differentiable in $b$ and
has a local maximum at $b$ then $f'(b)\ge 0$ while if it has a local minimum at $b$ then $f'(b)\le 0$.
%%%%%
%%%%%
\end{document}
