\documentclass[12pt]{article}
\usepackage{pmmeta}
\pmcanonicalname{IntegrationOfDifferentialBinomial}
\pmcreated{2013-03-22 14:45:49}
\pmmodified{2013-03-22 14:45:49}
\pmowner{rspuzio}{6075}
\pmmodifier{rspuzio}{6075}
\pmtitle{integration of differential binomial}
\pmrecord{5}{36408}
\pmprivacy{1}
\pmauthor{rspuzio}{6075}
\pmtype{Theorem}
\pmcomment{trigger rebuild}
\pmclassification{msc}{26A36}

\endmetadata

% this is the default PlanetMath preamble.  as your knowledge
% of TeX increases, you will probably want to edit this, but
% it should be fine as is for beginners.

% almost certainly you want these
\usepackage{amssymb}
\usepackage{amsmath}
\usepackage{amsfonts}

% used for TeXing text within eps files
%\usepackage{psfrag}
% need this for including graphics (\includegraphics)
%\usepackage{graphicx}
% for neatly defining theorems and propositions
%\usepackage{amsthm}
% making logically defined graphics
%%%\usepackage{xypic}

% there are many more packages, add them here as you need them

% define commands here
\begin{document}
\textbf{Theorem.}\, Let $a$, $b$, $c$, $\alpha$, $\beta$ be given real numbers and \,$\alpha\beta \neq 0$. \,The antiderivative
$$I = \int x^a(\alpha+\beta x^b)^c\,dx$$
is expressible by \PMlinkescapetext{means} of the elementary functions only in the three cases: \,
$(1)\,\, \frac{a+1}{b}+c\in\mathbb{Z}$,\,\,
$(2)\,\, \frac{a+1}{b}\in\mathbb{Z}$,\,\,
$(3)\,\, c\in\mathbb{Z}$

In accordance with P. L. Chebyshev (1821$-$1894), who has proven this theorem, the expression \,$x^a(\alpha+\beta x^b)^c\,dx$\, is called a {\em differential binomial}.

It may be worth noting that the differential binomial may be expressed in terms of the incomplete beta function and the hypergeometric function.  Define $y = \beta x^b / \alpha$.  Then we have
 $$I = {1 \over b} \alpha^{{a + 1 \over b} + c} \beta^{-{a + 1 \over b}} B_y \left( {1 + a \over b}, c - 1 \right)$$
 $$= {1 \over 1 + a} \alpha^{{a + 1 \over b} + c} \beta^{-{a + 1 \over b}} y^{1 + a \over b} F \left( {a + 1 \over b}, 2-c; {1 + a + b \over b}; y \right)$$

Chebyshev's theorem then follows from the theorem on elementary cases of the hypergeometric function.
%%%%%
%%%%%
\end{document}
