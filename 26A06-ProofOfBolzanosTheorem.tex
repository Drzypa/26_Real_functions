\documentclass[12pt]{article}
\usepackage{pmmeta}
\pmcanonicalname{ProofOfBolzanosTheorem}
\pmcreated{2013-03-22 15:43:29}
\pmmodified{2013-03-22 15:43:29}
\pmowner{cvalente}{11260}
\pmmodifier{cvalente}{11260}
\pmtitle{proof of Bolzano's theorem}
\pmrecord{7}{37674}
\pmprivacy{1}
\pmauthor{cvalente}{11260}
\pmtype{Proof}
\pmcomment{trigger rebuild}
\pmclassification{msc}{26A06}

\endmetadata

% this is the default PlanetMath preamble.  as your knowledge
% of TeX increases, you will probably want to edit this, but
% it should be fine as is for beginners.

% almost certainly you want these
\usepackage{amssymb}
\usepackage{amsmath}
\usepackage{amsfonts}

% used for TeXing text within eps files
%\usepackage{psfrag}
% need this for including graphics (\includegraphics)
%\usepackage{graphicx}
% for neatly defining theorems and propositions
%\usepackage{amsthm}
% making logically defined graphics
%%%\usepackage{xypic}

% there are many more packages, add them here as you need them

% define commands here
\begin{document}
Consider the compact interval $[a,b], a<b$ and a continuous real valued function $f$. If $f(a).f(b)<0$ then there exists $c\in(a,b)$ such that $f(c)=0$

WLOG consider $f(a)<0$ and $f(b)>0$. The other case can be proved using $-f(x)$ which will also verify the theorem's conditions.

consider $a_1 = \frac{a+b}{2}$, three cases can occur:

\begin{itemize}
\item $f(a_1)=0$, in this case the theorem is proved $c=a_1$
\item $f(a_1)>0$, in this case consider the interval $I_1 = (a,a_1)$
\item $f(a_1)<0$, in this case consider the interval $I_1 = (a_1,b)$
\end{itemize}

so starting with an open interval $I_0 = (a,b)$ we get another open interval $I_1 \subset I_0$ with length half of the original $|I_1| = \frac{|I_0|}{2}$.

Repeat the procedure to the interval $I_n$ and get another interval $I_{n+1}$.

We can thus define a succession of open intervals $I_n$ such that $I_{n+1} \subset I_n$, $|I_n|=2^{-n}|I_0|$, such that $I_n = (a_n,b_n)$ and $f(a_n)<0<f(b_n)$.

The succession $c_{2n} = a_n, c_{2n+1}=b_n$ is Cauchy by construction since $m>n \implies |c_m-c_n|<2^{-[n/2]}|I_0|$.

$c_n$ is therefore convergent $c_n\to c\in [a,b]$, and since $a_n$ and $b_n$ are sub-successions, they converge to the same limit. 

$f$ is continuous in $[a,b]$ so $x_n \to x \implies f(x_n) \to f(x)$

By construction

$f(a_n)<0$ and $f(b_n)>0$ so in the limit $\lim_{n \to \infty} f(a_n) = f(\lim_{n \to \infty} a_n)= f(c)\le 0$ and $\lim_{n \to \infty} f(b_n) = f(c) \ge 0$.

So there exists $c \in [a,b]$ such that $0 \le f(c) \le 0 \implies f(c)=0$.

But since $f(a).f(b)<0$, neither $f(a)=0$ nor $f(b)=0$ and since $f(c)=0$, $c \in (a,b)$
%%%%%
%%%%%
\end{document}
