\documentclass[12pt]{article}
\usepackage{pmmeta}
\pmcanonicalname{Extremum}
\pmcreated{2013-03-22 12:59:44}
\pmmodified{2013-03-22 12:59:44}
\pmowner{bshanks}{153}
\pmmodifier{bshanks}{153}
\pmtitle{extremum}
\pmrecord{20}{33373}
\pmprivacy{1}
\pmauthor{bshanks}{153}
\pmtype{Definition}
\pmcomment{trigger rebuild}
\pmclassification{msc}{26B12}
\pmsynonym{extrema}{Extremum}
\pmrelated{Plateau}
\pmrelated{RelationsBetweenHessianMatrixAndLocalExtrema}
\pmrelated{LeastAndGreatestValueOfFunction}
\pmrelated{SpeediestInclinedPlane}
\pmdefines{global minima}
\pmdefines{global maxima}
\pmdefines{local minima}
\pmdefines{local maxima}
\pmdefines{global minimum}
\pmdefines{global maximum}
\pmdefines{local minimum}
\pmdefines{local maximum}
\pmdefines{strict local minima}
\pmdefines{strict local maxima}
\pmdefines{strict local minimum}
\pmdefines{strict local maximum}
\pmdefines{saddle point}

% this is the default PlanetMath preamble.  as your knowledge
% of TeX increases, you will probably want to edit this, but
% it should be fine as is for beginners.

% almost certainly you want these
\usepackage{amssymb}
\usepackage{amsmath}
\usepackage{amsfonts}

% used for TeXing text within eps files
%\usepackage{psfrag}
% need this for including graphics (\includegraphics)
%\usepackage{graphicx}
% for neatly defining theorems and propositions
%\usepackage{amsthm}
% making logically defined graphics
%%%\usepackage{xypic}

% there are many more packages, add them here as you need them

% define commands here
\begin{document}
Extrema are minima and maxima. The \PMlinkescapetext{singular} forms of these \PMlinkescapetext{words} are extremum, minimum, and maximum. 

Extrema may be ``global'' or ``local''. A global minimum of a function $f$ is the lowest value that $f$ ever achieves. If you imagine the function as a surface, then a global minimum is the lowest point on that surface. Formally, it is said that $f\colon U \to V$ has a \emph{global minimum} at $x$ if $\forall u \in U, f(x) \leq f(u)$. 

A local minimum of a function $f$ is a point $x$ which has less value than all points ``next to'' it. If you imagine the function as a surface, then a local minimum is the \PMlinkescapetext{bottom} of a ``valley'' or ``bowl'' in the surface somewhere. Formally, it is said that $f\colon U \to V$ has a \emph{local minimum} at $x$ if $\exists$ a neighborhood $N$ of $x$ such that $\forall y \in N$, $f(x) \leq f(y)$.

If you flip the $\leq$ signs above to $\geq$, you get the definitions of global and local maxima.

A ``strict local minima'' or ``strict local maxima'' means that nearby points are strictly less than or strictly greater than the critical point, rather than $\leq$ or $\geq$. For instance, a strict local minima at $x$ has a neighborhood $N$ such that $\forall y \in N$, $(f(x) < f(y) \textrm{ or } y = x)$.  

A \emph{saddle point} is a critical point which is not a local extremum. 

A related concept is  plateau.

Finding minima or maxima is an important task which is part of the \PMlinkescapetext{field} of optimization.
This task is also important in Physics where the minima correspond to equilibria.
%%%%%
%%%%%
\end{document}
