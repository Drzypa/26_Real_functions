\documentclass[12pt]{article}
\usepackage{pmmeta}
\pmcanonicalname{AverageValueOfFunction}
\pmcreated{2013-03-22 19:01:54}
\pmmodified{2013-03-22 19:01:54}
\pmowner{pahio}{2872}
\pmmodifier{pahio}{2872}
\pmtitle{average value of function}
\pmrecord{12}{41904}
\pmprivacy{1}
\pmauthor{pahio}{2872}
\pmtype{Definition}
\pmcomment{trigger rebuild}
\pmclassification{msc}{26D15}
\pmclassification{msc}{11-00}
\pmrelated{ArithmeticMean}
\pmrelated{Mean3}
\pmrelated{Countable}
\pmrelated{GaussMeanValueTheorem}
\pmrelated{Expectation}
\pmrelated{MeanSquareDeviation}
\pmdefines{average value}

\endmetadata

% this is the default PlanetMath preamble.  as your knowledge
% of TeX increases, you will probably want to edit this, but
% it should be fine as is for beginners.

% almost certainly you want these
\usepackage{amssymb}
\usepackage{amsmath}
\usepackage{amsfonts}

% used for TeXing text within eps files
%\usepackage{psfrag}
% need this for including graphics (\includegraphics)
%\usepackage{graphicx}
% for neatly defining theorems and propositions
 \usepackage{amsthm}
% making logically defined graphics
%%%\usepackage{xypic}

% there are many more packages, add them here as you need them

% define commands here

\theoremstyle{definition}
\newtheorem*{thmplain}{Theorem}

\begin{document}
The set of the values of a real function $f$ defined on an interval \,$[a,\,b]$\, is usually uncountable, and therefore for being able to speak of an \emph{average value} of $f$ in the sense of the average value 
\begin{align}
A.V. \;=\; \frac{a_1\!+\!a_2\!+\ldots+\!a_n}{n} \;=\; \frac{\sum_{j=1}^na_j}{\sum_{j=1}^n1}
\end{align}
of a finite list \,$a_1,\,a_2,\,\ldots,\, a_n$\, of numbers, one has to replace the sums with integrals.\, Thus one could define
$$A.V.(f) \;:=\; \frac{\int_a^b\!f(x)\,dx}{\int_a^b\!1\,dx},$$
i.e.
\begin{align}
A.V.(f) \;:=\; \frac{1}{b\!-\!a}\int_a^b\!f(x)\,dx.
\end{align}
For example, the average value of $x^2$ on the interval \,$[0,\,1]$\, is $\frac{1}{3}$ and the average value of $\sin{x}$
on the interval \,$[0,\,\pi]$\, is $\frac{2}{\pi}$.\\

The definition (2) may be extended to complex functions $f$ on an arc $\gamma$ of a rectifiable curve via the contour integral
\begin{align}
A.V.(f) \;:=\; \frac{1}{l(\gamma)}\int_\gamma\!f(z)\,dz
\end{align}
where $l(\gamma)$ is the \PMlinkname{length}{ArcLength} of the arc.\, If especially $\gamma$ is a closed curve in a simply connected domain where $f$ is analytic, then the average value of $f$ on $\gamma$ is always 0, as the Cauchy integral theorem implies.


%%%%%
%%%%%
\end{document}
