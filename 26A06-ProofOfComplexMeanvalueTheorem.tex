\documentclass[12pt]{article}
\usepackage{pmmeta}
\pmcanonicalname{ProofOfComplexMeanvalueTheorem}
\pmcreated{2013-03-22 14:34:39}
\pmmodified{2013-03-22 14:34:39}
\pmowner{Wolfgang}{5320}
\pmmodifier{Wolfgang}{5320}
\pmtitle{proof of complex mean-value theorem}
\pmrecord{21}{36136}
\pmprivacy{1}
\pmauthor{Wolfgang}{5320}
\pmtype{Proof}
\pmcomment{trigger rebuild}
\pmclassification{msc}{26A06}

% this is the default PlanetMath preamble.  as your knowledge
% of TeX increases, you will probably want to edit this, but
% it should be fine as is for beginners.

% almost certainly you want these
\usepackage{amssymb}
\usepackage{amsmath}
\usepackage{amsfonts}

% used for TeXing text within eps files
%\usepackage{psfrag}
% need this for including graphics (\includegraphics)
%\usepackage{graphicx}
% for neatly defining theorems and propositions
%\usepackage{amsthm}
% making logically defined graphics
%%%\usepackage{xypic}

% there are many more packages, add them here as you need them

% define commands here
\begin{document}
\def\Re{\operatorname{Re}}
\def\Im{\operatorname{Im}}
The function $h(t)=\Re \frac{f(a+t(b-a))-f(a)}{b-a}$ is a function defined on [0,1].
We have $h(0)=0$ and $h(1)=\Re\frac{f(b)-f(a)}{b-a}$.
By the ordinary mean-value theorem, there is a number $t$, $0<t<1$, such that $h'(t)=h(1)-h(0)$.
To evaluate $h'(t)$, we use the assumption that $f$ is complex differentiable (holomorphic). The derivative of $\frac{f(a+t(b-a))-f(a)}{b-a}$ is equal to $f'(a+t(b-a))$, then $h'(t) =\Re(f'(a+t(b-a)))$, so $u=a+t(b-a)$ satisfies the required equation.
The proof of the second assertion can be deduced from the result just proved by applying it to the function f multiplied by i.
%%%%%
%%%%%
\end{document}
