\documentclass[12pt]{article}
\usepackage{pmmeta}
\pmcanonicalname{HessianMatrix}
\pmcreated{2013-03-22 12:59:41}
\pmmodified{2013-03-22 12:59:41}
\pmowner{cvalente}{11260}
\pmmodifier{cvalente}{11260}
\pmtitle{Hessian matrix}
\pmrecord{31}{33370}
\pmprivacy{1}
\pmauthor{cvalente}{11260}
\pmtype{Definition}
\pmcomment{trigger rebuild}
\pmclassification{msc}{26B12}
%\pmkeywords{second derivative}
\pmrelated{Gradient}
\pmrelated{PartialDerivative}
\pmrelated{SymmetricMatrix}
\pmrelated{ComplexHessianMatrix}
\pmrelated{HessianForm}
\pmrelated{DirectionalDerivative}
\pmdefines{Hessian}

\endmetadata

% this is the default PlanetMath preamble.  as your knowledge
% of TeX increases, you will probably want to edit this, but
% it should be fine as is for beginners.

% almost certainly you want these
\usepackage{amssymb}
\usepackage{amsmath}
\usepackage{amsfonts}

% used for TeXing text within eps files
%\usepackage{psfrag}
% need this for including graphics (\includegraphics)
%\usepackage{graphicx}
% for neatly defining theorems and propositions
%\usepackage{amsthm}
% making logically defined graphics
%%%\usepackage{xypic}

% there are many more packages, add them here as you need them

% define commands here
\begin{document}
Let $x \in \mathbb{R}^n$ and let $f\colon\mathbb{R}^n\to\mathbb{R}$ be a real-valued function having 2nd-order partial derivatives in an open set $U$ containing $x$.  The \emph{Hessian matrix} of $f$ is the matrix of second partial derivatives evaluated at $x$: 

\begin{equation}
\mathbf{H}(x):=
\begin{bmatrix}
\displaystyle{\frac{\partial^2 f}{\partial x_1^2}} & \displaystyle{\frac{\partial^2 f}{\partial x_1\partial x_2}} & \ldots & \displaystyle{\frac{\partial^2 f}{\partial x_1\partial x_n}}
\\ \displaystyle{\frac{\partial^2 f}{\partial x_2\partial x_1}} & \displaystyle{\frac{\partial^2 f}{\partial x_2^2}} & \ldots & \displaystyle{\frac{\partial^2 f}{\partial x_2\partial x_n}} 
\\ \vdots & \vdots & \ddots & \vdots
\\ \displaystyle{\frac{\partial^2 f}{\partial x_n\partial x_1}} & \displaystyle{\frac{\partial^2 f}{\partial x_n\partial x_2}} & \ldots & \displaystyle{\frac{\partial^2 f}{\partial x_n^2}}
\end{bmatrix}.
\end{equation}

If  $f$ is in $C^2(U)$, $\mathbf{H}(x)$ is \PMlinkname{symmetric}{SymmetricMatrix} because of the equality of mixed partials.  Note that $\mathbf{H}(x)=\mathbf{J}(\nabla f)$, the Jacobian of the gradient of $f$.

Given  a vector $\boldsymbol{v}\in\mathbb{R}^n$, the \emph{Hessian} of $f$ at $\boldsymbol{v}$ is:
\begin{equation}
\mathbf{H}(x)(\boldsymbol{v}):=\frac{1}{2}\boldsymbol{v}\mathbf{H}(x)\boldsymbol{v}^{\operatorname{T}}.
\end{equation}
Here we view $\boldsymbol{v}$ as a $1$ by $n$ matrix so that $\boldsymbol{v}^{\operatorname{T}}$ is the transpose of $\boldsymbol{v}$.  

\textbf{Remark}.  The Hessian of $f$ at $\boldsymbol{v}$ is a quadratic form, since $\mathbf{H}(x)(r\boldsymbol{v})=r^2\mathbf{H}(x)(\boldsymbol{v})$ for any $r\in\mathbb{R}$. 

 If $f$ is further assumed to be in $C^2(U)$, and $x$ is a critical point of $f$ such that $\mathbf{H}(x)$ is \PMlinkname{positive definite}{PositiveDefinite}, then 
$x$ is a strict local minimum of $f$.

This is not difficult to show. Since $\mathbf{H}(x)$ is \PMlinkname{positive definite}{PositiveDefinite}, the Rayleigh-Ritz theorem shows that there is a  $c > 0$ such that for all $h \in \mathbb{R}^n$, 
$h^T\mathbf{H}(x)h \ge 2c \Vert h\Vert^2$. Thus by
\PMlinkname{Taylor's theorem}{TaylorPolynomialsInBanachSpaces} (\PMlinkescapetext{weaker} form)
$$
f(x + h ) = f(x) + \frac{1}{2} h^T\mathbf{H}(x)h + o(\Vert h \Vert^2) \ge c \Vert h \Vert^2 + o(\Vert h\Vert^2).$$
For small $\Vert h \Vert$ the first \PMlinkescapetext{term} on the \PMlinkescapetext{right dominates} the second, so that both sides are positive for small $\Vert h\Vert$.
%%%%%
%%%%%
\end{document}
