\documentclass[12pt]{article}
\usepackage{pmmeta}
\pmcanonicalname{GeneralisedNdimensionalRiemannSum}
\pmcreated{2013-03-22 13:37:40}
\pmmodified{2013-03-22 13:37:40}
\pmowner{vernondalhart}{2191}
\pmmodifier{vernondalhart}{2191}
\pmtitle{Generalised N-dimensional Riemann Sum}
\pmrecord{4}{34270}
\pmprivacy{1}
\pmauthor{vernondalhart}{2191}
\pmtype{Definition}
\pmcomment{trigger rebuild}
\pmclassification{msc}{26B12}

% this is the default PlanetMath preamble.  as your knowledge
% of TeX increases, you will probably want to edit this, but
% it should be fine as is for beginners.

% almost certainly you want these
\usepackage{amssymb}
\usepackage{amsmath}
\usepackage{amsfonts}

% used for TeXing text within eps files
%\usepackage{psfrag}
% need this for including graphics (\includegraphics)
%\usepackage{graphicx}
% for neatly defining theorems and propositions
%\usepackage{amsthm}
% making logically defined graphics
%%%\usepackage{xypic}

% there are many more packages, add them here as you need them

% define commands here

\newcommand{\R}{\mathbb{R}}
\begin{document}
Let $I = [a_1, b_1] \times \cdots \times [a_N, b_N]$ be an $N$-cell in $\R^N$. For each $j = 1, \ldots, N$, let $a_j = t_{j,0} < \ldots < t_{j,N} = b_j$ be a partition $P_j$ of $[a_j, b_j]$. We define a partition $P$ of $I$ as
\[
P := P_1 \times \cdots \times P_N
\]
Each partition $P$ of $I$ generates a subdivision of $I$ (denoted by $(I_\nu)_\nu$) of the form
\[
I_\nu = [t_{1,j},t_{1,j+1}]\times \cdots \times [t_{N,k},t_{N,k+1}]
\]

Let $f : U \to \R^M$ be such that $I \subset U$, and let $(I_\nu)_\nu$ be the corresponding subdivision of a partition $P$ of $I$. For each $\nu$, choose $x_\nu \in I_\nu$. Define
\[
S(f,P) := \sum_\nu f(x_\nu)\mu(I\nu)
\]
As the Riemann sum of $f$ corresponding to the partition $P$.

A partition $Q$ of $I$ is called a refinement of $P$ if $P \subset Q$.
%%%%%
%%%%%
\end{document}
