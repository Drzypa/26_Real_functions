\documentclass[12pt]{article}
\usepackage{pmmeta}
\pmcanonicalname{LocalMinimumOfConvexFunctionIsNecessarilyGlobal}
\pmcreated{2013-03-22 13:33:37}
\pmmodified{2013-03-22 13:33:37}
\pmowner{stevecheng}{10074}
\pmmodifier{stevecheng}{10074}
\pmtitle{local minimum of convex function is necessarily global}
\pmrecord{13}{34165}
\pmprivacy{1}
\pmauthor{stevecheng}{10074}
\pmtype{Theorem}
\pmcomment{trigger rebuild}
\pmclassification{msc}{26B25}
\pmsynonym{extremal value of convex/concave functions}{LocalMinimumOfConvexFunctionIsNecessarilyGlobal}
\pmsynonym{local maximum of concave function is necessarily global}{LocalMinimumOfConvexFunctionIsNecessarilyGlobal}

\endmetadata

% this is the default PlanetMath preamble.  as your knowledge
% of TeX increases, you will probably want to edit this, but
% it should be fine as is for beginners.

% almost certainly you want these
\usepackage{amssymb}
\usepackage{amsmath}
\usepackage{amsfonts}
\usepackage{amsthm}

% used for TeXing text within eps files
%\usepackage{psfrag}
% need this for including graphics (\includegraphics)
\usepackage{graphicx}
% for neatly defining theorems and propositions
%\usepackage{amsthm}
% making logically defined graphics
%%%\usepackage{xypic}

% there are many more packages, add them here as you need them

% define commands here

\newtheorem{thm}{Theorem}
\begin{document}
\begin{thm}
A local minimum (resp. local maximum) of a convex function
(resp. concave function) on a
convex subset of a topological vector space, is always a global extremum.

\begin{proof}
Let $f:S\to \mathbb{R}$ be a convex function
on a convex set $S$ in a topological vector space.

Suppose $x$ is a local minimum for $f$;
that is, there is an open neighborhood $U$ of $x$
where $f(x) \leq f(\xi)$  for all $\xi\in U$.
We prove $f(x) \leq f(y)$ for arbitrary $y \in S$.

Consider the convex combination $(1-t)x + ty$ for $0 \leq t \leq 1$:
\begin{figure}[!htb]
\begin{center}
  \includegraphics{convex-minimum.eps}
\end{center}
\end{figure}
Since scalar multiplication and vector addition are, by definition,
continuous in a topological vector space, the convex combination approaches
$x$ as $t \to 0$.  Therefore for small enough $t$,
$(1-t)x + ty$ is in the neighborhood $U$.
Then
\begin{align*}
f(x) &\leq f\bigl(ty + (1-t) x\bigr)  & \text{for small $t > 0$} \\
&\leq t \, f(y) + (1-t) f(x) & \text{since $f$ is convex.}
\end{align*}
Rearranging \PMlinkescapetext{terms}, we have $f(x) \leq f(y)$.

To show the analogous situation for a concave function $f$,
the above reasoning can be applied after replacing $f$ with $-f$.
\end{proof}
\end{thm}

%%%%%
%%%%%
\end{document}
