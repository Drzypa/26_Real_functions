\documentclass[12pt]{article}
\usepackage{pmmeta}
\pmcanonicalname{DerivationOfGeometricMeanAsTheLimitOfThePowerMean}
\pmcreated{2013-03-22 14:17:13}
\pmmodified{2013-03-22 14:17:13}
\pmowner{Mathprof}{13753}
\pmmodifier{Mathprof}{13753}
\pmtitle{derivation of geometric mean as the limit of the power mean}
\pmrecord{8}{35741}
\pmprivacy{1}
\pmauthor{Mathprof}{13753}
\pmtype{Derivation}
\pmcomment{trigger rebuild}
\pmclassification{msc}{26D15}
\pmrelated{LHpitalsRule}
\pmrelated{PowerMean}
\pmrelated{WeightedPowerMean}
\pmrelated{ArithmeticGeometricMeansInequality}
\pmrelated{ArithmeticMean}
\pmrelated{GeometricMean}
\pmrelated{DerivationOfZerothWeightedPowerMean}

\endmetadata

% this is the default PlanetMath preamble.  as your knowledge
% of TeX increases, you will probably want to edit this, but
% it should be fine as is for beginners.

% almost certainly you want these
\usepackage{amssymb}
\usepackage{amsmath}
\usepackage{amsfonts}

% used for TeXing text within eps files
%\usepackage{psfrag}
% need this for including graphics (\includegraphics)
%\usepackage{graphicx}
% for neatly defining theorems and propositions
%\usepackage{amsthm}
% making logically defined graphics
%%%\usepackage{xypic}

% there are many more packages, add them here as you need them

% define commands here

\newtheorem{theorem}{Theorem}
\newtheorem{defn}{Definition}
\newtheorem{prop}{Proposition}
\newtheorem{lemma}{Lemma}
\newtheorem{cor}{Corollary}
\begin{document}
\PMlinkescapeword{fix}
\PMlinkescapeword{calculate}
Fix $x_1, x_2, \ldots, x_n \in \mathbb{R}^+$.  Then let 
\[
\mu(r) := \left(\frac{x_1^r+\cdots+x_n^r}{n}\right)^{1/r}.
\]

For $r\neq 0$, by definition $\mu(r)$ is the $r$th power mean of the $x_i$.  It is also clear that $\mu(r)$ is a differentiable function for $r\neq 0$.  What is $\lim_{r\to 0} \mu(r)$?  

We will first calculate $\lim_{r\to 0} \log\mu(r)$ using \PMlinkname{l'H\^opital's rule}{LHpitalsRule}.
\begin{align*}
\lim_{r\to 0} \log\mu(r) & = \lim_{r\to 0} \frac{\log\left(\frac{x_1^r+\cdots +x_n^r}{n}\right)}{r}\\
& = \lim_{r\to 0} \frac{\left(\frac{x_1^r\log x_1+\cdots+x_n^r\log x_n}{n}\right)}{\left(\frac{x_1^r+\cdots+x_n^r}{n}\right)}\\
& = \lim_{r\to 0} \frac{x_1^r\log x_1+\cdots+x_n^r\log x_n}{x_1^r+\cdots+x_n^r}\\
& = \frac{\log x_1+\cdots+\log x_n}{n}\\
& = \log \sqrt[n]{x_1\cdots x_n}.
\end{align*}

It follows immediately that 
\[
\lim_{r\to 0} \left(\frac{x_1^r+\cdots+x_n^r}{n}\right)^{1/r} = \sqrt[n]{x_1\cdots x_n}.
\]
%%%%%
%%%%%
\end{document}
