\documentclass[12pt]{article}
\usepackage{pmmeta}
\pmcanonicalname{PartialFractionsOfExpressions}
\pmcreated{2013-03-22 14:20:27}
\pmmodified{2013-03-22 14:20:27}
\pmowner{pahio}{2872}
\pmmodifier{pahio}{2872}
\pmtitle{partial fractions of expressions}
\pmrecord{29}{35812}
\pmprivacy{1}
\pmauthor{pahio}{2872}
\pmtype{Definition}
\pmcomment{trigger rebuild}
\pmclassification{msc}{26C15}
\pmsynonym{partial fractions}{PartialFractionsOfExpressions}
%\pmkeywords{multiplicity}
\pmrelated{ALectureOnThePartialFractionDecompositionMethod}
\pmrelated{PartialFractionsForPolynomials}
\pmrelated{ConjugatedRootsOfEquation2}
\pmrelated{MixedFraction}
\pmdefines{fractional expression}

\endmetadata

% this is the default PlanetMath preamble.  as your knowledge
% of TeX increases, you will probably want to edit this, but
% it should be fine as is for beginners.

% almost certainly you want these
\usepackage{amssymb}
\usepackage{amsmath}
\usepackage{amsfonts}

% used for TeXing text within eps files
%\usepackage{psfrag}
% need this for including graphics (\includegraphics)
%\usepackage{graphicx}
% for neatly defining theorems and propositions
%\usepackage{amsthm}
% making logically defined graphics
%%%\usepackage{xypic}

% there are many more packages, add them here as you need them

% define commands here
\begin{document}
\PMlinkescapeword{decomposition}

Let\, $R(z) = \frac{P(z)}{Q(z)}$\, be a {\em fractional expression}, i.e., a quotient of the polynomials $P(z)$ and $Q(z)$ such that $P(z)$ is not divisible by $Q(z)$.\, Let's restrict to the case that the coefficients are real or complex numbers. 

If the distinct complex zeros of the denominator are\, $b_1,\,b_2,\,\ldots,\,b_t$\, with the multiplicities\, $\tau_1,\,\tau_2,\,\ldots,\,\tau_t$ ($t \ge 1$), and the numerator has not common zeros, then $R(z)$ can be decomposed uniquely as the sum
$$R(z) \;=\; H(z)+
\sum_{j=1}^t\left(\frac{A_{j1}}{z-b_j}+\frac{A_{j2}}{(z-b_j)^2}+\ldots
        +\frac{A_{j\tau_j}}{(z-b_j)^{\tau_j}}\right),$$
where $H(z)$ is a polynomial and the $A_{jk}$'s are certain complex numbers.

Let us now take the special case that all coefficients of $P(z)$ and $Q(z)$ are real.\, Then the \PMlinkescapetext{{\em imaginary}} (i.e. non-real) zeros of $Q(z)$ are pairwise complex conjugates, with same multiplicities, and the corresponding linear \PMlinkname{factors}{Product} of $Q(z)$ may be pairwise multiplied to quadratic polynomials of the form\, $z^2\!+\!pz\!+\!q$\, with real $p$'s and $q$'s and\, $p^2 < 4q$.\, Hence the above decomposition leads to the unique decomposition of the form
\begin{align*}R(x) \;=\quad & H(x)+
\sum_{i=1}^m\left(\frac{A_{i1}}{x-b_i}+\frac{A_{i2}}{(x-b_i)^2}+\ldots
+\frac{A_{i\mu_i}}{(x-b_i)^{\mu_i}}\right)\\
&+\sum_{j=1}^n\left(\frac{B_{j1}x+C_{j1}}{x^2+p_jx+q_j}+
\frac{B_{j2}x+C_{j2}}{( x^2+p_jx+q_j)^2}+\ldots
+\frac{B_{j\nu_j}x+C_{j\nu_j}}{( x^2+p_jx+q_j)^{\nu_j}}\right),
\end{align*}
where $m$ is the number of the distinct real zeros and $2n$ the number of the distinct \PMlinkescapetext{imaginary} zeros of the denominator $Q(x)$ of the fractional expression\, $R(x) = \frac{P(x)}{Q(x)}$.\, The coefficients $A_{ik}$, $B_{jk}$ and $C_{jk}$ are uniquely determined real numbers.

Cf. the partial fractions of {\em fractional numbers}.\\

\textbf{Example.} 
$$\frac{-x^5\!+\!6x^4\!-\!7x^3\!+\!15x^2\!-\!4x\!+\!3}
{(x\!-\!1)^3(x^2\!+\!1)^2} \;=\;
-\frac{1}{x\!-\!1}\!+\!\frac{3}{(x\!-\!1)^3}\!+
\!\frac{x}{x^2\!+\!1}\!+\!\frac{2x\!-\!1}{(x^2\!+\!1)^2}$$
%%%%%
%%%%%
\end{document}
