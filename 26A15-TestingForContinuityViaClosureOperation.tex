\documentclass[12pt]{article}
\usepackage{pmmeta}
\pmcanonicalname{TestingForContinuityViaClosureOperation}
\pmcreated{2013-03-22 19:09:11}
\pmmodified{2013-03-22 19:09:11}
\pmowner{CWoo}{3771}
\pmmodifier{CWoo}{3771}
\pmtitle{testing for continuity via closure operation}
\pmrecord{8}{42057}
\pmprivacy{1}
\pmauthor{CWoo}{3771}
\pmtype{Result}
\pmcomment{trigger rebuild}
\pmclassification{msc}{26A15}
\pmclassification{msc}{54C05}

\endmetadata

\usepackage{amssymb,amscd}
\usepackage{amsmath}
\usepackage{amsfonts}
\usepackage{mathrsfs}

% used for TeXing text within eps files
%\usepackage{psfrag}
% need this for including graphics (\includegraphics)
%\usepackage{graphicx}
% for neatly defining theorems and propositions
\usepackage{amsthm}
% making logically defined graphics
%%\usepackage{xypic}
\usepackage{pst-plot}

% define commands here
\newcommand*{\abs}[1]{\left\lvert #1\right\rvert}
\newtheorem{prop}{Proposition}
\newtheorem{thm}{Theorem}
\newtheorem{ex}{Example}
\newcommand{\real}{\mathbb{R}}
\newcommand{\pdiff}[2]{\frac{\partial #1}{\partial #2}}
\newcommand{\mpdiff}[3]{\frac{\partial^#1 #2}{\partial #3^#1}}
\begin{document}
\begin{prop} Let $X,Y$ be topological spaces, and $f:X\to Y$ a function.  Then the following are equivalent: 
\begin{enumerate}
\item $f$ is continuous,
\item for any closed set $D\subseteq Y$, the set $f^{-1}(D)$ is closed in $X$,
\item $f(\overline{A})\subseteq \overline{f(A)}$, where $\overline{A}$ is the closure of $A$,
\item $\overline{f^{-1}(B)} \subseteq f^{-1}(\overline{B})$,
\item $f^{-1}(C^{\circ}) \subseteq f^{-1}(C)^{\circ}$, where $C^{\circ}$ is the interior of $C$.
\end{enumerate}
\end{prop}

\begin{proof}
\begin{itemize}
\item $(1)\Leftrightarrow (2)$.  Use the identity $f^{-1}(A-B)=f^{-1}(A)-f^{-1}(B)$ for any function $f$.  Then $f^{-1}(Y-D)=X-f^{-1}(D)$.  So if $D$ is closed (or open), $f^{-1}(Y-D)$ is open (or closed), whence $f^{-1}(D)$ is closed (or open).
\item $(2)\Leftrightarrow (3)$.  Suppose first that $f:X\to Y$ is continuous.  Since $$\overline{f(A)}=\bigcap \lbrace C \mid C\mbox{ closed in }Y,\mbox{ and }f(A)\subseteq C\rbrace,$$ $A\subseteq f^{-1}f(A)\subseteq f^{-1}(C)$, which is closed in $X$.  So $\overline{A} \subseteq f^{-1}(C)$, and therefore $f(\overline{A})\subseteq ff^{-1}(C)\subseteq C$.  As a result, $$f(\overline{A})\subseteq \bigcap \lbrace C \mid C\mbox{ closed in }Y,\mbox{ and }f(A)\subseteq C\rbrace = \overline{f(A)}.$$

Conversely, let $V$ be closed in $Y$.  Then $\overline{V}=V$.  Let $U=f^{-1}(V)$.  So $f(U)=V$.  Let $W=\overline{U}$.  Then $f(W)=f(\overline{U})\subseteq \overline{f(U)}=\overline{V}=V$.  So $W\subseteq f^{-1}f(W)\subseteq f^{-1}(V)=U \subseteq \overline{U}=W$.  As a result, $U=W$ is closed.

\item $(3)\Leftrightarrow (4)$.  First, assume $(2)$.  Let $B\subseteq Y$ and $A=f^{-1}(B)$.  So $f(A)\subseteq B$.  Then $f(\overline{A})\subseteq \overline{f(A)} \subseteq  \overline{B}$.  As a result, $\overline{f^{-1}(B)}=\overline{A}\subseteq f^{-1}f(\overline{A})\subseteq f^{-1}(\overline{B})$.

Conversely, assume $(3)$.  Let $A\subseteq X$ and $B=f(A)$.  So $A\subseteq f^{-1}(B)$.  Then $$f(\overline{A}) \subseteq f(\overline{f^{-1}(B)}) \subseteq ff^{-1}(\overline{B})\subseteq \overline{B} =\overline{f(A)}.$$

\item $(4)\Leftrightarrow (5)$.  First, assume $(3)$.  We use the identity: $C^{\circ}=Y-\overline{Y-C}$.  Then 
\begin{eqnarray*}
f^{-1}(C^{\circ}) &=& f^{-1}(Y-\overline{Y-C})=f^{-1}(Y)-f^{-1}(\overline{Y-C})\subseteq X-\overline{f^{-1}(Y-C)} \\ &=& X-\overline{f^{-1}(Y)-f^{-1}(C)}=X-\overline{X-f^{-1}(C)}=f^{-1}(C)^{\circ}.
\end{eqnarray*}

Conversely, assume $(4)$.  We use the identity $\overline{B}=Y-(Y-B)^{\circ}$.  Then 
\begin{eqnarray*}
\overline{f^{-1}(B)} &=& X-(X-f^{-1}(B))^{\circ} = X-f^{-1}(Y-B)^{\circ} \\ &\subseteq& X-f^{-1}((Y-B)^{\circ}) =f^{-1}(Y-(Y-B)^{\circ})=f^{-1}(\overline{B}).
\end{eqnarray*}
\end{itemize}
\end{proof}
%%%%%
%%%%%
\end{document}
