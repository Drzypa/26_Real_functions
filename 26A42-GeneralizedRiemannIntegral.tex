\documentclass[12pt]{article}
\usepackage{pmmeta}
\pmcanonicalname{GeneralizedRiemannIntegral}
\pmcreated{2013-03-22 13:40:03}
\pmmodified{2013-03-22 13:40:03}
\pmowner{rspuzio}{6075}
\pmmodifier{rspuzio}{6075}
\pmtitle{generalized Riemann integral}
\pmrecord{12}{34328}
\pmprivacy{1}
\pmauthor{rspuzio}{6075}
\pmtype{Definition}
\pmcomment{trigger rebuild}
\pmclassification{msc}{26A42}
\pmsynonym{Kurzweil-Henstock integral}{GeneralizedRiemannIntegral}
\pmsynonym{gauge integral}{GeneralizedRiemannIntegral}
\pmdefines{generalized Riemann integrable}
\pmdefines{gauge}

\endmetadata

% this is the default PlanetMath preamble.  as your knowledge
% of TeX increases, you will probably want to edit this, but
% it should be fine as is for beginners.

% almost certainly you want these
\usepackage{amssymb}
\usepackage{amsmath}
\usepackage{amsfonts}

% used for TeXing text within eps files
%\usepackage{psfrag}
% need this for including graphics (\includegraphics)
\usepackage{graphicx}
% for neatly defining theorems and propositions
%\usepackage{amsthm}
% making logically defined graphics
%%%\usepackage{xypic}

% there are many more packages, add them here as you need them

% define commands here
\begin{document}
A \emph{gauge} $\delta$ is a function which assigns to every real number $x$ an interval $\delta (x)$ such that $x \in \delta (x)$.

Given a gauge $\delta$, a partition ${U_i}_{i=1}^n$ of an interval $[a,b]$ is said to be $\delta$-fine if, for every point $x \in [a,b]$, the set $U_i$ containing $x$ is a subset of $\delta (x)$

A function $f : [a, b] \rightarrow \mathbb{R}$ is said to be \textbf{generalized Riemann integrable} on $[a,b]$ if there exists a number $L \in \mathbb{R}$ such that for every $\epsilon > 0$ there exists a gauge $\delta_{\epsilon}$ on $[a,b]$ such that if $\dot{\mathcal{P}}$ is any $\delta_{\epsilon}$-fine partition of $[a,b]$, then
\[| S(f ; \dot{\mathcal{P}}) - L | < \epsilon,\]
where $S(f ; \dot{\mathcal{P}})$ is any Riemann sum for $f$ using the partition $\dot{\mathcal{P}}$. The collection of all generalized Riemann integrable functions is usually denoted by $\mathcal{R}^{*}[a,b]$.

If $f \in \mathcal{R}^{*}[a,b]$ then the number $L$ is uniquely determined, and is called the \textbf{generalized Riemann integral} of $f$ over $[a,b]$.

The reason that this is called a generalized Riemann integral is that, in the special case where $\delta (x) = [x - y, x + y]$ for some number $y$, we recover the Riemann integral as a special case.

\begin{figure}[!htb]
\begin{center}
\includegraphics{riemann.eps}
\caption{Riemann sum over a $\delta$-fine partition}
\end{center}
\end{figure}


%%%%%
%%%%%
\end{document}
