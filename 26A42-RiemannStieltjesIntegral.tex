\documentclass[12pt]{article}
\usepackage{pmmeta}
\pmcanonicalname{RiemannStieltjesIntegral}
\pmcreated{2013-03-22 12:51:13}
\pmmodified{2013-03-22 12:51:13}
\pmowner{Mathprof}{13753}
\pmmodifier{Mathprof}{13753}
\pmtitle{Riemann-Stieltjes integral}
\pmrecord{11}{33187}
\pmprivacy{1}
\pmauthor{Mathprof}{13753}
\pmtype{Definition}
\pmcomment{trigger rebuild}
\pmclassification{msc}{26A42}
\pmrelated{RiemannSum}
\pmrelated{IntegralSign}
\pmdefines{Riemann-Stieltjes sum}
\pmdefines{integrator}

% this is the default PlanetMath preamble.  as your knowledge
% of TeX increases, you will probably want to edit this, but
% it should be fine as is for beginners.

% almost certainly you want these
\usepackage{amssymb}
\usepackage{amsmath}
\usepackage{amsfonts}

% used for TeXing text within eps files
%\usepackage{psfrag}
% need this for including graphics (\includegraphics)
%\usepackage{graphicx}
% for neatly defining theorems and propositions
%\usepackage{amsthm}
% making logically defined graphics
%%%\usepackage{xypic}

% there are many more packages, add them here as you need them

% define commands here
\begin{document}
Let $f$ and $\alpha$ be bounded, real-valued functions defined upon a closed finite interval $I = [ a, b ]$ of $\mathbb{R} (a \neq b)$, $P = \{ x_{0}, ..., x_{n} \}$ a partition of $I$, and $t_{i}$ a point of the subinterval $[ x_{i - 1}, x_{i} ]$. A sum of the form

$$S(P, f, \alpha) = \sum_{i = 1}^{n} f(t_{i}) (\alpha(x_{i}) - \alpha(x_{i - 1}))$$

is called a \textbf{Riemann-Stieltjes sum} of $f$ with respect to $\alpha$. $f$ is said to be \textbf{Riemann Stieltjes integrable with respect to} $\alpha$ on $I$ if there exists $A \in \mathbb{R}$ such that given any $\epsilon > 0$ there exists a partition $P_{\epsilon}$ of $I$ for which, for all $P$ finer than $P_{\epsilon}$ and for every choice of points $t_{i}$, we have


$$|S(P, f, \alpha) - A| < \epsilon$$


If such an $A$ exists, then it is unique and is known as the \textbf{Riemann-Stieltjes integral of $f$ with respect to $\alpha$}. $f$ is known as the \textbf{integrand} and $\alpha$ the \textbf{integrator}. The integral is denoted by


$$\int_{a}^{b}fd\alpha \quad \textrm{or} \quad \int_{a}^{b}f(x)d\alpha(x)$$

%%%%%
%%%%%
\end{document}
