\documentclass[12pt]{article}
\usepackage{pmmeta}
\pmcanonicalname{ProofOfMonotonicityCriterion}
\pmcreated{2013-03-22 13:45:14}
\pmmodified{2013-03-22 13:45:14}
\pmowner{paolini}{1187}
\pmmodifier{paolini}{1187}
\pmtitle{proof of monotonicity criterion}
\pmrecord{6}{34454}
\pmprivacy{1}
\pmauthor{paolini}{1187}
\pmtype{Proof}
\pmcomment{trigger rebuild}
\pmclassification{msc}{26A06}

% this is the default PlanetMath preamble.  as your knowledge
% of TeX increases, you will probably want to edit this, but
% it should be fine as is for beginners.

% almost certainly you want these
\usepackage{amssymb}
\usepackage{amsmath}
\usepackage{amsfonts}

% used for TeXing text within eps files
%\usepackage{psfrag}
% need this for including graphics (\includegraphics)
%\usepackage{graphicx}
% for neatly defining theorems and propositions
%\usepackage{amsthm}
% making logically defined graphics
%%%\usepackage{xypic}

% there are many more packages, add them here as you need them

% define commands here
\begin{document}
Let us start from the implications ``$\Rightarrow$''.

Suppose that $f'(x)\ge 0$ for all $x\in (a,b)$. We want to prove that therefore $f$ is increasing. So take $x_1,x_2\in [a,b]$ with $x_1<x_2$. Applying the mean-value Theorem on the interval $[x_1,x_2]$ we know that there exists a point $x\in (x_1,x_2)$ such that
\[
  f(x_2)-f(x_1) = f'(x) (x_2-x_1)
\]
and being $f'(x)\ge 0$ we conclude that $f(x_2)\ge f(x_1)$.

This proves the first claim. The other three cases can be achieved with minor modifications: replace all ``$\ge$'' respectively with $\le$, $>$ and $<$.

Let us now prove the implication ``$\Leftarrow$'' for the first and second statement.

Given $x\in (a,b)$ consider the ratio
\[
  \frac{f(x+h)-f(x)}{h}.
\]
If $f$ is increasing the numerator of this ratio is $\ge 0$ when $h>0$ and is $\le 0$ when $h<0$. Anyway the ratio is $\ge 0$ since the denominator has the same sign of the numerator. Since we know by hypothesis that the function $f$ is differentiable in $x$ we can pass to the limit to conclude that 
\[
f'(x)=\lim_{h\to 0} \frac{f(x+h)-f(x)}{h} \ge 0.
\]

If $f$ is decreasing the ratio considered turns out to be $\le 0$ hence the conclusion $f'(x)\le 0$.

Notice that if we suppose that $f$ is strictly increasing we obtain the this ratio is $>0$, but passing to the limit as $h\to 0$ we cannot conclude that $f'(x)>0$ but only (again) $f'(x)\ge 0$.
%%%%%
%%%%%
\end{document}
