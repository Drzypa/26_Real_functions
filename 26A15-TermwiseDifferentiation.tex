\documentclass[12pt]{article}
\usepackage{pmmeta}
\pmcanonicalname{TermwiseDifferentiation}
\pmcreated{2013-03-22 14:38:38}
\pmmodified{2013-03-22 14:38:38}
\pmowner{Mathprof}{13753}
\pmmodifier{Mathprof}{13753}
\pmtitle{termwise differentiation}
\pmrecord{9}{36231}
\pmprivacy{1}
\pmauthor{Mathprof}{13753}
\pmtype{Theorem}
\pmcomment{trigger rebuild}
\pmclassification{msc}{26A15}
\pmclassification{msc}{40A30}
\pmsynonym{differentiating a series}{TermwiseDifferentiation}
%\pmkeywords{uniform convergence}
\pmrelated{PowerSeries}
\pmrelated{IntegrationOfLaplaceTransformWithRespectToParameter}
\pmrelated{IntegralOfLimitFunction}

\endmetadata

% this is the default PlanetMath preamble.  as your knowledge
% of TeX increases, you will probably want to edit this, but
% it should be fine as is for beginners.

% almost certainly you want these
\usepackage{amssymb}
\usepackage{amsmath}
\usepackage{amsfonts}

% used for TeXing text within eps files
%\usepackage{psfrag}
% need this for including graphics (\includegraphics)
%\usepackage{graphicx}
% for neatly defining theorems and propositions
 \usepackage{amsthm}
% making logically defined graphics
%%%\usepackage{xypic}

% there are many more packages, add them here as you need them

% define commands here
\theoremstyle{definition}
\newtheorem*{thmplain}{Theorem}
\begin{document}
\begin{thmplain}
\,\,If in the open interval $I$,
all the \PMlinkescapetext{terms} of the series 
  \begin{align}     
          f_1(x)\!+\!f_2(x)\!+\cdots
  \end{align}
have continuous derivatives,
the series converges  having sum $S(x)$ and
the differentiated series\, $f_1'(x)\!+\!f_2'(x)\!+\!\cdots$\, 
\PMlinkname{converges uniformly}{SumFunctionOfSeries} on the interval $I$,
then the series (1) can be differentiated termwise, i.e. in every point of $I$ the sum function $S(x)$ is differentiable and
       $$\frac{d\,S(x)}{dx} = f_1'(x)\!+\!f_2'(x)\!+\cdots$$
The situation implies also that the series (1) converges uniformly on $I$.
\end{thmplain}
%%%%%
%%%%%
\end{document}
