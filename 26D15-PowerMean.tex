\documentclass[12pt]{article}
\usepackage{pmmeta}
\pmcanonicalname{PowerMean}
\pmcreated{2013-03-22 11:47:17}
\pmmodified{2013-03-22 11:47:17}
\pmowner{drini}{3}
\pmmodifier{drini}{3}
\pmtitle{power mean}
\pmrecord{14}{30266}
\pmprivacy{1}
\pmauthor{drini}{3}
\pmtype{Definition}
\pmcomment{trigger rebuild}
\pmclassification{msc}{26D15}
\pmclassification{msc}{16D10}
\pmclassification{msc}{00-01}
\pmclassification{msc}{34-00}
\pmclassification{msc}{35-00}
\pmrelated{WeightedPowerMean}
\pmrelated{ArithmeticGeometricMeansInequality}
\pmrelated{ArithmeticMean}
\pmrelated{GeometricMean}
\pmrelated{HarmonicMean}
\pmrelated{GeneralMeansInequality}
\pmrelated{RootMeanSquare3}
\pmrelated{ProofOfGeneralMeansInequality}
\pmrelated{DerivationOfZerothWeightedPowerMean}
\pmrelated{DerivationOfHarmonicMeanAsTheLimitOfThePowerMean}

\usepackage{amssymb}
\usepackage{amsmath}
\usepackage{amsfonts}
\usepackage{graphicx}
%%%%\usepackage{xypic}
\begin{document}
The $r$-th power mean of the numbers $x_1,x_2,\ldots,x_n$ is defined as:

$$M^r(x_1,x_2,\ldots,x_n)=\left(\frac{x_1^r+x_2^r+\cdots+x_n^r}{n}\right)^{1/r}.$$
\smallskip

The arithmetic mean is a special case when $r=1$.
The power mean is a continuous function of $r$, and taking limit when $r\to0$ gives us the geometric mean:
$$M^0(x_1,x_2,\ldots,x_n)=\sqrt[n]{x_1 {x_{2}} \cdots x_n}.$$
\smallskip

Also, when $r=-1$ we get
$$M^{-1}(x_1,x_2,\ldots,x_n)=\frac{n}{\frac{1}{x_1}+\frac{1}{x_2}+\cdots+\frac{1}{x_n}}$$
the harmonic mean.

A generalization of power means are weighted power means.
%%%%%
%%%%%
%%%%%
%%%%%
\end{document}
