\documentclass[12pt]{article}
\usepackage{pmmeta}
\pmcanonicalname{IntermediateValueTheoremForExtendedRealNumbers}
\pmcreated{2013-03-22 15:35:15}
\pmmodified{2013-03-22 15:35:15}
\pmowner{matte}{1858}
\pmmodifier{matte}{1858}
\pmtitle{intermediate value theorem for extended real numbers}
\pmrecord{6}{37498}
\pmprivacy{1}
\pmauthor{matte}{1858}
\pmtype{Theorem}
\pmcomment{trigger rebuild}
\pmclassification{msc}{26A06}
\pmrelated{ExtendedRealNumbers}

% this is the default PlanetMath preamble.  as your knowledge
% of TeX increases, you will probably want to edit this, but
% it should be fine as is for beginners.

% almost certainly you want these
\usepackage{amssymb}
\usepackage{amsmath}
\usepackage{amsfonts}
\usepackage{amsthm}

\usepackage{mathrsfs}

% used for TeXing text within eps files
%\usepackage{psfrag}
% need this for including graphics (\includegraphics)
%\usepackage{graphicx}
% for neatly defining theorems and propositions
%
% making logically defined graphics
%%%\usepackage{xypic}

% there are many more packages, add them here as you need them

% define commands here

\newcommand{\sR}[0]{\mathbb{R}}
\newcommand{\sC}[0]{\mathbb{C}}
\newcommand{\sN}[0]{\mathbb{N}}
\newcommand{\sZ}[0]{\mathbb{Z}}

 \usepackage{bbm}
 \newcommand{\Z}{\mathbbmss{Z}}
 \newcommand{\C}{\mathbbmss{C}}
 \newcommand{\F}{\mathbbmss{F}}
 \newcommand{\R}{\mathbbmss{R}}
 \newcommand{\Q}{\mathbbmss{Q}}



\newcommand*{\norm}[1]{\lVert #1 \rVert}
\newcommand*{\abs}[1]{| #1 |}



\newtheorem{thm}{Theorem}
\newtheorem{defn}{Definition}
\newtheorem{prop}{Proposition}
\newtheorem{lemma}{Lemma}
\newtheorem{cor}{Corollary}
\begin{document}
\begin{thm}
Let $\overline{\R}$ be the extended real numbers, and
suppose $f\colon \overline{\R}\to \overline{\R}$ is a continuous function.
Suppose $x_1<x_2\in \overline{\R}$ are such that $f(x_1)\neq f(x_2)$. If
$y\in(f(x_1),f(x_2))$, then
for some $c\in (x_1,x_2)$ we have 
$$
   f(c)=y.
$$
\end{thm}

\begin{proof}
As $\overline{\R}$ is homeomorphic to $[0,1]$, we can assume that $f$ is a function
$f\colon[0,1]\to \overline{\R}$. For simplicity, 
let us also assume that $x_1=0$,$x_2=1$, and $f(0)<f(1)$. Then
for some $\varepsilon>0$ we have 
$$
  f(0)<y-\varepsilon<y<y+\varepsilon < f(1).
$$
Let $g\colon [0,1]\to \R$ be the continuous function
$$
  g(x) = \operatorname{max}\{ \operatorname{min}\{ f(x), y+\varepsilon\}, y-\varepsilon\}.
$$
Now $g(0)=y-\varepsilon$ and $g(1)=y+\varepsilon$, 
so for some $c\in(0,1)$, we have $g(c)=y$, and thus $f(c)=y$. 
\end{proof}
%%%%%
%%%%%
\end{document}
