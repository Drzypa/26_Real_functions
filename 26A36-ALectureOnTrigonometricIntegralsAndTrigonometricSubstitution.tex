\documentclass[12pt]{article}
\usepackage{pmmeta}
\pmcanonicalname{ALectureOnTrigonometricIntegralsAndTrigonometricSubstitution}
\pmcreated{2013-03-22 15:38:39}
\pmmodified{2013-03-22 15:38:39}
\pmowner{alozano}{2414}
\pmmodifier{alozano}{2414}
\pmtitle{A lecture on trigonometric integrals and trigonometric substitution}
\pmrecord{4}{37576}
\pmprivacy{1}
\pmauthor{alozano}{2414}
\pmtype{Feature}
\pmcomment{trigger rebuild}
\pmclassification{msc}{26A36}
\pmrelated{ALectureOnIntegrationByParts}
\pmrelated{ALectureOnIntegrationBySubstitution}
\pmrelated{ALectureOnThePartialFractionDecompositionMethod}

% this is the default PlanetMath preamble.  as your knowledge
% of TeX increases, you will probably want to edit this, but
% it should be fine as is for beginners.

% almost certainly you want these
\usepackage{amssymb}
\usepackage{amsmath}
\usepackage{amsthm}
\usepackage{amsfonts}

% used for TeXing text within eps files
%\usepackage{psfrag}
% need this for including graphics (\includegraphics)
%\usepackage{graphicx}
% for neatly defining theorems and propositions
%\usepackage{amsthm}
% making logically defined graphics
%%%\usepackage{xypic}

% there are many more packages, add them here as you need them

% define commands here

\newtheorem{thm}{Theorem}[section]
\newtheorem{conj}[thm]{Conjecture}
\newtheorem{cor}[thm]{Corollary}
\newtheorem{lem}[thm]{Lemma}
\newtheorem{prop}[thm]{Proposition}
\newtheorem{defn}[thm]{Definition}
\newtheorem{remark}[thm]{Remark}
\newtheorem{exe}{Problem}
\newtheorem*{exe1}{Problem 1}
\newtheorem*{exe2}{Problem 2}
\newtheorem*{exe3}{Problem 3}
\newtheorem*{exe4}{Problem 4}

\theoremstyle{definition}
\newtheorem{exa}[thm]{Example}

\def\notdiv{\ \mathbin{\mkern-8mu|\!\!\!\smallsetminus}}
\newcommand{\Qoft}{\mathbb{Q}(T)}  %use in linux
\newcommand{\done}{\Box} %use in linux
\newcommand{\R}{\ensuremath{\mathbb{R}}}
\newcommand{\C}{\ensuremath{\mathbb{C}}}
\newcommand{\Z}{\ensuremath{\mathbb{Z}}}
\newcommand{\Q}{\mathbb{Q}}
\newcommand{\peri}{\operatorname{Perimeter}}
\newcommand{\lc}{\lim_{x\to c}}
\newcommand{\lzero}{\lim_{x\to 0}}
\newcommand{\lhzero}{\lim_{h\to 0}}
\begin{document}
\section{Trigonometric Integrals}

First, we must recall a few trigonometric identities:
\begin{eqnarray}
\sin^2x+\cos^2x &=& 1\\
\sec^2x &=& 1 + \tan^2 x\\
\sin^2x &=& \frac{1-\cos (2x)}{2}\\
\cos^2x &=& \frac{1+\cos(2x)}{2}\\
\sin(2x) &=& 2\sin x \cos x\\
\cos(2x) &=& \cos^2x - \sin^2 x.
\end{eqnarray}

The most usual integrals which involve trigonometric functions can
be solved using the identities above.

\begin{exa}
$\int \sin x dx = -\cos x +C$ and $\int \cos x dx = \sin x +C$ are
immediate integrals.
\end{exa}
\begin{exa}
For $\int \sin^2x dx,\ \int \cos^2x dx$ we use formulas (3) and
(4) respectively, e.g.
$$\int \sin^2x dx =\int \frac{1-\cos (2x)}{2} dx = \frac{1}{2}
\int (1 - \cos(2x)) dx =
\frac{1}{2}\left(x-\frac{\sin(2x)}{2}\right)+C.$$
\end{exa}
\begin{exa}
\label{combo} For integrals of the form $\int \cos^m x\sin x\ dx$
or $\int \sin^m x \cos x\ dx$ we use substitution with $u=\cos x$
or $u=\sin x$ respectively, e.g.
$$\int \cos^2x \sin x dx = \int -u^2
du = -\frac{u^3}{3} + C= -\frac{\cos^3 x}{3} +C.\ [u=\cos x,\
du=-\sin x dx]$$
\end{exa}

In the following examples, we use equations (1) in the forms
$\sin^2x=1-\cos^2x$ or $\cos^2x=1-\sin^2x$ to transform the
integral into one of the type described in Example \ref{combo}.

\begin{exa}
\begin{eqnarray*}
\int \sin^3 x dx &=& \int \sin^2x\sin x dx=\int(1-\cos^2 x)\sin x
dx\\
&=& \int \sin x dx - \int \cos^2 x \sin x dx\\
&=& -\cos x + \frac{\cos^3 x}{3} + C.
\end{eqnarray*}
Similarly one can solve $\int \cos^3 x dx$.
\end{exa}
\begin{exa}
\begin{eqnarray*}
\int \cos^3x\sin^2 x dx &=& \int \cos^2 x \cos x \sin^2x dx = \int
(1-\sin^2x)\cos x \sin^2 x dx\\
&=& \int \cos x \sin^2 x dx - \int \cos x \sin^4 x dx\\
&=& \frac{\sin^3 x}{3} - \frac{\sin^5 x}{5} + C.
\end{eqnarray*}
\end{exa}
\begin{exa}
In order to solve $\int \cos^5 x \sin^3 x dx$ we express it first
as $\int \cos^5x \sin^2 x \sin x=\int \cos^5 x (1-\cos^2x)\sin x
dx$ and then proceed as in the previous example.
\end{exa}
One can use similar tricks to solve integrals which involve
products of powers of $\sec x $ and $\tan x$, by using Equation
(2). Also, recall that the derivative of $\tan x$ is $\sec^2 x$
while the derivative of $\sec x$ is $\sec x \tan x$.
\begin{exa}
\begin{eqnarray*}
\int \tan^5x\sec^4 x dx &=& \int \tan^5 x \sec^2 x \sec^2x dx =
\int
\tan^5 x(1+\tan^2x)\sec^2 x dx\\
&=& \int \tan^5 x \sec^2 x dx + \int \tan^7 x \sec^2 x dx\\
&=& \frac{\tan^6 x}{6} + \frac{\tan^8 x}{8} + C.
\end{eqnarray*}
\end{exa}
\begin{exa}
\begin{eqnarray*}
\int \tan^3x\sec^4 x dx &=& \int \tan x \tan^2 x \sec^4x dx = \int
\tan x(\sec^2x-1)\sec^4 x dx\\
&=& \int \tan x \sec x \sec^5 x dx - \int \tan x \sec x\sec^3 x dx\\
&=& \frac{\sec^6 x}{6} - \frac{\sec^4 x}{4} + C.
\end{eqnarray*}
\end{exa}

\section{Trigonometric Substitutions}

One can easily deduce that $\int_0^1 \sqrt{1-x^2}dx$ has value
$\frac{\pi}{4}$. Why? Simply because the graph of the function
$y=\sqrt{1-x^2}$ is half a circumference of radius $r=1$ (because
if you square both sides of $y=\sqrt{1-x^2}$ you obtain
$x^2+y^2=1$ which is the equation of a circle or radius $r=1$).
Therefore, the area under the graph is a quarter of the area of a
circle.

How does one compute $\int_0^1 \sqrt{1-x^2} dx$ without using the
geometry of the problem? This is the prototype of integral where a
trigonometric substitution will work very nicely. Notice that
neither substitution nor integration by parts will work
appropriately.

\begin{exa}
Suppose we want to solve $\int_0^1 \sqrt{1-x^2}dx$ with analytic
methods. We will use a substitution $x=\sin \theta$ (so $\theta$
will be our new variable of integration), because, as we know from
Equation (1), $\sqrt{1-x^2}=\sqrt{1-\sin^2 \theta}=\cos \theta$,
thus getting rid of the pesky square root. Notice that
$dx=\cos\theta d\theta$. We also need to find the new limits of
integration with respect to the new variable of integration,
namely $\theta$. When $x=0=\sin\theta$ we must have $\theta=0$.
Similarly, when $x=1=\sin\theta$ one has $\theta=\pi/2$. We are
now ready to integrate:
\begin{eqnarray*}
\int_0^1 \sqrt{1-x^2}dx &=& \int_0^{\pi/2} (\cos \theta)\cos
\theta
d\theta=\int_0^{\pi/2} \cos^2 \theta d\theta\\
&=& \int_0^{\pi/2} \frac{1+\cos(2\theta)}{2}d\theta =
\frac{1}{2}\left(\theta+\frac{\sin(2\theta)}{2}\right)_0^{\pi/2}=\pi/4.
\end{eqnarray*}
Notice that we made use of Equation (4) in the second line.
\end{exa}
\begin{exa}
Similarly, one can solve $\int_0^r\sqrt{r^2 - x^2}dx$ by using a
substitution $x=r\sin\theta$. Indeed,
$\sqrt{r^2-x^2}=\sqrt{r^2-r^2\sin^2 \theta}=r\cos \theta$ and
$dx=r\cos\theta d\theta$. The limits of integration with respect
to $\theta$ are again $\theta=0$ to $\theta=\pi/2$ (check this!).
Thus:
\begin{eqnarray*}
\int_0^r \sqrt{r^2-x^2}dx &=& \int_0^{\pi/2} r^2(\cos \theta)\cos
\theta
d\theta=r^2\int_0^{\pi/2} \cos^2 \theta d\theta\\
&=& r^2\int_0^{\pi/2} \frac{1+\cos(2\theta)}{2}d\theta =
\frac{r^2}{2}\left(\theta+\frac{\sin(2\theta)}{2}\right)_0^{\pi/2}=r^2\pi/4.
\end{eqnarray*}
Thus, we have proved that a quarter of a circle of radius $r$ has
area $r^2\pi/4$ which implies that the area of such a circle is
$\pi r^2$, as usual.
\end{exa}

The trigonometric substitutions {\it usually} work when
expressions like $\sqrt{r^2-x^2}$, $\sqrt{r^2+x^2}$,
$\sqrt{x^2-r^2}$ appear in the integral at hand, for some real
number $r$. Here is a table of the suggested change of variables
in each particular case:\\

\begin{center}
\begin{tabular}{|c|c|c|}
  \hline
  % after \\: \hline or \cline{col1-col2} \cline{col3-col4} ...
  {\it If you see...} & {\it try this...} & {\it because...} \\
  \hline
  $\sqrt{1-x^2}$ & $x=\sin \theta$ & $\sqrt{1-\sin^2\theta}=\cos \theta$ \\
  $\sqrt{r^2-x^2}$ & $x=r\sin\theta$ & $\sqrt{r^2-\sin^2\theta}=r\cos\theta$ \\
  $\sqrt{1+x^2}$ & $x=\tan \theta$ & $\sqrt{1+\tan^2\theta}=\sec \theta$ \\
  $\sqrt{r^2+x^2}$ & $x=r\tan\theta$ & $\sqrt{r^2+\tan^2\theta}=r\sec\theta$ \\
  $\sqrt{x^2-1}$ & $x=\sec \theta$ & $\sqrt{\sec^2\theta-1}=\tan \theta$ \\
  $\sqrt{x^2-r^2}$ & $x=r\sin\theta$ & $\sqrt{\sec^2\theta-1}=r\tan\theta$ \\
  \hline
\end{tabular}
\end{center}

\begin{remark}
The above are ``suggested'' substitutions, they may not be the
most ideal choice! For example, for the integral $\int
2x\sqrt{1-x^2}dx$, the change $u=1-x^2$ will work much better than
$x=\sin \theta$.
\end{remark}

\begin{exa}
We would like to find the value of
$$\int_{\sqrt{2}}^2 \frac{1}{x^3\sqrt{x^2-1}} dx.$$
Since neither a $u$-substitution nor integration by parts seem
appropriate, we try $x=\sec \theta$, $dx=\sec\theta \tan \theta
d\theta$. When $x=\sqrt{2}=\sec\theta$ one has $\theta=\pi/4$
while $x=2$ implies $\theta=\pi/3$. Hence:
\begin{eqnarray*}\int_{\sqrt{2}}^2 \frac{1}{x^3\sqrt{x^2-1}} dx &=&
\int_{\pi/4}^{\pi/3} \frac{\sec\theta \tan\theta}{\sec^3\theta
\tan\theta} d\theta = \int_{\pi/4}^{\pi/3}
\frac{1}{\sec^2\theta}d\theta=\int_{\pi/4}^{\pi/3} \cos^2 \theta
d\theta
\end{eqnarray*}
and the last integral is easy to compute using Equation (4).
\end{exa}
%%%%%
%%%%%
\end{document}
