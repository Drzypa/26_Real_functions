\documentclass[12pt]{article}
\usepackage{pmmeta}
\pmcanonicalname{ProofOfPropertiesOfDerivativesByPureAlgebra}
\pmcreated{2013-03-22 16:00:03}
\pmmodified{2013-03-22 16:00:03}
\pmowner{Algeboy}{12884}
\pmmodifier{Algeboy}{12884}
\pmtitle{proof of properties of derivatives by pure algebra}
\pmrecord{5}{38028}
\pmprivacy{1}
\pmauthor{Algeboy}{12884}
\pmtype{Proof}
\pmcomment{trigger rebuild}
\pmclassification{msc}{26B05}
\pmclassification{msc}{46G05}
\pmclassification{msc}{26A24}
\pmrelated{RulesOfCalculusForDerivativeOfPolynomial}

\endmetadata

\usepackage{latexsym}
\usepackage{amssymb}
\usepackage{amsmath}
\usepackage{amsfonts}
\usepackage{amsthm}

%%\usepackage{xypic}

%-----------------------------------------------------

%       Standard theoremlike environments.

%       Stolen directly from AMSLaTeX sample

%-----------------------------------------------------

%% \theoremstyle{plain} %% This is the default

\newtheorem{thm}{Theorem}

\newtheorem{coro}[thm]{Corollary}

\newtheorem{lem}[thm]{Lemma}

\newtheorem{lemma}[thm]{Lemma}

\newtheorem{prop}[thm]{Proposition}

\newtheorem{conjecture}[thm]{Conjecture}

\newtheorem{conj}[thm]{Conjecture}

\newtheorem{defn}[thm]{Definition}

\newtheorem{remark}[thm]{Remark}

\newtheorem{ex}[thm]{Example}



%\countstyle[equation]{thm}



%--------------------------------------------------

%       Item references.

%--------------------------------------------------


\newcommand{\exref}[1]{Example-\ref{#1}}

\newcommand{\thmref}[1]{Theorem-\ref{#1}}

\newcommand{\defref}[1]{Definition-\ref{#1}}

\newcommand{\eqnref}[1]{(\ref{#1})}

\newcommand{\secref}[1]{Section-\ref{#1}}

\newcommand{\lemref}[1]{Lemma-\ref{#1}}

\newcommand{\propref}[1]{Prop\-o\-si\-tion-\ref{#1}}

\newcommand{\corref}[1]{Cor\-ol\-lary-\ref{#1}}

\newcommand{\figref}[1]{Fig\-ure-\ref{#1}}

\newcommand{\conjref}[1]{Conjecture-\ref{#1}}


% Normal subgroup or equal.

\providecommand{\normaleq}{\unlhd}

% Normal subgroup.

\providecommand{\normal}{\lhd}

\providecommand{\rnormal}{\rhd}
% Divides, does not divide.

\providecommand{\divides}{\mid}

\providecommand{\ndivides}{\nmid}


\providecommand{\union}{\cup}

\providecommand{\bigunion}{\bigcup}

\providecommand{\intersect}{\cap}

\providecommand{\bigintersect}{\bigcap}










\begin{document}
\begin{thm}
The derivative satisfies the following rules:
\begin{itemize}
\item[Linearity]
\[\frac{d}{dx}(f(x)+g(x))=\frac{df}{dx}+\frac{dg}{dx},\qquad
\frac{d}{dx}(af(x))=a\frac{df}{dx},\]
for $f(x),g(x)\in R[x]$ and $a\in R$.
\item[Power Rule]
\[\frac{d}{dx}(x^n)=nx^{n-1}.\]
\item[Product Rule]
\[\frac{d}{dx}(f(x)g(x))=\frac{df}{dx}g(x)+f(x)\frac{dg}{dx}.\]
\end{itemize}
\end{thm}
\begin{remark}
The following proofs apply to \PMlinkname{derivatives by pure algebra}{DerivativesByPureAlgebra}.  While the nature of the proofs are 
similar to the usual proofs, the notion of a limit is replaced by modular
arithmetic in $R[x,h]/(h)$.
\end{remark}
\begin{proof}
\noindent\textit{Power rule.}
\begin{eqnarray*}
\frac{d}{dx}(x^n) 
  & \equiv & \frac{(x+h)^n-x^n}{h} \\
   & = & \sum_{j=1}^{n} \binom{i}{j} x^{n-j} h^{j-1}\\
  & \equiv & \binom{n}{1} x^{n-1}=nx^{n-1}.
\end{eqnarray*}

\noindent\textit{Linearity rule.}
For all $f(x),g(x)\in R[x]\cong R[x,h]/(h)$, it follows
\[\frac{(f+g)(x+h)-(f+g)(x)}{h} \equiv \frac{f(x+h)+g(x+h)-f(x)-g(x)}{h}
    \equiv \frac{f(x+h)-f(x)}{h}+\frac{g(x+h)-g(x)}{h}.\]
Furthermore, for all $a\in R$
\[\frac{(af)(x+h)-(af)(x)}{h}\equiv \frac{af(x+h)-af(x)}{h}=a\frac{f(x+h)-f(x)}{h}.\]

\noindent\textit{Product rule.}
In $R[x,h]$ modulo $(h)$ we have:
\begin{eqnarray*}
\frac{d}{dx}(fg) 
  & \equiv & \frac{f(x+h)g(x+h))-f(x)g(x)}{h}\\
  & \equiv & \frac{f(x+h)g(x+h)-f(x)g(x+h)+f(x)g(x+h)-f(x)g(x)}{h}\\
  & \equiv & \frac{(f(x+h)-f(x))g(x+h)+f(x)(g(x+h)-g(x))}{h}\\
  & \equiv & \frac{f(x+h)-f(x)}{h}g(x+h)+f(x)\frac{g(x+h)-g(x)}{h}\\
  & \equiv & \frac{df}{dx}g(x)+f(x)\frac{dg}{dx}. 
\end{eqnarray*}
\end{proof}


%%%%%
%%%%%
\end{document}
