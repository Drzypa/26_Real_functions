\documentclass[12pt]{article}
\usepackage{pmmeta}
\pmcanonicalname{IntegralRelatedToArcSine}
\pmcreated{2013-03-22 18:44:58}
\pmmodified{2013-03-22 18:44:58}
\pmowner{pahio}{2872}
\pmmodifier{pahio}{2872}
\pmtitle{integral related to arc sine}
\pmrecord{8}{41526}
\pmprivacy{1}
\pmauthor{pahio}{2872}
\pmtype{Example}
\pmcomment{trigger rebuild}
\pmclassification{msc}{26A09}
\pmrelated{SubstitutionNotation}
\pmrelated{ArcSine}
\pmrelated{Arcosh}
\pmrelated{MethodsOfEvaluatingImproperIntegrals}
\pmrelated{CyclometricFunctions}

\endmetadata

% this is the default PlanetMath preamble.  as your knowledge
% of TeX increases, you will probably want to edit this, but
% it should be fine as is for beginners.

% almost certainly you want these
\usepackage{amssymb}
\usepackage{amsmath}
\usepackage{amsfonts}

% used for TeXing text within eps files
%\usepackage{psfrag}
% need this for including graphics (\includegraphics)
%\usepackage{graphicx}
% for neatly defining theorems and propositions
%\usepackage{amsthm}
% making logically defined graphics
%%%\usepackage{xypic}

% there are many more packages, add them here as you need them

% define commands here
\newcommand{\sijoitus}[2]%
{\operatornamewithlimits{\Big/}_{\!\!\!#1}^{\,#2}}
\begin{document}
We want to evaluate the integral
\begin{align}
\int_1^\infty\!\left(\arcsin\frac{1}{x}-\frac{1}{x}\right)dx.
\end{align}
Therefore we put an extra variable $t$ to the integrand and thus get the function
$$I(t) \;:=\; \int_1^\infty\!\left(\arcsin\frac{t}{x}-\frac{t}{x}\right)dx,$$
and in \PMlinkescapetext{order} to obtain a simpler integral, we \PMlinkname{differentiate it under the integral sign}{DifferentiationUnderIntegralSign}, then integrate:
\begin{align*}
I'(t) & \;=\; \int_1^\infty\!\left(\frac{1}{\sqrt{1\!-\!\frac{t^2}{x^2}}}\cdot\frac{1}{x}-\frac{1}{x}\right)dx\\
& \;=\; \int_1^\infty\left(\frac{1}{\sqrt{\frac{x^2}{t^2}\!-\!1}}\cdot\frac{1}{t}-\frac{1}{x}\right)dx\\
& \;=\; \sijoitus{x=1}{\quad\infty}\!\left[\ln\left(\frac{x}{t}+\sqrt{\frac{x^2}{t^2}\!-\!1}\right)-\ln{x}\right]\\
& \;=\; \sijoitus{x=1}{\quad\infty}\!\ln\frac{1+\sqrt{1\!-\!\frac{t^2}{x^2}}}{t}\\
& \;=\; \ln\frac{2}{t}-\ln\frac{1+\sqrt{1\!-\!t^2}}{t} \;=\; \ln{2}-\ln(1+\sqrt{1\!-\!t^2})
\end{align*}
The gotten expression implies, since\, $I(0) = \int_1^\infty(\arcsin{0}-0)dx = 0$,\, that
$$I(t) \;=\; \int_0^t[\ln{2}-\ln(1+\sqrt{1\!-\!t^2})]\,dt \;=\; t\ln{2}-\int_0^t\ln(1+\sqrt{1\!-\!t^2})\,dt,$$
and consequently
\begin{align*}
I(1) & \;=\; \ln{2}-\!\int_0^1\ln(1+\sqrt{1\!-\!t^2})\,dt \;=\; 
\ln{2}-\!\sijoitus{0}{\quad1}\!t\ln(1+\sqrt{1\!-\!t^2})-\!\int_0^1\frac{t^2\,dt}{(1+\sqrt{1\!-\!t^2})\sqrt{1\!-\!t^2}}\\
& \;=\; \ln{2}-\!\int_0^1\frac{t^2\,dt}{1\!-\!t^2+\sqrt{1\!-\!t^2}}.
\end{align*}
Here, the \PMlinkname{substitution}{ChangeOfVariableInDefiniteIntegral} \,$t = \sin{u}$\, helps, yielding
$$I(1) \;=\; \ln{2}-\int_0^{\frac{\pi}{2}}\!(1-\cos{u})\,du \;=\; \ln{2}-\frac{\pi}{2}+1.$$
Accordingly, we have the result
$$\int_1^\infty\!\left(\arcsin\frac{1}{x}-\frac{1}{x}\right)dx \;=\; 1+\ln{2}-\frac{\pi}{2}.$$\\


For the convergence, see the French version of \PMlinkexternal{this article}{http://en.wikipedia.org/wiki/Improper_integral}.

%%%%%
%%%%%
\end{document}
