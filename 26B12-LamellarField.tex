\documentclass[12pt]{article}
\usepackage{pmmeta}
\pmcanonicalname{LamellarField}
\pmcreated{2013-03-22 14:43:44}
\pmmodified{2013-03-22 14:43:44}
\pmowner{pahio}{2872}
\pmmodifier{pahio}{2872}
\pmtitle{lamellar field}
\pmrecord{18}{36360}
\pmprivacy{1}
\pmauthor{pahio}{2872}
\pmtype{Definition}
\pmcomment{trigger rebuild}
\pmclassification{msc}{26B12}
\pmsynonym{lamellar}{LamellarField}
\pmsynonym{irrotational}{LamellarField}
\pmsynonym{conservative}{LamellarField}
\pmsynonym{laminar}{LamellarField}
\pmrelated{CurlFreeField}
\pmrelated{PoincareLemma}
\pmrelated{VectorPotential}
\pmrelated{GradientTheorem}
\pmdefines{scalar potential}
\pmdefines{potential}
\pmdefines{rotor}

\endmetadata

% this is the default PlanetMath preamble.  as your knowledge
% of TeX increases, you will probably want to edit this, but
% it should be fine as is for beginners.

% almost certainly you want these
\usepackage{amssymb}
\usepackage{amsmath}
\usepackage{amsfonts}

% used for TeXing text within eps files
%\usepackage{psfrag}
% need this for including graphics (\includegraphics)
%\usepackage{graphicx}
% for neatly defining theorems and propositions
%\usepackage{amsthm}
% making logically defined graphics
%%%\usepackage{xypic}

% there are many more packages, add them here as you need them

% define commands here
\begin{document}
A vector field \,$\vec{F} = \vec{F}(x,\,y,\,z)$,\, defined in an open set $D$ of $\mathbb{R}^3$, is\, {\em lamellar}\, if the condition 
                   $$\nabla\!\times\!\vec{F} = \vec{0}$$
is satisfied in every point \,$(x,\,y,\,z)$\, of $D$.

Here, $\nabla\!\times\!\vec{F}$ is the curl or {\em rotor} of $\vec{F}$.\, The condition is equivalent with both of the following:
\begin{itemize}
\item The line integrals 
                $$\oint_s \vec{F}\cdot d\vec{s}$$
taken around any \PMlinkescapetext{closed} contractible curve $s$ vanish.
\item The vector field has a \PMlinkescapetext{{\em scalar potential}}\, $u = u(x,\,y,\,z)$\, which has continuous partial derivatives and which is up to a \PMlinkescapetext{constant term} unique in a simply connected domain; the scalar potential means that
                      $$\vec{F} = \nabla u.$$
\end{itemize}
The scalar potential has the expression
$$u = \int_{P_0}^P\vec{F}\cdot d\vec{s},$$
where the point $P_0$ may be chosen freely,\, $P = (x,\,y,\,z)$.

\textbf{Note.}\, In physics, $u$ is in general replaced with\, $V = -u$.\, If the $\vec{F}$ is interpreted as a \PMlinkescapetext{force}, then the potential $V$ is equal to the work made by the \PMlinkescapetext{force} when its point of application is displaced from $P_0$ to infinity.
%%%%%
%%%%%
\end{document}
