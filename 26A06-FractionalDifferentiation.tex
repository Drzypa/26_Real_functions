\documentclass[12pt]{article}
\usepackage{pmmeta}
\pmcanonicalname{FractionalDifferentiation}
\pmcreated{2013-03-22 16:18:46}
\pmmodified{2013-03-22 16:18:46}
\pmowner{Wkbj79}{1863}
\pmmodifier{Wkbj79}{1863}
\pmtitle{fractional differentiation}
\pmrecord{21}{38437}
\pmprivacy{1}
\pmauthor{Wkbj79}{1863}
\pmtype{Definition}
\pmcomment{trigger rebuild}
\pmclassification{msc}{26A06}
\pmsynonym{Grunwald-Letnikov differentiation}{FractionalDifferentiation}
\pmrelated{HigherOrderDerivativesOfSineAndCosine}
\pmdefines{fractional derivative}
\pmdefines{left-hand Grunwald-Letnikov derivative}
\pmdefines{left hand Grundwald Letnikov derivative}
\pmdefines{right-hand Grundwald-Letnikov derivative}
\pmdefines{right hand Grundwald-Letnikov derivative}

\endmetadata

% this is the default PlanetMath preamble.  as your knowledge
% of TeX increases, you will probably want to edit this, but
% it should be fine as is for beginners.

% almost certainly you want these
\usepackage{amssymb}
\usepackage{amsmath}
\usepackage{amsfonts}

% used for TeXing text within eps files
%\usepackage{psfrag}
% need this for including graphics (\includegraphics)
%\usepackage{graphicx}
% for neatly defining theorems and propositions
%\usepackage{amsthm}
% making logically defined graphics
%%%\usepackage{xypic}

% there are many more packages, add them here as you need them

% define commands here

\begin{document}
The idea of Grunwald-Letnikov differentiation comes from the following formulas of \PMlinkname{backward}{BackwardDifference} and forward difference \PMlinkescapetext{equations}. Within this entry, $[ \cdot ]$ will be used to denote the greatest integer function and $\Gamma$ will be used to denote the gamma function.

{\bf Backward difference} 

\begin{equation}D_{-}(f)(x) = \lim_{h\to 0}\frac{f(x)-f(x-h)}{h} 
\end{equation}

\begin{equation}D^n_{-}(f)(x)=\lim_{h\to 0}\frac{1}{h^n}\sum_{k=0}^n
\frac{(-1)^k n!}{k! (n-k)!}f(x-kh) \end{equation}

For derivatives of integer orders, we only requires to specifies one point $x\in {\mathbb R}$. Fractional derivatives, like fractional definite integrals, require an interval $[a,b]$ to be specified for the function $f:{\mathbb R}\to {\mathbb R}$ we are talking about.

{\bf Definition 1:  Left-hand Grunwald-Letnikov derivative} 

\begin{equation}D^p_{-}(f)(x)=
\lim_{h\to 0}\frac{1}{h^p}\sum_{k=0}^{\left[\frac{b-a}{h}\right]}
\frac{(-1)^k\Gamma (p+1)}{k! \Gamma (p-k+1)} f(x-kh) \end{equation}

{\bf Forward difference}

\begin{equation}D_{+}(f)(x) = \lim_{h\to 0}\frac{f(x+h)-f(x)}{h}
\end{equation}

\begin{equation}D^n_{+}(f)(x) = \lim_{h\to 0}\frac{1}{h^n}\sum^n_{k=0}
\frac{(-1)^k n!}{k! (n-k)!} f(x+(n-k-1)h) \end{equation}

{\bf Definition 2:  Right-hand Grunwald-Letnikov derivative}

\begin{equation}D^p_{+}(f)(x)=
\lim_{h\to 0}\frac{1}{h^p}\sum_{k=0}^{\left[\frac{b-a}{h}\right]}
\frac{(-1)^k\Gamma (p+1)}{k! \Gamma (p-k+1)} f(x+(m-k-1)h) \end{equation}

{\bf Theorem 1: Properties of fractional derivatives}

\begin{itemize}
\item {\rm Linearity}:  $D^p_{\pm}(a f+ b g)(x) = a D^p_{\pm}(f)(x) + b D^p_{\pm}(g)(x)$ where $a,b\in {\mathbb R}$ are any real constants
\item {\rm Iteration}:  $D^p_{\pm}D^q_{\pm}(f)(x) = D^{p+q}_{\pm}(f)(x)$
\item {\rm Chain rule}:  $\displaystyle{\frac{d^\beta f(g(x))}{dx^{\beta}} 
=\sum_{k=0}^{\infty}\frac{\Gamma(1+\beta)}{\Gamma(1+k)\Gamma(1-k+\beta)}
\frac{d^{\beta-k}1}{dx^{\beta-k}}
\frac{d^k f(g(x))}{dx^k} }$
\item {\rm Leibniz Rule}:   $\displaystyle{\frac{d^\beta (f(x)g(x))}{dx^\beta}
=\sum_{k=0}^{\infty}\frac{\Gamma(1+\beta)}{\Gamma(1+k)\Gamma(1-k+\beta)}
\frac{d^k f(x)}{dx^k} \frac{d^{\beta-k}g(x)}{dx^{\beta-k}} }$
\end{itemize}

{\bf Theorem 2: Table of fractional derivatives} 

\begin{itemize}
\item $\displaystyle{ D^{\alpha}_{\pm}(x^p)
=\frac{\Gamma (p+1)x^{p-\alpha}}{\Gamma (p-\alpha+1)} }$ where $\alpha,p\in {\mathbb R}$ and $\Gamma(x)$
\item $\displaystyle{ D^{\alpha}_{\pm}( e^{\lambda x} )
=\lambda^{\alpha} e^{\lambda x} }$ for all $\lambda\in {\mathbb R}$
\item $\displaystyle{ D^{\alpha}_{\pm} (\sin x) = \sin \left(x+\frac{\alpha \pi}{2}\right)}$
\item $\displaystyle{ D^{\alpha}_{\pm} (\cos x) = \cos \left(x+\frac{\alpha \pi}{2}\right)}$
\item $\displaystyle{ D^{\alpha}_{\pm} (e^{i x}) =\cos \left(x+\frac{\pi\alpha}{2}\right)+i\sin \left(x+\frac{\pi\alpha}{2}\right) }$
\end{itemize}
%%%%%
%%%%%
\end{document}
