\documentclass[12pt]{article}
\usepackage{pmmeta}
\pmcanonicalname{GeneralizedIntermediateValueTheorem}
\pmcreated{2013-03-22 17:17:44}
\pmmodified{2013-03-22 17:17:44}
\pmowner{azdbacks4234}{14155}
\pmmodifier{azdbacks4234}{14155}
\pmtitle{generalized intermediate value theorem}
\pmrecord{8}{39639}
\pmprivacy{1}
\pmauthor{azdbacks4234}{14155}
\pmtype{Theorem}
\pmcomment{trigger rebuild}
\pmclassification{msc}{26A06}
%\pmkeywords{continuous}
%\pmkeywords{connected}
%\pmkeywords{order}
%\pmkeywords{order topology}
\pmrelated{OrderTopology}
\pmrelated{TotalOrder}
\pmrelated{Continuous}
\pmrelated{ConnectedSpace}
\pmrelated{ConnectednessIsPreservedUnderAContinuousMap}

%packages
\usepackage{amsmath,mathrsfs,amsfonts,amsthm}
%theorem environments
\theoremstyle{plain}
\newtheorem*{thm*}{Theorem}
\newtheorem*{lem*}{Lemma}
\newtheorem*{cor*}{Corollary}
\newtheorem*{prop*}{Proposition}
%delimiters
\newcommand{\set}[1]{\{#1\}}
\newcommand{\medset}[1]{\big\{#1\big\}}
\newcommand{\bigset}[1]{\bigg\{#1\bigg\}}
\newcommand{\Bigset}[1]{\Bigg\{#1\Bigg\}}
\newcommand{\abs}[1]{\vert#1\vert}
\newcommand{\medabs}[1]{\big\vert#1\big\vert}
\newcommand{\bigabs}[1]{\bigg\vert#1\bigg\vert}
\newcommand{\Bigabs}[1]{\Bigg\vert#1\Bigg\vert}
\newcommand{\norm}[1]{\Vert#1\Vert}
\newcommand{\mednorm}[1]{\big\Vert#1\big\Vert}
\newcommand{\bignorm}[1]{\bigg\Vert#1\bigg\Vert}
\newcommand{\Bignorm}[1]{\Bigg\Vert#1\Bigg\Vert}
\newcommand{\vbrack}[1]{\langle#1\rangle}
\newcommand{\medvbrack}[1]{\big\langle#1\big\rangle}
\newcommand{\bigvbrack}[1]{\bigg\langle#1\bigg\rangle}
\newcommand{\Bigvbrack}[1]{\Bigg\langle#1\Bigg\rangle}
\newcommand{\sbrack}[1]{[#1]}
\newcommand{\medsbrack}[1]{\big[#1\big]}
\newcommand{\bigsbrack}[1]{\bigg[#1\bigg]}
\newcommand{\Bigsbrack}[1]{\Bigg[#1\Bigg]}
%operators
\DeclareMathOperator{\Hom}{Hom}
\DeclareMathOperator{\Tor}{Tor}
\DeclareMathOperator{\Ext}{Ext}
\DeclareMathOperator{\Aut}{Aut}
\DeclareMathOperator{\End}{End}
\DeclareMathOperator{\Inn}{Inn}
\DeclareMathOperator{\lcm}{lcm}
\DeclareMathOperator{\ord}{ord}
\DeclareMathOperator{\rank}{rank}
\DeclareMathOperator{\tr}{tr}
\DeclareMathOperator{\Mat}{Mat}
\DeclareMathOperator{\Gal}{Gal}
\DeclareMathOperator{\GL}{GL}
\DeclareMathOperator{\SL}{SL}
\DeclareMathOperator{\SO}{SO}
\DeclareMathOperator{\ann}{ann}
\DeclareMathOperator{\im}{im}
\DeclareMathOperator{\Char}{char}
\DeclareMathOperator{\Spec}{Spec}
\DeclareMathOperator{\supp}{supp}
\DeclareMathOperator{\diam}{diam}
\DeclareMathOperator{\Ind}{Ind}
\DeclareMathOperator{\vol}{vol}

\begin{document}
\begin{thm*}
Let $f:X\rightarrow Y$ be a continuous function with $X$ a connected space and $Y$ a totally ordered set in the order topology. If $x_1,x_2\in X$ and $y\in Y$ lies between $f(x_1)$ and $f(x_2)$, then there exists $x\in X$ such that $f(x)=y$.
\end{thm*}
\begin{proof}
The sets $U=f(X)\cap(-\infty,y)$ and $V=f(X)\cap(y,\infty)$ are disjoint open subsets of $f(X)$ in the subspace topology, and they are both non-empty, as $f(x_1)$ is contained in one and $f(x_2)$ is contained in the other. If $y\notin f(X)$, then $U\cup V$ constitutes a \PMlinkescapetext{separation} of the space $f(X)$, contradicting the hypothesis that $f(X)$ is the continuous image of the connected space $X$. Thus there must exist $x\in X$ such that $f(x)=y$.
\end{proof}
This version of the intermediate value theorem reduces to the familiar one of \PMlinkid{real analysis}{7599} when $X$ is taken to be a closed interval in $\mathbb{R}$ and $Y$ is taken to be $\mathbb{R}$. 

\begin{thebibliography}{1}
\bibitem{munkres}
J. Munkres, \emph{Topology}, 2nd ed. Prentice Hall, 1975.
\end{thebibliography}
%%%%%
%%%%%
\end{document}
