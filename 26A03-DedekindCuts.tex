\documentclass[12pt]{article}
\usepackage{pmmeta}
\pmcanonicalname{DedekindCuts}
\pmcreated{2013-03-22 12:38:34}
\pmmodified{2013-03-22 12:38:34}
\pmowner{rmilson}{146}
\pmmodifier{rmilson}{146}
\pmtitle{Dedekind cuts}
\pmrecord{26}{32907}
\pmprivacy{1}
\pmauthor{rmilson}{146}
\pmtype{Definition}
\pmcomment{trigger rebuild}
\pmclassification{msc}{26A03}
\pmsynonym{Schnitt}{DedekindCuts}
\pmrelated{RealNumber}

\endmetadata

\usepackage{amsmath}
\usepackage{amsthm}
\usepackage{amsfonts}
\usepackage{amssymb}

\newtheorem{thm}{Theorem}

\theoremstyle{definition}
\newtheorem*{defn}{Definition}
\theoremstyle{definition}
\newtheorem*{rem}{Remark}

\theoremstyle{definition}
\newtheorem*{nott}{Notation}

\newtheorem{lemma}{Lemma}
\newtheorem{cor}{Corollary}
\newtheorem*{eg}{Example}
\newtheorem*{ex}{Exercise}
\newtheorem*{prop}{Proposition}


\newcommand{\RR}{\mathbb{R}}
\newcommand{\QQ}{\mathbb{Q}}
\newcommand{\ZZ}{\mathbb{Z}}
\newcommand{\NN}{\mathbb{N}}
\newcommand{\leftbb}{[ \! [}
\newcommand{\rightbb}{] \! ]}
\newcommand{\bt}{\begin{thm}}
\newcommand{\et}{\end{thm}}
\newcommand{\Rel}{\mathbf{R}}
\newcommand{\er}{\thicksim}
\newcommand{\sqle}{\sqsubseteq}
\newcommand{\floor}[1]{\lfloor{#1}\rfloor}
\newcommand{\ceil}[1]{\lceil{#1}\rceil}
\begin{document}
\PMlinkescapeword{cut}
\PMlinkescapeword{sound}

The purpose of Dedekind cuts is to provide a sound logical foundation for the real number system.
Dedekind's motivation behind this project is to notice that a real number $\alpha$, intuitively,
is completely determined by the rationals strictly smaller than $\alpha$ and those strictly larger than
$\alpha$.  Concerning the completeness or continuity of the real line, Dedekind notes in \cite{Dedekind} that

\begin{quote}
If all points of the straight line fall into two classes such that every point of the first class lies to the
left of every point of the second class, then there exists one and only one point which produces this division
of all points into two classes, this severing of the straight line into two portions.
\end{quote}

Dedekind defines a point to produce the division of the real line if this point is either the
least or greatest element of either one of the classes mentioned above.  He further notes that the completeness 
property, as he just phrased it, is deficient in the rationals, which motivates the definition of reals as cuts 
of rationals.  Because all rationals greater than $\alpha$ are really 
just excess baggage, we prefer to sway
somewhat from Dedekind's original definition.  Instead, we adopt the following definition.

\begin{defn}
A \emph{Dedekind cut} is a subset $\alpha$ of the rational numbers $\QQ$ that satisfies these
properties:

\begin{enumerate}
\item
$\alpha$ is not empty.

\item
$\mathbb{Q} \setminus \alpha$ is not empty.

\item
$\alpha$ contains no greatest element

\item
For $x, y \in \QQ$, if $x \in \alpha$ and $y < x$, then $y \in \alpha$ as well.
\end{enumerate}
\end{defn}

Dedekind cuts are particularly appealing for two reasons.  First, they make it very easy to
prove the completeness, or continuity of the real line.  Also, they make it quite plain to
distinguish the rationals from the irrationals on the real line, and put the latter on a firm
logical foundation.
In the construction of the real numbers from Dedekind cuts, we make the following definition:

\begin{defn}
A \emph{real number} is a Dedekind cut.  We denote the set of all real numbers by $\RR$ and we
order them by set-theoretic inclusion, that is to say, for any $\alpha, \beta \in \RR$,
\[ \alpha < \beta \mbox{ if and only if } \alpha \subset \beta \]
where the inclusion is strict.  We further define $\alpha = \beta$ as real numbers 
if $\alpha$ and $\beta$ are equal as sets.  As usual, we write $\alpha \leq \beta$ 
if $\alpha < \beta$ or $\alpha=\beta$.  Moreover, a real number $\alpha$ is said to be \emph{irrational}
if $\QQ \setminus \alpha$ contains no least element.
\end{defn}

The Dedekind completeness property of real numbers, expressed as the supremum property, now becomes straightforward 
to prove.  In what follows, we will reserve Greek variables for real numbers, and Roman variables for rationals.

\begin{thm}
Every nonempty subset of real numbers that is bounded above has a least upper bound.
\end{thm}
\begin{proof}
Let $A$ be a nonempty set of real numbers, such that for every $\alpha \in A$ we have that
$\alpha \leq \gamma$ for some real number $\gamma$.  Now define the set 

\[ \sup A =  \bigcup_{\alpha \in A} \alpha. \]

We must show that this set is a real number.  This amounts to checking the four conditions
of a Dedekind cut. 
\begin{enumerate}
\item
$\sup A$ is clearly not empty, for it is the nonempty union of nonempty sets.

\item
Because $\gamma$ is a real number, there is some rational $x$ that is not in $\gamma$.  Since
every $\alpha \in A$ is a subset of $\gamma$, $x$ is not in any $\alpha$, so $x \not\in \sup A$
either.  Thus, $\QQ \setminus \sup A$ is nonempty.

\item
If $\sup A$ had a greatest element $g$, then $g \in \alpha$
for some $\alpha \in A$.  Then $g$ would be a greatest element of $\alpha$, but $\alpha$ is a real number, so 
by contrapositive, $\sup A$ has no greatest element. 
\item
Lastly, if $x \in \sup A$, then $x \in \alpha$ for some $\alpha$, so given any $y < x$ because $\alpha$ 
is a real number $y \in \alpha$, whence $y \in \sup A$. 
\end{enumerate}
Thus, $\sup A$ is a real number. Trivially, $\sup A$ is an upper bound of $A$, for every $\alpha \subseteq \sup A$.
It now suffices to prove that $\sup A \leq \gamma$, because $\gamma$ was an arbitrary
upper bound.  But this is easy, because
every $x \in \sup A$ is an element of $\alpha$ for some $\alpha \in A$, so because $\alpha \subseteq \gamma$, 
$x \in \gamma$.  Thus,
$\sup A$ is the least upper bound of $A$.  We call this real number the \emph{supremum} of A.
\end{proof}

To finish the construction of the real numbers, we must endow them with algebraic operations, define the additive
and multiplicative identity elements, prove that these definitions give a field, and prove further results about the order of the reals (such as the totality of this
order) -- in short, build a complete ordered field.  This task is somewhat laborious, but we include here the
appropriate definitions.  Verifying their correctness can be an instructive, albeit tiresome, exercise.  We use the 
same symbols for the operations on the reals as for the rational numbers; this should cause no confusion in context.

\begin{defn}
Given two real numbers $\alpha$ and $\beta$, we define
\begin{itemize}
\item
 The \emph{additive identity}, denoted $0$, is 
  \[0 := \{ x \in \QQ : x < 0 \} \]
\item
 The \emph{multiplicative identity}, denoted $1$, is
  \[1 := \{x \in \QQ : x < 1 \} \]
\item
 \emph{Addition} of $\alpha$ and $\beta$ denoted $\alpha + \beta$ is
\[ \alpha + \beta := \{x + y : x \in \alpha, ~y \in \beta \} \]
\item
 The \emph{opposite} of $\alpha$, denoted $-\alpha$, is
  \[ -\alpha := \{x \in \QQ: -x \not\in \alpha, \mbox{ but } -x \mbox{ is not the least element of }
    \QQ \setminus \alpha \} \]
\item
 The \emph{absolute value} of $\alpha$, denoted $|\alpha|$, is
\begin{equation*}
|\alpha| :=
 \begin{cases}
   \alpha, &\text{if } ~\alpha \geq 0 \\
  -\alpha, &\text{if } ~\alpha \leq 0
 \end{cases}
\end{equation*}
\item
 If $\alpha, \beta > 0$, then \emph{multiplication} of $\alpha$ and $\beta$, denoted $\alpha \cdot \beta$, is
  \[\alpha \cdot \beta := \{z \in \QQ : z \leq 0 \mbox{ or } z = xy \mbox{ for some } x \in \alpha, ~y \in \beta 
               \mbox{ with } x, y > 0 \} \]
 In general, 
 \begin{equation*}
 \alpha \cdot \beta :=
  \begin{cases}
    0,                    &\text{if } ~\alpha = 0  \text{ or } ~\beta = 0 \\
    |\alpha|\cdot|\beta|  &\text{if } ~\alpha > 0, ~\beta > 0 \text { or } ~\alpha < 0, ~\beta < 0 \\
    -(|\alpha|\cdot|\beta|) &\text{if } ~\alpha > 0, ~\beta < 0 \text { or } ~\alpha > 0, ~\beta < 0
   \end{cases}
  \end{equation*}
\item
The \emph{inverse} of $\alpha > 0$, denoted $\alpha^{-1}$, is
\[
\alpha^{-1} := \{x \in \QQ : x \leq 0 \text{ or } x > 0 \text{ and } (1/x) \not\in \alpha, \text{ but }
                1/x \text{ is not the least element of } \QQ \setminus \alpha \}
\]
If $\alpha < 0$, 
\[ \alpha^{-1} :=  -(|\alpha|)^{-1}  \]
 
\end{itemize}
\end{defn}

All that remains (!) is to check that the above definitions do indeed define a complete ordered field, and that
all the sets implied to be real numbers are indeed so.  
The properties of $\RR$ as an ordered field follow from these definitions and the properties of $\QQ$ as an
 ordered field. 
It is important to point out that in two steps, in showing that inverses and opposites are properly defined, 
we require an extra property of $\QQ$, not merely in its capacity as an ordered field.
This requirement is the Archimedean property.

Moreover, because $\RR$ is a field of characteristic $0$, it contains an isomorphic copy of $\QQ$.  The rationals
correspond to the Dedekind cuts $\alpha$ for which $\QQ \setminus \alpha$ contains a least member.

\begin{thebibliography}{99}
\bibitem{Courant&Robbins}
Courant, Richard and Robbins, Herbert.  {\it What is Mathematics?} pp. 68-72  Oxford University Press, Oxford, 1969
\bibitem{Dedekind}
Dedekind, Richard.  {\it Essays on the Theory of Numbers}  Dover Publications Inc, New York 1963
\bibitem{Rudin}
Rudin, Walter {\it Principles of Mathematical Analysis} pp. 17-21 McGraw-Hill Inc, New York, 1976
\bibitem{Spivak}
Spivak, Michael.  {\it Calculus} pp. 569-596  Publish or Perish, Inc. Houston, 1994

\end{thebibliography}
%%%%%
%%%%%
\end{document}
