\documentclass[12pt]{article}
\usepackage{pmmeta}
\pmcanonicalname{PowerRule}
\pmcreated{2013-03-22 12:28:03}
\pmmodified{2013-03-22 12:28:03}
\pmowner{mathcam}{2727}
\pmmodifier{mathcam}{2727}
\pmtitle{power rule}
\pmrecord{7}{32630}
\pmprivacy{1}
\pmauthor{mathcam}{2727}
\pmtype{Theorem}
\pmcomment{trigger rebuild}
\pmclassification{msc}{26A03}
\pmrelated{ProductRule}
\pmrelated{Derivation}
\pmrelated{Derivative}

\endmetadata

\usepackage{amssymb}
\usepackage{amsmath}
\usepackage{amsfonts}
\newcommand{\D}[1]{\ensuremath{\mathrm{d}#1}}
\newcommand{\DDX}{\ensuremath{\frac{\D{}}{\D{x}}}}
\begin{document}
The \emph{power rule} states that

\begin{equation*}
\DDX x^p = px^{p-1}, \quad p \in \mathbb{R}
\end{equation*}

This rule, when combined with the chain rule, product rule, and sum rule,
makes calculating many derivatives far more tractable.  This rule can be derived by repeated application of the product rule.
See the \PMlinkname{proof of the power rule}{ProofOfPowerRule}.

Repeated use of the above formula gives

\begin{align*}
\frac{d^i}{dx^i}x^k=
\begin{cases}
0&i>k\\
\frac{k!}{(k-i)!}x^{k-i}&i\leq k,
\end{cases}
\end{align*}
for $i,k\in\mathbb{Z}$.

\subsection*{Examples}

\begin{eqnarray*}
\DDX x^0 & = & \frac{0}{x} = 0 = \DDX 1 \\
\DDX x^1 & = & 1x^0 = 1 = \DDX x \\
\DDX x^2 & = & 2x \\
\DDX x^3 & = & 3x^2 \\
\DDX \sqrt{x} & = & \DDX x^{1/2} = \frac{1}{2}x^{-1/2} = \frac{1}{2\sqrt{x}} \\
\DDX 2x^e & = & 2ex^{e-1}
\end{eqnarray*}
%%%%%
%%%%%
\end{document}
