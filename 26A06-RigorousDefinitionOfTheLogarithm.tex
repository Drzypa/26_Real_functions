\documentclass[12pt]{article}
\usepackage{pmmeta}
\pmcanonicalname{RigorousDefinitionOfTheLogarithm}
\pmcreated{2013-03-22 17:00:37}
\pmmodified{2013-03-22 17:00:37}
\pmowner{rspuzio}{6075}
\pmmodifier{rspuzio}{6075}
\pmtitle{rigorous definition of the logarithm}
\pmrecord{18}{39292}
\pmprivacy{1}
\pmauthor{rspuzio}{6075}
\pmtype{Derivation}
\pmcomment{trigger rebuild}
\pmclassification{msc}{26A06}
\pmclassification{msc}{26A09}
\pmclassification{msc}{26-00}

% this is the default PlanetMath preamble.  as your knowledge
% of TeX increases, you will probably want to edit this, but
% it should be fine as is for beginners.

% almost certainly you want these
\usepackage{amssymb}
\usepackage{amsmath}
\usepackage{amsfonts}

% used for TeXing text within eps files
%\usepackage{psfrag}
% need this for including graphics (\includegraphics)
%\usepackage{graphicx}
% for neatly defining theorems and propositions
\usepackage{amsthm}
% making logically defined graphics
%%%\usepackage{xypic}

% there are many more packages, add them here as you need them

% define commands here
\newtheorem{dfn}{Definition}
\newtheorem{thm}{Theorem}
\begin{document}
In this entry, we shall construct the logarithm as a Dedekind cut
and then demonstrate some of its basic properties.  All that is
required in the way of background material are the properties of
integer powers of real numbers.

\begin{thm}
Suppose that $a,b,c,d$ are positive integers such that
$a/b = c/d$ and that $x>0$ and $y>0$ are  real numbers. 
Then $x^a \le y^b$ if and only if $x^c \le y^d$.
\end{thm}

\begin{proof}
Cross multiplying, the condition $a/b = c/d$ is equivalent to $ad = bc$.
By elementary properties of powers, $x^a \le y^b$ if and only if
$x^{ad} \le y^{bd}$.  Likewise, $x^c \le x^d$ if and only if $x^{bc} \le
y^{bd}$ which, since $bc = ad$, is equivalent to $x^{ad} \le y^{bd}$.
Hence, $x^a \le y^b$ if and only if $x^c \le x^d$.
\end{proof}

\begin{thm}
Suppose that $a,b,c,d$ are positive integers such that
$a/b \le c/d$ and that $x>1$ and $y>0$ are real numbers.  
If $x^c \le y^d$ then $x^a \le y^b$.
\end{thm}

\begin{proof}
Since we assumed that $b > 0$, we have that $x^c \le y^d$ is equivalent
to $x^{bc} \le y^{bd}$.  Likewise, since $d > 0$, we have that $x^a \le 
y^b$ is equivalent to $x^{ad} \le y^{bd}$.  Cross-multiplying, $a/b \le c/d$
is equivalent to $ad \le bc$.  Since $x > 1$, we have $x^{ad} \le x^{bc}$.
Combining the above statements, we conclude that $x^c \le y^d$ implies
$x^a \le y^b$.
\end{proof}

\begin{thm}
Suppose that $a,b,c,d$ are positive integers such that
$a/b > c/d$ and that $x>1$ and $y>0$ are real numbers.  
If $x^a > y^b$ then $x^c > y^d$.
\end{thm}

\begin{proof}
Since we assumed that $b > 0$, we have that $x^c > y^d$ is equivalent
to $x^{bc} > y^{bd}$.  Likewise, since $d > 0$, we have that $x^a > 
y^b$ is equivalent to $x^{ad} > y^{bd}$.  Cross-multiplying, $a/b > c/d$
is equivalent to $ad > bc$.  Since $x > 1$, we have $x^{ad} > x^{bc}$.
Combining the above statements, we conclude that $x^c > y^d$ implies
$x^a > y^b$.
\end{proof}

\begin{thm}
Let $x>1$ and $y$ be real numbers.
Then there exists an integer $n$ such that $x^n > y$.
\end{thm}

\begin{proof}
Write $x = 1 + h$.  Then we have $(1 + h)^n \ge 1 + n h$ for all
positive integers $n$.  This fact is easily proved by induction.
When $n = 1$, it reduces to the triviality $1 + h \ge h$.  If
$(1 + h)^n \ge 1 + n h$, then
\[
(1 + h)^{n+1} = (1 + h) (1 + h)^n \ge (1 + h) (1 + n h) =
1 + (n+1) h + n h^2 \ge 1 + (n+1) h.
\]
By the Archimedean property, there exists an integer $n$ such
that $1 + nh > y$, so $x^n > y$.
\end{proof}

\begin{thm}
Let $x>1$ and $y$ be real numbers. 
Then the pair of sets $(L,U)$ where
\begin{align}
L &= \{r \in \mathbb{Q} \mid (\exists a, b \in \mathbb{Z}) \quad
b > 0 ~\land~ r = a/b ~\land~ x^a \le y^b\} \\
U &= \{r \in \mathbb{Q} \mid (\exists a, b \in \mathbb{Z}) \quad
b > 0 ~\land~ r = a/b ~\land~ x^a > y^b\}
\end{align}
forms a Dedekind cut.
\end{thm}

\begin{proof}
Let $r$ be any rational number.  Then we have $r = a/b$ for some integers
$a$ and $b$ such that $b > 0$.  The possibilities $x^a \le y^b$ and
$x^a > y^b$ are exhaustive so $r$ must belong to at least one of $U$ and
$L$.  By theorem 1, it cannot belong to both.  By theorem 2, if $r \in L$
and $s \le r$, then $s \in L$ as well.  By theorem 3, if $r \in U$
and $s > r$, then $s \in U$ as well.  By theorem 4, neither $L$ nor $U$
are empty.  Hence, $(L,U)$ is a Dedekind cut and defines a
real number.
\end{proof}

\begin{dfn}
Suppose $x>1$ and $y>0$ are real numbers.   Then,
we define $\log_x y$ to be the real number defined by the cut $(L,U)$ of
the above theorem.
\end{dfn}
%%%%%
%%%%%
\end{document}
