\documentclass[12pt]{article}
\usepackage{pmmeta}
\pmcanonicalname{AdditionAndSubtractionFormulasForSineAndCosine}
\pmcreated{2013-03-22 16:59:01}
\pmmodified{2013-03-22 16:59:01}
\pmowner{Wkbj79}{1863}
\pmmodifier{Wkbj79}{1863}
\pmtitle{addition and subtraction formulas for sine and cosine}
\pmrecord{16}{39260}
\pmprivacy{1}
\pmauthor{Wkbj79}{1863}
\pmtype{Derivation}
\pmcomment{trigger rebuild}
\pmclassification{msc}{26A09}
\pmclassification{msc}{15-00}
\pmclassification{msc}{33B10}
\pmsynonym{addition and subtraction formulae for sine and cosine}{AdditionAndSubtractionFormulasForSineAndCosine}
\pmsynonym{addition formulas for sine and cosine}{AdditionAndSubtractionFormulasForSineAndCosine}
\pmsynonym{addition formulae for sine and cosine}{AdditionAndSubtractionFormulasForSineAndCosine}
\pmsynonym{subtraction formulas for sine and cosine}{AdditionAndSubtractionFormulasForSineAndCosine}
\pmsynonym{subtraction formulae for sine and cosine}{AdditionAndSubtractionFormulasForSineAndCosine}
\pmsynonym{addition formula for sine}{AdditionAndSubtractionFormulasForSineAndCosine}
\pmsynonym{subtraction}{AdditionAndSubtractionFormulasForSineAndCosine}
\pmrelated{AdditionFormula}
\pmrelated{DefinitionsInTrigonometry}
\pmrelated{DoubleAngleIdentity}
\pmrelated{MeanCurvatureAtSurfacePoint}
\pmrelated{DAlembertAndDBernoulliSolutionsOfWaveEquation}
\pmrelated{AdditionFormulas}

\endmetadata

\usepackage{amssymb}
\usepackage{amsmath}
\usepackage{amsfonts}

\usepackage{psfrag}
\usepackage{graphicx}
\usepackage{amsthm}
%%\usepackage{xypic}

\begin{document}
The rotation matrix $\displaystyle \left( \begin{array}{lr}
\cos \theta & -\sin \theta \\
\sin \theta & \cos \theta \end{array} \right)$ will be used to obtain the addition formulas for sine and cosine.

Recall that a vector in $\mathbb{R}^2$ can be rotated $\theta$ radians in the counterclockwise direction by multiplying on the left by the rotation matrix $\displaystyle \left( \begin{array}{lr}
\cos \theta & -\sin \theta \\
\sin \theta & \cos \theta \end{array} \right)$.  Because rotating by $\alpha+\beta$ radians is the same as rotating by $\beta$ radians followed by rotating by $\alpha$ radians, we obtain:

\begin{center}
$\begin{array}{rl}
\displaystyle \left( \begin{array}{lr}
\cos ( \alpha + \beta ) & -\sin ( \alpha + \beta ) \\
\sin ( \alpha + \beta ) & \cos ( \alpha + \beta ) \end{array} \right) & =\displaystyle \left( \begin{array}{lr}
\cos \alpha & -\sin \alpha \\
\sin \alpha & \cos \alpha \end{array} \right) \left( \begin{array}{lr}
\cos \beta & -\sin \beta \\
\sin \beta & \cos \beta \end{array} \right) \\
& \\
& =\displaystyle \left( \begin{array}{lr}
\cos \alpha \cos \beta -\sin \alpha \sin \beta & -\cos \alpha \sin \beta -\sin \alpha \cos \beta \\
\sin \alpha \cos \beta +\cos \alpha \sin \beta & -\sin \alpha \sin \beta +\cos \alpha \cos \beta \end{array} \right)
\end{array}$
\end{center}

Hence, $\sin ( \alpha + \beta )=\sin \alpha \cos \beta +\cos \alpha \sin \beta$ and $\cos ( \alpha + \beta )=\cos \alpha \cos \beta -\sin \alpha \sin \beta$.

Note that sine is an even function and that cosine is an odd function, \PMlinkname{i.e.}{Ie} $\sin(-x)=-\sin x$ and $\cos(-x)=-\cos x$.  These facts enable us to obtain the subtraction formulas for sine and cosine.

$$\sin(\alpha-\beta)=\sin(\alpha+(-\beta))=\sin(\alpha)\cos(-\beta)+\cos(\alpha)\sin(-\beta)=\sin(\alpha)\cos(\beta)-\cos(\alpha)\sin(\beta)$$

$$\cos(\alpha-\beta)=\cos(\alpha+(-\beta))=\cos(\alpha)\cos(-\beta)-\sin(\alpha)\sin(-\beta)=\cos(\alpha)\cos(\beta)+\sin(\alpha)\sin(\beta)$$
%%%%%
%%%%%
\end{document}
