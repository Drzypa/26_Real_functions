\documentclass[12pt]{article}
\usepackage{pmmeta}
\pmcanonicalname{Divergence}
\pmcreated{2013-03-22 12:55:08}
\pmmodified{2013-03-22 12:55:08}
\pmowner{rmilson}{146}
\pmmodifier{rmilson}{146}
\pmtitle{divergence}
\pmrecord{11}{33271}
\pmprivacy{1}
\pmauthor{rmilson}{146}
\pmtype{Definition}
\pmcomment{trigger rebuild}
\pmclassification{msc}{26B12}
\pmrelated{SourcesAndSinksOfVectorField}
\pmdefines{incompressible}
\pmdefines{divergence theorem}
\pmdefines{Gauss's theorem}

\usepackage{amsmath}
\usepackage{amsfonts}
\usepackage{amssymb}
\newcommand{\reals}{\mathbb{R}}
\newcommand{\natnums}{\mathbb{N}}
\newcommand{\cnums}{\mathbb{C}}
\newcommand{\znums}{\mathbb{Z}}
\newcommand{\lp}{\left(}
\newcommand{\rp}{\right)}
\newcommand{\lb}{\left[}
\newcommand{\rb}{\right]}
\newcommand{\supth}{^{\text{th}}}
\newtheorem{proposition}{Proposition}
\newtheorem{definition}[proposition]{Definition}

\newtheorem{theorem}[proposition]{Theorem}


\newcommand{\vi}{\mathbf{i}}
\newcommand{\vj}{\mathbf{j}}
\newcommand{\vk}{\mathbf{k}}
\newcommand{\be}{\mathbf{e}}

\newcommand{\vF}{\mathbf{F}}
\newcommand{\vN}{\mathbf{N}}


\newcommand{\bV}{\mathbf{V}}
\newcommand{\vnabla}{\nabla}

\newcommand{\vdiv}{\operatorname{div}}
\begin{document}
\paragraph{Basic Definition.}
Let $x,y,z$ be a system of Cartesian coordinates on $3$-dimensional
Euclidean space, and let $\vi, \vj, \vk$ be the
corresponding basis of unit vectors.  The \emph{divergence} of a continuously
differentiable vector field
$$\vF = F^1\vi+F^2\vj+F^3\vk,$$
is defined to be the function
$$\vdiv\vF=
\frac{\partial F^1}{\partial x}+
\frac{\partial F^2}{\partial y}+
\frac{\partial F^3}{\partial z}.$$
Another common notation for the divergence
is $\vnabla\cdot\vF$ (see gradient), a convenient mnemonic.

\paragraph{Physical interpretation.}
In physical \PMlinkescapetext{terms}, the divergence of a vector field
is the extent to which the vector field flow behaves like a source or
a sink at a given point.  Indeed, an alternative, but logically
equivalent definition, gives the divergence as the derivative of the
net flow of the vector field across the surface of a small sphere
relative to the surface area of the sphere.  To wit,
$$(\vdiv \vF)(p)= \lim_{r\rightarrow 0}
\int_{S} \!\!(\vF \cdot \vN)dS\;/\left(4 \pi r^2\right),
$$
where $S$ denotes the sphere of radius $r$ about a point
$p\in\reals^3$, and the integral is a surface integral taken with
respect to $\vN$, the normal to that sphere.  

The non-infinitesimal interpretation of divergence is given by Gauss's
Theorem. This theorem is a conservation law, stating that the volume total of
all sinks and sources, i.e. the volume integral of the divergence, is
equal to the net flow across the volume's boundary.  In symbols,
$$\int_V \vdiv \vF \, dV = \int_S (\vF\cdot \vN)\, dS,$$
where
$V\subset\reals^3$ is a compact region with a smooth boundary, and
$S=\partial V$ is that boundary oriented by outward-pointing normals.
We note that Gauss's theorem follows from the more general Stokes'
Theorem, which itself generalizes the fundamental theorem of calculus.

In light of the physical interpretation, a vector field with constant
zero divergence is called \emph{incompressible} -- in this case, no
\PMlinkescapetext{net} flow can occur across any
\PMlinkescapetext{closed} surface.

\paragraph{General definition.}
% By Clairaut's Theorem, the divergence operator annihilates the
% curl of any vector field which has continuous second partial
% derivatives.
The notion of divergence has meaning in the more general setting of
Riemannian geometry.  To that end, let $\bV$ be a vector field on a
Riemannian manifold.  The covariant derivative of $\bV$ is a type
$(1,1)$ tensor field. We define the \emph{divergence} of $\bV$ to be the
trace of that field.  In terms of coordinates (see tensor and Einstein
summation convention), we have
$$\vdiv \bV = V^i{}_{;i} \ .$$
%%%%%
%%%%%
\end{document}
