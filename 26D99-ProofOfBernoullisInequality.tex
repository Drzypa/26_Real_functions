\documentclass[12pt]{article}
\usepackage{pmmeta}
\pmcanonicalname{ProofOfBernoullisInequality}
\pmcreated{2013-03-22 12:38:14}
\pmmodified{2013-03-22 12:38:14}
\pmowner{danielm}{240}
\pmmodifier{danielm}{240}
\pmtitle{proof of Bernoulli's inequality}
\pmrecord{6}{32900}
\pmprivacy{1}
\pmauthor{danielm}{240}
\pmtype{Proof}
\pmcomment{trigger rebuild}
\pmclassification{msc}{26D99}

% this is the default PlanetMath preamble.  as your knowledge
% of TeX increases, you will probably want to edit this, but
% it should be fine as is for beginners.

% almost certainly you want these
\usepackage{amssymb}
\usepackage{amsmath}
\usepackage{amsfonts}

% used for TeXing text within eps files
%\usepackage{psfrag}
% need this for including graphics (\includegraphics)
%\usepackage{graphicx}
% for neatly defining theorems and propositions
%\usepackage{amsthm}
% making logically defined graphics
%%%\usepackage{xypic}

% there are many more packages, add them here as you need them

% define commands here
\begin{document}
Let $I$ be the interval $(-1, \infty)$ and 
$f :I \rightarrow \mathbb R$ the function defined as:
\[
   f(x) = (1 + x)^\alpha - 1 - \alpha x
\]
with $\alpha \in \mathbb R \setminus \lbrace 0, 1 \rbrace$ fixed.
Then $f$ is differentiable and its derivative is
\[
   f'(x) = \alpha (1 + x)^{\alpha - 1} - \alpha,
   \mbox{ for all } x \in I,
\]
from which it follows that $f'(x) = 0 \Leftrightarrow x = 0$.
\begin{enumerate}
\item
If $0 < \alpha < 1$ then $f'(x) < 0$ for all $x \in (0, \infty)$
and $f'(x) > 0$ for all $x \in (-1, 0)$ which means that $0$ is a
global maximum point for $f$.
Therefore 
$f(x) < f(0)$ for all $x \in I \setminus \lbrace 0 \rbrace$
which means that
$(1 + x)^\alpha < 1 + \alpha x$ for all $x \in (-1, 0)$.
\item
If $\alpha \notin [0, 1]$ then $f'(x) > 0$ for all $x \in (0, \infty)$
and $f'(x) < 0$ for all $x \in (-1, 0)$ meaning that $0$ is a global
minimum point for $f$.
This implies that 
$f(x) > f(0)$ for all $x \in I \setminus \lbrace 0 \rbrace$
which means that 
$(1 + x)^\alpha > 1 + \alpha x$ for all $x \in (-1, 0)$.
\end{enumerate}

Checking that the equality is satisfied for $x = 0$ or for \( \alpha \in
\lbrace 0, 1 \rbrace \) ends the proof.
%%%%%
%%%%%
\end{document}
