\documentclass[12pt]{article}
\usepackage{pmmeta}
\pmcanonicalname{ProofOffracftfstsleqfracfufsusleqfracfuftutForConvexF}
\pmcreated{2013-03-22 18:25:34}
\pmmodified{2013-03-22 18:25:34}
\pmowner{yesitis}{13730}
\pmmodifier{yesitis}{13730}
\pmtitle{proof of $\frac{f(t)-f(s)}{t-s}\leq\frac{f(u)-f(s)}{u-s}\leq\frac{f(u)-f(t)}{u-t}$ for convex $f$}
\pmrecord{6}{41080}
\pmprivacy{1}
\pmauthor{yesitis}{13730}
\pmtype{Proof}
\pmcomment{trigger rebuild}
\pmclassification{msc}{26A51}

\endmetadata

% this is the default PlanetMath preamble.  as your knowledge
% of TeX increases, you will probably want to edit this, but
% it should be fine as is for beginners.

% almost certainly you want these
\usepackage{amssymb}
\usepackage{amsmath}
\usepackage{amsfonts}

% used for TeXing text within eps files
%\usepackage{psfrag}
% need this for including graphics (\includegraphics)
%\usepackage{graphicx}
% for neatly defining theorems and propositions
%\usepackage{amsthm}
% making logically defined graphics
%%%\usepackage{xypic}

% there are many more packages, add them here as you need them

% define commands here

\begin{document}
We will prove
\begin{equation}\label{3}
    \displaystyle
    \frac{f(t)-f(s)}{t-s}\leq\frac{f(u)-f(s)}{u-s}.
\end{equation}
The proof of the right-most inequality is similar.

Suppose (\ref{3}) does not hold. Then for some $s, t, u$,
\begin{equation}\label{2}
    \displaystyle
    \frac{f(t)-f(s)}{t-s}>\frac{f(u)-f(s)}{u-s}.
\end{equation}
This inequality is just the statement of the slope of the line
segment $\overline{AB}, A=(t, f(t)), B=(s, f(s))$, being larger than
the slope of the segment $\overline{CB}, C=(u, f(u))$. Since $t$ is
between $s$ and $u$, and $f$ is continuous, this implies
\begin{equation}
    f(t)>h(x)=\frac{f(u)-f(s)}{u-s}(x-s)+f(s),
\end{equation}
$s<x<u$. This contradicts convexity of $f$ on $(a, b)$. Hence,
(\ref{2}) is false and (\ref{3}) follows.

Note that we have tacitly use the fact that $x=\lambda u + (1-\lambda)s$ and $h(x)=\lambda
f(u)+(1-\lambda)f(s)$ for some $\lambda$. 
%%%%%
%%%%%
\end{document}
