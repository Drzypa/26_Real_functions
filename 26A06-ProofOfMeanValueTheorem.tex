\documentclass[12pt]{article}
\usepackage{pmmeta}
\pmcanonicalname{ProofOfMeanValueTheorem}
\pmcreated{2013-03-22 12:40:57}
\pmmodified{2013-03-22 12:40:57}
\pmowner{Andrea Ambrosio}{7332}
\pmmodifier{Andrea Ambrosio}{7332}
\pmtitle{proof of mean value theorem}
\pmrecord{5}{32960}
\pmprivacy{1}
\pmauthor{Andrea Ambrosio}{7332}
\pmtype{Proof}
\pmcomment{trigger rebuild}
\pmclassification{msc}{26A06}

% this is the default PlanetMath preamble.  as your knowledge
% of TeX increases, you will probably want to edit this, but
% it should be fine as is for beginners.

% almost certainly you want these
\usepackage{amssymb}
\usepackage{amsmath}
\usepackage{amsfonts}

% used for TeXing text within eps files
%\usepackage{psfrag}
% need this for including graphics (\includegraphics)
%\usepackage{graphicx}
% for neatly defining theorems and propositions
%\usepackage{amsthm}
% making logically defined graphics
%%%\usepackage{xypic}

% there are many more packages, add them here as you need them

% define commands here
\begin{document}
Define $h(x)$ on $[a,b]$ by
\[ h(x) = f(x) - f(a) - \left( \frac{f(b)-f(a)}{b-a} \right) (x-a) \]
Clearly, $h$ is continuous on $[a,b]$, differentiable on $(a, b)$, and
\[ \begin{array}{ccl}
h(a) & = & f(a)-f(a)=0 \\
h(b) & = & f(b)-f(a)-\left( \frac{f(b)-f(a)}{b-a}\right)(b-a) = 0\\
\end{array} \]
Notice that $h$ satisfies the conditions of Rolle's Theorem.  Therefore, by Rolle's Theorem there exists $c \in (a,b)$ such that $h'(c)=0$.\\
%
However, from the definition of $h$ we obtain by differentiation that
\[ h'(x) = f'(x) - \frac{f(b)-f(a)}{b-a} \]
Since $h'(c)=0$, we therefore have
\[ f'(c) = \frac{f(b)-f(a)}{b-a} \]
as required.

\begin{thebibliography}{1}
\bibitem{spivak} Michael Spivak, {\em Calculus}, 3rd ed., Publish or Perish Inc., 1994. 
\end{thebibliography}
%%%%%
%%%%%
\end{document}
