\documentclass[12pt]{article}
\usepackage{pmmeta}
\pmcanonicalname{SolutionsOfXyYx}
\pmcreated{2014-12-16 16:29:29}
\pmmodified{2014-12-16 16:29:29}
\pmowner{pahio}{2872}
\pmmodifier{pahio}{2872}
\pmtitle{solutions of $x^y = y^x$}
\pmrecord{13}{42127}
\pmprivacy{1}
\pmauthor{pahio}{2872}
\pmtype{Result}
\pmcomment{trigger rebuild}
\pmclassification{msc}{26D20}
\pmclassification{msc}{26B35}
\pmclassification{msc}{26A09}
\pmclassification{msc}{11D61}
\pmsynonym{equation $x^y = y^x$}{SolutionsOfXyYx}
%\pmkeywords{exponentiation}
%\pmkeywords{power}
\pmrelated{CatalansConjecture}
\pmrelated{RationalAndIrrational}
\pmrelated{PerfectPower}

% this is the default PlanetMath preamble.  as your knowledge
% of TeX increases, you will probably want to edit this, but
% it should be fine as is for beginners.

% almost certainly you want these
\usepackage{amssymb}
\usepackage{amsmath}
\usepackage{amsfonts}

% used for TeXing text within eps files
%\usepackage{psfrag}
% need this for including graphics (\includegraphics)
%\usepackage{graphicx}
% for neatly defining theorems and propositions
 \usepackage{amsthm}
% making logically defined graphics
%%%\usepackage{xypic}

% there are many more packages, add them here as you need them

% define commands here

\theoremstyle{definition}
\newtheorem*{thmplain}{Theorem}

\begin{document}
The equation
\begin{align}
x^y \;=\; y^x
\end{align}
has trivial solutions on the line \,$y = x$.\, For other solutions one has the\\

\textbf{Theorem.}\, $1^\circ$. The only positive solutions of the equation (1) with\,  $1 < x < y$\, are in a parametric form
\begin{align}
x \;=\; (1\!+\!u)^{\frac{1}{u}}, \qquad y \;=\; (1\!+\!u)^{\frac{1}{u}+1}
\end{align}
where\, $u > 0$.\\
$2^\circ$. The only rational solutions of (1) are
\begin{align}
x \;=\; \left(1\!+\!\frac{1}{n}\right)^n\!, \qquad y \;=\; \left(1\!+\!\frac{1}{n}\right)^{n+1}
\end{align}
where\, $n = 1,\,2,\,3,\,\ldots$\\
$3^\circ$. Consequently, the only integer solution of (1) is
$$2^4 \;=\; 16 \;=\; 4^2.$$\\


\emph{Proof.} $1^\circ$. Let\, $(x,\,y)$\, be a solution of (1) with\, $1 < x < y$.\, Set\, $y = x\!+\!\delta$\, 
($\delta > 0$).\, Now
$$x^{x+\delta} \;=\; (x\!+\!\delta)^x,$$
from which we obtain easily
$$x \;=\; \left(1\!+\!\frac{\delta}{x}\right)^{\frac{x}{\delta}} \;:=\; (1\!+\!u)^{\frac{1}{u}},$$
where\, $u = \frac{\delta}{x}$.\, Then 
$$y \;=\; x\!+\!\delta \;=\; x\!\left(1\!+\!\frac{\delta}{x}\right) 
\;=\; (1\!+\!u)^{\frac{1}{u}}(1\!+\!u) \;=\; (1\!+\!u)^{\frac{1}{u}+1}.$$
$2^\circ$. The unit fractions\, $u = \frac{1}{n}$\, yield from (2) rational solutions (3).
Further, no irrational value of $u$ cannot make both $x$ and $y$ of (2) rational, since otherwise the ratio $1\!+\!u$ of the latter numbers would be irrational (cf. rational and irrational).\, Accordingly, for other rational solutions than (3), we must consider the values
$$u \;:=\; \frac{m}{n}$$
with coprime positive integers $m,\,n$ where\, $m > 1$.\, Make the antithesis that
$$x \;=\; \left(1\!+\!\frac{m}{n}\right)^{\frac{n}{m}} \;\in\; \mathbb{Q}.$$
Because the integers coprime with $m$ form a group with respect to the multiplication modulo $m$ (cf. prime residue classes), the congruence
$$nz \;\equiv\; 1 \pmod{m}$$
has a solution $z$.\, Thus we may write\, $nz = km\!+\!1$\, and rewrite the rational number
\begin{align}
\left[\left(1\!+\!\frac{m}{n}\right)^{\frac{n}{m}}\right]^z \;=\; \left(1\!+\!\frac{m}{n}\right)^{\frac{nz}{m}} 
\;=\; \left(1\!+\!\frac{m}{n}\right)^{\frac{km+1}{m}} 
\;=\; \left(1\!+\!\frac{m}{n}\right)^k\left(1\!+\!\frac{m}{n}\right)^{\frac{1}{m}}.
\end{align}
This product form tells that $\left(1\!+\!\frac{m}{n}\right)^{\frac{1}{m}}$ is rational.\, But the number
$$\left(1\!+\!\frac{m}{n}\right)^{\frac{1}{m}} \;=\; \sqrt[m]{\frac{m\!+\!n}{n}}$$ 
cannot be rational without the coprime integers $m\!+\!n$ and $n$ both being $m^{\mbox{th}}$ \PMlinkname{powers}{GeneralAssociativity}.\, If we had\, $n = v^m$,\, then by Bernoulli inequality,
$$(v\!+\!1)^m \;>\; v^m\!+\!mv \geqq n\!+\!m,$$
i.e. $m\!+\!n$ could not be a $m^{\mbox{th}}$ power.\, The contradictory situation means, by (4), that the antithesis is wrong.\, Therefore, the numbers (3) give the only rational solutions of (1).\\


\textbf{Note.}\, The value\, $n = 2$\, in (3) produces\, $x = \frac{9}{4}$,\, $y = \frac{27}{8}$,\, whence (1) reads
\begin{align}
\left(\frac{9}{4}\right)^{\frac{27}{8}} \;=\; \left(\frac{27}{8}\right)^{\frac{9}{4}}\!.
\end{align}
The truth of the equality (5) may also be checked by the calculation
$$\left(\frac{9}{4}\right)^{\frac{27}{8}} \;=\; \left[\left(\frac{9}{4}\right)^{\frac{1}{2}}\right]^{\frac{27}{4}} 
\;=\; \left(\frac{3}{2}\right)^{\frac{27}{4}} \;=\; \left[\left(\frac{3}{2}\right)^3\right]^{\frac{9}{4}} 
\;=\; \left(\frac{27}{8}\right)^{\frac{9}{4}}\!.$$


\begin{thebibliography}{8}
\bibitem{HG}{\sc P. Hohler \& P. Gebauer}:\, Kann man ohne Rechner entscheiden, ob $e^\pi$ oder $\pi^e$ gr\"osser ist? $-$ \emph{Elemente der Mathematik} \textbf{36} (1981).
\bibitem{SM}{\sc J. Sondow \& D. Marques}:\, Algebraic and transcendental solutions
of some exponential equations.\, $-$ \emph{Annales Mathematicae et Informaticae} \textbf{37} (2010); available directly at \PMlinkexternal{arXiv}{http://arxiv.org/pdf/1108.6096.pdf}.
\end{thebibliography}

%%%%%
%%%%%
\end{document}
