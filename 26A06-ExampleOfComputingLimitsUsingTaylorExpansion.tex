\documentclass[12pt]{article}
\usepackage{pmmeta}
\pmcanonicalname{ExampleOfComputingLimitsUsingTaylorExpansion}
\pmcreated{2013-03-22 15:39:48}
\pmmodified{2013-03-22 15:39:48}
\pmowner{stevecheng}{10074}
\pmmodifier{stevecheng}{10074}
\pmtitle{example of computing limits using Taylor expansion}
\pmrecord{15}{37597}
\pmprivacy{1}
\pmauthor{stevecheng}{10074}
\pmtype{Example}
\pmcomment{trigger rebuild}
\pmclassification{msc}{26A06}

\endmetadata

\usepackage{amssymb}
\usepackage{amsmath}
\usepackage{amsfonts}
\usepackage{amsthm}

% used for TeXing text within eps files
%\usepackage{psfrag}
% need this for including graphics (\includegraphics)
%\usepackage{graphicx}
% making logically defined graphics
%%%\usepackage{xypic}

% define commands here
\newcommand{\real}{\mathbb{R}}

\providecommand{\abs}[1]{\lvert#1\rvert}
\providecommand{\absW}[1]{\left\lvert#1\right\rvert}
\providecommand{\absB}[1]{\Bigl\lvert#1\Bigr\rvert}

\newtheorem*{prob}{Problem}
\begin{document}
\PMlinkescapeword{sounds}
\PMlinkescapeword{derivations}
\PMlinkescapeword{sort}

\begin{prob}
Evaluate
\[
\lim_{x \to -\infty} \bigl( x + \sqrt{x^2 + 23 x} \bigr)\,.
\]
\end{prob}

In beginners' courses in calculus, one is usually told to 
rationalize the above expression by multiplying by $
(x-\sqrt{x^2+ 23x})/(x-\sqrt{x^2+23 x})$.
This approach is somewhat dissatisfying, because it depends
on a specific \emph{algebraic} trick that works only for square roots
 --- for example, it would have been more difficult or impossible to rationalize, if instead, we had a cube root or even a transcendental function   ---
and this trick does not appeal to our intuition that
$\sqrt{x^2 + 23 x}$ should be approximately 
$\abs{x}$ for large $\abs{x}$.
Fortunately, there  is another approach that
exploits the \emph{analytic} properties of the functions involved.

\PMlinkname{L'H\^{o}pital's Rule}{LHpitalsRule} is one analytic approach, 
but in many cases, using Taylor expansion is even easier
and straightforward. Essentially, Taylor expansion approximates 
complicated functions by polynomials, whose limits are easy to evaluate.
We illustrate the method below.

First rewrite
\[
\lim_{x \to -\infty} \bigl(x+\sqrt{x^2 + 23 x} \bigr)
= \lim_{x \to +\infty} \bigl(\sqrt{x^2 - 23 x} - x \bigr) \,,
\]
so  that we do not have to worry about the pesky negatives any more.
Then, with the help of the binomial formula:
\[
(1+y)^{1/2} = 1 + \frac{1}{2} y + o(y)\,,\quad \textrm{as $y \to 0$,}
\]
(``$o$'' is Landau notation)
we obtain:
\begin{align*}
\sqrt{x^2 - 23 x} - x  
&= x \left( \sqrt{1 - \frac{23}{x}} - 1 \right) \\
&= x \left( 1 - \frac{1}{2} \frac{23}{x} + o\left(\frac{23}{x}\right) - 1 \right)\,,
\quad \textrm{as $x \to \infty$ (so $y = -\frac{23}{x} \to 0$)}\\
&= -\frac{23}{2} + x \, o\left(\frac{23}{x}\right) \\
&= -\frac{23}{2} + o(1)\,.
\end{align*}
Therefore
\[
\lim_{x \to -\infty} \bigl( x + \sqrt{x^2 + 23 x} \bigr) = \lim_{x\to\infty} \bigl( \sqrt{x^2 - 23 x} - x\bigr) = -\frac{23}{2}\,.
\]

\begin{prob}
Evaluate 
\[
\lim_{x \to 0} \left( \frac{1}{\ln(1+ x)} - \frac{1}{\tan x} \right)\,.
\]
\end{prob}
This example is admittedly artificial; it was made to be annoying to solve using
L'H\^{o}pital's Rule alone, but much simpler if one knows how to use 
the Taylor expansions:
\begin{align*}
\ln(1+x) &= x - \frac{1}{2} x^2 + o(x^2)\,, &  \text{as $x \to 0$, and} \\
\tan x &= x + o(x^2)\,, & \text{as $x \to 0$.}
\end{align*}
So we compute:
\begin{align*}
\frac{1}{\ln (1 + x) } - \frac{1}{\tan x}
= \frac{\tan x - \ln(1 + x)}{ \ln (1 + x) \tan x} 
&= \frac{\bigl(x + o(x^2)\bigr) - \bigl( x - \frac{1}{2} x^2 + o(x^2) \bigr)}{ \ln(1+x) \tan x} \\
&= \frac{\frac{1}{2} x^2 + o(x^2)}{\bigl(x + o(x) \bigr) \bigl( x + o(x) \bigr) } \\
&= \frac{\frac{1}{2} + o(1)}{\bigl(1 + o(1) \bigr) \bigl(1+o(1) \bigr)} \\
\end{align*}
Therefore, 
\[
\lim_{x \to 0} \left( \frac{1}{\ln(1+ x)} - \frac{1}{\tan x} \right) = \frac{1}{2}\,.
\]
The reader might reasonably ask how
did we know the right number of terms to use in the Taylor expansions.
The answer is to guess.
This is not as problematic as it sounds.  First, there is no harm
in using more terms than necessary in the expansion (only that there is more writing).
And if we used too few terms, we would know when we later encounter indeterminate forms such as $o(1)/x$ (as $x \to 0$) in our derivations.  If that happens, 
it is not hard to go back and add the needed terms.

Notice that all the essential information to evaluate the limit is contained in the first few derivatives of 
the functions involved \emph{at particular points} --- in the  above example, 
only at $x = 0$.
This information can be obtained  by manipulating series, 
unlike L'H\^{o}pital's Rule
which necessitates computing the derivative \emph{functions} at all points.
So even monstrous expressions like this one is tractable
with Taylor expansion:
\[
\lim_{x \to 0} \left( \frac{1}{\ln(1+ \tan(\sin x))} - \frac{1}{e^{\sin(\tan x)}  -1 }\right)\,.
\]

On the other hand, there definitely are situations where 
L'H\^{o}pital's Rule works but Taylor expansion does not: for instance,
\[
\lim_{x \to 0^+} x \ln x = \lim_{x \to 0^+} \frac{x}{1/\ln x}\,,
\]
because $1/\ln x$ cannot be expanded in a Taylor series about $x =0 $.

\begin{prob}
Here is a problem involving a different sort of limit:
does the following series converge?
\[
\sum_{n=1}^\infty \arctan(n^{-1})
\]
\end{prob}
Our intuition suggests that it does not, because 
$\arctan (n^{-1})$ should be approximately $n^{-1}$, and $\sum_n n^{-1}$  diverges.
However, the standard comparison test does not work
because $\arctan x \leq x$ (for $x \geq 0$) has the inequality in the wrong direction.
But with Taylor expansion the solution is a snap.
By expanding
\[
\arctan x = x + O(x^3) \,, \quad \text{as $x \to 0$.}
\]
and summing both sides, we get
\[
\sum_{n=1}^\infty \arctan (n^{-1} ) = \sum_{n=1}^\infty n^{-1} + \sum_{n=1}^\infty O(n^{-3})\,.
\]
As $\sum_n O(n^{-3})$ converges (being dominated by $C\sum_n n^{-3}$ for some constant $C$), $\sum_n \arctan(n^{-1})$ must diverge (to $\infty$).

(Of course, this problem could be solved by using the integral test,
but who really wants to integrate $\int \arctan (1/x) \, dx$?)
%%%%%
%%%%%
\end{document}
