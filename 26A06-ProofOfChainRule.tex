\documentclass[12pt]{article}
\usepackage{pmmeta}
\pmcanonicalname{ProofOfChainRule}
\pmcreated{2013-03-22 12:41:48}
\pmmodified{2013-03-22 12:41:48}
\pmowner{n3o}{216}
\pmmodifier{n3o}{216}
\pmtitle{proof of chain rule}
\pmrecord{6}{32978}
\pmprivacy{1}
\pmauthor{n3o}{216}
\pmtype{Proof}
\pmcomment{trigger rebuild}
\pmclassification{msc}{26A06}

% this is the default PlanetMath preamble.  as your knowledge
% of TeX increases, you will probably want to edit this, but
% it should be fine as is for beginners.

% almost certainly you want these
\usepackage{amssymb}
\usepackage{amsmath}
\usepackage{amsfonts}

% used for TeXing text within eps files
%\usepackage{psfrag}
% need this for including graphics (\includegraphics)
%\usepackage{graphicx}
% for neatly defining theorems and propositions
%\usepackage{amsthm}
% making logically defined graphics
%%%\usepackage{xypic}

% there are many more packages, add them here as you need them

% define commands here
\begin{document}
Let's say that $g$ is differentiable in $x_0$ and $f$ is differentiable in $y_0 = g(x_0)$. We define:
\[ \varphi(y) = \left \{ \begin{array}{ll}
\frac{f(y) - f(y_0)}{y-y_0} & \textrm{if $y \neq y_0$} \\
f'(y_0) & \textrm{if $y = y_0$}
\end{array} \right.
\]

Since $f$ is differentiable in $y_0$, $\varphi$ is continuous.
We observe that, for $x \neq x_0$,
\[ \frac{f(g(x))-f(g(x_0))}{x-x_0} = \varphi(g(x)) \frac{g(x) - g(x_0)}{x-x_0}, \]
in fact, if $g(x) \neq g(x_0)$, it follows at once from the definition of $\varphi$, while if $g(x) = g(x_0)$, both members of the equation are 0.

Since $g$ is continuous in $x_0$, and $\varphi$ is continuous in $y_0$,
\[ \lim_{x \to x_0} \varphi(g(x)) = \varphi(g(x_0)) = f'(g(x_0)), \]
hence
\begin{eqnarray*}
(f \circ g)'(x_0) &=& \lim_{x\to x_0} \frac{f(g(x))-f(g(x_0))}{x-x_0}  \\
  &=& \lim_{x\to x_0} \varphi(g(x)) \frac{g(x) - g(x_0)}{x-x_0}  \\
  &=& f'(g(x_0))g'(x_0). 
\end{eqnarray*}
%%%%%
%%%%%
\end{document}
