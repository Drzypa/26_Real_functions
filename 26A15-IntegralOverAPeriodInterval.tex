\documentclass[12pt]{article}
\usepackage{pmmeta}
\pmcanonicalname{IntegralOverAPeriodInterval}
\pmcreated{2013-03-22 18:43:57}
\pmmodified{2013-03-22 18:43:57}
\pmowner{pahio}{2872}
\pmmodifier{pahio}{2872}
\pmtitle{integral over a period interval}
\pmrecord{7}{41501}
\pmprivacy{1}
\pmauthor{pahio}{2872}
\pmtype{Theorem}
\pmcomment{trigger rebuild}
\pmclassification{msc}{26A15}
\pmclassification{msc}{26A42}
\pmsynonym{integral over a period}{IntegralOverAPeriodInterval}
\pmsynonym{integral of periodic function}{IntegralOverAPeriodInterval}
\pmrelated{DefiniteIntegral}
\pmrelated{IntegralsOfEvenAndOddFunctions}

% this is the default PlanetMath preamble.  as your knowledge
% of TeX increases, you will probably want to edit this, but
% it should be fine as is for beginners.

% almost certainly you want these
\usepackage{amssymb}
\usepackage{amsmath}
\usepackage{amsfonts}

% used for TeXing text within eps files
%\usepackage{psfrag}
% need this for including graphics (\includegraphics)
%\usepackage{graphicx}
% for neatly defining theorems and propositions
 \usepackage{amsthm}
% making logically defined graphics
%%%\usepackage{xypic}

% there are many more packages, add them here as you need them

% define commands here

\theoremstyle{definition}
\newtheorem*{thmplain}{Theorem}

\begin{document}
\PMlinkescapeword{period}
\textbf{Theorem.}\, If the real function $f$ is periodic and \PMlinkname{integrable}{RiemannIntegrable} over a \PMlinkname{period}{Periodic} interval, the value of integral over a period interval is always the same, i.e.
\begin{align}
\int_a^{a+p}\!f(x)\,dx \;=\; \int_0^pf(x)\,dx \quad \forall\,a \in \mathbb{R}
\end{align}
where $p$ is the period of $f$.\\

{\em Proof.}\, The right hand side of the equation (1) is manipulated, with one \PMlinkname{substitution}{ChangeOfVariableInDefiniteIntegral} \,$x = t\!+\!p$:
\begin{align*}
\int_0^pf(x)\,dx & \;=\; \int_0^af(x)\,dx+\int_a^pf(x)\,dx\\
& \;=\; \int_0^af(x)\,dx+\int_a^{a+p}\!f(x)\,dx-\int_p^{a+p}\!f(x)\,dx\\ 
& \;=\; \int_0^af(x)\,dx+\int_a^{a+p}\!f(x)\,dx-\int_0^{a}f(t\!+\!p)\,dt\\
& \;=\; \int_0^af(x)\,dx+\int_a^{a+p}\!f(x)\,dx-\int_0^{a}f(t)\,dt\\
& \;=\; \int_a^{a+p}\!f(x)\,dx
\end{align*}

\begin{thebibliography}{8}
\bibitem{J} {\sc Ernst Lindel\"of}: {\em Johdatus korkeampaan analyysiin}. Fourth edition. Werner S\"oderstr\"om Osakeyhti\"o, Porvoo ja Helsinki (1956).
\bibitem{L} {\em Fr{\aa}ga Lund om matematik}, \PMlinkexternal{here}{http://www.maths.lth.se/query/}.
\end{thebibliography}
%%%%%
%%%%%
\end{document}
