\documentclass[12pt]{article}
\usepackage{pmmeta}
\pmcanonicalname{HowToFindWhetherAGivenNumberIsPrimeOrNot}
\pmcreated{2014-08-12 17:53:37}
\pmmodified{2014-08-12 17:53:37}
\pmowner{burgess}{1001318}
\pmmodifier{burgess}{1001318}
\pmtitle{How to find whether a given number is prime or not?}
\pmrecord{1}{}
\pmprivacy{1}
\pmauthor{burgess}{1001318}
\pmtype{Topic}

% this is the default PlanetMath preamble.  as your knowledge
% of TeX increases, you will probably want to edit this, but
% it should be fine as is for beginners.

% almost certainly you want these
\usepackage{amssymb}
\usepackage{amsmath}
\usepackage{amsfonts}

% need this for including graphics (\includegraphics)
\usepackage{graphicx}
% for neatly defining theorems and propositions
\usepackage{amsthm}

% making logically defined graphics
%\usepackage{xypic}
% used for TeXing text within eps files
%\usepackage{psfrag}

% there are many more packages, add them here as you need them

% define commands here

\begin{document}

What is a prime number

A number is greater than 1 is called a prime number, if it has only two factors, namely 1 and the number itself.

Prime numbers up to 100 are:2, 3, 5, 7, 11, 13, 17, 19, 23, 29, 31, 37, 41, 43, 47, 53, 59, 61, 67, 71, 73, 79, 83, 89, 97

Procedure to find out the prime number

Suppose A is given number.

Step 1: Find a whole number nearly greater than the square root of A. K > square root(A)
Step 2: Test whether A is divisible by any prime number less than  K. If yes A is not a prime number. If not, A is prime  number.

Example:

Find out whether 337 is a prime number or not?

Step 1: 19 > square root (337) 
Prime numbers less than 19 are 2, 3, 5, 7, 11, 13, 17
Step 2: 337 is not divisible by any of them

Therefore 337 is a prime number

These are simple and easy tricks which are helpful to solve your math homework problems  .

\end{document}
