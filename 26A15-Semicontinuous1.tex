\documentclass[12pt]{article}
\usepackage{pmmeta}
\pmcanonicalname{Semicontinuous1}
\pmcreated{2013-03-22 14:00:16}
\pmmodified{2013-03-22 14:00:16}
\pmowner{bwebste}{988}
\pmmodifier{bwebste}{988}
\pmtitle{semicontinuous}
\pmrecord{13}{34844}
\pmprivacy{1}
\pmauthor{bwebste}{988}
\pmtype{Definition}
\pmcomment{trigger rebuild}
\pmclassification{msc}{26A15}
\pmdefines{lower semicontinuous}
\pmdefines{upper semicontinuous}
\pmdefines{lower semi-continuous}
\pmdefines{upper semi-continuous}

\endmetadata

% this is the default PlanetMath preamble.  as your knowledge
% of TeX increases, you will probably want to edit this, but
% it should be fine as is for beginners.

% almost certainly you want these
\usepackage{amssymb}
\usepackage{amsmath}
\usepackage{amsfonts}

% used for TeXing text within eps files
%\usepackage{psfrag}
% need this for including graphics (\includegraphics)
%\usepackage{graphicx}
% for neatly defining theorems and propositions
%\usepackage{amsthm}
% making logically defined graphics
%%%\usepackage{xypic}

% there are many more packages, add them here as you need them

% define commands here

\newcommand{\R}[0]{\mathbb{R}}
\newcommand{\C}[0]{\mathbb{C}}
\newcommand{\N}[0]{\mathbb{N}}
\newcommand{\Z}[0]{\mathbb{Z}}
\newcommand{\Q}[0]{\mathbb{Q}}

% The below lines should work as the command
% \renewcommand{\bibname}{References}
% without creating havoc when rendering an entry in 
% the page-image mode.
\makeatletter
\@ifundefined{bibname}{}{\renewcommand{\bibname}{References}}
\makeatother

\newcommand*{\norm}[1]{\lVert #1 \rVert}
\newcommand*{\abs}[1]{| #1 |}
\begin{document}
Suppose $X$ is a topological space, and $f$ is a function
from $X$ into the extended real numbers $\mathbb{R}^*$; $f:X\to \mathbb{R}^*$.
Then:
\begin{enumerate}
\item
If
$f^{-1}((\alpha,\infty])=\{x\in X \mid f(x) >\alpha\}$ 
is an open set in $X$ for all $\alpha\in \mathbb{R}$,
then $f$ is said to be {\bf lower semicontinuous}.
\item
If
$f^{-1}([-\infty,\alpha))=\{x\in X \mid f(x) <\alpha\}$ 
is an open set in $X$ for all $\alpha\in \mathbb{R}$,
then $f$ is said to be {\bf upper semicontinuous}.
\end{enumerate}

In other words, $f$ is lower semicontinuous, if $f$ is continuous with 
respect to the topology for $\mathbb{R}^*$ containing $\emptyset$ and 
open sets
$$
   U(\alpha) = (\alpha,\infty], \quad \quad \alpha\in \mathbb{R}\cup \{-\infty\}.
$$
It is not difficult to see that this is a topology. For example, 
for a union of sets $U(\alpha_i)$ we have $\cup_i U(\alpha_i)=U(\inf \alpha_i)$. 
Obviously, this topology is much coarser than
the usual topology for the extended numbers. 
However,
the sets $U(\alpha)$ can be seen as neighborhoods of infinity, so
in some sense, semicontinuous functions are "continuous at infinity"
(see example 3 below). 

\subsubsection{Examples}
\begin{enumerate}
\item A function $f\colon X\to \mathbb{R}^*$ is continuous if and only if 
it is lower and upper semicontinuous. 
\item Let $f$ be the characteristic function of a set $\Omega\subseteq X$. 
Then $f$ is lower (upper)
  semicontinuous if and only if $\Omega$ is open (closed).
This also holds for the function that
  equals $\infty$ in the set and $0\,$ outside. 
  
It follows that the characteristic function of $\Q$ is not
semicontinuous.

\item On $\mathbb{R}$, the function $f(x)=1/x$ for $x\neq 0$ and $f(0)=0$, is not
semicontinuous. This example illustrate how semicontinuous "at infinity".
\end{enumerate}
  
\subsubsection{Properties}

Let $f\colon X\to \mathbb{R}^*$ be a function.
\begin{enumerate}
\item Restricting $f$ to a subspace preserves semicontinuity.
\item Suppose $f$ is upper (lower) semicontinuous, $A$ is a topological space, and 
$\Psi\colon A\to X$ is a homeomorphism. Then $f\circ\Psi$ is upper (lower) semicontinuous. 
\item Suppose $f$ is upper (lower) semicontinuous, and 
$S\colon \mathbb{R}^*\to \mathbb{R}^*$ is a sense preserving homeomorphism. 
Then $S\circ f$ is upper (lower) semicontinuous. 
\item $f$ is lower semicontinuous if and only if 
$-f$ is upper semicontinuous.
\end{enumerate}

\begin{thebibliography}{9}
 \bibitem{rudin_real}
 W. Rudin, \emph{Real and complex analysis}, 3rd ed., McGraw-Hill Inc., 1987.
\bibitem{cohn}
 D.L. Cohn, \emph{Measure Theory}, Birkh\"auser, 1980.
 \end{thebibliography}
%%%%%
%%%%%
\end{document}
