\documentclass[12pt]{article}
\usepackage{pmmeta}
\pmcanonicalname{RulerFunction}
\pmcreated{2013-03-22 18:23:55}
\pmmodified{2013-03-22 18:23:55}
\pmowner{yesitis}{13730}
\pmmodifier{yesitis}{13730}
\pmtitle{ruler function}
\pmrecord{5}{41044}
\pmprivacy{1}
\pmauthor{yesitis}{13730}
\pmtype{Definition}
\pmcomment{trigger rebuild}
\pmclassification{msc}{26A99}

% this is the default PlanetMath preamble.  as your knowledge
% of TeX increases, you will probably want to edit this, but
% it should be fine as is for beginners.

% almost certainly you want these
\usepackage{amssymb}
\usepackage{amsmath}
\usepackage{amsfonts}

% used for TeXing text within eps files
%\usepackage{psfrag}
% need this for including graphics (\includegraphics)
%\usepackage{graphicx}
% for neatly defining theorems and propositions
%\usepackage{amsthm}
% making logically defined graphics
%%%\usepackage{xypic}

% there are many more packages, add them here as you need them

% define commands here

\begin{document}
The ruler function $f$ on the real line is defined as follows:

\begin{equation}
    f(x)=\left\{
           \begin{array}{ll}
             0, & \hbox{$x$ is irrational;} \\
             1/n, & \hbox{$x=m/n$, $m$ and $n$ are relatively primes}.
           \end{array}
         \right.
\end{equation}

Given a rational number $\frac{m}{n}$ in lowest terms, $n$ positive, the ruler function outputs the size (length) of a piece resulting from equally subdividing the unit interval into $n$, the number in the denominator, parts. It ``ignores'' inputs of irrational functions, sending them to 0.

The ruler function is so termed because it resembles a ruler. The following picture might be helpful: if $\frac{m}{n}$ in lowest terms  is a reasonably small rational number (which we assume positive). Then it can be ``read off'' on a ruler whose intervals of one unit size are each equally subdivided into $n$ parts measuring $\frac{1}{n}$ units each by 
\begin{enumerate}
\item{running one's finger through until the integer preceding it and then} 
\item{running through to the subsequent $r$th subunit, ``left-over'' from the division of $m$ by $n$.} 
\end{enumerate}
On the other hand, an irrational number can not be read off from any ruler no matter how fine we subdivide a unit interval in any ruler.

\begin{thebibliography}{1}
\bibitem{Du2003}
Dunham, W., \emph{Nondifferentiability of the Ruler Function}, Mathematics Magazine, Mathematical Association of America, 2003.

\bibitem{Sc1997}
Heuer, G.A., \emph{Functions Continuous at the Irrationals and Discontinuous at the Rationals}, The American Mathematical Monthly, Mathematical Association of America, 1965.
\end{thebibliography}
%%%%%
%%%%%
\end{document}
