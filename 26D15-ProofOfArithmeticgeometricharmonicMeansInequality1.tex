\documentclass[12pt]{article}
\usepackage{pmmeta}
\pmcanonicalname{ProofOfArithmeticgeometricharmonicMeansInequality}
\pmcreated{2013-03-22 12:41:25}
\pmmodified{2013-03-22 12:41:25}
\pmowner{drini}{3}
\pmmodifier{drini}{3}
\pmtitle{proof of arithmetic-geometric-harmonic means inequality}
\pmrecord{6}{32970}
\pmprivacy{1}
\pmauthor{drini}{3}
\pmtype{Proof}
\pmcomment{trigger rebuild}
\pmclassification{msc}{26D15}
\pmrelated{ArithmeticMean}
\pmrelated{GeometricMean}
\pmrelated{HarmonicMean}
\pmrelated{GeneralMeansInequality}
\pmrelated{WeightedPowerMean}
\pmrelated{PowerMean}

\endmetadata

\usepackage{graphicx}
%%%\usepackage{xypic} 
\usepackage{bbm}
\newcommand{\Z}{\mathbbmss{Z}}
\newcommand{\C}{\mathbbmss{C}}
\newcommand{\R}{\mathbbmss{R}}
\newcommand{\Q}{\mathbbmss{Q}}
\newcommand{\mathbb}[1]{\mathbbmss{#1}}
\newcommand{\figura}[1]{\begin{center}\includegraphics{#1}\end{center}}
\newcommand{\figuraex}[2]{\begin{center}\includegraphics[#2]{#1}\end{center}}
\begin{document}
Let $M$ be $\max\{x_1,x_2,x_3,\ldots,x_n\}$ and let $m$ be $\min\{x_1,x_2,x_3,\ldots,x_n\}$.

Then
$$M=\frac{M+M+M+\cdots+M}{n}\geq\frac{x_1+x_2+x_3+\cdots+x_n}{n}$$
$$m=\frac{n}{\frac{n}{m}}=
\frac{n}{\frac{1}{m}+\frac{1}{m}+\frac{1}{m}+\cdots+\frac{1}{m}}
\leq\frac{n}{\frac{1}{x_1}+\frac{1}{x_2}+\frac{1}{x_3}+\cdots+\frac{1}{x_n}}$$
where all the summations have $n$ terms.
So we have proved in this way the two inequalities at the extremes.

Now we shall prove the inequality between arithmetic mean and geometric mean.

\section{Case $n=2$}
We do first the case $n=2$.

\begin{eqnarray*}
(\sqrt{x_1}-\sqrt{x_2})^2 &\geq& 0\\
x_1-2\sqrt{x_1x_2}+x_2&\geq&0\\
x_1+x_2&\geq&2\sqrt{x_1x_2}\\
\frac{x_1+x_2}{2}&\geq&\sqrt{x_1x_2}
\end{eqnarray*}

\section{Case $n=2^k$}
Now we prove the inequality for any power of $2$ (that is, $n=2^k$ for some integer $k$) by using mathematical induction. 

\begin{eqnarray*}
\lefteqn{
    \frac{x_1+x_2+\cdots+x_{2^k}+x_{2^k+1}+\cdots+x_{2^{k+1}}}{2^{k+1}}
}\\
&=&
\frac{\left(
   \frac{x_1+x_2+\cdots+x_{2^k}}{2^k}
\right)
+
\left(
   \frac{x_{2^k+1}+x_{2^k+2}+\cdots+x_{2^{k+1}}}{2^k}
\right)}
{2}
\end{eqnarray*}
and using the case $n=2$ on the last expression we can state the following inequality
\begin{eqnarray*}
\lefteqn{
    \frac{x_1+x_2+\cdots+x_{2^k}+x_{2^k+1}+\cdots+x_{2^{k+1}}}{2^{k+1}}
}\\
&\ge&
\sqrt{
\left(
   \frac{x_1+x_2+\cdots+x_{2^k}}{2^k}
\right)
\left(
   \frac{x_{2^k+1}+x_{2^k+2}+\cdots+x_{2^{k+1}}}{2^k}
\right)
}\\
&\ge&
\sqrt{
\sqrt[2^k]{x_1x_2\cdots x_{2^k}}
\sqrt[2^k]{x_{2^k+1}x_{2^k+2}\cdots x_{2^{k+1}}}
}
\end{eqnarray*}
where the last inequality was obtained by applying the induction hypothesis with $n=2^k$. Finally, we see that the last expression is equal to
$\sqrt[2^{k+1}]{x_1x_2x_3\cdots x_{2^{k+1}}}$
and so we have proved the truth of the inequality when the number of terms is a power of two.

\section{Inequality for $n$ numbers implies inequality for $n-1$}
Finally, we prove that if the inequality holds for any $n$, it must also hold for $n-1$, and this proposition, combined with the preceding proof for powers of $2$, is enough to prove the inequality for any positive integer.

Suppose that 
$$\frac{x_1+x_2+\cdots+x_n}{n}\geq\sqrt[n]{x_1x_2\cdots x_n}$$
is known for a given value of $n$ (we just proved that it is true for powers of two, as example).
Then we can replace $x_n$ with the average of the first $n-1$ numbers. So
\begin{eqnarray*}
\lefteqn{
\frac{
x_1+x_2+\cdots+x_{n-1}+
\left(\frac{x_1+x_2+\cdots+x_{n-1}}{n-1} \right)
}{n}
}\\
&=&
\frac{(n-1)x_1+(n-1)x_2+\cdots+(n-1)x_{n-1}+x_1+x_2+\cdots+x_{n-1}}{n(n-1)}\\
&=&
\frac{n x_1+ n x_2 + \cdots + n x_{n-1}}{n(n-1)}\\
&=&
\frac{x_1+x_2 + \cdots + x_{n-1}}{(n-1)}
\end{eqnarray*}

On the other hand
\begin{eqnarray*}
\lefteqn{\sqrt[n]{
x_1x_2\cdots x_{n-1}
\left(\frac{x_1+x_2+\cdots+x_{n-1}}{n-1}\right)
}}\\
&=&
\sqrt[n]{x_1x_2\cdots x_{n-1}}
\sqrt[n]{\frac{x_1+x_2+\cdots+x_{n-1}}{n-1}}
\end{eqnarray*}
which, by hypothesis (the inequality holding for $n$ numbers) and the observations made above, leads to:
$$\left(\frac{x_{1}+x_{2}+\cdots+x_{n-1}}{n-1}\right)^{n}\ge 
(x_{1}x_{2}\cdots x_{n})\left(\frac{x_1+x_2+\cdots+x_{n-1}}{n-1}\right)
$$
and so
$$
\left(\frac{x_{1}+x_{2}+\cdots+x_{n-1}}{n-1}\right)^{n-1}\ge 
x_{1}x_{2}\cdots x_{n}
$$
from where we get that 
$$
\frac{x_{1}+x_{2}+\cdots+x_{n-1}}{n-1}\ge\sqrt[n-1]{x_{1}x_{2}\cdots x_{n}}.$$

So far we have proved the inequality between the arithmetic mean and the geometric mean. The geometric-harmonic inequality is easier. Let $t_i$ be $1/x_{i}$.

From 
$$\frac{t_{1}+t_{2}+\cdots+t_{n}}{n}\geq\sqrt[n]{t_{1}t_{2}t_{3}\cdots t_{n}}$$
we obtain
$$\frac{\frac{1}{x_{1}}+\frac{1}{x_{2}}+\frac{1}{x_{3}}+\cdots+\frac{1}{x_{n}}}{n}\geq
\sqrt[n]{\frac{1}{x_{1}}\frac{1}{x_{2}}\frac{1}{x_{3}}\cdots\frac{1}{x_{n}}}$$
and therefore
$$\sqrt[n]{x_{1}x_{2}x_{3}\cdots x_{n}}\geq
\frac{n}{
\frac{1}{x_{1}}+\frac{1}{x_{2}}+\frac{1}{x_{3}}+\cdots+\frac{1}{x_{n}}
}$$
and so, our proof is completed.
%%%%%
%%%%%
\end{document}
