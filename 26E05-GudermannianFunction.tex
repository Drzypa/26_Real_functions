\documentclass[12pt]{article}
\usepackage{pmmeta}
\pmcanonicalname{GudermannianFunction}
\pmcreated{2014-10-27 7:59:12}
\pmmodified{2014-10-27 7:59:12}
\pmowner{pahio}{2872}
\pmmodifier{pahio}{2872}
\pmtitle{Gudermannian function}
\pmrecord{27}{41997}
\pmprivacy{1}
\pmauthor{pahio}{2872}
\pmtype{Definition}
\pmcomment{trigger rebuild}
\pmclassification{msc}{26E05}
\pmclassification{msc}{26A48}
\pmclassification{msc}{33B10}
\pmclassification{msc}{26A09}
\pmsynonym{Gudermannian}{GudermannianFunction}
\pmrelated{ChainRule}
\pmrelated{EulerNumbers2}

\endmetadata

% this is the default PlanetMath preamble.  as your knowledge
% of TeX increases, you will probably want to edit this, but
% it should be fine as is for beginners.

% almost certainly you want these
\usepackage{amssymb}
\usepackage{amsmath}
\usepackage{amsfonts}

% used for TeXing text within eps files
%\usepackage{psfrag}
% need this for including graphics (\includegraphics)
%\usepackage{graphicx}
% for neatly defining theorems and propositions
 \usepackage{amsthm}
% making logically defined graphics
%%%\usepackage{xypic}
\usepackage{pstricks}
\usepackage{pst-plot}

% there are many more packages, add them here as you need them

% define commands here

\theoremstyle{definition}
\newtheorem*{thmplain}{Theorem}

\begin{document}
\PMlinkescapeword{atan}

The \emph{Gudermannian function}\, \textbf{gd}\, is defined by the definite integral
\begin{align}
\mbox{gd}\,x \;:=\; \int_0^x\!\frac{dt}{\cosh t} \;=\; \int_0^x\!\frac{2\,dt}{e^t+e^{-t}} 
\;=\; 2\!\int_0^x\!\frac{e^t}{e^{2t}+1}\,dt.
\end{align}
Because of the continuity of the integrand of (1), this equation defines a differentiable real function, which by the \PMlinkid{fundamental theorem of calculus}{5660} has the derivative 
\begin{align}
\frac{d}{dx}\,\mbox{gd}\,x \;=\; \frac{1}{\cosh{x}}.
\end{align}
From this we infer that the Gudermannian is an odd function.


Using the \PMlinkname{substitution}{ChangeOfVariableInDefiniteIntegral} \,$e^t \:= u$\, we can make from (1) the closed form
\begin{align}
\mbox{gd}\,x \;=\; 2\arctan{e^x}-\frac{\pi}{2}.
\end{align}
This enables to see that 
$$\lim_{x\to-\infty}\mbox{gd}\,x \;=\; -\frac{\pi}{2}, \qquad \lim_{x\to+\infty}\mbox{gd}\,x \;=\; +\frac{\pi}{2},$$
whence the \PMlinkescapetext{graph} of the gudermannian has the lines \,$y = \pm\frac{\pi}{2}$\, as asymptotes.

Besides (3), one has also e.g. the closed forms
$$\mbox{gd}\,x \;=\; \arcsin(\tanh{x}), \qquad \mbox{gd}\,x \;=\; \arctan(\sinh{x}),$$
since both of these composition functions have the derivative $\displaystyle\frac{1}{\cosh{x}}$, as (2), and vanish in the origin (see the fundamental theorem of integral calculus).\, Accordingly, we may write the formulae
\begin{align}
\sin(\mbox{gd}\,x) \;=\; \tanh{x}, \qquad \tan(\mbox{gd}\,x) \;=\; \sinh{x}
\end{align}
which are illustrated by the below right triangle.\, Cf. the properties of the catenary!
\begin{center}
\begin{pspicture}(-3,-0.5)(3,2.8)
\pspolygon(-2.5,0)(2,0)(2,2.5)
\rput(2.7,1.2){$\sinh{x}$}
\rput(-0.7,1.5){$\cosh{x}$}
\rput(0,-0.25){$1$}
\rput(-1.5,0.25){$\mbox{gd}\,x$}
\psarc(-2.5,0){0.5}{0}{29}
\psline(1.75,0)(1.75,0.25)(2,0.25)
\rput(-3,0){.}
\rput(3,2.8){.}
\end{pspicture}
\end{center}


Thus, Gudermannian function offers a beautiful \PMlinkescapetext{connection} between the \PMlinkid{trigonometric}{4676} and the hyperbolic functions without using imaginary numbers (cf. the paragraph 3 in hyperbolic identities).\\


The Taylor series expansion of $\mbox{gd}\,x$ may be expressed with the Euler numbers $E_n$; these have as generating function the derivative of Gudermannian function:
$$\frac{1}{\cosh{x}} \;:=\; \sum_{n=0}^\infty\frac{E_n}{n!}\,x^n\; \qquad (|x| < \frac{\pi}{2})$$
\begin{align}
\therefore \;\, \mbox{gd}\,x \;=\; \sum_{n=0}^\infty\frac{E_n}{(n\!+\!1)!}\,x^{n+1} \qquad (|x| < \frac{\pi}{2})
\end{align}
Since gd is \PMlinkescapetext{odd} (and its derivative \PMlinkname{even}{EvenFunction}), the numbers $E_n$ with odd indices are 0; the others are non-zero integers.\, The begin of (5) is
$$\mbox{gd}\,x \;=\; x-\frac{1}{3!}x^3+\frac{5}{5!}x^5-\frac{61}{7!}x^7+\frac{1385}{9!}x^9-+\ldots$$


\begin{center}
\begin{pspicture}(-6,-3.2)(6,2.5)
\psline[linestyle=dashed](-5.5,+1.5707)(5.5,+1.5707)
\psline[linestyle=dashed](-5.5,-1.5707)(5.5,-1.5707)
\pscurve[linecolor=blue](0,0)(0.1,0.09983)(0.2,0.1987)(0.3,0.2956)(0.4,0.3897)(0.5,0.4804)(0.6,0.5669)(0.7,0.6490)(0.8,0.7262)(0.9,0.7985)(1.0,0.8658)(1.1,0.9281)(1.2,0.9857)(1.3,1.0387)(1.4,1.0872)(1.5,1.1317)(1.6,1.1724)(1.7,1.2094)(1.8,1.2432)(1.9,1.2739)(2.0,1.3018)(2.1,1.3271)(2.2,1.3501)(2.3,1.3710)(2.4,1.3899)(2.5,1.4070)(2.6,1.4225)(2.7,1.4366)(2.8,1.4493)(2.9,1.4609)(3.0,1.4713)(3.1,1.4808)(3.2,1.4893)(3.3,1.4971)(3.4,1.5041)(3.5,1.5104)(3.6,1.5162)(3.7,1.5214)(3.8,1.5261)(3.9,1.5303)(4.0,1.5342)(4.1,1.5377)(4.2,1.5408)(4.3,1.5437)(4.4,1.5462)(4.5,1.5486)(4.6,1.5507)(4.7,1.5526)(4.8,1.5543)(4.9,1.5559)(5.0,1.5573)(5.1,1.5586)(5.2,1.5598)(5.3,1.5608)(5.4,1.5618)(5.5,1.5626)
\pscurve[linecolor=blue](0,0)(-0.1,-0.09983)(-0.2,-0.1987)(-0.3,-0.2956)(-0.4,-0.3897)(-0.5,-0.4804)(-0.6,-0.5669)(-0.7,-0.6490)(-0.8,-0.7262)(-0.9,-0.7985)(-1.0,-0.8658)(-1.1,-0.9281)(-1.2,-0.9857)(-1.3,-1.0387)(-1.4,-1.0872)(-1.5,-1.1317)(-1.6,-1.1724)(-1.7,-1.2094)(-1.8,-1.2432)(-1.9,-1.2739)(-2.0,-1.3018)(-2.1,-1.3271)(-2.2,-1.3501)(-2.3,-1.3710)(-2.4,-1.3899)(-2.5,-1.4070)(-2.6,-1.4225)(-2.7,-1.4366)(-2.8,-1.4493)(-2.9,-1.4609)(-3.0,-1.4713)(-3.1,-1.4808)(-3.2,-1.4893)(-3.3,-1.4971)(-3.4,-1.5041)(-3.5,-1.5104)(-3.6,-1.5162)(-3.7,-1.5214)(-3.8,-1.5261)(-3.9,-1.5303)(-4.0,-1.5342)(-4.1,-1.5377)(-4.2,-1.5408)(-4.3,-1.5437)(-4.4,-1.5462)(-4.5,-1.5486)(-4.6,-1.5507)(-4.7,-1.5526)(-4.8,-1.5543)(-4.9,-1.5559)(-5.0,-1.5573)(-5.1,-1.5586)(-5.2,-1.5598)(-5.3,-1.5608)(-5.4,-1.5618)(-5.5,-1.5626)
\rput(0,-2.8){$\textrm{The curve  }y = \mbox{gd}\,x$}
\psaxes{->}(0,0)(-5.5,-1.8)(5.5,2.4)
\rput[a](5.6,-0.25){$x$}
\rput[r](0.3,2.4){$y$}
\rput[l](-6.5,0){.}
\end{pspicture}
\end{center}

%%%%%
%%%%%
\end{document}
