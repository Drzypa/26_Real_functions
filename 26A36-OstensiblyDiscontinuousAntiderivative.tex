\documentclass[12pt]{article}
\usepackage{pmmeta}
\pmcanonicalname{OstensiblyDiscontinuousAntiderivative}
\pmcreated{2013-03-22 18:37:08}
\pmmodified{2013-03-22 18:37:08}
\pmowner{pahio}{2872}
\pmmodifier{pahio}{2872}
\pmtitle{ostensibly discontinuous antiderivative}
\pmrecord{7}{41353}
\pmprivacy{1}
\pmauthor{pahio}{2872}
\pmtype{Example}
\pmcomment{trigger rebuild}
\pmclassification{msc}{26A36}
\pmrelated{CyclometricFunctions}

\endmetadata

% this is the default PlanetMath preamble.  as your knowledge
% of TeX increases, you will probably want to edit this, but
% it should be fine as is for beginners.

% almost certainly you want these
\usepackage{amssymb}
\usepackage{amsmath}
\usepackage{amsfonts}

% used for TeXing text within eps files
%\usepackage{psfrag}
% need this for including graphics (\includegraphics)
%\usepackage{graphicx}
% for neatly defining theorems and propositions
 \usepackage{amsthm}
% making logically defined graphics
%%%\usepackage{xypic}

% there are many more packages, add them here as you need them

% define commands here

\theoremstyle{definition}
\newtheorem*{thmplain}{Theorem}

\begin{document}
The real function
\begin{align}
x \;\mapsto\; \frac{1}{5-3\cos{x}}
\end{align}
is continuous for any $x$ (the denominator is always positive) and therefore it has an antiderivative, defined for all $x$.\, Using the universal trigonometric substitution
$$\cos{x} \;:=\; \frac{1\!-\!t^2}{1\!+\!t^2}, \qquad  dx \;=\; \frac{2dt}{1\!+\!t^2},
 \qquad t \,=\, \tan\frac{x}{2},$$
we obtain
$$5-3\cos{x} \;=\; \frac{5(1\!+\!t^2)-3(1\!-\!t^2)}{1\!+\!t^2} \;=\; \frac{2(1\!+\!4t^2)}{1\!+\!t^2},$$
whence
$$\int\!\frac{dx}{5-3\cos{x}} \;=\; \int\!\frac{dt}{1\!+\!4t^2} \,=\, \frac{1}{2}\arctan2t+C 
\,=\, \frac{1}{2}\arctan\!\left(2\tan\frac{x}{2}\right)+C.$$
This result is not defined in the odd multiples of $\pi$, and it seems that the function (1) does not have a continuous antiderivative.

However, one can check that the function
\begin{align}
x \;\mapsto\; \frac{x}{4}+\frac{1}{2}\arctan\frac{\sin{x}}{3-\cos{x}}+C
\end{align}
is everywhere continuous and has as its derivative the function (1); one has
$$\left|\frac{\sin{x}}{3-\cos{x}}\right| \;\leqq\; \frac{1}{3\!-\!1} \;=\; \frac{1}{2} \;<\; \frac{\pi}{2}.$$

\begin{thebibliography}{9}
\bibitem{J} {\sc Ernst Lindel\"of}: {\em Johdatus korkeampaan analyysiin}. Fourth edition. Werner S\"oderstr\"om Osakeyhti\"o, Porvoo ja Helsinki (1956).
\end{thebibliography}

%%%%%
%%%%%
\end{document}
