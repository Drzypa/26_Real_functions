\documentclass[12pt]{article}
\usepackage{pmmeta}
\pmcanonicalname{ContinuityOfConvexFunctionsAlternateProof}
\pmcreated{2013-03-22 18:25:28}
\pmmodified{2013-03-22 18:25:28}
\pmowner{yesitis}{13730}
\pmmodifier{yesitis}{13730}
\pmtitle{continuity of convex functions, alternate proof}
\pmrecord{4}{41078}
\pmprivacy{1}
\pmauthor{yesitis}{13730}
\pmtype{Proof}
\pmcomment{trigger rebuild}
\pmclassification{msc}{26B25}
\pmclassification{msc}{26A51}

% this is the default PlanetMath preamble.  as your knowledge
% of TeX increases, you will probably want to edit this, but
% it should be fine as is for beginners.

% almost certainly you want these
\usepackage{amssymb}
\usepackage{amsmath}
\usepackage{amsfonts}

% used for TeXing text within eps files
%\usepackage{psfrag}
% need this for including graphics (\includegraphics)
%\usepackage{graphicx}
% for neatly defining theorems and propositions
%\usepackage{amsthm}
% making logically defined graphics
%%%\usepackage{xypic}

% there are many more packages, add them here as you need them

% define commands here

\begin{document}
Let $f$ be convex and $y\in(a, b)$ be arbitrary but fixed. Then
\begin{eqnarray}
    f(\lambda x+(1-\lambda)y) &\leq& \lambda f(x)+(1-\lambda)f(y) \\
    f(\lambda x+(1-\lambda)y)-f(y) &\leq&
    \lambda(f(x)-f(y))\leq\lambda|f(x)-f(y)|.
\end{eqnarray}

Fix a number $c>\sup\{|f(u)-f(v)|: u,v\in(a, b)\}$. Then
\begin{equation}\label{5}
    |f(\lambda
    x+(1-\lambda)y)-f(y)|\leq\lambda|f(x)-f(y)|<\lambda c.
\end{equation}

Given $\epsilon>0$, let $\lambda$ range over $(0, \epsilon/c)$ if
$\epsilon/c<1$, or $\lambda=1$ otherwise. Then it is easy to see
that $f(\lambda x+(1-\lambda)y)$ and $f(y)$ lie within $\epsilon$
distance of each other when $\lambda$ varies as specified.

Continuity of $f$ now follows--for $x<y$, the left-hand limit equals
$f(y)$ and for $y<x$, the right-hand limit also equals $f(y)$, hence
the limit is $f(y)$.
%%%%%
%%%%%
\end{document}
