\documentclass[12pt]{article}
\usepackage{pmmeta}
\pmcanonicalname{RationalFunction}
\pmcreated{2013-03-22 13:38:54}
\pmmodified{2013-03-22 13:38:54}
\pmowner{CWoo}{3771}
\pmmodifier{CWoo}{3771}
\pmtitle{rational function}
\pmrecord{6}{34300}
\pmprivacy{1}
\pmauthor{CWoo}{3771}
\pmtype{Definition}
\pmcomment{trigger rebuild}
\pmclassification{msc}{26C15}
\pmsynonym{rational expression}{RationalFunction}
\pmrelated{PolynomialRing}
\pmrelated{FractionField}
\pmrelated{RealFunction}
\pmrelated{PropertiesOfEntireFunctions}
\pmrelated{IntegrationOfFractionPowerExpressions}

% this is the default PlanetMath preamble.  as your knowledge
% of TeX increases, you will probably want to edit this, but
% it should be fine as is for beginners.

% almost certainly you want these
\usepackage{amssymb}
\usepackage{amsmath}
\usepackage{amsfonts}

% used for TeXing text within eps files
%\usepackage{psfrag}
% need this for including graphics (\includegraphics)
%\usepackage{graphicx}
% for neatly defining theorems and propositions
%\usepackage{amsthm}
% making logically defined graphics
%%%\usepackage{xypic}

% there are many more packages, add them here as you need them

% define commands here
\def\sse{\subseteq}
\def\bigtimes{\mathop{\mbox{\Huge $\times$}}}
\def\impl{\Rightarrow}
\begin{document}
A real function $R(x)$ of a single variable $x$ is called
\emph{\PMlinkescapetext{rational}} if it can be written as a quotient
\[ R(x) = \frac{P(x)}{Q(x)}, \]
where $P(x)$ and $Q(x)$ are polynomials in $x$ with real coefficients. When one is only interested in algebraic properties of $R(x)$ or $P(x)$ and $Q(x)$, it is convenient to forget that they define functions and simply treat them as algebraic expressions in $x$. In this case $R(x)$ is referred to as a \emph{rational expression}.

In general, a rational function (expression) $R(x_1,\ldots,x_n)$ has the form
\[ R(x_1,\ldots,x_n) = \frac{P(x_1,\ldots,x_n)}{Q(x_1,\ldots,x_n)}, \]
where $P(x_1,\ldots,x_n)$ and $Q(x_1,\ldots,x_n)$ are polynomials in the
variables $(x_1,\ldots,x_n)$ with coefficients in some field or
ring $S$.

In this sense, $R(x_1,\ldots,x_n)$ can be regarded as an element of the fraction
field $S(x_1,\ldots,x_n)$ of the polynomial ring $S[x_1,\ldots,x_n]$.
%%%%%
%%%%%
\end{document}
