\documentclass[12pt]{article}
\usepackage{pmmeta}
\pmcanonicalname{ClassicalStokesTheorem}
\pmcreated{2013-03-22 15:27:52}
\pmmodified{2013-03-22 15:27:52}
\pmowner{stevecheng}{10074}
\pmmodifier{stevecheng}{10074}
\pmtitle{classical Stokes' theorem}
\pmrecord{6}{37316}
\pmprivacy{1}
\pmauthor{stevecheng}{10074}
\pmtype{Theorem}
\pmcomment{trigger rebuild}
\pmclassification{msc}{26B20}
\pmrelated{GeneralStokesTheorem}
\pmrelated{GaussGreenTheorem}
\pmrelated{GreensTheorem}

\usepackage{amssymb}
\usepackage{amsmath}
\usepackage{amsfonts}
\usepackage{amsthm}
\usepackage{enumerate}

% used for TeXing text within eps files
%\usepackage{psfrag}
% need this for including graphics (\includegraphics)
%\usepackage{graphicx}
% making logically defined graphics
%%%\usepackage{xypic}

% define commands here
\newcommand{\complex}{\mathbb{C}}
\newcommand{\real}{\mathbb{R}}
\newcommand{\rat}{\mathbb{Q}}
\newcommand{\nat}{\mathbb{N}}

\providecommand{\abs}[1]{\lvert#1\rvert}
\providecommand{\absW}[1]{\left\lvert#1\right\rvert}
\providecommand{\absB}[1]{\Bigl\lvert#1\Bigr\rvert}
\providecommand{\norm}[1]{\lVert#1\rVert}
\providecommand{\normW}[1]{\left\lVert#1\right\rVert}
\providecommand{\normB}[1]{\Bigl\lVert#1\Bigr\rVert}
\providecommand{\defnterm}[1]{\emph{#1}}

\DeclareMathOperator{\D}{D}
\DeclareMathOperator{\linspan}{span}

\newcommand{\vF}{\mathbf{F}}
\newcommand{\vA}{\mathbf{A}}
\newcommand{\vs}{\mathbf{s}}
\newcommand{\vn}{\mathbf{n}}
\newcommand{\vt}{\mathbf{t}}
\newcommand{\vu}{\mathbf{u}}
\newcommand{\vv}{\mathbf{v}}

\DeclareMathOperator{\curl}{curl}
\begin{document}
Let $M$ be a compact, oriented two-dimensional differentiable manifold (surface) with boundary in $\real^3$,
and $\vF$ be a $C^2$-smooth vector field defined on an open set in $\real^3$ containing $M$.
Then
\[
\iint_M (\nabla \times \vF) \cdot d\vA = \int_{\partial M} \vF \cdot d\vs\,.
\]
Here, the boundary of $M$, $\partial M$ (which is a curve)
is given the induced orientation from $M$.  The symbol $\nabla \times \vF$
denotes the curl of $\vF$.
The symbol $d \vs$ denotes the line element $ds$ with a direction
parallel to the unit tangent vector $\vt$ to $\partial M$, while $d \vA$ denotes
the area element $dA$ of the surface $M$ with a direction parallel to the unit outward normal $\vn$
to $M$.  In precise terms:
\[
d\vA = \vn \, dA\,, \quad d\vs = \vt \, ds\,.
\]

The classical Stokes' theorem reduces to Green's theorem on the plane if the surface $M$
is taken to lie in the xy-plane.

The classical Stokes' theorem, and 
the other ``Stokes' type'' theorems
are special cases of the general Stokes' theorem involving
differential forms.
In fact, in the proof we present below, we appeal to the general Stokes' theorem.

\section*{Physical interpretation}
(To be written.)

\section*{Proof using differential forms}
The proof becomes a triviality once we express
$(\nabla \times \vF) \cdot d\vA$
and $\vF \cdot d\vs$ in terms of differential forms.

\begin{proof}
Define the differential forms $\eta$ and $\omega$ by
\begin{align*}
\eta_p(\vu, \vv) &= \langle \curl \vF(p), \vu \times \vv \rangle\,, \\
\omega_p(\vv) &= \langle \vF(p), \vv \rangle\,.
\end{align*}
for points $p \in \real^3$, and tangent vectors $\vu, \vv \in \real^3$.
The symbol $\langle, \rangle$ denotes the dot product in $\real^3$.
Clearly, the functions $\eta_p$ and $\omega_p$ are linear and alternating in 
$\vu$ and $\vv$.

We claim 
\begin{align}
\eta &= \nabla \times \vF \cdot d\vA & \text{ on $M$. } \\
\omega &= \vF \cdot d\vs & \text{ on $\partial M$.}
\end{align}

To prove (1), it suffices to check
it holds true when we evaluate the left- and right-hand sides
on an orthonormal basis $\vu, \vv$ for the tangent space of $M$
corresponding to the orientation of $M$,
given by the unit outward normal $\vn$.
We calculate
\begin{align*}
\nabla \times \vF \cdot d\vA(\vu,\vv) &= 
\langle \curl \vF, \vn \rangle \,
dA(\vu, \vv) 
& \text{definition of $d\vA = \vn \, dA$}
\\
&= 
\langle \curl \vF, \vn \rangle 
& \text{definition of volume form $dA$}
\\
&= 
\langle \curl \vF, \vu \times \vv \rangle 
& \text{since $\vu \times \vv = \vn$} \\
&=
\eta(\vu, \vv)\,.
\end{align*}

For equation (2), similarly, we only have to check that it holds
when both sides are evaluated at $\vv = \vt$,
the unit tangent vector of $\partial M$
with the induced orientation of $\partial M$.
We calculate again,
\begin{align*}
\vF \cdot d\vs(\vt) &= \langle \vF, \vt \rangle \, ds(\vt) 
& \text{definition of $d\vs = \vt \, ds$} \\
&= \langle \vF, \vt \rangle 
& \text{definition of volume form $ds$} \\
&= \omega(\vt)\,.
\end{align*}

Furthermore, $d \omega$ = $\eta$.
(This can be checked by a calculation
in Cartesian coordinates, but in fact this equation
is one of the coordinate-free \emph{definitions} of the curl.)

The classical Stokes' Theorem now follows
from the general Stokes' Theorem,
\[
\int_M \eta = \int_M d\omega = \int_{\partial M} \omega\,. \qedhere
\]
\end{proof}

\begin{thebibliography}{3}
\bibitem{Spivak} Michael Spivak. {\it Calculus on Manifolds}. Perseus Books, 1998.
\end{thebibliography}
%%%%%
%%%%%
\end{document}
