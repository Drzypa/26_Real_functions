\documentclass[12pt]{article}
\usepackage{pmmeta}
\pmcanonicalname{SincFunction}
\pmcreated{2013-03-22 14:17:22}
\pmmodified{2013-03-22 14:17:22}
\pmowner{mathcam}{2727}
\pmmodifier{mathcam}{2727}
\pmtitle{sinc function}
\pmrecord{20}{35744}
\pmprivacy{1}
\pmauthor{mathcam}{2727}
\pmtype{Definition}
\pmcomment{trigger rebuild}
\pmclassification{msc}{26A06}
\pmsynonym{sine cardinal}{SincFunction}
\pmsynonym{cardinal sine}{SincFunction}
\pmrelated{SineIntegral}
\pmrelated{SineIntegralInInfinity}
\pmrelated{LimitOfDisplaystyleFracsinXxAsXApproaches0}
\pmrelated{Asymptote}
\pmrelated{LaplaceTransformOfIntegralSine}

% this is the default PlanetMath preamble.  as your knowledge
% of TeX increases, you will probably want to edit this, but
% it should be fine as is for beginners.

% almost certainly you want these
\usepackage{amssymb}
\usepackage{amsmath}
\usepackage{amsfonts}

% used for TeXing text within eps files
%\usepackage{psfrag}
% need this for including graphics (\includegraphics)
%\usepackage{graphicx}
% for neatly defining theorems and propositions
\usepackage{amsthm}
% making logically defined graphics
%%%\usepackage{xypic}
\usepackage{pstricks}
\usepackage{pst-plot}
% there are many more packages, add them here as you need them

% define commands here

\newcommand{\sR}[0]{\mathbb{R}}
\newcommand{\sC}[0]{\mathbb{C}}
\newcommand{\sN}[0]{\mathbb{N}}
\newcommand{\sZ}[0]{\mathbb{Z}}

 \usepackage{bbm}
 \newcommand{\Z}{\mathbbmss{Z}}
 \newcommand{\C}{\mathbbmss{C}}
 \newcommand{\R}{\mathbbmss{R}}
 \newcommand{\Q}{\mathbbmss{Q}}



\newcommand*{\norm}[1]{\lVert #1 \rVert}
\newcommand*{\abs}[1]{| #1 |}
\begin{document}
{\bf Definition}
The \emph{$\operatorname{sinc}$-\PMlinkescapetext{function}} is the function 
$\operatorname{sinc}:\sR\to \sR$ defined as
 \begin{eqnarray*}
 \operatorname{sinc}(x) &=& \left\{ \begin {array}{ll} \frac{\sin x}{x} & \mbox{when}\,\, x\neq 0, \\
 1 & \mbox{when}\,\, x= 0.
 \end{array} \right.
 \end{eqnarray*}

In some situations, it is more convenient to work with an alternative "normalized variant," in which for $x\neq 0$ we redefine the function as
\begin{align*}
\operatorname{sinc}(x)=\frac{\sin(\pi x)}{\pi x}.
\end{align*}

The remainder of this entry deals with the initial definition, though most properties can clearly be suitably modified for the normalized version.

\begin{center}
\begin{pspicture}(-9,-2)(9,2)
\psaxes[Dx=11,Dy=1]{->}(0,0)(-8.5,-1.5)(8.5,2)
\rput(8.5,-0.2){$x$}
\rput(0.2,2){$y$}
\rput(1,-0.25){$\pi$}
\rput(-1.1,-0.25){$-\pi$}
\psplot[linecolor=blue]{-8}{-0.01}{x 180 mul sin x div 3 div}
\psplot[linecolor=blue]{0.01}{8}{x 180 mul sin x div 3 div}
\psdot[linecolor=red](0,1)
\end{pspicture}
\end{center}




\subsubsection*{Properties}
\begin{itemize}
\item Using a Taylor expansion of $\sin$, one can show that $\operatorname{sinc}$
is infinitely many times differentiable.
In particular, $\operatorname{sinc}$ is continuous. In this sense, the value $1$
for $x=0$ is motivated. 
\item Jordan's inequality implies that $|\operatorname{sinc}(x)|\le 1$ for all $x\in \sR$.  More generally, one can also show that all derivatives of $\operatorname{sinc}$ are bounded by 1.  See \PMlinkname{this entry}{AllDerivativesOfSincAreBoundedBy1}.
\item $\operatorname{sinc}$ is an even function.
\item The zeros of $\operatorname{sinc}$ are $x=\pm \pi, \pm 2\pi,\ldots$.
\item $\operatorname{sinc}$ is in $L^2$, but not in $L^1$, and \cite{gearhart}
\begin{eqnarray*}
\int_{-\infty}^\infty \operatorname{sinc}(x)\,dx &=& \pi,\\
\int_{-\infty}^\infty \operatorname{sinc}^2(x)\,dx &=& \pi,
\end{eqnarray*}
where the first of these denotes an improper Riemann integral.
\item For all $x\in \sR$, we have \cite{gearhart}
\begin{eqnarray*}
\operatorname{sinc} (x) &=& \sum_{k=0}^\infty (-1)^k \frac{x^{2k}}{(2k+1)!}, \\
\operatorname{sinc} (x) &=& \prod_{k=1}^\infty \left(1-\frac{x^2}{(k\pi)^2}\right), \\
\operatorname{sinc} (x) &=& \prod_{k=1}^\infty \cos \frac{x}{2^k}.
\end{eqnarray*}
\item The Fourier transform of $\operatorname{sinc}$ is the box function, i.e.
\begin{align*}
\operatorname{sinc}(x) = {1\over 2} \int_{-1}^1 e^{ixt} dt.
\end{align*}
\item The $\operatorname{sinc}$ function satisfies the differential equation
\begin{align*}
 x \operatorname{sinc}'' + 2 \operatorname{sinc}' + x \operatorname{sinc} = 0
\end{align*}
This is a consequence of a comment in the sine integral entry.
\item There is no known simple expression for the integral of
sinc. However, this function is known as the sine integral. 
\end{itemize}

\subsubsection*{Synonym and Etymology}
The $\operatorname{sinc}$ function is also 
called \PMlinkescapetext{\emph{sine cardinal}}
or  \PMlinkescapetext{\emph{cardinal sine}}.

\subsubsection*{Use}

The sinc function is relevant in several fields.  For one, its Fourier transform is a box, so it is the frequency respose of a perfect on/off sampling device, and therefore often the correct way to interpolate between frequencies in a sampled signal.  The resulting function is in fact analytic on the entire complex plane.

\begin{thebibliography}{9}
 \bibitem{gearhart} W.B. Gearhart, H.S.Shultz, \emph{The Function $\frac{\sin x }{x}$},
The College Mathematics Journal, March 1990, Volume 21, Number 2, pp. 90-99.
\PMlinkexternal{(online)}{http://www.maa.org/pubs/Calc_articles/ma003.pdf}.
 \end{thebibliography}
%%%%%
%%%%%
\end{document}
