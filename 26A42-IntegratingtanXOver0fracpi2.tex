\documentclass[12pt]{article}
\usepackage{pmmeta}
\pmcanonicalname{IntegratingtanXOver0fracpi2}
\pmcreated{2013-03-22 15:57:47}
\pmmodified{2013-03-22 15:57:47}
\pmowner{Wkbj79}{1863}
\pmmodifier{Wkbj79}{1863}
\pmtitle{integrating $\tan x$ over $[0,\frac{\pi}{2}]$}
\pmrecord{14}{37977}
\pmprivacy{1}
\pmauthor{Wkbj79}{1863}
\pmtype{Example}
\pmcomment{trigger rebuild}
\pmclassification{msc}{26A42}
\pmclassification{msc}{45-01}
\pmrelated{ImproperLimits}
\pmrelated{OneSidedLimit}

% this is the default PlanetMath preamble.  as your knowledge
% of TeX increases, you will probably want to edit this, but
% it should be fine as is for beginners.

% almost certainly you want these
\usepackage{amssymb}
\usepackage{amsmath}
\usepackage{amsfonts}

% used for TeXing text within eps files
%\usepackage{psfrag}
% need this for including graphics (\includegraphics)
%\usepackage{graphicx}
% for neatly defining theorems and propositions
%\usepackage{amsthm}
% making logically defined graphics
%%%\usepackage{xypic}

% there are many more packages, add them here as you need them

% define commands here

\begin{document}
Note that what is meant by $\displaystyle \int\limits_0^{\frac{\pi}{2}} \tan x \, dx$ is actually $\displaystyle \lim_{t \to \frac{\pi}{2}^-} \int\limits_0^t \tan x \, dx$, since $\tan x$ is defined on $[0, \frac{\pi}{2})$ but not at $\frac{\pi}{2}$.

\begin{center}
$\begin{array}{ll}
\displaystyle \int\limits_0^{\frac{\pi}{2}} \tan x \, dx & \displaystyle = \lim_{t \to \frac{\pi}{2}^-} \int\limits_0^t \tan x \, dx \\
& \\
& \displaystyle  = \lim_{t \to \frac{\pi}{2}^-} \ln |\sec x| \bigg|_0^t \\
& \\
& \displaystyle = \lim_{t \to \frac{\pi}{2}^-} \ln |\sec t| - \ln |\sec 0| \\
& \\
& \displaystyle = \lim_{t \to \frac{\pi}{2}^-} \ln |\sec t| \\
& \\
& \displaystyle = \infty. \end{array}$ \end{center}

%%%%%
%%%%%
\end{document}
