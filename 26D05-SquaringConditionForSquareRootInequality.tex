\documentclass[12pt]{article}
\usepackage{pmmeta}
\pmcanonicalname{SquaringConditionForSquareRootInequality}
\pmcreated{2013-03-22 17:55:56}
\pmmodified{2013-03-22 17:55:56}
\pmowner{pahio}{2872}
\pmmodifier{pahio}{2872}
\pmtitle{squaring condition for square root inequality}
\pmrecord{6}{40427}
\pmprivacy{1}
\pmauthor{pahio}{2872}
\pmtype{Theorem}
\pmcomment{trigger rebuild}
\pmclassification{msc}{26D05}
\pmclassification{msc}{26A09}
\pmsynonym{squaring condition}{SquaringConditionForSquareRootInequality}
\pmrelated{StrangeRoot}

% this is the default PlanetMath preamble.  as your knowledge
% of TeX increases, you will probably want to edit this, but
% it should be fine as is for beginners.

% almost certainly you want these
\usepackage{amssymb}
\usepackage{amsmath}
\usepackage{amsfonts}

% used for TeXing text within eps files
%\usepackage{psfrag}
% need this for including graphics (\includegraphics)
%\usepackage{graphicx}
% for neatly defining theorems and propositions
 \usepackage{amsthm}
% making logically defined graphics
%%%\usepackage{xypic}

% there are many more packages, add them here as you need them

% define commands here

\theoremstyle{definition}
\newtheorem*{thmplain}{Theorem}

\begin{document}
Of the inequalities \;$\sqrt{a} \lessgtr b$, 
\begin{itemize}
\item both are undefined when\; $a < 0$;
\item both can be sidewise squared when\; $a \geqq 0$\, and\, $b \geqq 0$;
\item $\sqrt{a} > b$\, is identically true if\, $a \geqq 0$\, and\, $b < 0$.
\item $\sqrt{a} < b$\, is identically untrue if\, $b < 0$;

\end{itemize}


The above theorem may be utilised for solving inequalities involving square roots.\\

\textbf{Example.}\, Solve the inequality
\begin{align}
\sqrt{2x+3} \;>\; x.
\end{align}
The reality condition\, $2x+3 \geqq 0$\, requires that\, $x \geqq -1\frac{1}{2}$.\, For using the theorem, we distinguish two cases according to the sign of the right hand side:

$1^\circ$:\; $-1\frac{1}{2} \leqq x < 0$.\; The inequality is identically true; we have for (1) the partial solution\, $-1\frac{1}{2} \leqq x < 0$.

$2^\circ$:\; $x \geqq 0$.\; Now we can square both \PMlinkescapetext{sides}, obtaining
$$2x+3 \;>\; x^2$$
$$x^2-2x-3 \;<\; 0$$
The zeros of $x^2\!-\!2x\!-\!3$ are\, $x = 1\pm2$,\, i.e. $-1$ and $3$. Since the graph of the polynomial function is a parabola opening upwards, the polynomial attains its negative values when\, $-1 < x < 3$ (see quadratic inequality).\, Thus we obtain for (1) the partial solution\, $0 \leqq x < 3$.

Combining both partial solutions we obtain the total solution
$$-1\frac{1}{2} \;\leqq\; x \;<\; 3.$$


%%%%%
%%%%%
\end{document}
