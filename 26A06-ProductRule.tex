\documentclass[12pt]{article}
\usepackage{pmmeta}
\pmcanonicalname{ProductRule}
\pmcreated{2013-03-22 12:27:57}
\pmmodified{2013-03-22 12:27:57}
\pmowner{mathcam}{2727}
\pmmodifier{mathcam}{2727}
\pmtitle{product rule}
\pmrecord{12}{32628}
\pmprivacy{1}
\pmauthor{mathcam}{2727}
\pmtype{Theorem}
\pmcomment{trigger rebuild}
\pmclassification{msc}{26A06}
\pmrelated{Derivative}
\pmrelated{ProofOfProductRule}
\pmrelated{ProductRule}
\pmrelated{PowerRule}
\pmrelated{ProofOfPowerRule}
\pmrelated{SumRule}
\pmrelated{ZeroesOfDerivativeOfComplexPolynomial}

\usepackage{amssymb}
\usepackage{amsmath}
\usepackage{amsfonts}
\newcommand{\D}[1]{\ensuremath{\mathrm{d}#1}}
\begin{document}
The \emph{product rule} states that if $f:\mathbb{R}\rightarrow\mathbb{R}$ and $g:\mathbb{R}\rightarrow\mathbb{R}$ are functions in one variable both differentiable at a point $x_0$, then the derivative of the product of the two functions, denoted $f\cdot g$, at $x_0$ is given by
\begin{equation*}
\frac{\D{}}{\D{x}}\left(f\cdot g\right)(x_0) = f(x_0)g'(x_0) + f'(x_0)g(x_0).
\end{equation*}

\subsubsection*{Proof}
See the \PMlinkname{proof of the product rule}{ProofOfProductRule}.
\subsection{Generalized Product Rule}
More generally, for differentiable functions $f_1, f_2,\ldots,f_n$ in one variable, all differentiable at $x_0$, we have
\begin{align*}
D(f_1\cdots f_n)(x_0)=\sum_{i=1}^n\left(f_i(x_0)\cdots f_{i-1}(x_0)\cdot Df_i(x_0)\cdot f_{i+1}(x_0)\cdots f_n(x_0)\right).
\end{align*}

Also see \PMlinkname{Leibniz' rule}{LeibnizRule}.

\subsubsection*{Example}

The derivative of $x\ln|x|$ can be found by application of this rule.
Let $f(x) = x, g(x) = \ln|x|$, so that $f(x)g(x) = x\ln|x|$.  Then $f'(x) = 1$ and
$g'(x) = \frac{1}{x}$.  Therefore, by the product rule,

\begin{eqnarray*}
\frac{\D{}}{\D{x}}(x\ln|x|) & = & f(x)g'(x) + f'(x)g(x) \\
& = & \frac{x}{x} + 1\cdot\ln|x| \\
& = & \ln|x| + 1
\end{eqnarray*}
%%%%%
%%%%%
\end{document}
