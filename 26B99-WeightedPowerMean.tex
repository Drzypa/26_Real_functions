\documentclass[12pt]{article}
\usepackage{pmmeta}
\pmcanonicalname{WeightedPowerMean}
\pmcreated{2013-03-22 11:47:20}
\pmmodified{2013-03-22 11:47:20}
\pmowner{drini}{3}
\pmmodifier{drini}{3}
\pmtitle{weighted power mean}
\pmrecord{12}{30267}
\pmprivacy{1}
\pmauthor{drini}{3}
\pmtype{Definition}
\pmcomment{trigger rebuild}
\pmclassification{msc}{26B99}
\pmclassification{msc}{00-01}
\pmclassification{msc}{26-00}
\pmrelated{ArithmeticGeometricMeansInequality}
\pmrelated{ArithmeticMean}
\pmrelated{GeometricMean}
\pmrelated{HarmonicMean}
\pmrelated{PowerMean}
\pmrelated{ProofOfArithmeticGeometricHarmonicMeansInequality}
\pmrelated{RootMeanSquare3}
\pmrelated{ProofOfGeneralMeansInequality}
\pmrelated{DerivationOfHarmonicMeanAsTheLimitOfThePowerMean}

\usepackage{amssymb}
\usepackage{amsmath}
\usepackage{amsfonts}
\usepackage{graphicx}
%%%%\usepackage{xypic}
\begin{document}
If $w_1,w_2,\ldots,w_n$ are positive real numbers such that $w_1+w_2+\cdots+w_n=1$, we define the \emph{$r$-th weighted power mean} of the $x_i$ as:

$$M_w^r(x_1,x_2,\ldots,x_n)=\left({w_1x_1^r+w_2x_2^r+\cdots+w_nx_n^r}\right)^{1/r}.$$

When all the $w_i=\frac{1}{n}$ we get the standard power mean.
The weighted power mean is a continuous function of $r$, and taking limit when $r\to0$ gives us
$$M_w^0=x_1^{w_1}x_2^{w_2}\cdots w_n^{w_n}.$$

We can weighted use power means to generalize the power means inequality:
If $w$ is a set of weights, and if $r<s$ then
$$M_w^r \leq M_w^s.$$
%%%%%
%%%%%
%%%%%
%%%%%
\end{document}
