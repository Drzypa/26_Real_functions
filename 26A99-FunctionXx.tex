\documentclass[12pt]{article}
\usepackage{pmmeta}
\pmcanonicalname{FunctionXx}
\pmcreated{2014-07-20 12:01:34}
\pmmodified{2014-07-20 12:01:34}
\pmowner{pahio}{2872}
\pmmodifier{pahio}{2872}
\pmtitle{function $x^x$}
\pmrecord{17}{38566}
\pmprivacy{1}
\pmauthor{pahio}{2872}
\pmtype{Example}
\pmcomment{trigger rebuild}
\pmclassification{msc}{26A99}
\pmsynonym{$x$ to the $x$th power}{FunctionXx}
\pmrelated{ExponentialFunction}
\pmrelated{GrowthOfExponentialFunction}
\pmrelated{ImproperIntegral}
\pmrelated{PowerTowerSequence}

% this is the default PlanetMath preamble.  as your knowledge
% of TeX increases, you will probably want to edit this, but
% it should be fine as is for beginners.

% almost certainly you want these
\usepackage{amssymb}
\usepackage{amsmath}
\usepackage{amsfonts}

% used for TeXing text within eps files
%\usepackage{psfrag}
% need this for including graphics (\includegraphics)
%\usepackage{graphicx}
% for neatly defining theorems and propositions
 \usepackage{amsthm}
% making logically defined graphics
%%%\usepackage{xypic}
\usepackage{pstricks}
\usepackage{pst-plot}

% there are many more packages, add them here as you need them

% define commands here

\theoremstyle{definition}
\newtheorem*{thmplain}{Theorem}

\begin{document}
We list some properties of the real function\, $x \mapsto x^x$\,
which may be called {\it 2-tetration}.\, Cf. also [1].\\

\begin{itemize}
  \item The function is defined only for positive values of $x$ (and for negative integers $x$ which we neglect in the following items); cf. fraction power and power function.
  \item $x^x$ can be \PMlinkname{composed}{Composition} of\, $e^t$\, and\, 
$t = x\ln{x}$.
  \item $\displaystyle\lim_{x\to 0+}x^x = 1$,\, as one sees by substituting\, $t = x\ln{x}$\, into the Taylor expansion of $e^t$.
  \item $x^x$ is differentiable: Using the chain rule on\, $e^{x\ln{x}}$, one obtains: 
\[\frac{d}{dx}(x^x) = x^x(1+\ln{x}).\]
  \item The function has absolute minimum value\,
 $\displaystyle e^{-\frac{1}{e}} \approx 0.6922$\, and achieves all real values above this.
  \item $\displaystyle\lim_{x\to\infty}\frac{e^x}{x^x} =
\lim_{x\to\infty}e^{(1-\ln{x})x} = 0$\, (cf. growth of exponential function)
  \item $x^x$ has no elementary function as antiderivative.
  \item $\displaystyle \int_0^1 x^x\,dx = 
\sum_{n=1}^{\infty}\frac{(-1)^{n+1}}{n^n} = 0.7834305107\ldots$ 
(see the entry \PMlinkname{Sophomore's dream}{SophomoreSDream}).
  \item $y := x^x$\, satisfies the differential equations
$$yy'' - (y')^2  - y^2/x = 0$$
and
$$y^3 y''' - y^2 (y'')^2 + 2 y (y')^2 y'' - 3 y^2 y' y'' - (y')^4 + 2 y (y')^3 = 0.$$

\end{itemize}

\begin{center}
\begin{pspicture}(-0.5,-0.5)(3.5,9)
\psaxes[Dx=1,Dy=1]{->}(0,0)(-0.5,-0.5)(3.5,8.5)
\rput(3.55,-0.2){$x$}
\rput(0.2,8.48){$y$}
\rput(-0.3,-0.3){$0$}
\psplot[linecolor=blue]{0.01}{2.4}{x x exp}
\psdot[linecolor=red](0,1)
\psdot[linecolor=blue](0.35,0.69)
\psline[linestyle=dashed](0.35,0)(0.35,0.69)
\rput(0.35,-0.36){$\frac{1}{e}$}
\rput(3,5){$y = x^x$}
\end{pspicture}\\

\begin{thebibliography}{8}
\bibitem{SM}{\sc J. Sondow \& D. Marques}:\, Algebraic and transcendental solutions
of some exponential equations.\, $-$ \emph{Annales Mathematicae et Informaticae} \textbf{37} (2010); available directly at \PMlinkexternal{arXiv}{http://arxiv.org/pdf/1108.6096.pdf}.
\end{thebibliography}





\end{center}

%%%%%
%%%%%
\end{document}
