\documentclass[12pt]{article}
\usepackage{pmmeta}
\pmcanonicalname{DerivativeForParametricForm}
\pmcreated{2013-03-22 17:30:48}
\pmmodified{2013-03-22 17:30:48}
\pmowner{pahio}{2872}
\pmmodifier{pahio}{2872}
\pmtitle{derivative for parametric form}
\pmrecord{9}{39904}
\pmprivacy{1}
\pmauthor{pahio}{2872}
\pmtype{Derivation}
\pmcomment{trigger rebuild}
\pmclassification{msc}{26B05}
\pmclassification{msc}{46G05}
\pmclassification{msc}{26A24}
%\pmkeywords{parametric form}
\pmrelated{GoniometricFormulae}
\pmrelated{CurvatureOfNielsensSpiral}
\pmrelated{Parameter}

% this is the default PlanetMath preamble.  as your knowledge
% of TeX increases, you will probably want to edit this, but
% it should be fine as is for beginners.

% almost certainly you want these
\usepackage{amssymb}
\usepackage{amsmath}
\usepackage{amsfonts}

% used for TeXing text within eps files
%\usepackage{psfrag}
% need this for including graphics (\includegraphics)
%\usepackage{graphicx}
% for neatly defining theorems and propositions
 \usepackage{amsthm}
% making logically defined graphics
%%%\usepackage{xypic}

% there are many more packages, add them here as you need them

% define commands here

\theoremstyle{definition}
\newtheorem*{thmplain}{Theorem}

\begin{document}
Instead of the usual way\, $y = f(x)$\, to present plane curves it is in many cases more comfortable to express both coordinates, $x$ and $y$, by means of a suitable auxiliary variable, the parametre.  It is true e.g. for the cycloid curve.\\

Suppose we have the parametric form
\begin{align}
x = x(t),\quad y = y(t).
\end{align}
For getting now the derivative $\displaystyle\frac{dy}{dx}$ in a point $P_0$ of the curve, we chose another point $P$ of the curve.  If the values of the parametre $t$ corresponding these points are $t_0$ and $t$, we thus have the points\, $(x(t_0),\,y(t_0))$\, and\, $(x(t),\,y(t))$\, and the slope of the secant line through the points is the difference quotient
\begin{align}
\frac{y(t)-y(t_0)}{x(t)-x(t_0)} = \frac{\frac{y(t)-y(t_0)}{t-t_0}}{\frac{x(t)-x(t_0)}{t-t_0}}.
\end{align}
Let us assume that the functions (1) are differentiable when\, $t = t_0$\, and that\, $x'(t_0) \neq 0$.  As we let\, $t\to t_0$, the left side of (2) tends to the derivative $\frac{dy}{dx}$ and the \PMlinkescapetext{right} side to the quotient $\frac{y'(t_0)}{x'(t_0)}$.  Accordingly we have the result
\begin{align}
\left(\frac{dy}{dx}\right)_{\!t=t_0} =\, \frac{y'(t_0)}{x'(t_0)}.
\end{align}
Note that the \PMlinkescapetext{formula} (3)
may be written
$$\frac{dy}{dx} = \frac{\frac{dy}{dt}}{\frac{dx}{dt}}.$$

\textbf{Example.}  For the cycloid
  $$x = a(\varphi-\sin{\varphi}),\quad y = a(1-\cos{\varphi}),$$
we obtain
$$\frac{dy}{dx} = \frac{\frac{d}{d\varphi}(1-\cos\varphi)}{\frac{d}{d\varphi}(\varphi-\sin\varphi)} 
= \frac{\sin\varphi}{1-\cos\varphi} = \cot\frac{\varphi}{2}.$$ 


%%%%%
%%%%%
\end{document}
