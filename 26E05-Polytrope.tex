\documentclass[12pt]{article}
\usepackage{pmmeta}
\pmcanonicalname{Polytrope}
\pmcreated{2014-02-23 21:50:23}
\pmmodified{2014-02-23 21:50:23}
\pmowner{pahio}{2872}
\pmmodifier{pahio}{2872}
\pmtitle{polytrope}
\pmrecord{16}{42110}
\pmprivacy{1}
\pmauthor{pahio}{2872}
\pmtype{Definition}
\pmcomment{trigger rebuild}
\pmclassification{msc}{26E05}
\pmclassification{msc}{26A15}
\pmclassification{msc}{26A09}
\pmclassification{msc}{26A03}
\pmsynonym{polytropic equation}{Polytrope}
\pmrelated{GraphOfEquationXyConstant}
\pmrelated{AsymptotesOfGraphOfRationalFunction}
\pmrelated{PowerFunction}
\pmdefines{cubic hyperbola}

% this is the default PlanetMath preamble.  as your knowledge
% of TeX increases, you will probably want to edit this, but
% it should be fine as is for beginners.

% almost certainly you want these
\usepackage{amssymb}
\usepackage{amsmath}
\usepackage{amsfonts}

% used for TeXing text within eps files
%\usepackage{psfrag}
% need this for including graphics (\includegraphics)
%\usepackage{graphicx}
% for neatly defining theorems and propositions
 \usepackage{amsthm}
% making logically defined graphics
%%%\usepackage{xypic}
\usepackage{pstricks}
\usepackage{pst-plot}

% there are many more packages, add them here as you need them

% define commands here

\theoremstyle{definition}
\newtheorem*{thmplain}{Theorem}

\begin{document}
\PMlinkescapeword{constant}
\PMlinkescapeword{complex}
\PMlinkescapeword{entropy}
\paragraph{Mathematical concept}
Let $n$ be a nonnegative constant.\, A 
\emph{polytropic equation} expresses that the real variable 
$y$ is inversely proportional to the power $x^n$ of the real 
variable $x$.\, So, it is a question of the equation
\begin{align}
y \;=\; \frac{c}{x^n}
\end{align}
where $c$ is another constant.

The graph of a polytropic equation is called a 
\emph{polytrope}.\, It has the coordinate axes as asymptotes 
for\, $n \neq 0$.\, Special cases of polytrope are the 
hyperbola \,$y = \frac{c}{x}$\, and the 
\emph{cubic hyperbola}
$$y = \frac{c}{x^2}.$$\\

Below one sees the \PMlinkescapetext{right} halves of three polytropes given by the integer $n$ values 0 (green), 1 (cyan) and 2 (blue); farther below a whole cubic hyperbola.

\begin{center}
\begin{pspicture}(-0.5,-0.5)(7,5.7) 
\psaxes[Dx=1,Dy=1]{->}(0,0)(5.4,5.4)
\rput(5.6,-0.2){$x$}
\rput(-0.2,5.6){$y$}
\psplot[linecolor=green]{0}{5}{1}
\psplot[linecolor=cyan]{0.2}{5}{1 x div}
\psplot[linecolor=blue]{0.45}{5}{1 x div x div}
\psdot(1,1)
\rput(3.5,3){$\displaystyle y \,=\, \frac{1}{x^n}$ \;for\; $n = 0,\,1,\,2$}
\end{pspicture}
\end{center}

\begin{center}
\begin{pspicture}(-7,-0.5)(7,5.7) 
\psaxes[Dx=1,Dy=1]{->}(0,0)(-5.4,0)(5.4,5.4)
\rput(5.6,-0.2){$x$}
\rput(-0.2,5.6){$y$}
\psdot(0,0)
\psplot[linecolor=blue]{-5}{-0.45}{1 x div x div}
\psplot[linecolor=blue]{0.45}{5}{1 x div x div}
\rput(3.5,3){Cubic hyperbola\;\; $\displaystyle y \,=\, \frac{1}{x^2}$}
\end{pspicture}
\end{center} 


\paragraph{An application to thermodynamics: The reversible polytropic process}
When a gas undergoes a reversible process in which there is heat transfer, the process frequently takes place in such a manner that a plot of $\log{p}$ vs. $\log{v}$ is a straight line. Here $p$ denotes absolute pressure (e.g. in psia) and $v$ specific volume (e.g. in $ft^3/lbm$).  For such a process
$$pv^n = \mbox{constant}.$$
This is called a \emph{polytropic process}.  Indeed this equation represents a \emph{constitutive equation} since it can be verifiable experimentally in a lab for different processes of heat transfer, i.e. for distinct rational values of $n$. It is obvious, from the above equation, that
$$\frac{d \log{p}}{d \log{v}} = -n,$$
where $-n$ is the slope of the straight line on the mentioned logarithmic chart. Typical polytropic processes are tabulated as follow. \\

\begin{center}
\begin{tabular}{|c|c|c|c|}
\hline 
\multicolumn{4}{|c|}{\bf Some Typical Polytropic Processes} \\
\hline 
{\bf Process Name} & {\bf Constant \PMlinkescapetext{Property}} & {\bf $n$} 
& {\bf Physical \PMlinkescapetext{Units}} \\
\hline
Isobaric & $p$ & 0 & lbf/in$^2$ abs. \\
\hline
Isothermal & $T$ & 1 & $^{\circ}\mathrm{R}$ \\
\hline
Isentropic & $s$ & $k$ & Btu/lbm-$^{\circ}\mathrm{R}$ \\
\hline
Isochoric(Isovolumetric) & $v$ & $\infty$ & ft$^3$/lbm \\
\hline
\end{tabular}
\end{center} 

Legend:\, $p$ $=$ abs. pressure;\, $T$ $=$ abs. temperature;\, $s$ $=$ specific entropy;\, $v$ $=$ specific volume \\


In general, one may have more complex processes of heat transfer where $n$ is any rational number (e.g.\, $n = -2,\,-1,\, -0.5$, etc.). \,Specific examples are the expansion of the combustion gases in the cylinders of water-cooled internal combustion engines and reciprocating machines. \,In such cases the pressure and volume during a polytropic process are measured, as might be done with an \emph{engine indicator}, so that the logarithm of the pressure and volume are plotted in \PMlinkescapetext{order} to know the \emph{polytropic constant} $n$.  \\


%%%%%
%%%%%
\end{document}
