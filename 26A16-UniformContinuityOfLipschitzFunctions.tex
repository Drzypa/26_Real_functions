\documentclass[12pt]{article}
\usepackage{pmmeta}
\pmcanonicalname{UniformContinuityOfLipschitzFunctions}
\pmcreated{2013-03-22 15:06:16}
\pmmodified{2013-03-22 15:06:16}
\pmowner{paolini}{1187}
\pmmodifier{paolini}{1187}
\pmtitle{uniform continuity of Lipschitz functions}
\pmrecord{8}{36834}
\pmprivacy{1}
\pmauthor{paolini}{1187}
\pmtype{Theorem}
\pmcomment{trigger rebuild}
\pmclassification{msc}{26A16}

% this is the default PlanetMath preamble.  as your knowledge
% of TeX increases, you will probably want to edit this, but
% it should be fine as is for beginners.

% almost certainly you want these
\usepackage{amssymb}
\usepackage{amsmath}
\usepackage{amsfonts}

% used for TeXing text within eps files
%\usepackage{psfrag}
% need this for including graphics (\includegraphics)
%\usepackage{graphicx}
% for neatly defining theorems and propositions
\usepackage{amsthm}
% making logically defined graphics
%%%\usepackage{xypic}

% there are many more packages, add them here as you need them

% define commands here

\newtheorem{proposition}{Proposition}
\begin{document}
\begin{proposition}
An H\"o{}lder continuous mapping is uniformly continuous.
In particular any Lipschitz continuous mapping is uniformly continuous.
\end{proposition}
\begin{proof}
Let $f\colon X\to Y$ be a mapping such that for some $C>0$ and $\alpha$ with $0 < \alpha \le 1$ one
has
\[
  d_Y(f(p),f(q)) \le C d_X(p,q)^\alpha.
\]
For every given $\epsilon > 0$, choose
$\delta=\left(\epsilon/(C+1)\right)^{\frac 1 \alpha}$.
If $p,q\in X$ are given points satisfying
\[
d_X(p,q)<\delta
\]
then
\[
d_Y(f(p),f(q))\leq C \delta^\alpha \le C\frac{\epsilon}{C+1} < \epsilon,
\]
as desired.
\end{proof}
%%%%%
%%%%%
\end{document}
