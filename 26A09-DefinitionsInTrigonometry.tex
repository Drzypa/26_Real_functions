\documentclass[12pt]{article}
\usepackage{pmmeta}
\pmcanonicalname{DefinitionsInTrigonometry}
\pmcreated{2013-03-22 13:55:08}
\pmmodified{2013-03-22 13:55:08}
\pmowner{Daume}{40}
\pmmodifier{Daume}{40}
\pmtitle{definitions in trigonometry}
\pmrecord{10}{34676}
\pmprivacy{1}
\pmauthor{Daume}{40}
\pmtype{Definition}
\pmcomment{trigger rebuild}
\pmclassification{msc}{26A09}
\pmrelated{Trigonometry}
\pmrelated{Sinusoid}
\pmrelated{ComplexSineAndCosine}
\pmrelated{ExampleOnSolvingAFunctionalEquation}
\pmrelated{DerivativesOfSineAndCosine}
\pmrelated{AdditionFormulasForSineAndCosine}
\pmrelated{AdditionFormulaForTangent}
\pmrelated{GoniometricFormulae}
\pmrelated{OsculatingCurve}
\pmdefines{sine}
\pmdefines{cosine}
\pmdefines{exponential}
\pmdefines{tangent}
\pmdefines{cotangent}
\pmdefines{secant}
\pmdefines{cosecant}
\pmdefines{trigonometric function}

\endmetadata

\usepackage{amssymb}
\usepackage{amsmath}
\usepackage{amsfonts}
\usepackage{graphicx}
\begin{document}
\PMlinkescapeword{terms}
\PMlinkescapeword{names}
\PMlinkescapeword{inverses}
\includegraphics{trig.eps}

\textbf{Informal definitions}

Given a triangle $ABC$ with a signed angle $x$ at $A$ and a
right angle at $B$, the ratios
$$\frac{BC}{AC}\qquad \frac{AB}{AC}\qquad \frac{BC}{AB}$$
are dependent only on the angle $x$, and therefore define functions,
denoted by
$$\sin x\qquad \cos x\qquad \tan x$$
respectively, where the names are short for \emph{sine, cosine} and
\emph{tangent}. Their inverses are rather less important,
but also have names:
\begin{eqnarray*}
\cot x &=& \frac{AB}{BC} = \frac{1}{\tan x} \text{  (cotangent)} \\
\csc x &=& \frac{AC}{BC} = \frac{1}{\sin x} \text{  (cosecant)} \\
\sec x &=& \frac{AC}{AB} = \frac{1}{\cos x} \text{  (secant)}
\end{eqnarray*}
From Pythagoras's theorem we have $\cos^2 x+\sin^2 x = 1$ for all (real) $x$.
Also it is ``clear'' from the diagram at left that functions $\cos$ and $\sin$
are periodic with period $2\pi$. However:

\textbf{Formal definitions}

The above definitions are not fully rigorous, because we have not defined
the word \emph{angle}. We will sketch a more rigorous approach.

The power series
$$\sum_{n=0}^\infty\frac{x^n}{n!}$$
converges uniformly on compact subsets of $\mathbb{C}$ and its sum,
denoted by $\exp(x)$ or by $e^x$, is therefore an entire function of $x$,
called the exponential function.
$f(x)=\exp(x)$ is the unique solution of the boundary value problem
$$f(0)=1\qquad f'(x)=f(x)$$
on $\mathbb{R}$.
The sine and cosine functions, for real arguments, are defined in terms
of $\exp$, simply by
$$\exp(ix)=\cos x + i(\sin x)\;.$$
Thus
$$\cos x = 1-\frac{x^2}{2!}+\frac{x^4}{4!}-\frac{x^6}{6!}+\cdots$$
$$\sin x = \frac{x}{1!}-\frac{x^3}{3!}+\frac{x^5}{5!}-\cdots$$
Although it is not self-evident, $\cos$ and $\sin$ are periodic functions on
the real line, and have the same period. That period is denoted by $2\pi$.
%%%%%
%%%%%
\end{document}
