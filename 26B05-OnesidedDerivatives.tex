\documentclass[12pt]{article}
\usepackage{pmmeta}
\pmcanonicalname{OnesidedDerivatives}
\pmcreated{2013-03-22 15:39:00}
\pmmodified{2013-03-22 15:39:00}
\pmowner{pahio}{2872}
\pmmodifier{pahio}{2872}
\pmtitle{one-sided derivatives}
\pmrecord{9}{37582}
\pmprivacy{1}
\pmauthor{pahio}{2872}
\pmtype{Definition}
\pmcomment{trigger rebuild}
\pmclassification{msc}{26B05}
\pmclassification{msc}{26A24}
\pmsynonym{left derivative}{OnesidedDerivatives}
\pmsynonym{right derivative}{OnesidedDerivatives}
\pmrelated{Differentiable}
\pmrelated{OneSidedLimit}
\pmrelated{DifferntiableFunction}
\pmrelated{OneSidedContinuity}
\pmrelated{SemicubicalParabola}
\pmdefines{left-sided derivative}
\pmdefines{right-sided derivative}

\endmetadata

% this is the default PlanetMath preamble.  as your knowledge
% of TeX increases, you will probably want to edit this, but
% it should be fine as is for beginners.

% almost certainly you want these
\usepackage{amssymb}
\usepackage{amsmath}
\usepackage{amsfonts}

% used for TeXing text within eps files
%\usepackage{psfrag}
% need this for including graphics (\includegraphics)
%\usepackage{graphicx}
% for neatly defining theorems and propositions
 \usepackage{amsthm}
% making logically defined graphics
%%%\usepackage{xypic}

% there are many more packages, add them here as you need them

% define commands here

\theoremstyle{definition}
\newtheorem*{thmplain}{Theorem}
\begin{document}
\begin{itemize}
\item If the real function $f$ is defined in the point $x_0$ and on some interval left from this and if the left-hand one-sided limit\, 
$\lim_{h\to 0-}\frac{f(x_0+h)-f(x_0)}{h}$\, exists, then this limit is defined to be the {\em left-sided derivative} of $f$ in $x_0$.
\item If the real function $f$ is defined in the point $x_0$ and on some interval right from this and if the right-hand one-sided limit\, 
$\lim_{h\to 0+} \frac{f(x_0+h)-f(x_0)}{h}$\, exists, then this limit is defined to be the {\em right-sided derivative} of $f$ in $x_0$.
\end{itemize}

It's apparent that if $f$ has both the left-sided and the right-sided derivative in the point $x_0$ and these are equal, then $f$ is differentiable in $x_0$ and $f'(x_0)$ equals to these one-sided derivatives.\, Also inversely.

\textbf{Example.}\, The real function\, $x \mapsto x\sqrt{x}$\, is defined for\, 
$x \geqq 0$\, and differentiable for\, $x > 0$\, with\, 
$f'(x) \equiv \frac{3}{2}\sqrt{x}$.\, The function also has the right derivative in $0$:
 $$\lim_{h \to 0+}\frac{h\sqrt{h}- 0\sqrt{0}}{h} = \lim_{h \to 0+}\sqrt{h} = 0$$

\textbf{Remark.}\, For a function\, $f\!: [a,\,b] \to \mathbb{R}$,\,
to have a right-sided derivative at\, $x = a$ with value $d$,
is equivalent to saying that there is an extension $g$
of $f$ to some open interval containing\, $[a,\,b]$\,
and satisfying\, $g'(a) = d$.\, Similarly for left-sided derivatives.
%%%%%
%%%%%
\end{document}
