\documentclass[12pt]{article}
\usepackage{pmmeta}
\pmcanonicalname{Majorization}
\pmcreated{2013-03-22 14:30:22}
\pmmodified{2013-03-22 14:30:22}
\pmowner{rspuzio}{6075}
\pmmodifier{rspuzio}{6075}
\pmtitle{majorization}
\pmrecord{8}{36043}
\pmprivacy{1}
\pmauthor{rspuzio}{6075}
\pmtype{Definition}
\pmcomment{trigger rebuild}
\pmclassification{msc}{26D99}
\pmdefines{majorize}
\pmdefines{majorization}

% this is the default PlanetMath preamble.  as your knowledge
% of TeX increases, you will probably want to edit this, but
% it should be fine as is for beginners.

% almost certainly you want these
\usepackage{amssymb}
\usepackage{amsmath}
\usepackage{amsfonts}

% used for TeXing text within eps files
%\usepackage{psfrag}
% need this for including graphics (\includegraphics)
%\usepackage{graphicx}
% for neatly defining theorems and propositions
%\usepackage{amsthm}
% making logically defined graphics
%%%\usepackage{xypic}

% there are many more packages, add them here as you need them

% define commands here
\begin{document}
For any real vector $x = (x_1,x_2,\ldots,x_n)\in \mathbb{R}^n$, let $x_{(1)}\geq x_{(2)} \geq \cdots \geq x_{(n)}$ denote the components of $x$ in non-increasing order.

For $x, y \in \mathbb{R}^n$, we say that $x$ is majorized by $y$, or $y$ majorizes $x$, if
\begin{align*}
 \sum_{i=1}^m x_{(i)} &\leq \sum_{i=1}^m y_{(i)}, \quad \text{ for $m=1,\ldots, n-1$, and} \\
 \sum_{i=1}^n x_{(i)} &= \sum_{i=1}^n y_{(i)}
\end{align*} 
A common notation for ``$x$ is majorized by $y$'' is $x \prec y$.


{\bf Remark:}

A canonical example is that, if $y_1$, $y_2, \ldots, y_n$ are non-negative real numbers such that their sum is equal to 1, then
\[
 \left(\frac{1}{n},\ldots,\frac{1}{n} \right) \prec (y_1,\ldots,y_n).
\]
In general, $x\prec y$ vaguely means that the components of $x$ is less spread out than are the components of~$y$.


{\bf Reference}

\begin{itemize}
\item G. H. Hardy, J. E. Littlewood and G. P{\'o}lya, {\em Inequalities}, 2nd edition, 1952, Cambridge University Press, London.

\item A. W. Marshall and I. Olkin, {\em Inequalities: Theory of Majorization and Its Applications}, 1979, Acadamic Press, New York.
\end{itemize}
%%%%%
%%%%%
\end{document}
