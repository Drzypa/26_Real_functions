\documentclass[12pt]{article}
\usepackage{pmmeta}
\pmcanonicalname{PropertiesOfVectorvaluedFunctions}
\pmcreated{2013-03-22 19:02:42}
\pmmodified{2013-03-22 19:02:42}
\pmowner{pahio}{2872}
\pmmodifier{pahio}{2872}
\pmtitle{properties of vector-valued functions}
\pmrecord{9}{41923}
\pmprivacy{1}
\pmauthor{pahio}{2872}
\pmtype{Topic}
\pmcomment{trigger rebuild}
\pmclassification{msc}{26A42}
\pmclassification{msc}{26A36}
\pmclassification{msc}{26A24}
\pmrelated{ProductAndQuotientOfFunctionsSum}

\endmetadata

% this is the default PlanetMath preamble.  as your knowledge
% of TeX increases, you will probably want to edit this, but
% it should be fine as is for beginners.

% almost certainly you want these
\usepackage{amssymb}
\usepackage{amsmath}
\usepackage{amsfonts}

% used for TeXing text within eps files
%\usepackage{psfrag}
% need this for including graphics (\includegraphics)
%\usepackage{graphicx}
% for neatly defining theorems and propositions
 \usepackage{amsthm}
% making logically defined graphics
%%%\usepackage{xypic}

% there are many more packages, add them here as you need them

% define commands here

\theoremstyle{definition}
\newtheorem*{thmplain}{Theorem}

\begin{document}
If\, $F = (f_1,\,\ldots,\,f_n)$\, and\, $G = (g_1,\,\ldots,\,g_n)$\, are vector-valued and $u$ a real-valued function of the real variable $t$, one defines the vector-valued functions $F\!+\!G$ and $uF$ componentwise as
$$F\!+\!G \;:=\; (f_1\!+\!g_1,\,\ldots,\,f_n\!+\!g_n), \quad uF \;:=\; (uf_1,\,\ldots,\,uf_n)$$
and the real valued dot product as
$$F \cdot G \;:=\; f_1g_1\!+\ldots+\!f_ng_n.$$
If\, $n = 3$,\, one my define also the vector-valued cross product function as
$$F\!\times\!G \;:=\; 
\left( 
\left|\begin{matrix}
f_2 & f_3 \\
g_2 & g_3
\end{matrix}\right|\!,\, 
\left|\begin{matrix}
f_3 & f_1 \\
g_3 & g_1
\end{matrix}\right|\!,\,
\left|\begin{matrix}
f_1 & f_2 \\
g_1 & g_2
\end{matrix}\right|
\right)\!.$$\\

It's not hard to verify, that if $F$, $G$ and $u$ are differentiable on an interval, so are also
$F\!+\!G$, $uF$ and $F\cdot G$, and the formulae
$$(F\!+\!G)' \;=\; F'\!+\!G', \quad (uF)' \;=\; u'F\!+\!uF', \quad (F\cdot G)' \;=\; F'\cdot G+F\cdot G'$$
are valid, in $\mathbb{R}^3$ additionally
$$(F\!\times\!G)' \;=\; F'\!\times\!G+F\!\times\!G'.$$\\

Likewise one can verify the following theorems.\\

\textbf{Theorem 1.}\, If $u$ is continuous in the point $t$ and $F$ in the point $u(t)$, then
$$H \;=\; F\!\circ\!u \;:=\; (f_1\!\circ\!u,\,\ldots,\,f_n\!\circ\!u)$$
is continuous in the point $t$.\, If $u$ is differentiable in the point $t$ and $F$ in the point $u(t)$, then the composite function $H$ is differentiable in $t$ and the chain rule
$$H'(t) \;=\; F'(u(t))\,u'(t)$$
is in \PMlinkescapetext{force}.\\

\textbf{Theorem 2.}\, If $F$ and $G$ are integrable on\, $[a,\,b]$,\, so is also $c_1F\!+\!c_2G$, where $c_1,\,c_2$ are real constants, and 
$$\int_a^b\!(c_1F\!+\!c_2G)\,dt \;=\; c_1\int_a^b\!F\,dt+c_2\int_a^b\!G\,dt.$$\\

\textbf{Theorem 3.}\, Suppose that $F$ is continuous on the interval $I$ and\, $c \in I$.\, Then the vector-valued function 
$$t\; \mapsto\; \int_c^t\!F(\tau)\,d\tau \;:=\; G(t) \quad \forall\, t \in I$$
is differentiable on $I$ and satisfies\, $G' = F$.\\

\textbf{Theorem 4.}\, Suppose that $F$ is continuous on the interval\, $[a,\,b]$\, and $G$ is an arbitrary function such that\, $G' = F$\, on this interval.\, Then
$$\int_a^b\!F(t)\,dt \;=\; G(b)\!-\!G(a).$$

Theorem 2 may be generalised to

\textbf{Theorem 5.}\, If $F$ is integrable on\, $[a,\,b]$\, and\, $C = (c_1,\,\ldots,\,c_n)$\, is an arbitrary vector of $\mathbb{R}^n$, then dot product $C\cdot F$ is integrable on this interval and
$$\int_a^b\!C\cdot F(t)\,dt \;=\; C\cdot\!\int_a^b\!F(t)\,dt.$$





%%%%%
%%%%%
\end{document}
