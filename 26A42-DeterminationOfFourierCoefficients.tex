\documentclass[12pt]{article}
\usepackage{pmmeta}
\pmcanonicalname{DeterminationOfFourierCoefficients}
\pmcreated{2013-03-22 18:22:47}
\pmmodified{2013-03-22 18:22:47}
\pmowner{pahio}{2872}
\pmmodifier{pahio}{2872}
\pmtitle{determination of Fourier coefficients}
\pmrecord{7}{41022}
\pmprivacy{1}
\pmauthor{pahio}{2872}
\pmtype{Derivation}
\pmcomment{trigger rebuild}
\pmclassification{msc}{26A42}
\pmclassification{msc}{42A16}
\pmsynonym{calculation of Fourier coefficients}{DeterminationOfFourierCoefficients}
%\pmkeywords{Fourier coefficients}
\pmrelated{UniquenessOfFourierExpansion}
\pmrelated{FourierSineAndCosineSeries}
\pmrelated{OrthogonalityOfChebyshevPolynomials}

% this is the default PlanetMath preamble.  as your knowledge
% of TeX increases, you will probably want to edit this, but
% it should be fine as is for beginners.

% almost certainly you want these
\usepackage{amssymb}
\usepackage{amsmath}
\usepackage{amsfonts}

% used for TeXing text within eps files
%\usepackage{psfrag}
% need this for including graphics (\includegraphics)
%\usepackage{graphicx}
% for neatly defining theorems and propositions
 \usepackage{amsthm}
% making logically defined graphics
%%%\usepackage{xypic}

% there are many more packages, add them here as you need them

% define commands here

\theoremstyle{definition}
\newtheorem*{thmplain}{Theorem}

\begin{document}
Suppose that the real function $f$ may be presented as sum of the Fourier series:
\begin{align}
f(x) \;=\; \frac{a_0}{2}+\sum_{m=0}^\infty(a_m\cos{mx}+b_m\sin{mx})
\end{align}
Therefore, $f$ is periodic with period $2\pi$.\, For expressing the Fourier coefficients $a_m$ and $b_m$
with the function itself, we first multiply the series (1) by $\cos{nx}$ ($n \in \mathbb{Z}$) and integrate from $-\pi$ to $\pi$.\, Supposing that we can integrate termwise, we may write
\begin{align}
\int_{-\pi}^\pi\!f(x)\cos{nx}\,dx \,=\, \frac{a_0}{2}\!\int_{-\pi}^\pi\!\cos{nx}\,dx
+\!\sum_{m=0}^\infty\!\left(a_m\!\int_{-\pi}^\pi\!\cos{mx}\cos{nx}\,dx+b_m\!\int_{-\pi}^\pi\!\sin{mx}\cos{nx}\,dx\right)\!.
\end{align}
When\, $n = 0$,\, the equation (2) reads
\begin{align}
\int_{-\pi}^\pi f(x)\,dx = \frac{a_0}{2}\cdot2\pi = \pi a_0,
\end{align}
since in the sum of the right hand side, only the first addend is distinct from zero. 

When $n$ is a positive integer, we use the product formulas of the trigonometric identities, getting
$$\int_{-\pi}^\pi\cos{mx}\cos{nx}\,dx 
= \frac{1}{2}\int_{-\pi}^\pi[\cos(m-n)x+\cos(m+n)x]\,dx,$$
$$\int_{-\pi}^\pi\sin{mx}\cos{nx}\,dx 
= \frac{1}{2}\int_{-\pi}^\pi[\sin(m-n)x+\sin(m+n)x]\,dx.$$
The latter expression vanishes always, since the sine is an odd function.\, If\, $m \neq n$,\, the former equals zero because the antiderivative consists of sine terms which vanish at multiples of $\pi$; only in the case\, $m = n$\, we obtain from it a non-zero result $\pi$.\, Then (2) reads
\begin{align}
\int_{-\pi}^\pi f(x)\cos{nx}\,dx = \pi a_n
\end{align}
to which we can include as a special case the equation (3).

By multiplying (1) by $\sin{nx}$ and integrating termwise, one obtains similarly
\begin{align}
\int_{-\pi}^\pi f(x)\sin{nx}\,dx = \pi b_n.
\end{align}
The equations (4) and (5) imply the formulas
$$a_n \;=\; \frac{1}{\pi}\int_{-\pi}^\pi f(x)\cos{nx}\,dx \quad (n = 0,\,1,\,2,\,\ldots)$$
and
$$b_n \;=\; \frac{1}{\pi}\int_{-\pi}^\pi f(x)\sin{nx}\,dx \quad (n = 1,\,2,\,3,\,\ldots)$$
for finding the values of the Fourier coefficients of $f$.



%%%%%
%%%%%
\end{document}
