\documentclass[12pt]{article}
\usepackage{pmmeta}
\pmcanonicalname{FractionPower}
\pmcreated{2014-09-21 12:12:39}
\pmmodified{2014-09-21 12:12:39}
\pmowner{pahio}{2872}
\pmmodifier{pahio}{2872}
\pmtitle{fraction power}
\pmrecord{21}{37340}
\pmprivacy{1}
\pmauthor{pahio}{2872}
\pmtype{Definition}
\pmcomment{trigger rebuild}
\pmclassification{msc}{26A03}
\pmsynonym{fractional power}{FractionPower}
%\pmkeywords{fractional number exponent}
\pmrelated{PowerFunction}
\pmrelated{GeneralPower}
\pmrelated{IntegrationOfFractionPowerExpressions}
\pmrelated{NthRootFormulas}

% this is the default PlanetMath preamble.  as your knowledge
% of TeX increases, you will probably want to edit this, but
% it should be fine as is for beginners.

% almost certainly you want these
\usepackage{amssymb}
\usepackage{amsmath}
\usepackage{amsfonts}

% used for TeXing text within eps files
%\usepackage{psfrag}
% need this for including graphics (\includegraphics)
%\usepackage{graphicx}
% for neatly defining theorems and propositions
 \usepackage{amsthm}
% making logically defined graphics
%%%\usepackage{xypic}

% there are many more packages, add them here as you need them

% define commands here

\theoremstyle{definition}
\newtheorem*{thmplain}{Theorem}
\begin{document}
Let $m$ be an integer and $n$ a positive factor of $m$.\, If $x$ is a positive real number, we may write the identical equation
       $$(x^{\frac{m}{n}})^n \;=\; x^{\frac{m}{n}\cdot n} \;=\; x^m$$
and therefore the definition of \PMlinkname{$n^\mathrm{th}$ root}{NthRoot} gives the \PMlinkescapetext{formula}
\begin{align}
\sqrt[n]{x^m} \;=\;  x^{\frac{m}{n}}.
\end{align}
Here, the exponent $\frac{m}{n}$ is an integer.\, For enabling the validity of (1) for the cases where $n$ does not divide $m$ we must set the following

\textbf{Definition.}\, Let $\frac{m}{n}$\, be a fractional number, i.e. an integer $m$ not divisible by the integer $n$, which latter we assume to be positive.\, For any positive real number $x$ we define the\, {\em fraction power}\, $x^{\frac{m}{n}}$ as the $n^\mathrm{th}$ \PMlinkescapetext{root}
\begin{align}
             x^{\frac{m}{n}} \;:=\; \sqrt[n]{x^m}.
\end{align}

\textbf{Remarks}
\begin{enumerate}

\item The existence of the \PMlinkescapetext{root} in the right 
hand side of (2) is proved 
\PMlinkname{here}{existenceofnthroot}.

\item The defining equation (2) is independent on the form of the exponent $\frac{m}{n}$:\, If\, $\frac{k}{l} = \frac{m}{n}$,\, then we have\, $(\sqrt[n]{x^m})^{ln} = [(\sqrt[n]{x^m})^n]^l = x^{lm} = x^{kn} = 
[(\sqrt[l]{x^k})^l]^n = (\sqrt[l]{x^k})^{ln}$,\, and because the mapping\, 
$y\mapsto y^{ln}$\, is injective in $\mathbb{R}_+$, the positive numbers $\sqrt[l]{x^k}$ and $\sqrt[n]{x^m}$ must be equal.

\item The fraction power function\, $x\mapsto x^{\frac{m}{n}}$\, is a special case of power function.

\item The presumption that $x$ is positive signifies that one can not identify all $n^\mathrm{th}$ \PMlinkname{roots}{NthRoot} $\sqrt[n]{x}$ and the powers $x^{\frac{1}{n}}$.\, For example, $\sqrt[3]{-8}$ equals $-2$ and\, 
$\frac{2}{6} = \frac{1}{3}$,\, but one \textbf{must not} \PMlinkescapetext{calculate}
 $$(-8)^{\frac{1}{3}} \;=\; (-8)^{\frac{2}{6}} \;=\; \sqrt[6]{(-8)^2} \;=\; \sqrt[6]{64} \;=\; 2.$$
The point is that $(-8)^{\frac{1}{3}}$ is not defined in $\mathbb{R}$.\, Here we have\, $l = 6$\, and the mapping\, 
$y\mapsto y^{ln}$\, is not injective in\, $\mathbb{R}_-\cup\mathbb{R}_+$.\,
 --- Nevertheless, some people and books may use also for negative $x$ the equality\, $\sqrt[n]{x} = x^{\frac{1}{n}}$\, and more generally\, $\sqrt[n]{x^m} = x^{\frac{m}{n}}$\, where one then insists that\, $\gcd(m,\,n) = 1.$

\item According to the preceding item, for the negative values of $x$ the derivative of \PMlinkname{odd roots}{NthRoot}, e.g. $\sqrt[3]{x}$, ought to be calculated as follows:
$$\frac{d\sqrt[3]{x}}{dx} \;=\; \frac{d(-\sqrt[3]{-x})}{dx} 
\;=\; -\frac{d(-x)^\frac{1}{3}}{dx} \;=\; -\frac{1}{3}\!\cdot\!(-x)^{-\frac{2}{3}}(-1) 
\;=\;\frac{1}{3\sqrt[3]{(-x)^2}} \;=\; \frac{1}{3\sqrt[3]{x^2}}$$
The result is similar as $\frac{d\sqrt[3]{x}}{dx}$ for positive $x$'s, although the \PMlinkescapetext{odd} root functions are not special cases of the power function.
\end{enumerate}
%%%%%
%%%%%
\end{document}
