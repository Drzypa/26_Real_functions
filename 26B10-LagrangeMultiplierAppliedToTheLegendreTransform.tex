\documentclass[12pt]{article}
\usepackage{pmmeta}
\pmcanonicalname{LagrangeMultiplierAppliedToTheLegendreTransform}
\pmcreated{2013-03-22 18:52:47}
\pmmodified{2013-03-22 18:52:47}
\pmowner{dh2718}{16929}
\pmmodifier{dh2718}{16929}
\pmtitle{Lagrange multiplier applied to the Legendre transform}
\pmrecord{7}{41727}
\pmprivacy{1}
\pmauthor{dh2718}{16929}
\pmtype{Example}
\pmcomment{trigger rebuild}
\pmclassification{msc}{26B10}

% this is the default PlanetMath preamble.  as your knowledge
% of TeX increases, you will probably want to edit this, but
% it should be fine as is for beginners.

% almost certainly you want these
\usepackage{amssymb}
\usepackage{amsmath}
\usepackage{amsfonts}

% used for TeXing text within eps files
%\usepackage{psfrag}
% need this for including graphics (\includegraphics)
%\usepackage{graphicx}
% for neatly defining theorems and propositions
%\usepackage{amsthm}
% making logically defined graphics
%%%\usepackage{xypic}

% there are many more packages, add them here as you need them

% define commands here

\begin{document}
Since the Legendre transform is a stationary point of a function, Lagrange multipliers should be the natural choice for handling constraints. However, there is a problem due to the fact that we are mostly interested in its functional dependence on the transform parameter, and not on its value.\\\\
\textbf{THE LEGENDRE-LAGRANGE PROBLEM}\\

Let $f(\overline{x})$ be a function of the real n-vector $\overline{x}$ and $g(\overline{p})$ its Legendre transform defined by
\begin{align}
g(\overline{p}) \;=\;\overline{p}.\overline{x}-f(\overline{x})\quad\quad p_i\;=\;\frac{\partial f}{\partial x_i}
\end{align}
The vector $\overline{p}$ is a function of $\overline{x}$, or conversely, $\overline{x}$ is a function of $\overline{p}$, so that $g(\overline{p})$ is a function of $\overline{p}$ alone.
The Legendre transform is also alternatively defined as the maximum of the function $\overline{p}.\overline{x}-f(\overline{x})$. This maximum is reached for $\overline{x}$ satifying the second set of equations in (1).
Suppose now that the components $x_i$ are not all independent, but are linked by the constraint:
\begin{align}
h(\overline{x})\;=\;0
\end{align}
This equation defines one of the components, $x_\alpha$ for example, as a function of all the others. Putting this function into (1) would give us the transform $g(\overline{p})$ as a function of the vector $\overline{p}$ with n-1 components. It would be nice if we could instead use a Lagrange multiplier $k$ and compute the maximum of the function $$\overline{p}.\overline{x}-f(\overline{x})+kh(\overline{x})$$ with all its n components. But then, how does the constraint (2) on $\overline{x}$ translate to $\overline{p}$? The answer is amazingly simple, as we shall see next.\\\\
\textbf{DIRECT COMPUTATION OF THE TRANSFORM}\\

We are going first to compute the Legendre transform the hard way, without the help of a multiplier, by considering $x_\alpha$ as a function of the other components: 
\begin{align}
p_i=\frac{df}{dx_i}=\frac{\partial f}{\partial x_i}+\frac{\partial f}{\partial x_\alpha}\frac{\partial x_\alpha}{\partial x_i}\quad\quad \frac{dh}{dx_i}=\frac{\partial h}{\partial x_i}+\frac{\partial h}{\partial x_\alpha}\frac{\partial x_\alpha}{\partial x_i}=0\quad\quad i\neq\alpha
\end{align}
Taking the value of \Large $\frac{\partial x_\alpha}{\partial x_i}$ \normalsize from the second set of equations (3) and putting it into the first set, we get:
\begin{align}
p_i\;=\;\frac{\partial f}{\partial x_i}-\Phi\frac{\partial h}{\partial x_i}\quad \mbox{with} \quad \Phi\;=\;\frac{\frac{\partial f}{\partial x_\alpha}}{\frac{\partial h}{\partial x_\alpha}}\quad\mbox{and}\quad i\neq\alpha 
\end{align}
From definition (1), the Legendre transform $g$ is therefore:
\begin{align}
g(\overline{p})\;=\;\sum_{i\neq\alpha} x_i\left( \frac{\partial f}{\partial x_i} - \Phi\frac{\partial h}{\partial x_i}\right)-f(\overline{x})
\end{align}\\
\textbf{BACK TO THE LAGRANGE MULTIPLIER}\\

In the first equation (4), if we set $i=\alpha$ we get $p_\alpha = 0$ and conversely, getting the value of $\Phi$. So, in (5), we may remove the condition $i\neq\alpha$ and sum over all the n components of $\overline{p}$. The constraint $p_\alpha = 0$ reduces the number of independent components to n-1. In fact, we are back to the traditional Lagrange method with the multiplier $\Phi$ and an additional constraint. We even have the choice between n such constraints. They generate up to n functionally different transforms corresponding to the n possible forms of the function $\overline{x}$, according to which component $x_\alpha$ we choose to eliminate. 
The method extends easily to the case of more than one constraint:\\ $h_1=h_2=...h_m=0$. We use m multipliers $\Phi_1, \Phi_2...\Phi_m$. They are computed by equating to zero any set of m components from $\overline{p}$.
\begin{thebibliography}{1}
\bibitem {A} \PMlinkexternal{Fersanz at PM - Legendre Transform}{http://planetmath.org/encyclopedia/LegendreTransform.html}\\
This link is actually broken but, hopefully, should be operative soon.
\bibitem {B} \PMlinkexternal{Wikipedia - Legendre transformation}{http://en.wikipedia.org/wiki/Legendre_transformation}
\end{thebibliography}

%%%%%
%%%%%
\end{document}
