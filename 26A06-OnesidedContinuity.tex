\documentclass[12pt]{article}
\usepackage{pmmeta}
\pmcanonicalname{OnesidedContinuity}
\pmcreated{2013-03-22 17:57:50}
\pmmodified{2013-03-22 17:57:50}
\pmowner{pahio}{2872}
\pmmodifier{pahio}{2872}
\pmtitle{one-sided continuity}
\pmrecord{6}{40469}
\pmprivacy{1}
\pmauthor{pahio}{2872}
\pmtype{Definition}
\pmcomment{trigger rebuild}
\pmclassification{msc}{26A06}
\pmrelated{OneSidedLimit}
\pmrelated{OneSidedDerivatives}
\pmrelated{OneSidedContinuityBySeries}
\pmdefines{continuous from the left}
\pmdefines{continuous from the right}
\pmdefines{from the left continuous}
\pmdefines{from the right continuous}
\pmdefines{continuous on closed interval}

\endmetadata

% this is the default PlanetMath preamble.  as your knowledge
% of TeX increases, you will probably want to edit this, but
% it should be fine as is for beginners.

% almost certainly you want these
\usepackage{amssymb}
\usepackage{amsmath}
\usepackage{amsfonts}

% used for TeXing text within eps files
%\usepackage{psfrag}
% need this for including graphics (\includegraphics)
%\usepackage{graphicx}
% for neatly defining theorems and propositions
 \usepackage{amsthm}
% making logically defined graphics
%%%\usepackage{xypic}

% there are many more packages, add them here as you need them

% define commands here

\theoremstyle{definition}
\newtheorem*{thmplain}{Theorem}

\begin{document}
The real function $f$ is {\em continuous from the left} in the point\, $x = x_0$\, iff
$$\lim_{x\to x_0-}f(x) = f(x_0).$$

The real function $f$ is {\em continuous from the right} in the point\, $x = x_0$\, iff
$$\lim_{x\to x_0+}f(x) = f(x_0).$$

The real function $f$ is {\em continuous on the closed interval} \,$[a,\,b]$\, iff it is continuous at all points of the open interval \,$(a,\,b)$,\, from the right continuous at $a$ and from the left continuous at $b$.\\


\textbf{Examples.}\, The ceiling function $\lceil{x}\rceil$ is from the left continuous at each integer, the mantissa function $x\!-\!\lfloor{x}\rfloor$ is from the right continuous at each integer.





%%%%%
%%%%%
\end{document}
