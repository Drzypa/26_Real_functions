\documentclass[12pt]{article}
\usepackage{pmmeta}
\pmcanonicalname{WeierstrassCriterionOfUniformConvergence}
\pmcreated{2013-03-22 14:38:21}
\pmmodified{2013-03-22 14:38:21}
\pmowner{pahio}{2872}
\pmmodifier{pahio}{2872}
\pmtitle{Weierstrass' criterion of uniform convergence}
\pmrecord{9}{36225}
\pmprivacy{1}
\pmauthor{pahio}{2872}
\pmtype{Theorem}
\pmcomment{trigger rebuild}
\pmclassification{msc}{26A15}
\pmclassification{msc}{40A30}
\pmsynonym{Weierstrass' M-test}{WeierstrassCriterionOfUniformConvergence}

\endmetadata

% this is the default PlanetMath preamble.  as your knowledge
% of TeX increases, you will probably want to edit this, but
% it should be fine as is for beginners.

% almost certainly you want these
\usepackage{amssymb}
\usepackage{amsmath}
\usepackage{amsfonts}

% used for TeXing text within eps files
%\usepackage{psfrag}
% need this for including graphics (\includegraphics)
%\usepackage{graphicx}
% for neatly defining theorems and propositions
 \usepackage{amsthm}
% making logically defined graphics
%%%\usepackage{xypic}

% there are many more packages, add them here as you need them

% define commands here
\theoremstyle{definition}
\newtheorem*{thmplain}{Theorem}
\begin{document}
\begin{thmplain}
\, \,Let the real functions $f_1(x)$, $f_2(x)$, ... be defined in the interval $[a, b]$. \,If they all \PMlinkescapetext{satisfy} the condition 
         $$|f_n(x)| \leqq M_n \quad \forall\,x\in[a, b],$$
with $\sum_{n = 1}^{\infty}M_n$ a convergent series of \PMlinkescapetext{constant terms}, then the function series 
                      $$f_1(x)\!+\!f_2(x)\!+\!\cdots$$
\PMlinkname{converges uniformly}{SumFunctionOfSeries} on the interval $[a, b]$.
\end{thmplain}

The theorem is valid also for the series with complex function terms, when one replaces the interval with a subset of $\mathbb{C}$.
%%%%%
%%%%%
\end{document}
