\documentclass[12pt]{article}
\usepackage{pmmeta}
\pmcanonicalname{IteratedLimitInmathbbR2}
\pmcreated{2013-03-22 17:28:27}
\pmmodified{2013-03-22 17:28:27}
\pmowner{pahio}{2872}
\pmmodifier{pahio}{2872}
\pmtitle{iterated limit in $\mathbb{R}^2$}
\pmrecord{10}{39860}
\pmprivacy{1}
\pmauthor{pahio}{2872}
\pmtype{Definition}
\pmcomment{trigger rebuild}
\pmclassification{msc}{26B12}
\pmclassification{msc}{26A06}
\pmdefines{iterated limit}

% this is the default PlanetMath preamble.  as your knowledge
% of TeX increases, you will probably want to edit this, but
% it should be fine as is for beginners.

% almost certainly you want these
\usepackage{amssymb}
\usepackage{amsmath}
\usepackage{amsfonts}

% used for TeXing text within eps files
%\usepackage{psfrag}
% need this for including graphics (\includegraphics)
%\usepackage{graphicx}
% for neatly defining theorems and propositions
 \usepackage{amsthm}
% making logically defined graphics
%%%\usepackage{xypic}

% there are many more packages, add them here as you need them

% define commands here

\theoremstyle{definition}
\newtheorem*{thmplain}{Theorem}

\begin{document}
Let $f$ be a function from a subset $S$ of\, $\mathbb{R}^2$\, to\, $\mathbb{R}$ and\, $(a,\, b)$\, an accumulation point of $S$.  The limits
$$\lim_{x\to a}\left(\lim_{y\to b}f(x,\,y)\right) \quad\mbox{and}\quad \lim_{y\to b}\left(\lim_{x\to a}f(x,\,y)\right)$$
are called {\em iterated limits}.\\

\textbf{Example 1.}  If\; $\displaystyle f(x,\,y) := \frac{x\sin\frac{1}{x}+y}{x+y}$,\, then 
\begin{itemize}
\item $\lim_{x\to 0}\left(\lim_{y\to 0}f(x,\,y)\right) = \lim_{x\to0}\sin\frac{1}{x}$ does not exist
\item $\lim_{y\to 0}\left(\lim_{x\to 0}f(x,\,y)\right) = \lim_{y\to0}1 = 1$
\item the usual limit $\lim_{(x,y)\to(0,0)}f(x,\,y)$ does not exist.
\end{itemize}

\textbf{Example 2.}  If\; $\displaystyle f(x,\,y) := \frac{x^2}{x^2+y^2}$,\, then 
\begin{itemize}
\item $\lim_{x\to 0}\left(\lim_{y\to 0}f(x,\,y)\right) = \lim_{x\to0} \frac{x^2}{x^2} = 1$
\item $\lim_{y\to 0}\left(\lim_{x\to 0}f(x,\,y)\right) = \lim_{y\to0} 0 = 0$
\item the usual limit $\lim_{(x,y)\to(0,0)}f(x,\,y)$ again does not exist, \PMlinkescapetext{even} though both of the iterated limits do.
\end{itemize}

So far we have studied examples that present discontinuity at its point of accumulation. We now expose an illustrative example where such discontinuity can be avoided. \\

\textbf{Example 3.}  Consider the function
$$f(x,\,y) := \frac{x\sin{x}\cosh{y}+y\cos{x}\sinh{y}}{x^2+y^2};$$
then (we apply \PMlinkname{l'H\^opital's rule}{LHpitalsRule} throughout)
\begin{itemize}
\item 
$\lim_{x\to 0}\left(\lim_{y\to 0}f(x,\,y)\right) =
\lim_{x\to 0}\left(\lim_{y\to 0}\frac{x\sin x\cosh y+y\cos x\sinh y}{x^2+y^2}\right)= 
\lim_{x\to 0}\frac{x\sin x}{x^2}=\lim_{x\to 0}\frac{\sin x}{x}=\lim_{x\to 0}\cos x=1$
\item 
$\lim_{y\to 0}\left(\lim_{x\to 0}f(x,\,y)\right) =
\lim_{y\to 0}\left(\lim_{x\to 0}\frac{x\sin x\cosh y+y\cos x\sinh y}{x^2+y^2}\right)= 
\lim_{y\to 0}\frac{y\sinh y}{y^2}=\lim_{y\to 0}\frac{\sinh y}{y}=\lim_{y\to 0}\cosh y=1$
\item  the usual limit $\lim_{(x,y)\to(0,0)}f(x,\,y)$ exists in this case.  An essential reason which assures the continuity of this function, arises from the fact that\, $f(x,\,y) \equiv \Re(\frac{\sin z}{z})$,\, $z = x+iy$,\, i.e. it is the real part of the analytic function \,$w := \frac{\sin z}{z}$\, having the removable singularity at\, $z = 0$ (see the entry complex sine and cosine). 
\end{itemize}


%%%%%
%%%%%
\end{document}
