\documentclass[12pt]{article}
\usepackage{pmmeta}
\pmcanonicalname{ExampleOfInfinitesimalHyperrealNumber}
\pmcreated{2013-03-22 17:25:57}
\pmmodified{2013-03-22 17:25:57}
\pmowner{asteroid}{17536}
\pmmodifier{asteroid}{17536}
\pmtitle{example of infinitesimal hyperreal number}
\pmrecord{4}{39808}
\pmprivacy{1}
\pmauthor{asteroid}{17536}
\pmtype{Example}
\pmcomment{trigger rebuild}
\pmclassification{msc}{26E35}

% this is the default PlanetMath preamble.  as your knowledge
% of TeX increases, you will probably want to edit this, but
% it should be fine as is for beginners.

% almost certainly you want these
\usepackage{amssymb}
\usepackage{amsmath}
\usepackage{amsfonts}

% used for TeXing text within eps files
%\usepackage{psfrag}
% need this for including graphics (\includegraphics)
%\usepackage{graphicx}
% for neatly defining theorems and propositions
%\usepackage{amsthm}
% making logically defined graphics
%%%\usepackage{xypic}

% there are many more packages, add them here as you need them

% define commands here

\begin{document}
The hyperreal number $\{\frac{1}{n}\}_{n \in \mathbb{N}}\; \in {}^*\mathbb{R}\;$ is infinitesimal.

{\bf Proof -} Let $\mathcal{F}$ be the nonprincipal ultrafilter \PMlinkescapetext{fixed} in the \PMlinkescapetext{parent} \PMlinkname{entry}{Hyperreal}.

$\{n \in \mathbb{N} : 0 < \frac{1}{n}\} = \mathbb{N} \in \mathcal{F}\;\;\;\;$ so $\;\;0 < \{\frac{1}{n}\}_{n \in \mathbb{N}}$.

Given any positive $a \in \mathbb{R}$ we have that $\{n \in \mathbb{N} : a \leq \frac{1}{n}\}$ is finite,
 so $\{n \in \mathbb{N} : \frac{1}{n} < a\} \in \mathcal{F}$ and therefore $\{\frac{1}{n}\}_{n \in \mathbb{N}} < \{a\}_{n \in \mathbb{N}}$.

Thus $0 < \{\frac{1}{n}\}_{n \in \mathbb{N}} < \{a\}_{n \in \mathbb{N}}$ for every positive real number $\{a\}_{n \in \mathbb{N}} \in \mathbb{R}$, and so $\{\frac{1}{n}\}_{n \in \mathbb{N}}\;$ is infinitesimal.$\square$
%%%%%
%%%%%
\end{document}
