\documentclass[12pt]{article}
\usepackage{pmmeta}
\pmcanonicalname{CircularSegment}
\pmcreated{2013-03-22 19:05:02}
\pmmodified{2013-03-22 19:05:02}
\pmowner{pahio}{2872}
\pmmodifier{pahio}{2872}
\pmtitle{circular segment}
\pmrecord{10}{41972}
\pmprivacy{1}
\pmauthor{pahio}{2872}
\pmtype{Definition}
\pmcomment{trigger rebuild}
\pmclassification{msc}{26B10}
\pmclassification{msc}{51M04}
%\pmkeywords{chord}
%\pmkeywords{circular arc}
\pmrelated{LineSegment}
\pmrelated{SphericalSegment}
\pmrelated{ExampleOfCalculusOfVariations}
\pmdefines{height of circular segment}

\endmetadata

% this is the default PlanetMath preamble.  as your knowledge
% of TeX increases, you will probably want to edit this, but
% it should be fine as is for beginners.

% almost certainly you want these
\usepackage{amssymb}
\usepackage{amsmath}
\usepackage{amsfonts}

% used for TeXing text within eps files
%\usepackage{psfrag}
% need this for including graphics (\includegraphics)
%\usepackage{graphicx}
% for neatly defining theorems and propositions
 \usepackage{amsthm}
% making logically defined graphics
%%%\usepackage{xypic}
\usepackage{pstricks}
\usepackage{pst-plot}

% there are many more packages, add them here as you need them

% define commands here

\theoremstyle{definition}
\newtheorem*{thmplain}{Theorem}

\begin{document}
A chord of a circle \PMlinkescapetext{divides} the corresponding disk into two \emph{circular segments}.\, The perimetre of a circular segment consists thus of the chord ($c$) and a circular arc ($a$).

The magnitude $r$ of the radius of circle and the magnitude $\alpha$ of a central angle naturally determine uniquely the magnitudes of the corresponding arc and chord, and these may be directly calculated from
\begin{align}
\begin{cases}
a \;=\; r\alpha,\\
c \;=\; 2r\sin\frac{\alpha}{2}.
\end{cases}
\end{align}
Conversely, the magnitudes of $a$ and $c$ ($< a$) uniquely determine $r$ and $\alpha$ from the pair of equations (1), but $r$ and $\alpha$ are generally not \PMlinkescapetext{expressible} in a closed form; this becomes clear from the relationship\, $\frac{c}{a}\cdot\frac{\alpha}{2} = \sin\frac{\alpha}{2}$\, implied by (1).\\

\begin{center}
\begin{pspicture}(-4,-3)(4,3)
\psdot[linewidth=0.02](0,0)
\psarc[linecolor=blue](0,0){2.5}{20}{130}
\psarc[linecolor=red](0,0){2.5}{130}{380}
\psline[linecolor=red,linewidth=0.04](-1.607,1.915)(2.349,0.855)
\psline[linecolor=blue](-1.607,1.915)(2.349,0.855)
\psline[linestyle=dotted](-1.607,1.915)(0,0)(2.349,0.855)
\rput(0.1,0.3){$\alpha$}
\rput(0.6,2.6){$a$}
\rput(0.35,1.15){$c$}
\rput(-0.9,0.8){$r$}
\rput(-4,-3){.}
\rput(4,3){.}
\end{pspicture}
\end{center}

The area of a circular segment is obtained by subtracting from [resp. adding to] the area of the corresponding sector the area of the isosceles triangle having the chord as \PMlinkname{base}{BaseAndHeightOfTriangle} [the adding concerns the case where the central angle is greater than the straight angle]:
$$A \;=\; \frac{\alpha}{2\pi}\cdot\pi r^2\mp\frac{1}{2}r^2\sin\alpha \;=\;\frac{r^2}{2}(\alpha\mp\sin\alpha)$$\\

The \PMlinkescapetext{\emph{height}} of the circular segment, i.e. the distance of the \PMlinkname{midpoints}{ArcLength} of the arc and the chord, may be expressed in the following forms:
$$h \;=\; \left(1-\cos\frac{\alpha}{2}\right)r \;=\; r-\sqrt{r^2-\frac{c^2}{4}} \;=\; \frac{c}{2}\tan\frac{\alpha}{4}$$


%%%%%
%%%%%
\end{document}
