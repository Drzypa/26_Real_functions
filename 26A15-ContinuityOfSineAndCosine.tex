\documentclass[12pt]{article}
\usepackage{pmmeta}
\pmcanonicalname{ContinuityOfSineAndCosine}
\pmcreated{2013-03-22 19:15:37}
\pmmodified{2013-03-22 19:15:37}
\pmowner{pahio}{2872}
\pmmodifier{pahio}{2872}
\pmtitle{continuity of sine and cosine}
\pmrecord{4}{42189}
\pmprivacy{1}
\pmauthor{pahio}{2872}
\pmtype{Theorem}
\pmcomment{trigger rebuild}
\pmclassification{msc}{26A15}

% this is the default PlanetMath preamble.  as your knowledge
% of TeX increases, you will probably want to edit this, but
% it should be fine as is for beginners.

% almost certainly you want these
\usepackage{amssymb}
\usepackage{amsmath}
\usepackage{amsfonts}

% used for TeXing text within eps files
%\usepackage{psfrag}
% need this for including graphics (\includegraphics)
%\usepackage{graphicx}
% for neatly defining theorems and propositions
 \usepackage{amsthm}
% making logically defined graphics
%%%\usepackage{xypic}

% there are many more packages, add them here as you need them

% define commands here

\theoremstyle{definition}
\newtheorem*{thmplain}{Theorem}

\begin{document}
\textbf{Theorem.}\, The real functions \;$x\mapsto\sin{x}$\; 
and\; $x\mapsto\cos{x}$\; are continuous at every real number $x$.\\

{\em Proof.}\, Let $\varepsilon$ be an arbitrary positive number.\, 
Denote\, $\Delta\sin{x} =: \sin{z}-\sin{x}$,\, 
$\Delta\cos{x} =: \cos{z}-\cos{x}$\, where we suppose that\, 
$|z-x| < \frac{\pi}{2}$.\, We may interpret $|z-x|$ as an arc 
of the unit circle of the $xy$-plane.\, Let's think in the 
circle the right triangle with hypotenuse the chord of the arc and 
the catheti (i.e. the shorter sides) vertical and horizontal.\, Then 
$|\Delta\sin{x}|$ and $|\Delta\cos{x}|$ are just these cathets; so we have
$$|\Delta\sin{x}| \;\leqq\; |z-x|, \quad |\Delta\cos{x}| \;\leqq\; |z-x|.$$
If we make\, $|z-x| < \varepsilon$,\, then also\, $|\Delta\sin{x}|$ and 
$|\Delta\cos{x}|$ are less than $\varepsilon$.\, It means that both 
functions are continuous at $x$.



\begin{thebibliography}{9}
\bibitem{NP}{\sc E. Lindel\"of:} {\em Johdatus korkeampaan analyysiin}. Nelj\"as painos.\, Werner S\"oderstr\"om Osakeyhti\"o, Porvoo ja Helsinki (1956).
\end{thebibliography}
%%%%%
%%%%%
\end{document}
