\documentclass[12pt]{article}
\usepackage{pmmeta}
\pmcanonicalname{ExampleOfTestsForLocalExtremaInLagrangeMultiplierMethod}
\pmcreated{2013-03-22 19:12:19}
\pmmodified{2013-03-22 19:12:19}
\pmowner{scineram}{4030}
\pmmodifier{scineram}{4030}
\pmtitle{example of tests for local extrema in Lagrange multiplier method}
\pmrecord{9}{42120}
\pmprivacy{1}
\pmauthor{scineram}{4030}
\pmtype{Example}
\pmcomment{trigger rebuild}
\pmclassification{msc}{26B12}
\pmclassification{msc}{49K35}
\pmclassification{msc}{49-00}
%\pmkeywords{lagrange multiplier function}

\endmetadata

% this is the default PlanetMath preamble.  as your knowledge
% of TeX increases, you will probably want to edit this, but
% it should be fine as is for beginners.

% almost certainly you want these
\usepackage{amssymb}
\usepackage{amsmath}
\usepackage{amsfonts}

% used for TeXing text within eps files
%\usepackage{psfrag}
% need this for including graphics (\includegraphics)
%\usepackage{graphicx}
% for neatly defining theorems and propositions
\usepackage{amsthm}
% making logically defined graphics
%%%\usepackage{xypic}

% there are many more packages, add them here as you need them

% define commands here

\begin{document}
Let $n\in\mathbb{N}^+$ and $c\in\mathbb{R}$. We want to find the local extrema of the function
\[f\colon\mathbb{R}^n\to\mathbb{R},\quad x\mapsto\sum_{1\le i<j\le n}x_ix_j\]
subject to the condition $g=0$, where
\[g\colon\mathbb{R}^n\to\mathbb{R},\quad x\mapsto\sum_{1\le i\le n}x_i-c.\]
The first and second order partial derivatives are for all $i,j\in\{1,\dots,n\}$
\[\partial_if(x)=\sum_{k\neq i}x_k,\quad\partial_ig(x)=1,\]
\[\partial_i\partial_jf(x)=1-\delta_{i,j},\quad\partial_i\partial_jg(x)=0,\]
where $\delta_{i,j}$ is the Kroenecker-delta. Thus the necessary condition $f'(x)=\lambda g'(x)$ together with $g(x)=0$ gives the system of equations
\[\sum_{j\neq i}x_j=\lambda,\quad i\in\{1,\dots,n\},\]
\[\sum_{1\le j\le n}x_j=c.\]
By summing the first $n$ equations and then substituting in the last we get
\[(n-1)c=n\lambda,\]
\[x_i=\sum_{1\le j\le n}x_j-\sum_{j\neq i}x_j=c-\lambda=\frac{c}{n},\quad i\in\{1,\dots,n\}.\]
Thus there is only one point, where local extremum is possible. We apply the test in the parent entry to the matrix
\[D^2(f-\lambda g)(x)=[1-\delta_{i,j}]_{i,j=1}^n=nP-I,\]
where $P$ is the matrix containing $1/n$ in all entries, and $I$ is the identity matrix. $P$ is a rank one projection. Therefore the second derivative has spectrum $\sigma(nP-I)=\{n-1,-1\}$, where $-1$ has multiplicity $n-1$, and $n-1$ has multiplicity $1$. Thus the second derivative of $f-\lambda g$ is indefinit, so it has no local extrema. However the nullspace of $g'(x)$ is precisely the nullspace of $P$, thus the second derivative is strictly negative on the tangent space $T_x(M)$, so the vector $(c/n,\dots,c/n)$ is a local maximum of $f$ subject to $g=0$.
%%%%%
%%%%%
\end{document}
