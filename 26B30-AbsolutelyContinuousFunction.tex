\documentclass[12pt]{article}
\usepackage{pmmeta}
\pmcanonicalname{AbsolutelyContinuousFunction}
\pmcreated{2013-03-22 15:18:47}
\pmmodified{2013-03-22 15:18:47}
\pmowner{matte}{1858}
\pmmodifier{matte}{1858}
\pmtitle{absolutely continuous function}
\pmrecord{13}{37116}
\pmprivacy{1}
\pmauthor{matte}{1858}
\pmtype{Definition}
\pmcomment{trigger rebuild}
\pmclassification{msc}{26B30}
\pmclassification{msc}{26A46}
\pmrelated{SingularFunction}
\pmrelated{AbsolutelyContinuous}
\pmdefines{fundamental theorem of calculus for the Lebesgue integral}

\endmetadata

% this is the default PlanetMath preamble.  as your knowledge
% of TeX increases, you will probably want to edit this, but
% it should be fine as is for beginners.

% almost certainly you want these
\usepackage{amssymb}
\usepackage{amsmath}
\usepackage{amsfonts}
\usepackage{amsthm}

\usepackage{mathrsfs}

% used for TeXing text within eps files
%\usepackage{psfrag}
% need this for including graphics (\includegraphics)
%\usepackage{graphicx}
% for neatly defining theorems and propositions
%
% making logically defined graphics
%%%\usepackage{xypic}

% there are many more packages, add them here as you need them

% define commands here

\newcommand{\sR}[0]{\mathbb{R}}
\newcommand{\sC}[0]{\mathbb{C}}
\newcommand{\sN}[0]{\mathbb{N}}
\newcommand{\sZ}[0]{\mathbb{Z}}

 \usepackage{bbm}
 \newcommand{\Z}{\mathbbmss{Z}}
 \newcommand{\C}{\mathbbmss{C}}
 \newcommand{\F}{\mathbbmss{F}}
 \newcommand{\R}{\mathbbmss{R}}
 \newcommand{\Q}{\mathbbmss{Q}}



\newcommand*{\norm}[1]{\lVert #1 \rVert}
\newcommand*{\abs}[1]{| #1 |}



\newtheorem{thm}{Theorem}
\newtheorem{defn}{Definition}
\newtheorem{prop}{Proposition}
\newtheorem{lemma}{Lemma}
\newtheorem{cor}{Corollary}
\begin{document}
\PMlinkescapeword{order}

\PMlinkescapetext{Absolute continuity} is the precise condition one needs to 
impose in order for the fundamental theorem of calculus 
to hold for the Lebesgue integral.

\PMlinkescapeword{absolutely continuous}
\PMlinkescapeword{property}
{\bf Definition}
Suppose $[a,b]$ be a closed bounded interval of $\R$.
Then a function $f\colon [a,b]\to\C$ is 
   {\bf absolutely continuous} on $[a,b]$,
if for any $\varepsilon>0$, there is a $\delta>0$ such that the following
condition holds:
\begin{itemize}
\item[($\ast$)] If $(a_1,b_1), \ldots, (a_n,b_n)$ is a finite
collection of disjoint open intervals in $[a,b]$
such that
$$ 
  \sum_{i=1}^n (b_i-a_i)< \delta,
$$
then
$$ 
   \sum_{i=1}^n |f(b_i)-f(a_i)|< \varepsilon.
$$
\end{itemize}

\begin{thm}[\PMlinkescapetext{Fundamental theorem of calculus for the Lebesgue integral}]
Let $f\colon [a,b] \to \C$ be a
function. Then $f$ is absolutely continuous if and only if
there is a function $g\in L^1(a,b)$ (i.e. a $g\colon(a,b)\to \C$ with
$\displaystyle \int_a^b |g|< \infty$), such that
$$ 
   f(x) = f(a) + \int_a^x g(t) dt
$$
for all $x\in[a,b]$.
What is more, if $f$ and $g$ are as above, then $f$ is differentiable
almost everywhere and $f'=g$
almost everywhere. (Above, both integrals are Lebesgue integrals.)
\end{thm}

See \cite{jones,aliprantis} for proof.

See also \cite{wikiabs}, and \cite{barcenas} for a discussion 
about different proofs.

\begin{thebibliography}{9}
\bibitem{wikiabs} Wikipedia, entry on 
   \PMlinkexternal{Absolute continuity}{http://en.wikipedia.org/wiki/Absolute_continuity}.
\bibitem{jones}
F. Jones, \emph{Lebesgue Integration on Euclidean Spaces},
Jones and Barlett Publishers, 1993.
\bibitem{aliprantis}
C.D. Aliprantis, O. Burkinshaw, \emph{Principles of Real Analysis},
2nd ed., Academic Press, 1990.
\bibitem{barcenas}  D. B'arcenas,
\emph{The Fundamental Theorem of
Calculus for Lebesgue Integral},
Divulgaciones Matem\'aticas, Vol. 8, No. 1, 2000, pp. 75-85.
\end{thebibliography}
%%%%%
%%%%%
\end{document}
