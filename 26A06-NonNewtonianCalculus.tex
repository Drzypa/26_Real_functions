\documentclass[12pt]{article}
\usepackage{pmmeta}
\pmcanonicalname{NonNewtonianCalculus}
\pmcreated{2016-06-21 10:28:10}
\pmmodified{2016-06-21 10:28:10}
\pmowner{smithpith}{22173}
\pmmodifier{smithpith}{22173}
\pmtitle{Non-Newtonian Calculus}
\pmrecord{1523}{41754}
\pmprivacy{1}
\pmauthor{smithpith}{22173}
\pmtype{Topic}
\pmclassification{msc}{26A06}
\pmclassification{msc}{00-02}
%\pmkeywords{Non-Newtonian}
%\pmkeywords{calculus}
%\pmkeywords{calculi}
%\pmkeywords{nonlinear}
%\pmkeywords{multiplicative}
%\pmkeywords{derivative}
%\pmkeywords{average}
%\pmkeywords{integral}
%\pmkeywords{means}
%\pmkeywords{geometric}
%\pmkeywords{anageometric}
%\pmkeywords{bigeometric}
%\pmkeywords{quadratic}
%\pmkeywords{anaquadratic}
%\pmkeywords{biquadratic}
%\pmkeywords{harmonic}
%\pmkeywords{anaharmonic}
%\pmkeywords{biharmonic}
%\pmkeywords{Jane Grossman}
%\pmkeywords{Michael Grossman}
%\pmkeywords{Robert Katz}

\endmetadata

% this is the default PlanetMath preamble.  as your knowledge
% of TeX increases, you will probably want to edit this, but
% it should be fine as is for beginners.

% almost certainly you want these
\usepackage{amssymb}
\usepackage{amsmath}
\usepackage{amsfonts}

% used for TeXing text within eps files
%\usepackage{psfrag}
% need this for including graphics (\includegraphics)
%\usepackage{graphicx}
% for neatly defining theorems and propositions
%\usepackage{amsthm}
% making logically defined graphics
%%%\usepackage{xypic}

% there are many more packages, add them here as you need them

% define commands here

\begin{document}



Non-Newtonian Calculus







Home
Contents

Home
Multiplicative Calculus
Brief History
Applications
Citations
Reviews
Comments
Quotations
References
Links/Reading
Appendix 1
Appendix 2
Appendix 3
Dedication




Brief Description
The non-Newtonian calculi are alternatives to the classical calculus of Newton and Leibniz. They provide a wide variety of mathematical tools for use in science, engineering, and mathematics.
There are infinitely many non-Newtonian calculi. Like the classical calculus, each of them possesses, among other things: a derivative, an integral, a natural average, a special class of functions having a constant derivative, and two Fundamental Theorems which reveal that the derivative and integral are 'inversely' related. Nevertheless, most non-Newtonian calculi are markedly different from the classical calculus.     
     
For example, infinitely many non-Newtonian calculi are nonlinear in the sense that each of them has a nonlinear derivative or integral. Among these calculi are the geometric calculus, bigeometric calculus, harmonic calculus, biharmonic calculus, quadratic calculus, and biquadratic calculus. Furthermore, in the geometric calculus and in the bigeometric calculus, the derivative and integral are both multiplicative. (Please see the Multiplicative Calculus section of this website.)
Of course, in the classical calculus the linear functions are the functions having a constant derivative. However, in the geometric calculus, the exponential functions are the functions having a constant derivative. And in the bigeometric calculus, the power functions are the functions having a constant derivative. (The geometric derivative and the bigeometric derivative are closely related to the well-known logarithmic derivative and elasticity, respectively.)

The well-known arithmetic average (of functions) is the natural average in the classical calculus, but the well-known geometric average is the natural average in the geometric calculus. And the well-known harmonic average and quadratic average (or root mean square) are closely related to the natural averages in the harmonic and quadratic calculi, respectively.

Furthermore, unlike the classical derivative, the bigeometric derivative is scale invariant (or scale free), i.e., it is invariant under all changes of scale (or unit) in function arguments and in function values.
Non-Newtonian calculus has been applied in a variety of scientific, engineering, and mathematical fields. In particular, the geometric and bigeometric calculi have been used extensively. For more information, please see the Applications and Citations sections of this website.

NOTE. The aforementioned geometric calculus should not be confused with the geometric calculus advocated by David Hestenes involving so-called Clifford algebras.
NOTE. Non-Newtonian calculus was recommended as a topic for the 21st-century college-mathematics-curriculum - at the 27th International Conference on Technology in Collegiate Mathematics (March of 2015). The conference is sponsored by Pearson PLC, the largest education company and the largest book publisher in the world. (Please see item [224] in the References section.)
NOTE. The six books on non-Newtonian calculus and related matters by Jane Grossman, Michael Grossman, and Robert Katz are indicated below, and are available at some academic libraries, public libraries, and booksellers such as Amazon.com. On the Internet, each of the books can be read and downloaded, free of charge, at HathiTrust, Google Books, and the Digital Public Library of America.

Michael Grossman and Robert Katz.  Non-Newtonian Calculus, ISBN 0912938013, 1972. [15] 
Michael Grossman. The First Nonlinear System of Differential and Integral Calculus, ISBN 0977117006, 1979. (The geometric calculus) [11] 
Jane Grossman, Michael Grossman, Robert Katz. The First Systems of Weighted Differential and Integral Calculus, ISBN 0977117014, 1980. [9]
Jane Grossman. Meta-Calculus: Differential and Integral, ISBN 0977117022, 1981. [7]
Michael Grossman. Bigeometric Calculus: A System with a Scale-Free Derivative, ISBN 0977117030, 1983. [10]
Jane Grossman, Michael Grossman, and Robert Katz. Averages: A New Approach, ISBN 0977117049, 1983. [8]



A Quotation from Gauss

The following Carl Friedrich Gauss quotation is from Carl Friedrich Gauss: Werke, Volume 8, page 298; and from Memorabilia Mathematica or The Philomath's Quotation Book (1914) by Robert Edouard Moritz, quotation #1215.

"In general the position as regards all such new calculi is this - That one cannot accomplish by them anything that could not be accomplished without them. However, the advantage is, that, provided such a calculus corresponds to the inmost nature of frequent needs, anyone who masters it thoroughly is able - without the unconscious inspiration of genius which no one can command - to solve the respective problems, indeed to solve them mechanically in complicated cases in which, without such aid, even genius becomes powerless. Such is the case with the invention of general algebra, with the differential calculus, and in a more limited region with Lagrange's calculus of variations, with my calculus of congruences, and with Mobius' calculus. Such conceptions unite, as it were, into an organic whole countless problems which otherwise would remain isolated and require for their separate solution more or less application of inventive genius."




Three Reviews of the Book Non-Newtonian Calculus

The following review was written by David Pearce MacAdam, and appeared in the Journal of the Optical Society of America (Volume 63, January of 1973), a publication of  The Optical Society, which is a member of the American Institute of Physics.

"This [Non-Newtonian Calculus] is an exciting little book, for two reasons: first, its content, and second, its presentation. The content consists primarily of a brief nonaxiomatic description of the fundamentals of various calculi ...

"For each calculus, a gradient, a derivative, and an average are defined; a basic theorem (essentially a mean-value theorem) is stated; an integral is defined; and a fundamental theorem of integral calculus is stated. In Chapter 1, the authors review the essentials of Newtonian calculus and establish the format of presentation that they follow in the later chapters on the various non-Newtonian calculi. Classical problems studied by such men as Galileo and Newton provide examples of, and motivation for, the authors' work. The authors' reference to these problems serves to emphasize the relevance of their results to the concerns of modern science. The greatest value of these non-Newtonian calculi may prove to be their ability to yield simpler physical laws than the Newtonian calculus. Throughout, this book exhibits a clarity of vision characteristic of important mathematical creations.

"The authors have written this book for engineers and scientists, as well as for mathematicians. They have made a dramatic break with tradition and omitted all proofs and many of the mathematical details that place so much of contemporary mathematics writing out of the reach of scientists and engineers. Instead, they have included details that help develop an intuitive conception of their calculi and relate their calculi to well-knows classical problems. The authors apparently feel that their results are of sufficient importance to scientists and engineers to justify these departures from the more traditional style of writing in mathematics. The writing is clear, concise, and very readable. No more than a working knowledge of [classical] calculus is assumed. Mathematicians who feel that their results are of importance to scientists and engineers, but who find little interest among those workers for their results, might consider presenting their work as Grossman and Katz have done. This would do much to reopen channels of communication between mathematicians and scientists and do much to advance both disciplines."

The following excerpt is from a review written by Ivor Grattan-Guinness that appeared in Middlesex Math Notes (Volume 3, pages 47 - 50, 1977), a publication of Middlesex University in London, England.

"There is enough here [in Non-Newtonian Calculus] to indicate that non-Newtonian calculi ... have considerable potential as alternative approaches to traditional problems. This very original piece of mathematics will surely expose a number of missed opportunities in the history of the subject."


The following excerpt is from a review written by H. Gollmann that appeared in Internationale Mathematische Nachrichten (Number 105, 1972), a publication of Österreichische Mathematische Gesellschaft in Vienna, Austria.

"The possibilities opened up by the [non-Newtonian] calculi seem to be immense."



Acknowledgement                                                                                                                                                           

Thanks to David Lukas and Kenneth Lukas for their expert advice about the Internet and website construction. In 2003, they suggested the idea of using the Internet to broadcast information about non-Newtonian calculus. Dave, who set up the first NNC website, is a computer engineer. Ken, a talented musician, uses the computer to compose music and to create videos and movies. They knew that posting information on the Internet would be an effective way to inform people about non-Newtonian calculus. That information has turned out to be of considerable interest to many researchers who until they read about it on the Internet knew nothing about non-Newtonian calculus, but subsequently found applications in various fields. 



Contacts
    
      Name: Michael Grossman
      E-mail address: smithpith@yahoo.com

      Name: Robert Katz
      Address: 12 Green Street; Rockport, MA 01966; United States  



Last Edit
21 June 2016

=====================================================================================================================================================
Multiplicative Calculus



Contents

Home
Multiplicative Calculus
Brief History
Applications
Citations
Reviews
Comments
Quotations
References
Links/Reading
Appendix 1
Appendix 2
Appendix 3
Dedication


The geometric and bigeometric calculi have been the two most often used non-Newtonian calculi. In each of these two calculi, the use of multiplication/division to combine/compare numbers is crucial.

Each of these two calculi is a 'multiplicative calculus' in the sense that its derivative and integral are multiplicative operators. It turns out that infinitely many non-Newtonian calculi are multiplicative calculi. (But, infinitely many non-Newtonian calculi are not multiplicative calculi.)


Because there are many multiplicative calculi,  the expression "the multiplicative calculus" should be avoided, and no one specific calculus should be named "multiplicative calculus". Nevertheless, some authors have used the name "multiplicative calculus" for the geometric calculus, while others have used the same name for the bigeometric calculus. It is hoped that the scientific community will soon reach accord with regard to names for these two calculi. Our suggestion is simply to use the names "geometric calculus" and "bigeometric calculus", respectively. Interestingly, this matter is discussed by Dorota Aniszewska and Marek Rybaczuk in their article "Multiplicative Hénon map" [288].

Similarly, the expression "the product calculus" should be avoided, and no one specific calculus should be named "product calculus".

Furthermore, some authors have used the expression "Volterra multiplicative calculus" for a system created by Vito Volterra in 1887 for the purpose of solving systems of linear (classical) differential equations.  Since neither the derivative nor the integral in Volterra's system is a multiplicative operator, the Volterra system is not a multiplicative calculus.  [143]

In fact, some authors have erroneously referred to the bigeometric calculus as the "Volterra calculus". The Volterra system is not a non-Newtonian calculus, and is markedly different from both the bigeometric calculus and the geometric calculus. (Vito Volterra, 1860 - 1940, was a brilliant and influential Jewish/Italian scientist.) [143]

NOTE.  Various applications of the geometric and bigeometric calculi are indicated in the Applications and Citations sections of this website. 
NOTE. The six books on non-Newtonian calculus and related matters by Jane Grossman, Michael Grossman, and Robert Katz are indicated below, and are available at some academic libraries, public libraries, and booksellers such as Amazon.com. On the Internet, each of the books can be read and downloaded, free of charge, at HathiTrust, Google Books, and the Digital Public Library of America.
Michael Grossman and Robert Katz.  Non-Newtonian Calculus, ISBN 0912938013, 1972.  [15] 
Michael Grossman. The First Nonlinear System of Differential and Integral Calculus, ISBN 0977117006, 1979. (The geometric calculus) [11] 
Jane Grossman, Michael Grossman, Robert Katz. The First Systems of Weighted Differential and Integral Calculus, ISBN 0977117014, 1980. [9]
Jane Grossman. Meta-Calculus: Differential and Integral, ISBN 0977117022, 1981. [7]
Michael Grossman. Bigeometric Calculus: A System with a Scale-Free Derivative, ISBN 0977117030, 1983. [10]
Jane Grossman, Michael Grossman, and Robert Katz. Averages: A New Approach, ISBN 0977117049, 1983. [8] 

=========================================================================================================================================================================
Brief History


Contents

Home
Multiplicative Calculus
Brief History
Applications
Citations
Reviews
Comments
Quotations
References
Links/Reading
Appendix 1
Appendix 2
Appendix 3
Dedication


The non-Newtonian calculi were created in the period from 1967 to 1970 by Michael Grossman and Robert Katz. In July of 1967, they created an infinite family of calculi that includes the classical calculus, the geometric calculus, the harmonic calculus, and the quadratic calculus. In August of 1970, they created infinitely-many other calculi, including the bigeometric calculus, the anageometric calculus, the biharmonic calculus, the anaharmonic calculus, the biquadratic calculus, and the anaquadratic calculus. All of the calculi can be described simultaneously within the framework of a general theory. They decided to use the adjective "non-Newtonian" to indicate any of the calculi other than the classical calculus.

In 1972, Grossman and Katz completed their book Non-Newtonian Calculus. [15]  It contains discussions of the nine aforementioned non-Newtonian calculi, the general theory of non-Newtonian calculus, and heuristic guides for application. Subsequently, with Jane Grossman, they wrote several other books/articles on non-Newtonian calculus, and on related matters such as weighted calculus, meta-calculus, averages, and means. (Please see items [7 - 15, 34, 35] in the References.)


Michael Grossman and Robert Katz began their development of non-Newtonian calculus on 14 July 1967. Prior to that day, non-Newtonian calculus was unknown to them, and (apparently) to everyone else. In Non-Newtonian Calculus (1972), Grossman and Katz included the following paragraph (page 82):
"However, since we have nowhere seen a discussion of even one specific non-Newtonian calculus, and since we have not found a notion that encompasses the *-average, we are inclined to the view that the non-Newtonian calculi have not been known and recognized heretofore. But only the mathematical community can decide that."



NOTE. For more information about the history of non-Newtonian calculus, please see Appendix 2 of this website.

NOTE. The six books on non-Newtonian calculus and related matters by Jane Grossman, Michael Grossman, and Robert Katz are indicated below, and are available at some academic libraries, public libraries, and booksellers such as Amazon.com. On the Internet, each of the books can be read and downloaded, free of charge, at HathiTrust, Google Books, and the Digital Public Library of America.
Michael Grossman and Robert Katz.  Non-Newtonian Calculus, ISBN 0912938013, 1972. [15] 
Michael Grossman. The First Nonlinear System of Differential and Integral Calculus, ISBN 0977117006, 1979. (The geometric calculus) [11] 
Jane Grossman, Michael Grossman, Robert Katz. The First Systems of Weighted Differential and Integral Calculus, ISBN 0977117014, 1980. [9]
Jane Grossman. Meta-Calculus: Differential and Integral, ISBN 0977117022, 1981. [7]
Michael Grossman. Bigeometric Calculus: A System with a Scale-Free Derivative, ISBN 0977117030, 1983. [10]
Jane Grossman, Michael Grossman, and Robert Katz. Averages: A New Approach, ISBN 0977117049, 1983. [8]

=====================================================================================================================================================
Applications
Contents

Home
Multiplicative Calculus
Brief History
Applications
Citations
Reviews
Comments
Quotations
References
Links/Reading
Appendix 1
Appendix 2
Appendix 3
Dedication


Non-Newtonian calculus has application in a variety of scientific, engineering, and mathematical fields. Among them are fractal theory, including fractal structures and fractal materials; dynamical systems, including chaos theory and Lorenz systems; differential equations, including Runge-Kutta methods; computer science, including imaging, signal processing, and artificial-intelligence; and growth/decay analysis in economics, finance, biology, and chemistry. Some examples are indicated below.


In the abstracts to his two seminars both called "Non-Newtonian calculus for the dynamics of random fractal structures", Wojbor Woyczynski (Case Western Reserve University) asserted: "Many natural phenomena, from microscopic bacteria growth, through macroscopic turbulence, to the large scale structure of the Universe, display a fractal character. For studying the time evolution of such "rough" objects, the classical, "smooth" Newtonian calculus is not enough." In an abstract to another seminar he asserted: "Random fractals, a quintessentially 20th century idea, arise as natural models of various physical, biological (think your mother's favorite cauliflower dish), and economic (think Wall Street, or the Horseshoe Casino) phenomena, and they can be characterized in terms of the mathematical concept of fractional dimension. Surprisingly, their  time evolution can be analyzed by employing a non-Newtonian calculus utilizing integration and differentiation of fractional order." [90, 104, 146]

Non-Newtonian calculus is used by Martin Ostoja-Starzewski and his research team in their work on fractal materials at the University of Illinois at Urbana-Champaign. [163] From Professor Ostoja-Starzewski's 2013 media-upload "The inner workings of fractal materials", University of Illinois at Urbana-Champaign: "Together with a small, highly focused research team, Ostoja-Starzewski is working across disciplines to unite methods from solid mechanics, advanced continuum mechanics, statistical physics and mathematics. Some of the specific mathematical theories they use include probability theory and non-Newtonian calculus. These approaches allow them to focus on different fractal structures, including morphogenesis of fractals at elastic-inelastic transitions in solids, composites and soils, as well as materials that have anomalous heat conduction properties and fractal patterns that are seen in biological materials."

The bigeometric calculus was used in an article on fractals and multiplicative dynamical systems by Dorota Aniszewska and Marek Rybaczuk (both from Wroclaw University of Technology in Poland). [131] In that article they state: "Describing the evolution of defects [in materials] treated as fractals implies usage of the multiplicative derivative, because the ordinary [classical] additive derivative of a function depending on fractal dimension or measure does not exist. ... The goal of this paper is chaos examination in multiplicative dynamical systems described with the multiplicative derivative." (The expression "multiplicative derivative" refers here to the bigeometric derivative.)

The bigeometric calculus was used in an article on fractals and material science by M. Rybaczuk and P. Stoppel (both from Wroclaw University of Technology in Poland). [18]

The bigeometric calculus was used in the article "Critical growth of fractal patterns in biological systems"  by Marek Rybaczuk of Wroclaw University of Technology in Poland. [238]

The bigeometric calculus was used in an article on fractal dimension and dimensional spaces by Marek Rybaczuk (Wroclaw University of Technology in Poland), Alicja Kedziab (Medical Academy of Wroclaw in Poland), and Witold Zielinskia (Wroclaw University of Technology in Poland). [132] 

The bigeometric calculus was used in the book Measurements, Dimensions, Invariant Models and Fractals by Wacław Kasprzak, Bertold Lysik, Marek Rybaczuk (all from Wroclaw University of Technology in Poland). [203]

The bigeometric calculus was used in the article "Physical stability and critical effects in models of fractal defects evolution based on single fractal approximation" by Dorota Aniszewska and Marek Rybaczuk (both from Wroclaw University of Technology in Poland). [228]

The bigeometric calculus was used in an article [126] on chaos in multiplicative dynamical systems by Dorota Aniszewska and Marek Rybaczuk, both from the Wroclaw University of Technology in Poland. They showed that "all classical conditions concerning chaotic behavior can be extended to multiplicative [dynamical] systems". Their work involves one-dimensional multiplicative versions of logistic equations, and multi-dimensional nonlinear dynamical systems described by means of the bigeometric derivative. A multiplicative version of the classical Lorenz system (as well as the Lyapunov exponent and the Runge-Kutta method) was used "for analysis of stability and chaotic behavior".
The bigeometric calculus was used in an article on multiplicative differential equations by Dorota Aniszewska (Wroclaw University of Technology in Poland). [1, 129]

The bigeometric calculus was used in an article on a multiplicative Lorenz system by Dorota Aniszewska and Marek Rybaczuk (both from Wroclaw University of Technology in Poland). [130]
The bigeometric calculus was used by Marek Rybaczuk (Wroclaw University of Technology in Poland) in his lecture "Fractal Models of Defects Evolution" at the 2004 South African Conference on Applied Mechanics. [258]
A perceptive discussion about non-Newtonian calculi and proper usage of the expression "multiplicative calculus" is included in the article "Multiplicative Hénon map" by Dorota Aniszewska and Marek Rybaczuk (both from Wroclaw University of Technology in Poland). (Please see the Multiplicative Calculus section of this website.) The article's main topic is an application of geometric arithmetic [15], namely a multiplicative version of the Hénon map, which is a function that arose in the study of dynamical systems, fractals, and chaos theory. [288]

According to [21], "in dimensional spaces (in a similar way to physical quantities) you can multiply and divide quantities which have different dimensions but you cannot add and subtract quantities with different dimensions. This means that the classical additive derivative is undefined because the difference f(x+deltax)-f(x) has no value. However in dimensional spaces, the geometric derivative and the bigeometric derivative remain well-defined. Multiplicative dynamical systems can become chaotic even when the corresponding classical additive system does not because the additive and multiplicative derivatives become inequivalent if the variables involved also have a varying fractal dimension."

The geometric calculus and the bigeometric calculus were used by Bugce Eminaga (Girne American University in Cyprus), Hatice Aktore (Eastern Mediterranean University in North Cyprus), and Mustafa Riza (Eastern Mediterranean University in North Cyprus) in their article re dynamical systems called "A modified quadratic Lorenz attractor". [237] From that article: "In Section 3, the modified quadratic Lorenz attractor is translated into geometric and bigeometric calculus, and the solutions of the the system are obtained using the corresponding multiplicative Runge-Kutta methods."

The geometric calculus was used by Mustafa Riza (Eastern Mediterranean University in North Cyprus), Hatice Aktore (Eastern Mediterranean University in North Cyprus), and Bugce Eminaga (Girne American University in Cyprus) in their lecture re dynamical systems called "A modified quadratic Lorenz attractor in geometric multiplicative calculus" at the 28th International Conference of the Jangjeon Mathematical Society. [233]
The geometric calculus was used by Luc Florack and Hans van Assen (both of the Eindhoven University of Technology in the Netherlands) in their work on biomedical image analysis and "complex imaging frameworks such as diffusion tensor imaging". In their article "Multiplicative calculus in biomedical image analysis" [88], they assert: "We advocate the use of an alternative calculus in biomedical image analysis, known as multiplicative (a.k.a. non-Newtonian) calculus. ... The purpose of this article is to provide a condensed review of multiplicative calculus and to illustrate its potential use in biomedical image analysis. ... Examples have been given in the context of cardiac strain analysis and diffusion tensor imaging to illustrate the relevance of multiplicative calculus in biomedical image analysis, and to support our recommendation for further investigation into practical as well as fundamental issues." (The expression "multiplicative calculus" refers here to the geometric calculus.) In Professor Florack's article [96] he states: "Multiplicative calculus provides a natural framework in problems involving positive images and positivity preserving operators. In increasingly important, complex imaging frameworks, such as diffusion tensor imaging, it complements standard calculus in a nontrivial way. The purpose of this article is to illustrate the basics of multiplicative calculus and its application to the regularization of positive definite matrix fields." (The expression "multiplicative calculus" refers here to the geometric calculus.) [88, 96, 199, 111]

The advantage of using the geometric or bigeometric calculus in some applications was demonstrated by Nico Persch, Christopher Schroers, Simon Setzer, and Joachim Weickert (Gottfried Wilhelm Leibniz Prize winner), all from Saarland University in Germany, in their two articles on imaging science called "Introducing more physics into variational depth–from–defocus" [240], and "Physically inspired depth-from-defocus" [241] . From the former article: "Moreover, we advocate to replace the traditional Euler-Lagrange formalism by a multiplicative variant." From the latter article: "For  the  minimisation  of  our  energy  functional, we show the advantages of a multiplicative Euler-Lagrange formalism ... Our work is an example how one can benefit from physically refined modelling in conjunction with multiplicative calculi. It is our hope that both concepts will receive more popularity in future computer vision models."

The geometric calculus was used in two articles on medical-imaging science by Kiyoko Tateishi (Saint Marianna University School of Medicine in Japan and The University of Tokushima in Japan), Yusaku Yamaguchi (Shikoku Medical Center for Children and Adults in Japan), Omar M. A. Al-Ola (Tanta University in Egypt), Takeshi Kojima (The University of Tokushima in Japan), and Tetsuya Yoshinaga (The University of Tokushima in Japan). [271, 272] The articles are entitled "Continuous analog of multiplicative algebraic reconstruction technique for computed tomography" and "Noise reduction in computed tomography using a multiplicative continuous-time image reconstruction method". Both articles were presented at the SPIE conference Medical Imaging 2016: Physics of Medical Imaging in February/March of 2016. (SPIE, an affiliate of the American Institute of Physics, is an international society for optical engineering with more than 18,000 members.)

The geometric calculus was used in an article about radiation therapy by T. Yoshinaga, Y. Tanaka, K. Fujimoto (all from the Institute of Health Biosciences at The University of Tokushima in Japan). [289] From the article: "In this paper, we propose an iterative method as discretization of the differential equation using the geometric multiplicative calculus, and show the effectiveness of our method. ... We derived the iterative method by using the first-order approximation based on the geometric multiplicative calculus applied to the differential equation."

The geometric calculus was used by Luc Florack (Eindhoven University of Technology in the Netherlands) in his presentation "Neuro and cardio imaging" at the 2011 BIRS Workshop (Banff International Research Station for Mathematical Innovation and Discovery). [195]

Application of the geometric calculus to image analysis is discussed in the article "Direction-controlled DTI [Diffusion Tensor Imaging] interpolation" by Luc Florack, Tom Dela Haije, and Andrea Fuster, all from Eindhoven University of Technology in the Netherlands. [231] From that article: "The methodology ... exploits [geometric] calculus to implement positivity preserving "linear" operations."

Non-Newtonian calculus was used by Xiaohong Gong, Yali Zhou, Hao Zhou, and Yinfei Zheng (all from Zhejiang University in Hangzhou, China) in an article on ultrasound imaging. [211] From that article: "The proposed technique combines a new multiplicative gradient operator of non-Newtonian type with the traditional Canny operator to generate the initial edge map, ....  Thus, the proposed method is very suitable for fast and accurate edge detection of medical ultrasound images."

Non-Newtonian calculus was used in the study of contour detection in images with multiplicative noise by Marco Mora, Fernando Córdova-Lepe, and Rodrigo Del-Valle (all of Universidad Católica del Maule in Chile). [99] 

Non-Newtonian calculus was used in the article "A multiplicative gradient-based anisotropic diffusion approach for speckle noise removal" by Romulus Terebes (Technical University of Cluj-Napoca [TUCN], Romania), Monica Borda (TUCN, Romania), Christian Germain (IMS Laboratory, Bordeaux, France), Raul Malutan (TUCN, Romania), and Ioana Iles (TUCN, Romania). [268] From that article: "Reference [..] introduces a novel gradient operator better suited for edge detection in ultrasound or SAR imaging [SAR: synthetic aperture radar]. This new operator, the multiplicative gradient, ... is developed using non-Newtonian calculus."

A course on non-Newtonian calculus was conducted in the summer-term of 2012 by Joachim Weickert (Gottfried Wilhelm Leibniz Prize winner), Laurent Hoeltgen, and other faculty from the Mathematical Image Analysis Group of Saarland University in Germany. Among the topics covered were applications of non-Newtonian calculus to digital image processing, rates of return, and other growth processes. [106]

The geometric calculus was used in a course given by Joachim Weickert (Gottfried Wilhelm Leibniz Prize winner) at Saarland University in Germany in the summer-term of 2012: "Differential Equations in Image Processing and Computer Vision, CS 101". [173]

In their article "A non-Newtonian examination of the theory of exogenous economic growth", Diana Andrada Filip (Babes-Bolyai University of Cluj-Napoca, Romania) and Cyrille Piatecki (LEO, Orléans University, France) assert: "In this paper, we have tried to present how a non-Newtonian calculus could be applied to repostulate and analyse the neoclassical [Solow-Swan] exogenous growth model [in economics]. ... In fact, one must acknowledge that it’s only under the effort of Grossman & Katz (1972) ... that such a non-Newtonian calculus emerged to give a natural answer to many growth phenomena. ... We must underscore that to discover that there was a non-Newtonian way to look to differential equations has been a great surprise for us. It opens the question to know if there are major fields of economic analysis which can be profoundly re-thought in the light of this discovery." [82, 34, 121]

Applications of non-Newtonian calculus to economics and statistics are discussed in the article "An overview on non-Newtonian calculus and its potential applications to economics" by Diana Andrada Filip (Babes-Bolyai University of Cluj-Napoca, Romania) and Cyrille Piatecki (LEO, Orléans University, France). [181] From that article: "In this paper, after a brief presentation of [the geometric] calculus, we try to show how it could be used to re-explore from another perspective classical economic theory, more particularly economic growth and the maximum-likelihood method from statistics."

A strong argument for "a non-Newtonian economic analysis" based on a new multiplicative accounting system and non-Newtonian calculus is presented in the article "In defense of a non-Newtonian economic analysis through an accounting paradigm" by Diana Andrada Filip (Babes-Bolyai University of Cluj-Napoca in Romania) and Cyrille Piatecki (Orléans University in France). [216] In that article they state: In that article they state: "The double-entry bookkeeping promoted by Luca Pacioli in the fifteenth century could be considered a strong argument in behalf of the multiplicative calculus which can be developed from the Grossman and Katz non-Newtonian calculus concept provided that one goes from an additive bookkeeping system to a multiplicative one. ... If for instance a non-Newtonian analysis of economic growth should be implemented, ... the involved [accounting system] must also be non-Newtonian, that is non-additive."

Discussions concerning a new multiplicative accounting system and the advantages of using the geometric calculus in economic analysis are included in an article by Diana Andrada Filip (Babes-Bolyai University of Cluj-Napoca in Romania) and Cyrille Piatecki (Orléans University in France). [149] 

The geometric calculus was used by Hasan Özyapıcı (Eastern Mediterranean University in Cyprus), İlhan Dalcı (Eastern Mediterranean University in Cyprus), and Ali Özyapıcı (Cyprus International University) in their article "Integrating accounting and multiplicative calculus: an effective estimation of learning curve". [290] From the Abstract: "The results of this study are also expected to help researchers, practitioners, economists, business managers, and cost and managerial accountants to understand how to construct a multiplicative based learning curve to improve such decisions as pricing, profit planning, capacity management, and budgeting." (The expression "multiplicative calculus" refers here to the geometric calculus.)

The non-Newtonian approach to accounting [82, 121, 149, 181, 216]  was advocated by Amelia Correa (St. Andrews College in India) and Romar Correa (University of Mumbai in India) in their article "Accounting for Financialization: Stock-Flow-Consistent Political Economy". [259]

Many applications of non-Newtonian calculus have been made by Agamirza E. Bashirov, Mustafa Riza, and Yucel Tandogdu (all of Eastern Mediterranean University in North Cyprus); Emine Misirli Kurpinar and Yusuf Gurefe (both of Ege University in Turkey); and Ali Ozyapici (Lefke European University in Turkey). Their work has application to differential equations, calculus of variations, finite-difference methods, approximation theory, multivariable calculus, complex analysis, actuarial science, finance, economics, biology, and demographics. [2, 24, 27, 33, 84, 87, 94, 95, 123, 140, 145, 157, 200] The article [2] was "submitted by Steven G. Krantz" and published in 2008 by the Journal of Mathematical Analysis and Applications. From [94]: "This work is aimed to show that various problems from different fields can be modeled more efficiently using [geometric] calculus, in place of Newtonian calculus. ... In this study it becomes evident that the [geometric] calculus methodology has some advantages over [classical] calculus in modeling some processes in areas such as actuarial science, finance, economics, biology, demographics, etc."

Non-Newtonian calculus was used by Ugur Kadak (Gazi University in Turkey) and Muharrem Ozluk (Batman University in Turkey) in their article "Generalized Runge-Kutta method with respect to non-Newtonian calculus". [210]

The geometric calculus was used in the article "The Runge-Kutta Method in geometric multiplicative calculus" by Mustafa Riza and Hatice Aktore (both from Eastern Mediterranean University in North Cyprus). The article includes applications to selected well-known topics in biology, physics, and mathematics, including the Baranyi model for bacterial growth and the Rössler attractor problem. [169]

The bigeometric calculus was used in the article "Bigeometric calculus - a modelling tool" by Mustafa Riza (Eastern Mediterranean University in North Cyprus) and Bugce Eminaga (Girne American University in Cyprus). The article includes a new mathematical model for studying tumor therapy with oncolytic virus. In the article, Professors Riza and Eminaga state: "The results show that the Bigeometric Runge-Kutta method is superior to the ordinary Runge-Kutta method for a certain family of problems." [178]

The geometric and bigeometric calculi were used in the article "Bigeometric Calculus and Runge Kutta Method" by Mustafa Riza (Eastern Mediterranean University in North Cyprus) and Bugce Eminaga (Girne American University in Cyprus). The article (a revision of article [178]), includes new mathematical models (of the growth of cells, genes, bacteria, and viruses) for studying such things as tumor therapy with oncolytic virus and cell-cycle-specific cancer-chemotherapy. In the article, Professors Riza and Eminaga state: "Bigeometric Runge-Kutta method is, at least for a particular set of initial value problems, superior with respect to accuracy and computation-time to the ordinary Runge-Kutta method." [215] 

The bigeometric calculus was used in "New solution method for electrical systems represented by ordinary differential equation", an article recommended by Piero Malcovati (University of Pavia in Italy) and written by Bülent Bilgehan (Girne American University in Cyprus), Buğçe Eminağa (Girne American University in Cyprus), and Mustafa Riza (Eastern Mediterranean University in Cyprus). [250] From the article's abstract: "The aim is to examine the existing models from bigeometric calculus point of view to obtain accuracy on the results. This work is an application of bigeometric Runge–Kutta (BRK4) method ... . This type of work arises from applications where the systems are defined by ordinary differential equations such as noise, filter, audio, chaotic circuits, etc. ... The improvement in this work is obtained by introducing bigeometric calculus in the process of seeking a solution to differential equations. ... The applicability is tested against the classical method called Runge–Kutta (RK4). Simulation results confirm the application of BRK4 method in electrical circuit analysis. The new method also provides better results for all types of input signals, i.e., linear, nonlinear, constant or Gaussian." From the article: "There has been great research in geometric (multiplicative) and bigeometric calculus within the recent years."

The geometric calculus and the bigeometric calculus were used by Hatice Aktore (Eastern Mediterranean University in North Cyprus) in an article on multiplicative Runge-Kutta methods. [133]

A lecture about a complex multiplicative Runge-Kutta method was presented by Hatice Aktore and Mustafa Riza (both of Eastern Mediterranean University in North Cyprus) at the 2012 International Conference on Applied Analysis and Algebra at Yıldız Technical University in Istanbul, Turkey. [148]

The geometric calculus was used by Yusuf Gurefe (Usak University in Turkey) and Emine Misirli (Ege University in Turkey) in their lecture "New Runge-Kutta methods for numerical solutions of multiplicative initial value problems" at the 2014 International Conference on Recent Advances in Pure and Applied Mathematics at Antalta, Turkey. In the abstract to that lecture they state: "[Geometric] calculus, which is defined in a manner analogous to the concepts in the classical calculus, has become important in recent years." [207]

The geometric calculus was used in the article "New 2-point implicit block multistep method for multiplicative initial value problems" by Yusuf Gurefe (Usak University in Turkey) and Emine Misirli (Ege University in Turkey). [291]

The geometric calculus is among the topics in the mathematics textbook Mathematical Analysis: Fundamentals by Agamirza Bashirov. [179] Included is application of the geometric calculus to differential equations. From the Abstract to Chapter 11: "An interesting feature of this chapter is an introduction to multiplicative calculus, which is an alternative to the [classical] calculus of Newton and Leibnitz. By use of methods of multiplicative calculus it is proved that an infinitely-many times differentiable function may not be analytic." (The expression "multiplicative calculus" refers here to the geometric calculus.)

Application of the bigeometric derivative to the theory of elasticity in economics was made by Fernando Córdova-Lepe (Universidad Católica del Maule in Chile) . (He referred to the bigeometric derivative as the "multiplicative derivative.") [3, 4, 105] Elasticity is also discussed in Non-Newtonian Calculus [15], Bigeometric Calculus: A System with a Scale-Free Derivative [10], and The First Systems of Weighted Differential and Integral Calculus [9].
 
The geometric calculus was used by Ali Uzer (Fatih University in Turkey) in an article on wave theory in physics: "Improvement of the diffraction coefficient of GTD  by using multiplicative [the geometric] calculus". [185] In the abstract to the article he states: "The given approximation techniques [based on the geometric calculus] ... can be extended to other high frequency techniques in electromagnetics."

Non-Newtonian calculus was used by Ali Uzer (Fatih University in Turkey) to develop a multiplicative type of calculus for complex-valued functions of a complex variable, which he applied to wave theory in physics. [78, 158]

The geometric calculus was used in the article "A q-analogue of the multiplicative calculus: q-multiplicative calculus" by Gokhan Yener and Ibrahim Emiroglu (both of Yildiz Technical University in Turkey). [267] From that article: "According to our research, we strongly believe that q-multiplicative calculus, which is the extended version of multiplicative calculus [i.e., the geometric calculus], can show the way for further research fields with new definitions theories and applications."

The geometric calculus was used by Paolo Perrone (University of Milan in Italy and Max Planck Institute in Germany/Italy) in his article "A gauge theoretic approach to quantum physics". [194]

The geometric calculus was used by Ali Ozyapici and Bulent Bilgehan (both of Girne American University in Cyprus/Turkey) in their lecture "Applications of multiplicative calculus to exponential signal processing" at the 2013 Algerian Turkish International Days on Mathematics. [162] In the abstract to that lecture they state: "In this work, we study the representation of signals based on multiplicative calculus. ... Consequently, we demonstrate that multiplicative calculus representation results in a highly efficient model for the representation of exponential type signals." (The expression "multiplicative calculus" refers here to the geometric calculus.)

The geometric calculus was used by Bulent Bilgehan (Girne American University in Cyprus/Turkey) in his lecture "Finite product representation via multiplicative calculus in signal processing" at the First International Symposium on Engineering, Artificial Intelligence & Applications at Girne American University in Cyprus/Turkey. (The expression "multiplicative calculus" refers here to the geometric calculus.) [171]

The geometric calculus was used by Bulent Bilgehan (Girne American University in Cyprus/Turkey) in his article about signal processing called "Efficient approximation for linear and non-linear signal representation". [222] The Abstract: "This paper focuses on optimum representation for both linear and non-linear type signals which have a wide range of applications in the analysis and processing of real-world signals, that is, noise, filtering, audio, image etc. Accurate representation of signals, usually is not an easy process. The optimum representation is achieved by introducing exponential bases within multiplicative calculus which enables direct processing to reveal the unknown fitting parameters. Simulation tests confirm that the newly introduced models produce accurate results while using substantially less computation and provide support for applying the new model in the field of parametric linear, non-linear signal representation for processing." (The expression "multiplicative calculus" refers here to the geometric calculus.)

The geometric calculus was used in the article "Finite product representation via multiplicative calculus and its applications to exponential signal processing" by Ali Ozyapici (Cyprus International University in Cyprus/Turkey) and Bülent Bilgehan (Girne American University in Cyprus/Turkey). (The expression "multiplicative calculus" refers here to the geometric calculus.) [225] The Abstract: "In this paper, the multiplicative least square method is introduced and is applied to integrals for the finite product representation of the positive functions. Hence, many nonlinear functions can be represented by well-behaved exponential functions. Product representation produces an accurate representation of signals, especially where exponentials occur. Some real applications of nonlinear exponential signals will be selected to demonstrate the applicability and efficiency of the proposed representation."

Non-Newtonian calculus was used in the 2016 doctoral dissertation of Michael Valenzuela at the University of Arizona in the United States. The dissertation is entitled "Machine learning, optimization, and anti-training with sacrificial data". (In computer science, machine learning is a branch of artificial intelligence.) [279] From the dissertation: "Grossman and Katz [Non-Newtonian Calculus] mention several alternative calculi including: geometric, anageometric, bigeometric, quadratic, anaquadratic, biquadratic, harmonic, anaharmonic, and biharmonic. ... Non-Newtonian calculus has been used to derive optimization algorithms that perform better than traditional Newton based methods for Expectation-Maximization algorithms. However, Non-Newtonian calculus goes beyond simply being useful for optimization, it is useful for the other half of learning: modeling. The second order approximation using geometric calculus may produce the Gaussian curve ... . The nth order approximation using bigeometric calculus produces an nth polynomial on a log-log plot. ... Non-Newtonian generalized Taylor expansions produce nth order models, which are rarely polynomials. ... Non-Newtonian models sometimes make sense to use. Non-Newtonian models follow from non- Newtonian calculi. ... Here are a few rules of thumb for non-Newtonian models. If a meta-model is primarily concerned with learning probabilities, non-parametric distributions, or anything else where the multiplication is the primary operation, then the geometric calculi may be of interest. If working in a domain where the squares are additive, as is common the case when estimating the variance of a sum of independent random variables, then the quadratic calculi may produce meaningful models."
An article concerning minimization methods based on the geometric and bigeometric calculi was written by Ali Ozyapici (Girne American University in Cyprus/Turkey), Mustafa Riza (Eastern Mediterranean University in North Cyprus), Bulent Bilgehan (Girne American University), and Agamirza E. Bashirov (Eastern Mediterranean University). [176] From the Abstract to the article: "Theory and applications of [geometric] and [bigeometric] calculi have been evolving rapidly over the recent years. As numerical minimization methods have a wide range of applications in science and engineering, the idea of the design of minimization methods based on [geometric] and [bigeometric] calculi is self-evident. In this paper, the well-known Newton minimization method for one and two variables is developed in the framework of [geometric] and [bigeometric] calculi. The efficiency of these proposed minimization methods is demonstrated by examples, ... . One of the striking results of the proposed method is that the rate of convergence and the range of initial values are considerably larger compared to the original method."

Application of non-Newtonian calculus to information technology was made by S. L. Blyumin of the Lipetsk State Technical University in Russia. [23]

An article concerning minimization methods based on the geometric and bigeometric calculi was written by Ali Ozyapici (Girne American University in Cyprus/Turkey), Mustafa Riza (Eastern Mediterranean University in North Cyprus), Bulent Bilgehan (Girne American University), and Agamirza E. Bashirov (Eastern Mediterranean University). [176] From the Abstract to the article: "Theory and applications of [geometric] and [bigeometric] calculi have been evolving rapidly over the recent years. As numerical minimization methods have a wide range of applications in science and engineering, the idea of the design of minimization methods based on [geometric] and [bigeometric] calculi is self-evident. In this paper, the well-known Newton minimization method for one and two variables is developed in the framework of [geometric] and [bigeometric] calculi. The efficiency of these proposed minimization methods is demonstrated by examples, ... . One of the striking results of the proposed method is that the rate of convergence and the range of initial values are considerably larger compared to the original method."

Non-Newtonian calculus was used in the article "On line and double integrals in the non-Newtonian sense" by Ahmet Faruk Çakmak (Yıldız Technical University in Turkey) and Feyzi Başar (Fatih University in Turkey). [201] From the Abstract: "This paper is devoted to line and double integrals in the sense of non-Newtonian calculus (*-calculus). Moreover, in the sense of *-calculus, the fundamental theorem of calculus for line integrals and double integrals are stated and proved, and some applications are presented."

Application of non-Newtonian calculus to functional analysis was made by Cengiz Türkmen and  Feyzi Basar, both of Fatih University in Turkey. Their work was presented at the First International Conference on Analysis and Applied Mathematics, whose purpose was "to bring together mathematicians working in the area of analysis and applied mathematics to share new trends of applications of math". [112]

The geometric calculus was used in the article "On multiplicative fractional calculus" by Thabet Abdeljawad (Prince Sultan University in Saudi Arabia). [251]  From that article: "In this work, we bring together [the geometric] calculus and fractional calculus." 

The geometric calculus was used in the article "On geometric fractional calculus" by Thabet Abdeljawad (Prince Sultan University in Saudi Arabia). [270]
 

Non-Newtonian calculus was used by Ahmet Faruk Cakmak (Yıldız Technical University in Turkey) and Feyzi Basar (Fatih University in Turkey) to yield "some new results on sequence spaces with respect to non-Newtonian calculus". [122] From that article: " ... Grossman and Katz (Non-Newtonian Calculus, Lee Press, Pigeon Cove, Massachusetts, 1972) introduced the non-Newtonian calculi, [including] the geometric, anageometric, and bigeometric calculi."

Non-Newtonian calculus was used in the article “Non-Newtonian fuzzy numbers” by Ugur Kadak (Gazi University in Turkey). [230]

The geometric calculus was used in the article “A new look at the classical sequence spaces by using multiplicative calculus” by Yusuf Gurefe (Usak University in Turkey), Ugur Kadak (Bozok University in Turkey), Emine Misirli (Ege University in Turkey), and Alia Kurdi (University Politehnica of Bucharest in Romania). (The expression "multiplicative calculus" refers here to the geometric calculus.) [232]

Non-Newtonian calculus was used in the article "On the classical paranormed sequence spaces and related duals over the non-Newtonian complex field" by Ugur Kadak (Bozok University in Turkey), Murat Kirişci (Istanbul University in Turkey), and Ahmet Faruk Cakmak (Yıldız Technical University in Turkey). [226]

Non-Newtonian calculus was used in the article "Cesaro summable sequence spaces over the non-Newtonian complex field" by Ugur Kadak (Bozok University in Turkey). [253]

Non-Newtonian calculus was used in an article by S. L. Blyumin (Lipetsk State Technical University in Russia) involving binary arithmetic-operations and functional equations. [180]

The geometric calculus was used in an article on marketing by David Godes (University of Maryland). [192]

Several of the books and articles by Jane Grossman, Michael Grossman, and Robert Katz contain discussion about applications of non-Newtonian calculus to subjects such as growth/decay analysis, analytic geometry, vectors, least-squares methods, centroids, complex numbers, sigmoidal functions, relativistic composition of speeds, measurement (physics), psychophysics, weighted calculus, meta-calculus, averages (of functions), and means (of two positive numbers). [15, 11, 9, 7, 10, 8, 12, 14, 34, 35]

Non-Newtonian calculus may have application to neuroscience. According to Roberto Sotero Diaz (Hotchkiss Brain Institute of the University of Calgary in Canada): "... I’m very interested in the application of non-Newtonian calculus to computational neuroscience, specifically for solving biophysical models of the generation of neuronal activity. The sigmoidal calculus, as introduced in ... Non-Newtonian Calculus has the potential to be a very useful approach to the problems I want to solve ... ." [15]

The non-Newtonian averages (of functions) were used to construct a family of means (of two positive numbers). [8, 14] Included among those means are some well-known ones such as the arithmetic mean, the geometric mean, the harmonic mean, the power means, the logarithmic mean, the identric mean, and the Stolarsky mean.  The family of means was used to yield simple proofs of some familiar inequalities. [14] Publications [8, 14] about that family are cited in articles [29-32, 118, 153].

Non-Newtonian calculus was used by James R. Meginniss (Claremont Graduate School and Harvey Mudd College) to create a theory of probability that is adapted to human behavior and decision making. [16] In that article he asserts: "The purpose of this paper is to present a new theory of probability that is adapted to human behavior and decision making. Two basic ideas will be used: that probabilities do not obey the laws of ordinary arithmetic and calculus, but instead are governed by the laws of one of the non-Newtonian calculi and its corresponding arithmetic; and that ... ."

Non-Newtonian calculus was used in the famous 2006 report "Stern Review on the Economics of Climate Change", according to a 2012 critique of that report (called "What is Wrong with Stern?") by former UK Cabinet Minister Peter Lilley and economist Richard Tol. "Stern Review on the Economics of Climate Change", which is over 700-pages long, was commissioned by the UK government, was written by a team led by Nicholas Stern (former Chief Economist at the World Bank), and has drawn worldwide attention. [116, 165]

The geometric calculus was used in an article on "statistics of acoustically induced bubble-nucleation events in in-vitro blood" by Jérôme Gateau, Nicolas Taccoen, Mickaël Tanter, and Jean-François Aubry (their affiliations: Institut Langevin; ESPCI ParisTech; CNRS UMR 7587, INSERM U979; Université Paris Diderot, Paris 7). [166]

Non-Newtonian calculus was used in the article "A new theoretical discrete growth distribution with verification for microbial counts in water" by James Englehardt (University of Miami), Jeff Swartout (US EPA Facilities, Cincinnati, OH, USA), and Chad Loewenstine (BLDG 27958-A, Quantico, VA, USA). [287]

Non-Newtonian calculus was used in the article "The discrete Weibull distribution: an alternative for correlated counts with confirmation for microbial counts in water" by James D. Englehardt (University of Miami) and Ruochen Li (Shenzhen, China). [85] 

The geometric calculus was used in the article "Distributions of autocorrelated first-order kinetic outcomes: illness severity" by James D. Englehardt (University of Miami). (The expression "product integral" used in the article refers to the geometric integral.) [227] From that article: "... [the geometric integral] is fundamental to developing the continuously multiplicative nature of the continuous first-order kinetic rate law, not otherwise obvious."

"Application of geometric calculus in numerical analysis and difference sequence spaces" is the title of an article by Khirod Boruah and Bipan Hazarika, both from Rajiv Gandhi University in India. Included in the article is discussion of the "advantages of geometric interpolation formulae over ordinary interpolation formulae". (In that article the word "geometric" refers to the geometric calculus, not to the subject of geometry.) From the article: "The main aim of this paper is ... and to obtain the Geometric Newton-Gregory interpolation formulae which are more useful than Newton-Gregory interpolation formulae." Also from the article: "In this paper, we have defined geometric difference sequence space and obtained the Geometric Newton-Gregory interpolation formulae. Our main aim is to bring up geometric calculus to the attention of researchers in the branch of numerical analysis and to demonstrate its usefulness. We think that geometric calculus may especially be useful as a mathematical tool for economics, management and finance." [276]

Geometric arithmetic [15] and geometric complex-numbers were used in the article "Generalized geometric difference sequence spaces and its duals" by Khirod Boruah (Rajiv Gandhi University in India), Bipan Hazarika (Rajiv Gandhi University in India), and Mikail Et (Firat University in Turkey). From that article: "Generally speaking multiplicative calculus is a methodology that allows one to have a different look at problems which can be investigated via calculus. In some cases, for example for growth related problems, the use of multiplicative calculus is advocated instead of a traditional Newtonian one." (The expression "multiplicative calculus" refers here to the geometric calculus.) [277]
The bigeometric calculus was used by William Campillay and Manuel Pinto (both of the Universidad de Santiago de Chile) in a lecture on bigeometric differential-equations at the VIII Congreso de Análisis Funcional y Ecuaciones de Evolución at Universidad de Santiago de Chile. [172]
The geometric integral is discussed in the article "Product integrals and sum integrals" by Raymond A. Guenther (University of Nebraska at Omaha). [236]

Weighted non-Newtonian calculus [9] was used by Ziyue Liu and Wensheng Guo (both of the University of Pennsylvania) in an article on spline smoothing. [119]

Weighted non-Newtonian calculus [9] was used by David Baqaee (Harvard University) in an article on an axiomatic foundation for intertemporal decision making. [86]

The First Systems of Weighted Differential and Integral Calculus [9] was used in the book Minimization of Climatic Vulnerabilities on Mini-hydro Power Plants: Fuzzy AHP, Fuzzy ANP Techniques and Neuro-Genetic Model Approach by Mrinmoy Majumder (National Institute of Technology Agartala in India). [274] From page 26: "The weight function was proposed by Grossman et al. (1980) [The First Systems of Weighted Differential and Integral Calculus ] which represents the weighted impact of a set of parameters based on their priority values and magnitude."  

Non-Newtonian calculus was used by Stanley Paul Palasek (Sonoran Science Academy in Tucson, Arizona) in a biology project on opioid peptide delivery at an Intel® International Science and Engineering Fair. [97]

Non-Newtonian calculus was used by J. I. King in a study on atmospheric temperature (optical measure theory and inverse transfer theory). [102] (This study is used in [89].)

A lecture about the bigeometric calculus was presented by Ahmet Faruk Çakmak at the 2011 International Conference on Applied Analysis and Algebra at Yıldız Technical University in Istanbul, Turkey. [107]

The geometric calculus was used by Gunnar Sparr (Lund Institute of Technology, in Sweden) in an article on computer vision. (The "multiplicative derivative" referred to in the article is the geometric derivative.) [155]

Non-Newtonian calculus, and related matters are topics in the article "Matrix transformations between certain sequence spaces over the non-Newtonian complex field" by Ugur Kadak and Hakan Efe (both of Gazi University in Turkey). [189]

Non-Newtonian calculus and related matters are topics in the article "The construction of Hilbert spaces over the non-Newtonian field" by Ugur Kadak and Hakan Efe (both of Gazi University in Turkey). [190]

The geometric calculus was used by Uğur Kadak (Gazi University in Turkey) and Yusuf Gurefe (Bozok University in Turkey) in their presentation at the 2012 Analysis and Applied Mathematics Seminar Series of Fatih University in Istanbul, Turkey. [117]

Application of non-Newtonian analysis to function spaces was made by Ahmet Faruk Cakmak (Yıldız Technical University in Turkey) and Feyzi Basar (Fatih University in Turkey) in their lecture at the 2012 conference The Algerian-Turkish International Days on Mathematics, at University of Badji Mokhtar at Annaba, in Algeria. [127]

Non-Newtonian analysis was used by Ahmet Faruk Cakmak (Yıldız Technical University in Turkey) and Feyzi Basar (Fatih University in Turkey) in their 2015 article “Some sequence spaces and matrix transformations in multiplicative sense”, and in their 2014 lecture with the same name at the Çankırı Karatekin University Mathematics Days, Çankırı, Turkey. [217, 204] In that article the authors state: "As an alternative to the classical calculus of Newton and Leibniz, [the geometric] calculus gives more convenient results in some specific problems."

Application of non-Newtonian analysis to "continuous and bounded functions over the field of non-Newtonian/geometric complex numbers" was made by Zafer Cakir (Gumushane University, Turkey). [137, 159]

Non-Newtonian calculus was one of the topics of discussion at the 2013 Algerian-Turkish International Days on Mathematics conference at Fatih University in Istanbul, Turkey. [138]

Non-Newtonian calculus is listed among the topics covered in the International Journal on Recent Trends in Life Science and Mathematics (IJLSM). [167]

Non-Newtonian calculus and related matters are used in the article "Certain sequence spaces over the non-Newtonian complex field" by Sebiha Tekin and Feyzi Basar, both of Fatih University in Turkey. [144]

The geometric integral is useful in stochastics. [22]

The geometric calculus is the topic of an article by Dick Stanley of the University of California at Berkeley. [19]

Various student projects regarding the geometric calculus are discussed in an article by Duff Campbell of the University of California at Berkeley. [193]

The geometric calculus was used in statistics and data analysis by Jarno van Roosmalen (Eindhoven University of Technology in the Netherlands) in his bachelor project on "multiplicative principal component analysis". [120]

Non-Newtonian calculus may have application in situations involving discontinuous phenomena. [35]

The geometric calculus was the subject of Christopher Olah's lecture at the Singularity Summit on 13 October 2012. [134] Singularity University's Singularity Summit is a conference on robotics, artificial intelligence, brain-computer interfacing, and other emerging technologies including genomics and regenerative medicine. [135]  Christopher Olah is a Thiel Fellow.

A graduate-seminar involving non-Newtonian calculus was conducted by Jared Burns at the University of Pittsburgh on 13 December 2012. [154]

The geometric and bigeometric calculi were used in a seminar (December of 2013) given by Ali Ozyapici and held by the Faculty of Engineering at Girne American University in Cyprus: "Applications of Multiplicative Calculi to Economical and Numerical Problems". [175]

Non-Newtonian calculus was used by Ugur Kadak (Gazi University in Turkey), Feyzi Basar (Fatih University in Turkey), and Hakan Efe (Gazi University in Turkey) in an article on sequence spaces. [160]

Non-Newtonian calculus was used by Ugur Kadak (Gazi University in Turkey), Feyzi Basar (Fatih University in Turkey), and Hakan Efe (Gazi University in Turkey) in their lecture “Construction of the duals of classical sets of sequences and related matrix transformations with non-Newtonian calculus” at the 2013 Algerian Turkish International Days on Mathematics. [186]

Non-Newtonian calculus and related matters are topics in the article "Determination of the Kothe-Toeplitz duals over the non-Newtonian complex field" by Ugur Kadak (Gazi University in Turkey). [183]

Non-Newtonian calculus was used by Ahmet Faruk Çakmak (Yıldız Technical University in Turkey) and Feyzi Basar (Fatih University in Turkey) in an article on function spaces. [161]

A lecture entitled "Difference sequence spaces and non-Newtonian calculus" was presented by Khiord Boruah (Rajiv Gandhi University in India) at the National Conference on Recent Trends of Mathematics and its Applications, in May of 2014. [188]

Several specific non-Newtonian calculi are discussed in an article by H. Vic Dannon (Gauge Institute in USA). [196]

Several specific non-Newtonian calculi are discussed in a book by H. Vic Dannon (Gauge Institute in USA). [197]

The geometric calculus was used by Tolgay Karanfiller (Cyprus International University) in his lecture "Numerical solution of non-linear equations via multiplicative calculus" at the 2014 International Conference on Recent Advances in Pure and Applied Mathematics at Antalta, Turkey. [209]

The mathematics department of Eastern Mediterranean University in North Cyprus has established a research group for the purpose of studying and applying the geometric calculus and the bigeometric calculus. [110]

The geometric calculus and some of its applications are the topics of the 2009 doctoral dissertation of Ali Ozyapici at Ege University in Turkey. The dissertation is entitled "Multiplicative calculus and its applications". [191]  

Non-Newtonian calculus and some of its applications are the topics of the 2011 doctoral dissertation of Ugur Kadak at Gazi University in Turkey. The dissertation is entitled "Non-Newtonian analysis and its applications". [187]

The geometric calculus and some of its applications are the topics of the 2113 doctoral dissertation of Yusuf Gurefe at Ege University in Turkey. The dissertation is entitled "Multiplicative differential equations and applications". [208]

Non-Newtonian analysis was used in the 2014 doctoral dissertation of Ahmet Faruk Cakmak at Yıldız Technical University in Turkey. The dissertation is entitled "Some new sequence spaces over a new field". [205]

Non-Newtonian calculus is a featured topic of the 2016 doctoral dissertation of Zakaria Adnan at Kwame Nkrumah University of Science and Technology in Ghana. The dissertation is entitled "An analysis of Runge-Kutta method in non-Newtonian calculus". [278]

Knowledge of the geometric calculus ("multiplicative calculus") is a requirement for the master's degree in computer-engineering at Inonu University (Malatya, Turkey). [136]

The geometric calculus ("multiplicative calculus") is included on the list "Proposed Topics for the Master's Degree" of the Institute of Mathematics of the Jagiellonian University in Poland. [174]

Non-Newtonian calculus was used in the 2005 master's thesis of Ali Ozyapici at Eastern Mediterranean University in North Cyprus. [220]

Non-Newtonian calculus was used in the 2011 master's thesis of Hatice Aktore at Eastern Mediterranean University in North Cyprus. [221]

The geometric calculus was used in the 2014 master's thesis of Jaafar Anwar H. Ameen at Eastern Mediterranean University in North Cyprus. [219]

The First Systems of Weighted Differential and Integral Calculus [9] is used by Riswan Efendi (Universiti Teknologi Malaysia), Zuhaimy Ismail (Universiti Teknologi Malaysia), Nor Haniza Sarmin (Universiti Teknologi Malaysia), and Mustafa Mat Deris (Universiti Tun Hussein Onn Malaysia) in their article "A reversal model of fuzzy time series in regional load forecasting". [245]

The geometric calculus was presented in the book Alternative Picture of the World, Volume 1 by Leonid G. Kreidik and George Shpenkov (both from the Institute of Mathematics & Physics at the University of Technology & Agriculture in Poland). From the book: "The multiplicative calculus allows us to see a great many facts which would be impossible to find by the classical additive calculus." (The expression "multiplicative calculus" refers here to the geometric calculus.) [285]

The geometric calculus was presented in the article "Additive and multiplicative judgements of dialectical logic. Additive and multiplicative differentials and integrals of dialectical judgements" by Leonid G. Kreidik of the Dialectical Academy in Russia-Belarus. From the article: "The multiplicative calculus allows us to see a great many facts which would be impossible to find by the classical additive calculus." (The expression "multiplicative calculus" refers here to the geometric calculus.) [256]

Several applications of discrete “multiplicative” [i.e., geometric] calculus have been made by Mohammad Jahanshahi (Azad Islamic University of Karadj in Iran) and his colleagues. [150, 151, 152, 202, 286]

Geometric arithmetic [15] was used in an article about signal processing by Norman Zacharias (Leibniz Institute for Neurobiology, Germany), Cezary Sieluzycki (Leibniz Institute for Neurobiology, Germany), Wojciech Kordecki (University of Business in Wroczaw, Poland) , Reinhard Konig (Leibniz Institute for Neurobiology, Germany), and Peter Heil (Leibniz Institute for Neurobiology, Germany). From that article: "We therefore propose geometric, rather than arithmetic, averaging of the M100 component across subjects and suggest a novel and superior normalization procedure.  ... According to our knowledge, it has never been investigated whether ERFs and ERPs follow the additive or the multiplicative model, but this knowledge is crucial, ... ." [266]

Geometric arithmetic [15] was used in two articles about "multiplicative parameters-applications in economics and finance" by Helena Jasiulewicz (Wrocław University of Environmental and Life Sciences in Poland) and Wojciech Kordecki (The Witelon State University of Applied Sciences in Poland). Among the concepts discussed: geometric mean, geometric-mean investment strategy, geometric standard deviation, lognormal distributions, and the Pareto distribution (a power-law probability distribution). From those articles: "When is it better to use arithmetic (additive) parameters and when geometric (multiplicative) ones?" [264, 265]

Application of geometric arithmetic [15] to wavelet analysis was made in the article "Geometric wavelet approximations and differencing" by Abdourrahmane Mahamane Atto, Emmanuel Trouve, Jean Marie Nicolas (the former two from Polytech Annecy-Chambery in France, and the latter from Télécom ParisTech in France). From that article: "This paper introduces the concept of geometric wavelets defined from multiplicative algebras. ... In particular, when the acquisition system yields a multiplicative interaction model involving a non-constant signal, then geometric representation frameworks such as that presented in this paper are expected to be more relevant than additive frameworks." [212]

The geometric derivative is included in Mathematica's "Stack Exchange" website, which has links to articles [2], [88], and [19]. [198]

NOTE. Non-Newtonian calculus was recommended as a topic for the 21st-century college-mathematics-curriculum - at the 27th International Conference on Technology in Collegiate Mathematics (March of 2015). The conference is sponsored by Pearson PLC, the largest education company and the largest book publisher in the world. [224]

=================================================================================================================================================================
Citations
Contents

Home
Multiplicative Calculus
Brief History
Applications
Citations
Reviews
Comments
Quotations
References
Links/Reading
Appendix 1
Appendix 2
Appendix 3
Dedication

Non-Newtonian Calculus [15] is cited in the book The Rainbow of Mathematics: A History of the Mathematical Sciences by the eminent mathematics-historian Ivor Grattan-Guinness. [6]

Non-Newtonian calculus is cited in an article on atmospheric temperature by Robert G. Hohlfeld, Thomas W. Drueding, and John F. Ebersole (all from U.S. Air Force Geophysics Laboratory, Atmospheric Sciences Division) . [89]

Non-Newtonian Calculus [15] is cited in an article on means by Jane Tang. [20] 

The geometric calculus is cited in a book on the phenomena of growth and structure-building by Manfred Peschel and Werner Mende (both of the German Academy of Sciences Berlin). [25]
The geometric calculus is the topic of an article and a seminar by Michael Coco of Lynchburg University. [115]
Non-Newtonian calculus is cited in a book on the energy crisis by R. Gagliardi. [26]
 
Non-Newtonian Calculus is cited in a 2009 doctoral dissertation on nonlinear dynamical systems by David Malkin at University College London. [36]

The geometric calculus is cited by Daniel Karrasch in his 2012 doctoral dissertation "Hyperbolicity and invariant manifolds for finite time processes" at the Technical University of Dresden in Germany. [141]

The geometric calculus is cited in the article "Investigation of the solutions of the Cauchy problem and boundary-value problems for the ordinary differential equations with continuously changing order of the derivative" by N.A. Aliyev and R.G. Ahmadov (both from Baku State University in Azerbaijan). [283]

The book Averages: A New Approach [8] is cited by Christoph von Hagke in his 2012 doctoral dissertation "Coupling between climate and tectonics?" at Freie Universität Berlin. [223]

Non-Newtonian calculus is cited in the e-book Economic Statistics. [91]

Non-Newtonian differentiation was the topic of a lecture by Karol Kosar and Ivan Kupka at a student conference at Comenius University in Slovakia. [93]

Non-Newtonian calculus is cited in the article "Topological generalization of Cauchy's mean value theorem" by Ivan Kupka (Comenius University in Slovakia). [269]

The First Nonlinear System of Differential And Integral Calculus [11] is cited in the article "L'Hopital's rule and Taylor's Theorem for product calculus" by Alex Twist and Michael Spivey. [103] 

Non-Newtonian Calculus is cited in an article on petroleum engineering by Raymond W. K. Tang and William E. Brigham (both of Stanford University). [37]

The First Nonlinear System of Differential and Integral Calculus was cited in a lecture presented by Bruno Ćurko at the 2011 annual international symposium "Days of Frane Petric - From Petric to Boskovic" at Cres, Croatia. [109] 

Non-Newtonian calculus and Robert Katz are cited in a book on popular-culture by Paul Dickson. [28]


The geometric calculus ("multiplicative calculus") is cited by Orhan Tug (Ishik University in Iraq) and Feyzi Basar (Fatih University in Turkey) in their article "On the spaces of Norlund null and Norlund convergent sequences". [281]

Non-Newtonian calculus is cited in Science Education International: The ICASE Journal: "In mathematics, limits and diversity can be seen in the difference between the Arabic numbers and Roman numerals, Euclidean geometry and non-Euclidean geometry, Newtonian calculus and non-Newtonian calculus, and the existence of multiple ways to solve a mathematical problem." [38]

Non-Newtonian Calculus is cited in Gordon Mackay's book Comparative Metamathematics. (The eighteen previous editions of Comparative Metamathematics are d The True Nature of Mathematics.) [139]

Non-Newtonian Calculus is cited in the journal Search. [77]

Non-Newtonian Calculus is cited in the journal Science Weekly. [64]

Non-Newtonian Calculus is cited in the journal Annals of Science. [66]

Non-Newtonian Calculus is cited in the journal Science Progress. [67]

Non-Newtonian Calculus is cited in the journal Allgemeines Statistisches Archiv. [69]

Non-Newtonian Calculus is cited in the journal Il Nuovo Cimento della Societa Italiana di Fisica: A. [70]

The First Systems of Weighted Differential and Integral Calculus [9] is cited in the e-book Alimony Defense: A Complete Guide, edited by Premakh. [275] From that e-book: "The concept of weighted average can be extended to functions. Weighted averages of functions play an important role in the systems of weighted differential and integral calculus [in the classical case and in the non-Newtonian case]". 

The First Systems of Weighted Differential and Integral Calculus [9] is cited by the authors indicated below in their article on the global burden of cholera: Mohammad Ali, Anna Lena Lopez, Young Ae You, Young Eun Kim, Binod Sah, Brian Maskery, and John Clemens (all of the United Nations' International Vaccine Institute, Snu Research Park, San 4-8 Nakseongdae-dong Gwanak-gu, Seoul, Korea, 151 - 919). [98]

The First Systems of Weighted Differential and Integral Calculus [9] is cited by the authors indicated below in their article on thermochemistry of ammonium based ionic liquids: Sergey P. Verevkin and Vladimir N. Emer'yanenko (both of the University of Rostock, in Germany), Ingo Krossing (University of Freiburg, in Germany), and Roland Kalb (Proionic Production of Ionic Substances GmbH, in Graz, Austria). [108]

The First Systems of Weighted Differential and Integral Calculus [9]  is cited by P. Arun Raj Kumar and S. Selvakumar (both of the National Institute of Technology, Tiruchirappalli, in India) in their article "Detection of distributed denial of service attacks using an ensemble of adaptive and hybrid neuro-fuzzy systems". [147]      

The First Systems of Weighted Differential and Integral Calculus [9] is cited by Riswan Efendi and Zuhaimy Ismail (both of Universiti Teknologi Malaysia) together with Mustafa Mat Deris (Universiti Tun Husein Onn Malaysia) in their article "Improved weight fuzzy time series as used in the exchange rates forecasting of US dollar to ringgit Malaysia". [142]  

The First Systems of Weighted Differential and Integral Calculus [9] is cited by Riswan Efendi and Zuhaimy Ismail (both of Universiti Teknologi Malaysia) together with Mustafa Mat Deris (Universiti Tun Husein Onn Malaysia) in their article "A new linguistic out-sample approach of fuzzy time series for daily forecasting of Malaysian electricity load demand". [246]   

The First Systems of Weighted Differential and Integral Calculus [9] is cited by ZHENG Xu and LI Jian-Zhong (School of Computer Science and Technology, Harbin Institute of Technology, Harbin, China) in their work on wireless sensor networks in computer science. [125] 

The First Systems of Weighted Differential and Integral Calculus [9] is cited by Hong-Kyu Kim (Korea University, Seoul), Mirim Lee (Korea University, Seoul), Kwang-Ryeol Lee (Korea Institute of Science and Technology, Seoul), and Jae-Chul Lee (Korea University, Seoul) in their article "How can a minor element added to a binary amorphous alloy simultaneously improve the plasticity and glass-forming ability?". [248]    

The First Systems of Weighted Differential and Integral Calculus [9] is cited by Jie Zhang, Li Li, Luying Peng, Yingxian Sun, Jue Li (the first four from Tongji University School of Medicine in Shanghai, China; and the latter from The First Hospital of China Medical University, Shenyang, China) in their article "An Efficient Weighted Graph Strategy to Identify Differentiation Associated Genes in Embryonic Stem Cells". [156]  

The First Systems of Weighted Differential and Integral Calculus [9] is cited by Christoffel Wilhelmus Janse Rensburg (North-West University at Potchefstroom, in South Africa) in his master-of-science dissertation (in computer science and information systems) "The relationship between process maturity models and the use and effectiveness of systems development methodologies". [247]

The First Systems of Weighted Differential and Integral Calculus [9] is cited in the article "Ideality equation" by Alex Lyubomirskiy (GEN3 Partners). [273] 
The First Systems of Weighted Differential and Integral Calculus [9] is cited in the article "Service ratio-optimal, content coherence-aware data push systems" by Christos Liaskos (Foundation of Research and Technology in Hellas, Crete, Greece) and Ageliki Tsioliaridou (Democritus University of Thrace, Xanthi, Greece). [280]

The First Systems of Weighted Differential and Integral Calculus [9] is cited in the article "Developing seismic vulnerability curves for typical Iranian buildings" by Mehdi Sadeghi1 (Islamic Azad University in Iran), Mohsen Ghafory-Ashtiany (International Institute of Earthquake Engineering and Seismology in Iran), and Naghmeh Pakdel-Lahiji (Islamic Azad University in Iran). [284]

The First Systems of Weighted Differential and Integral Calculus [9] is cited in the journal Praxis der Mathematik. [79]

Meta-Calculus: Differential and Integral [7] is cited in the journal Indian Journal of Theoretical Physics. [80]

 Each of the following two books is cited in the journal Publicationes Mathematicae. [56]
   1) Non-Newtonian Calculus: Volume 19, page 351, 1972.
   2) Bigeometric Calculus: A System with a Scale-Free Derivative: Volume 32, page 282, 1985.

Each of the following two books is cited in the journal Acta Scientiarum Mathematicarum. [60] 
   1) Non-Newtonian Calculus: Volume 33, page 361, 1972.
   2) The First Nonlinear System of Differential and Integral Calculus: Volumes 42-43, page 225, 1980. 

Each of the following six books is cited in the journal Industrial Mathematics. [61]
   1) Non-Newtonian Calculus: Volumes 43-45, page 91, 1994 .
   2) The First Nonlinear System of Differential and Integral Calculus: Volumes 28-30, page 143, 1978.
   3) The First Systems of Weighted Differential and Integral Calculus: Volumes 31-33, page 66, 1981.
   4) Meta-Calculus: Differential and Integral: Volumes 31-33, page 83, 1981.
   5) Bigeometric Calculus: A System with a Scale-Free Derivative: Volumes 33-34, page 91, 1983.
   6) Averages: A New Approach: Volumes 33-34, page 91, 1983.

Each of the following two books is cited in the journal Economic Books: Current Selections. [81]
   1) The First Systems of Weighted Differential and Integral Calculus: Volume 9, page 29, 1982.
   2) Meta-Calculus: Differential and Integral: Volume 9, page 29, 1982. 

Non-Newtonian calculus is cited in the journal Ciência e cultura. [39]

Non-Newtonian calculus is cited in the journal American Statistical Association: 1997 Proceedings of the Section on Bayesian Statistical Science. [40]

Non-Newtonian Calculus is cited in the Australian Journal of Statistics. [73]

Non-Newtonian Calculus is cited in the journal Physique au Canada. [83]

Non-Newtonian Calculus is cited in the journal Synthese. [74]

Non-Newtonian Calculus is cited in the journal Mathematical Education. [75]

Non-Newtonian Calculus is cited in the the journal Institute of Mathematical Statistics Bulletin. [76]



=================================================================================================================================================
Reviews
Contents

Home
Multiplicative Calculus
Brief History
Applications
Citations
Reviews
Comments
Quotations
References
Links/Reading
Appendix 1
Appendix 2
Appendix 3
Dedication



Non-Newtonian Calculus was reviewed by Ivor Grattan-Guinness in the journal Middlesex Math Notes . [101]                                                                                                                                                                               
Excerpt: "There is enough here [in Non-Newtonian Calculus] to indicate that non-Newtonian calculi ... have considerable potential as alternative approaches to traditional problems. This very original piece of mathematics will surely expose a number of missed opportunities in the history of the subject."
Each of the following six books was reviewed in the journal Internationale Mathematische Nachrichten. [53] 
  1) Non-Newtonian Calculus [15]: Number 105, 1972.
       Excerpt: "The possibilities opened up by the [non-Newtonian] calculi seem to be immense."
  2) The First Nonlinear System of Differential and Integral Calculus [11]: Volumes 35-36, page 42, 1981.
  3) The First Systems of Weighted Differential and Integral Calculus [9]: Volumes 35-36, page 40, 1981.
  4) Meta-Calculus: Differential and Integral [7]: Volumes 35-36, page 140, 1981.
  5) Bigeometric Calculus: A System with a Scale-Free Derivative [10]: Volumes 37-38, page 266, 1983.
  6) Averages: A New Approach [8]: Volumes 37-38, page 266, 1983.
Non-Newtonian Calculus was reviewed by David Pearce MacAdam in the Journal of the Optical Society of America. [100] 
Excerpt: "This [Non-Newtonian Calculus] is an exciting little book. ... The greatest value of these non-Newtonian calculi may prove to be their ability to yield simpler physical laws than the Newtonian calculus. Throughout, this book exhibits a clarity of vision characteristic of important mathematical creations. ... The authors have written this book for engineers and scientists, as well as for mathematicians. ... The writing is clear, concise, and very readable. No more than a working knowledge of [classical] calculus is assumed."

Each of the following five books was reviewed by Ralph P. Boas, Jr. in the journal Mathematical Reviews. [47] 
  1) The First Nonlinear System of Differential and Integral Calculus [11]: Mathematical Reviews, 1980.
  2) The First Systems of Weighted Differential and Integral Calculus [9]: Mathematical Reviews, 1981.
  3) Meta-Calculus: Differential and Integral [7]: Mathematical Reviews, 1982.
  4) Bigeometric Calculus: A System with a Scale-Free Derivative [10]: Mathematical Reviews, 1984.
       Excerpt: "It seems plausible that people who need to study functions from this point of view might well be able to formulate problems more clearly by using [bigeometric] calculus instead of [classical] calculus."
  5) Averages: A New Approach [8]: Mathematical Reviews, 1984.

Non-Newtonian Calculus [15] was reviewed in the journal Mathematical Reviews in 1978. [47]

Non-Newtonian Calculus was reviewed in the magazine Choice. [41]

Non-Newtonian Calculus was reviewed in the journal American Mathematical Monthly. [48]

The First Nonlinear System of Differential And Integral Calculus [11], a book about the geometric calculus, was reviewed in the journal American Mathematical Monthly. [52]

Bigeometric Calculus: A System with a Scale-Free Derivative [10] was reviewed in Mathematics Magazine. [49]

Bigeometric Calculus: A System with a Scale-Free Derivative was reviewed in the journal The Mathematics Student. [58]

The article "An introduction to non-Newtonian calculus" [12] was reviewed by K. Strubecker in the journal  Zentralblatt Math (Zbl 0418.26008) [43].

The article "A new approach to means of two positive numbers" [14] was reviewed in Zentralblatt Math (Zbl 0586.26014) [43].

Each of the following three books was reviewed by K. Strubecker in Zentralblatt Math [43].
   1) Non-Newtonian Calculus [15]: Zbl 0228.26002.
   2) The First Systems of Weighted Differential and Integral Calculus [9]: Zbl 0443.26005.
   3) Meta-Calculus: Differential and Integral [7]: Zbl 0493.26001.

The article "A new approach to means of two positive numbers" [14] was reviewed in the journal ZDM (1986c.10787) [50].

Each of the following five books was reviewed in ZDM [50].
   1) Non-Newtonian Calculus[15]: 1982a.00259.
   2) The First Nonlinear System of Differential and Integral Calculus [11]: 1982a.00243.
   3) The First Systems of Weighted Differential and Integral Calculus [9]: 1982a.00248.
   4) Bigeometric Calculus: A System with a Scale-Free Derivative [10]: 19861.06868.
   5) Averages: A New Approach [8]: 19861.06873.

Each of the following six books was reviewed in the journal Scientific Annals of Alexandru Ioan Cuza University of Iaşi: Mathematics Section. [55]
   1) Non-Newtonian Calculus: Volumes 17-18, 1972.
   2) The First Nonlinear System of Differential and Integral Calculus: Volumes 26-27, 1980.
   3) The First Systems of Weighted Differential and Integral Calculus: Volumes 27-28, 1981.
   4) Meta-Calculus: Differential and Integral: Volumes 28-29, 1982.
   5) Bigeometric Calculus: A System with a Scale-Free Derivative: Volumes 29-30, 1983.
   6) Averages: A New Approach: Volumes 29-30, 1983. 

Each of the following three books was reviewed in the journal Nieuw Tijdschrift Voor Wiskunde. [57]
   1) The First Nonlinear System of Differential And Integral Calculus: Volume 68, page 104, 1981.
   2) The First Systems of Weighted Differential and Integral Calculus: Volumes 69-70, page 235, 1982.
   3) Meta-Calculus: Differential and Integral: Volumes 69-70, page 236, 1982.

Each of the following two books was reviewed by Leo Barsotti in the journal Boletim da Sociedade Paranaense de Matemática. [54]
  1) The First Nonlinear System of Differential and Integral Calculus: Volume 2, page 32, 1981.
  2) The First Systems of Weighted Differential and Integral Calculus: Volume 2, pages 32-33, 1981.

Each of the following three books was reviewed in the journal L'Enseignement Mathématique. [59]
  1) The First Nonlinear System of Differential and Integral Calculus: page 52, 1980.
  2) Bigeometric Calculus: A System with a Scale-Free Derivative: page 83, 1982.
  3) Averages: A New Approach: page 83, 1982. 

Each of the following two books was reviewed by P. Wilker in the journal Revue de mathématique élémentaires. [92]
  1) The First Nonlinear System of Differential and Integral  Calculus, Volumes 37-40.
  2) The First Systems of Weighted Differential and Integral Calculus: Volumes 37-40.

Non-Newtonian Calculus was reviewed by Otakar Zich in the journal Kybernetika. [45]

Non-Newtonian Calculus was reviewed in the journal Wissenschaftliche Zeitschrift: Mathematisch-Naturwissenschaftliche Reihe. [51]

Non-Newtonian Calculus was reviewed by M. Dutta in the Indian Journal of History of Science. [42] 

Non-Newtonian Calculus was reviewed by Karel Berka in the journal Theory and Decision. [44]

Non-Newtonian Calculus was reviewed by David Preiss in  the journal Aplikace Matematiky. [46]

Non-Newtonian Calculus was reviewed in the journal Physikalische Blätter. [62]

Non-Newtonian Calculus was reviewed in the journal "Scientia"; Rivista di Scienza. [63]

Non-Newtonian Calculus was reviewed in the journal Philosophia mathematica. [65]

Non-Newtonian Calculus was reviewed in the journal Revue du CETHEDEC. [68]

Non-Newtonian Calculus was reviewed in the journal Bollettino della Unione Matematica Italiana. [71]

Non-Newtonian Calculus was reviewed in the journal Cahiers du Centre d'Etudes de Recherche Opérationnelle. [72]

The First Nonlinear System of Differential and Integral Calculus was anonymously reviewed at abebooks.com in 2013. [168]
Excerpt: "In the spirit of non-Euclidean geometry ... , so [Michael Grossman] and his colleague Robert Katz have invented [non-Newtonian] calculus. ...  Totally 'out there' in scope, it appears genuine and valid."

Non-Newtonian Calculus was reviewed at amazon.com by Steven Lesko in 2006. [170]
Excerpt: "A Top Mathematical Breakthrough: This humble, eloquent masterpiece gives a concise and readable introduction to a most powerful tool. ... Up there with the likes of Non-Euclidean geometry ..."

=============================================================================================================================================================
Comments
Contents

Home
Multiplicative Calculus
Brief History
Applications
Citations
Reviews
Comments
Quotations
References
Links/Reading
Appendix 1
Appendix 2
Appendix 3
Dedication

"Your ideas [in Non-Newtonian Calculus] seem quite ingenious."
 - Dirk J. Struik, Massachusetts Institute of Technology, USA; from his correspondence, dated 20 April 1972, with Grossman, Grossman, and Katz.
"The monographs on non-Newtonian calculus by [Grossman, Grossman, and Katz] appear to be very useful and innovative."
 - Kenneth J. Arrow, Nobel-Laureate, Stanford University, USA ; from his correspondence, dated 26 November 1980, with Grossman, Grossman, and Katz.
"There is enough here [in Non-Newtonian Calculus] to indicate that non-Newtonian calculi ... have considerable potential as alternative approaches to traditional problems. This very original piece of mathematics will surely expose a number of missed opportunities in the history of the subject."
  - Ivor Grattan-Guinness, Middlesex University, England; from his 1977 review of Non-Newtonian Calculus [101].
"Non-Newtonian Calculus, by Michael Grossman and Robert Katz, is a fascinating and (potentially) extremely important piece of mathematical theory. That a whole family of differential and integral calculi, parallel to but nonlinear with respect to ordinary Newtonian (or Leibnizian) calculus, should have remained undiscovered (or uninvented) for so long is astonishing -- but true. Every mathematician and worker with mathematics owes it to himself to look into the discoveries of Grossman and Katz. The theory has proved to be most valuable in several research studies in which I am engaged. I predict that non-Newtonian calculus will come to be recognized as the most important mathematical discovery of the Twentieth Century." 
 - James R. Meginniss, Claremont Graduate School and Harvey Mudd College, USA; from a 1980 memorandum to his colleagues.
"The purpose of this paper is to present a new theory of probability that is adapted to human behavior and decision making."
 -  James R. Meginniss, Claremont Graduate School and Harvey Mudd College, USA; from his 1980 article "Non-Newtonian calculus applied to probability, utility, and Bayesian analysis" [16].
"The possibilities opened up by the [non-Newtonian] calculi seem to be immense."
 - H. Gollmann, Graz, Austria; from his 1972 review of Non-Newtonian Calculus [53].
 "This [Non-Newtonian Calculus] is an exciting little book ... The greatest value of these non-Newtonian calculi may prove to be their ability to yield simpler physical laws than the Newtonian calculus. Throughout, this book exhibits a clarity of vision characteristic of important mathematical creations. ... The authors have written this book for engineers and scientists, as well as for mathematicians. ... The writing is clear, concise, and very readable. No more than a working knowledge of [classical] calculus is assumed."
 - David Pearce MacAdam, Cape Cod Community College, USA; from his 1973 review of Non-Newtonian Calculus [100].
"It is known that non-Newtonian  calculus models real life problems more accurately."
 - R. C. Mittal, Indian Institute of Technology, India; from the ResearchGate website on 12 November 2014. [218]
"After a long period of silence in the field of non-Newtonian calculus introduced by Grossman and Katz [15] in 1972, the field experienced a revival with the mathematically comprehensive description of the geometric calculus by Bashirov et al. [2], which initiated a kickstart of numerous publications in this field."
 - Mustafa Riza (Eastern Mediterranean University in North Cyprus) and Bugce Eminaga (Girne American University in Cyprus);  from their 2015 article "Bigeometric Calculus and Runge Kutta Method" [215].
"Grossman and Katz have shown that it is possible to create infinitely many calculi independently. They constructed a comprehensive family of calculi, including the Newtonian (or Leibnizian) calculus, the geometric calculus, the bigeometric calculus, and infinitely-many other calculi. ... The geometric and bigeometric calculi have become more and more popular in the past decade."
 - Mustafa Riza (Eastern Mediterranean University in North Cyprus) and Bugce Eminaga (Girne American University in Cyprus);  from their 2015 article "Bigeometric Calculus and Runge Kutta Method" [215].
"There has been great research in geometric (multiplicative) and bigeometric calculus within recent years."
 - Bülent Bilgehan (Girne American University in Cyprus), Bugce Eminaga (Girne American University in Cyprus), and Mustafa Riza (Eastern Mediterranean University in Cyprus); from their 2016 article "New solution method for electrical systems represented by ordinary differential equation" [250]

"Random fractals, a quintessentially 20th century idea, arise as natural models of various physical, biological (think your mother's favorite cauliflower dish), and economic (think Wall Street, or the Horseshoe Casino) phenomena, and they can be characterized in terms of the mathematical concept of fractional dimension. Surprisingly, their time evolution can be analyzed by employing a non-Newtonian calculus ..."
 - Wojbor Woyczynski, Case Western Reserve University, USA; from an abstract to his 2013 seminar "Fractional calculus for random fractals". [146]
"Non-Newtonian calculus for the dynamics of random fractal structures"
 - Wojbor Woyczynski, Case Western Reserve University, USA; from his seminars (in 2011 and 2012) "Non-Newtonian calculus for the dynamics of random fractal structures". [90, 104]
"Many natural phenomena, from microscopic bacteria growth, through macroscopic turbulence, to the large scale structure of the Universe, display a fractal character. For studying the time evolution of such "rough" objects, the classical, "smooth" Newtonian calculus is not enough."
 - Wojbor Woyczynski, Case Western Reserve University, USA; from an abstract to his seminars (in 2011 and 2012) "Non-Newtonian calculus for the dynamics of random fractal structures". [90, 104]
"Together with a small, highly focused research team, Ostoja-Starzewski is working across disciplines to unite methods from solid mechanics, advanced continuum mechanics, statistical physics and mathematics. Some of the specific mathematical theories they use include probability theory and non-Newtonian calculus. These approaches allow them to focus on different fractal structures, including morphogenesis of fractals at elastic-inelastic transitions in solids, composites and soils, as well as materials that have anomalous heat conduction properties and fractal patterns that are seen in biological materials."
 - Martin Ostoja-Starzewski, University of Illinois at Urbana-Champaign, USA; from the 2013 media-upload "The inner workings of fractal materials", University of Illinois at Urbana-Champaign. [163]
"Describing the evolution of defects [in materials] treated as fractals implies usage of the multiplicative derivative, because the ordinary [classical] additive derivative of a function depending on fractal dimension or measure does not exist. ... The goal of this paper is chaos examination in multiplicative dynamical systems described with the multiplicative derivative."  (The expression "multiplicative derivative" refers here to the bigeometric derivative.)
- Dorota Aniszewska and Marek Rybaczuk, both from Wroclaw University of Technology in Poland; from their 2008  article "Lyapunov type stability and Lyapunov exponent for exemplary multiplicative dynamical systems". [131]
" ... evolution of fractal characteristics will be examined with the help of dynamical system theory or more precisely in terms of multiplicative calculus." (The expression "multiplicative calculus" refers here to the bigeometric calculus.)
 - Dorota Aniszewska and Marek Rybaczuk, both from Wroclaw University of Technology in Poland; from their 2009 article "Fractal characteristics of defects evolution in parallel fibre reinforced composite in quasi-static process of fracture". [184]
"We advocate the use of an alternative calculus in biomedical image analysis, known as multiplicative (a.k.a. non-Newtonian) calculus. ... The purpose of this article is to provide a condensed review of multiplicative calculus and to illustrate its potential use in biomedical image analysis. ... Examples have been given in the context of cardiac strain analysis and diffusion tensor imaging to illustrate the relevance of multiplicative calculus in biomedical image analysis, and to support our recommendation for further investigation into practical as well as fundamental issues." (The expression "multiplicative calculus" refers here to the geometric calculus.)
 - Luc Florack and Hans van Assen, both of Eindhoven University of Technology in The Netherlands; from their 2012 article "Multiplicative calculus in biomedical image analysis" [88].
"Multiplicative calculus provides a natural framework in problems involving positive images and positivity preserving operators. In increasingly important, complex imaging frameworks, such as diffusion tensor imaging, it complements standard calculus in a nontrivial way. The purpose of this article is to illustrate the basics of multiplicative calculus and its application to the regularization of positive definite matrix fields." (The expression "multiplicative calculus" refers here to the geometric calculus.)
 - Luc Florack, Eindhoven University of Technology in The Netherlands; from his 2012 article "Regularization of positive definite matrix fields based on multiplicative calculus" [96].

"Our work is an example of how one can benefit from physically refined modelling in conjunction with multiplicative calculi. It is our hope that both concepts will receive more popularity in future computer vision models." 
 - Nico Persch, Christopher Schroers, Simon Setzer, and Joachim Weickert (Gottfried Wilhelm Leibniz Prize winner), all from Saarland University in Germany; from their article on imaging science called "Physically inspired depth-from-defocus". [241]

"The proposed technique combines a new multiplicative gradient operator of non-Newtonian type with the traditional Canny operator to generate the initial edge map ....  Thus, the proposed method is very suitable for fast and accurate edge detection of medical ultrasound images."
- Xiaohong Gong, Yali Zhou, Hao Zhou, and Yinfei Zheng, all from Zhejiang University at Hangzhou in China; from their 2014 article "Ultrasound image edge detection based on a novel multiplicative gradient and Canny operator". [211]

"Reference [..] introduces a novel gradient operator better suited for edge detection in ultrasound or SAR imaging [SAR: synthetic aperture radar]. This new operator, the multiplicative gradient, ... is developed using non-Newtonian calculus."
 - Romulus Terebes (Technical University of Cluj-Napoca [TUCN], Romania), Monica Borda (TUCN, Romania), Christian Germain (IMS Laboratory, Bordeaux, France), Raul Malutan (TUCN, Romania), and Ioana Iles (TUCN, Romania); from their 2015 article "A multiplicative gradient-based anisotropic diffusion approach for speckle noise removal". [268]
"This work presents a new operator of non-Newtonian type which [has] shown [to] be more efficient in contour detection [in images with multiplicative noise] than the traditional operators. ... In our view, the work proposed in (Grossman and Katz, 1972) stands as a foundation ... Innovative applications of non-Newtonian calculus can be found in the field of Bayesian Analysis (Meginniss, 1980)."  [15, 16]
 - Marco Mora, Fernando Córdova-Lepe, and Rodrigo Del-Valle, all of Universidad Católica del Maule in Chile; from their 2012 article [99].
"In our proseminar we'll learn some of the most exciting non-linear calculations and consider the applications for which they may be of particular interest. The applications range from rates of return and other growth processes to highly active areas of digital image processing."
 -  Joachim Weickert (Gottfried Wilhelm Leibniz Prize winner), Saarland University in Germany; from his description of his non-Newtonian calculus course "Analysis beyond Newton and Leibniz", Saarland University, 2012. [106]
"In many circumstances, multiplicative calculus is highly natural; for example, the decay of a radioactive material and the unconstrained growth of a bacterial colony [yield] constant multiplicative derivatives."  (The expression "multiplicative calculus" refers here to the geometric calculus.)
 - Christopher Olah, Thiel Fellow; from Christopher Olah's Blog, 10 June 2011. [134, 135, 177]
"In this paper, we have tried to present how a non-Newtonian calculus could be applied to repostulate and analyse the neoclassical [Solow-Swan] exogenous growth model [in economics]. ... In fact, one must acknowledge that it’s only under the effort of Grossman & Katz (1972) ... that such a non-Newtonian calculus emerged to give a natural answer to many growth phenomena. ... We must underscore that to discover that there was a non-Newtonian way to look to differential equations has been a great surprise for us. It opens the question to know if there are major fields of economic analysis which can be profoundly re-thought in the light of this discovery."
 - Diana Andrada Filip (Babes-Bolyai University of Cluj-Napoca in Romania) and Cyrille Piatecki (Orléans University in France); from their 2014 article [82].
"In this paper, after a brief presentation of [the geometric] calculus, we try to show how it could be used to re-explore from another perspective classical economic theory, more particularly economic growth and the maximum-likelihood method from statistics."
- Diana Andrada Filip (Babes-Bolyai University of Cluj-Napoca in Romania) and Cyrille Piatecki (Orléans University in France); from their 2014 article [181].
"... [non-Newtonian calculus] could help to acquire new insight on classical subjects, or solve directly some problems which could only be reached by approximations."
 - Diana Andrada Filip (Babes-Bolyai University of Cluj-Napoca in Romania) and Cyrille Piatecki (Orléans University in France); from their 2014 article [181].
"The double-entry bookkeeping promoted by Luca Pacioli in the fifteenth century could be considered a strong argument in behalf of the multiplicative calculus, which can be developed from the Grossman and Katz non-Newtonian calculus concept." (The expression "multiplicative calculus" refers here to the geometric calculus.)
 - Diana Andrada Filip (Babes-Bolyai University of Cluj-Napoca in Romania) and Cyrille Piatecki (Orléans University in France); from their 2014 article [149].
"The results of this study are also expected to help researchers, practitioners, economists, business managers, and cost and managerial accountants to understand how to construct a multiplicative based learning curve to improve such decisions as pricing, profit planning, capacity management, and budgeting." 
 - Hasan Özyapıcı (Eastern Mediterranean University in Cyprus), İlhan Dalcı (Eastern Mediterranean University in Cyprus), and Ali Ozyapici (Cyprus International University); from their article "Integrating accounting and multiplicative calculus: an effective estimation of learning curve". [290] (The expression "multiplicative calculus" refers here to the geometric calculus.)

"This work is aimed to show that various problems from different fields can be modeled more efficiently using [geometric] calculus, in place of Newtonian calculus. ... In this study it becomes evident that the [geometric] calculus methodology has some advantages over [classical] calculus in modeling some processes in areas such as actuarial science, finance, economics, biology, demographics, etc."
 - Agamirza E. Bashirov (Eastern Mediterranean University in North Cyprus), Emine Misirli (Ege University in Turkey), Yucel Tandogdu (Eastern Mediterranean University in North Cyprus) Ali Ozyapici, Lefke European University in Turkey); from their 2011 article [94].
"In [1967] Michael Grossman and Robert Katz gave definitions of a new kind of derivative and integral ...  and thus established a new calculus, called multiplicative calculus [the geometric calculus]. ... We think that [the geometric calculus] can especially be useful as a mathematical tool for economics and finance ... In the present paper our aim is to bring [the geometric] calculus to the attention of researchers ...  and [to] demonstrate its usefulness."
 - Agamirza E. Bashirov (Eastern Mediterranean University in North Cyprus), Emine Misirli Kurpinar (Ege University in Turkey), and Ali Ozyapici (Ege University in Turkey); from their 2008 article [2].
"In 2011, Bashirov et al. ["On modeling with multiplicative differential equations"] exploit the efficiency of  [the geometric] calculus over the Newtonian calculus. They demonstrated that the [geometric calculus] differential equations are more suitable than the ordinary differential equations in investigating some problems in various fields. Furthermore, Bashirov et al. [" Multiplicative calculus and its applications"] illustrated the usefulness of [the geometric] calculus with some interesting applications."
 - Feng Gu (Hangzhou Normal University in China) and Yeol-Je Cho (Gyeongsang National University in Korea); from their 2015 article [249].
"Bigeometric calculus - a modelling tool"
 - Mustafa Riza (Eastern Mediterranean University in North Cyprus) and Bugce Eminaga (Girne American University in Cyprus); from their 2014 article "Bigeometric calculus - a modelling tool" [178], which includes a new mathematical model for studying tumor therapy with oncolytic virus.
"Bigeometric Runge-Kutta method is, at least for a particular set of initial value problems, superior with respect to accuracy and computation-time to the ordinary Runge-Kutta method."
 - Mustafa Riza (Eastern Mediterranean University in North Cyprus) and Bugce Eminaga (Girne American University in Cyprus); from their 2015 article "Bigeometric Calculus and Runge Kutta Method" [215], which includes new mathematical models (of the growth of cells, genes, bacteria, and viruses) for studying such things as tumor therapy with oncolytic virus and cell-cycle-specific cancer-chemotherapy.
"While one problem can be easily expressed in one calculus, the same problem can not be expressed as easily [in another]."
  - Emine Misirli and Yusuf Gurefe, both of Ege University in Turkey; from their 2009 lecture [123].
"If non-Newtonian calculus is employed together with classical calculus in the formulations, then many of the complicated phenomena in physics or engineering may be analyzed more easily."
 - Ahmet Faruk Çakmak (Yıldız Technical University in Turkey) and Feyzi Basar (Fatih University in Turkey); from their 2014 article "Certain spaces of functions over the field of non-Newtonian complex numbers". [161]
In this paper, the multiplicative least square method is introduced and is applied to integrals for the finite product representation of the positive functions. Hence, many nonlinear functions can be represented by well-behaved exponential functions. Product representation produces an accurate representation of signals, especially where exponentials occur. Some real applications of nonlinear exponential signals will be selected to demonstrate the applicability and efficiency of proposed representation."
 - Ali Ozyapici (Cyprus International University in Cyprus/Turkey) and Bulent Bilgehan (Girne American University in Cyprus/Turkey); from their article "Finite product representation via multiplicative calculus and its applications to exponential signal processing" [225]
"The main goal of this paper is chaos examination in systems described with multiplicative [bigeometric] differential equations."
 - Dorota Aniszewska and Marek Rybaczuk, both from Wroclaw University of Technology in Poland; from their 2010 article "Chaos in multiplicative systems". [126] 
"Theory and applications of [geometric] and [bigeometric] calculi have been evolving rapidly over the recent years. As numerical minimization methods have a wide range of applications in science and engineering, the idea of the design of minimization methods based on [geometric] and [bigeometric] calculi is self-evident. In this paper, the well-known Newton minimization method for one and two variables is developed in the framework of [geometric] and [bigeometric] calculi. The efficiency of these proposed minimization methods is demonstrated by examples, ... . One of the striking results of the proposed method is that the rate of convergence and the range of initial values are considerably larger compared to the original method."
 - Ali Ozyapici, Mustafa Riza, Bulent Bilgehan, Agamirza E. Bashirov (the first and third authors are from Girne American University in Cyprus/Turkey, the second and fourth from Eastern Mediterranean University in North Cyprus); from the Abstract to their 2013 article [176].
"Grossman and Katz [Non-Newtonian Calculus] mention several alternative calculi including: geometric, anageometric, bigeometric, quadratic, anaquadratic, biquadratic, harmonic, anaharmonic, and biharmonic. ... Non-Newtonian calculus has been used to derive optimization algorithms that perform better than traditional Newton based methods for Expectation-Maximization algorithms. However, Non-Newtonian calculus goes beyond simply being useful for optimization, it is useful for the other half of learning: modeling. The second order approximation using geometric calculus may produce the Gaussian curve ... . The nth order approximation using bigeometric calculus produces an nth polynomial on a log-log plot. ... Non-Newtonian generalized Taylor expansions produce nth order models, which are rarely polynomials. ... Non-Newtonian models sometimes make sense to use. Non-Newtonian models follow from non- Newtonian calculi. ... Here are a few rules of thumb for non-Newtonian models. If a meta-model is primarily concerned with learning probabilities, non-parametric distributions, or anything else where the multiplication is the primary operation, then the geometric calculi may be of interest. If working in a domain where the squares are additive, as is common the case when estimating the variance of a sum of independent random variables, then the quadratic calculi may produce meaningful models."
Michael Valenzuela (University of Arizona in the United States); from his 2016 doctoral dissertation "Machine learning, optimization, and anti-training with sacrificial data" (In computer science, machine learning is a branch of artificial intelligence.) [279]

"It seems plausible that people who need to study functions from this point of view might well be able to formulate problems more clearly by using [bigeometric] calculus instead of [classical] calculus."
 - Ralph P. Boas, Jr., Northwestern University, USA; from his 1984 review of  Bigeometric Calculus: A System with a Scale-Free Derivative [47].
"The more innovative parts of the Stern Review – the non-Newtonian calculus in Chapter 13, for instance – have yet to be submitted to learned journals."  
 - Richard Tol, University of Sussex, England; from his forward to "What is Wrong with Stern?", a critique (4 September 2012) of the 2006 report "Stern Review on the Economics of Climate Change". [116, 165]
"... I’m very interested in the application of non-Newtonian calculus to computational neuroscience, specifically for solving biophysical models of the generation of neuronal activity. The sigmoidal calculus, as introduced in your book Non-Newtonian Calculus has the potential to be a very useful approach to the problems I want to solve ... ." [15]
 - Roberto Sotero Diaz, Hotchkiss Brain Institute of the University of Calgary, Canada; from his correspondence, dated 31 October 2014, with Grossman, Grossman, and Katz. 
"I am pretty sure non-Newtonian calculus would improve option-pricing models [in finance]." 
 - Raymond Tang, financial advisor, Etechadvisors Inc., Laguna Beach, CA, USA; from his correspondence, dated 15 April 2014, with Grossman and Grossman. (Also, please see [37].)
"An interesting feature of this chapter is an introduction to multiplicative calculus, which is an alternative to the [classical] calculus of Newton and Leibnitz. By use of methods of multiplicative calculus it is proved that an infinitely-many times differentiable function may not be analytic." (The expression "multiplicative calculus" refers here to the geometric calculus.)
 - Agamirza E. Bashirov, Eastern Mediterranean University, North Cyprus; from the Abstract to Chapter 11 of his 2014 textbook Mathematical Analysis: Fundamentals [179].
"As readers of this blog are probably aware, I’m a rather big fan of multiplicative calculus ... So, it should come as no surprise that when I was given an opportunity to speak at Singularity Summit last fall, a conference largely concerned with exponential trends in technology, I decided to try to persuade people of the utility of multiplicative calculus and the value of mathematical abstractions."  (The expression "multiplicative calculus" refers here to the geometric calculus.)
 - Christopher Olah, Thiel Fellow; from Christopher Olah's Blog, 07 May 2013. [134, 135, 164]

"In this [project] you will explore alternative 'calculi'.  How can there be multiple calculi? ... I would like you to explore one (or a family) of non-Newtonian or multiplicative calculi.  Make sure to (i) be precise with your definitions and assumptions, (ii) describe how your non-Newtonian calculus differs from the standard calculus, (iii) perform a sample computation or two illustrating key concepts from your calculus, (iv) describe an application  of  non-Newtonian  calculus,  or  at  least  a  mathematical  or  real-world  situation where it might be useful."
Active Link                 Active Link             Active Link
 - Justin Webster, College of Charleston, USA; from his instructions to students for the project "Alternative Calculi: Multiplicative or Non-Newtonian Calculus" in his Introductory Calculus course (MATH 120), 2015. [239]



NOTE. The six books on non-Newtonian calculus and related matters by Jane Grossman, Michael Grossman, and Robert Katz are indicated below, and are available at some academic libraries, public libraries, and booksellers such as Amazon.com. On the Internet, each of the books can be read and downloaded, free of charge, at HathiTrust, Google Books, and the Digital Public Library of America.
Michael Grossman and Robert Katz.  Non-Newtonian Calculus, ISBN 0912938013, 1972. [15] 
Michael Grossman. The First Nonlinear System of Differential and Integral Calculus, ISBN 0977117006, 1979. (The geometric calculus) [11] 
Jane Grossman, Michael Grossman, Robert Katz. The First Systems of Weighted Differential and Integral Calculus, ISBN 0977117014, 1980. [9]
Jane Grossman. Meta-Calculus: Differential and Integral, ISBN 0977117022, 1981. [7]
Michael Grossman. Bigeometric Calculus: A System with a Scale-Free Derivative, ISBN 0977117030, 1983. [10]
Jane Grossman, Michael Grossman, and Robert Katz. Averages: A New Approach, ISBN 0977117049, 1983. [8]

=============================================================================================================================================

Quotations
Contents

Home
Multiplicative Calculus
Brief History
Applications
Citations
Reviews
Comments
Quotations
References
Links/Reading
Appendix 1
Appendix 2
Appendix 3
Dedication

"For each successive class of phenomena, a new calculus or a new geometry, as the case might be, which might prove not wholly inadequate to the subtlety of nature."
  - Quoted without citation by Henry John Stephen Smith in Nature, Volume 8, page 450 (1873).

"In general the position as regards all such new calculi is this - That one cannot accomplish by them anything that could not be accomplished without them. However, the advantage is, that, provided such a calculus corresponds to the inmost nature of frequent needs, anyone who masters it thoroughly is able - without the unconscious inspiration of genius which no one can command - to solve the respective problems, indeed to solve them mechanically in complicated cases in which, without such aid, even genius becomes powerless. Such is the case with the invention of general algebra, with the differential calculus, and in a more limited region with Lagrange's calculus of variations, with my calculus of congruences, and with Mobius' calculus. Such conceptions unite, as it were, into an organic whole countless problems which otherwise would remain isolated and require for their separate solution more or less application of inventive genius."
 - Carl Friedrich Gauss, as quoted in Carl Friedrich Gauss: Werke, Volume 8, page 298; and as quoted in Robert Edouard Moritz's book Memorabilia Mathematica or The Philomath's Quotation Book, quotation #1215 (1914).

      "But if some mind very different from ours were to look upon some property of some curved line as we do on the evenness of a straight line, he would not recognize as such the evenness of a straight line; nor would he arrange the elements of his geometry according to that very different system, and would investigate quite other relationships as I have suggested in my notes.
     "We fashion our geometry on the properties of a straight line because that seems to us to be the simplest of all. But really all lines that are continuous and of a uniform nature are just as simple as one another. Another kind of mind which might form an equally clear mental perception of some property of any one of these curves, as we do of the congruence of a straight line, might believe these curves to be the simplest of all, and from that property of these curves build up the elements of a very different geometry, referring all other curves to that one, just as we compare them to a straight line. Indeed, these minds, if they noticed and formed an extremely clear perception of some property of, say, the parabola, would not seek, as our geometers do, to rectify the parabola, they would endeavor, if one may coin the expression, to parabolify the straight line." 
Roger Joseph Boscovich, as quoted in "Boscovich's mathematics", an article by J. F. Scott, in the book Roger Joseph Boscovich, edited by Lancelot Law Whyte (1961); and as quoted in "Transient pressure analysis in composite reservoirs" (1982) by Raymond W. K. Tang and William E. Brigham (1982).

"... all the ingenious analysis which has evolved from the hypothesis of linearity is at best a first approximation to the applicable mathematics of the future."
 - E. T. Bell, from his book The Development of Mathematics (1945), and as quoted in the book The First Nonlinear System of Differential and Integral Calculus (1979).

"It has long been recognized that biological growth is multiplicative in style, and not accretionary or additive."
 - Peter B. Medawar (Nobel-Laureate), from his book The Uniqueness of the Individual (1958), and as quoted in the article "Which growth rate?" (1987) by Jane Grossman, Michael Grossman, and Robert Katz. 

"... the norm of biological growth - the standard to which all actual instances of growth must be referred - is exponential growth."
 - Peter B. Medawar (Nobel-Laureate), from his book Pluto's Republic (1982), and as quoted in the article "Which growth rate?" (1987) by Jane Grossman, Michael Grossman, and Robert Katz. 

"It is departure from exponential growth that calls for comment and explanation, just as with departure from uniform motion in a straight line."
Peter B. Medawar (Nobel-Laureate),  from his book The Uniqueness of the Individual (1958), and as quoted in the article "Which growth rate?" (1987) by Jane Grossman, Michael Grossman, and Robert Katz.
"A large part of mathematics which becomes useful developed with absolutely no desire to be useful, and in a situation where nobody could possibly know in what area it would become useful; and there were no general indications that it would ever be so. By and large it is uniformly true in mathematics that there is a time lapse between a mathematical discovery and the moment when it is useful; and that this lapse of time can be anything from 30 to 100 years, in some cases even more; and that the whole system seems to function without any direction, without any reference to usefulness, and without any desire to do things which are useful."
 - John von Neumann, as quoted in Out of the Mouths of Mathematicians: A Quotation Book for Philomaths by R. Schmalz (1993).

"Insight must precede application."
 - Max Planck, as quoted by the Max Planck Society and used as its motto.

"Nowadays every school child learns a bit of the theories of George Boole and Georg Cantor. Theories originally perceived as exotic and fanciful are indeed the foundation of our computer age. But too often we seem to miss the fundamental lesson: today's abstractions become tomorrow's applications."
 - Lynn Arthur Steen, from his address "Mathematics: Our Invisible Culture" at a symposium at Dickinson College in Pennsylvania in September of 1985.

"As an answer to those who are in the habit of saying to every new fact, 'What is its use?', [Benjamin] Franklin says, 'What is the use of an infant?'"   
 - Michael Faraday, as quoted by I. Bernard Cohen in his article "Faraday  and Franklin's "Newborn Baby"" in Proceedings of the American Philosophical Society, Volume 131, Number 2 (1987). 

"It is the fate of the scientist to face the constant demand that he show his learning to have some “practical use.”  Yet it may not be of interest to him to have such a “practical use” exist; he may feel that the delight of learning, of understanding, of probing the universe, is its own reward."
Isaac Asimov, from his article "Of what use?", an introduction to the book The Greatest Adventure: Basic Research That Shapes Our Lives (1974).

"The reward of a thing well done is to have done it."
 - Ralph Waldo Emerson from his Essays: Second Series, 1844, as quoted by Robert Katz in a conversation with Michael Grossman in the 1970s.

"A century ago, [Einstein's] general relativity had no obvious “impact” ... . It didn’t even have a clear goal, except intellectually."
 - Philip Ball, from his article "There’s no space for today’s young Einsteins" in The Guardian (12 February 2016).

"There seem to be two kinds of discovery. In one kind, the goal is given first, and then the mind goes from the goal to the means, that is, from the question to the solution. In the other kind, the mind goes from the means to the goal, that is, the mind first discovers a fact and then seeks a use for it. In mathematics, and elsewhere,  most significant discoveries are of the second kind. ... An outstanding example in mathematics is the exhaustive study of the conics by the Greeks, and then, some two thousand years later, Kepler's stunning application of the Greek findings to the movement of the planets in the solar system."
- Howard Eves, from his book Mathematical Circles Squared, pages 167 and 168 (1972).

"... we find in the history of ideas mutations which do not seem to correspond to any obvious need, and at first sight appear as mere playful whimsies - such as Apollonius' work on conic sections, or the non-Euclidean geometries, whose practical value became apparent only later."
- Arthur Koestler, from his book The Sleepwalkers (1959).

"Newton's epoch-making works (1669, 1671) were offered to the Royal Society and Cambridge University Press but, incredible as it now seems, were rejected for publication."
 - John Stillwell, from his book Mathematics and Its History (2010).

"More than 350 years after the Roman Catholic Church condemned Galileo, Pope John Paul II is poised to rectify one of the Church's most infamous wrongs -- the persecution of the Italian astronomer and physicist for proving the Earth moves around the Sun."
 - Alan Cowell, from his article "After 350 Years, Vatican Says Galileo Was Right: It Moves" in The New York Times, 31 October 1992.

"Yet Gauss, universally regarded as the foremost mathematician of his day, did not publicize his findings. ...  In an 1829 letter to a confidant, Gauss observed that he had no plans 'to work up my very extensive researches for publication, and perhaps they will never appear in my lifetime, for I fear the howl of the Boeotians if I speak my opinion out-loud.' While today's reader may miss a bit of this classical allusion, suffice it to say that being called a "Boeotian" is being labeled an unimaginative, crudely obtuse dullard. Obviously Gauss had little regard for the reception of the mathematical community for his new ideas."
 - William Dunham as quoted in his book Journey Through Genius (1990).

"Of all of Gauss' creations, the most revolutionary in concept and the most weighty in affecting the course of mathematics and science is non-Euclidean geometry. ... Since Gauss did not make known many of his brilliant creations, other mathematicians [such as Lobatchevsky and Bolyai] worked toward and arrived at the same results independently. ... In the case of non-Euclidean geometry, Gauss had an additional reason for not publishing his work. ... [A]nyone professing to present another geometry conflicting with Euclid's would undoubtedly have been judged insane not only by the general public, which was ignorant of all geometry, but by mathematicians, scientists, and philosophers. ... There are limits to allowable heterodoxy, even in mathematics, and these limits are soon reached in every era."
 - Morris Kline, from his book Mathematics and the Physical World (1959)

"For thirty years or so after the publication of Lobatchevsky's and Bolyai's works all but a few mathematicians ignored the non-Euclidean geometries. ... The mere fact that there can be alternative geometries was in itself a shock."
 - Morris Kline, from his book Mathematics: The Loss of Certainty (1980).

"Georg Cantor (1845 - 1918), the creator of transfinite set theory, is one of the most imaginative and controversial figures in the history of mathematics. ... Because his views were unorthodox, they stimulated lively debate and at times vigorous denunciation. Leopold Kronecker considered Cantor a scientific charlatan, a renegade, a "corrupter of youth," but Bertrand Russell described him as one of the greatest intellects of the nineteenth century. David Hilbert believed Cantor had created a new paradise for mathematicians, though others, notably Henri Poincare, thought set theory and Cantor's transfinite numbers represented a grave mathematical malady, a perverse pathological illness that would one day be cured. ... Cantor's creation of transfinite set theory was an achievement of major consequence in the history of mathematics."
 - Joseph Dauben, from his book Georg Cantor: His Mathematics and Philosophy of the Infinite, pages 1 and 6 (1990).

"Resistance to irrationals [i.e., irrational numbers] continued for thousands of years. In the late nineteenth century, when the gifted German mathematician Georg Cantor did groundbreaking work to put them on firmer footing, his former teacher, a crab named Leopold Kronecker who "opposed" the irrationals, violently disagreed with Cantor and sabotaged his career at every turn. Cantor, unable to tolerate this, had a breakdown and spent his last days in a mental institution."
 - Leonard Mlodinow, from his book Euclid's Window: The Story of Geometry from Parallel Lines to Hyperspace (2001).

"Mathematicians, let it be known, are often no less illogical, no less closed-minded, and no less predatory than most men. Like other closed minds they shield their obtuseness behind the curtain of established ways of thinking while they hurl charges of madness against the men who would tear apart the fabric."  
 - Morris Kline, from his book Mathematics in Western Culture (1953).

"The mathematician Georg Cantor spoke of a law of conservation of ignorance. A false conclusion once arrived at and widely accepted is not easily dislodged and the less it is understood the more tenaciously it is held."
Morris Kline, from his book Mathematics: The Loss of Certainty (1980).

"Gregor Mendel is accorded a special place in the history of genetics. ... Poignantly overshadowing the creative brilliance of Mendel's work is the fact that it was virtually ignored for thirty-four years. Only after the dramatic rediscovery in 1900 - sixteen years after Mendel's death - was Mendel rightfully recognized as the founder of genetics."
 - Daniel L. Hartl and Vitezslav Orel, from their article "What did Gregor Mendel think he discovered?", Genetics, Volume 131, Genetics Society of America (1992).
"Some truly revolutionary scientific theories may take years or decades to win general acceptance among scientists. This is certainly true of plate tectonics, one of the most important and far-ranging geological theories of all time; when first proposed, it was ridiculed, but steadily accumulating evidence finally prompted its acceptance, with immense consequences for geology, geophysics, oceanography, and paleontology. And the man who first proposed this theory was a brilliant interdisciplinary scientist, Alfred Wegener. ... Reaction to Wegener's theory was almost uniformly hostile, and often exceptionally harsh and scathing ... ."
 - from the article "Alfred Wegener (1880-1930)" at the website of the University of California Museum of Paleontology.
"Nikola Tesla was one of the greatest electrical inventors who ever lived. ...  Ridiculed by his contemporaries, his ideas frequently appeared in works of science fiction. ... Tesla was so far ahead of his time that many of his ideas are only appearing today. His legacy can been seen in everything from microwave ovens to MX missiles."
 - from the abstract "Who Was Nikola Tesla?" for The Public Broadcasting Service (PBS) program Tesla: Master of Lightning, which first aired on PBS in December of 2000.
"Daniel Shechtman, who has won the chemistry Nobel for discovering quasicrystals, was initially lambasted for 'bringing disgrace' on his research group. -  A scientist whose work was so controversial he was ridiculed and asked to leave his research group has won the [2011] Nobel Prize in Chemistry. Daniel Shechtman, 70, a researcher at Technion-Israel Institute of Technology in Haifa, received the award for discovering seemingly impossible crystal structures in frozen gobbets of metal that resembled the beautiful patterns seen in Islamic mosaics. ... In an interview this year with the Israeli newspaper, Haaretz, Shechtman said: "People just laughed at me." He recalled how Linus Pauling, a colossus of science and a double Nobel laureate, mounted a frightening "crusade" against him. After telling Shechtman to go back and read a crystallography textbook, the head of his research group asked him to leave for "bringing disgrace" on the team."
 - Ian Sample, from his article "Nobel Prize in chemistry for dogged work on 'impossible' quasicrystals" in The Guardian, 5 October 2011.
"When Ludwig Boltzmann [1844 - 1906] first established statistical mechanics, he was ridiculed by his colleagues Ernst Mach, Wilhelm Ostwald, and others for his atomistic ideas, and it took three decades until experiments made the reality of atoms and molecules concrete enough to convince the community."
 - Christoph von der Malsburg, William A. Phillips, and Wolf Singer, from their book Dynamic Coordination in the Brain: From Neurons to Mind (2010).
"Chandrasekhar developed a theory about the nature of stars for which he would be awarded the Nobel Prize 53 years later, in 1983. His theory challenged the common scientific notion of the 1930s that all stars, after burning up their fuel, became faint, planet-sized remnants known as white dwarfs. ... Initially his theory was rejected by peers and professional journals in England. The distinguished astronomer Sir Arthur Eddington publicly ridiculed his suggestion that stars could collapse into such objects, which are now known as black holes. ... Today, the extremely dense neutron stars and black holes implied by Chandrasekhar’s early work are a central part of the field of astrophysics."
 from the The University of Chicago's press release "Subrahmanyan Chandrasekhar", 22 August 1985.

"Few things inhibit the undertaking of a new venture more than the fear of ridicule."
 - Robert Katz, as quoted in Paul Dickson's book The Official Rules [28] (2014).
"The obstacles mainly were in the very beginning, in the late '60s, when we proposed the idea that tumors need to recruit their own private blood supply. That was met with almost universal hostility and ridicule and disbelief by other scientists. ...  A lot of people would walk out of my presentations. There were many critics, very great experts who kept saying this couldn't be. ... We had ten years of really tough ridicule. I was sometimes very upset."
 ― Judah Folkman, from an interview on June 18, 1999, published in the online-article "Judah Folkman Interview" by the Academy of Achievement on 21 September 2010. (Judah Folkman (1933 - 2008) founded angiogenesis research, a field of biology which revolutionized biomedical research and clinical drug development. He created a novel approach to understanding and treating many diseases, including cancer.)

"Ignaz Semmelweis (born Semmelweis Ignác Fülöp; 1 July 1818 – 13 August 1865) was a Hungarian physician of German extraction now known as an early pioneer of antiseptic procedures. ... Semmelweis discovered that the incidence of puerperal fever (also known as "childbed fever") could be drastically cut by the use of hand disinfection in obstetrical clinics. ... The so-called Semmelweis reflex — a metaphor for a certain type of human behaviour characterized by reflex-like rejection of new knowledge because it contradicts entrenched norms, beliefs or paradigms — is named after Semmelweis, whose perfectly reasonable hand washing suggestions were ridiculed and rejected by his contemporaries."
 - from the article "Ignaz Semmelweis" at the Wikipedia website on 15 August 2015.

"In 1891, Dr. William B. Coley [1862 - 1936] injected streptococcal organisms into a patient with inoperable cancer. He thought that the infection he produced would have the side effect of shrinking the malignant tumor. He was successful, and this was one of the first examples of immunotherapy. ... William Coley's intuitions were correct: Stimulating the immune system may be effective in treating cancer.  ...  But he was a man before his time, and he met with severe criticism. Despite this criticism, however, Coley stuck with his ideas, and today we are recognizing their potential value."
 - Edward F. McCarthy, from his article "The toxins of William B. Coley and the treatment of bone and soft-tissue sarcomas" in The Iowa Orthopaedic Journal, Volume 26, US National Library of Medicine, National Institutes of Health (2006).

"Biologist Lynn Margulis died on November 22nd. Her major work was in cell evolution ...  Her ideas were generally either ignored or ridiculed when she first proposed them; [her theory of] symbiosis in cell evolution is now considered one of the great scientific breakthroughs."
 - John Brockman, from his article "Lynn Margulis 1938-2011" on the Edge.org website, 23 November 2011. 

"In Western Australia, a young doctor, working in relative isolation, pursued another hypothesis, the possibility that these chronic stomach ailments [such as ulcers and gastritis] were caused by a microscopic corkscrew-shaped organism called Helicobacter pylori. For over a decade, Dr. Barry Marshall endured the hostility and ridicule of a medical establishment deeply invested in the received wisdom that peptic ulcers were a chronic condition requiring a lifetime of treatment. ... Today his discovery is recognized as one of the greatest breakthroughs in medicine since the polio vaccine."
 - from the Academy of Achievement article "Barry Marshall: Profile", http://www.achievement.org/autodoc/page/mar1pro-1, February 26, 2010. (In 2005, Barry Marshall was awarded The Nobel Prize in Physiology or Medicine.)

"As you well know, for many years lots of people, especially various pure mathematicians, claimed that our work was useless. But, despite their discouraging and sometimes arrogant comments, we always knew that non-Newtonian calculus has considerable potential for application in science, engineering, and mathematics. ... And we were right!!"
 - Michael Grossman, from his letter to Robert Katz on 21 July 2014. 

"After a long period of silence in the field of non-Newtonian calculus introduced by Grossman and Katz [15] in 1972, the field experienced a revival with the mathematically comprehensive description of the geometric calculus by Bashirov et al. [2], which initiated a kickstart of numerous publications in this field."
Mustafa Riza and Bugce Eminaga, from their article "Bigeometric Calculus and Runge Kutta Method" [215] (2015).

"There is enough here [in Non-Newtonian Calculus] to indicate that non-Newtonian calculi ... have considerable potential as alternative approaches to traditional problems. This very original piece of mathematics will surely expose a number of missed opportunities in the history of the subject."
  - Ivor Grattan-Guinness,  from his review of Non-Newtonian Calculus [101] (1977).
"[B]y far the most usual way of handling phenomena so novel that they would make for a serious rearrangement of our preconceptions is to ignore them altogether, or to abuse those who bear witness for them."
 - William James, as quoted in the book Pragmatism: A Series of Lectures by William James, 1906-1907 (2008).
"I'm pleased that, towards the end of his life, [Benoit Mandelbrot] received due recognition, because it took a long time for the mathematical community to understand something that must have been obvious to him: fractals were important. They were a game changer, opening up completely new ways to think about many aspects of the natural world. But for a long time it was not difficult to find professional research mathematicians who stoutly maintained that fractals and chaos were completely useless and that all of the interest in them was pure hype. This attitude persisted into the current century, when fractals had been around for at least twenty-five years and chaos for forty. That this attitude was narrow-minded and unimaginative is easy to establish, because by that time both areas were being routinely used in branches of science ranging from astrophysics to zoology. It was clear that the critics hadn’t deigned to sully their lily-white hands by picking up a random copy of Nature or Science and finding out what was in it."
 - Ian Stewart, from the article "The influence of Benoît B. Mandelbrot on mathematics", edited by Michael F. Barnsley and Michael Frame, in Notices of the American Mathematical Society, October of 2012. 

"Benoit Mandelbrot survived Nazi-occupied France to become one of the most creative thinkers of the 20th century. ... He coined the term “fractal” in 1975, from the Latin word “fractus,” meaning broken or shattered, to better measure rough shapes and irregular surfaces, from graphs of the stock market to coastlines. ... [H]is fractal sets have turned out to have a fabulous number of applications in many additional fields, including mathematics, economics, the sciences and the arts. ... IBM employed him for decades as a researcher (“I was in an industrial laboratory because academia found me unsuitable,” Mandelbrot explained at the time). ... Mandelbrot’s new ideas were laughed at widely when first developed ... Mandelbrot was likewise seen as making little sense in his adopted homeland, since French mathematics was governed for decades by the accomplished Bourbaki group, led by André Weil — brother of the philosopher Simone Weil — and other trendsetters. To such intellectuals, Mandelbrot was a visibly freakish phenomenon. ... Mandelbrot proved to be a uniquely serious innovator, a Kepler of the century past."
  - Benjamin Ivry, from his article "Benoit Mandelbrot Influenced Art and Mathematics" in The Jewish Daily Forward of 23 November 2012.

"Aczel ... has skipped over what I consider the pernicious effects of Bourbaki. In France, where Bourbaki ruled the roost for a while, applied mathematics received a sharp slap in the face; more than that: a body blow. In the USA, the New Math, located in the high clouds of abstraction, drew its breath of life from Bourbaki. After expending the energies of enthusiasts and spending tens of millions in cash, after abusing the patience of teachers, parents, and a goodly proportion of professional mathematicians, the New Math was finally acknowledged to be an abject and unmitigated failure. (And the failure was predictable, according to some.)
 - Philip J. Davis, from his review "Departed Glories: Bourbaki" of Amir D. Aczel's book The Artist and the Mathematician: The Story of Nicolas Bourbaki, the Genius Mathematician Who Never Existed, in SIAM NEWS (Society for Industrial and Applied Mathematics) on 12 January 2006.
"Throughout the centuries there were men who took first steps down new roads armed with nothing but their own vision. Their goals differed, but they all had this in common: that the step was first, the road new, the vision unborrowed, and the response they received -- hatred. ... The first motor was considered foolish. The first airplane was considered impossible. The power loom was considered vicious. Anesthesia was considered sinful. But the men of unborrowed vision went ahead. They fought, they suffered and they paid. But they won."  
 - Ayn Rand, from her book The Fountainhead (1943).
“The human mind treats a new idea the same way the body treats a strange protein; it rejects it.”
 - Peter B. Medawar , from his book The Art of the Soluble (1967).

"Scientists sometimes boast by implication when they criticize or minimize the achievements of others."
 - Stanislaw Ulam, from his book Adventures of a Mathematician (1976).
"Like all revolutionary new ideas, the subject [of space exploration] has had to pass through three stages, which may be summed up by these reactions: (1) 'It's crazy - don't waste my time.' (2) 'It's possible, but it's not worth doing.' (3) 'I always said it was a good idea.'"
 - Arthur C. Clarke, from his book Report on Planet Three and Other Speculations (1972).
"Every great advance in science has issued from a new audacity of imagination."
- John Dewey, as quoted in the book The Quest for Certainty: Gifford Lectures, a series of lectures by John Dewey in 1929.
"Science goes where you imagine it.”
 - Judah Folkman, as quoted in the article "Folkman Looks Ahead" by Claudia Kalb in Newsweek magazine on 18 February 2001. 
"Perhaps the only thing that saves science from invalid conventional wisdom that becomes effectively permanent is the presence of mavericks in every generation - people who keep challenging convention and thinking up new ideas ... ."    
 - David M. Raup, from his book The Nemesis Affair: A Story of the Death of Dinosaurs and the Ways of Science (1999).
"In mathematics the art of asking questions is more valuable than solving problems."
 - Georg Cantor, from his doctoral The geometric calculus and some of its applications are the topics of the 2009 doctoral dissertation of Ali Ozyapici at Ege University in Turkey. The dissertation is entitled "Multiplicative calculus and its applications". [191]  

Non-Newtonian calculus and some of its applications are the topics of the 2011 doctoral dissertation of Ugur Kadak at Gazi University in Turkey. The dissertation is entitled "Non-Newtonian analysis and its applications". [187]

The geometric calculus and some of its applications are the topics of the 2113 doctoral dissertation of Yusuf Gurefe at Ege University in Turkey. The dissertation is entitled "Multiplicative differential equations and applications". [208]

Non-Newtonian analysis was used in the 2014 doctoral dissertation of Ahmet Faruk Cakmak at Yıldız Technical University in Turkey. The dissertation is entitled "Some new sequence spaces over a new field". [205]

Non-Newtonian calculus is a featured topic of the 2016 doctoral dissertation of Zakaria Adnan at Kwame Nkrumah University of Science and Technology in Ghana. The dissertation is entitled "An analysis of Runge-Kutta method in non-Newtonian calculus". [278] (1867).
"Now, my uncle [Szolem Mandelbrot], who was a [brilliant] mathematician given to strong opinions, was very scornful of some of his peers. He said that they were very, very good, but they were just theorem-provers. They have an extraordinary arsenal of techniques, remember many previous results, and put them together in new ways. But they don't have the creativity to ask new questions. So in mathematics there has been historically this more or less sharp distinction between those who are best known for asking questions and those who are best known for proving theorems that others have conjectured. The greatest mathematician in my private pantheon has been Henri Poincaré. Altogether a very great man, he started many branches of mathematics from scratch, but he acknowledged himself that he didn't prove any difficult theorem and cared about proofs less than about concepts."
 - Benoit Mandelbrot, from the interview "A Radical Mind" with Benoit Mandelbrot, conducted by Bill Jersey on 24 April  2005, posted at PBS's NOVA Online on 01 October 2008, and broadcast on PBS's NOVA television-program Hunting the Hidden Dimension on 24 August 2011. 
" I would say that mathematics is the science of skillful operations with concepts and rules invented just for this purpose. The principal emphasis is on the invention of concepts."
 - Eugene Wigner, from his article "The unreasonable effectiveness of mathematics in the natural sciences," in the journal Communications in Pure and Applied Mathematics, Volume 13, Number 1 (February 1960). 

"The formulation of a problem is far more often essential than its solution, which may be merely a matter of mathematical or experimental skill. To raise new questions, new possibilities, to regard old problems from a new angle requires creative imagination and marks real advance in science."  
 - Albert Einstein and Leopold Infield, from their book The Evolution of Physics (1938). 
"It seems plausible that people who need to study functions from this point of view might well be able to formulate problems more clearly by using [bigeometric] calculus instead of [classical] calculus."
 - Ralph P. Boas, Jr., from his review [47] of Bigeometric Calculus: A System with a Scale-Free Derivative (1984).
"It is known that non-Newtonian  calculus models real life problems more accurately."
 - R. C. Mittal, from the ResearchGate website on 12 November 2014 [218].
“Every discovery I made while at IBM fell well outside the scope of any university department. ... [M]y work on the distribution of galaxies would not be published until it became known and understood, but could not become known and understood until it was published. ... The stigma attached to being, to a degree, a vanity-press book is erased by large sales ... and wide influence."
 - Benoit Mandelbrot, from "A maverick's apprenticeship: The Wolf Prize for Physics", an essay he wrote on receiving the 1993 Wolf Prize for Physics (1993).
"Some insights are resisted with such intensity that it may take decades before they're widely accepted among scientists. For some, such as HIV, heliocentrism, and evolution, pockets of resistance remain decades or centuries after the war is won. ... Are you ready to have the top scientists in your field criticizing your work in journals and dissing you at meetings? Because history shows that the deeper your idea cuts into the heart of a field, the more your peers are likely to challenge you. Human nature being what it is, what ought to be reasoned discussion may turn personal, even nasty. ... Progress is made when good scientists keep working -- and keep supporting what they believe is true -- despite the criticism."
 - Anne Sasso, from her article "Audacity, Part 5: Rejection and Ridicule" in the magazine Science (American Association for the Advancement of Science), 11 June 2010. 

"A new scientific innovation does not triumph by convincing its opponents and making them see the light, but rather because its opponents eventually die, and a new generation grows up that is familiar with it."
 - Max Planck, from his book The Philosophy of Physics (1936).
"Flying in the face of the Establishment with unconventional ideas and methods ... is highly esteemed in academia --  until somebody actually does it."
 - Edward Tenner, from his article "Benoit Mandelbrot the Maverick, 1924-2010" in The Atlantic magazine (16 October 2010).

NOTE. The six books on non-Newtonian calculus and related matters by Jane Grossman, Michael Grossman, and Robert Katz are indicated below, and are available at some academic libraries, public libraries, and booksellers such as Amazon.com. On the Internet, each of the books can be read and downloaded, free of charge, at HathiTrust, Google Books, and the Digital Public Library of America.
Michael Grossman and Robert Katz.  Non-Newtonian Calculus, ISBN 0912938013, 1972. [15] 
Michael Grossman. The First Nonlinear System of Differential and Integral Calculus, ISBN 0977117006, 1979. (The geometric calculus) [11] 
Jane Grossman, Michael Grossman, Robert Katz. The First Systems of Weighted Differential and Integral Calculus, ISBN 0977117014, 1980. [9]
Jane Grossman. Meta-Calculus: Differential and Integral, ISBN 0977117022, 1981. [7]
Michael Grossman. Bigeometric Calculus: A System with a Scale-Free Derivative, ISBN 0977117030, 1983. [10]
Jane Grossman, Michael Grossman, and Robert Katz. Averages: A New Approach, ISBN 0977117049, 1983. [8]




========================================================================================================================================================================
References
Contents

Home
Multiplicative Calculus
Brief History
Applications
Citations
Reviews
Comments
Quotations
References
Links/Reading
Appendix 1
Appendix 2
Appendix 3
Dedication
[1] Dorota Aniszewska. "Multiplicative Runge-Kutta methods", Nonlinear Dynamics, Volume 50, Numbers 1-2, Springer,  2007.
[2] Agamirza E. Bashirov, Emine Misirli Kurpinar, and Ali Ozyapici. "Multiplicative calculus and its applications", Journal of Mathematical Analysis and Applications, Volume 337, Issue 1, pages 36 - 48,  http://dx.doi.org/10.1016/j.jmaa.2007.03.081, Elsevier, January 2008. 

[3] Fernando Córdova-Lepe. "From quotient operation toward a proportional calculus", Journal of Mathematics, Game Theory and Algebra, 2004. (Also, please see [105].)
[4] Fernando Córdova-Lepe. "The multiplicative derivative as a measure of elasticity in economics", TMAT Revista Latinoamericana de Ciencias e Ingeniería, Volume 2, Number 3, 2006.

[5] Felix R. Gantmacher. The Theory of Matrices, Volumes 1 and 2, Chelsea Publishing Company, 1959.
[6] Ivor Grattan-Guinnness. The Rainbow of Mathematics: A History of the Mathematical Sciences, pages 332 and 774, ISBN 0393320308, W. W. Norton & Company, 2000.
[7] Jane Grossman. Meta-Calculus: Differential and Integral , ISBN 0977117022, 1981. 

[8] Jane Grossman, Michael Grossman, and Robert Katz. Averages: A New Approach,  ISBN 0977117049, 1983.
[9] Jane Grossman, Michael Grossman, Robert Katz. The First Systems of Weighted Differential and Integral Calculus, ISBN 0977117014, 1980.

[10] Michael Grossman. Bigeometric Calculus: A System with a Scale-Free Derivative, ISBN 0977117030, 1983.

[11] Michael Grossman. The First Nonlinear System of Differential and Integral Calculus, ISBN 0977117006, 1979. (A detailed account of the geometric calculus.)

[12] Michael Grossman. "An introduction to non-Newtonian calculus", International Journal of Mathematical Education in Science and Technology, Volume 10, Number 4, pages 525-528, Taylor and Francis, 1979.

[13] Michael Grossman and Robert Katz, "Isomorphic calculi", International Journal of Mathematical Education in Science and Technology, Volume 15, Number 2, pages 253-263, Taylor and Francis, 1984.

[14] Michael Grossman and Robert Katz. "A new approach to means of two positive numbers", International Journal of Mathematical Education in Science and Technology, Volume 17, Number 2, pages 205-208, Taylor and Francis, 1986.

[15] Michael Grossman and Robert Katz. Non-Newtonian Calculus, ISBN 0912938013, Lee Press, 1972.
[16] James R. Meginniss. "Non-Newtonian calculus applied to probability, utility, and Bayesian analysis", American Statistical Association: Proceedings of the Business and Economic Statistics Section, pages 40 -410, 1980.
[17] "Vito Volterra". Wikipedia article (Internet).

[18] M. Rybaczuk and P. Stoppel. "The fractal growth of fatigue defects in materials", International Journal of Fracture, Volume 103, Issue 1, pages 71 - 94, Springer, 2000.

[19] Dick Stanley. "A multiplicative calculus", Primus, Volume 9, Issue 4, 1999.

[20] Jane Tang. "On the construction and interpretation of means", International Journal of Mathematical Education in Science and Technology, Volume 14, Number 1,  pages 55–57, Taylor and Francis, 1983. 

[21] "Multiplicative calculus". Wikipedia article (Internet).

[22] "Product integral". Wikipedia article (Internet).

[23] S. L. Blyumin."Discreteness versus continuity in information technologies: quantum calculus and its alternatives", Automation and Remote Control, Volume 72, Number 11, 2402-2407, DOI: 10.1134/S0005117911110142, Springer, 2011. (Russian version: S. L. Blyumin. "Discreteness versus continuity in information technologies: quantum calculus and its alternatives", Reference 13, Lipetsk State Technical University, 2008.)
[24] Mustafa Riza, Ali Ozyapici, and Emine Misirli Kurpinar. "Multiplicative finite difference methods", Quarterly of Applied Mathematics, Volume 67, pages 745 - 754, Online ISSN 1552-4485, Brown University, May 2009.
[25] Manfred Peschel and Werner Mende.The Predator-Prey Model: Do We Live in a Volterra World?, page 246, ISBN 0387818480, Springer, 1986. 
[26] R. Gagliardi. The Mathematics of the Energy Crisis, page 76, Intergalactic Publishing  Company, 1978.

[27] Ali Ozyapici and Emine Misirli Kurpinar. "Notes on Multiplicative Calculus", 20th International Congress of the Jangjeon Mathematical Society, article 67, page 80,  August 2008.

[28] Paul Dickson. The Official Rules, pages 179 and 180, ISBN 9780486797175, Courier Corporation, 2014.

[29] Horst Alzer. "Bestmogliche abschatzungen fur spezielle mittelwerte", Reference 19; Univ. u Novom Sadu, Zb. Rad. Prirod.-Mat. Fak., Ser. Mat. 23/1; 1993.

[30] V. S. Kalnitsky. "Means generating the conic sections and the third degree polynomials", Reference 7, Saint Petersburg Mathematical Society Preprint 2004-04, 2004.

[31] Methanias Colaço Júnior, Manoel Mendonça, Francisco Rodrigues. "Mining software change history in an industrial environment", Reference 20, XXIII Brazilian Symposium on Software Engineering, 2009.

[32] Nicolas Carels and Diego Frias. "Classifying coding DNA with nucleotide statistics", Reference 36, Bioinformatics and Biology Insights 2009:3, Libertas Academica, pages 141-154, 2009.

[33]  Ali Ozyapici and Emine Misirli Kurpinar. "Exponential Approximation on Multiplicative Calculus",  6th ISAAC  Congress, page 471, 2007.

[34] Jane Grossman, Michael Grossman, and Robert Katz. "Which growth rate?", International Journal of Mathematical Education in Science and Technology, Volume 18, Number 1, pages 151-154, Taylor and Francis, 1987.

[35] Michael Grossman. "Calculus and discontinuous phenomena", International Journal of Mathematical Education in Science and Technology, Volume 19, Number 5, pages 777-779, Taylor and Francis, 1988.

[36] David Malkin. "The evolutionary impact of gradual complexification on complex systems", doctoral thesis at University College London's Computer Science Department, 2009.

[37] Raymond W. K. Tang and William E. Brigham. "Transient pressure analysis in composite reservoirs", Reference 18, Stanford University: Petroleum Research Institute (with United States Department of Energy), 1982.

[38] Science Education International, International Council of Associations for Science Education, Volumes 2-3, page 24, 1991.

[39] Ciência e Cultura, Sociedade Brasileira para o Progresso da Ciência, Volume 32, Issues 5-8, page 829, 1980.

[40] American Statistical Association: 1997 Proceedings of the Section on Bayesian Statistical Science, page 176, 1997.

[41] Choice, Association of College and Research Libraries, American Library Association, Volume 9, Issues 8-12, 1972.

[42] Indian Journal of History of Science, Indian National Science Academy, Volumes 6 - 8, page 154, 1971-1973.

[43] Zentralblatt MATH, FIZ Karlsruhe.

[44] Theory and Decision, Springer, Volume 6, page 237, 1975.

[45] Kybernetika, Československá Kybernetická Společnost, Volume 9, page 155, 1973.

[46] Aplikace Matematiky, Československá Akademie Věd. Matematický Ustav, Volume 18, page 208, 1973.

[47] Mathematical Reviews, American Mathematical Society.

[48] American Mathematical Monthly, Mathematical Association of America, May of 1973.

[49] Mathematics Magazine, Mathematical Association of America, Volume 57, Number 2, page 119, 1984.

[50] ZDM, Springer.

[51] Wissenschaftliche Zeitschrift: Mathematisch-Naturwissenschaftliche Reihe, University of Leipzig, Volume 22, page 97, 1973.

[52] American Mathematical Monthly, Mathematical Association of America, June/July of 1980. 

[53] Internationale Mathematische Nachrichten, Number 105, Österreichische Mathematische Gesellschaft,  1972.

[54] Boletim da Sociedade Paranaense de Matemática, Sociedade Paranaense de Matemática.

[55] Analele științifice ale Universității "Al. I. Cuza" din Iași: Secțiunea Matematică, Universitatea "Al. I Cuza".

[56] Publicationes Mathematicae, Kossuth Lajos Tudományegyetem: Matematikai Intézet.  

[57] Nieuw Tijdschrift Voor Wiskunde, P. Noordhoff.  

[58] The Mathematics Student, Indian Mathematical Society, Volumes 53-54, page 57, 1987.

[59] L'Enseignement Mathématique, International Commission on the Teaching of Mathematics.

[60] Acta Scientiarum Mathematicarum, Institutum Bolyaianum Universitatis Szegediensis.

[61] Industrial Mathematics, Industrial Mathematics Society.

[62] Physikalische Blätter, Physik Verlag, Volume 29, page 48, 1973.

[63] "Scientia"; Rivista di Scienza, ResearchGATE, Volume 107, page 919, 1972.

[64] Science Weekly, American Association for the Advancement of Science,Volume 176, page 954, 1972.

[65] Philosophia mathematica, Canadian Society for the History and Philosophy of Mathematics, Volumes 9-14, page 96, 1972.

[66] Annals of Science, British Society for the History of Science, Volumes 29-30, page 424, 1972.

[67] Science Progress, Science Progress, Volume 60, page 428, 1972.

[68] Revue du CETHEDEC, Centre d'Etudes Théoriques de la Détection et des Communications, Volume 9, page 110, 1972.

[69] Allgemeines Statistisches Archiv, Deutsche Statistische Gesellschaft, Volumes 56-57, page 403, 1972. 

[70] Il Nuovo Cimento della Societa Italiana di Fisica: A, Societa Italiana di Fisica, page 851, 1972.

[71] Bollettino della Unione Matematica Italiana, Unione Matematica Italiana, page 289, 1972.

[72] Cahiers du Centre d'Etudes de Recherche Opérationnelle, Centre d'Etudes de Recherche Opérationnelle, Volumes 14-15, page 85, 1972.

[73] Australian Journal of Statistics. Statistical Society of Australia, Volumes 14-15, 1972.

[74] Synthese, D. Reidel Publishing Company, Volume 26, page 181, 1973.

[75] Mathematical Education, India University Grants Commission, Volume 2, 1985.

[76] Institute of Mathematical Statistics Bulletin, Institute of Mathematical Statistics, Volumes 1-2, 1972.

[77] Search, Australian & New Zealand Association for the Advancement of Science (ANZAAS), Volume 3, page 457, 1972.

[78] Ali Uzer. "Multiplicative type complex calculus as an alternative to the classical calculus", Computers & Mathematics with Applications, DOI:10.1016/j.camwa.2010.08.089, Elsevier, 2010.

[79] Praxis der Mathematik, Aulis Verlag Deubner, Volume 23, page 94, 1981.

[80] Indian Journal of Theoretical Physics, Institute of Theoretical Physics (India), Volume 31, page 176, 1983.

[81] Economic Books: Current Selections, University of Pittsburgh, Department of Economics, Volume 9, page 29, 1982.

[82] Diana Andrada Filip and Cyrille Piatecki. "A non-Newtonian examination of the theory of exogenous economic growth", hal.archives-ouvertes.fr/hal-00945781, CNCSIS - UEFISCSU (project number PNII IDEI 2366/2008) and Laboratoire d’Economie d’Orléans (LEO), Mathematica Aeterna, Volume 4, Number 2, Hilaris, 2014.

[83] Physique au Canada, Canadian Association of Physicists, Volumes 27-28, page 88, 1971.

[84] Emine Misirli and Yusuf Gurefe. "Multiplicative Adams Bashforth–Moulton methods", Numerical Algorithms, DOI: 10.1007/s11075-010-9437-2, Volume 57, Number 4, pages 425 - 439, Springer, 2011.

[85]  James D. Englehardt  and  Ruochen Li. "The discrete Weibull distribution: an alternative for correlated counts with confirmation for microbial counts in water", Risk Analysis, Volume 31, Issue 3, Pages 370–381, DOI: 10.1111/j.1539-6924.2010.01520.x, Wiley, 2011.

[86] David Baqaee. "Intertemporal choice: a Nash bargaining approach", Reserve Bank of New Zealand, Research: Discussion Paper Series, ISSN 1177-7567,  DP2010/08, September 2010.

[87] Agamirza E. Bashirov and Mustafa Riza. "Complex multiplicative calculus", arXiv.org, Cornell University, arXiv:1103.1462v1, 2011.

[88] Luc Florack and Hans van Assen. "Multiplicative calculus in biomedical image analysis", Journal of Mathematical Imaging and Vision, Volume 42, Number 1, DOI: 10.1007/s10851-011-0275-1, Springer, 2012.

[89] Robert G. Hohlfeld, Thomas W. Drueding, and John F. Ebersole. "Application of optical measure theory to atmospheric temperature sounding from TOVS radiances", U.S. Air Force Geophysics Laboratory, Atmospheric Sciences Division, GL-TR-89-0120, 1989.

[90] Wojbor Woyczynski . "Non-Newtonian calculus for the dynamics of random fractal structures: linear and nonlinear", seminar at The Ohio State University on 22 April 2011.

[91] Economic Statistics, page 41, eM Publications, Google eBook.

[92] Revue de mathématique élémentaires, Birkhäuser, 1982.

[93] Karol Kosar and Ivan Kupka. "Zovšeobecnená derivácia" ("Generalized derivative"), Student Conference, Comenius University at Bratislava in Slovakia, ISBN 978-80-89186-68-6, page 62, 2010.

[94] Agamirza E. Bashirov, Emine Misirli, Yucel Tandogdu, and Ali Ozyapici. "On modeling with multiplicative differential equations", Applied Mathematics - A Journal of Chinese Universities, Volume 26, Number 4, pages 425-428, ISSN: 1993-0445, DOI: 10.1007/s11766-011-2767-6, China Society for Industrial and Applied Mathematics, Springer, 2011.

[95] Agamirza E. Bashirov and Mustafa Riza. "On complex multiplicative differentiation", TWMS Journal of Applied and Engineering Mathematics, Volume 1, Number 1, pages 75-85, 2011.

[96] Luc Florack. "Regularization of positive definite matrix fields based on multiplicative calculus", Reference 9, Scale Space and Variational Methods in Computer Vision, Lecture Notes in Computer Science, Volume 6667/2012, pages 786-796, DOI: 10.1007/978-3-642-24785-9_66, Springer, 2012.

[97] Stanley Paul Palasek. "Nonlinear modeling and optimization of peptide delivery", Intel® International Science and Engineering Fair, PH029, Society for Science and the Public, 2011.

[98] Mohammad Ali, Anna Lena Lopez, Young Ae You, Young Eun Kim, Binod Sah, Brian Maskery, and John Clemens. "The global burden of cholera", Bulletin of the World Health Organization, Research Article ID: BLT.11.093427, United Nations, 2012.

[99] Marco Mora, Fernando Córdova-Lepe, and Rodrigo Del-Valle. "A non-Newtonian gradient for contour detection in images with multiplicative noise", Pattern Recognition Letters, Volume 33, Issue 10, pages 1245-1256, http://dx.doi.org/10.1016/j.patrec.2012.02.012, International Association for Pattern Recognition, Elsevier, 2012. 

[100] Journal of the Optical Society of America, The Optical Society, Volume 63, January of 1973.

[101] Middlesex Math Notes, Middlesex University, London, England, Volume 3, pages 47 - 50, 1977.

[102] J. I. King. "Optical measure and the non-Newtonian calculus of inverse transfer theory", private communication, 1989. (This study is used in [89].)
[103] Alex Twist and Michael Spivey. "L'Hopital's rule and Taylor's Theorem for product calculus", University of Puget Sound, 2010.
[104] Wojbor Woyczynski . "Non-Newtonian calculus for the dynamics of random fractal structures: linear and nonlinear", seminar at Cleveland State University on 02 May 2012.
[105] F. Córdova-Lepe and M. Pinto. "From quotient operation toward a proportional calculus", International Journal of Mathematics, Game Theory and Algebra, Volume 18, Number 6, pages 527-536, 2009. 
[106] Joachim Weickert, Laurent Hoeltgen, and other faculty from the Mathematical Image Analysis Group of Saarland University in Germany. University Course: "Analysis beyond Newton and Leibniz", Saarland University, Summer 2012.
[107] Ahmet Faruk Çakmak. "Some new studies on bigeometric calculus", International Conference on Applied Analysis and Algebra, Yıldız Technical University, Istanbul, Turkey, 2011.
[108] Sergey P. Verevkin, Vladimir N. Emel'yanenko, Ingo Krossing, and Roland Kalb. "Thermochemistry of ammonium based ionic liquids: tetra-alkyl ammonium nitrates - experiments and computations", The Journal of Chemical Thermodynamics, Volume 51, pages 107–113, http://dx.doi.org/10.1016/j.jct.2012.02.035, Elsevier, 2012.
[109] Bruno Ćurko. "The presence of Ruder Boskovic in the digital world", proceedings of the annual international symposium "Days of Frane Petric - From Petric to Boskovic", ISSN 1884-2236, project/theme: 191-1911112-1092, pages 143-145, The Croatian Philosophical Society in Zagreb, 2011.
[110] Mathematics Department of Eastern Mediterranean University in North Cyprus. Research Group: Multiplicative Calculus.
[111] Luc Florack. "Regularization of positive definite matrix fields based on multiplicative calculus", presented in 2011 at the Third International Conference on Scale Space and Variational Methods In Computer Vision, Ein-Gedi Resort, Dead Sea, Israel, Lecture Notes in Computer Science: 6667, ISBN 978-3-642-24784-2,  Springer, 2012.
[112] Cengiz Türkmen and Feyzi Başar. "Some basic results on the sets of sequences with geometric calculus", First International Conference on Analysis and Applied Mathematics, American Institue of Physics: Conference Proceedings, Volume 1470, pages 95-98, ISBN 978-0-7354-1077-0, doi:http://dx.doi.org/10.1063/1.4747648, 2012.
[113] Ali Ozyapici. "Non-Newtonian calculi", Master of Science thesis, Eastern Mediterranean University, Department of Mathematics, 2005.
[114] Vito Volterra and Bohuslav Hostinský. Opérations Infinitésimales Linéaires: Applications aux Equations Différentielles et Fonctionnelles, Gauthier-Villars, Paris, 1938.
[115] Michael Coco. "Multiplicative calculus", seminar at Virginia Commonwealth University's Analysis Seminar in April of 2008. 
[116] Nicholas Stern et al. "Stern Review on the Economics of Climate Change", Cambridge University Press, DRR10368, 2006. (Also, please see the review "What is Wrong with Stern?"  published by the Global Warming Policy Foundation.) 
[117] Uğur Kadak and Yusuf Gurefe. "Construction of metric spaces by using multiplicative calculus on reals", Analysis and Applied Mathematics Seminar Series, Fatih University, Mathematics Department, Istanbul, Turkey, 30 April 2012.
[118] Sunchai Pitakchonlasup, and Assadaporn Sapsomboon. "A comparison of the efficiency of applying association rule discovery on software archive using support-confidence model and support-new confidence model", Reference 13, International Journal of Machine Learning and Computing, Volume 2, Number 4, pages 517-520, International Association of Computer Science and Information Technology Press, August 2012.
[119] Ziyue Liu and Wensheng Guo. "Data driven adaptive spline smoothing": Supplement, Statistica Sinica, Volume 20, pages 1143-1163, 2010.

[120] Jarno van Roosmalen. "Multiplicative principal component analysis", Bachelor project, Eindhoven University of Technology, Netherlands, 22 April 2012.
[121] Diana Andrada Filip and Cyrille Piatecki. "A non-Newtonian examination of the theory of exogenous economic growth", 2012 seminar at Laboratoire d’Economie d’Orléans (LEO), Orléans University in France, Groupement de Recherche Européen, hal.archives-ouvertes.fr/hal-00945781, CNCSIS - UEFISCSU (project number PNII IDEI 2366/2008) and Laboratoire d’Economie d’Orléans (LEO), Mathematica Aeterna, Hilaris, 2014. 
[122] Ahmet Faruk Cakmak and Feyzi Basar. "Some new results on sequence spaces with respect to non-Newtonian calculus", Journal of Inequalities and Applications, SpringerOpen, 2012:228, doi:10.1186/1029-242X-2012-228, October of 2012.
[123] Emine Misirli and Yusuf Gurefe. "The new numerical algorithms for solving multiplicative differential equations", International Conference of Mathematical Sciences, Maltepe University, Istanbul, Turkey, 04-10 August 2009.
[124] Z. Avazzadeh, Z. Beygi Rizi, G. B. Loghmani, and F. M. Maalek Ghaini. "A numerical solution of nonlinear parabolic-type Volterra partial integro-differential equations using radial basis functions", Engineering Analysis with Boundary Elements, ISSN 0955-7997, Volume 36, Number 5, pages 881 - 893, Elsevier, 2012.   

[125] ZHENG Xu and LI Jian-Zhong. "Approximate aggregation algorithm for weighted data in wireless sensor networks", Journal of Software, Volume 23, Supplement 1, ISSN 1000-9825,  pages 108 - 119, 2012.

[126] Dorota Aniszewska and Marek Rybaczuk. "Chaos in multiplicative systems", from pages 9 - 16 in the book Chaotic Systems: Theory and Applications by Christos H. Skiadas and Ioannis Dimotikalis, ISBN 9814299715, World Scientific, 2010.
[127] Ahmet Faruk Cakmak and Feyzi Basar. "Space of continuous functions over the field of non-Newtonian real numbers", lecture at The Algerian-Turkish International Days on Mathematics, University of Badji Mokhtar at Annaba, Algeria, October of 2012.
[128] Muttalip Ozavsar and Adem Cengiz Cevikel. "Fixed points of multiplicative contraction mappings on multiplicative metric spaces", arXiv preprint arXiv:1205.5131, 2012.   
[129] Dorota Aniszewska. "Multiplicative Runge–Kutta methods", Nonlinear Dynamics, Volume 50, Numbers 1-2 ,  Springer, 2007.
[130] Dorota Aniszewska and Marek Rybaczuk. "Analysis of the multiplicative Lorenz system", Chaos, Solitons & Fractals, Volume 25, Issue 1, pages 79 - 90, Elsevier, 2005.
[131] Dorota Aniszewska and Marek Rybaczuk. "Lyapunov type stability and Lyapunov exponent for exemplary multiplicative dynamical systems", Nonlinear Dynamics, Volume 54, Issue 4, Springer, 2008.
[132] Marek Rybaczuka, Alicja Kedzia, and Witold Zielinskia. "The concept of physical and fractal dimension II - The differential calculus in dimensional spaces", Chaos, Solitons & Fractals, Volume 12, Issue 13, pages 2537 - 2552, http://dx.doi.org/10.1016/S0960-0779(00)00231-9, Elsevier, 2001.  
[133] Hatice Aktore. "Multiplicative Runge-Kutta methods", Master of Science thesis, Eastern Mediterranean University, Department of Mathematics, 2011.
[134] Christopher Olah, "Multiplicative calculus for analyzing exponential trends" , a lecture at the Singularity Summit on 13 October 2012.  
[135] Singularity Summit, 13 October 2012.
[136] Inonu University, Computer Engineering Department. Master's Degree, 2013.
[137] Zafer Cakir. "Spaces of continuous and bounded functions over the field of non-Newtonian complex numbers", lecture at The Algerian-Turkish International Days on Mathematics, University of Badji Mokhtar at Annaba, Algeria, October of 2012.
[138] The 2013 Algerian-Turkish International Days on Mathematics conference, Fatih University, Istanbul, Turkey, 12-14 September 2013.
[139] Gordon Mackay. Comparative Metamathematics, ISBN: 978-0557249572, 2011.
[140] Ali Ozyapici. "Alternatives to the classical calculus: multiplicative calculi", Dokuz Eylul University (Turkey), Mathematics-Department Seminar, 25 December 2008.
[141] Daniel Karrasch. "Hyperbolicity and invariant manifolds for finite time processes", doctoral thesis, Technical University of Dresden in Germany, 2012.
[142] Riswan Efendi, Zuhaimy Ismail, and Mustafa Mat Deris. "Improved weight fuzzy time series as used in the exchange rates forecasting of US dollar to ringgit Malaysia", International Journal of Computational Intelligence and Applications, DOI: 10.1142/S1469026813500053, Volume 12, Issue 01, Imperial College Press, March 2013.
[143] Antonin Slavik. Product Integration, Its History and Applications, ISBN 80-7378-006-2, Matfyzpress, Prague, 2007.
[144] Sebiha Tekin and Feyzi Basar. "Certain sequence spaces over the non-Newtonian complex field", Abstract and Applied Analysis (2013), 1--11, doi:10.1155/2013/739319, http://projecteuclid.org/euclid.aaa/1393511955, Project Euclid, Hindawi, 2013.
[145] Agamirza E. Bashirov. "On line integrals and double multiplicative integrals", TWMS Journal of Applied and Engineering Mathematics, Volume 3, Number 1, pages 103 - 107, 2013.
[146]  Wojbor Woyczynski . "Fractional calculus for random fractals", seminar at Case Western Reserve University on 03 April 2013. 
[147] P. Arun Raj Kumar and S. Selvakumar. "Detection of distributed denial of service attacks using an ensemble of adaptive and hybrid neuro-fuzzy systems", Computer Communications, Volume 36, Issue 3, pages 303 - 319, http://dx.doi.org/10.1016/j.comcom.2012.09.010, Elsevier, February of 2013.
[148] Hatice Aktore and Mustafa Riza. "Complex multiplicative Runge-Kutta method", International Conference on Applied Analysis and Algebra, Yıldız Technical University, Istanbul, Turkey, 2012.
[149] Diana Andrada Filip and Cyrille Piatecki. "In defense of a non-Newtonian economic analysis", hal.archives-ouvertes.fr/hal-00945782, CNCSIS – UEFISCSU (Babes-Bolyai University of Cluj-Napoca, Romania) and  LEO (Orléans University, France), 2014.
[150] M. Jahanshahi, N. Aliev, and H. R. Khatami. "An analytic-numerical method for solving difference equations with variable coefficients by discrete multiplicative integration", Dynamical Systems and Applications, Proceedings, pages 425—435,  World Scientific, July 2004.      

[151] H. R. Khatami, M. Jahanshahi, and N. Aliev. "An analytical method for some nonlinear difference equations by discrete multiplicative differentiation", Dynamical Systems and Applications, Proceedings, pages 455—462,  World Scientific, July 2004.         

[152] N. Aliev, N. Azizi, and M. Jahanshahi. "Invariant functions for discrete derivatives and their applications to solve non-homogenous linear and non-linear difference equations", International Mathematical Forum, Volume 2, Number 11, pages 533–542, Hikari Ltd, 2007.      

[153] Methanias Colaco Rodrigues Junior. "Identificacao E Validacao Do Perfil Neurolinguistic O De Programadores Atraves Da Mineracao De Repositorios De Engenharia De Software", thesis, Multiinstitutional Program in Computer Science: Federal University of Bahia (Brazil), State University of Feira de Santana (Brazil), and Salvador University (Brazil), IEVDOP neurolinguistic - repositorio.ufba.br, 2011.

[154] Jared Burns. "M-Calculi: Multiplying and Means", graduate-seminar, University of Pittsburgh, Mathematics Department, 13 December 2012.

[155] Gunnar Sparr. "A Common Framework for Kinetic Depth Reconstruction and Motion for Deformable Objects", Lecture Notes in Computer Science, Volume 801, Springer, Proceedings of the Third European Conference on Computer Vision, Stockholm, Sweden, pages 471-482, May of 1994.

[156] Jie Zhang, Li Li, Luying Peng, Yingxian Sun, Jue Li. "An Efficient Weighted Graph Strategy to Identify Differentiation Associated Genes in Embryonic Stem Cells", PLoS ONE, 8(4): e62716, doi:10.1371/journal.pone.0062716, April of 2013.

[157] Agamirza E. Bashirov and Mustafa Riza. "On Complex Multiplicative Integration", arXiv.org, Cornell University, arXiv:1307.8293, 2013.

[158] Ali Uzer. "Exact solution of conducting half plane problems in terms of a rapidly convergent series and an application of the multiplicative calculus", Turkish Journal of Electrical Engineering & Computer Sciences, DOI: 10.3906/elk-1306-163, 2013.

[159] Zafer Cakir. "Spaces of continuous and bounded functions over the field of geometric complex numbers", Journal of Inequalities and Applications, Volume 2013:363, doi:10.1186/1029-242X-2013-363, ISSN 1029-242X, Springer, 2013.

[160] Ugur Kadak, Feyzi Basar, and Hakan Efe. "Construction of the duals of classical sequence spaces with respect to non-Newtonian calculus on reals", Fatih University, www.fatih.edu.tr/~akaya/ABSTRACTS.pdf#page=66, 2013.

[161] Ahmet Faruk Çakmak and Feyzi Basar. "Certain spaces of functions over the field of non-Newtonian complex numbers", Abstract and Applied Analysis, Volume 2014, Article ID 236124, http://dx.doi.org/10.1155/2014/236124, Hindawi Publishing Corporation, 2014.

[162] Ali Ozyapici and Bulent Bilgehan. "Applications of multiplicative calculus to exponential signal processing", Algerian Turkish International Days on Mathematics, http://www.fatih.edu.tr/~akaya/ABSTRACTS.pdf#page=174, 2013.

[163] Martin Ostoja-Starzewski. "The inner workings of fractal materials", Media-Upload, University of Illinois at Urbana-Champaign, http://web.mechse.illinois.edu/media/uploads/web_sites/82/files/p18_20_martin_ostoja_starzewski_hi_res.20130904.52279b9f04d249.63951844.pdf, 2013.

[164] Christopher Olah. Christopher Olah's Blog, 07 May 2013.

[165] Andrew Orlowski. "Was Stern 'wrong for the right reasons' ... or just wrong?", The Register, 4 September 2012.

[166] Jérôme Gateau, Nicolas Taccoen, Mickaël Tanter, and Jean-François Aubry. "Statistics of acoustically induced bubble-nucleation events in in-vitro blood: a feasibility study", Ultrasound in Medicine and Biology,  ISSN: 0301-5629, Volume 39, Issue 10, pages 1812-25, doi: 10.1016/j.ultrasmedbio.2013.04.011, Elsevier, October 2013.

[167] The International Journal on Recent Trends in Life Science and Mathematics (IJLSM), ijlsm.org., 2013.

[168] AbeBooks.com, book-search/title/integral-calculus/first-edition/page-1/, 2013.

[169] Mustafa Riza and Hatice Aktore. "The Runge-Kutta method in geometric multiplicative calculus", LMS Journal of Computation and Mathematics, Volume 18, Issue 01, DOI: http://dx.doi.org/10.1112/S1461157015000145, London Mathematical Society, Cambridge University Press, 2015.

[170] Amazon.com, Non-Newtonian Calculus, 2006.

[171] Bulent Bilgehan. "Finite product representation via multiplicative calculus in signal processing", The International Symposium on Engineering, Artificial Intelligence and Applications, Girne American University in Cyprus/Turkey, November, 2013.

[172] William Campillay and Manuel Pinto. "Proportional-Differential Equations", VIII Congreso de Análisis Funcional y Ecuaciones de Evolución, Universidad de Santiago de Chile, November, 2013.

[173] Joachim Weickert. University Course CS 101: Differential Equations in Image Processing and Computer Vision, Saarland University in Germany, Summer 2012.

[174] Institute of Mathematics of the Jagiellonian University in Poland. "Proposed Topics for the Master's Degree", since 2010.

[175] Ali Ozyapici. Seminar: "Applications of Multiplicative Calculi to Economical and Numerical Problems", Girne American University in Cyprus, December of 2013.

[176] Ali Özyapıcı, Mustafa Riza, Bülent Bilgehan, and Agamirza E. Bashirov. "On multiplicative minimization methods", Numerical Algorithms, ISSN: 1017-1398 (Print), ISSN: 1572-9265 (Online), DOI 10.1007/s11075-013-9813-9, Springer, December of 2013.

[177] Christopher Olah. Christopher Olah's Blog, "Alien Mathematics, Numbers, and Polynomial Centric Societies", 10 June 2011.

[178] Mustafa Riza and Bugce Eminaga . "Bigeometric calculus - a modelling tool", arXiv.org, Cornell University, arXiv:1402.2877, 2014.

[179] Agamirza Bashirov. Mathematical Analysis: Fundamentals, Elsevier: Academic Press, DOI: 10.1016/B978-0-12-801001-3.00013-5, ISBNs 0128010509 and 9780128010501, 2014.

[180] S. L. Blyumin. "Binary arithmetic operations - Functional Equations" (Russian:"БИНАРНЫЕ АРИФМЕТИЧЕСКИЕ ОПЕРАЦИИ"), http://www.stu.lipetsk.ru/files/materials/2415/2008_01_010.pdf, Mathematical Modelling, 38, UDC 512.534.1, (Russian: МАТЕМАТИЧЕСКОЕ МОДЕЛИРОВАНИЕ, 38, УДК,512.534.1), News of Universities Chernozemya, Number 1 (11), Lipetsk State Technical University, 2008.

[181] Diana Andrada Filip and Cyrille Piatecki. "An overview on non-Newtonian calculus and its potential applications to economics", halshs.archives-ouvertes.fr/docs/00/94/57/88/PDF/nncam.pdf, Applied Mathematics - A Journal of Chinese Universities, Volume 28,  China Society for Industrial and Applied Mathematics, Springer, 2014.

[182] Xiaoju He, Meimei Song, and Danping Chen. "Common fixed points for weak commutative mappings on a multiplicative metric space", Fixed Point Theory and Applications,  http://www.fixedpointtheoryandapplications.com/content/2014/1/48, Springer, 2014. 

[183] Ugur Kadak. "Determination of the Kothe-Toeplitz duals over the non-Newtonian complex field", National Center for Biotechnology Information, PMC4085728,  Scientific World Journal, doi:10.1155/2014/438924‎, 2014.

[184] Dorota Aniszewska and Marek Rybaczuk. "Fractal characteristics of defects evolution in parallel fibre reinforced composite in quasi-static process of fracture", Theoretical and Applied Fracture Mechanics, Elsevier, 2009.

[185] Ali Uzer. "Improvement of the diffraction coefficient of GTD [geometrical theory of diffraction] by using multiplicative calculus", IEEE Transactions on Antennas and Propagation, Volume 62, Issue 7, ISSN 0018-926X, DOI 10.1109/TAP.2014.2319852, IEEE Antennas and Propagation Society, July of 2014.

[186] Uğur Kadak, Feyzi Başar & Hakan Efe. “Construction of the duals of classical sets of sequences and related matrix transformations with non-Newtonian calculus”, Algerian Turkish International Days on Mathematics, Fatih University in Turkey, 2013.  

[187] Ugur Kadak. "Non-Newtonian analysis and its applications", doctoral thesis, Gazi University in Turkey, http://websitem.gazi.edu.tr/site/ugurkadak/posts/view/id/81101, 2011.

[188] Khiord Boruah. "Difference sequence spaces and non-Newtonian calculus", National Conference on Recent Trends of Mathematics and its Applications, RTMA-2014, Rajiv Gandhi University in India, May of 2014.

[189] Ugur Kadak and Hakan Efe. "Matrix transformations between certain sequence spaces over the non-Newtonian complex field", National Center for Biotechnology Information, PMC4090579, Scientific World Journal, doi:10.1155/2014/705818, 2014.

[190] Ugur Kadak and Hakan Efe. "The construction of Hilbert spaces over the non-Newtonian field", International Journal of Analysis, http://www.researchgate.net/publication/262919697_The_construction_of_Hilbert_spaces_over_the_non-Newtonian_field, Hindawi Publishing Corporation, 2014.

[191] Ali Ozyapici. "Multiplicative calculus and its applications", doctoral thesis, Ege University in Turkey, http://www.yarbis.yildiz.edu.tr/common/uploads/242578d8ce/TezGosterdoktora.pdf, 2009.

[192] David Godes. "Product policy in markets with word-of-mouth communication", Social Science Research Network, SSRN-id2275617, 2013.

[193] Duff Campbell. "Multiplicative calculus and student projects", Primus, Volume 9, Issue 4, 1999.

[194] Paolo Perrone. "A gauge theoretic approach to quantum physics", INFN-Istituto Nazionale di Fisica Nucleare (Italian National Institute for Nuclear Physics), www.infn.it/thesis/PDF/getfile.php?filename=8193--laurea.pdf, 2013.

[195] Luc Florack. "Neuro and cardio imaging", BIRS Workshop, 11w5018, Banff International Research Station for Mathematical Innovation and Discovery, Banff (Alberta, Canada), 2011.

[196] H. Vic Dannon. "Power Means Calculus: Product Calculus, Harmonic Mean Calculus, and Quadratic Mean Calculus", Gauge Institute Journal of Math and Physics, Volume 4, No 4, 2008.

[197] H. Vic Dannon. Power Means Calculus and Fractional Calculus; Gauge Institute; ISBN-10: 098012879X; ISBN-13: 978-0980128796; 2011.

[198] Mathematica. "Define product derivative",  Mathematica Stack Exchange, 2014.

[199] Luc Florack. "Regularization of positive definite matrix fields based on multiplicative calculus"; from pages 786 - 796 in the book Scale Space and Variational Methods in Computer Vision: Third International Conference (Ein-Gedi, Israel, 29 May - 2 June 2011) - Revised Selected Papers by Alfred M. Bruckstein, Bart ter Haar Romeny, Alexander M. Bronstein, Michael M. Bronstein; SSVM 2011; Springer Science & Business Media; 2012.

[200] Emine Misirli and Ali Ozyapici. "Exponential approximations on multiplicative calculus", Proceedings of the Jangjeon Mathematical Society, Volume 12, Number 2, ISSN 1598-7264, pages 227 - 236, 2009.

[201] Ahmet Faruk Çakmak and Feyzi Başar. "On line and double integrals in the non-Newtonian sense", 2014 International Conference on Analysis and Applied Mathematics, ISBN: 978-0-7354-1247-7,  American Institute of Physics Conference Proceedings, Volume 1611, http://dx.doi.org/10.1063/1.4893869, 2014.

[202] Mohammad Jahanshahi. "Discrete additive and multiplicative calculus for making the mathematical models of discrete physical and natural phenomena", 3rd International Conference on 21st Century Mathematics, School of Mathematical Sciences at Government College University, Lahore, Pakistan, 2007.

[203] Wacław Kasprzak, Bertold Lysik, Marek Rybaczuk. "Measurements, Dimensions, Invariant Models and Fractals", Ukrainian Society on Fracture Mechanics, Wroclaw University of Technology, SPOLOM, ISBNs  9666652145 and 9789666652143, 2004.

[204] Ahmet Faruk Cakmak and Feyzi Basar. “Some sequence spaces and matrix transformations in multiplicative sense”, lecture at the Çankırı Karatekin University Mathematics Days, Çankırı, Turkey, 2014.

[205] Ahmet Faruk Çakmak. "Some new sequence spaces over a new field", doctoral thesis, Yıldız Technical University in Turkey, http://www.yarbis.yildiz.edu.tr/common/uploads/1881fb8e20/AFCResume.pdf, 2014. 

[206] Muhammad Sarwar and Badshah-e-Rome. "Some unique fixed point theorems in multiplicative metric space", arXiv.org, Cornell University, arXiv:1410.3384v1, 2014.

[207] Yusuf Gurefe and Emine Misirli. "New Runge-Kutta methods for numerical solutions of multiplicative initial value problems", 2014 International Conference on Recent Advances in Pure and Applied Mathematics at Antalya in Turkey, Abstract Book of ICRAPAM 2014, Page 121, ISBN:978-975-00211-1-4, http://www.icrapam.org/ICRAPAMAbsBook.pdf, 2014.

[208] Yusuf Gurefe. "Multiplicative differential equations and applications", 2013 doctoral Thesis, Ege University in Turkey, Abstract Book of ICRAPAM 2014, Page 121, ISBN:978-975-00211-1-4, http://www.icrapam.org/ICRAPAMAbsBook.pdf, 2014..

[209] Tolgay Karanfiller. "Numerical solution of non-linear equations via multiplicative calculus",  2014 International Conference on Recent Advances in Pure and Applied Mathematics at Antalya in Turkey, Abstract Book of ICRAPAM 2014, Page 149, ISBN:978-975-00211-1-4, http://www.icrapam.org/ICRAPAMAbsBook.pdf, 2014.

[210] Ugur Kadak and Muharrem Ozluk. "Generalized Runge-Kutta method with respect to non-Newtonian calculus", Abstract and Applied Analysis, Volume 2015, Article ID 594685, Hindawi Publishing Corporation, 2015.

[211] Xiaohong Gong, Yali Zhou, Hao Zhou, and Yinfei Zheng. "Ultrasound image edge detection based on a novel multiplicative gradient and Canny operator", Ultrasonic Imaging, uix.sagepub.com, doi: 10.1177/0161734614554461, Sage Journals, 2014.

[212] Abdourrahmane Mahamane Atto, Emmanuel Trouve, Jean Marie Nicolas. "Geometric wavelet approximations and differencing", https://hal.archives-ouvertes.fr/hal-00950823/document, 2014.

[213] Oratai Yamaod and Wutiphol Sintunavarat. "Some fixed point results for generalized contraction mappings with cyclic (α,β)-admissible mapping in multiplicative metric spaces", Journal of Inequalities and Applications, 2014:488, SpringerOpen, 2014.

[214] M. Abbas, B. Ali, and I.Y. Suleiman. "Common fixed points of locally contractive mappings in multiplicative metric spaces with application", International Journal of Mathematics and Mathematical Sciences, Volume 2015 (2015), Article ID 218683, http://dx.doi.org/10.1155/2015/218683, Hindawi Publishing Corporation, 2015. 

[215] Mustafa Riza and Bugce Eminaga. "Bigeometric Calculus and Runge Kutta Method", arXiv.org, Cornell University, arXiv:1402.2877v2, 2015. 

[216] Diana Andrada Filip and Cyrille Piatecki. "In defense of a non-Newtonian economic analysis through an accounting paradigm", World Economics Association, February of 2015.

[217] Ahmet Faruk Cakmak and Feyzi Basar. “Some sequence spaces and matrix transformations in multiplicative sense”, TWMS Journal of Applied and Engineering Mathematics, Volume 6, Number 1, pages 27-37, 2015.

[218] R. C. Mittal. ResearchGate website, 12 November 2014.

[219] Jaafar Anwar H. Ameen. "Continuous nowhere differentiable functions", master's thesis, Eastern Mediterranean University in North Cyprus, 2014.

[220] Ali Ozyapici. "Non-Newtonian calculi", master's thesis, Eastern Mediterranean University in North Cyprus, 2005.

[221] Hatice Aktore. "Multiplicative Runge-Kutta methods", master's thesis, Eastern Mediterranean University in North Cyprus, 2011.

[222] Bulent Bilgehan. "Efficient approximation for linear and non-linear signal representation", IET Signal Processing, DOI: 10.1049/iet-spr.2014.0070, Online ISSN 1751-9683, Institution of Engineering and Technology (IET), Spring 2015.

[223] Christoph von Hagke. "Coupling between climate and tectonics?", doctoral thesis, Freie Universität Berlin in Germany, 2012.

[224] Eric Gaze. Keynote Speech at the 27th International Conference on Technology in Collegiate Mathematics (ICTCM), Pearson PLC, YouTube,  YouTube, March of 2015.

[225] Ali Ozyapici and Bülent Bilgehan. "Finite product representation via multiplicative calculus and its applications to exponential signal processing", Numerical Algorithms, ISSN: 1017-1398 (Print), ISSN: 1572-9265 (Online), DOI 10.1007/s11075-015-0004-8, Springer, May of 2015.

[226] Ugur Kadak, Murat Kirişci, and Ahmet Faruk Cakmak. "On the classical paranormed sequence spaces and related duals over the non-Newtonian complex field", Journal of Function Spaces, Hindawi Publishing Company, May of 2015.  

[227] James D. Englehardt. "Distributions of autocorrelated first-order kinetic outcomes: illness severity", PLOS ONE, DOI: 10.1371/journal.pone.0129042, Public Library of Science (PLOS), 2015. 

[228] Dorota Aniszewska and Marek Rybaczuk. "Physical stability and critical effects in models of fractal defects evolution based on single fractal approximation", Chaos, Solitons & Fractals, Volume 32, Issue 1, Elsevier, 2007.

[229] Mujahid Abbas, Manuel De la Sen, and Talat Nazir. "Common fixed points of generalized rational type cocyclic mappings in multiplicative metric- spaces", Discrete Dynamics in Nature and Society, Volume 2015, Article ID 532725, http://dx.doi.org/10.1155/2015/532725, Hindawi Publishing Corporation,  2015.

[230] Ugur Kadak. “Non-Newtonian fuzzy numbers”, Iranian Journal of Fuzzy Systems, http://ijfs.usb.ac.ir/article_2114_0.html, University of Sistan and Baluchestan in Iran, 2015.

[231] Luc Florack, Tom Dela Haije, and Andrea Fuster. "Direction-controlled DTI interpolation", from pages 149 - 162 in the book Visualization and Processing of Higher Order Descriptors for Multi-Valued Data by Ingrid Hotz and Thomas Schultz, ISBN 3319150901, Springer, 2015.

[232] Yusuf Gurefe, Ugur Kadak, Emine Misirli, and Alia Kurdi. “A new look at the classical sequence spaces by using multiplicative calculus”, UPB Scientific Bulletin, ISSN 1223-7027, Series A, Volume 78, Issue 2, 2016.

[233] Mustafa Riza, Hatice Aktore, and Bugce Eminaga. "A modified quadratic Lorenz attractor in geometric multiplicative calculus", 28th International Conference of the Jangjeon Mathematical Society, Antalya, Turkey, May of 2015.

[234] Chirasak Mongkolkeha and Wutiphol Sintunavarat. "Best proximity points for multiplicative proximal contraction mapping on multiplicative metric spaces", The Journal of Nonlinear Science and Applications, Print-ISSN 2008-1898, Online-ISSN 2008-1901, http://www.emis.de/journals/TJNSA/includes/files/articles/user_articles/20150203173148_8362, June of 2015.

[235]  Cenap Duyar, Birsen Sagir, and Oguz Ogur. "Some basic topological properties on non-Newtonian real line", British Journal of Mathematics & Computer Science, 9(4): XX-XX, Article no.BJMCS.2015.204, ISSN: 2231-0851, DOI: 10.9734/BJMCS/2015/17941, 2015.

[236] Raymond A. Guenther. "Product integrals and sum integrals", International Journal of Mathematical Education in Science and Technology, Volume 14, Issue 2, Taylor & Francis, 1983, published online in July of 2006.

[237] Bugce Eminaga, Hatice Aktore, and Mustafa Riza. "A modified quadratic Lorenz attractor", arXiv.org, Cornell University, arXiv:1508.06840v1, 2015.

[238] Marek Rybaczuk. "Critical growth of fractal patterns in biological systems", Acta of Bioengineering and Biomechanics, Volume 1, Number 1, Wroclaw University of Technology, 1999.

[239] Justin Webster. MATH 120: Introductory Calculus course, Project Assignment, College of Charleston, 2015.


[240] Nico Persch, Christopher Schroers, Simon Setzer, and Joachim Weickert. "Introducing more physics into variational depth–from–defocus", from pages 15 - 27 in the book Lecture Notes in Computer Science, Volume 8753 2014, ISBN 9783319117515, DOI 10.1007/978-3-319-11752-2, Springer, 2014; Proceedings, Pattern Recognition: 36th German Conference, GCPR 2014, Münster, Germany, September 2-5, 2014, Münster, Germany, September 2-5, 2014.

[241] Nico Persch, Christopher Schroers, Simon Setzer, and Joachim Weickert. "Physically inspired depth-from-defocus", Universitat des Saarlandes, Fachrichtung 6.1 - Mathematik, Preprint No. 355, February of 2015.

[242] Shin Min Kang, Poonam Nagpal, Sudhir Kumar Garg, and Sanjay Kumar."Fixed points for multiplicative expansive mappings in multiplicative metric space", International Journal of Mathematical Analysis, Volume 9, Number 39, www.m-hikari.com, http://dx.doi.org/10.12988/ijma.2015.54130, Hikari Ltd., 2015.

[243] Shin Min Kang, Parveen Kumar, Sanjay Kumar, and Bu Young Lee. "Common fixed points for compatible mappings of types in multiplicative metric spaces", International Journal of Mathematical Analysis, Volume 9, Number 36, www.m-hikari.com, http://dx.doi.org/10.12988/ijma.2015.53104, Hikari Ltd., 2015.

[244] Parveen Kumar, Sanjay Kumar, and Shin Min Kang. "Common fixed point theorems for subcompatible and occasionally weakly compatible mappings in multiplicative metric spaces", International Journal of Mathematical Analysis, Volume 9, Number 36, http://dx.doi.org/10.12988/, Hikari Ltd, 2015.

[245] Riswan Efendi, Zuhaimy Ismail , Nor Haniza Sarmin, and Mustafa Mat Deris. "A reversal model of fuzzy time series in regional load forecasting", International Journal of Energy and Statistics, Volume 03, Issue 01, DOI: 10.1142/S2335680415500039, World Scientific, 2015.

[246] Riswan Efendi, Zuhaimy Ismail, and Mustafa Mat Deris. "A new linguistic out-sample approach of fuzzy time series for daily forecasting of Malaysian electricity load demand", Applied Soft Computing, Volume 28, Issue C, Doi: 10.1016/j.asoc.2014.11.043, Elsevier, 2015.
[247] Christoffel Wilhelmus Janse Rensburg. "The relationship between process maturity models and the use and effectiveness of systems development methodologies", master-of-science dissertation, North-West University at Potchefstroom in South Africa, 2012.
[248] Hong-Kyu Kim, Mirim Lee, Kwang-Ryeol Lee, and Jae-Chul Lee. "How can a minor element added to a binary amorphous alloy simultaneously improve the plasticity and glass-forming ability?", Acta Materialia, Volume 61, Issue 17, ISSN: 1359-6454, Elsevier, 2013.
[249] Feng Gu and Yeol-Je Cho. "Common fixed point results for four maps satisfying ϕ-contractive condition in multiplicative metric spaces", Fixed Point Theory and Applications, Volume 2015:165,  doi:10.1186/s13663-015-0412-4, Springer, 2015.
[250] Bülent Bilgehan, Buğçe Eminağa, and Mustafa Riza. "New solution method for electrical systems represented by ordinary differential equation", Journal of Circuits, Systems and Computers, Volume 25, Issue 02, DOI: 10.1142/S0218126616500110, World Scientific, 2016.
[251] Thabet Abdeljawad. "On multiplicative fractional calculus", arXiv.org, Cornell University, arXiv:1510.04176, 2015.
[252] Muhammad Usman Ali. "Caristi mapping in multiplicative metric spaces", Science International (Lahore), 27(5), 3917-3919, ISSN:1013-5316, CODEN: SINTE 8, 2015. 
 
[253] Ugur Kadak. "Cesaro summable sequence spaces over the non-Newtonian complex field", Journal of Probability and Statistics, Hindawi Publishing Corporation, October of 2015.
[254] Demet Binbaşıoǧlu, Serkan Demiriz, and Duran Türkoǧlu. "Fixed points of non-Newtonian contraction mappings on non-Newtonian metric spaces", Journal of Fixed Point Theory and Applications, Springer, November of 2015. 
[255] Parveen Kumar, Sanjay Kumar, and Shin Min Kang. "Common fixed points for weakly compatible mappings in multiplicative metric spaces", International Journal of Mathematical Analysis, Volume 9, Number 42, http://dx.doi.org/10.12988/ijma.2015.56162, Hikari Ltd., 2015.
[256] Leonid G. Kreidik. "Additive and multiplicative judgements of dialectical logic. Additive and multiplicative differentials and integrals of dialectical judgements", Journal of Theoretical Dialectics-Physics-Mathematics, Issue [B-01], Dialectical Academy in Russia-Belarus, 2005.
[257] Clement Boateng Ampadu. "Fixed point theorems in multiplicative soft metric spaces and multiplicative quasi-dislocated soft metric spaces", American Mathematical Society meeting, California State University at Fullerton, 24 October 2015.
[258] Marek Rybaczuk. "Fractal Models of Defects Evolution", 2004 South African Conference on Applied Mechanics (Johannesburg, South Africa), January of 2004. 
[259] Amelia Correa and Romar Correa. "Accounting for Financialization: Stock-Flow-Consistent Political Economy", World Review of Political Economy, Volume 6, Number 2 , Pluto Journals,  DOI: 10.13169/worlrevipolieco, 2015.
[260] Hui-hui Zheng and Feng Gu. "Multiplicative metric space two pairs self-mapping common fixed-point theorem" (乘积度量空间中两对自映象的公共不动点定理), Journal of Hangzhou Normal University (Natural Science), Volume 14, Number 4, doi:10.3969/j, issn.1674-232X, 2015.
[261] K. Abodayeh, A. Pitea, W. Shatanawi, and T. Abdeljawad. "Remarks on multiplicative metric spaces and related fixed points", arXiv.org, arXiv:1512.03771v1, 2015.
[262] Yumnam Rohen, Laishram Shanjit, and P.P. Murthy. "Fixed points of R-weakly commuting mappings in multiplicative metric space", The 11th International Conference on Fixed Point Theory And Its Applications, Galatasaray University and Atilim University in Turkey, abstracted by Topology Atlas at York University in Toronto, Canada, July of 2015.
[263] Parveen Kumar, Sanjay Kumar, and Shin Min Kang. "Common fixed points for intimate mappings in multiplicative metric spaces", International Journal of Pure and Applied Mathematics, Volume 103, Number 4, ISSN:1311-8080 (printed version), ISSN:1314-3395 (on-line version), 2015.
[264] Helena Jasiulewicz, Wojciech Kordecki. "Additive versus multiplicative parameters-applications in economics and finance", arXiv.org, Cornell University, arXiv:1306.4994, 2013.
[265] Helena Jasiulewicz, Wojciech Kordecki. "Multiplicative parameters and estimators: applications in economics and finance", Annals of Operations Research, DOI: 10.1007/s10479-015-2035-x, Springer, October of 2015.
[266] Norman Zacharias, Cezary Sieluzycki, Wojciech Kordecki, Reinhard Konig, and Peter Heil. "The M10 0 component of evoked magnetic fields differs by scaling factors: Implications for signal averaging", Psychophysiology, Vol 48, Issue 8, DOI:10.1111/j.1469-8986.2011.01183.x, Wiley Periodicals, Society for Psychophysiological Research, 2011. 
[267] Gokhan Yener and Ibrahim Emiroglu. "A q-analogue of the multiplicative calculus: q-multiplicative calculus", Discrete and Continuous Dynamical Systems - Series S (DCDS-S), Volume 8, Number 6, doi:10.3934/dcdss.2015.8.1435, American Institute of Mathematical Sciences (AIMS), December of 2015.
[268] Romulus Terebes, Monica Borda, Christian Germain, Raul Malutan, and Ioana Iles. "A multiplicative gradient-based anisotropic diffusion approach for speckle noise removal", E-Health and Bioengineering Conference (EHB), IEEE Conference Publications, Institute of Electrical and Electronics Engineers, XPlore Digital Library, DOI: 10.1109/EHB.2015.7391605, 2015.
[269] Ivan Kupka. "Topological generalization of Cauchy's mean value theorem", Annales Academiae Scientiarum Fennicae, Volume 41, Issue 1, doi:10.5186/aasfm.2016.4120, Academia Scientiarum Fennica (in Finland), 2016.
[270] Thabet Abdeljawad. "On geometric fractional calculus", Journal of Semigroup Theory and Applications, Volume 2016 (2016), Article ID 2, ISSN: 2051-2937, 2016.
[271] Kiyoko Tateishi, Yusaku Yamaguchi, Omar M. A. Al-Ola, Takeshi Kojima, and Tetsuya Yoshinaga. "Continuous analog of multiplicative algebraic reconstruction technique for computed tomography", SPIE conference at San Diego in California, 27 February - 3 March 2016, Paper 9783-168, Proceedings SPIE 9783, Medical Imaging 2016: Physics of Medical Imaging, 97834Q, doi: 10.1117/12.2214598, 2016.
[272] Yusaku Yamaguchi, Takeshi Kojima, and Tetsuya Yoshinaga. "Noise reduction in computed tomography using a multiplicative continuous-time image reconstruction method", SPIE conference at San Diego in California, 27 February - 3 March 2016, Paper 9783-171, Proceedings SPIE 9783, Medical Imaging 2016: Physics of Medical Imaging, 97834T, doi: 10.1117/12.2216439, 2016. 
[273] Alex Lyubomirskiy. "Ideality equation", The Triz Journal, 3 April 2016. (“TRIZ” is the Russian acronym for the “Theory of Inventive Problem Solving.” G.S. Altshuller and his colleagues in the former U.S.S.R. developed the method between 1946 and 1985.)
[274] Mrinmoy Majumder. Minimization of Climatic Vulnerabilities on Mini-hydro Power Plants: Fuzzy AHP, Fuzzy ANP Techniques and Neuro-Genetic Model Approach, ISBNs  9812873147 and 9789812873149, Springer, 2016.
[275] Premakh. Alimony Defense: A Complete Guide, AnVi OpenSource Knowledge Trust, e-book, 2016.
[276] Khirod Boruah and Bipan Hazarika. "Application of geometric calculus in numerical analysis and difference sequence spaces", arXiv.org, Cornell University,  arXiv:1603.09479, 2016.
[277] Khirod Boruah, Bipan Hazarika, and Mikail Et. "Generalized geometric difference sequence spaces and its duals", Cornell University, arXiv:1603.09497v1, 2016.
[278] Zakaria Adnan. "An analysis of Runge-Kutta method in non-Newtonian calculus", doctoral dissertation, Kwame Nkrumah University of Science and Technology in Ghana, ir.knust.edu.gh, 2016.
[279] Michael Valenzuela. "Machine learning, optimization, and anti-training with sacrificial data", doctoral dissertation, University of Arizona in the United States, http://arizona.openrepository.com/arizona/handle/10150/605111, 2016.
[280] Christos Liaskos and Ageliki Tsioliaridou. "Service ratio-optimal, content coherence-aware data push systems", TMIS (Transactions on Management Information Systems), Volume 6, Issue 4, Article 15, ACM (Association for Computing Machinery), January of 2016.

[281] Orhan Tug and Feyzi Basar. "On the spaces of Norlund null and Norlund convergent sequences", TWMS Journal of Applied and Engineering Mathematics, Volume 7, Number 1, pages 76-87, 2016.

[282] Peter D. Lax. "The Flowering of applied Mathematics in America", Siam Review, Volume 31, Number 4, DOI:10.1137/1031123, December of 1989.

[283] N.A. Aliyev and R.G. Ahmadov. "Investigation of the solutions of the Cauchy problem and boundary-value problems for the ordinary differential equations with continuously changing order of the derivative", arXiv.org, Cornell University, arXiv:1605.06601v1, 2016.

[284] Mohsen Ghafory-Ashtiany, and Naghmeh Pakdel-Lahiji. "Developing seismic vulnerability curves for typical Iranian buildings", Proceedings of the Institution of Mechanical Engineers, Journal of Risk and Reliability,  Volume 229, Number 6, Sage Journals, December of 2015.

[285] Leonid G. Kreidik and George Shpenkov. Alternative Picture of the World, Volume 1, Published by George Shpenkov, Institute of Mathematics & Physics at the University of Technology & Agriculture (UTA) in Bydgoszcz, Poland, 1996.
[286] M. Jahanshahi and N. Aliev. "Discrete additive and multiplicative calculus in graduate mathematics and their applications for solving difference equations", The Mathematics Education into the 21st Century Project, Proceedings of the 10th International Conference: "Models in Developing Mathematics Education", University of Applied Sciences in Dresden, Germany, September 11-17, 2009.

[287] James Englehardt, Jeff Swartout, and Chad Loewenstine. "A new theoretical discrete growth distribution with verification for microbial counts in water", Risk Analysis, Volume 29, Number 6, DOI: 10.1111/j.1539-6924.2008.01194.x, Wiley, 2009.
[288] Dorota Aniszewska and Marek Rybaczuk. "Multiplicative Hénon map", AIP Conference Proceedings: Volume 1738, Number 480060-1–480060-4, International Conference of Numerical Analysis and Applied Mathematics 2015 (ICNAAM 2015), ISBN: 978-0-7354-1392-4, http://dx.doi.org/10.1063/1.4952296, American Institute of Physics, 2016.

[289] T. Yoshinaga, Y. Tanaka, K. Fujimoto. "Iterative method as discretization of continuous-time method based on dose-volume constrained optimization for intensity-modulated radiation therapy treatment planning", European Society of Radiology, ECR 2015, C-0519, DOI: 10.1594/ecr2015/C-0519, 2015.

[290] Hasan Özyapıcı, İlhan Dalcı, and Ali Özyapıcı. "Integrating accounting and multiplicative calculus: an effective estimation of learning curve", Journal: Computational and Mathematical Organization Theory, Print ISSN: 1381-298X, Online ISSN: 1572-9346, DOI: 10.1007/s10588-016-9225-1, Springer, 2016.

[291] Yusuf Gurefe and Emine Misirli. "New 2-point implicit block multistep method for multiplicative initial value problems", AIP Conference Proceedings: Volume 1738, International Conference of Numerical Analysis and Applied Mathematics 2015 (ICNAAM 2015), ISBN: 978-0-7354-1392-4, http://dx.doi.org/10.1063/1.4952296, American Institute of Physics, 2016.


=================================================================================================================================================
Links/Reading
Contents

Home
Multiplicative Calculus
Brief History
Applications
Citations
Reviews
Comments
Quotations
References
Links/Reading
Appendix 1
Appendix 2
Appendix 3
Dedication
Links


HathiTrust.org for online and downloadable books.

Digital Public Library of America for online and downloadable books.

Google Books for online books.

Amazon.com for books to buy.
WorldCat.org for library books.


Reading
Michael Grossman. "An introduction to non-Newtonian calculus", International Journal of Mathematical Education in Science and Technology, Volume 10, Number 4, pages 525-528, Taylor and Francis, 1979.
Michael Grossman and Robert Katz. "A new approach to means of two positive numbers", International Journal of Mathematical Education in Science and Technology, Volume 17, Number 2, pages 205 -208, Taylor and Francis, 1986.
Jane Grossman, Michael Grossman, and Robert Katz. "Which growth rate?", International Journal of Mathematical Education in Science and Technology, Volume 18, Number 1, pages 151 - 154, Taylor and Francis, 1987.
Michael Grossman and Robert Katz. "Isomorphic calculi", International Journal of Mathematical Education in Science and Technology, Volume 15, Issue 2, pages 253-263, DOI:10.1080/0020739840150214, Taylor and Francis, 1984.
Michael Grossman. "Calculus and discontinuous phenomena", International Journal of Mathematical Education in Science and Technology, Volume 19, Number 5, pages 777-779, Taylor and Francis, 1988.
Robert Katz. Axiomatic Analysis, D. C. Heath and Company, 1964.
Robert Edouard Moritz. "Quotientiation, an extension of the differentiation process", Proceedings of the Nebraska Academy of Sciences, Volume VII, pages 112 - 117, 1897 - 1890.



NOTE. The six books on non-Newtonian calculus and related matters by Jane Grossman, Michael Grossman, and Robert Katz are indicated below, and are available at some academic libraries, public libraries, and booksellers such as Amazon.com. On the Internet, each of the books can be read and downloaded, free of charge, at HathiTrust, Google Books, and the Digital Public Library of America.
Michael Grossman and Robert Katz.  Non-Newtonian Calculus, ISBN 0912938013, 1972. [15] 
Michael Grossman. The First Nonlinear System of Differential and Integral Calculus, ISBN 0977117006, 1979. (The geometric calculus) [11] 
Jane Grossman, Michael Grossman, Robert Katz. The First Systems of Weighted Differential and Integral Calculus, ISBN 0977117014, 1980. [9]
Jane Grossman. Meta-Calculus: Differential and Integral, ISBN 0977117022, 1981. [7]
Michael Grossman. Bigeometric Calculus: A System with a Scale-Free Derivative, ISBN 0977117030, 1983. [10]
Jane Grossman, Michael Grossman, and Robert Katz. Averages: A New Approach, ISBN 0977117049, 1983. [8]

=====================================================================================================================================================
Appendix 1
Contentsf

Home
Multiplicative Calculus
Brief History
Applications
Citations
Reviews
Comments
Quotations
References
Links/Reading
Appendix 1
Appendix 2
Appendix 3
Dedication

My Response to Some Critics, by Michael Grossman



It's an interesting fact that in any given non-Newtonian calculus, the derivative and integral can be expressed in terms of the classical derivative and classical integral, respectively (Non-Newtonian Calculus [15], page 31). Unfortunately, some critics of non-Newtonian calculus have fallaciously argued that it follows from that fact that the non-Newtonian calculi are useless.

Indeed, if that argument is valid, then each of the absurd arguments indicated below would also be valid:
The operation of multiplication of two positive integers is useless because of the fact that it can be expressed in terms of (repeated) addition.
The operation of multiplication of two positive numbers is useless because of the fact that it can be expressed in terms of addition (by using logarithms).
The operation of division of two positive numbers is useless because of the fact that it can be expressed in terms of subtraction (by using logarithms). 
The geometric average (of two positive numbers) is useless because of the fact that it can be expressed in terms of the arithmetic average (by using logarithms).
The logarithmic derivative is useless because of the fact that it can be expressed in terms of the classical derivative.
The elasticity concept (used in economics, etc.) is useless because of the fact that it can be expressed in terms of the classical derivative.
The classical calculus is useless because of the fact that the classical derivative and classical integral can each be expressed in the context of the real number system (e.g., by using 'epsilon-delta' formulations).

Interestingly enough, the great mathematician, astronomer, and physicist Carl Friedrich Gauss (1777-1855) discussed the usefulness of new calculi a long time ago:   
"In general the position as regards all such new calculi is this - That one cannot accomplish by them anything that could not be accomplished without them. However, the advantage is, that, provided such a calculus corresponds to the inmost nature of frequent needs, anyone who masters it thoroughly is able - without the unconscious inspiration of genius which no one can command - to solve the respective problems, indeed to solve them mechanically in complicated cases in which, without such aid, even genius becomes powerless. Such is the case with the invention of general algebra [and] with the differential calculus ... . Such conceptions unite, as it were, into an organic whole countless problems which otherwise would remain isolated and require for their separate solution more or less application of inventive genius."
 - Carl Friedrich Gauss, as quoted in Carl Friedrich Gauss: Werke, Volume 8, page 298; and as quoted in Robert Edouard Moritz's  book Memorabilia Mathematica or The Philomath's Quotation Book, quotation #1215 (1914). 
 

Finally, it's worth noting that the classical derivative and classical integral can each be expressed in terms of the corresponding operator of any given non-Newtonian calculus. (Please see page 28 of The First Nonlinear System of Differential and Integral Calculus [11], and page 34 of Bigeometric Calculus: A System With a Scale-free Derivative [10].) 

 - Michael Grossman






==================================================================================================================================================
Appendix 2
Contents

Home
Multiplicative Calculus
Brief History
Applications
Citations
Reviews
Comments
Quotations
References
Links/Reading
Appendix 1
Appendix 2
Appendix 3
Dedication



Recollections and Reflections, by Michael Grossman

In this section I present some recollections and reflections about my participation in the creation and development of non-Newtonian calculus (NNC). I alone am responsible for the content. It's been an amazing experience. Many memories still seem vivid. I recommend that you read the Brief History section of the website before continuing.

Robert (Bob) Katz was one of my mathematics professors when I was a student at Tufts University (1960 to 1964). We eventually became good friends. Our common interests include mathematics, music, and hiking. While at Tufts, I assisted him in the writing of his textbook Axiomatic Analysis (D. C. Heath, 1964), and then in teaching from it. Axiomatic Analysis contains a simple, clear, and orderly treatment of  basic logic and the real number system, and provides students with important training in logical and creative thinking. Incidentally we were both graduate students at the Yale University mathematics department, although Bob preceded me by about twenty years.

A remarkable person, Bob is brilliant, amazingly imaginative, hardworking, caring, and generous. He is a superb mathematics writer, and a magnificent teacher - the best teacher I ever had. I have been privileged to be his student, colleague, and friend.

We began our work  on NNC on 14 July 1967. We happened to come across a simple algebraic identity in a statistics book that led us to create the first non-Newtonian calculus, the geometric calculus. This is explained on page 85 of our book Non-Newtonian Calculus. Shortly afterwards, we created infinitely more, but not all, non-Newtonian calculi. We were thrilled!  We realized that the geometric calculus, and possibly many other non-Newtonian calculi, would be useful in science, engineering, and mathematics. The geometric calculus is well suited for use in many situations where multiplication/division, rather than addition/subtraction, are the most natural ways of combining/comparing numbers.

By August of 1967, we had produced a table, which we called the CHART, on which we compared the concepts of the classical calculus with the corresponding concepts of the geometric calculus and the corresponding concepts of our most general (up to that time) non-Newtonian calculus. I have a vague memory of being in the back seat of Bob's car, heading west on the Massachusetts Turnpike. I had a pad of paper and a pencil, and I was working on the CHART. Bob and his wonderful wife Rosalie were in the front seats. We were driving from their home in Boston to Poughkeepsie, New York, where they had some family matters to take care of. Bob and I thought it would be a good idea for me to accompany him so we could finish the CHART quickly. I must have been concentrating intensely on my work because I have no memory of the city of Poughkeepsie or the sights we passed during that trip.

We decided to show the CHART to a distinguished Harvard mathematics professor. At that time Bob was a mathematics editor at a major publishing company based in Boston. The professor and Bob had met a few times when the professor was working with the company as a consultant on the famous SMSG mathematics project called "The New Math". They had both helped with Mary Dolciani's textbooks for SMSG. Bob and I both lived near the professor's home in Cambridge, Massachusetts. (I lived with my parents and sister Dotty in Somerville, Massachusetts.) We drove to his house. His wife told us he was not home, but that she would give the CHART to him. We thanked her, and drove away feeling cheerful. But unfortunately the professor never responded. We then showed the CHART to other pure mathematicians. But after a few months, it became clear that the CHART did not interest them.

So we tried another approach: writing a book that we called Dialogs on Non-Newtonian Calculus. We completed the book in August of 1968. It was a sequence of dialogs, written in the spirit of Galileo's 1632 classic Dialogue Concerning the Two Chief World Systems. Our book contained discussions between a student and a mathematician about the non-Newtonian calculi we had constructed in 1967. The dialog format enabled us to present the material in an informal way with lots of questions and answers. I remember hiking with Bob in the White Mountains of New Hampshire, to celebrate the completion of the book. But months later, the book had been rejected by all the publishers we had contacted.

Finally, in 1969, we decided to write a book in the form of a research report. We called it Non-Newtonian Calculus. Interestingly, in August of 1970 while writing the book we unexpectedly created infinitely more non-Newtonian calculi, including the bigeometric calculus. (Roughly, we did this by transforming function values and arguments, not just values.) Again we were thrilled! We knew that the bigeometric calculus like the geometric calculus and maybe many other non-Newtonian calculi would be useful. In fact, just as the geometric calculus is well suited for use in certain situations where multiplication/division are naturally used, the bigeometric calculus is well suited for use in many other situations where multiplication/division are naturally used.          

Non-Newtonian Calculus was first published in 1972. It contains among other things discussions about nine specific non-Newtonian calculi, including the geometric and bigeometric calculi, and the general theory of non-Newtonian calculus. We published the book ourselves in order to maintain control of the contents. We tried to make the book readable for scientists and engineers, as well as for mathematicians. Well aware that people are reluctant to accept new ideas without good reasons, we worked hard to develop motivations and explanations for each concept. Included in the book are various ideas concerning potential applications, including a chapter with heuristic guides for choosing an appropriate calculus. We were determined to write the book clearly and concisely, and made a special effort to avoid mistakes.

We distributed copies of Non-Newtonian Calculus to individual purchasers, to libraries, and to journals for review. We received enthusiastic encouragement from a few people. The response we received from Dirk J. Struik, the eminent historian and mathematician, was particularly pleasing.  We sent him a copy of Non-Newtonian Calculus in 1972. He graciously invited us to visit him at his home in Belmont, Massachusetts. Our visit was enjoyable and memorable. His deep historical perspective is uncommon among research mathematicians. ("Mathematicians are notoriously bad historians", according to the mathematician Peter D. Lax. [282]) Professor Struik recognized NNC's potential for application, and he asked us to keep him informed.
"Your ideas [in Non-Newtonian Calculus] seem quite ingenious."
 - Dirk J. Struik, Massachusetts Institute of Technology, USA; from his letter dated 20 April 1972.

Cautiously optimistic, we patiently waited to see if anybody would discover some application(s) of NNC. But much to our dismay, it turned out that various pure mathematicians said NNC was useless. (Please see Appendix 1.) Some of them were rude and arrogant. It became discouraging. 

The negativity of some pure mathematicians toward NNC surprised us at first, but eventually we became painfully accustomed to it. In the beginning we were naively unaware that new and unusual ideas, even in science and mathematics, are often ignored, ridiculed, or demeaned by the academic establishment, even in the 20th and 21st centuries. This matter is perceptively discussed in the following two quotations, and in Scott Berkun's The Myths of Innovation (O'Reilly Media, Inc., 2010), a book I wish Bob and I had read before we began our work on NNC.

"Some insights are resisted with such intensity that it may take decades before they're widely accepted among scientists. For some, such as HIV, heliocentrism, and evolution, pockets of resistance remain decades or centuries after the war is won. ... Are you ready to have the top scientists in your field criticizing your work in journals and dissing you at meetings? Because history shows that the deeper your idea cuts into the heart of a field, the more your peers are likely to challenge you. Human nature being what it is, what ought to be reasoned discussion may turn personal, even nasty. ... Progress is made when good scientists keep working -- and keep supporting what they believe is true -- despite the criticism."
 - Anne Sasso; from her article "Audacity, Part 5: Rejection and Ridicule" in the magazine Science (American Association for the Advancement of Science) (11 June 2010). 

"Flying in the face of the Establishment with unconventional ideas and methods ... is highly esteemed in academia --  until somebody actually does it."
 - Edward Tenner; from his article "Benoit Mandelbrot the Maverick, 1924-2010" in The Atlantic magazine (16 October 2010).

Jane Tang and I got married in 1972. She joined Bob and me in our work on NNC, and over a period of several years we came up with various new ideas, and wrote more books and some articles. We showed that the classical calculus and each non-Newtonian calculus can be 'weighted' in a manner explained in our book The First Systems of Weighted Differential and Integral Calculus (1980). In each of these 'weighted calculi' a weight function and a weighted average play a central role. We then created the 'meta-calculi', in each of which two weight functions play a central role. The meta-calculi can be used for financial-investment analysis, and are discussed in our book Meta-Calculus: Differential and Integral (1981). Our book Averages: A New Approach (1983) includes a method we devised for using averages of functions to construct means of two positive numbers; and our article "A new approach to means of two positive numbers" (1986) contains examples of how that method can be used to prove inequalities. Our article "Which growth rate?" (1987) includes application of the geometric calculus to the study of growth and decay. That article provides reasons for using the geometric derivative for studying the growth of organisms. Our book The First Nonlinear System of Differential and Integral Calculus, published in 1979, contains a detailed treatment of the geometric calculus. (In that book, we used the expression "exponential calculus" instead of "geometric calculus".) And our book Bigeometric Calculus: A System with a Scale-Free Derivative, published in 1983, contains a detailed treatment of the bigeometric calculus. Between 1972 and 1988, we wrote five books and several journal articles. These publications received some favorable responses, but also discouraging criticism from some pure mathematicians.

I don't remember the exact date. It might have been in 1978, but in any case it was a beautiful spring day at Bob and Rosalie's home in Rockport, Masachusetts, where they had lived since about 1969. Bob, Rosalie, Jane, and I met with the distinguished English mathematics-historian Ivor Grattan-Guinness and Professor Thomas Hawkins, his colleague from Boston University, where Professor Grattan-Guinness had come to deliver a lecture. It was an interesting and enjoyable meeting. Professor Grattan-Guinness was good-natured, extremely knowledgeable, and obviously interested in NNC. He was impressed by the potential and originality of our work. Here's an excerpt from his 1977 review of Non-Newtonian Calculus:
"There is enough here [in Non-Newtonian Calculus] to indicate that non-Newtonian calculi ... have considerable potential as alternative approaches to traditional problems. This very original piece of mathematics will surely expose a number of missed opportunities in the history of the subject."
Ivor Grattan-Guinness, Middlesex University, England; from his review of Non-Newtonian Calculus in Middlesex Math Notes, Middlesex University, London, England, Volume 3, pages 47 - 50, 1977.

An interesting application of NNC was made in 1980 by James R. Meginniss, an econometric specialist at the Claremont Graduate School and Harvey Mudd College. In his article "Non-Newtonian calculus applied to probability, utility, and Bayesian analysis", Professor Meginniss presented "a new theory of probability that is adapted to human behavior and decision making". In May of 1980, Bob and I met Professor Meginniss at Purdue University in Indiana when he presented his article to the American Statistical Association. He told us he had learned about NNC from one of his students who found a copy of Non-Newtonian Calculus in their school library and realized that NNC could be used in Professor Meginniss' research.

For me, one of the most difficult aspects of the NNC project was making sure that we made no mistakes. I had worked hard on the proofs, and then I checked them and rechecked them to make sure they were correct. Even after our last publication, I continued to check the proofs. The books had been reviewed, and undoubtedly examined carefully, by first-rate mathematicians, and no errors were found. And yet, for years I was plagued with doubt. I couldn't convince myself that there was nothing wrong with our work.

"Mathematicians are notoriously difficult to convince."
 - John Stuart Mill, as quoted in Robert Katz's book Axiomatic Analysis (1964).

" Proofs should be treated with skepticism until mathematicians have had a chance to review them thoroughly, Yau [Shing-Tung Yau] told us. Until then, he said, 'it’s not math—it’s religion.'”
 - Sylvia Nasar and David Gruber, from their article "Manifold destiny" in the The New Yorker magazine, 28 August 2006.

But ironically it turned out that there was something wrong with me: obsessive-compulsive disorder (OCD). By the spring of 1988, my checking and rechecking had become all-consuming. It resulted in anxiety and depression, so bad that I could not function normally. That's when a psychiatrist diagnosed me with OCD. It was the first time I had ever heard of OCD, which is an anxiety disorder. Some OCD patients compulsively wash their hands, disturbed that their hands might be dirty and unable to ever convince themselves that their hand-washing was effective. Other patients compulsively check their door locks, unable to ever convince themselves that their doors are locked and their houses are safe. In my case the obsession is mathematics, and the compulsion is checking my NNC notes to make sure that all the proofs are valid. In any case, no matter what the obsession/compulsion, OCD is torturous and extremely depressing.

"Obsessive-compulsive disorder can make people do weird things. The mathematician Kurt Gödel was so afraid of tainted food that he would eat only portions his wife tasted first; after she became too ill to do this, he starved to death. The inventor Nikola Tesla spent the last 10 years of his life living in a hotel, eating at precisely 8 o’clock every evening, always using a stack of exactly 18 linen napkins to clean his cutlery; he felt compelled to walk around the block three times before entering his laboratory, and he was so afraid of germs that he would not allow his friends near him. My favorite anecdote from the book [The Man Who Couldn’t Stop: OCD and the True Story of a Life Lost in Thought]: A Canadian man whose OCD was so unbearable that he attempted suicide by shooting himself in the head — but succeeded only in lobotomizing himself in such a way that he was cured."
 - Scott Stossel, from his review in The New York Times newspaper (30 January 2015) of David Adam's book The Man Who Couldn’t Stop: OCD and the True Story of a Life Lost in Thought.


Many months of behavior therapy with an excellent psychologist, and a daily dose of antidepressant medicine, prescribed by excellent psychiatrists, have enabled me to recover enough to lead a normal life - except that I can not do mathematics anymore. Jane and I gave up our positions in the mathematics department at the University of Massachusetts Lowell, and moved to Florida. Jane taught herself investing during the many extended periods when I was busy checking my NNC notes; and as a result of her wise investment decisions, we now have a comfortable life.

Bob didn't understand my condition, and became frustrated when his well-intentioned suggestions failed to help me. On 19 October 1989, Bob sent me a letter indicating that he felt it would be best for both of us if we suspended our friendship.

The support I received from my wife Jane Tang Grossman, my parents Ruth and Milton Grossman, my sisters Linda Meier and Dorothy Lukas, and the Grossman and Tang families gave me reason to try hard to recover from my OCD, a seemingly bottomless pit. Anxiety and depression are extraordinarily difficult to combat. Depression is not the same as sadness. It's far worse. My psychiatrists and psychologist have been invaluable. Without them I would not have recovered as much as I have, if at all. I am deeply grateful to Drs. Morton Miller, Patricia Tahan, and Jeanne Yetz for their skill, guidance, and patience.

Jane and I have been living in Jupiter, Florida, since 1990. We enjoy the beaches. And we enjoy the freedom to pursue our interests. Jane, who makes piano and mathematics tutorials on YouTube, has established an online following, and I help her with her website (https://sites.google.com/site/pianoandmathtutorials/).

In 2003, my nephews David Lukas and Kenneth Lukas urged me to take what turned out to be a pivotal step: using the Internet to broadcast information about NNC. Dave is a computer engineer. Ken, a talented musician, uses the computer to compose music and to create videos and movies. They knew that posting information on the Internet would be an excellent way to inform people about NNC. On 2 April 2003, Dave set up the first NNC website. Later, on 28 September 2003, I set up another NNC website. I also made all six of our books on NNC and related matters available to be read and downloaded, free of charge, at HathiTrust, Google Books, and the Digital Public Library of America; and I made the books available for purchase at booksellers such as Amazon.com. ... As a result researchers who previously knew nothing about our work have found applications of NNC in many fields!

"After a long period of silence in the field of non-Newtonian calculus introduced by Grossman and Katz [15] in 1972, the field experienced a revival with the mathematically comprehensive description of the geometric calculus by Bashirov et al. [2], which initiated a kickstart of numerous publications in this field."
 - Mustafa Riza and Bugce Eminaga,  from their article "Bigeometric Calculus and Runge Kutta Method" [215] (2015).


Written by Agamirza E. Bashirov, Emine Misirli Kurpinar, and Ali Ozyapici, the article  [2]  ("Multiplicative calculus and its applications") was published in the Journal of Mathematical Analysis and Applications in 2008. It provides detailed development of the geometric calculus and some applications. From the Abstract: "In the present paper our aim is to bring [the geometric] calculus to the attention of researchers and to demonstrate its usefulness." The article has been of considerable interest to, and used by, many scientists, engineers, and applied mathematicians.

The trigger for writing that article may well have been an e-mail sent to me in 2004 by Professor Ali Ozyapici, who was then a graduate student. He was excited by information about NNC that we had posted on the Internet, and wanted to learn more. I was happy to hear from him and we corresponded by e-mail for several months. In 2006, I received an e-mail from Professor Misirli Kurpinar, Ali Ozyapici's teacher at that time. She expressed her interest in NNC and its possible applications. I have also had occasion to correspond with Professors Bashirov and Riza. Those four professors and some of their colleagues are now pursuing applications of NNC. In fact, Ali Ozyapici received his PhD degree in 2009 with a dissertation entitled "Multiplicative [geometric] calculus and its applications"; and he received his master's degree  in 2005 with a thesis called "Non-Newtonian Calculi".

Researchers around the world have applied non-Newtonian calculus in a variety of fields. Among them are fractal theory, including fractal structures and fractal materials; dynamical systems, including chaos theory and Lorenz systems; differential equations, including Runge-Kutta methods; computer science, including imaging, signal processing, and artificial-intelligence; and growth/decay analysis in economics, finance, biology, and chemistry.


"Random fractals, a quintessentially 20th century idea, arise as natural models of various physical, biological (think your mother's favorite cauliflower dish), and economic (think Wall Street, or the Horseshoe Casino) phenomena, and they can be characterized in terms of the mathematical concept of fractional dimension. Surprisingly, their time evolution can be analyzed by employing a non-Newtonian calculus ..."
 - Wojbor Woyczynski, Case Western Reserve University, USA; from an abstract to his 2013 seminar [146].

"The goal of this paper is chaos examination in multiplicative dynamical systems described with the [bigeometric] derivative."
 - Dorota Aniszewska and Marek Rybaczuk, both from Wroclaw University of Technology in Poland; from their 2008  article "Lyapunov type stability and Lyapunov exponent for exemplary multiplicative dynamical systems". [131]

" ... evolution of fractal characteristics will be examined with the help of dynamical system theory or more precisely in terms of [the bigeometric] calculus." 
Dorota Aniszewska and Marek Rybaczuk, both from Wroclaw University of Technology in Poland; from their 2009 article "Fractal characteristics of defects evolution in parallel fibre reinforced composite in quasi-static process of fracture". [184]

"In 2011, Bashirov et al. ["On modeling with multiplicative differential equations"] exploit the efficiency of  [the geometric] calculus over the Newtonian calculus. They demonstrated that the [geometric calculus] differential equations are more suitable than the ordinary differential equations in investigating some problems in various fields. Furthermore, Bashirov et al. [" Multiplicative calculus and its applications"] illustrated the usefulness of [the geometric] calculus with some interesting applications."
 - Feng Gu (Hangzhou Normal University in China) and Yeol-Je Cho (Gyeongsang National University in Korea); from their 2015 article [249].

"Theory and applications of non-Newtonian calculus have been evolving rapidly over the recent years. As numerical methods have a wide range of applications in science and engineering, the idea of the design of such numerical methods based on non-Newtonian calculus is self-evident. In this paper, the well-known Runge-Kutta method for ordinary differential equations is developed in the frameworks of non-Newtonian calculus given in generalized form and then tested for different generating functions. The efficiency of the proposed non-Newtonian Euler and Runge-Kutta methods is exposed by examples, and the results are compared with the exact solutions."
 - Ugur Kadak (Gazi University, in Turkey) and Muharrem Ozluk (Batman University in Turkey); from their 2015 article "Generalized Runge-Kutta Method with respect to Non-Newtonian Calculus". [210]

"We advocate the use of an alternative calculus in biomedical image analysis, known as multiplicative (a.k.a. non-Newtonian) calculus.
 - Luc Florack and Hans van Assen, both of Eindhoven University of Technology in The Netherlands; from their 2012 article "Multiplicative calculus in biomedical image analysis". [88]

"In our proseminar we'll learn some of the most exciting non-linear calculations and consider the applications for which they may be of particular interest. The applications range from rates of return and other growth processes to highly active areas of digital image processing."
Joachim Weickert (Gottfried Wilhelm Leibniz Prize winner), Saarland University in Germany; from his description of his non-Newtonian calculus course "Analysis beyond Newton and Leibniz", Saarland University, 2012. [106]

"In this paper, the multiplicative least square method is introduced and is applied to integrals for the finite product representation of the positive functions. Hence, many nonlinear functions can be represented by well-behaved exponential functions. Product representation produces an accurate representation of signals, especially where exponentials occur. Some real applications of nonlinear exponential signals will be selected to demonstrate the applicability and efficiency of proposed representation."
 - Ali Ozyapici (Cyprus International University in Cyprus/Turkey) and Bulent Bilgehan (Girne American University in Cyprus/Turkey); from their article "Finite product representation via multiplicative calculus and its applications to exponential signal processing" [225]

"Grossman and Katz [Non-Newtonian Calculus] mention several alternative calculi including: geometric, anageometric, bigeometric, quadratic, anaquadratic, biquadratic, harmonic, anaharmonic, and biharmonic. ... Non-Newtonian calculus has been used to derive optimization algorithms that perform better than traditional Newton based methods for Expectation-Maximization algorithms. However, Non-Newtonian calculus goes beyond simply being useful for optimization, it is useful for the other half of learning: modeling. The second order approximation using geometric calculus may produce the Gaussian curve ... . The nth order approximation using bigeometric calculus produces an nth polynomial on a log-log plot. ... Non-Newtonian generalized Taylor expansions produce nth order models, which are rarely polynomials. ... Non-Newtonian models sometimes make sense to use. Non-Newtonian models follow from non- Newtonian calculi. ... Here are a few rules of thumb for non-Newtonian models. If a meta-model is primarily concerned with learning probabilities, non-parametric distributions, or anything else where the multiplication is the primary operation, then the geometric calculi may be of interest. If working in a domain where the squares are additive, as is common the case when estimating the variance of a sum of independent random variables, then the quadratic calculi may produce meaningful models."
 - Michael Valenzuela (University of Arizona in the United States); from his 2016 doctoral dissertation "Machine learning, optimization, and anti-training with sacrificial data" (In computer science, machine learning is a branch of artificial intelligence.) [279].

"In this paper, after a brief presentation of [the geometric] calculus, we try to show how it could be used to re-explore from another perspective classical economic theory, more particularly economic growth and the maximum-likelihood method from statistics."
 - Diana Andrada Filip (Babes-Bolyai University of Cluj-Napoca in Romania) and Cyrille Piatecki (Orléans University in France); from their 2014 article "An overview on non-Newtonian calculus and its potential applications to economics".  [181]

"In many circumstances, [the geometric] calculus is highly natural; for example, the decay of a radioactive material and the unconstrained growth of a bacterial colony [yield] constant [geometric] derivatives." 
 - Christopher Olah, Thiel Fellow; from Christopher Olah's Blog, 10 June 2011. [134, 135, 177]


On 14 October 2003, I got a pleasant surprise: an e-mail from Bob. He found my e-mail address on one of our NNC websites and decided to contact me, our first contact since 19 October 1989.  I was happy to find out that he and Rosalie were doing well. They were still living in Rockport. Bob and I continued to exchange e-mails for about 2 years, but then the e-mailing ended, as Bob decided to stop using the Internet.

In 2008, I created a "Non-Newtonian calculus" article on Wikipedia, the controversial Internet "encyclopedia". I kept updating it with information about new applications of NNC. But some people objected to the article, saying that NNC was "not notable". They wanted to delete the article from Wikipedia. The issue became a source of stress for me. In 2011, they did vote to delete the article. I decided to stop participating in Wikipedia. But ironically, through no effort of mine, Wikipedia now has a Category called "Non-Newtonian calculus", and articles within that Category concerning NNC and related matters.

In February of 2014, Jane decided to call Bob by telephone. At that time Bob was 91 years old and Rosalie was 87. (I was 71.) We were pleased to find that he and Rosalie were still doing well. He sounded as strong and mentally sharp as ever. Since then we've been talking on the telephone regularly. In August of 2014, my sister Dotty (Dave and Kenny's mother) and I visited Bob and Rosalie at their home in Rockport. It was a memorable visit, the first time Bob and I had seen each other in 25 years, and we had a wonderful time. I was overjoyed to see them. We had a lot to talk about, especially recent applications of NNC. 

A nice surprise occurred on 13 May 2015. I happened to come across the website for the 27th International Conference on Technology in Collegiate Mathematics (March of 2015). The conference is sponsored by Pearson PLC, the largest education company and the largest book publisher in the world. The website included an interesting keynote-speech by Eric Gaze concerning the 21st-century college-mathematics-curriculum. I was delighted to learn that NNC is among the featured topics recommended for that curriculum. [224]                      

Finally, a statement by an applied-mathematician in India resonates in my mind:
"It is known that non-Newtonian  calculus models real life problems more accurately."
 - R. C. Mittal, Indian Institute of Technology; from the ResearchGate website on 12 November 2014. [218]

It pleases me that this statement was made by a person I had never heard of, who lives on the other side of the world, and who is an excellent applied-mathematician. Not only does he believe that NNC models many real life problems more accurately than the classical calculus, he believes it is known that NNC models many real life problems more accurately than the classical calculus. This suggests that knowledge of NNC will continue to spread around the world, and that NNC will continue to be applied in a variety of fields.

 - Michael Grossman



=========================================================================================================================================================================
Appendix 3
Contents

Home
Multiplicative Calculus
Brief History
Applications
Citations
Reviews
Comments
Quotations
References
Links/Reading
Appendix 1
Appendix 2
Appendix 3
Dedication
A Letter to Robert Katz from Michael Grossman


21 July 2014

Dear Bob,

For the past several years, I've been loading lots of information onto the Internet regarding our work on non-Newtonian calculus (NNC) and related matters. This effort has turned out to be worthwhile, because lately, there have been far more applications of NNC than ever before. Apparently some people have seen the stuff I've put on the Internet, and have found it useful.

One of the most exciting areas of NNC application is fractal geometry, a subject created in the 1970s by Benoit Mandelbrot, a brilliant maverick mathematician/scientist whose work was, like ours, at first dismissed as pointless. Now fractal geometry is being used practically everywhere, e.g., in  biology, chemistry, physics, economics, engineering, psychology, physiology, and medicine. Fractal geometry, unlike Euclidean geometry, is concerned with roughness. In fact, fractal geometry is a new mathematical tool for investigating roughness. 

Here's where NNC comes in. I've learned from various sources on the Internet that the bigeometric calculus is being used in fractal geometry. In fact, the bigeometric derivative is now used to analyze the rate of change of the roughness of a fractal over time.

"Random fractals, a quintessentially 20th century idea, arise as natural models of various physical, biological (think your mother's favorite cauliflower dish), and economic (think Wall Street, or the Horseshoe Casino) phenomena, and they can be characterized in terms of the mathematical concept of fractional dimension. Surprisingly, their time evolution can be analyzed by employing a non-Newtonian calculus [the bigeometric calculus] ..."
 - Professor Wojbor Woyczynski, Case Western Reserve University, USA; from an abstract to his seminar at Case Western Reserve University on 03 April 2013.

"Many natural phenomena, from microscopic bacteria growth, through macroscopic turbulence, to the large scale structure of the Universe, display a fractal character. For studying the time evolution of such "rough" objects, the classical, "smooth" Newtonian calculus is not enough."
 - Professor Wojbor Woyczynski, Case Western Reserve University, USA; from an abstract to his seminar "Non-Newtonian calculus for the dynamics of random fractal structures" at The Ohio State University on 22 April 2011, and at Cleveland State University on 02 May 2012.

"Together with a small, highly focused research team, Ostoja-Starzewski is working across disciplines to unite methods from solid mechanics, advanced continuum mechanics, statistical physics and mathematics. Some of the specific mathematical theories they use include probability theory and non-Newtonian calculus. These approaches allow them to focus on different fractal structures ... "
 - Professor Martin Ostoja-Starzewski, University of Illinois at Urbana-Champaign, USA; from the 2013 media-upload "The inner workings of fractal materials", University of Illinois at Urbana-Champaign.

Another important area of NNC application is biomedical image analysis. Of course, biomedical imaging provides physicians and researchers with invaluable noninvasive-methods for examining living tissues. And biomedical imaging is used so extensively that some of its nomenclature has become commonplace: "X-ray", "ultrasound",  "MRI", "CAT-scan", "PET-scan". Remarkably, it turns out that the geometric calculus is a useful tool for biomedical image analysis.

"We advocate the use of an alternative calculus in biomedical image analysis, known as multiplicative (a.k.a. non-Newtonian) calculus. ... The purpose of this article is to provide a condensed review of multiplicative calculus and to illustrate its potential use in biomedical image analysis." (The expression "multiplicative calculus" refers here to the geometric calculus.)
 - Professors Luc Florack and Hans van Assen, both of Eindhoven University of Technology in The Netherlands; from their article "Multiplicative calculus in biomedical image analysis", Journal of Mathematical Imaging and Vision, Springer, 2012.

"Multiplicative calculus provides a natural framework in problems involving positive images and positivity preserving operators. In increasingly important, complex imaging frameworks, such as diffusion tensor imaging, it complements standard calculus in a nontrivial way. The purpose of this article is to illustrate the basics of multiplicative calculus and its application to the regularization of positive definite matrix fields." (The expression "multiplicative calculus" refers here to the geometric calculus.)
 - Professors Luc Florack, Eindhoven University of Technology in The Netherlands; from his article "Regularization of positive definite matrix fields based on multiplicative calculus", Scale Space and Variational Methods in Computer Vision, Lecture Notes in Computer Science, Springer, 2012.

"This work presents a new operator of non-Newtonian type which [has] shown [to] be more efficient in contour detection [in images with multiplicative noise] than the traditional operators. ... In our view, the work proposed in (Grossman and Katz, 1972) stands as a foundation ... ."  
 - Professors Marco Mora, Fernando Córdova-Lepe, and Rodrigo Del-Valle, all of Universidad Católica del Maule in Chile; from their article "A non-Newtonian gradient for contour detection in images with multiplicative noise", Pattern Recognition Letters, International Association for Pattern Recognition, Elsevier, 2012. 

"In our proseminar we'll learn some of the most exciting non-linear calculations and consider the applications for which they may be of particular interest. The applications range from rates of return and other growth processes to highly active areas of digital image processing."
 - Professor Joachim Weickert (Gottfried Wilhelm Leibniz Prize winner), Saarland University in Germany; from his description of his non-Newtonian calculus course  "Analysis beyond Newton and Leibniz", Saarland University, 2012. 

In addition to fractal geometry and biomedical image analysis, I've come across many other areas of NNC application, some of which we anticipated years ago: 
growth/decay analysis (e. g., in economics and biology), dynamical systems and chaos theory, finance (e.g., rates of return), the theory of elasticity in economics, marketing, wave theory in physics, the economics of climate change, signal processing, atmospheric temperature, information technology, pathogen counts in treated water, actuarial science, tumor therapy in medicine, materials science/engineering, demographics, differential equations (including multiplicative Lorenz systems and Runge-Kutta methods), calculus of variations, finite-difference methods, averages of functions, means of two positive numbers, weighted calculus, meta-calculus, approximation theory, least-squares methods, multivariable calculus, complex analysis, functional analysis, probability theory, utility theory, Bayesian analysis, stochastics, and decision making.

The non-Newtonian calculi most often used are the geometric and bigeometric calculi, which, of course, is not surprising.

As you well know, for many years lots of people, especially various pure mathematicians, claimed that our work was useless. But, despite their discouraging and sometimes arrogant comments, we always knew that NNC has considerable potential for application in science, engineering, and mathematics. ... And we were right!!      


From:
Mike

===========================================================================================================================================================================

Dedication
Contents

Home
Multiplicative Calculus
Brief History
Applications
Citations
Reviews
Comments
Quotations
References
Links/Reading
Appendix 1
Appendix 2
Appendix 3
Dedication

Together with my sisters Linda Meier and Dorothy (Dotty) Lukas, I am profoundly grateful to my parents Ruth Grossman (13 March 1916 - 26 November 2002) and Milton Grossman (17 April 1911 - 1 July 2003), to whom I dedicate this website. 

Linda, Dotty, and I could not have been more fortunate. Our parents were extraordinary. They gave us unconditional love, unlimited attention, and inspired us by the examples they set. They taught us about honesty, the difference between right and wrong, hard-work, respect for others, the importance of education and independent thinking, the value of money, and the fact that "money isn't everything". Our bad behavior always resulted in a calm, logical, and convincing explanation of how we misbehaved, why it was wrong, and what we had to do to avoid repeating our mistake.

My parents were pleased and curious about non-Newtonian calculus. But they hadn't studied enough mathematics to really understand non-Newtonian calculus or its potential for application. I wish they had seen the various applications of non-Newtonian calculus that have appeared on the Internet since 2003. They would have been happy.

At a restaurant years ago, my wife Jane offered to pay for her parents Nancy and Michael Tang's dinners. A wonderful and brilliant person, Michael Tang half-jokingly responded: "Don't bother. You can never repay us!" Well, Linda, Dotty, and I could never repay our parents, either.

 - Michael Grossman

======================================================================================================================================================================================================================================================================================================================================================================================================================================================================================================================================================

































































































\end{document}
