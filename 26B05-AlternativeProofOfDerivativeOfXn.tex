\documentclass[12pt]{article}
\usepackage{pmmeta}
\pmcanonicalname{AlternativeProofOfDerivativeOfXn}
\pmcreated{2013-03-22 15:59:31}
\pmmodified{2013-03-22 15:59:31}
\pmowner{Wkbj79}{1863}
\pmmodifier{Wkbj79}{1863}
\pmtitle{alternative proof of derivative of $x^n$}
\pmrecord{12}{38013}
\pmprivacy{1}
\pmauthor{Wkbj79}{1863}
\pmtype{Proof}
\pmcomment{trigger rebuild}
\pmclassification{msc}{26B05}
\pmclassification{msc}{26A24}
\pmrelated{DerivativeOfXn}
\pmrelated{DerivativesByPureAlgebra}

\endmetadata

\usepackage{amssymb}
\usepackage{amsmath}
\usepackage{amsfonts}

\usepackage{psfrag}
\usepackage{graphicx}
\usepackage{amsthm}
%%\usepackage{xypic}
\begin{document}
The typical derivative formula

$$\frac{df}{dx}=\lim_{h \to 0}\frac{f(x+h)-f(x)}{h}$$

combined with the binomial theorem yield an alternative way to prove that

$$ \frac{d}{dx}(x^n)=nx^{n-1}$$

for any positive integer $n$.

\begin{proof}
\begin{center}
$\begin{array}{ll}
\displaystyle \frac{d}{dx}(x^n) & \displaystyle =\lim_{h \to 0}\frac{(x+h)^n-x^n}{h} \\
& \\
& \displaystyle =\lim_{h \to 0}\frac{\displaystyle \left( \sum_{j=0}^n {n \choose j} x^j h^{n-j} \right) - x^n}{h}\\
& \\
& \displaystyle =\lim_{h \to 0}\frac{\displaystyle x^n+nx^{n-1}h+h^2 \left( \sum_{j=0}^{n-2} {n \choose j} x^j h^{n-2-j} \right) - x^n}{h} \\
& \\
& \displaystyle =\lim_{h \to 0}\frac{\displaystyle nx^{n-1}h+h^2 \sum_{j=0}^{n-2} {n \choose j} x^j h^{n-2-j}}{h} \\
& \\
& \displaystyle =\lim_{h \to 0} \left( nx^{n-1}+h \sum_{j=0}^{n-2} {n \choose j} x^j h^{n-2-j} \right) \\
& \\
& \displaystyle = nx^{n-1} \end{array}$
\end{center}
\end{proof}
%%%%%
%%%%%
\end{document}
