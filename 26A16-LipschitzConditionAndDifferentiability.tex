\documentclass[12pt]{article}
\usepackage{pmmeta}
\pmcanonicalname{LipschitzConditionAndDifferentiability}
\pmcreated{2013-03-22 11:57:50}
\pmmodified{2013-03-22 11:57:50}
\pmowner{Mathprof}{13753}
\pmmodifier{Mathprof}{13753}
\pmtitle{Lipschitz condition and differentiability}
\pmrecord{34}{30776}
\pmprivacy{1}
\pmauthor{Mathprof}{13753}
\pmtype{Theorem}
\pmcomment{trigger rebuild}
\pmclassification{msc}{26A16}
\pmsynonym{mean value inequality}{LipschitzConditionAndDifferentiability}
\pmrelated{Derivative2}

\endmetadata

\usepackage{amsmath}
\usepackage{amsfonts}
\usepackage{amssymb}
\newcommand{\reals}{\mathbb{R}}
\newcommand{\natnums}{\mathbb{N}}
\newcommand{\cnums}{\mathbb{C}}
\newcommand{\znums}{\mathbb{Z}}
\newcommand{\lp}{\left(}
\newcommand{\rp}{\right)}
\newcommand{\lb}{\left[}
\newcommand{\rb}{\right]}
\newcommand{\supth}{^{\text{th}}}
\newtheorem{proposition}{Proposition}
\newtheorem{definition}[proposition]{Definition}

\newcommand{\lin}{\operatorname{lin}}

\newcommand{\Df}{{\operatorname{D}f}}


\newtheorem{theorem}[proposition]{Theorem}
\begin{document}
If $X$ and $Y$ are Banach spaces, e.g. $\reals^n$, one can inquire about the relation
between differentiability and the Lipschitz condition.    If $f$ is Lipschitz, the ratio
$$\frac{ \Vert f(q)-f(p)\Vert}{\Vert q-p \Vert},\quad p,q\in X$$
is bounded but is not assumed to converge to a limit.

\begin{proposition}
  Let $f:X\to Y$ be a \PMlinkname{continuously differentiable mapping}{DifferentiableMapping} between
  Banach spaces.  If $K\subset X$ is a compact
  subset, then the restriction $f:K\to Y$ satisfies the Lipschitz
  condition.
\end{proposition}
\emph{Proof.}
Let $\lin(X,Y)$ denote the Banach space of bounded linear maps from
$X$ to $Y$.  Recall that the norm $\Vert T\Vert$ of a linear mapping
$T\in\lin(X,Y)$ is defined by
$$\Vert T \Vert = \sup \{ \frac{\Vert Tu \Vert}{\Vert u\Vert} : u\neq 0\}.$$

Let $\Df:X\to \lin(X,Y)$ denote the derivative of $f$.  By definition
$\Df$ is continuous, which really means that
$\Vert \Df \Vert: X\to \reals$
is a continuous function.  Since
$K\subset X$ is compact, there exists a finite upper bound $B_1>0$ for
$\Vert \Df\Vert$ restricted to $K$.  In particular, this means that
$$\Vert \Df(p) u \Vert \leq \Vert \Df(p)\Vert \Vert u\Vert \leq B_1
\Vert u\Vert,$$
for all $p\in K,\; u\in X$.

Next, consider the secant mapping $s:X\times X\to\reals$ defined by
$$s(p,q) = 
\begin{cases}
  \displaystyle \frac{\Vert f(q) - f(p) - \Df(p)(q-p)\Vert}{\Vert q-p\Vert} & q\neq
  p \\
  0 & p=q
\end{cases}
$$
This mapping is continuous, because $f$ is assumed to be continuously
differentiable.  Hence, there is a finite
upper bound $B_2>0$ for $s$ restricted to the compact set $K\times K$.  It
follows that for all $p,q\in K$ we have
\begin{align*}
  \Vert f(q) - f(p) \Vert &\leq \Vert f(q) - f(p) -
  \Df(p)(q-p)\Vert  + \Vert \Df(p)(q-p)\Vert\\
  &\leq B_2 \Vert q-p\Vert +   B_1 \Vert q-p\Vert\\
  &= (B_1+B_2)\Vert q-p\Vert
\end{align*}
Therefore $B_1+ B_2$ is the desired Lipschitz constant.  QED

Neither condition is stronger. For example, the function $f:\reals \to \reals$
given by $f(x) = x^2$ is differentiable but not Lipschitz.
%%%%%
%%%%%
%%%%%
\end{document}
