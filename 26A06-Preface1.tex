\documentclass[12pt]{article}
\usepackage{pmmeta}
\pmcanonicalname{Preface1}
\pmcreated{2014-08-03 22:47:47}
\pmmodified{2014-08-03 22:47:47}
\pmowner{PMBookProject}{1000683}
\pmmodifier{rspuzio}{6075}
\pmtitle{Preface}
\pmrecord{3}{87450}
\pmprivacy{1}
\pmauthor{PMBookProject}{6075}
\pmtype{Definition}
\pmclassification{msc}{26A06}

% this is the default PlanetMath preamble.  as your knowledge
% of TeX increases, you will probably want to edit this, but
% it should be fine as is for beginners.

% almost certainly you want these
\usepackage{amssymb}
\usepackage{amsmath}
\usepackage{amsfonts}

% need this for including graphics (\includegraphics)
\usepackage{graphicx}
% for neatly defining theorems and propositions
\usepackage{amsthm}

% making logically defined graphics
%\usepackage{xypic}
% used for TeXing text within eps files
%\usepackage{psfrag}

% there are many more packages, add them here as you need them

% define commands here

\begin{document}
\begin{center}
PREFACE
\end{center}

The significance of the Calculus, the possibility of applying
it in other fields, its usefulness, ought to be kept constantly
and vividly before the student during his study of the subject,
rather than be deferred to an uncertain future.

Not only for students who intend to become engineers, but
also for those planning a profound study of other sciences, the
usefulness of the Calculus is universally recognized by teachers;
it should be consciously realized by the student himself. It is
obvious that students interested primarily in mathematics,
particularly if they expect to instruct others, should recognize
the same fact.

To all these, and even to the student who expects only general 
culture, the use of certain types of applications tends to
make the subject more real and tangible, and offers a basis for
an interest that is not artificial. Such an interest is necessary
to secure proper attention and to insure any real grasp of the
essential ideas.

For this reason, the attempt is made in this book to present
as many and as varied applications of the Calculus as it is
possible to do without venturing into technical fields whose
subject matter is itself unknown and incomprehensible to the
student, and without abandoning an orderly presentation of
fundamental principles.

The same general tendency has led to the treatment of
topics with a view toward bringing out their essential usefulness. 
Thus the treatment of the logarithmic derivative is
vitalized by its presentation as the relative rate of change of a
quantity; and it is fundamentally connected with the important
`` compound interest law,'' which arises in any phenomenon in
the relative rate of increase (logarithmic derivative) is
constant.

Another instance of the same tendency is the attempt, in the
introduction of the precise concept of curvature, to explain the
reason for the adoption of this, as opposed to other simpler
but cruder measures of bending. These are only instances, of
two typical kinds, of the way in which the effort to bring out
the usefulness of the subject has influenced the presentation of
even the traditional topics.

Rigorous forms of demonstration are not insisted upon, especially 
where the precisely rigorous proofs would be beyond
the present grasp of the student. Rather the stress is laid upon
the student's certain comprehension of that which is done, and
his conviction that the results obtained are both reasonable and
useful. At the same time, all effort has been made to avoid
those grosser errors and actual misstatements of fact which
have often offended the teacher in texts otherwise attractive
and teachable.

Thus a proof for the formula for differentiating a logarithm
is given which lays stress on the very meaning of logarithms;
while it is not absolutely rigorous, it is at least just as rigorous
as the more traditional proof which makes use of the limit of
$(1+1/n)^{n}$ as $n$ becomes infinite, and it is far more convincing
and instructive. The proof used for the derivative of the sine
of an angle is quite as sound as the more traditional proof
(which is also indicated), and makes use of fundamentally useful 
concrete concepts connected with circular motion. These
two proofs again illustrate the tendency to make the subject
vivid, tangible, and convincing to the student; this tendency
will be found to dominate, in so far as it was found possible,
every phase of every topic.

Many traditional theorems are omitted or reduced in importance. 
In many cases, such theorems are reproduced in exercises, with 
a sufficient hint to enable the student to master
them. Thus Taylor's Theorem in several variables, for which
wide applications are not apparent until further study of 
mathematics and science, is presented in this manner.

On the other hand, many theorems of importance, both from
mathematical and scientific grounds, which have been omitted
traditionally, are included. Examples of this sort are the brief
treatment of simple harmonic motion, the wide application of
Cavalieri's theorem and the prismoid formula, other approximation 
formulas, the theory of least squares (under the head
of exercises in maxima and minima), and many other topics.

The Exercises throughout are colored by the views expressed
above, to bring out the usefulness of the subject and to give
tangible concrete meaning to the concepts involved. Yet formal
exercises are not at all avoided, nor is this necessary if the
student's interest has been secured through conviction of the
usefulness of the topics considered. Far more exercises are
stated than should be attempted by any one student. This will
lend variety, and will make possible the assignment of different
problems to different students and to classes in successive
years. It is urged that care be taken in selecting from the
exercises, since the lists are graded so that certain groups of
exercises prepare the student for other groups which follow;
but it is unnecessary that all of any group be assigned, and it is
urged that in general less than half be used for any one student. 
Exercises that involve practical applications and others
that involve bits of theory to be worked out by the student are
of frequent occurrence. These should not be avoided, for they
are in tune with the spirit of the whole book; great care has
been taken to select these exercises to avoid technical concepts
strange to the student or proofs that are too difficult.

An effort is made to remove many technical difficulties by
the intelligent use of tables. Tables of Integrals and many
other useful tables are appended; it is hoped that these will
be found usable and helpful.

Parts of the book may be omitted without destroying the
essential unity of the whole. Thus the rather complete treatment 
of Differential Equations (of the more elementary types)
can be omitted. Even the chapter on Functions of Several
Variables can be omitted, at least except for a few paragraphs,
without vital harm; and the same may be said of the chapter
on Approximations. The omission of entire chapters, of course,
would only be contemplated where the pressure of time is unusual; 
but many paragraphs may be omitted at the discretion
of the teacher.

Although care has been exercised to secure a consistent order
of topics, some teachers may desire to alter it; for example,
an earlier introduction of transcendental functions and of portions 
of the chapter on Approximations may be desired, and is
entirely feasible. But it is urged that the comparatively early
introduction of Integration as a summation process be retained,
since this further impresses the usefulness of the subject, and
accustoms the student to the ideas of derivative and integral
before his attention is diverted by a variety of formal rules.

Purely destructive criticism and abandonment of coherent
arrangement are just as dangerous as ultra-conservatism. This
book attempts to preserve the essential features of the Calculus,
to give the student a thorough training in mathematical reasoning, 
to create in him a sure mathematical imagination, and
to meet fairly the reasonable demand for enlivening and enriching 
the subject through applications at the expense of purely
formal work that contains no essential principle.

\begin{flushright}
E. W. DAVIS,

W. C. BRENKE,

E. R. HEDRICK, Editor.
\end{flushright}

June, 1912.
\end{document}
