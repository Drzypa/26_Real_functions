\documentclass[12pt]{article}
\usepackage{pmmeta}
\pmcanonicalname{CharacterizationsOfMajorization}
\pmcreated{2013-03-22 15:26:37}
\pmmodified{2013-03-22 15:26:37}
\pmowner{Mathprof}{13753}
\pmmodifier{Mathprof}{13753}
\pmtitle{characterizations of majorization}
\pmrecord{7}{37291}
\pmprivacy{1}
\pmauthor{Mathprof}{13753}
\pmtype{Theorem}
\pmcomment{trigger rebuild}
\pmclassification{msc}{26D99}
\pmrelated{BirkoffVonNeumannTheorem}
\pmrelated{MuirheadsTheorem}
\pmdefines{Pigou-Dalton transfer}

% this is the default PlanetMath preamble.  as your knowledge
% of TeX increases, you will probably want to edit this, but
% it should be fine as is for beginners.

% almost certainly you want these
\usepackage{amssymb}
\usepackage{amsmath}
\usepackage{amsfonts}

% used for TeXing text within eps files
%\usepackage{psfrag}
% need this for including graphics (\includegraphics)
%\usepackage{graphicx}
% for neatly defining theorems and propositions
%\usepackage{amsthm}
% making logically defined graphics
%%%\usepackage{xypic}

% there are many more packages, add them here as you need them

% define commands here
\begin{document}
 Let $\mathcal{E}_n$ be the set of all $n\times n$ permutation
matrices that exchange two components. Such matrices have the form
\[
\begin{bmatrix}
\ddots& \\
&0 && 1 \\
&&\ddots \\
&1 && 0 \\
&&&&\ddots
\end{bmatrix}
\]
A matrix $T$ is called a {\it Pigou-Dalton transfer} (PDT) if
\[
  T = \alpha I + (1-\alpha)E
\]
for some $\alpha$ between 0 and 1, and $E\in \mathcal{E}_n$.


The following are equivalent
\begin{enumerate}
\item $x$ is \PMlinkname{majorized}{Majorization} by $y$.

\item $x=Dy$ for a  doubly stochastic matrix $D$.

\item $x=T_1T_2\cdots T_k y$ for finitely many PDT $T_1,\ldots,
T_k$.

\item $\sum_{i=1}^n \theta(x_i) \leq \sum_{i=1}^n \theta(y_i)$ for
all convex function $\theta$.

\item $x$ lies in the convex hull whose vertex set is
\[
  \big\{ (y_{\pi(1)},y_{\pi(2)},\ldots, y_{\pi(n)}):\,
\pi \text{ is a permutation  of }\{1,\ldots, n\}\big\}.
\]

\item For any $n$ non-negative real numbers $a_1,\ldots, a_n$,
\[
 \sum_{\pi} a_1^{x_{\pi(1)}} a_2^{x_{\pi(2)}} \cdots
 a_n^{x_{\pi(n)}} \leq
  \sum_{\pi} a_1^{y_{\pi(1)}} a_2^{y_{\pi(2)}} \cdots
 a_n^{y_{\pi(n)}}
\]
where summation is taken over all permutations of $\{1,\ldots,
n\}$.
\end{enumerate}

The equivalence of the above conditions are due to Hardy,
Littlewood, P{\'o}lya, Birkhoff, von Neumann and Muirhead.


{\bf Reference}

\begin{itemize}
\item G. H. Hardy, J. E. Littlewood and G. P{\'o}lya, {\em
Inequalities}, 2nd edition, 1952, Cambridge University Press,
London.

\item A. W. Marshall and I. Olkin, {\em Inequalities: Theory of
Majorization and Its Applications}, 1979, Acadamic Press, New
York.
\end{itemize}
%%%%%
%%%%%
\end{document}
