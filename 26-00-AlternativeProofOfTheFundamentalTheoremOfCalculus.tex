\documentclass[12pt]{article}
\usepackage{pmmeta}
\pmcanonicalname{AlternativeProofOfTheFundamentalTheoremOfCalculus}
\pmcreated{2013-03-22 15:55:24}
\pmmodified{2013-03-22 15:55:24}
\pmowner{ruffa}{7723}
\pmmodifier{ruffa}{7723}
\pmtitle{alternative proof of the fundamental theorem of calculus}
\pmrecord{4}{37930}
\pmprivacy{1}
\pmauthor{ruffa}{7723}
\pmtype{Proof}
\pmcomment{trigger rebuild}
\pmclassification{msc}{26-00}

\endmetadata

% this is the default PlanetMath preamble.  as your knowledge
% of TeX increases, you will probably want to edit this, but
% it should be fine as is for beginners.

% almost certainly you want these
\usepackage{amssymb}
\usepackage{amsmath}
\usepackage{amsfonts}

% used for TeXing text within eps files
%\usepackage{psfrag}
% need this for including graphics (\includegraphics)
%\usepackage{graphicx}
% for neatly defining theorems and propositions
%\usepackage{amsthm}
% making logically defined graphics
%%%\usepackage{xypic}

% there are many more packages, add them here as you need them

% define commands here

\begin{document}
An alternative proof for the first part involves the use of a formula derived by the method of exhaustion:

\[
\int_{a}^{b}f(t)dt=\left(  b-a\right)  \sum_{n=1}^{\infty}\sum_{m=1}^{2^{n}
-1}\left(  -1\right)  ^{m+1}2^{-n}f\left(  a+m(b-a)/2^{n}\right)  .
\]


Given that

\[
F(x)=\int_{a}^{x}f(t)dt,
\]


and

\[
F^{\prime}(x)=\lim_{\Delta x\rightarrow0}\frac{F(x+\Delta x)-F(x)}{\Delta
x}=\lim_{\Delta x\rightarrow0}\frac{1}{\Delta x}\int_{x}^{x+\Delta x}f(t)dt,
\]


the above formula leads to:

\[
F^{\prime}(x)=\lim_{\Delta x\rightarrow0}\frac{(x+\Delta x-x)}{\Delta x}
\sum_{n=1}^{\infty}\sum_{m=1}^{2^{n}-1}\left(  -1\right)  ^{m+1}2^{-n}f\left(
x+m\Delta x/2^{n}\right)  ,
\]


or

\[
F^{\prime}(x)=\sum_{n=1}^{\infty}\sum_{m=1}^{2^{n}-1}\left(  -1\right)
^{m+1}2^{-n}f\left(  x\right)  .
\]


Since it can be shown that

\[
\sum_{n=1}^{\infty}\sum_{m=1}^{2^{n}-1}\left(  -1\right)  ^{m+1}2^{-n}
=\sum_{n=1}^{\infty}2^{-n}=1,
\]


It follows that

\[
F^{\prime}(x)=f(x).
\]


The second part of the proof is identical to the parent.

%%%%%
%%%%%
\end{document}
