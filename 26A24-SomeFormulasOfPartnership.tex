\documentclass[12pt]{article}
\usepackage{pmmeta}
\pmcanonicalname{SomeFormulasOfPartnership}
\pmcreated{2014-08-26 11:08:16}
\pmmodified{2014-08-26 11:08:16}
\pmowner{burgess}{1001318}
\pmmodifier{burgess}{1001318}
\pmtitle{Some formulas of partnership}
\pmrecord{1}{}
\pmprivacy{1}
\pmauthor{burgess}{1001318}
\pmtype{Definition}

% this is the default PlanetMath preamble.  as your knowledge
% of TeX increases, you will probably want to edit this, but
% it should be fine as is for beginners.

% almost certainly you want these
\usepackage{amssymb}
\usepackage{amsmath}
\usepackage{amsfonts}

% need this for including graphics (\includegraphics)
\usepackage{graphicx}
% for neatly defining theorems and propositions
\usepackage{amsthm}

% making logically defined graphics
%\usepackage{xypic}
% used for TeXing text within eps files
%\usepackage{psfrag}

% there are many more packages, add them here as you need them

% define commands here

\begin{document}
1. Partnership : When two or more than two persons run a business jointly, they are called partners and the deal is known  as partnership
2. Ratio of Division of Gains:
(i)When investments of all the partners are for the same time, the gain or loss is distributed among the partners in the ratio of their investments.
Suppose A and B invest Rs. x and Rs. y respectively for a year in a business, then at end of the year
(A's share of profit) : (B's share of profit) = x:y
(ii) When investments are for different time periods, then equivalent capitals are calculated for a unit of time by taking (capital x number of units of time).Now, gain or loss is divided in the ratio of these capitals.
Suppose A invests Rs. x for p months and B invests Rs. y for q months, then  
(A's share of profit) : (B's share of profit) = px : qy
3. Working and Sleeping Partners:
A partner who manages the business is known as a working partner and the one who simply invests the money is a sleeping partner
I hope these concepts will be helpful for you to solve some of your math problems.

\end{document}
