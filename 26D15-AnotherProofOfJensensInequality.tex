\documentclass[12pt]{article}
\usepackage{pmmeta}
\pmcanonicalname{AnotherProofOfJensensInequality}
\pmcreated{2013-03-22 15:52:53}
\pmmodified{2013-03-22 15:52:53}
\pmowner{Andrea Ambrosio}{7332}
\pmmodifier{Andrea Ambrosio}{7332}
\pmtitle{another proof of Jensen's inequality}
\pmrecord{12}{37881}
\pmprivacy{1}
\pmauthor{Andrea Ambrosio}{7332}
\pmtype{Proof}
\pmcomment{trigger rebuild}
\pmclassification{msc}{26D15}
\pmclassification{msc}{39B62}

\endmetadata

% this is the default PlanetMath preamble.  as your knowledge
% of TeX increases, you will probably want to edit this, but
% it should be fine as is for beginners.

% almost certainly you want these
\usepackage{amssymb}
\usepackage{amsmath}
\usepackage{amsfonts}

% used for TeXing text within eps files
%\usepackage{psfrag}
% need this for including graphics (\includegraphics)
%\usepackage{graphicx}
% for neatly defining theorems and propositions
%\usepackage{amsthm}
% making logically defined graphics
%%%\usepackage{xypic}

% there are many more packages, add them here as you need them

% define commands here
\begin{document}
First of all, it's clear that defining
\[
\lambda _{k}=\frac{\mu _{k}}{\sum_{k=1}^{n}\mu _{k}}
\]
we have
\[
\sum_{k=1}^{n}\lambda _{k}=1
\]

so it will we enough to prove only the simplified version.

Let's proceed by induction.

1) $n=2$; we have to show that, for any $x_{1}$and $x_{2}$ in $[a,b]$, 
\[
f\left( \lambda _{1}x_{1}+\lambda _{2}x_{2}\right) \leq \lambda
_{1}f(x_{1})+\lambda _{2}f(x_{2}).
\]
But, since $\lambda _{1}+\lambda _{2}$ must be equal to 1, we can put $%
\lambda _{2}=1-\lambda _{1}$, so that the thesis becomes 
\[
f\left( \lambda _{1}x_{1}+\left( 1-\lambda _{1}\right) x_{2}\right) \leq
\lambda _{1}f(x_{1})+\left( 1-\lambda _{1}\right) f(x_{2}),
\]
which is true by definition of a convex function.

2) Taking as true that $f\left( \sum_{k=1}^{n-1}\mu _{k}x_{k}\right) \leq
\sum_{k=1}^{n-1}\mu _{k}f\left( x_{k}\right) $, where $\sum_{k=1}^{n-1}\mu
_{k}=1$, we have to prove that 
\[
f\left( \sum_{k=1}^{n}\lambda _{k}x_{k}\right) \leq \sum_{k=1}^{n}\lambda
_{k}f\left( x_{k}\right) ,
\]
where $\sum_{k=1}^{n}\lambda _{k}=1$.

First of all, let's observe that 
\[
\sum_{k=1}^{n-1}\frac{\lambda _{k}}{1-\lambda _{n}}=\frac{\left(
\sum_{k=1}^{n}\lambda _{k}\right) -\lambda _{n}}{1-\lambda _{n}}=\frac{%
1-\lambda _{n}}{1-\lambda _{n}}=1
\]
and that if all $x_{k}\in \lbrack a,b]$, $\sum_{k=1}^{n-1}\frac{\lambda _{k}%
}{1-\lambda _{n}}x_{k}$ belongs to $[a,b]$ as well. In fact, $\frac{\lambda
_{k}}{1-\lambda _{n}}$ being non-negative, 
\[
a\leq x_{k}\leq b\Rightarrow \frac{\lambda _{k}}{1-\lambda _{n}}a\leq \frac{%
\lambda _{k}}{1-\lambda _{n}}x_{k}\leq \frac{\lambda _{k}}{1-\lambda _{n}}b,
\]
and, summing over $k$,%
\[
a\sum_{k=1}^{n-1}\frac{\lambda _{k}}{1-\lambda _{n}}\leq \sum_{k=1}^{n-1}%
\frac{\lambda _{k}}{1-\lambda _{n}}x_{k}\leq b\sum_{k=1}^{n-1}\frac{\lambda
_{k}}{1-\lambda _{n}},
\]%
that is%
\[
a\leq \sum_{k=1}^{n-1}\frac{\lambda _{k}}{1-\lambda _{n}}x_{k}\leq b.
\]

We have, by definition of a convex function:
\begin{eqnarray*}
f\left( \sum_{k=1}^{n}\lambda _{k}x_{k}\right)  &=&f\left(
\sum_{k=1}^{n-1}\lambda _{k}x_{k}+\lambda _{n}x_{n}\right)  \\
&=&f\left( \left( 1-\lambda _{n}\right) \sum_{k=1}^{n-1}\frac{\lambda _{k}}{%
1-\lambda _{n}}x_{k}+\lambda _{n}x_{n}\right)  \\
&\leq &\left( 1-\lambda _{n}\right) f\left( \sum_{k=1}^{n-1}\frac{\lambda
_{k}}{1-\lambda _{n}}x_{k}\right) +\lambda _{n}f\left( x_{n}\right) .
\end{eqnarray*}
But, by inductive hypothesis, since $\sum_{k=1}^{n-1}\frac{\lambda _{k}}{%
1-\lambda _{n}}=1$, we have:
\[
f\left( \sum_{k=1}^{n-1}\frac{\lambda _{k}}{1-\lambda _{n}}x_{k}\right) \leq
\sum_{k=1}^{n-1}\frac{\lambda _{k}}{1-\lambda _{n}}f\left( x_{k}\right) ,
\]
so that%
\begin{eqnarray*}
f\left( \sum_{k=1}^{n}\lambda _{k}x_{k}\right)  &\leq &\left( 1-\lambda
_{n}\right) f\left( \sum_{k=1}^{n-1}\frac{\lambda _{k}}{1-\lambda _{n}}%
x_{k}\right) +\lambda _{n}f\left( x_{n}\right)  \\
&\leq &\left( 1-\lambda _{n}\right) \sum_{k=1}^{n-1}\frac{\lambda _{k}}{%
1-\lambda _{n}}f\left( x_{k}\right) +\lambda _{n}f\left( x_{n}\right)  \\
&=&\sum_{k=1}^{n-1}\lambda _{k}f\left( x_{k}\right) +\lambda _{n}f\left(
x_{n}\right)  \\
&=&\sum_{k=1}^{n}\lambda _{k}f\left( x_{k}\right) 
\end{eqnarray*}
which is the thesis.
%%%%%
%%%%%
\end{document}
