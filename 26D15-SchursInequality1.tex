\documentclass[12pt]{article}
\usepackage{pmmeta}
\pmcanonicalname{SchursInequality}
\pmcreated{2013-03-22 13:19:30}
\pmmodified{2013-03-22 13:19:30}
\pmowner{rspuzio}{6075}
\pmmodifier{rspuzio}{6075}
\pmtitle{Schur's inequality}
\pmrecord{11}{33836}
\pmprivacy{1}
\pmauthor{rspuzio}{6075}
\pmtype{Theorem}
\pmcomment{trigger rebuild}
\pmclassification{msc}{26D15}
%\pmkeywords{sum}
%\pmkeywords{inequality}
%\pmkeywords{Schur}

\endmetadata

% this is the default PlanetMath preamble.  as your knowledge
% of TeX increases, you will probably want to edit this, but
% it should be fine as is for beginners.

% almost certainly you want these
\usepackage{amssymb}
\usepackage{amsmath}
\usepackage{amsfonts}
\usepackage{amsthm}

% used for TeXing text within eps files
%\usepackage{psfrag}
% need this for including graphics (\includegraphics)
%\usepackage{graphicx}
% for neatly defining theorems and propositions
%\usepackage{amsthm}
% making logically defined graphics
%%%\usepackage{xypic}

% there are many more packages, add them here as you need them

% define commands here

\newcommand{\mc}{\mathcal}
\newcommand{\mb}{\mathbb}
\newcommand{\mf}{\mathfrak}
\newcommand{\ol}{\overline}
\newcommand{\ra}{\rightarrow}
\newcommand{\la}{\leftarrow}
\newcommand{\La}{\Leftarrow}
\newcommand{\Ra}{\Rightarrow}
\newcommand{\nor}{\vartriangleleft}
\newcommand{\Gal}{\text{Gal}}
\newcommand{\GL}{\text{GL}}
\newcommand{\Z}{\mb{Z}}
\newcommand{\R}{\mb{R}}
\newcommand{\Q}{\mb{Q}}
\newcommand{\C}{\mb{C}}
\newcommand{\<}{\langle}
\renewcommand{\>}{\rangle}
\begin{document}
If $a$, $b$, and $c$ are non-negative real numbers and $k\geq 1$ is real, then the following inequality holds: 
\[
a^k(a-b)(a-c)+b^k(b-c)(b-a)+c^k(c-a)(c-b)\geq 0
\]

\begin{proof}
We can assume without loss of generality that $c\leq b\leq a$ via a permutation of the variables (as both sides are symmetric in those variables).  Then collecting terms, we wish to show that 
\begin{align*}
(a-b)\left(a^k(a-c)-b^k(b-c)\right)+c^k(a-c)(b-c)\geq 0
\end{align*}
which is clearly true as every term on the left is positive.\end{proof}

There are a couple of special cases worth noting:
\begin{itemize}
\item Taking $k=1$, we get the well-known $$ a^3 + b^3 + c^3 + 3abc \geq ab(a+b) + ac(a+c) + bc(b+c) $$
\item If $c=0$, we get $(a-b)(a^{k+1}-b^{k+1})\geq0$.
\item If $b=c=0$, we get $a^{k+2}\geq0$.
\item If $b=c$, we get $a^{k}(a-c)^2\geq 0$.
\end{itemize}
%%%%%
%%%%%
\end{document}
