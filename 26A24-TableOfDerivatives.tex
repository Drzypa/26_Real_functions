\documentclass[12pt]{article}
\usepackage{pmmeta}
\pmcanonicalname{TableOfDerivatives}
\pmcreated{2013-03-22 17:34:48}
\pmmodified{2013-03-22 17:34:48}
\pmowner{CWoo}{3771}
\pmmodifier{CWoo}{3771}
\pmtitle{table of derivatives}
\pmrecord{31}{39992}
\pmprivacy{1}
\pmauthor{CWoo}{3771}
\pmtype{Feature}
\pmcomment{trigger rebuild}
\pmclassification{msc}{26A24}
\pmrelated{IntegralTables}
\pmrelated{Derivative2}
\pmrelated{GeneralFormulasForIntegration}

\endmetadata

\usepackage{amssymb,amscd}
\usepackage{amsmath}
\usepackage{amsfonts}
\usepackage{mathrsfs}
\usepackage{tabls}

% used for TeXing text within eps files
% need this for including graphics (\includegraphics)
%\usepackage{graphicx}
% for neatly defining theorems and propositions
\usepackage{amsthm}
% making logically defined graphics
%%\usepackage{xypic}
\usepackage{pst-plot}
\usepackage{psfrag}

% define commands here
\newtheorem{prop}{Proposition}
\newtheorem{thm}{Theorem}
\newtheorem{ex}{Example}
\newcommand{\real}{\mathbb{R}}
\newcommand{\ints}{\mathbb{Z}}
\newcommand{\pdiff}[2]{\frac{\partial #1}{\partial #2}}
\newcommand{\mpdiff}[3]{\frac{\partial^#1 #2}{\partial #3^#1}}
\newcommand{\sech}{\operatorname{sech}}
\newcommand{\csch}{\operatorname{csch}}
\newcommand{\arsinh}{\operatorname{arsinh}}
\newcommand{\arcosh}{\operatorname{arcosh}}
\newcommand{\artanh}{\operatorname{artanh}}
\newcommand{\arcoth}{\operatorname{arcoth}}
\begin{document}
\PMlinkescapeword{mode}
\PMlinkescapeword{open}
\PMlinkescapeword{line}
\PMlinkescapeword{lines}
\PMlinkescapeword{code}
\PMlinkescapeword{section}
Below are some tables of some real-valued functions and their corresponding derivatives:

\subsubsection*{Basic rules}

\begin{center}
\begin{tabular}{|c|c|}
\hline
$f(x)$ & $\displaystyle{\frac{df(x)}{dx}} = f'(x)$ \\
\hline\hline
$f(x) + g(x)$ & $f'(x)+g'(x)$ \\
\hline
$f(x)g(x)$ & $f'(x)g(x)+f(x)g'(x)$ \\
\hline
$\displaystyle \frac{f(x)}{g(x)},\, g\neq 0$ & $\displaystyle \frac{f'(x)g(x)-f(x)g'(x)}{g(x)^2}$ \\
\hline
$f(g(x))$&$f'(g(x))g'(x)$ \\
\hline
$f^{-1}(x)$& $\displaystyle{\frac{1}{f'(f^{-1}(x))}}$ \\
\hline
% add your function here & add its derivative here \\
% \hline 
\end{tabular}
\end{center}

\subsubsection*{\PMlinkname{Polynomials}{Polynomial} and powers}

\begin{center}
\begin{tabular}{|c|c|c|}
\hline
$f(x)$ & $f'(x)$ & applicable domain \\
\hline\hline
$c\in \mathbb{R}$ & 0 & $x\in \mathbb{R}$ \\
\hline
$x^r$ & $rx^{r-1}$ & $x\in \mathbb{R}$ \\
\hline
$\sqrt{x}$ & $\displaystyle\frac{1}{2\sqrt{x}}$ & $x>0$ \\
\hline
$|x|$ & $\displaystyle\frac{x}{|x|}=\frac{|x|}{x}$ & $x\ne 0$ \\
\hline
% add your function here & add its derivative here & specify where f is differentiable \\
% \hline 
\end{tabular}
\end{center}

\subsubsection*{Exponential and logarithmic functions}

\begin{center}
\begin{tabular}{|c|c|c|}
\hline
$f(x)$ & $f'(x)$ & applicable domain \\
\hline\hline
$\exp(x)=e^x$ & $\exp(x)=e^x$ & $x\in \mathbb{R}$ \\
\hline
$a^x$ & $a^x\ln{a}$ & $x\in \mathbb{R}$ \\
\hline
$\ln x$ & $\displaystyle{\frac{1}{x}}$ & $x>0$ \\
\hline
$x^x$ & $x^x(1+\ln x)$ & $x>0$ \\
\hline
% add your function here & add its derivative here & specify where f is differentiable \\
% \hline 
\end{tabular}
\end{center}

\subsubsection*{\PMlinkname{Trigonometric functions}{Trigonometry}}

\begin{center}
\begin{tabular}{|c|c|c|}
\hline
$f(x)$ & $f'(x)$ & applicable domain \\
\hline\hline
$\sin{x}$ & $\cos{x}$ & $x\in \mathbb{R}$ \\
\hline
$\cos{x}$ & $-\sin{x}$ & $x\in \mathbb{R}$ \\
\hline
$\tan{x}$ & $\sec^2{x}$ & $x\ne n\pi+\displaystyle{\frac{\pi}{2}},\, n\in \mathbb{Z}$  \\
\hline
$\cot{x}$ & $-\csc^2{x}$ & $x\ne n\pi,\, n\in \mathbb{Z}$ \\
\hline
$\sec{x}$ & $\sec{x}\tan{x}$ & $x\ne n\pi+\displaystyle{\frac{\pi}{2}},\, n\in \mathbb{Z}$ \\
\hline
$\csc{x}$ & $-\csc{x}\cot{x}$ & $x\ne n\pi,\, n\in \mathbb{Z}$ \\
\hline
$\arcsin{x}$ & $\displaystyle\frac{1}{\sqrt{1-x^2}}$ & $|x|<1$ \\
\hline
$\arccos{x}$ & $\displaystyle-\frac{1}{\sqrt{1-x^2}}$ & $|x|<1$ \\
\hline
$\arctan{x}$ & $\displaystyle\frac{1}{1+x^2}$ & $x\in \mathbb{R}$ \\
\hline
% add your function here & add its derivative here & specify where f is differentiable \\
% \hline 
\end{tabular}
\end{center}

\subsubsection*{\PMlinkname{Hyperbolic functions}{HyperbolicFunctions}}

\begin{center}
\begin{tabular}{|c|c|c|}
\hline
$f(x)$ & $f'(x)$ & applicable domain \\
\hline\hline
$\sinh{x}$ & $\cosh{x}$ & $x\in \mathbb{R}$ \\
\hline
$\cosh{x}$ & $\sinh{x}$ & $x\in \mathbb{R}$ \\
\hline
$\tanh{x}$ & $\sech^2{x}$ & $x\in \mathbb{R}$ \\
\hline 
$\coth{x}$ & $-\csch^2{x}$ & $x\ne 0$ \\
\hline
$\sech{x}$ & $-\sech{x}\tanh{x}$ & $x\in \mathbb{R}$ \\
\hline
$\csch{x}$ & $-\csch{x}\coth{x}$ & $x\ne 0$ \\
\hline
$\arsinh{x}$ & $\displaystyle\frac{1}{\sqrt{x^2\!+\!1}}$ & $x\ne 0$ \\
\hline
$\arcosh{x}$ & $\displaystyle\frac{1}{\sqrt{x^2\!-\!1}}$ & $|x|>1$ \\
\hline
$\artanh{x}$ & $\displaystyle\frac{1}{1\!-\!x^2}$ & $-1 < x < 1$ \\
\hline
$\arcoth{x}$ & $\displaystyle\frac{1}{1\!-\!x^2}$ & $|x| > 1$ \\
\hline
% add your function here & add its derivative here & specify where f is differentiable \\
% \hline 

\end{tabular}
\end{center}

\subsubsection*{Other functions} (see error function, logarithmic integral, sine integral, Gudermannian, Hermite polynomials)

\begin{center}
\begin{tabular}{|c|c|c|}
\hline
$f(x)$ & $f'(x)$ & applicable domain \\
\hline\hline
$\mbox{Erf}\,x\,$ & $\displaystyle\frac{2}{\sqrt{\pi}}e^{-x^2}$ & $x\in \mathbb{R}$ \\
\hline
$\mbox{Li}\,x$ & $\displaystyle\frac{1}{\ln{x}}$ & $x > 1$ \\
\hline
$\mbox{Si}\,x$ & $\displaystyle\mbox{sinc}\,x$ & $x \in \mathbb{R}$ \\
\hline
$\mbox{gd}\,x$ & $\displaystyle\frac{1}{\cosh{x}}$ &  $x \in \mathbb{R}$\\
\hline 
$\mbox{gd}^{-1}x$ & $\displaystyle\frac{1}{\cos{x}}$ &  $|x| < \frac{\pi}{2}$\\
\hline 
$H_n(x)$ & $2nH_{n-1}(x)$ & $x \in \mathbb{R}$ \\
 \hline 
% add your function here & add its derivative here & specify where f is differentiable \\
% \hline 


\end{tabular}
\end{center}




\textbf{Instructions on how to add a function and its derivative}.  Open the entry in edit mode.  Using the appropriate table for your function (or make a new table if applicable), make a copy of the two lines of comment (starting with \%) in the code (within the tabular environment) and paste it immediately before the comment.  For functions outside of the ``Basic rules'' section, include the appropriate domain.  Uncomment the lines (take out the \% symbols) after completing.  Preview before saving the entry.
%%%%%
%%%%%
\end{document}
