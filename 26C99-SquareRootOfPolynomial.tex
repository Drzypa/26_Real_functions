\documentclass[12pt]{article}
\usepackage{pmmeta}
\pmcanonicalname{SquareRootOfPolynomial}
\pmcreated{2013-03-22 15:32:06}
\pmmodified{2013-03-22 15:32:06}
\pmowner{pahio}{2872}
\pmmodifier{pahio}{2872}
\pmtitle{square root of polynomial}
\pmrecord{17}{37428}
\pmprivacy{1}
\pmauthor{pahio}{2872}
\pmtype{Algorithm}
\pmcomment{trigger rebuild}
\pmclassification{msc}{26C99}
\pmclassification{msc}{12E05}
\pmsynonym{calculation of square root of polynomial}{SquareRootOfPolynomial}
\pmrelated{SquareOfSum}
\pmrelated{BombellisMethodOfComputingSquareRoots}

% this is the default PlanetMath preamble.  as your knowledge
% of TeX increases, you will probably want to edit this, but
% it should be fine as is for beginners.

% almost certainly you want these
\usepackage{amssymb}
\usepackage{amsmath}
\usepackage{amsfonts}

% used for TeXing text within eps files
%\usepackage{psfrag}
% need this for including graphics (\includegraphics)
%\usepackage{graphicx}
% for neatly defining theorems and propositions
 \usepackage{amsthm}
% making logically defined graphics
%%%\usepackage{xypic}

% there are many more packages, add them here as you need them

% define commands here

\theoremstyle{definition}
\newtheorem*{thmplain}{Theorem}
\begin{document}
The \PMlinkescapetext{{\em square root of a polynomial}}\, $f$,\, denoted by $\sqrt{f}$, is any polynomial $g$ having the square\, $g^2$ equal to $f$.\, For example,\, $\sqrt{9x^2\!-\!30x\!+\!25} = 3x\!-\!5$ or $-3x\!+\!5$.

A polynomial needs not have a square root, but if it has a square root $g$, then also the opposite polynomial $-g$ is its square root.

\textbf{Algorithm.}\, The idea of the squaring \PMlinkescapetext{formula}
$$(a\!+\!b\!+\!c+..)^2 =(a)a+(2a\!+\!b)b+(2a\!+\!2b\!+\!c)c+..$$
(see the square of sum) gives a method for getting the square root of a polynomial:
\begin{itemize}
 \item The \PMlinkescapetext{terms of the radicand are ordered according to the rising or falling powers of certain letter (the first term must have a positive coefficient and even exponents).
 \item The leading term of the root is equal to the square root of the first term of the radicand.
 \item The second term of the root is equal to the first term of the first remainder divided by the double leading term.
 \item The third term of the root is equal to first term of the second remainder  divided by the double leading term}. 
 \item And so on.
\end{itemize}

In the examples below, on the \PMlinkescapetext{left under the lines there are the remainders, on the right under the lines the corresponding sums}.

\textbf{Example 1.}\, $\sqrt{9x^4\!+\!6x^3\!-\!11x^2\!-\!4x\!+\!4} =$ ? 

\begin{tabular}{lcrrrrrrrrrrr}
$\sqrt{}
$&$(9x^4$&$+6x^3$&$-11x^2$&$-4x$&$+4)$&$=$&$\pm$&$(3x^2$&$ +x$&$-2)$\\
$       $&$ 9x^4$&$     $&$      $&$   $&$   $&$ $&$   $&$ 3x^2$&$   $&$   $\\
\cline{2-2} \cline{9-9}
$       $&$     $&$ 6x^3$&$-11x^2$&$   $&$   $&$ $&$   $&$ 6x^2$&$  +x$&$  $\\
$       $&$     $&$ 6x^3$&$  +x^2$&$   $&$   $&$ $&$   $&$     $&$   x$&$  $\\
\cline{3-4} \cline{9-10}
$       $&$     $&$     $&$-12x^2$&$-4x$&$ +4$&$ $&$   $&$ 6x^2$&$ +2x$&$-2$\\
$       $&$     $&$     $&$-12x^2$&$-4x$&$ +4$&$ $&$   $&$     $&$    $&$-2$\\
\cline{4-6} \cline{9-11}
$       $&$     $&$     $&$      $&$   $&$ 0$&$  $&$   $&$     $&$    $&$   $\\
\end{tabular}

\textbf{Example 2.}\, $\sqrt{x^6\!-\!2x^5\!-\!x^4\!+\!3x^2\!+\!2x\!+\!1} =$ ? 

\begin{tabular}{lcrrrrrrrrrrrrrr} $\sqrt{}$&
$(1$&$+2x$&$+3x^2$&$$&$-x^4$&$-2x^5$&$+x^6)$&$=$&
               $\pm$&$(1$&$ +x$&$+x^2$&$-x^3)$\\
$$&$1$&$$&$$&$$&$$&$$&$$&$$&$$&$1$&$$&$$&$$\\
\cline{2-2} \cline{11-11} $$&
$$&$2x$&$+3x^2$&$$&$$&$$&$$&$$&$$&$2$&$+x$&$$&$$\\
$$&$$&$2x$&$+x^2$&$$&$$&$$&$$&$$&$$&$$&$+x$&$$&$$\\
\cline{3-4} \cline{11-12} $$&
$$&$$&$2x^2$&$$&$-x^4$&$$&$$&$$&$$&$2$&$+2x$&$+x^2$&$$\\
$$&$$&$$&$2x^2$&$+2x^3$&$+x^4$&$$&$$&$$&$$$$&&$$&$x^2$&$$\\
\cline{4-6} \cline{11-13} $$&
$$&$$&$$&$-2x^3$&$-2x^4$&$-2x^5$&$+x^6$&$$&$$&$2$&$+2x$&$+2x^2$&$-x^3$\\
$$&$$&$$&$$&$-2x^3$&$-2x^4$&$-2x^5$&$+x^6$&$$&$$&$$&$$&$$&$-x^3$\\
\cline{5-8} \cline{11-14} $$&
$$&$$&$$&$$&$$&$$&$0$&$$&$$&$$&$$&$$&$$\\
\end{tabular}

\textbf{Remark.}\, The procedure may give a Taylor series expansion of the square root, if it is not a polynomial.\, E.g. we get
$$\sqrt{1+x} = 1+\frac{1}{2}x-\frac{1}{8}x^2+\frac{1}{16}x^3
  -\frac{5}{128}x^4+-...$$

\begin{thebibliography}{9}
\bibitem{MR}{} {\em Meyers Rechenduden}.\, Erster verbesserter Neudruck.\, Bibliographisches Institut AG, Mannheim (1960).
\end{thebibliography}
%%%%%
%%%%%
\end{document}
