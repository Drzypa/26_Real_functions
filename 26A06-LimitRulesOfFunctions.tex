\documentclass[12pt]{article}
\usepackage{pmmeta}
\pmcanonicalname{LimitRulesOfFunctions}
\pmcreated{2013-03-22 14:51:21}
\pmmodified{2013-03-22 14:51:21}
\pmowner{pahio}{2872}
\pmmodifier{pahio}{2872}
\pmtitle{limit rules of functions}
\pmrecord{20}{36528}
\pmprivacy{1}
\pmauthor{pahio}{2872}
\pmtype{Theorem}
\pmcomment{trigger rebuild}
\pmclassification{msc}{26A06}
\pmclassification{msc}{30A99}
\pmsynonym{limit rules of sequences}{LimitRulesOfFunctions}
%\pmkeywords{limit rule}
\pmrelated{GrowthOfExponentialFunction}
\pmrelated{ImproperLimits}
\pmrelated{DerivativesOfSineAndCosine}
\pmrelated{ListOfCommonLimits}
\pmrelated{LimitExamples}
\pmrelated{ProductAndQuotientOfFunctionsSum}
\pmrelated{DerivationOfPlasticNumber}

% this is the default PlanetMath preamble.  as your knowledge
% of TeX increases, you will probably want to edit this, but
% it should be fine as is for beginners.

% almost certainly you want these
\usepackage{amssymb}
\usepackage{amsmath}
\usepackage{amsfonts}

% used for TeXing text within eps files
%\usepackage{psfrag}
% need this for including graphics (\includegraphics)
%\usepackage{graphicx}
% for neatly defining theorems and propositions
 \usepackage{amsthm}
% making logically defined graphics
%%%\usepackage{xypic}

% there are many more packages, add them here as you need them

% define commands here
\theoremstyle{definition}
\newtheorem{thmplain}{Theorem}
\begin{document}
\begin{thmplain}
 \, Let $f$ and $g$ be two \PMlinkname{real}{RealFunction} or complex functions.\, Suppose that there exist the limits \,$\lim_{x\to x_0}f(x)$\, and\, $\lim_{x\to x_0}g(x)$.\, Then there exist the limits\, $\lim_{x\to x_0}[f(x)\!\pm\!g(x)]$,\, $\lim_{x\to x_0}f(x)g(x)$\, and, if\, $\lim_{x\to x_0}g(x)\neq 0$,\, also\, $\lim_{x\to x_0}f(x)/g(x)$, and
\begin{enumerate}
\item $\lim_{x\to x_0}[f(x)\!\pm\!g(x)] 
\;=\; \lim_{x\to x_0}f(x)\pm\lim_{x\to x_0}g(x),$
\item $\lim_{x\to x_0}f(x)g(x) \;=\; \lim_{x\to x_0}f(x)\cdot\lim_{x\to x_0}g(x),$
\item $\lim_{x\to x_0}\frac{f(x)}{g(x)} \;=\; 
\frac{\lim_{x\to x_0}f(x)}{\lim_{x\to x_0}g(x)},$
\item $\lim_{x\to x_0}c \;=\; c
\quad\mathrm{where}\,\,c\,\,\mathrm{is\,\,a\,\,constant}.$
\end{enumerate}
\end{thmplain}

These rules are used in limit calculations and in proving the corresponding differentiation rules (sum rule, product rule etc.). 

In \PMlinkescapetext{theorem} 1, the domains of $f$ and $g$ could be any topological space (not necessarily $\mathbb{R}$ or $\mathbb{C}$).\\

There are \PMlinkescapetext{similar} limit rules of \PMlinkname{sequences}{Sequence}.\\

As well, one often needs the

\begin{thmplain}
 \,If there exists the limit\, $\lim_{x\to x_0}f(x) = a$\, and if $g$ is continuous at the point\, $x = a$, then there exists the limit\, $\lim_{x\to x_0}g(f(x))$, and 
           $$\lim_{x\to x_0}g(f(x)) \;=\; g(\lim_{x\to x_0}f(x)).$$
\end{thmplain}
%%%%%
%%%%%
\end{document}
