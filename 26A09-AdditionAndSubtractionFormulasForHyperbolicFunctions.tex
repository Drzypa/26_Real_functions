\documentclass[12pt]{article}
\usepackage{pmmeta}
\pmcanonicalname{AdditionAndSubtractionFormulasForHyperbolicFunctions}
\pmcreated{2013-03-22 17:50:45}
\pmmodified{2013-03-22 17:50:45}
\pmowner{Wkbj79}{1863}
\pmmodifier{Wkbj79}{1863}
\pmtitle{addition and subtraction formulas for hyperbolic functions}
\pmrecord{5}{40317}
\pmprivacy{1}
\pmauthor{Wkbj79}{1863}
\pmtype{Derivation}
\pmcomment{trigger rebuild}
\pmclassification{msc}{26A09}
\pmclassification{msc}{33B10}
\pmsynonym{addition and subtraction formulae for hyperbolic functions}{AdditionAndSubtractionFormulasForHyperbolicFunctions}
\pmsynonym{addition formulas for hyperbolic functions}{AdditionAndSubtractionFormulasForHyperbolicFunctions}
\pmsynonym{addition formulae for hyperbolic functions}{AdditionAndSubtractionFormulasForHyperbolicFunctions}
\pmsynonym{subtraction formulas for hyperbolic functions}{AdditionAndSubtractionFormulasForHyperbolicFunctions}
\pmsynonym{subtraction formulae for hyperbolic functions}{AdditionAndSubtractionFormulasForHyperbolicFunctions}
\pmsynonym{addition form}{AdditionAndSubtractionFormulasForHyperbolicFunctions}
\pmrelated{AdditionFormula}
\pmrelated{HyperbolicIdentities}
\pmrelated{AdditionFormulas}

\usepackage{amssymb}
\usepackage{amsmath}
\usepackage{amsfonts}
\usepackage{pstricks}
\usepackage{psfrag}
\usepackage{graphicx}
\usepackage{amsthm}
%%\usepackage{xypic}

\newcommand{\ds}{\displaystyle}
\begin{document}
The addition formulas for hyperbolic sine, hyperbolic cosine, and hyperbolic tangent will be achieved via brute \PMlinkescapetext{force}.

\begin{align*}
\sinh(x+y) & =\frac{e^{x+y}-e^{-(x+y)}}{2} \\
& =\frac{e^xe^y-e^xe^{-y}+e^xe^{-y}-e^{-x}e^{-y}}{2} \\
& =e^x\left(\frac{e^y-e^{-y}}{2}\right)+e^{-y}\left(\frac{e^x-e^{-x}}{2}\right) \\
& =(\cosh x+\sinh x)\sinh y+(\cosh y-\sinh y)\sinh x \\
& =\cosh x\sinh y+\sinh x\sinh y+\sinh x\cosh y-\sinh x\sinh y \\
& =\sinh x\cosh y+\cosh x\sinh y
\end{align*}

\begin{align*}
\cosh(x+y) & =\frac{e^{x+y}+e^{-(x+y)}}{2} \\
& =\frac{e^xe^y-e^xe^{-y}+e^xe^{-y}+e^{-x}e^{-y}}{2} \\
& =e^x\left(\frac{e^y-e^{-y}}{2}\right)+e^{-y}\left(\frac{e^x+e^{-x}}{2}\right) \\
& =(\cosh x+\sinh x)\sinh y+(\cosh y-\sinh y)\cosh x \\
& =\cosh x\sinh y+\sinh x\sinh y+\cosh x\cosh y-\cosh x\sinh y \\
& =\cosh x\cosh y+\sinh x\sinh y
\end{align*}

\begin{align*}
\tanh(x+y) & =\frac{\sinh(x+y)}{\cosh(x+y)} \\
& =\frac{\sinh x\cosh y+\cosh x\sinh y}{\cosh x\cosh y+\sinh x\sinh y} \\
& =\frac{\ds\frac{\sinh x}{\cosh x} \cdot \frac{\cosh y}{\cosh y}+\frac{\cosh x}{\cosh x} \cdot \frac{\sinh y}{\cosh y}}
        {\ds\frac{\cosh x}{\cosh x} \cdot \frac{\cosh y}{\cosh y}+\frac{\sinh x}{\cosh x} \cdot \frac{\sinh y}{\cosh y}} \\
& =\frac{\tanh x+\tanh y}{1+\tanh x\tanh y}
\end{align*}

Note that $\sinh$ and $\tanh$ are odd functions and $\cosh$ is an even function, \PMlinkname{i.e.}{Ie} $\sinh(-t)=-\sinh t$, $\tanh(-t)=-\tanh t$, and $\cosh(-t)=\cosh t$.  These facts enable us to obtain the subtraction formulas.

\[
\sinh(x-y)=\sinh(x+(-y))=\sinh x\cosh(-y)+\cosh x\sinh(-y)=\sinh x\cosh y-\cosh x\sinh y
\]

\[
\cosh(x-y)=\cosh(x+(-y))=\cosh x\cosh(-y)+\sinh x\sinh(-y)=\cosh x\cosh y-\sinh x\sinh y
\]

\[
\tanh(x-y)=\tanh(x+(-y))=\frac{\tanh x+\tanh(-y)}{1+\tanh x\tanh(-y)}=\frac{\tanh x-\tanh y}{1-\tanh x\tanh y}
\]
%%%%%
%%%%%
\end{document}
