\documentclass[12pt]{article}
\usepackage{pmmeta}
\pmcanonicalname{GrowthOfExponentialFunction}
\pmcreated{2013-03-22 14:51:32}
\pmmodified{2013-03-22 14:51:32}
\pmowner{pahio}{2872}
\pmmodifier{pahio}{2872}
\pmtitle{growth of exponential function}
\pmrecord{18}{36532}
\pmprivacy{1}
\pmauthor{pahio}{2872}
\pmtype{Theorem}
\pmcomment{trigger rebuild}
\pmclassification{msc}{26A12}
\pmclassification{msc}{26A06}
\pmrelated{MaximalNumber}
\pmrelated{LimitRulesOfFunctions}
\pmrelated{NaturalLogarithm}
\pmrelated{AsymptoticBoundsForFactorial}
\pmrelated{MinimalAndMaximalNumber}
\pmrelated{FunctionXx}
\pmrelated{Growth}
\pmrelated{LimitsOfNaturalLogarithm}
\pmrelated{DerivativeOfLimitFunctionDivergesFromLimitOfDerivatives}

\endmetadata

% this is the default PlanetMath preamble.  as your knowledge
% of TeX increases, you will probably want to edit this, but
% it should be fine as is for beginners.

% almost certainly you want these
\usepackage{amssymb}
\usepackage{amsmath}
\usepackage{amsfonts}

% used for TeXing text within eps files
%\usepackage{psfrag}
% need this for including graphics (\includegraphics)
%\usepackage{graphicx}
% for neatly defining theorems and propositions
 \usepackage{amsthm}
% making logically defined graphics
%%%\usepackage{xypic}

% there are many more packages, add them here as you need them

% define commands here
\theoremstyle{definition}
\newtheorem*{thmplain}{Theorem}
\begin{document}
\textbf{Lemma.} 
   $$\lim_{x\to\infty}\frac{x^a}{e^x} = 0$$
for all \PMlinkescapetext{constant} values of $a$.

{\em Proof.}\, Let $\varepsilon$ be any positive number.\, Then we get:

$$0 < \frac{x^a}{e^x} \leqq \frac{x^{\lceil a \rceil}}{e^x} < 
\frac{x^{\lceil a \rceil}}{\frac{x^{\lceil a\rceil+1}}{(\lceil a\rceil+1)!}}
 = \frac{(\lceil a\rceil+1)!}{x} < \varepsilon$$
as soon as\, $x > \max\{1, \frac{(\lceil a\rceil+1)!}{\varepsilon}\}$.\, Here, $\lceil\cdot\rceil$ \PMlinkescapetext{means} the ceiling function;\, $e^x$ has been estimated downwards by taking only one of the all positive \PMlinkescapetext{terms of the series expansion} 
$$e^x = 1+\frac{x}{1!}+\frac{x^2}{2!}+\cdots+\frac{x^n}{n!}+\cdots$$\\

\begin{thmplain}
 \, The \PMlinkescapetext{growth} of the real exponential function\,\, $x\mapsto b^x$\,\, exceeds all power functions, i.e.
     $$\lim_{x\to\infty}\frac{x^a}{b^x} = 0$$
with $a$ and $b$ any \PMlinkescapetext{constants},\, $b > 1$.
\end{thmplain}

{\em Proof.}\, Since\, $\ln b > 0$,\, we obtain by using the lemma the result
   $$\lim_{x\to\infty}\frac{x^a}{b^x} = 
\lim_{x\to\infty}\left(\frac{x^{\frac{a}{\ln b}}}{e^x}\right)^{\ln b} = 0^{\ln b} = 0.$$\\

\textbf{Corollary 1.}\,  $\displaystyle\lim_{x\to 0+}x\ln{x} = 0.$

{\em Proof.}\, According to the lemma we get
$$0 = \lim_{u\to\infty}\frac{-u}{e^u} = 
\lim_{x\to 0+}\frac{-\ln{\frac{1}{x}}}{\frac{1}{x}} = \lim_{x\to 0+}x\ln{x}.$$\\

\textbf{Corollary 2.}\,  $\displaystyle\lim_{x\to\infty}\frac{\ln{x}}{x} = 0.$

{\em Proof.}\, Change in the lemma\, $x$\, to\, $\ln{x}$.\\

\textbf{Corollary 3.}\,  $\displaystyle\lim_{x\to\infty}x^{\frac{1}{x}} = 1.$ \, (Cf. limit of nth root of n.)

{\em Proof.}\, By corollary 2, we can write:\, $\displaystyle x^{\frac{1}{x}} = e^{\frac{\ln{x}}{x}}\longrightarrow e^0 = 1$\, as\, $x\to\infty$ (see also theorem 2 in limit rules of functions).
%%%%%
%%%%%
\end{document}
