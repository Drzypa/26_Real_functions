\documentclass[12pt]{article}
\usepackage{pmmeta}
\pmcanonicalname{PowerFunction}
\pmcreated{2013-03-22 14:46:32}
\pmmodified{2013-03-22 14:46:32}
\pmowner{pahio}{2872}
\pmmodifier{pahio}{2872}
\pmtitle{power function}
\pmrecord{17}{36421}
\pmprivacy{1}
\pmauthor{pahio}{2872}
\pmtype{Definition}
\pmcomment{trigger rebuild}
\pmclassification{msc}{26A99}
\pmrelated{PropertiesOfTheExponential}
\pmrelated{FractionPower}
\pmrelated{CubeOfANumber}
\pmrelated{Polytrope}
\pmrelated{PowerTowerSequence}
\pmrelated{LaplaceTransformOfLogarithm}
\pmdefines{natural power function}
\pmdefines{root function}
\pmdefines{fraction power function}

\endmetadata

% this is the default PlanetMath preamble.  as your knowledge
% of TeX increases, you will probably want to edit this, but
% it should be fine as is for beginners.

% almost certainly you want these
\usepackage{amssymb}
\usepackage{amsmath}
\usepackage{amsfonts}

% used for TeXing text within eps files
%\usepackage{psfrag}
% need this for including graphics (\includegraphics)
%\usepackage{graphicx}
% for neatly defining theorems and propositions
 \usepackage{amsthm}
% making logically defined graphics
%%%\usepackage{xypic}

% there are many more packages, add them here as you need them

% define commands here
\theoremstyle{definition}
\newtheorem*{thmplain}{Theorem}
\begin{document}
A real {\em power function} \,$f\!:\,\mathbb{R}_+\to\mathbb{R}$\, has the form
 $$f(x) \;=\; x^a$$
where $a$ is a given real number.

\begin{thmplain}
 \,The power function\, $x\mapsto x^a$\, is differentiable with the derivative \,$x\mapsto ax^{a-1}$\, and strictly increasing if\, $a > 0$\, and strictly decreasing if\, $a < 0$\, (and \PMlinkescapetext{constant} 1 if\, 
$a = 0$).
\end{thmplain}

The power functions comprise the {\em natural power functions}\, $x\mapsto x^n$\, with\, $n = 0,\,1,\,2,\,\ldots$,\, the {\em root functions}\, $x\mapsto \sqrt[n]{x} = x^{\frac{1}{n}}$\, with\, $n = 1,\,2,\,3,\,\ldots$\, and other {\em fraction power functions}\, $x\mapsto x^a$\, with $a$ any fractional number.



\textbf{Note.} \,The power $x^a$ may of course be meaningful also for other than positive values of $x$, if $a$ is an integer.\, On the other hand, e.g. $(-1)^{\sqrt{2}}$ has no real values --- see the general power.
%%%%%
%%%%%
\end{document}
