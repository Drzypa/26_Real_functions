\documentclass[12pt]{article}
\usepackage{pmmeta}
\pmcanonicalname{LengthOfCurveInAMetricSpace}
\pmcreated{2013-03-22 16:50:27}
\pmmodified{2013-03-22 16:50:27}
\pmowner{Mathprof}{13753}
\pmmodifier{Mathprof}{13753}
\pmtitle{length of curve in a metric space}
\pmrecord{8}{39085}
\pmprivacy{1}
\pmauthor{Mathprof}{13753}
\pmtype{Definition}
\pmcomment{trigger rebuild}
\pmclassification{msc}{26B15}
\pmdefines{length of a curve}

\endmetadata

% this is the default PlanetMath preamble.  as your knowledge
% of TeX increases, you will probably want to edit this, but
% it should be fine as is for beginners.

% almost certainly you want these
\usepackage{amssymb}
\usepackage{amsmath}
\usepackage{amsfonts}

% used for TeXing text within eps files
%\usepackage{psfrag}
% need this for including graphics (\includegraphics)
%\usepackage{graphicx}
% for neatly defining theorems and propositions
%\usepackage{amsthm}
% making logically defined graphics
%%%\usepackage{xypic}

% there are many more packages, add them here as you need them

% define commands here

\begin{document}
Suppose that $(X,d)$ is a metric space. Let $f$ be a curve, so that
$f: [0,1] \to X$ is a continuous function, and let $0=t_0 < t_1 < \cdots < t_n=1$ and
$x_i = f(t_i)$ for $0 \le i \le n$. 
The set $\{x_0, x_1, \ldots , x_n\}$
is called a partition of the curve. 
The \emph{\PMlinkescapetext{length} of the curve} is defined to be 
the supremum over all partitions of the quantity $\sum_{i=1}^n d(x_i , x_{i-1})$.


%%%%%
%%%%%
\end{document}
