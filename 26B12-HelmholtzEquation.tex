\documentclass[12pt]{article}
\usepackage{pmmeta}
\pmcanonicalname{HelmholtzEquation}
\pmcreated{2013-03-22 13:09:09}
\pmmodified{2013-03-22 13:09:09}
\pmowner{Mathprof}{13753}
\pmmodifier{Mathprof}{13753}
\pmtitle{Helmholtz equation}
\pmrecord{11}{33592}
\pmprivacy{1}
\pmauthor{Mathprof}{13753}
\pmtype{Definition}
\pmcomment{trigger rebuild}
\pmclassification{msc}{26B12}
\pmclassification{msc}{35-00}
\pmsynonym{Helmholtz differential equation}{HelmholtzEquation}
\pmsynonym{reduced wave equation}{HelmholtzEquation}
\pmrelated{WaveEquation}
\pmrelated{PoissonsEquation}

% this is the default PlanetMath preamble.  as your knowledge
% of TeX increases, you will probably want to edit this, but
% it should be fine as is for beginners.

% almost certainly you want these
\usepackage{amssymb}
\usepackage{amsmath}
\usepackage{amsfonts}

% used for TeXing text within eps files
%\usepackage{psfrag}
% need this for including graphics (\includegraphics)
\usepackage{graphicx}
% for neatly defining theorems and propositions
%\usepackage{amsthm}
% making logically defined graphics
%%%\usepackage{xypic}

% there are many more packages, add them here as you need them

% define commands here
\newcommand{\vA}{\mathbf{A}}
\newcommand{\vB}{\mathbf{B}}
\newcommand{\vx}{\mathbf{x}}
\newcommand{\vN}{\mathbf{N}}
\newcommand{\bV}{\mathbf{V}}
\newcommand{\vnabla}{\nabla}
\begin{document}
The \emph{Helmholtz equation} is a partial differential equation which, in scalar form is $$\vnabla^2f+k^2f = 0,$$ or in vector form is $$\vnabla^2\vA+k^2\vA = 0,$$ where $\vnabla^2$ is the Laplacian.
The solutions of this equation represent the solution of the wave equation, which is of great interest in physics.

Consider a wave equation $$\frac{\partial^2\psi}{\partial t^2} = c^2\vnabla^2\psi$$
with wave speed $c$. If we look for time harmonic standing waves of frequency $\omega$,
$$\psi(\vx,t) = e^{-j\omega t}\phi(\vx)$$
we find that $\phi(x)$ satisfies the Helmholtz equation:
$$(\vnabla^2+k^2)\phi = 0$$
where $k=\omega/c$ is the wave number.

Usually the Helmholtz equation is solved by the separation of variables method, in Cartesian, spherical or cylindrical coordinates.
%%%%%
%%%%%
\end{document}
