\documentclass[12pt]{article}
\usepackage{pmmeta}
\pmcanonicalname{VersionOfTheFundamentalLemmaOfCalculusOfVariations}
\pmcreated{2013-03-22 19:12:04}
\pmmodified{2013-03-22 19:12:04}
\pmowner{pahio}{2872}
\pmmodifier{pahio}{2872}
\pmtitle{version of the fundamental lemma of calculus of variations}
\pmrecord{4}{42116}
\pmprivacy{1}
\pmauthor{pahio}{2872}
\pmtype{Theorem}
\pmcomment{trigger rebuild}
\pmclassification{msc}{26A15}
\pmclassification{msc}{26A42}
\pmsynonym{fundamental lemma of calculus of variations}{VersionOfTheFundamentalLemmaOfCalculusOfVariations}
%\pmkeywords{calculus of variations}
\pmrelated{EulerLagrangeDifferentialEquation}

% this is the default PlanetMath preamble.  as your knowledge
% of TeX increases, you will probably want to edit this, but
% it should be fine as is for beginners.

% almost certainly you want these
\usepackage{amssymb}
\usepackage{amsmath}
\usepackage{amsfonts}

% used for TeXing text within eps files
%\usepackage{psfrag}
% need this for including graphics (\includegraphics)
%\usepackage{graphicx}
% for neatly defining theorems and propositions
 \usepackage{amsthm}
% making logically defined graphics
%%%\usepackage{xypic}

% there are many more packages, add them here as you need them

% define commands here

\theoremstyle{definition}
\newtheorem*{thmplain}{Theorem}

\begin{document}
\textbf{Lemma.}\, If a real function $f$ is continuous on the interval \,$[a,\,b]$\, and if
$$\int_a^b\!f(x)\varphi(x)\,dx \;=\; 0$$
for all functions $\varphi$ continuously differentiable on the interval and vanishing at its end points, then\, 
$f(x) \,\equiv\, 0$\, on the whole interval.\\

\emph{Proof.}\, We make the antithesis that $f$ does not vanish identically.\, Then there exists a point $x_0$ of the open interval \,$(a,\,b)$\, such that\, $f(x_0) \neq 0$;\, for example\, $f(x_0) > 0$.\, The continuity of $f$ implies that there are the numbers $\alpha$ and $\beta$ such that\, $a < \alpha < x_0 < \beta < b$\, and\, $f(x) > 0$\, for all\, 
$x \in [\alpha,\,\beta]$.\, Now the function $\varphi_0$ defined by
\begin{align*}
\varphi_0(x) \;:=\; 
\begin{cases}
(x\!-\!\alpha)^2(x\!-\!\beta)^2 \quad \mbox{for}\;\; \alpha \leqq x \leqq \beta,\\
0 \qquad \mbox{otherwise}
\end{cases}
\end{align*}
fulfils the requirements for the functions $\varphi$.\, Since both $f$ and $\varphi_0$ are positive on the open interval\, $(\alpha,\,\beta)$,\, we however have
$$\int_a^b\!f(x)\varphi_0(x)\,dx \;=\; \int_\alpha^\beta\!f(x)\varphi_0(x)\,dx \;>\; 0.$$
Thus the antithesis causes a contradiction.\, Consequently, we must have\, $f(x) \,\equiv\, 0$.
%%%%%
%%%%%
\end{document}
