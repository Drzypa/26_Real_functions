\documentclass[12pt]{article}
\usepackage{pmmeta}
\pmcanonicalname{simIsAnEquivalenceRelation}
\pmcreated{2013-03-22 16:13:16}
\pmmodified{2013-03-22 16:13:16}
\pmowner{Wkbj79}{1863}
\pmmodifier{Wkbj79}{1863}
\pmtitle{$\sim$ is an equivalence relation}
\pmrecord{9}{38319}
\pmprivacy{1}
\pmauthor{Wkbj79}{1863}
\pmtype{Result}
\pmcomment{trigger rebuild}
\pmclassification{msc}{26A12}

\usepackage{amssymb}
\usepackage{amsmath}
\usepackage{amsfonts}

\usepackage{psfrag}
\usepackage{graphicx}
\usepackage{amsthm}
%%\usepackage{xypic}

\begin{document}
Note that $\sim$ as defined in the entry Landau notation is an equivalence relation on the set of all functions from $\mathbb{R}^+$ to $\mathbb{R}^+$.  This set of functions will be denoted in this entry as $F$.

\emph{\PMlinkname{Reflexive}{Reflexive}:} For any $f \in F$, $\displaystyle \lim_{x \to \infty} \frac{f(x)}{f(x)}=1$, and $f \sim f$.

\emph{Symmetric:} If $f,g \in F$ with $f \sim g$, then $\displaystyle \lim_{x \to \infty} \frac{f(x)}{g(x)}=1$.  Thus:

\begin{center}
$\begin{array}{ll}
\displaystyle \lim_{x \to \infty} \frac{g(x)}{f(x)} & \displaystyle =\lim_{x \to \infty} \frac{1}{\left( \frac{f(x)}{g(x)} \right)} \\
\\
& \displaystyle =\frac{1}{1} \\
\\
& =1 \end{array}$
\end{center}

Therefore, $g \sim f$.

\PMlinkname{\emph{Transitive}}{Transitive3}\emph{:} If $f,g,h \in F$ with $f \sim g$ and $g \sim h$, then $\displaystyle \lim_{x \to \infty} \frac{f(x)}{g(x)}=1$ and $\displaystyle \lim_{x \to \infty} \frac{g(x)}{h(x)}=1$.  Thus:

\begin{center}
$\begin{array}{ll}
\displaystyle \lim_{x \to \infty} \frac{f(x)}{h(x)} & \displaystyle =\lim_{x \to \infty} \left( \frac{f(x)}{g(x)} \cdot \frac{g(x)}{h(x)} \right) \\
\\
& =1 \cdot 1 \\
\\
& =1 \end{array}$
\end{center}

Therefore, $f \sim h$.
%%%%%
%%%%%
\end{document}
