\documentclass[12pt]{article}
\usepackage{pmmeta}
\pmcanonicalname{DirectSumOfEvenoddFunctionsexample}
\pmcreated{2013-03-22 13:34:24}
\pmmodified{2013-03-22 13:34:24}
\pmowner{mathcam}{2727}
\pmmodifier{mathcam}{2727}
\pmtitle{direct sum of even/odd functions (example)}
\pmrecord{6}{34191}
\pmprivacy{1}
\pmauthor{mathcam}{2727}
\pmtype{Example}
\pmcomment{trigger rebuild}
\pmclassification{msc}{26A06}
\pmrelated{DirectSumOfHermitianAndSkewHermitianMatrices}
\pmrelated{ProductAndQuotientOfFunctionsSum}

\endmetadata

% this is the default PlanetMath preamble.  as your knowledge
% of TeX increases, you will probably want to edit this, but
% it should be fine as is for beginners.

% almost certainly you want these
\usepackage{amssymb}
\usepackage{amsmath}
\usepackage{amsfonts}

% used for TeXing text within eps files
%\usepackage{psfrag}
% need this for including graphics (\includegraphics)
%\usepackage{graphicx}
% for neatly defining theorems and propositions
%\usepackage{amsthm}
% making logically defined graphics
%%%\usepackage{xypic}

% there are many more packages, add them here as you need them

% define commands here
\begin{document}
\newcommand{\sR}[0]{\mathbb{R}}

{\bf Example.} Direct sum of even and odd functions 

Let us define the sets
\begin{eqnarray*}
F &=& \{ f\, |\, f\,\mbox{ is a function from}\, \sR\, \mbox{ to}\, \sR \}, \\
F_+ &=& \{ f\in F \,|\, f(x)=f(-x) \,\mbox{for all}\, x\in \sR\}, \\
F_- &=& \{ f\in F \,|\, f(x)=-f(-x)\,\mbox{for all}\, x\in \sR\}.
\end{eqnarray*}

In other words, $F$ contain all functions from $\sR$ to $\sR$, $F_+\subset F$
contain all even functions, and $F_-\subset F$ contain all odd functions.
All of these spaces  have a natural vector space structure:
for functions  $f$ and $g$ we define
$f+g$ as the function $x\mapsto f(x)+g(x)$. Similarly, if $c$ is
a real constant, then $cf$ is the
function $x\mapsto cf(x)$.  With these operations, the zero vector
is the mapping $x\mapsto 0$.

We claim that $F$ is the direct sum of $F_+$ and $F_-$, i.e.,
that
\begin{eqnarray}
\label{eq10}
F &=& F_+ \oplus F_-.
\end{eqnarray}

To prove this claim, let us first note that  $F_\pm$ are vector subspaces of $F$.
Second, given an arbitrary function $f$ in $F$, we can define
\begin{eqnarray*}
f_+(x) &=& \frac{1}{2}\big( f(x) + f(-x) \big), \\
f_-(x) &=& \frac{1}{2}\big( f(x) - f(-x) \big).
\end{eqnarray*}
Now $f_+$ and $f_-$ are even and odd functions and $f=f_+ + f_-$.
Thus any function in $F$ can be split into two components $f_+$ and $f_-$,
such that $f_+ \in F_+$ and $f_-\in F_-$.
To show that the sum is direct, suppose $f$ is an element in $F_+\cap F_-$.
Then we have that $f(x)=-f(-x)=-f(x)$, so $f(x)=0$ for all $x$, i.e., $f$ is
the zero vector in $F$. We have established equation \ref{eq10}.
%%%%%
%%%%%
\end{document}
