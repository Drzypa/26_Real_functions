\documentclass[12pt]{article}
\usepackage{pmmeta}
\pmcanonicalname{AngleMultiplicationAndDivisionFormulaeForTangent}
\pmcreated{2013-03-22 17:00:15}
\pmmodified{2013-03-22 17:00:15}
\pmowner{rspuzio}{6075}
\pmmodifier{rspuzio}{6075}
\pmtitle{angle multiplication and division formulae for tangent}
\pmrecord{8}{39286}
\pmprivacy{1}
\pmauthor{rspuzio}{6075}
\pmtype{Result}
\pmcomment{trigger rebuild}
\pmclassification{msc}{26A09}

\endmetadata

% this is the default PlanetMath preamble.  as your knowledge
% of TeX increases, you will probably want to edit this, but
% it should be fine as is for beginners.

% almost certainly you want these
\usepackage{amssymb}
\usepackage{amsmath}
\usepackage{amsfonts}

% used for TeXing text within eps files
%\usepackage{psfrag}
% need this for including graphics (\includegraphics)
%\usepackage{graphicx}
% for neatly defining theorems and propositions
%\usepackage{amsthm}
% making logically defined graphics
%%%\usepackage{xypic}

% there are many more packages, add them here as you need them

% define commands here

\begin{document}
From the angle addition formula for the tangent, we may derive formulae
for tangents of multiples of angles:

\begin{align*}
\tan (2x) &= {2 \tan x \over 1 - \tan^2 x} \\
\tan (3x) &= {3 \tan x - \tan^3 x \over 1 - 3 \tan^2 x} \\
\tan (4x) &= {4 \tan x - 3 \tan^3 x \over 1 - 6 \tan^2 x + \tan^4 x}
\end{align*}

These formulae may be derived from a recursion.  Write $\tan x = w$ and
write $\tan (nx) = u_n / v_n$ where the $u$'s and the $v$'s are
polynomials in $w$.  Then we have the initial values $u_1 = w$ and 
$v_1 = 1$ and the recursions
\begin{align*}
u_{n+1} &= u_n + w v_n \\
v_{n+1} &= v_n - w u_n ,
\end{align*}
which follow from the addition formula.  Moreover, if we know the tangent
of an angle and are interested in finding the tangent of a multiple of
that angle, we may use our recursions directly without first having to derive 
the multiple angle formulae.  From these recursions, one may show that the
$u$'s will only involve odd powers of $w$ and the $v$'s will only involve
even powers of $w$.

Proceeding in the opposite direction, one may consider bisecting an angle.
Solving for $\tan x$ in the duplication formula above, one arrives at the
following half-angle formula:
\[
\tan \left( {x \over 2} \right) =
\sqrt{ 1 + {1 \over \tan^2 x}} - {1 \over \tan x}
\]
Expressing the tangent in terms of sines and cosines and simplifying, one 
finds the following equivalent formulae:
\[
\tan \left( {x \over 2} \right) =
{1 - \cos x \over \sin x} =
{\sin x \over 1 + \cos x} =
\pm\sqrt{ 1 - \cos x \over 1 + \cos x }
\]
%%%%%
%%%%%
\end{document}
