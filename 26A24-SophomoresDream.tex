\documentclass[12pt]{article}
\usepackage{pmmeta}
\pmcanonicalname{SophomoresDream}
\pmcreated{2014-07-20 10:46:23}
\pmmodified{2014-07-20 10:46:23}
\pmowner{pahio}{2872}
\pmmodifier{pahio}{2872}
\pmtitle{Sophomore's dream}
\pmrecord{11}{88119}
\pmprivacy{1}
\pmauthor{pahio}{2872}
\pmtype{Derivation}
\pmclassification{msc}{26A24}

% this is the default PlanetMath preamble.  as your knowledge
% of TeX increases, you will probably want to edit this, but
% it should be fine as is for beginners.

% almost certainly you want these
\usepackage{amssymb}
\usepackage{amsmath}
\usepackage{amsfonts}

% need this for including graphics (\includegraphics)
\usepackage{graphicx}
% for neatly defining theorems and propositions
\usepackage{amsthm}

% making logically defined graphics
%\usepackage{xypic}
% used for TeXing text within eps files
%\usepackage{psfrag}

% there are many more packages, add them here as you need them

% define commands here

\begin{document}
The integral
\begin{align}
I := \int_0^1 x^x\, dx
\end{align}
may be expanded to a rapidly converging series as follows.

Changing the integrand to a power of $e$ and using the power series expansion of the 
exponential function gives us
\begin{align}
I = \int_0^1 e^{x\ln x}\, dx 
= \int_0^1 \sum_{n=0}^\infty\frac{(x\ln x)^n}{n!}\, dx.
\end{align}
Here the series is uniformly convergent on $[0,1]$ and may be 
integrated termwise:
\begin{align}
I = \sum_{n=0}^\infty\int_0^1\frac{x^n(\ln x)^n}{n!}dx.
\end{align}
The last equation of the \PMlinkname{parent entry}{ExampleOfDifferentiationUnderIntegralSign} 
then gives in the case $m = n$ from (3) the result
\begin{align}
I = \sum_{n=0}^\infty\frac{(-1)^n}{(n\!+\!1)^{n+1}},
\end{align}
i.e.,
\begin{align}
\int_0^1 x^x\, dx \;=\; 
1-\frac{1}{2^2}+\frac{1}{3^3}-\frac{1}{4^4}+-\ldots
\end{align}

Cf. the \PMlinkname{function $x^x$}{FunctionXX}.\\

Since the series (5) satisfies the conditions of 
\PMlinkname{Leibniz' theorem for alternating series}{LeibnizEstimateForAlternatingSeries}, 
one may easily estimate the error made when a partial sum of (5) is used for the exact value of the integral 
$I$.\, If one for example takes for $I$ the sum of nine first
terms, the first omitted term is $-\frac{1}{10^{10}}$; thus the 
error is negative and its absolute value less than $10^{-10}$.\\




\end{document}
