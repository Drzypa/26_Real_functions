\documentclass[12pt]{article}
\usepackage{pmmeta}
\pmcanonicalname{AllDerivativesOfSincAreBoundedBy1}
\pmcreated{2013-03-22 15:39:03}
\pmmodified{2013-03-22 15:39:03}
\pmowner{matte}{1858}
\pmmodifier{matte}{1858}
\pmtitle{all derivatives of sinc are bounded by $1$}
\pmrecord{10}{37583}
\pmprivacy{1}
\pmauthor{matte}{1858}
\pmtype{Result}
\pmcomment{trigger rebuild}
\pmclassification{msc}{26A06}

% this is the default PlanetMath preamble.  as your knowledge
% of TeX increases, you will probably want to edit this, but
% it should be fine as is for beginners.

% almost certainly you want these
\usepackage{amssymb}
\usepackage{amsmath}
\usepackage{amsfonts}
\usepackage{amsthm}

\usepackage{mathrsfs}

% used for TeXing text within eps files
%\usepackage{psfrag}
% need this for including graphics (\includegraphics)
%\usepackage{graphicx}
% for neatly defining theorems and propositions
%
% making logically defined graphics
%%%\usepackage{xypic}

% there are many more packages, add them here as you need them

% define commands here

\newcommand{\sR}[0]{\mathbb{R}}
\newcommand{\sC}[0]{\mathbb{C}}
\newcommand{\sN}[0]{\mathbb{N}}
\newcommand{\sZ}[0]{\mathbb{Z}}

 \usepackage{bbm}
 \newcommand{\Z}{\mathbbmss{Z}}
 \newcommand{\C}{\mathbbmss{C}}
 \newcommand{\F}{\mathbbmss{F}}
 \newcommand{\R}{\mathbbmss{R}}
 \newcommand{\Q}{\mathbbmss{Q}}



\newcommand*{\norm}[1]{\lVert #1 \rVert}
\newcommand*{\abs}[1]{| #1 |}



\newtheorem{thm}{Theorem}
\newtheorem{defn}{Definition}
\newtheorem{prop}{Proposition}
\newtheorem{lemma}{Lemma}
\newtheorem{cor}{Corollary}
\begin{document}
Let us show that all derivatives of $\operatorname{sinc}$ are bounded by $1$. 

First of all, let us \PMlinkescapetext{point} out that $\operatorname{sinc}(t)\le 1$ is 
bounded by the Jordan's inequality. To \PMlinkescapetext{bound} the derivatives, let 
us write $\operatorname{sinc}$ as a Fourier integral,
$$
  \operatorname{sinc}(t) = {1\over 2} \int_{-1}^1 e^{ixt} \, dx.
$$
Let $k=1,2,\ldots$. Then
$$
{d^k\over dt^k} \operatorname{sinc}(t) = {1\over 2} \int_{-1}^1 (ix)^k e^{ixt} \, dx.
$$
and
\begin{eqnarray*}
\left\vert {d^k\over dt^k} \operatorname{sinc}(t) \right\vert &=& \left\vert {1\over 2} \int_{-1}^1 (ix)^k e^{ixt} \, dx \right\vert \\
&\le& {1\over 2} \int_{-1}^1 \vert (ix)^k e^{ixt} \vert \, dx \\
&\le& {1\over 2} \int_{-1}^1 \vert x \vert^k \, dx \\
&\le& {1\over 2} \cdot 2 \int_0^1 \vert x \vert^k \, dx \\
&\le& \int_0^1  x^k  \, dx \\
&\le& \frac{1}{k+1} \\
&<& 1.
\end{eqnarray*}
%%%%%
%%%%%
\end{document}
