\documentclass[12pt]{article}
\usepackage{pmmeta}
\pmcanonicalname{RigorousDefinitionOfTrigonometricFunctions}
\pmcreated{2013-03-22 16:22:11}
\pmmodified{2013-03-22 16:22:11}
\pmowner{CWoo}{3771}
\pmmodifier{CWoo}{3771}
\pmtitle{rigorous definition of trigonometric functions}
\pmrecord{9}{38508}
\pmprivacy{1}
\pmauthor{CWoo}{3771}
\pmtype{Derivation}
\pmcomment{trigger rebuild}
\pmclassification{msc}{26A09}
\pmrelated{TrigonometricFormulasFromSeries}

\endmetadata

% this is the default PlanetMath preamble.  as your knowledge
% of TeX increases, you will probably want to edit this, but
% it should be fine as is for beginners.

% almost certainly you want these
\usepackage{amssymb}
\usepackage{amsmath}
\usepackage{amsfonts}

% used for TeXing text within eps files
%\usepackage{psfrag}
% need this for including graphics (\includegraphics)
%\usepackage{graphicx}
% for neatly defining theorems and propositions
%\usepackage{amsthm}
% making logically defined graphics
%%%\usepackage{xypic}

% there are many more packages, add them here as you need them

% define commands here

\begin{document}
It is possible to define the trigonometric functions rigorously by means of a
process based upon the angle addition identities.  A sketch of how this is done
is provided below.

To begin, define a sequence $\{c_n\}_{n=1}^\infty$ by the initial condition 
$c_1 = 1$ and the recursion
 \[c_{n+1} = 1 - \sqrt{1 - {c_n \over 2}}.\]
Likewise define a sequence $\{s_n\}_{n=1}^\infty$ by the conditions $s_1 = 1$ and
 \[s_{n+1} = \sqrt{c_n \over 2}.\]
(In both equations above, we take the positive square root.) 
It may be shown that both of these sequences are strictly decreasing and approach $0$.

Next, define a sequence of $2 \times 2$ matrices as follows:
 \[m_n = \left( \begin{matrix} 1 - c_n & s_n \\ - s_n & 1 - c_n \end{matrix} \right) \]
Using the recursion relations which define $c_n$ and $s_n$, it may be shown that
$m_{n+1}^2 = m_n$,  More grenerally, using induction, this can be generalised to
$m_{n+k}^{2^k} = m_n$.

It is easy to check that the product of any two matrices of the form
 \[\left( \begin{matrix} x & y \\ -y & x \end{matrix} \right) \]
is of the same form.  Hence, for any integers $k$ and $n$, the matrix 
$m_n^k$ will be of this form.  We can therefore define functions $S$ and
$C$ from rational numbers whose denominator is a power of two to real
numbers by the following equation:
 \[\left( \begin{matrix} C \left( {n \over 2^k} \right) & S \left( {n 
\over 2^k} \right) \\ - S \left( {n \over 2^k} \right) & C \left( {n 
\over 2^k} \right) \end{matrix} \right) = \left( \begin{matrix} 1 - c_k 
& s_k \\ - s_k & 1 - c_k \end{matrix} \right)^n.\] 

From the recursion relations, we may prove the following identities:
\begin{eqnarray*} S^2 (r) + C^2 (r) &=& 1 \cr
S (p + q) &=& S(p) C(q) + S(q) C(p) \cr
C (p + q) &=& C(p) C(q) - S(p) S(q) \end{eqnarray*}

From the fact that $c_n \to 0$ and $s_n \to 0$ as $n \to \infty$, it 
follows that, if $\{p_n\}_{n=1}^\infty$ and $\{q_n\}_{n=1}^\infty$ are 
two sequences of rational numbers whose denominators are powers of two
such that $\lim_{n \to \infty} p_n = \lim_{n \to \infty} q_n$, then
$\lim_{n \to \infty} C(p_n) = \lim_{n \to \infty} C(q_n)$ and
$\lim_{n \to \infty} S(p_n) = \lim_{n \to \infty} S(q_n)$.  Therefore,
we may define functions by the conditions that, for any convergent
series of rational numbers $\{r_n\}_{n=0}^\infty$ whose denominators
are powers of two,
 \[\cos \left( \pi  \lim_{n \to \infty} r_n \right) = 
\lim_{n \to \infty} C(r_n)\]
and
 \[\sin \left( \pi  \lim_{n \to \infty} r_n \right) = 
\lim_{n \to \infty} S(r_n).\] 
By continuity, we see that these functions satisfy the angle addition
identities.

\textbf{Application}.  Let us use the definitions above to find $\sin(\frac{\pi}{2})$ and $\cos(\frac{\pi}{2})$.  Let $r_i:=\frac{1}{2}$ for every positive integer $i$.  Then we need to find $C(\frac{1}{2})$ and $S(\frac{1}{2})$.  We use the matrix above defining $C$ and $S$, and set $n=k=1$:
 \[\left( \begin{matrix} C \left( {\frac{1}{2}} \right) & S \left( {\frac{1}{2}} \right) \\ - S \left( {\frac{1}{2}} \right) & C \left( {\frac{1}{2}} \right) \end{matrix} \right) = \left( \begin{matrix} 1 - c_1 
& s_1 \\ - s_1 & 1 - c_1 \end{matrix} \right) = \left( \begin{matrix} 0 & 1 \\ -1 & 0 \end{matrix} \right).\] 
As a result, $\cos(\frac{\pi}{2})= \cos(\pi \lim_{i \to \infty} \frac{1}{2}) = 
\lim_{i \to \infty} C(\frac{1}{2}) = C(\frac{1}{2}) = 0$.  Similarly, $\sin(\frac{\pi}{2})= 1$.
%%%%%
%%%%%
\end{document}
