\documentclass[12pt]{article}
\usepackage{pmmeta}
\pmcanonicalname{ProofOfQuotientRule}
\pmcreated{2013-03-22 12:38:58}
\pmmodified{2013-03-22 12:38:58}
\pmowner{drini}{3}
\pmmodifier{drini}{3}
\pmtitle{proof of quotient rule}
\pmrecord{5}{32915}
\pmprivacy{1}
\pmauthor{drini}{3}
\pmtype{Proof}
\pmcomment{trigger rebuild}
\pmclassification{msc}{26A06}

\usepackage{graphicx}
%%%\usepackage{xypic} 
\usepackage{bbm}
\newcommand{\Z}{\mathbbmss{Z}}
\newcommand{\C}{\mathbbmss{C}}
\newcommand{\R}{\mathbbmss{R}}
\newcommand{\Q}{\mathbbmss{Q}}
\newcommand{\mathbb}[1]{\mathbbmss{#1}}
\newcommand{\figura}[1]{\begin{center}\includegraphics{#1}\end{center}}
\newcommand{\figuraex}[2]{\begin{center}\includegraphics[#2]{#1}\end{center}}
\newtheorem{dfn}{Definition}
\begin{document}
Let $F(x) = f(x)/g(x)$.  Then

\begin{eqnarray*}
F'(x) & = & \lim_{h \to0} \frac{F(x+h)-F(x)}{h} = \lim_{h \to0} \frac{\frac{f(x+h)}{g(x+h)}-\frac{f(x)}{g(x)}}{h}\\
& = & \lim_{h\to0} \frac{f(x+h)g(x)-f(x)g(x+h)}{hg(x+h)g(x)}
\end{eqnarray*}

Like the product rule, the key to this proof is subtracting and adding the same quantity.  We separate $f$ and $g$ in the above expression by subtracting and adding the term $f(x)g(x)$ in the numerator.

\begin{eqnarray*}
F'(x) & = & \lim_{h \to0} \frac{f(x+h)g(x)-f(x)g(x)+f(x)g(x)-f(x)g(x+h)}{hg(x+h)g(x)} \\
& = & \lim_{h \to0} \frac{g(x)\frac{f(x+h)-f(x)}{h}-f(x) \frac{g(x+h)-g(x)}{h}}{g(x+h)g(x)} \\
& = & \frac{\lim_{h \to0}g(x) \cdot \lim_{h \to0} \frac{f(x+h)-f(x)}{h}-\lim_{h \to0} f(x) \cdot \lim_{h \to0} \frac{g(x+h)-g(x)}{h}}{\lim_{h \to0}g(x+h) \cdot \lim_{h \to0}g(x)} \\
& = & \frac{g(x)f'(x)-f(x)g'(x)}{\lbrack g(x) \rbrack ^2}
\end{eqnarray*}
%%%%%
%%%%%
\end{document}
