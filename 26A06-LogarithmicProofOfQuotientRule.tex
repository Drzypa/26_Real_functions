\documentclass[12pt]{article}
\usepackage{pmmeta}
\pmcanonicalname{LogarithmicProofOfQuotientRule}
\pmcreated{2013-03-22 16:18:51}
\pmmodified{2013-03-22 16:18:51}
\pmowner{Wkbj79}{1863}
\pmmodifier{Wkbj79}{1863}
\pmtitle{logarithmic proof of quotient rule}
\pmrecord{7}{38439}
\pmprivacy{1}
\pmauthor{Wkbj79}{1863}
\pmtype{Proof}
\pmcomment{trigger rebuild}
\pmclassification{msc}{26A06}
\pmclassification{msc}{97D40}
\pmrelated{LogarithmicProofOfProductRule}

\endmetadata

\usepackage{amssymb}
\usepackage{amsmath}
\usepackage{amsfonts}

\usepackage{psfrag}
\usepackage{graphicx}
\usepackage{amsthm}
%%\usepackage{xypic}

\begin{document}
Following is a proof of the quotient rule using the natural logarithm, the chain rule, and implicit differentiation.  Note that circular reasoning does not occur, as each of the concepts used can be proven independently of the quotient rule.

\begin{proof}
Let $f$ and $g$ be differentiable functions and $\displaystyle y=\frac{f(x)}{g(x)}$.  Then $\ln y=\ln f(x)-\ln g(x)$.  Thus, $\displaystyle \frac{1}{y} \cdot \frac{dy}{dx}=\frac{f'(x)}{f(x)}-\frac{g'(x)}{g(x)}$.  Therefore,

\begin{center}
$\begin{array}{rl}
\displaystyle \frac{dy}{dx} & \displaystyle = y \left( \frac{f'(x)}{f(x)}-\frac{g'(x)}{g(x)} \right) \\
& \\
& \displaystyle = \frac{f(x)}{g(x)} \left( \frac{f'(x)}{f(x)}-\frac{g'(x)}{g(x)} \right) \\
& \\
& \displaystyle = \frac{f'(x)}{g(x)}-\frac{f(x)g'(x)}{(g(x))^2} \\
& \\
& \displaystyle = \frac{g(x)f'(x)-f(x)g'(x)}{(g(x))^2}. \end{array}$
\end{center}
\end{proof}

Once students are familiar with the natural logarithm, the chain rule, and implicit differentiation, they typically have no problem following this proof of the quotient rule.  Actually, with some prompting, they can produce a proof of the quotient rule \PMlinkescapetext{similar} to this one.  This exercise is a great way for students to review many concepts from calculus.
%%%%%
%%%%%
\end{document}
