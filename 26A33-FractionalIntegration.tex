\documentclass[12pt]{article}
\usepackage{pmmeta}
\pmcanonicalname{FractionalIntegration}
\pmcreated{2013-03-22 16:17:47}
\pmmodified{2013-03-22 16:17:47}
\pmowner{rspuzio}{6075}
\pmmodifier{rspuzio}{6075}
\pmtitle{fractional integration}
\pmrecord{17}{38416}
\pmprivacy{1}
\pmauthor{rspuzio}{6075}
\pmtype{Definition}
\pmcomment{trigger rebuild}
\pmclassification{msc}{26A33}
\pmsynonym{fractional integral}{FractionalIntegration}

% this is the default PlanetMath preamble.  as your knowledge
% of TeX increases, you will probably want to edit this, but
% it should be fine as is for beginners.

% almost certainly you want these
\usepackage{amssymb}
\usepackage{amsmath}
\usepackage{amsfonts}

% used for TeXing text within eps files
%\usepackage{psfrag}
% need this for including graphics (\includegraphics)
%\usepackage{graphicx}
% for neatly defining theorems and propositions
%\usepackage{amsthm}
% making logically defined graphics
%%%\usepackage{xypic}

% there are many more packages, add them here as you need them

% define commands here

\begin{document}
The basic idea of "Riemann-Liouville" type fractional integration comes from the following observation:

Given any integrable function $f:{\mathbb R}\mapsto {\mathbb R}$ in one variable, we have the following Cauchy Integration Formula:

\begin{displaymath}D^{-n}(f)(x)=\int_{t_n=0}^x dt_n\ldots \int_{t_1=0}^{t_2}
f(t_1)\,dt_1 =\frac{1}{(n-1)!} \int_{t=0}^x f(t)(x-t)^{n-1}\,dt
\end{displaymath}

when switching the index from integer $n$ to non-integer $\alpha$ gives the ideas of the following definitions: 

{\bf Definition 1:}  {\rm Left-Hand Riemann-Liouville Integration}

\begin{displaymath}I^{\alpha}_L (f)(s,t)=
\frac{1}{\Gamma(\alpha)}\int_{u=s}^tf(u)(t-u)^{\alpha-1}\,du
=\int_{u=s}^t f(u)\,dg^{\alpha}_t(u) \end{displaymath}

where \begin{displaymath}g^{\alpha}_t(u)=\frac{t^{\alpha}-(t-u)^{\alpha}}
{\Gamma(\alpha+1)}\end{displaymath}

{\bf Definition 2:}  {\rm Right-Hand Riemann-Liouville Integration}

\begin{displaymath}I^{\alpha}_R (f)(s,t)=
\frac{1}{\Gamma(\alpha)}\int_{u=s}^tf(u)(u-s)^{\alpha-1}\,du
=\int_{u=s}^t f(u)\,dh^{\alpha}_t(u) \end{displaymath}

where \begin{displaymath}h^{\alpha}_t(u)=\frac{s^{\alpha}+(u-s)^{\alpha}}
{\Gamma(\alpha+1)}\end{displaymath}

{\bf Definition 3:}  {\rm Riesz Potential}

\begin{displaymath}I^{\alpha}_C (f)(s,t;p)=
\frac{1}{\Gamma(\alpha)}\int_{u=s}^tf(u)|u-p|^{\alpha-1}\,du
=\int_{u=s}^t f(u)\,dr^{\alpha}_p(u) \end{displaymath}

where \begin{displaymath}r^{\alpha}_p(u)=\frac{p^{\alpha}+{\rm sign}(u-p)
|u-p|^{\alpha}}{\Gamma(\alpha+1)}\end{displaymath},

${\rm sign}(x)=1$ for $x>0$, ${\rm sign}(x)=0$ for $x=0$, ${\rm sign}(x)=-1$ for $x<0$

and $\Gamma(x)$ is the gamma function of $x$
%%%%%
%%%%%
\end{document}
