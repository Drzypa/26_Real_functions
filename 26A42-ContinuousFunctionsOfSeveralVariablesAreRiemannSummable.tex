\documentclass[12pt]{article}
\usepackage{pmmeta}
\pmcanonicalname{ContinuousFunctionsOfSeveralVariablesAreRiemannSummable}
\pmcreated{2013-03-22 15:07:56}
\pmmodified{2013-03-22 15:07:56}
\pmowner{paolini}{1187}
\pmmodifier{paolini}{1187}
\pmtitle{continuous functions of several variables are Riemann summable}
\pmrecord{9}{36875}
\pmprivacy{1}
\pmauthor{paolini}{1187}
\pmtype{Theorem}
\pmcomment{trigger rebuild}
\pmclassification{msc}{26A42}

\endmetadata

% this is the default PlanetMath preamble.  as your knowledge
% of TeX increases, you will probably want to edit this, but
% it should be fine as is for beginners.

% almost certainly you want these
\usepackage{amssymb}
\usepackage{amsmath}
\usepackage{amsfonts}

% used for TeXing text within eps files
%\usepackage{psfrag}
% need this for including graphics (\includegraphics)
%\usepackage{graphicx}
% for neatly defining theorems and propositions
\usepackage{amsthm}
% making logically defined graphics
%%%\usepackage{xypic}

% there are many more packages, add them here as you need them

% define commands here
\newcommand{\R}{\mathbb R}
\newtheorem{theorem}{Theorem}
\begin{document}
\begin{theorem}
Continuous functions defined on compact subsets of $\R^n$ are Riemann integrable.
\end{theorem}
\begin{proof}
Let $D\subset \R^n$ be a compact subset of $\R^n$ and let $f\colon D\to \R$ be a continuous function. 
Since $f$ is defined on a compact set, $f$ is uniformly continuous i.e.\ given $\epsilon>0$ there exists $\delta>0$ such that $|x-y|\le \delta \Rightarrow |f(x)-f(y)|\le \epsilon$. 
Let $R>0$ be large enough so that $D\subset (-R,R)^n$ (such an $R$ exists because $D$ is bounded).
Let $P$ be a polyrectangle such that $D\subset \cup P \subset (-R,R)^n$ and such that every rectangle $R$ in $P$ has diameter which is less then $\delta$. So one has $\sup_R f(x)-\inf_R f(x) \le \epsilon$ and hence
\[
 S^*(f,P)-S_*(f,P)\le \epsilon \sum_{Q\in P} \mathrm{meas} (Q) \le \epsilon \mathrm{meas}(P) \le \epsilon \mathrm{meas}[-R,R]^n = \epsilon 2^n R^n.
\]
Letting $\epsilon\to 0$ one concludes that $S^*(f)=S_*(f)$.
\end{proof}
%%%%%
%%%%%
\end{document}
