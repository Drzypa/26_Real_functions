\documentclass[12pt]{article}
\usepackage{pmmeta}
\pmcanonicalname{IncreasingdecreasingmonotoneFunction}
\pmcreated{2013-03-22 13:36:05}
\pmmodified{2013-03-22 13:36:05}
\pmowner{Koro}{127}
\pmmodifier{Koro}{127}
\pmtitle{increasing/decreasing/monotone function}
\pmrecord{12}{34228}
\pmprivacy{1}
\pmauthor{Koro}{127}
\pmtype{Definition}
\pmcomment{trigger rebuild}
\pmclassification{msc}{26A06}
\pmclassification{msc}{26A48}
\pmdefines{increasing}
\pmdefines{decreasing}
\pmdefines{strictly increasing}
\pmdefines{strictly decreasing}
\pmdefines{monotone}
\pmdefines{monotonic}
\pmdefines{strictly monotone}
\pmdefines{strictly monotonic}
\pmdefines{weakly increasing}
\pmdefines{weakly decreasing}
\pmdefines{strongly increasing}
\pmdefines{strongly decreasing}
\pmdefines{strongly monotone}
\pmdefines{weakly monotone}
\pmdefines{stronly mono}

% this is the default PlanetMath preamble.  as your knowledge
% of TeX increases, you will probably want to edit this, but
% it should be fine as is for beginners.

% almost certainly you want these
\usepackage{amssymb}
\usepackage{amsmath}
\usepackage{amsfonts}

% used for TeXing text within eps files
%\usepackage{psfrag}
% need this for including graphics (\includegraphics)
%\usepackage{graphicx}
% for neatly defining theorems and propositions
%\usepackage{amsthm}
% making logically defined graphics
%%%\usepackage{xypic}

% there are many more packages, add them here as you need them

% define commands here
\begin{document}
\newcommand{\sR}[0]{\mathbb{R}}

\PMlinkescapeword{increasing}
\PMlinkescapeword{strictly increasing}
\PMlinkescapeword{decreasing}
\PMlinkescapeword{strictly decreasing}
\PMlinkescapeword{monotone}
\PMlinkescapeword{strictly monotone}

{\bf Definition}
Let $A$ be a subset of $\sR$, and let $f$ be a function from $f:A\to \sR$.
Then
\begin{enumerate}
\item $f$ is \emph{increasing} or \emph{weakly increasing}, if
$x\le y$ implies that $f(x)\le f(y)$ (for all $x$ and $y$ in $A$).
\item $f$ is \emph{strictly increasing} or \emph{strongly increasing}, if
$x< y$ implies that $f(x)< f(y)$.
\item $f$ is \emph{decreasing} or \emph{weakly decreasing}, if
$x\le y$ implies that $f(x)\ge f(y)$.
\item $f$ is \emph{strictly decreasing} or \emph{strongly decreasing} if
$x< y$ implies that $f(x)> f(y)$.
\item $f$ is \emph{monotone},
if $f$ is either increasing or decreasing.
\item $f$ is \emph{strictly monotone} or \emph{strongly monotone},
if $f$ is either strictly increasing or strictly decreasing.
\end{enumerate}

{\bf Theorem} Let $X$ be a bounded or unbounded open interval of $\sR$.
In other words, let $X$ be an interval of the form $X=(a,b)$, where $a,b\in\sR\cup\{-\infty,\infty\}$.
Futher, let $f:X\to \sR$ be a monotone function. 
\begin{enumerate}
\item The set of points where $f$ is discontinuous is at most 
countable \cite{aliprantis, rudin}. 
\item [Lebesgue] $f$ is differentiable almost 
everywhere (\cite{jones}, pp. 514). 
\end{enumerate}

\begin{thebibliography}{9}
\bibitem{aliprantis}
 C.D. Aliprantis, O. Burkinshaw, \emph{Principles of Real Analysis},
 2nd ed., Academic Press, 1990.
\bibitem{rudin}
 W. Rudin, \emph{Principles of Mathematical Analysis}, McGraw-Hill Inc., 1976.
 \bibitem{jones}
F. Jones, \emph{Lebesgue Integration on Euclidean Spaces}, 
Jones and Barlett Publishers, 1993.
 \end{thebibliography}
%%%%%
%%%%%
\end{document}
