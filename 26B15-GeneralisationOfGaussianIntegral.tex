\documentclass[12pt]{article}
\usepackage{pmmeta}
\pmcanonicalname{GeneralisationOfGaussianIntegral}
\pmcreated{2013-03-22 18:43:36}
\pmmodified{2013-03-22 18:43:36}
\pmowner{pahio}{2872}
\pmmodifier{pahio}{2872}
\pmtitle{generalisation of Gaussian integral}
\pmrecord{6}{41495}
\pmprivacy{1}
\pmauthor{pahio}{2872}
\pmtype{Derivation}
\pmcomment{trigger rebuild}
\pmclassification{msc}{26B15}
\pmclassification{msc}{26A36}
\pmrelated{SubstitutionNotation}

% this is the default PlanetMath preamble.  as your knowledge
% of TeX increases, you will probably want to edit this, but
% it should be fine as is for beginners.

% almost certainly you want these
\usepackage{amssymb}
\usepackage{amsmath}
\usepackage{amsfonts}

% used for TeXing text within eps files
%\usepackage{psfrag}
% need this for including graphics (\includegraphics)
%\usepackage{graphicx}
% for neatly defining theorems and propositions
%\usepackage{amsthm}
% making logically defined graphics
%%%\usepackage{xypic}

% there are many more packages, add them here as you need them

% define commands here
\newcommand{\sijoitus}[2]%
{\operatornamewithlimits{\Big/}_{\!\!\!#1}^{\,#2}}
\begin{document}
The integral
$$\int_0^\infty\!e^{-x^2}\cos{tx}\,dx \;:=\; w(t)$$
is a generalisation of the Gaussian integral \,$w(0) = \frac{\sqrt{\pi}}{2}$.\, For evaluating it we first form its derivative which may be done by \PMlinkname{differentiating under the integral sign}{DifferentiationUnderIntegralSign}:
$$w'(t) \;=\; \int_0^\infty\!e^{-x^2}(-x)\sin{tx}\,dx 
\;=\; \frac{1}{2}\int_0^\infty\!e^{-x^2}(-2x)\sin{tx}\,dx$$
Using integration by parts this yields
$$w'(t) \;=\; \frac{1}{2}\!\sijoitus{x=0}{\quad\infty}\!e^{-x^2}\sin{tx}-\frac{t}{2}\int_0^\infty\!e^{-x^2}\cos{tx}\,dx
\,=\, \frac{1}{2}(0-0)-\frac{t}{2}\int_0^\infty\!e^{-x^2}\cos{tx}\,dx \,=\, -\frac{t}{2}w(t).$$
Thus $w(t)$ satisfies the linear differential equation
$$\frac{dw}{dt} \;=\; -\frac{1}{2}tw,$$
where one can \PMlinkname{separate the variables}{SeparationOfVariables} and integrate:
$$\int\!\frac{dw}{w} \;=\; -\frac{1}{2}\int\!t\,dt.$$
So,\, $\ln{w} \,=\, -\frac{1}{4}t^2+\ln{C}$,\, i.e.\, $w = w(t) = Ce^{-\frac{1}{4}t^2}$,\, 
and since there is the initial condition\, $w(0) = \frac{\sqrt{\pi}}{2}$, we obtain the result
$$w(t) \;=\; \frac{\sqrt{\pi}}{2}e^{-\frac{1}{4}t^2}.$$


%%%%%
%%%%%
\end{document}
