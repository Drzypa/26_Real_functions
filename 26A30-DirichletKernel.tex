\documentclass[12pt]{article}
\usepackage{pmmeta}
\pmcanonicalname{DirichletKernel}
\pmcreated{2013-03-22 14:11:53}
\pmmodified{2013-03-22 14:11:53}
\pmowner{mathwizard}{128}
\pmmodifier{mathwizard}{128}
\pmtitle{Dirichlet kernel}
\pmrecord{10}{35629}
\pmprivacy{1}
\pmauthor{mathwizard}{128}
\pmtype{Definition}
\pmcomment{trigger rebuild}
\pmclassification{msc}{26A30}
\pmrelated{ExampleOfTelescopingSum}

\endmetadata

% this is the default PlanetMath preamble.  as your knowledge
% of TeX increases, you will probably want to edit this, but
% it should be fine as is for beginners.

% almost certainly you want these
\usepackage{amssymb}
\usepackage{amsmath}
\usepackage{amsfonts}

% used for TeXing text within eps files
%\usepackage{psfrag}
% need this for including graphics (\includegraphics)
%\usepackage{graphicx}
% for neatly defining theorems and propositions
%\usepackage{amsthm}
% making logically defined graphics
%%%\usepackage{xypic}

% there are many more packages, add them here as you need them

% define commands here
\begin{document}
The \emph{Dirichlet \PMlinkescapetext{kernel}} $D_n$ of order $n$ is defined as
$$D_n(t)=\sum_{k=-n}^ne^{ikt}.$$
It can be represented as
$$D_n(t)=\frac{\sin\left(n+\frac{1}{2}\right)t}{\sin\frac{t}{2}}.$$
\textbf{Proof:} It is
\begin{align*}
\sum_{k=-n}^ne^{ikt}&= e^{-int}\frac{1-e^{i(2n+1)t}}{1-e^{it}}\\
&=\frac{e^{i\left(n+\frac{1}{2}\right)t}-e^{-i\left(n+\frac{1}{2}\right)t}} {e^{i\frac{t}{2}}-e^{-i\frac{t}{2}}}\\
&=\frac{\sin\left(n+\frac{1}{2}\right)t}{\sin\frac{t}{2}}.\qquad\qquad\Box
\end{align*}
The Dirichlet kernel arises in the analysis of periodic functions because for any function $f$ of period $2\pi$, the convolution of $D_N$ and $f$ results in the Fourier-series approximation of order $n$:
$$(D_N*f)(x)=\frac{1}{2\pi}\int_{-\pi}^\pi f(y)D_n(x-y)dy=\sum_{k=-n}^n\hat{f}(k)e^{ikx}.$$
%%%%%
%%%%%
\end{document}
