\documentclass[12pt]{article}
\usepackage{pmmeta}
\pmcanonicalname{ALectureOnThePartialFractionDecompositionMethod}
\pmcreated{2013-03-22 15:39:12}
\pmmodified{2013-03-22 15:39:12}
\pmowner{alozano}{2414}
\pmmodifier{alozano}{2414}
\pmtitle{a lecture on the partial fraction decomposition method}
\pmrecord{5}{37586}
\pmprivacy{1}
\pmauthor{alozano}{2414}
\pmtype{Feature}
\pmcomment{trigger rebuild}
\pmclassification{msc}{26A42}
\pmclassification{msc}{28-00}
\pmrelated{ALectureOnIntegrationBySubstitution}
\pmrelated{ALectureOnIntegrationByParts}
\pmrelated{ALectureOnTrigonometricIntegralsAndTrigonometricSubstitution}
\pmrelated{PartialFractionsOfExpressions}
\pmrelated{PartialFractionsForPolynomials}

% this is the default PlanetMath preamble.  as your knowledge
% of TeX increases, you will probably want to edit this, but
% it should be fine as is for beginners.

% almost certainly you want these
\usepackage{amssymb}
\usepackage{amsmath}
\usepackage{amsthm}
\usepackage{amsfonts}

% used for TeXing text within eps files
%\usepackage{psfrag}
% need this for including graphics (\includegraphics)
%\usepackage{graphicx}
% for neatly defining theorems and propositions
%\usepackage{amsthm}
% making logically defined graphics
%%%\usepackage{xypic}

% there are many more packages, add them here as you need them

% define commands here

\newtheorem{thm}{Theorem}[section]
\newtheorem{conj}[thm]{Conjecture}
\newtheorem{cor}[thm]{Corollary}
\newtheorem{lem}[thm]{Lemma}
\newtheorem{prop}[thm]{Proposition}
\newtheorem{defn}[thm]{Definition}
\newtheorem{remark}[thm]{Remark}
\newtheorem{exe}{Problem}
\newtheorem*{exe1}{Problem 1}
\newtheorem*{exe2}{Problem 2}
\newtheorem*{exe3}{Problem 3}
\newtheorem*{exe4}{Problem 4}

\theoremstyle{definition}
\newtheorem{exa}[thm]{Example}

\def\notdiv{\ \mathbin{\mkern-8mu|\!\!\!\smallsetminus}}
\newcommand{\Qoft}{\mathbb{Q}(T)}  %use in linux
\newcommand{\done}{\Box} %use in linux
\newcommand{\R}{\ensuremath{\mathbb{R}}}
\newcommand{\C}{\ensuremath{\mathbb{C}}}
\newcommand{\Z}{\ensuremath{\mathbb{Z}}}
\newcommand{\Q}{\mathbb{Q}}
\newcommand{\peri}{\operatorname{Perimeter}}
\newcommand{\lc}{\lim_{x\to c}}
\newcommand{\lzero}{\lim_{x\to 0}}
\newcommand{\lhzero}{\lim_{h\to 0}}
\newcommand{\linf}{\lim_{x\to \infty}}
\begin{document}
\section{Integrating Rational Functions}

A rational function is a function of the form
$y=\frac{p(x)}{q(x)}$ where $p(x)$ and $q(x)$ are polynomials
(with real coefficients). Here we are interested in how to solve
the integral
$$\int \frac{p(x)}{q(x)} dx$$
We already know how to integrate some functions of this type. As
we know, the chain rule tells us that the derivative of $y=\ln
(g(x))$ is $y'=\frac{g'(x)}{g(x)}$, where $g(x)$ is any other
function. Therefore: \begin{eqnarray}\label{log}\int
\frac{g'(x)}{g(x)}dx=\ln|g(x)|+C.
\end{eqnarray}
\begin{exa}
$$\int \frac{2x+1}{x^2+x+1} dx = \ln|x^2+x+1|+C.$$
\end{exa}
\begin{exa}
Equation (\ref{log}) may also be used to integrate any function of
the form $y=\frac{a}{bx+c}$, for any constants $a,b,c$. Indeed:
$$\int \frac{a}{bx+c}dx=\frac{a}{b}\int
\frac{b}{bx+c}dx=\frac{a}{b}\ln|bx+c| +C$$ or alternatively use a
substitution $u=bx+c$.
\end{exa}
\begin{exa}
Using a substitution we may also integrate any function of the
form $y=\frac{a}{(bx+c)^n}$, namely $u=bx+c$ does the job. For
example, using $u=2x+1$, $du=2dx$:
$$\int
\frac{3}{(2x+1)^4}dx=\frac{3}{2}\int\frac{1}{u^4}du=-\frac{1}{2}\frac{1}{u^3}+C=-\frac{1}{2(2x+1)^3}+C.$$
\end{exa}
\begin{exa}
The derivative of the arc tangent function, $y=\arctan x$, is
$y'=\frac{1}{1+x^2}$. Therefore:
$$\int \frac{1}{1+x^2} dx= \arctan x + C.$$
Thus, for any constant $a$, using a substitution $x=au$ one may
integrate
$$\int \frac{1}{a^2+x^2} dx= \frac{1}{a}\arctan \frac{x}{a} + C.$$
You may also use the arctangent function to integrate other
functions, by completing the square of the denominator. For
example, $x^2+2x+2=1+(x+1)^2$, thus:
$$\int \frac{1}{x^2+2x+2}dx=\int
\frac{1}{1+(x+1)^2}dx=\arctan(x+1)+C.$$
\end{exa}
\begin{exa}
The arctangent also allows us to integrate another family of
functions, namely:
$$\int \frac{bx+c}{a^2+x^2}dx.$$
The trick is to use the favorite strategy of Napoleon: {\it divide
and conquer}, i.e. we break the fraction into a sum of two (here
we pick $a=1$ for simplicity):
$$\int \frac{bx+c}{1+x^2}dx=\int \frac{bx}{1+x^2}dx+\int \frac{c}{1+x^2}dx=\frac{b}{2}\ln|1+x^2|+c\arctan x + C.$$
\end{exa}
\section{Partial Fraction Decomposition}
The objective of this method is to reduce any integral of the type
$\int \frac{p(x)}{q(x)}dx$ to a sum of integrals of the types
described in the previous sections. For example:
\begin{exa}
We would like to solve the following integral:
$$\int \frac{2}{x^2+3x-4}dx.$$
First, we factor the denominator: $$x^2+3x-4=(x+4)(x-1)$$ In order
to integrate, we are going to express the fraction in the
integrand as a sum:
$$\frac{2}{x^2+3x-4}=\frac{A}{x+4}+\frac{B}{x-1}$$
for some constants $A,B$ to be determined. The right hand side
(after realizing a common denominator) adds up to
$\frac{A(x-1)+B(x+4)}{(x+4)(x-1)}$. Therefore, in order to have an
equality we need the numerators to be equal:
\begin{eqnarray}\label{eq1}A(x-1)+B(x+4)=2\end{eqnarray} for {\bf all} values of $x$ (i.e. this
should be an equality of polynomials). Thus, we can substitute
values of $x$ and obtain equations relating $A$ and $B$ and
hopefully we'll be able to determine the value of the constants.
The easiest values to pick are the roots of the denominator of the
original fraction. In this case, when we plug $x=1$ in Eq.
(\ref{eq1}) we get $5B=2$, and so $B=2/5$. When we plug $x=-4$ we
obtain $-5A=2$ and so $A=-2/5$, and we are done! Now we can finish
the integral: \begin{eqnarray*}\int
\frac{2}{x^2+3x-4}dx &=& \int\frac{-2/5}{x+4}dx+\int\frac{2/5}{x-1}dx\\
&=&
-\frac{2}{5}\ln|x+4|+\frac{2}{5}\ln|x-1|+C=\frac{2}{5}\ln|\frac{x-1}{x+4}|
+C.\end{eqnarray*}
\end{exa}
{\bf The method.} The goal of the method, as explained above, is
to express any fraction $\frac{p(x)}{q(x)}$ as the sum of {\it
partial fractions} of the types discussed in the previous section.
\begin{enumerate}
\item If the degree of $p(x)$ is larger than the degree of $q(x)$
then use polynomial division to divide and obtain a quotient
$t(x)$ and remainder $r(x)$ polynomials such that
$p(x)=q(x)t(x)+r(x)$. Thus
$$\frac{p(x)}{q(x)}= t(x) + \frac{r(x)}{q(x)}$$
where the degree of $r$ is lower than the degree of $q$. Now use
the partial fraction decomposition with $r(x)/q(x)$.

\item Factor the denominator, $q(x)$, into irreducible polynomials
(over $\R$). Thus, we may express $q(x)$ as a product of linear
polynomials (perhaps to a power other than $1$) and irreducible
quadratic polynomials.

\item If a linear factor $(x-a)$ to the {\bf first} power appears
in the denominator of $q(x)$, the partial fraction decomposition
should have a term $\frac{A}{(x-a)}$, for some constant $A$ to be
determined.

\item If a linear factor to the {\bf nth} power, say $(x-b)^n$
appears in the denominator of $q(x)$, the partial fraction
decomposition should have terms
$$\frac{B_1}{(x-b)}+\frac{B_2}{(x-b)^2}+\ldots+\frac{B_n}{(x-b)^n}$$ for some
$n$ constants $B_1, B_2, \ldots, B_n$ to be determined.

\item If a quadratic polynomial $ax^2+bx+c$ to the $n$th power appears in the
factorization of $q(x)$, i.e. $(ax^2+bx+c)^n$ is a factor of $q(x)$, then the partial fraction decomposition
should have terms
$$\frac{C_1x+D_1}{ax^2+bx+c}+\frac{C_2x+D_2}{(ax^2+bx+c)^2}+\ldots+\frac{C_nx+D_n}{(ax^2+bx+c)^n}$$
for some constants $C_i,D_i$ to be determined.

\item Once you have the sum of all appropriate partial fractions
(see above), group together all these partial fractions into one
fraction $\frac{P(x)}{q(x)}$ (use a minimum common multiple for
the denominator! The minimum common multiple will actually be
$q(x)$). In order to have an equality you need to find appropriate
constants $A,B,\ldots$ such that $p(x)=P(x)$. For this, plug
values of $x$ to obtain equations relating the constants.
\end{enumerate}

\begin{exa}
Suppose we want to find the partial fraction decomposition of:
$$\frac{3x+2}{(x-1)(x-2)(x-3)^2(1+x^2)}$$
Then, we need constants $A,B,C,D,E,F$ such that
$$\frac{3x+2}{(x-1)(x-2)(x-3)^2(1+x^2)}=\frac{A}{x-1} + \frac{B}{x-2}+\frac{C}{x-3}+\frac{D}{(x-3)^2}+\frac{Ex+F}{1+x^2}.$$
The next step would be to add up all the partial fractions into
one big fraction $$\frac{P(x)}{(x-1)(x-2)(x-3)^2(1+x^2)}$$ and
find constants $A,B,C,D,E,F$ such that $P(x)=3x+2$ for all $x$.
\end{exa}
%%%%%
%%%%%
\end{document}
