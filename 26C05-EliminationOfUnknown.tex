\documentclass[12pt]{article}
\usepackage{pmmeta}
\pmcanonicalname{EliminationOfUnknown}
\pmcreated{2013-03-22 19:20:27}
\pmmodified{2013-03-22 19:20:27}
\pmowner{pahio}{2872}
\pmmodifier{pahio}{2872}
\pmtitle{elimination of unknown}
\pmrecord{10}{42289}
\pmprivacy{1}
\pmauthor{pahio}{2872}
\pmtype{Algorithm}
\pmcomment{trigger rebuild}
\pmclassification{msc}{26C05}
\pmclassification{msc}{13P10}
\pmclassification{msc}{12D99}
\pmsynonym{elementary method of elimination}{EliminationOfUnknown}
%\pmkeywords{elimination}
%\pmkeywords{bivariate polynomial}
\pmrelated{FactorizationOfPrimitivePolynomial}

\endmetadata

% this is the default PlanetMath preamble.  as your knowledge
% of TeX increases, you will probably want to edit this, but
% it should be fine as is for beginners.

% almost certainly you want these
\usepackage{amssymb}
\usepackage{amsmath}
\usepackage{amsfonts}

% used for TeXing text within eps files
%\usepackage{psfrag}
% need this for including graphics (\includegraphics)
%\usepackage{graphicx}
% for neatly defining theorems and propositions
 \usepackage{amsthm}
% making logically defined graphics
%%%\usepackage{xypic}

% there are many more packages, add them here as you need them

% define commands here

\theoremstyle{definition}
\newtheorem*{thmplain}{Theorem}

\begin{document}
Consider the simultaneous polynomial equations
\begin{align}
\begin{cases}
a(x,\,y) \;=:\; \sum_{i=0}^ma_i(y)x^i \;=\; 0,\\
b(x,\,y) \;=:\; \sum_{j=0}^nb_j(y)x^j \;=\; 0
\end{cases}
\end{align}
in two unknowns $x$ and $y$, where e.g.\, $m \ge n$.\, It is possible to eliminate one of the unknowns from (1), i.e. derive an \PMlinkname{equivalent}{Equivalent3} pair of polynomial equations
\begin{align*}
\begin{cases}
f(y) \;=\; 0,\\
g(x,\,y) \;=\; 0.
\end{cases}
\end{align*}

First we form the polynomial
\begin{align}
c(x,\,y) \;=:\; b_n(y)\,a(x,\,y)-a_m(y)\,x^{m-n}\,b(x,\,y),
\end{align}
the degree of which is less than $m$.\, When\, $(x_0,\,y_0)$\, is a solution of (1), then it satisfies 
\begin{align}
\begin{cases}
c(x,\,y) \;=\; 0\\
b(x,\,y) \;=\; 0.
\end{cases}
\end{align}
On the other hand, when\, $(x_1,\,y_1)$\, is a solution of (3), then (2) implies that it satisfies also (1), except possibly in the case\, $b_n(y_1) = 0$.\\
We can continue similarly until we arrive at a pair of equations
\begin{align}
\begin{cases}
f(y) \;=\; 0,\\
g(x,\,y) \;=\; 0
\end{cases}
\end{align}
Substituting the roots of the former of the equations (4) into the latter one, which in practice is usually of first degree with respect to $x$, one can get the corresponding values of $x$.\, Hence one obtains all solutions of the original system of equations (1).\, Since the cases\, $b_n(y_1) = 0$\, may yield wrong solutions, one should check them by substituting into (1).\\

\textbf{Note.}\, One can derive from the equations (1) an equation of lower degree also by eliminating from them the constant terms; the terms of resulting equation have as common factor $x$ or its higher power, which is removed by dividing.


%%%%%
%%%%%
\end{document}
