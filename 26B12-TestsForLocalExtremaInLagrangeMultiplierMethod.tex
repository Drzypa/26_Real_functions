\documentclass[12pt]{article}
\usepackage{pmmeta}
\pmcanonicalname{TestsForLocalExtremaInLagrangeMultiplierMethod}
\pmcreated{2013-03-22 15:28:52}
\pmmodified{2013-03-22 15:28:52}
\pmowner{stevecheng}{10074}
\pmmodifier{stevecheng}{10074}
\pmtitle{tests for local extrema in Lagrange multiplier method}
\pmrecord{6}{37336}
\pmprivacy{1}
\pmauthor{stevecheng}{10074}
\pmtype{Result}
\pmcomment{trigger rebuild}
\pmclassification{msc}{26B12}
\pmclassification{msc}{49-00}
\pmclassification{msc}{49K35}
%\pmkeywords{Lagrange multipler}
%\pmkeywords{Lagrangian multiplier}
%\pmkeywords{second derivative test}
%\pmkeywords{local extrema}
%\pmkeywords{Hessian}
\pmrelated{HessianForm}
\pmrelated{RelationsBetweenHessianMatrixAndLocalExtrema}
\pmdefines{projected Hessian}
\pmdefines{reduced Hessian}

\usepackage{amssymb}
\usepackage{amsmath}
\usepackage{amsfonts}
%\usepackage{amsthm}
\usepackage{enumerate}

% used for TeXing text within eps files
%\usepackage{psfrag}
% need this for including graphics (\includegraphics)
%\usepackage{graphicx}
% making logically defined graphics
%%%\usepackage{xypic}

% define commands here
\newcommand{\complex}{\mathbb{C}}
\newcommand{\real}{\mathbb{R}}
\newcommand{\rat}{\mathbb{Q}}
\newcommand{\nat}{\mathbb{N}}

\providecommand{\abs}[1]{\lvert#1\rvert}
\providecommand{\absW}[1]{\left\lvert#1\right\rvert}
\providecommand{\absB}[1]{\Bigl\lvert#1\Bigr\rvert}
\providecommand{\norm}[1]{\lVert#1\rVert}
\providecommand{\normW}[1]{\left\lVert#1\right\rVert}
\providecommand{\normB}[1]{\Bigl\lVert#1\Bigr\rVert}
\providecommand{\defnterm}[1]{\emph{#1}}

\DeclareMathOperator{\D}{D}
\DeclareMathOperator{\linspan}{span}
\begin{document}
Let $U$ be open in $\real^n$,
and $f\colon U \to \real$, $g\colon U \to \real^m$ be twice continuously differentiable functions.
Assume that $p \in U$ is a stationary point
for $f$ on $M = g^{-1}(\{ 0 \})$,
and $\D g$ has full rank everywhere\footnote{Actually, only $\D g(p)$ needs to have full rank, and the arguments presented here continue to hold in that case, although $M$ would not necessarily be a manifold then.} on $M$.
Then we know that $p$ is the solution
to the Lagrange multiplier system
\begin{equation}\label{lagrange}
\D f(p) = \lambda \cdot \D g(p)\,,
\end{equation}
for a Lagrange multiplier vector $\lambda = (\lambda_1, \dotsc, \lambda_m)$.

Our aim is to develop an analogue of the second derivative test for the stationary point $p$.

The most straightforward way to proceed is to consider a coordinate chart $\alpha\colon V \to M$
for the manifold $M$, and consider the Hessian of the function $f \circ \alpha\colon V \to \real^n$
at $0 = \alpha^{-1}(p)$.  This Hessian is in fact just the Hessian form of $f \colon M \to \real$
expressed in the coordinates of the chart $\alpha$.  But the whole point of using Lagrange multipliers
is to avoid calculating coordinate charts directly, so we find an equivalent expression
for $\D^2 (f\circ \alpha)(0)$ in terms of $\D^2 f(p)$ without mentioning derivatives of $\alpha$.

To do this, we differentiate $f \circ \alpha$ twice using the chain rule and product
 rule\footnote{Note that the ``product'' operation involved (second equality of \eqref{second-derivative})
is the operation of \emph{composition of two linear mappings}.
Think hard about this if you are not sure; it took me several tries to get this formula right,
since multi-variable iterated derivatives have a complicated structure.}.
To reduce clutter, from now on we use the prime notation for derivatives rather than $\D$.
\begin{equation}\label{second-derivative}
\begin{split}
(f \circ \alpha)''(0) &= ((f' \circ \alpha) \cdot \alpha')'(0)  \\
&= \bigl ( (f'' \circ \alpha) \cdot \alpha' \cdot \alpha' + (f' \circ \alpha) \cdot \alpha'' \bigr) (0) \\
&= f''(\alpha(0)) \cdot \alpha'(0) \cdot \alpha'(0) + f'(\alpha(0)) \cdot \alpha''(0) \\
&= f''(p) \cdot \alpha'(0) \cdot \alpha'(0) + f'(p) \cdot \alpha''(0)\,.
\end{split}
\end{equation}

If we interpret $(f \circ \alpha)''(0)$ as a bilinear mapping of vectors $u,v \in \real^{n-m}$,
then formula \eqref{second-derivative} really means
\begin{equation}\label{bilinear}
(f \circ \alpha)''(0) \cdot (u, v) = 
f''(p) \cdot \bigl(\alpha'(0)\cdot u, \alpha'(0) \cdot v\bigr) + f'(p) \cdot \bigl(\alpha''(0) \cdot (u, v)\bigr)\,.
\end{equation}
To obtain the quadratic form, we set $v = u$; also we abbreviate the vector $\alpha'(0) \cdot u$ by $h$,
which belongs to the tangent space $\mathrm{T}_p M$ of $M$ at $p$. So,
\begin{equation}\label{quadratic}
(f \circ \alpha)''(0) \cdot u^2 = 
f''(p) \cdot h^2 + f'(p) \cdot \bigl(\alpha''(0) \cdot u^2 \bigr)\,.
\end{equation}
Na\"ively, we might think that $(f \circ \alpha)''(0)$ is simply
$f''(p)$ restricted to the tangent space $\mathrm{T}_p M$.
This happens to be the first term in \eqref{quadratic}, but there is also
an additional contribution by the second term involving $\alpha''(0)$;
intuitively, $\alpha''(0)$ is the curvature of the surface (manifold) $M$,
``changing the geometry'' of the graph of $f$.

But the second term of \eqref{quadratic} still involves $\alpha$.
To eliminate it, we differentiate the equation $g \circ \alpha = 0$ twice.
\begin{equation}\label{g-second-derivative}
0 = (g \circ \alpha)''(0) = g''(p) \cdot \alpha'(0) \cdot \alpha'(0) + g'(p) \cdot \alpha''(0)\,.
\end{equation}
(It is derived the same way as \eqref{second-derivative} but with $f$ replaced by $g$.)
Now we can substitute \eqref{g-second-derivative} and \eqref{lagrange}
in \eqref{second-derivative} to eliminate the term $f'(p) \cdot \alpha''(0)$:
\begin{equation}\label{ff-second-derivative}
\begin{split}
(f \circ \alpha)''(0) &= f''(p) \cdot \alpha'(0) \cdot \alpha'(0) + \lambda \cdot g'(p) \cdot \alpha''(0) \\
&= f''(p) \cdot \alpha'(0) \cdot \alpha'(0) - \lambda \cdot g''(p) \cdot \alpha'(0) \cdot \alpha'(0)\,,
\end{split}
\end{equation}
or expressed as a quadratic form,
\begin{equation}\label{ff-quadratic}
(f \circ \alpha)''(0) \cdot u^2 = f''(p) \cdot h^2 - \lambda \cdot g''(p) \cdot h^2\,.
\end{equation}

Thus, to understand the nature of the stationary point $p$, we can study the modified Hessian:
\begin{equation}\label{lagrange-hessian}
f''(p) - \lambda \cdot g''(p)\,, \quad \text{ restricted to $\mathrm{T}_p M$. }
\end{equation}
For example, if this bilinear form is positive definite, then $p$ is a local minimum,
and if it is negative definite, then $p$ is a local maximum, and so on.
All the tests that apply to the usual Hessian in $\real^n$ apply to the modified Hessian \eqref{lagrange-hessian}.

In coordinates of $\real^n$, the modified Hessian \eqref{lagrange-hessian}
takes the form
\begin{equation}\label{lagrange-hessian-coord}
\sum_{i=1}^n \sum_{j=1}^n \left( \left.\frac{\partial^2 f}{\partial x^i \partial x^j}\right|_p - \sum_{k=1}^m \lambda_k \, \left.\frac{\partial^2 g^k}{\partial x^i \partial x^j}\right|_p \right)  h^i h^j\,.
\end{equation}
We emphasize that the vector $h$ can be restricted to lie in the tangent space $\mathrm{T}_p M$,
when studying the stationary point $p$ of $f$ restricted to $M$.

In matrix form \eqref{lagrange-hessian-coord}
can be written
\begin{equation}\label{lagrange-hessian-matrix}
B= \begin{bmatrix}
\dfrac{\partial^2 f}{\partial x^i \partial x^j} - \sum\limits_{k=1}^m \lambda_k \, \dfrac{\partial^2 g^k}{\partial x^i \partial x^j} 
\end{bmatrix}_{ij}\,.
\end{equation}
But again, the test vector $h$ need only lie on $\mathrm{T}_p M$,
so if we want to apply positive/negative definiteness tests for matrices,
they should instead be applied to the \emph{projected} or \emph{reduced} Hessian:
\begin{equation}
Z^\mathrm{t} B Z
\end{equation}
where the columns of the $n \times (n-m)$ matrix $Z$ form a \emph{basis}
for $\mathrm{T}_p M = \ker g'(p) \subset \real^n$.
%%%%%
%%%%%
\end{document}
