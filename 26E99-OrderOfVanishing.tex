\documentclass[12pt]{article}
\usepackage{pmmeta}
\pmcanonicalname{OrderOfVanishing}
\pmcreated{2013-03-22 17:57:15}
\pmmodified{2013-03-22 17:57:15}
\pmowner{pahio}{2872}
\pmmodifier{pahio}{2872}
\pmtitle{order of vanishing}
\pmrecord{6}{40453}
\pmprivacy{1}
\pmauthor{pahio}{2872}
\pmtype{Definition}
\pmcomment{trigger rebuild}
\pmclassification{msc}{26E99}
\pmsynonym{vanishing order}{OrderOfVanishing}
\pmrelated{Multiplicity}
\pmrelated{OsculatingCurve}

% this is the default PlanetMath preamble.  as your knowledge
% of TeX increases, you will probably want to edit this, but
% it should be fine as is for beginners.

% almost certainly you want these
\usepackage{amssymb}
\usepackage{amsmath}
\usepackage{amsfonts}

% used for TeXing text within eps files
%\usepackage{psfrag}
% need this for including graphics (\includegraphics)
%\usepackage{graphicx}
% for neatly defining theorems and propositions
 \usepackage{amsthm}
% making logically defined graphics
%%%\usepackage{xypic}

% there are many more packages, add them here as you need them

% define commands here

\theoremstyle{definition}
\newtheorem*{thmplain}{Theorem}

\begin{document}
\textbf{Definition.}\, Let $x_0$ be a \PMlinkname{zero}{ZeroOfAFunction} of the real function $\Delta$.\, The {\em order of vanishing} of $\Delta$ at $x_0$ is $n$, if\, $\displaystyle\lim_{x\to x_0}\frac{\Delta(x)}{x^n}$ has a non-zero finite value.

Usually, $x_0$ of the definition is 0.\\

\textbf{Example.}\, If the curves \, $y = f(x)$\, and\, $y = g(x)$\, have in the point \,$(x_0,\,y_0)$\, the order of contact $n$, then the difference \, $\Delta(h) := g(x_0+h)-f(x_0+h)$\, of the ordinates has $n\!+\!1$-order of vanishing.
%%%%%
%%%%%
\end{document}
