\documentclass[12pt]{article}
\usepackage{pmmeta}
\pmcanonicalname{NablaActingOnProducts}
\pmcreated{2013-03-22 15:27:05}
\pmmodified{2013-03-22 15:27:05}
\pmowner{pahio}{2872}
\pmmodifier{pahio}{2872}
\pmtitle{nabla acting on products}
\pmrecord{11}{37300}
\pmprivacy{1}
\pmauthor{pahio}{2872}
\pmtype{Topic}
\pmcomment{trigger rebuild}
\pmclassification{msc}{26B12}
\pmclassification{msc}{26B10}
\pmrelated{Nabla}
\pmrelated{NablaNabla}
\pmdefines{gradient of vector}
\pmdefines{divergence of dyad product}
\pmdefines{curl of dyad product}

% this is the default PlanetMath preamble.  as your knowledge
% of TeX increases, you will probably want to edit this, but
% it should be fine as is for beginners.

% almost certainly you want these
\usepackage{amssymb}
\usepackage{amsmath}
\usepackage{amsfonts}

% used for TeXing text within eps files
%\usepackage{psfrag}
% need this for including graphics (\includegraphics)
%\usepackage{graphicx}
% for neatly defining theorems and propositions
 \usepackage{amsthm}
% making logically defined graphics
%%%\usepackage{xypic}

% there are many more packages, add them here as you need them

% define commands here

\theoremstyle{definition}
\newtheorem*{thmplain}{Theorem}
\begin{document}
Let $f$, $g$ be differentiable scalar fields and $\vec{u}$, $\vec{v}$  differentiable vector fields in some domain of $\mathbb{R}^3$.\, There are following formulae:

\begin{itemize}

\item Gradient of a product function\\
$\nabla(fg) = (\nabla f)g+(\nabla g)f$

\item Divergence of a scalar-multiplied vector\\
$\nabla\cdot(f\vec{u}) = (\nabla f)\cdot\vec{u}+(\nabla\cdot\vec{u})f$

\item Curl of a scalar-multiplied vector\\
$\nabla\!\times\!(f\vec{u}) = (\nabla f)\times\vec{u}+(\nabla\!\times\!\vec{u})f$

\item Divergence of a vector product\\
$\nabla\cdot(\vec{u}\!\times\!\vec{v}) = 
  (\nabla\!\times\!\vec{u})\cdot\vec{v}-(\nabla\!\times\!\vec{v})\cdot\vec{u}$

\item Curl of a vector product\\
$\nabla\!\times\!(\vec{u}\!\times\!\vec{v}) = 
(\vec{v}\cdot\nabla)\vec{u}-(\vec{u}\cdot\nabla)\vec{v}
-(\nabla\cdot\vec{u})\vec{v}+(\nabla\cdot\vec{v})\vec{u}$

\item Gradient of a scalar product\\
$\nabla(\vec{u}\cdot\vec{v})\, =\, 
(\vec{v}\cdot\nabla)\vec{u}+(\vec{u}\cdot\nabla)\vec{v}
+\vec{v}\!\times\!(\nabla\!\times\!\vec{u})
+\vec{u}\!\times\!(\nabla\!\times\!\vec{v})$\\
or, using dyads,\\
$\nabla(\vec{u}\cdot\vec{v}) = (\nabla\vec{u})\cdot\vec{v}+(\nabla\vec{v})\cdot\vec{u}$

\item Gradient of a vector product\\
$\nabla(\vec{u}\!\times\!\vec{v}) = (\nabla\vec{u})\!\times\!\vec{v}-(\nabla\vec{v})\!\times\!\vec{u}$

\item Divergence of a dyad product\\
$\nabla\cdot(\vec{u}\,\vec{v}) = 
(\nabla\!\cdot\!\vec{u})\,\vec{v}+\vec{u}\cdot\nabla\vec{v}$

\item Curl of a dyad product\\
$\nabla\!\times\!(\vec{u}\,\vec{v}) = 
(\nabla\!\times\!\vec{u})\,\vec{v}-\vec{u}\times\!\nabla\vec{v}$
\end{itemize}

\textbf{Explanations}
\begin{enumerate}
 \item $\vec{v}\cdot\nabla$ means the operator\, 
$v_x\frac{\partial}{\partial x}+v_y\frac{\partial}{\partial y}
+v_z\frac{\partial}{\partial z}$.
 \item The {\em gradient of a vector} $\vec{w}$ is defined as the dyad\, 
$\nabla\vec{w} := \vec{i}\,\frac{\partial\vec{w}}{\partial x}
+\vec{j}\,\frac{\partial\vec{w}}{\partial y}
+\vec{k}\,\frac{\partial\vec{w}}{\partial z}$.
 \item The {\em divergence} and the {\em curl} of a dyad product are defined by the equation\\
$\nabla\!*\!(\vec{u}\vec{v}) := 
\vec{i}\!*\!\frac{\partial(\vec{u}\vec{v})}{\partial x}
\!+\!\vec{j}\!*\!\frac{\partial(\vec{u}\vec{v})}{\partial y}
\!+\!\vec{k}\!*\!\frac{\partial(\vec{u}\vec{v})}{\partial z}$,\, where the asterisks are dots or crosses and the partial derivatives of the dyad product \PMlinkescapetext{mean} the expression\, 
$\frac{\partial(\vec{u}\vec{v})}{\partial x} = 
\frac{\partial\vec{u}}{\partial x}\vec{v}+
\vec{u}\frac{\partial\vec{v}}{\partial x}$\, and so on.
\end{enumerate}
%%%%%
%%%%%
\end{document}
