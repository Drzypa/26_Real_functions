\documentclass[12pt]{article}
\usepackage{pmmeta}
\pmcanonicalname{Nabla}
\pmcreated{2013-03-22 14:00:20}
\pmmodified{2013-03-22 14:00:20}
\pmowner{stevecheng}{10074}
\pmmodifier{stevecheng}{10074}
\pmtitle{nabla}
\pmrecord{7}{34847}
\pmprivacy{1}
\pmauthor{stevecheng}{10074}
\pmtype{Definition}
\pmcomment{trigger rebuild}
\pmclassification{msc}{26A06}
\pmrelated{gradient}
\pmrelated{NablaActingOnProducts}
\pmrelated{Gradient}
\pmrelated{AlternateCharacterizationOfCurl}
\pmdefines{$\nabla$}

\endmetadata

% this is the default PlanetMath preamble.  as your knowledge
% of TeX increases, you will probably want to edit this, but
% it should be fine as is for beginners.

% almost certainly you want these
\usepackage{amssymb}
\usepackage{amsmath}
\usepackage{amsfonts}

% used for TeXing text within eps files
%\usepackage{psfrag}
% need this for including graphics (\includegraphics)
%\usepackage{graphicx}
% for neatly defining theorems and propositions
%\usepackage{amsthm}
% making logically defined graphics
%%%\usepackage{xypic}

% there are many more packages, add them here as you need them

% define commands here
% copied from "gradient"
\newcommand{\lp}{\left(}
\newcommand{\rp}{\right)}
\newcommand{\lb}{\left[}
\newcommand{\rb}{\right]}
\newcommand{\supth}{^{\text{th}}}

\newcommand{\vF}{\mathbf{F}}
\newcommand{\vs}{\mathbf{s}}
\newcommand{\vv}{\mathbf{v}}

\newcommand{\vi}{\mathbf{i}}
\newcommand{\vj}{\mathbf{j}}
\newcommand{\vk}{\mathbf{k}}
\newcommand{\ve}{\mathbf{e}}

\newcommand{\vx}{\mathbf{x}}
\newcommand{\vX}{\mathbf{X}}
\newcommand{\vA}{\mathbf{A}}

\newcommand{\grad}{\operatorname{grad}}
\newcommand{\curl}{\operatorname{curl}}
\begin{document}
Let $f:\mathbb{R}^n\to \mathbb{R}$ be a 
$C^1(\mathbb{R}^n)$ function, that is, a partially differentiable 
function in all its coordinates. The symbol $\nabla$, named 
\emph{nabla}, represents the gradient operator, whose action on $f(x_1,x_2,\ldots,x_n)$ is given  by
\begin{eqnarray*}
\nabla f&=&\left(f_{x_1},f_{x_2},\ldots,f_{x_n}\right)\\
&=&\left(
\frac{\partial f}{\partial x_1},\frac{\partial f}{\partial x_2},\ldots,\frac{\partial f}{\partial x_n}
\right)
\end{eqnarray*}

\section*{Properties}
\begin{enumerate}
\item If $f,g$ are functions, then 
\[
   \nabla(fg) = (\nabla f) g + f \nabla g.
\]
\item For any scalars $\alpha$ and $\beta$ and functions $f$ and $g$,
\[
   \nabla(\alpha f + \beta g) = \alpha \nabla f + \beta \nabla g.
\]
\end{enumerate}

\section*{The $\nabla$ symbolism}
Using the $\nabla$ formalism,
the divergence operator can be expressed as
$\nabla\cdot$, the curl operator as $\nabla\times$, and the
Laplacian operator as $\nabla^2$. To wit, for a given vector field
\[
\vA = A_x\, \vi + A_y\, \vj + A_z\, \vk,
\]
and a given function $f$
we have
\begin{align*}
\nabla\cdot \vA &= \frac{\partial A_x}{\partial x} +
\frac{\partial A_y}{\partial y} +\frac{\partial A_z}{\partial z} \\
\nabla\times \vA &=
\lp\frac{\partial A_z}{\partial y} - \frac{\partial A_y}{\partial z}
\rp \vi+
\lp \frac{\partial A_x}{\partial z} -
\frac{\partial A_z}{\partial x}\rp \vj+
\lp \frac{\partial A_y}{\partial x} -
\frac{\partial A_x}{\partial y}\rp \vk\\
\nabla^2 f &= \frac{\partial^2 f}{\partial x^2} +
\frac{\partial^2 f}{\partial y^2} +\frac{\partial^2 f}{\partial z^2}.
\end{align*}
%%%%%
%%%%%
\end{document}
