\documentclass[12pt]{article}
\usepackage{pmmeta}
\pmcanonicalname{FiniteLimitImplyingUniformContinuity}
\pmcreated{2013-03-22 19:00:20}
\pmmodified{2013-03-22 19:00:20}
\pmowner{pahio}{2872}
\pmmodifier{pahio}{2872}
\pmtitle{finite limit implying uniform continuity}
\pmrecord{5}{41874}
\pmprivacy{1}
\pmauthor{pahio}{2872}
\pmtype{Theorem}
\pmcomment{trigger rebuild}
\pmclassification{msc}{26A15}

\endmetadata

% this is the default PlanetMath preamble.  as your knowledge
% of TeX increases, you will probably want to edit this, but
% it should be fine as is for beginners.

% almost certainly you want these
\usepackage{amssymb}
\usepackage{amsmath}
\usepackage{amsfonts}

% used for TeXing text within eps files
%\usepackage{psfrag}
% need this for including graphics (\includegraphics)
%\usepackage{graphicx}
% for neatly defining theorems and propositions
 \usepackage{amsthm}
% making logically defined graphics
%%%\usepackage{xypic}

% there are many more packages, add them here as you need them

% define commands here

\theoremstyle{definition}
\newtheorem*{thmplain}{Theorem}

\begin{document}
\textbf{Theorem.}\, If the real function $f$ is continuous on the interval \,$[0,\,\infty)$\, and the limit 
\,$\displaystyle\lim_{x\to\infty}f(x)$\, exists as a finite number $a$, then $f$ is uniformly continuous on that interval.\\

\emph{Proof.}\, Let\, $\varepsilon > 0$.\, According to the limit condition, there is a positive number $M$ such that 
\begin{align}
|f(x)\!-\!a| \;<\; \frac{\varepsilon}{2} \quad \forall x > M.
\end{align}
The function is continuous on the finite interval\, $[0,\,M\!+\!1]$;\, hence $f$ is also uniformly continuous on this compact interval.\, Consequently, there is a positive number\, $\delta < 1$\, such that
\begin{align}
|f(x_1)\!-\!f(x_2)| \;<\; \varepsilon 
\quad \forall\, x_1,\,x_2 \in [0,\,M\!+\!1]\;\;\mbox{with}\;\;|x_1\!-\!x_2| < \delta.
\end{align}
Let $x_1,\,x_2$ be nonnegative numbers and\, $|x_1\!-\!x_2| < \delta$.\, Then\, $|x_1\!-\!x_2| < 1$\, and thus both numbers either belong to\, $[0,\,M\!+\!1]$\, or are greater than $M$.\, In the latter case, by (1) we have
\begin{align}
|f(x_1)\!-\!f(x_2)| \;=\; |f(x_1)\!-\!a\!+\!a\!-\!f(x_2)|\;\leqq\; |f(x_1)\!-\!a|+|f(x_2)\!-\!a| 
\;<\; \frac{\varepsilon}{2}+\frac{\varepsilon}{2} \;=\; \varepsilon.
\end{align}
So, one of the conditions (2) and (3) is always in \PMlinkescapetext{force}, whence the assertion is true.

%%%%%
%%%%%
\end{document}
