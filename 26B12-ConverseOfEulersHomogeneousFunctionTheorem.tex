\documentclass[12pt]{article}
\usepackage{pmmeta}
\pmcanonicalname{ConverseOfEulersHomogeneousFunctionTheorem}
\pmcreated{2013-03-22 18:07:56}
\pmmodified{2013-03-22 18:07:56}
\pmowner{pahio}{2872}
\pmmodifier{pahio}{2872}
\pmtitle{converse of Euler's homogeneous function theorem}
\pmrecord{7}{40683}
\pmprivacy{1}
\pmauthor{pahio}{2872}
\pmtype{Theorem}
\pmcomment{trigger rebuild}
\pmclassification{msc}{26B12}
\pmclassification{msc}{26A06}
\pmclassification{msc}{15-00}
\pmsynonym{converse of Euler's theorem on homogeneous functions}{ConverseOfEulersHomogeneousFunctionTheorem}
%\pmkeywords{Euler's theorem on homogeneous functions}
\pmrelated{ConverseTheorem}
\pmrelated{ChainRuleSeveralVariables}
\pmrelated{Logarithm}

\endmetadata

% this is the default PlanetMath preamble.  as your knowledge
% of TeX increases, you will probably want to edit this, but
% it should be fine as is for beginners.

% almost certainly you want these
\usepackage{amssymb}
\usepackage{amsmath}
\usepackage{amsfonts}

% used for TeXing text within eps files
%\usepackage{psfrag}
% need this for including graphics (\includegraphics)
%\usepackage{graphicx}
% for neatly defining theorems and propositions
 \usepackage{amsthm}
% making logically defined graphics
%%%\usepackage{xypic}

% there are many more packages, add them here as you need them

% define commands here

\theoremstyle{definition}
\newtheorem*{thmplain}{Theorem}

\begin{document}
\textbf{Theorem.}\, If the function $f$ of the real variables $x_1,\,\ldots,\,x_k$ satisfies the identity
\begin{align}
x_1\frac{\partial f}{\partial x_1}+\ldots+x_k\frac{\partial f}{\partial x_k} = nf,
\end{align}
then $f$ is a homogeneous function of degree $n$.\\


{\em Proof.}\, Let\, $f(tx_1,\,\ldots,\,tx_k) := \varphi(t)$.\, Differentiating with respect to $t$ we obtain
$$\varphi'(t) = x_1f'_{x_1}(tx_1,\,\ldots,\,tx_k)+\ldots+x_kf'_{x_k}(tx_1,\,\ldots,\,tx_k) 
= \frac{1}{t}[tx_1f'_{x_1}(tx_1,\,\ldots,\,tx_k)+\ldots+tx_kf'_{x_k}(tx_1,\,\ldots,\,tx_k)],$$
which by (1) may be written
$$\varphi'(t) = \frac{n}{t}f(tx_1,\,\ldots,\,tx_k) = \frac{n}{t}\varphi(t).$$
Accordingly,
$$\frac{\varphi'(t)}{\varphi(t)} = \frac{n}{t},$$
which implies the integrated form
$$\ln|\varphi(t)| = \ln{t^n}+\ln{C}$$
for any positive $t$.\, Thus we have\, $\varphi(t) = Ct^n$,\, where $C$ is independent on $t$.\, Choosing\, $t = 1$\, we see that\, $C = \varphi(1)$,\, and therefore\, $\varphi(t) = t^n\varphi(1)$.\, This last equation means that
$$f(tx_1,\,\ldots,\,tx_k) = t^nf(x_1,\,\ldots,\,x_k)$$
saying that $f$ is a (positively) homogeneous function of degree $n$.

\begin{thebibliography}{8}
\bibitem{lindelof}{\sc Ernst Lindel\"of}: {\em Differentiali- ja integralilasku
ja sen sovellutukset II}.\, Mercatorin Kirjapaino Osakeyhti\"o, Helsinki (1932).
\end{thebibliography} 

%%%%%
%%%%%
\end{document}
