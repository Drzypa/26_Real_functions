\documentclass[12pt]{article}
\usepackage{pmmeta}
\pmcanonicalname{VolumeOfSphericalCapAndSphericalSector}
\pmcreated{2013-03-22 18:19:14}
\pmmodified{2013-03-22 18:19:14}
\pmowner{pahio}{2872}
\pmmodifier{pahio}{2872}
\pmtitle{volume of spherical cap and spherical sector}
\pmrecord{7}{40950}
\pmprivacy{1}
\pmauthor{pahio}{2872}
\pmtype{Theorem}
\pmcomment{trigger rebuild}
\pmclassification{msc}{26B15}
\pmclassification{msc}{53A05}
\pmclassification{msc}{51M04}
\pmsynonym{volume of spherical cap}{VolumeOfSphericalCapAndSphericalSector}
\pmsynonym{volume of spherical sector}{VolumeOfSphericalCapAndSphericalSector}
\pmrelated{SubstitutionNotation}
\pmrelated{GreatCircle}
\pmrelated{Diameter2}
\pmrelated{PowerOfPoint}

% this is the default PlanetMath preamble.  as your knowledge
% of TeX increases, you will probably want to edit this, but
% it should be fine as is for beginners.

% almost certainly you want these
\usepackage{amssymb}
\usepackage{amsmath}
\usepackage{amsfonts}
\usepackage{amsthm}

\usepackage{mathrsfs}
\usepackage{pstricks}
\usepackage{pst-plot}

% used for TeXing text within eps files
%\usepackage{psfrag}
% need this for including graphics (\includegraphics)
%\usepackage{graphicx}
% for neatly defining theorems and propositions
%
% making logically defined graphics
%%%\usepackage{xypic}

% there are many more packages, add them here as you need them

% define commands here
% define commands here
\newcommand{\sijoitus}[2]%
{\operatornamewithlimits{\Big/}_{\!\!\!#1}^{\,#2}}

\newcommand{\sR}[0]{\mathbb{R}}
\newcommand{\sC}[0]{\mathbb{C}}
\newcommand{\sN}[0]{\mathbb{N}}
\newcommand{\sZ}[0]{\mathbb{Z}}

 \usepackage{bbm}
 \newcommand{\Z}{\mathbbmss{Z}}
 \newcommand{\C}{\mathbbmss{C}}
 \newcommand{\F}{\mathbbmss{F}}
 \newcommand{\R}{\mathbbmss{R}}
 \newcommand{\Q}{\mathbbmss{Q}}



\newcommand*{\norm}[1]{\lVert #1 \rVert}
\newcommand*{\abs}[1]{| #1 |}



\newtheorem{thm}{Theorem}
\newtheorem{defn}{Definition}
\newtheorem{prop}{Proposition}
\newtheorem{lemma}{Lemma}
\newtheorem{cor}{Corollary}
\begin{document}
\PMlinkescapeword{height}
\textbf{Theorem 1.}\, The volume of a spherical cap is\, $\pi h^2\!\left(r\!-\!\frac{h}{3}\right)$,\, when $h$ is its height and $r$ is the radius of the sphere.\\

{\em Proof.}\, The sphere may be formed by letting the circle \,$(x\!-\!r)^2\!+\!y^2 = r^2$,\, i.e.\, 
$y = (\pm)\sqrt{rx\!-\!x^2}$,\, rotate about the $x$-axis.\, Let the spherical cap be the portion \PMlinkescapetext{cut} from the sphere on the left of the plane at\, $x = h$\, perpendicular to the $x$-axis.
\begin{center}
\begin{pspicture}(-0.8,-3.5)(7,3.5)
\psdots(0,0)(1.2,0)(3,0)(5.95,0)
\rput(-0.2,-0.25){$0$}
\rput(1.2,-0.27){$h$}
\rput(3,-0.27){$r$}
\rput(0.2,3.4){$y$}
\rput(6.8,0.24){$x$}
\pscircle[linecolor=blue](3,0){3}
\psellipse[linecolor=blue](1.2,0)(0.45,2.39)
\psline{->}(0,-3.3)(0,3.3)
\psline[linestyle=dotted](0,0)(1.2,0)
\psline[linestyle=dashed](1.2,0)(6,0)
\psline(-0.7,0)(0,0)
\psline{->}(6,0)(6.7,0)
\rput(-0.8,-3.5){.}
\rput(7,3.5){.}
\end{pspicture}
\end{center}
Then the \PMlinkescapetext{formula} for the volume of solid of revolution yields the volume in question:
$$V = \pi\!\int_0^h(\sqrt{rx\!-\!x^2})^2\,dx = \pi\!\int_0^h(2rx\!-\!x^2)\,dx = 
\pi\!\sijoitus{x=0}{\quad h}\left(rx^2\!-\!\frac{x^3}{3}\right) = \pi{h}^2\!\left(r\!-\!\frac{h}{3}\right).\\$$


\textbf{Theorem 2.}\, The volume of a spherical sector is\, $\frac{2}{3}\pi{r}^2h$,\, where $h$ is the height of the spherical cap of the spherical sector and $r$ is the radius of the sphere.\\

{\em Proof.}\, The volume $V$ of the spherical sector equals to the sum or difference of the spherical cap and the circular cone depending on whether\, $h < r$\, or\, $h > r$.\, If the radius of the base circle of the cone is $\varrho$, then
$$
V = \begin{cases} 
\pi{h}^2(r\!-\!\frac{h}{3})+\frac{1}{3}\pi{\varrho}^2(r\!-\!h) &\mbox{when\, $h < r$,}\\
\pi{h}^2(r\!-\!\frac{h}{3})-\frac{1}{3}\pi{\varrho}^2(h\!-\!r) &\mbox{when\, $h > r$.}
\end{cases}
$$
But one can see that both expressions of $V$ are identical.\, Moreover, if $c$ is the great circle of the sphere having as a diameter the line of the axis of the cone and if $P$ is the midpoint of the base of the cone, then in both cases, the power of the point $P$ with respect to the circle $c$ is
$$\varrho^2 = (2r\!-\!h)h.$$
Substituting this to the expression of $V$ and simplifying give\, $V = \frac{2}{3}\pi{r}^2h$,\, Q.E.D.




%%%%%
%%%%%
\end{document}
