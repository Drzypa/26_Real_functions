\documentclass[12pt]{article}
\usepackage{pmmeta}
\pmcanonicalname{GradientInCurvilinearCoordinates}
\pmcreated{2013-03-22 15:27:32}
\pmmodified{2013-03-22 15:27:32}
\pmowner{stevecheng}{10074}
\pmmodifier{stevecheng}{10074}
\pmtitle{gradient in curvilinear coordinates}
\pmrecord{5}{37309}
\pmprivacy{1}
\pmauthor{stevecheng}{10074}
\pmtype{Result}
\pmcomment{trigger rebuild}
\pmclassification{msc}{26B12}
\pmclassification{msc}{26B10}
\pmrelated{gradient}
\pmrelated{Gradient}

\endmetadata

\usepackage{amssymb}
\usepackage{amsmath}
\usepackage{amsfonts}
%\usepackage{amsthm}
\usepackage{enumerate}

% used for TeXing text within eps files
%\usepackage{psfrag}
% need this for including graphics (\includegraphics)
%\usepackage{graphicx}
% making logically defined graphics
%%%\usepackage{xypic}

% define commands here
\newcommand{\complex}{\mathbb{C}}
\newcommand{\real}{\mathbb{R}}
\newcommand{\rat}{\mathbb{Q}}
\newcommand{\nat}{\mathbb{N}}

\providecommand{\abs}[1]{\lvert#1\rvert}
\providecommand{\absW}[1]{\left\lvert#1\right\rvert}
\providecommand{\absB}[1]{\Bigl\lvert#1\Bigr\rvert}
\providecommand{\norm}[1]{\lVert#1\rVert}
\providecommand{\normW}[1]{\left\lVert#1\right\rVert}
\providecommand{\normB}[1]{\Bigl\lVert#1\Bigr\rVert}
\providecommand{\defnterm}[1]{\emph{#1}}

\DeclareMathOperator{\D}{D}
\DeclareMathOperator{\linspan}{span}

\newcommand{\vi}{\mathbf{i}}
\newcommand{\vj}{\mathbf{j}}
\newcommand{\vk}{\mathbf{k}}
\newcommand{\ve}{\mathbf{e}}
\begin{document}
We give the formulas for the gradient expressed in various curvilinear coordinate systems.
We also show the metric tensors $g_{ij}$ so
that the reader may verify the results
by working from the basic formulas for the gradient.
\tableofcontents

\section{Cylindrical coordinate system}
In the
cylindrical system of coordinates $(r,\theta, z)$ we have
\[
g_{ij} =
\begin{pmatrix}
1 &0&0\\ 0&r^2&0\\ 0&0&1
\end{pmatrix}\,.
\]
So that
\begin{align*}
\nabla f 
&= \frac{\partial f}{\partial r}\,\ve_r +
\frac{1}{r}\frac{\partial f}{\partial\theta}\,\ve_\theta +
\frac{\partial f}{\partial z}\,\vk\,,
\end{align*}

where
\begin{align*}
\ve_r &= \frac{\partial}{\partial r} =
\frac{x}{r}\, \vi + \frac{y}{r}\, \vj \\
\ve_\theta &= \frac{1}{r}\frac{\partial}{\partial \theta} =
-\frac{y}{r}\, \vi + \frac{x}{r}\, \vj
\end{align*}
are the unit vectors in the direction of increase of $r$ and $\theta$.  Of course,
$\vi, \vj, \vk$ denote the unit vectors along the positive $x, y, z$ axes respectively.

The notations $\partial/\partial r, \partial/\partial \theta$, etc.,
denote the tangent vectors corresponding to infinitesimal changes in $r, \theta$, etc. respectively.
Concretely, in terms of Cartesian coordinates, $\partial/\partial r$
is the vector $\vi \, \partial x/\partial r + \vj \, \partial y/\partial r + \vk \, \partial z/\partial r$.
And similarly for the other variables.  (There is a deep reason for using the seemingly strange notation:
see Leibniz notation for vector fields for details.)


\section{Polar coordinate system}
This is just the special case of the cylindrical coordinate system where we chop off the $z$ coordinate.
Thus
\begin{align*}
\nabla f &= \frac{\partial f}{\partial r} \, \ve_r + \frac{1}{r} \frac{\partial f}{\partial \theta} \, \ve_\theta\,.
\end{align*}

\section{Spherical coordinate system}
To stave off confusion, note that this is the ``mathematicians'~'' convention for the spherical coordinate
system $(\rho,\phi,\theta)$.
That is, $\phi$ is the co-latitude angle, and $\theta$ is the longitudinal angle.

\[
g_{ij} =
\begin{pmatrix}
1&0&0 \\ 0&\rho^2&0\\ 0&0&\rho^2\sin^2\phi
\end{pmatrix}.
\]

\begin{align*}
\nabla f &= 
\frac{\partial f}{\partial\rho}\,\ve_\rho +
\frac{1}{\rho}\frac{\partial f}{\partial\phi}\,\ve_\phi +
\frac{1}{\rho\sin\phi}\frac{\partial f}{\partial\theta}\,\ve_\theta\,,
\end{align*}
where
\begin{align*}
\ve_\rho &= \frac{\partial}{\partial \rho} =
\frac{x}{\rho}\, \vi + \frac{y}{\rho}\, \vj + \frac{z}{\rho}\, \vk\\
\ve_\phi &= \frac{1}{\rho} \frac{\partial}{\partial \phi} = \frac{z
x}{r\rho} \, \vi +
\frac{ zy}{r\rho}\, \vj - \frac{r}{\rho}\, \vk\\
\ve_\theta &= \frac{1}{\rho\sin\theta}\frac{\partial}{\partial
\theta} = - \frac{y}{r}\, \vi +
\frac{x}{r}\, \vj
\end{align*}
are the unit vectors in the direction of increase of $\rho, \phi, \theta$, respectively.
%%%%%
%%%%%
\end{document}
