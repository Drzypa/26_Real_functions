\documentclass[12pt]{article}
\usepackage{pmmeta}
\pmcanonicalname{SumOfSeriesDependsOnOrder}
\pmcreated{2013-03-22 14:50:59}
\pmmodified{2013-03-22 14:50:59}
\pmowner{pahio}{2872}
\pmmodifier{pahio}{2872}
\pmtitle{sum of series depends on order}
\pmrecord{16}{36520}
\pmprivacy{1}
\pmauthor{pahio}{2872}
\pmtype{Example}
\pmcomment{trigger rebuild}
\pmclassification{msc}{26A06}
\pmclassification{msc}{40A05}
%\pmkeywords{conditional convergence}
\pmrelated{AbsoluteConvergence}
\pmrelated{OrderOfFactorsInInfiniteProduct}
\pmrelated{AlternatingHarmonicSeries}
\pmrelated{ConditionallyConvergentSeries}
\pmrelated{ConvergingAlternatingSeriesNotSatisfyingAllLeibnizConditions}
\pmrelated{FiniteChangesInConvergentSeries}
\pmrelated{FiniteChangesInConvergentSeries2}

\endmetadata

% this is the default PlanetMath preamble.  as your knowledge
% of TeX increases, you will probably want to edit this, but
% it should be fine as is for beginners.

% almost certainly you want these
\usepackage{amssymb}
\usepackage{amsmath}
\usepackage{amsfonts}

% used for TeXing text within eps files
%\usepackage{psfrag}
% need this for including graphics (\includegraphics)
%\usepackage{graphicx}
% for neatly defining theorems and propositions
%\usepackage{amsthm}
% making logically defined graphics
%%%\usepackage{xypic}

% there are many more packages, add them here as you need them

% define commands here
\begin{document}
According to the \PMlinkname{Leibniz' test}{LeibnizEstimateForAlternatingSeries}, 
the alternating series
$$1-\frac{1}{2}+\frac{1}{3}-\frac{1}{4}+\frac{1}{5}-\frac{1}{6}+\frac{1}{7}
-\frac{1}{8}+\frac{1}{9}-\frac{1}{10}+\frac{1}{11}-\frac{1}{12}+-\ldots$$
is convergent and has a positive sum ($= \ln{2}$; see the \PMlinkname{natural logarithm}{NaturalLogarithm2}).\, Denote it by $S$.\, We can \PMlinkescapetext{group pairwise its terms and multiply each term} by $\frac{1}{2}$ getting the two series
$S = (1-\frac{1}{2})+(\frac{1}{3}-\frac{1}{4})+(\frac{1}{5}-\frac{1}{6})+(\frac{1}{7}
-\frac{1}{8})+(\frac{1}{9}-\frac{1}{10})+\ldots,$

$\frac{1}{2}S = \frac{1}{2}-\frac{1}{4}+\frac{1}{6}-\frac{1}{8}+\frac{1}{10}-+\ldots.$

Then we add these two series termwise getting the sum

$1\frac{1}{2}S = 1+\frac{1}{3}-\frac{2}{4}+\frac{1}{5}+\frac{1}{7}
-\frac{2}{8}+\frac{1}{9}+\frac{1}{11}-\frac{2}{12}+\ldots.$

Hence, this last series \PMlinkescapetext{contains} exactly the same \PMlinkescapetext{terms} as the original, but its sum is fifty percent greater.\, This is possible because the original series is not absolutely convergent:\, the series which is formed of the absolute values of its \PMlinkescapetext{terms} is the divergent harmonic series.\\

P. S.\; -- For justification of the used manipulations of the series, see the \PMlinkescapetext{parent} entry.
%%%%%
%%%%%
\end{document}
