\documentclass[12pt]{article}
\usepackage{pmmeta}
\pmcanonicalname{ProofOfGronwallsLemma}
\pmcreated{2013-03-22 13:22:23}
\pmmodified{2013-03-22 13:22:23}
\pmowner{jarino}{552}
\pmmodifier{jarino}{552}
\pmtitle{proof of Gronwall's lemma}
\pmrecord{5}{33902}
\pmprivacy{1}
\pmauthor{jarino}{552}
\pmtype{Proof}
\pmcomment{trigger rebuild}
\pmclassification{msc}{26D10}

% this is the default PlanetMath preamble.  as your knowledge
% of TeX increases, you will probably want to edit this, but
% it should be fine as is for beginners.

% almost certainly you want these
\usepackage{amssymb}
\usepackage{amsmath}
\usepackage{amsfonts}

% used for TeXing text within eps files
%\usepackage{psfrag}
% need this for including graphics (\includegraphics)
%\usepackage{graphicx}
% for neatly defining theorems and propositions
%\usepackage{amsthm}
% making logically defined graphics
%%%\usepackage{xypic}

% there are many more packages, add them here as you need them

% define commands here
\begin{document}
The inequality
\begin{equation}
\phi(t)\leq K+L\int_{t_0}^t\psi(s)\phi(s)ds
\label{ineq}
\end{equation}
is equivalent to
\[
\frac{\phi(t)}{K+L\int_{t_0}^t\psi(s)\phi(s)ds}\leq 1
\]
Multiply by $L\psi(t)$ and integrate, giving
\[
\int_{t_0}^t \frac{L\psi(s)\phi(s)ds}{K+L\int_{t_0}^s\psi(\tau)\phi(\tau)d\tau}\leq L\int_{t_0}^t \psi(s)ds
\]
Thus
\[
\ln\left(K+L\int_{t_0}^t\psi(s)\phi(s)ds\right)-\ln K\leq L\int_{t_0}^t \psi(s)ds
\]
and finally
\[
K+L\int_{t_0}^t\psi(s)\phi(s)ds\leq K\exp\left(L\int_{t_0}^t\psi(s)ds\right)
\]
Using (\ref{ineq}) in the left hand side of this inequality gives the result.
%%%%%
%%%%%
\end{document}
