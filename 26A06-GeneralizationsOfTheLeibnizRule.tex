\documentclass[12pt]{article}
\usepackage{pmmeta}
\pmcanonicalname{GeneralizationsOfTheLeibnizRule}
\pmcreated{2013-03-22 14:30:18}
\pmmodified{2013-03-22 14:30:18}
\pmowner{GeraW}{6138}
\pmmodifier{GeraW}{6138}
\pmtitle{generalizations of the Leibniz rule}
\pmrecord{13}{36042}
\pmprivacy{1}
\pmauthor{GeraW}{6138}
\pmtype{Theorem}
\pmcomment{trigger rebuild}
\pmclassification{msc}{26A06}
\pmsynonym{Leibniz rule}{GeneralizationsOfTheLeibnizRule}
\pmrelated{MultinomialTheorem}
\pmrelated{NthDerivativeOfADeterminant}

% this is the default PlanetMath preamble.  as your knowledge
% of TeX increases, you will probably want to edit this, but
% it should be fine as is for beginners.

% almost certainly you want these
\usepackage{amssymb}
\usepackage{amsmath}
\usepackage{amsfonts}

% used for TeXing text within eps files
%\usepackage{psfrag}
% need this for including graphics (\includegraphics)
%\usepackage{graphicx}
% for neatly defining theorems and propositions
%\usepackage{amsthm}
% making logically defined graphics
%%%\usepackage{xypic}

% there are many more packages, add them here as you need them

% define commands here

\newcommand{\sR}[0]{\mathbb{R}}
\newcommand{\sC}[0]{\mathbb{C}}
\begin{document}
For the derivative, the product rule
$$
   (fg)' = f'g + fg'
$$
is known as the \emph{Leibniz rule}. Below are various ways it
can be generalized. 

\subsubsection*{Higher derivatives}
Let $f,g$ be real (or complex)
functions defined on an open interval of $\sR$. If
$f$ and $g$ are $k$ times differentiable, then
$$
   (fg)^{(k)} = \sum_{r=0}^k {k \choose r} f^{(k-r)} g^{(r)}.
$$

\subsubsection*{Generalized Leibniz rule for more functions}
Let $f_1,\ldots,f_r$ be real (or complex) valued functions 
that are defined on an open interval of $\mathbb{R}$.
If $f_1,\ldots,f_r$  are $n$ times differentiable, then
$$
   \frac{d^n}{dt^n}\prod_{i=1}^rf_i(t) = \sum_{n_1+\cdots+n_r=n} {n \choose n_1,n_2,\ldots,n_r} \prod_{i=1}^r \frac{d^{n_i}}{dt^{n_i}}f_i(t).
$$
where ${n \choose n_1,n_2,\ldots,n_r}$ is the multinomial coefficient.


\subsubsection*{Leibniz rule for multi-indices}
If $f,g:\sR^n \to \sR$ are smooth functions defined on an 
open set of $\sR^n$, and $j$ is a multi-index, then
$$ \partial^j(fg) = \sum_{i\le j} {j \choose i} \partial^i(f)\, \partial^{j-i}(g),$$
where $i$ is a multi-index.

\begin{thebibliography}{3}
\bibitem{Leibniz}
Leibniz, Gottfried W. {\it Symbolismus memorabilis calculi Algebraici et Infinitesimalis, in comparatione potentiarum et differentiarum; et de Lege Homogeneorum Transcendentali}, Miscellanea Berolinensia ad incrementum 
scientiarum, ex scriptis Societati Regiae scientarum pp. 160-165 (1710).
Available online at the \PMlinkexternal{digital library of the 
Berlin-Brandenburg Academy}{
http://bibliothek.bbaw.de/bibliothek-digital/digitalequellen/schriften/anzeige/index_html?band=01-misc/1&seite:int=184}.
\end{thebibliography}
%%%%%
%%%%%
\end{document}
