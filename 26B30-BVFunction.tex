\documentclass[12pt]{article}
\usepackage{pmmeta}
\pmcanonicalname{BVFunction}
\pmcreated{2013-03-22 15:12:32}
\pmmodified{2013-03-22 15:12:32}
\pmowner{paolini}{1187}
\pmmodifier{paolini}{1187}
\pmtitle{BV function}
\pmrecord{11}{36969}
\pmprivacy{1}
\pmauthor{paolini}{1187}
\pmtype{Definition}
\pmcomment{trigger rebuild}
\pmclassification{msc}{26B30}
\pmsynonym{function of bounded variation}{BVFunction}
\pmrelated{TotalVariation}
\pmdefines{total variation}

% this is the default PlanetMath preamble.  as your knowledge
% of TeX increases, you will probably want to edit this, but
% it should be fine as is for beginners.

% almost certainly you want these
\usepackage{amssymb}
\usepackage{amsmath}
\usepackage{amsfonts}

% used for TeXing text within eps files
%\usepackage{psfrag}
% need this for including graphics (\includegraphics)
%\usepackage{graphicx}
% for neatly defining theorems and propositions
\usepackage{amsthm}
% making logically defined graphics
%%%\usepackage{xypic}

% there are many more packages, add them here as you need them

% define commands here
\newcommand{\R}{\mathbb R}
\newtheorem{theorem}{Theorem}
\newtheorem{definition}{Definition}
\theoremstyle{remark}
\newtheorem{example}{Example}
\begin{document}
Functions of bounded variation, $BV$ functions, are functions whose distributional derivative is a finite Radon measure. More precisely:

\begin{definition}[functions of bounded variation]
Let $\Omega\subset \R^n$ be an open set. We say that a function $u\in L^1(\Omega)$ 
has \emph{bounded variation}, and write $u\in BV(\Omega)$, if there exists a finite Radon vector measure $Du\in\mathcal M(\Omega,\R^n)$ such that
\[
  \int_\Omega u(x)\,\mathrm{div}\phi(x)\, dx = - \int_\Omega \langle \phi(x), Du(x)\rangle
\] 
for every function $\phi\in C_c^1(\Omega,\R^n)$. The measure $Du$, 
represents the distributional derivative of $u$ since the above equality holds true for every $\phi\in C^\infty_c(\Omega,\R^n)$.
\end{definition}

Notice that $W^{1,1}(\Omega)\subset BV(\Omega)$. In fact if $u\in W^{1,1}(\Omega)$ one can choose $\mu:=\nabla u\mathcal L$ (where $\mathcal L$ is the Lebesgue measure on $\Omega$). 
The equality $\int u\mathrm{div \phi} = -\int \phi\, d\mu
= -\int \phi \nabla u$ 
is nothing else than the definition of weak derivative, and hence holds true.
One can easily find an example of a $BV$ functions which is not $W^{1,1}$.

An equivalent definition can be given as follows. 
\begin{definition}[variation]
Given $u\in L^1(\Omega)$ we define the \emph{variation} of $u$ in $\Omega$ as
\[
  V(u,\Omega):=\sup\{\int_\Omega u\mathrm{div}\phi\colon \phi\in\mathcal C_c^1(\Omega,\R^n),\ \Vert \phi\Vert_{L^\infty(\Omega)}\le 1\}.
\]
We define $BV(\Omega)=\{ u\in L^1(\Omega)\colon V(u,\Omega)<+\infty\}$.
\end{definition}.
%%%%%
%%%%%
\end{document}
