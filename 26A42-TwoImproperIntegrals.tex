\documentclass[12pt]{article}
\usepackage{pmmeta}
\pmcanonicalname{TwoImproperIntegrals}
\pmcreated{2013-03-22 18:43:11}
\pmmodified{2013-03-22 18:43:11}
\pmowner{pahio}{2872}
\pmmodifier{pahio}{2872}
\pmtitle{two improper integrals}
\pmrecord{6}{41487}
\pmprivacy{1}
\pmauthor{pahio}{2872}
\pmtype{Example}
\pmcomment{trigger rebuild}
\pmclassification{msc}{26A42}
\pmclassification{msc}{26A06}
\pmclassification{msc}{26A03}

\endmetadata

% this is the default PlanetMath preamble.  as your knowledge
% of TeX increases, you will probably want to edit this, but
% it should be fine as is for beginners.

% almost certainly you want these
\usepackage{amssymb}
\usepackage{amsmath}
\usepackage{amsfonts}

% used for TeXing text within eps files
%\usepackage{psfrag}
% need this for including graphics (\includegraphics)
%\usepackage{graphicx}
% for neatly defining theorems and propositions
 \usepackage{amsthm}
% making logically defined graphics
%%%\usepackage{xypic}

% there are many more packages, add them here as you need them

% define commands here

\theoremstyle{definition}
\newtheorem*{thmplain}{Theorem}

\begin{document}
Let us consider first the improper integral
$$I(k) \;:=\; \int_0^\infty\frac{1-\cos{kx}}{x^2}\,dx.$$
The derivative $I'(k)$ may be formed by \PMlinkname{differentiating under the integral sign}{DifferentiationUnderIntegralSign}:
$$I'(k) \;=\; \int_0^\infty\left(\frac{\partial}{\partial k}\frac{1-\cos{kx}}{x^2}\right) dx 
\;=\; \int_0^\infty\frac{\sin{kx}}{x}\,dx \;=\; \int_0^\infty\frac{\sin{t}}{t}\,dt$$
Here, the last form has been gotten by the \PMlinkname{substitution}{ChangeOfVariableInDefiniteIntegral} \,$kx = t$.\, 
But since by the \PMlinkname{parent entry}{SineIntegralInInfinity} we have
$$\int_0^\infty\frac{\sin{t}}{t}\,dt \;=\; \frac{\pi}{2}$$
and since\, $I(0) = 0$, we can write
$$I(k) \;=\; \int_0^k\frac{\pi}{2}\, dk \;=\; \frac{\pi k}{2}.$$
Thus we have evaluated the integral $I(k)$:
\begin{align}
\int_0^\infty\frac{1-\cos{kx}}{x^2}\,dx \;=\; \frac{\pi k}{2}.
\end{align}


The formula (1) gives
$$I(1) \;=\; \int_0^\infty\frac{1-\cos{x}}{x^2}\,dx \;=\; \frac{\pi}{2}.$$
We use here the consequence formula
$$1-\cos{x} \;=\; 2\sin^2{\frac{x}{2}}$$
of the double angle formula \,$\cos{2\alpha} = 1-2\sin^2{\alpha}$,\, obtaining
$$\frac{\pi}{2} \;=\; 2\int_0^\infty\frac{\sin^2\frac{x}{2}}{x^2}\,dx
\;=\; \int_0^\infty\frac{\sin^2{u}}{u^2}\, du,$$
where the substitution \,$\frac{x}{2} = u$\, has produced the last form.\, Accordingly, we can write as result the formula
\begin{align}
\int_0^\infty\!\left(\!\frac{\sin{x}}{x}\!\right)^2 dx \;=\; \frac{\pi}{2}.
\end{align}





%%%%%
%%%%%
\end{document}
