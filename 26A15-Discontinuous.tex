\documentclass[12pt]{article}
\usepackage{pmmeta}
\pmcanonicalname{Discontinuous}
\pmcreated{2013-03-22 13:45:01}
\pmmodified{2013-03-22 13:45:01}
\pmowner{mathwizard}{128}
\pmmodifier{mathwizard}{128}
\pmtitle{discontinuous}
\pmrecord{14}{34447}
\pmprivacy{1}
\pmauthor{mathwizard}{128}
\pmtype{Definition}
\pmcomment{trigger rebuild}
\pmclassification{msc}{26A15}
\pmclassification{msc}{54C05}
\pmdefines{removable discontinuity}
\pmdefines{saltus}
\pmdefines{jump}
\pmdefines{jump discontinuity}
\pmdefines{discontinuity of the second kind}
\pmdefines{discontinuity of the first kind}
\pmdefines{essential discontinuity}

% this is the default PlanetMath preamble.  as your knowledge
% of TeX increases, you will probably want to edit this, but
% it should be fine as is for beginners.

% almost certainly you want these
\usepackage{amssymb}
\usepackage{amsmath}
\usepackage{amsfonts}

% used for TeXing text within eps files
%\usepackage{psfrag}
% need this for including graphics (\includegraphics)
%\usepackage{graphicx}
% for neatly defining theorems and propositions
%\usepackage{amsthm}
% making logically defined graphics
%%%\usepackage{xypic}

% there are many more packages, add them here as you need them

% define commands here

\newcommand{\sR}[0]{\mathbb{R}}
\newcommand{\sC}[0]{\mathbb{C}}
\newcommand{\sN}[0]{\mathbb{N}}
\newcommand{\sZ}[0]{\mathbb{Z}}

% The below lines should work as the command
% \renewcommand{\bibname}{References}
% without creating havoc when rendering an entry in 
% the page-image mode.
\makeatletter
\@ifundefined{bibname}{}{\renewcommand{\bibname}{References}}
\makeatother

\newcommand*{\norm}[1]{\lVert #1 \rVert}
\newcommand*{\abs}[1]{| #1 |}
\begin{document}
\PMlinkescapeword{formula}
\PMlinkescapeword{types}
\PMlinkescapeword{satisfies}
\PMlinkescapeword{simple}
\section*{Definition}
 Suppose $A$ is an open set in $\sR$ (say an interval $A=(a,b)$, or $A=\sR$), 
 and $f:A\to \sR$ is a function.
 Then $f$ is \emph{discontinuous} at $x\in A$, if $f$ is not continuous
 at $x$. One also says that $f$ is discontinuous at all boundary points of $A$.

We know that $f$ is continuous at $x$ if and only 
if $\lim_{z\to x} f(z)=f(x)$. Thus, from the properties of the 
one-sided limits, which we denote by $f(x+)$ and $f(x-)$, it follows
 that $f$ is discontinuous at $x$ if and only if 
$f(x+)\neq f(x)$, or $f(x-)\neq f(x)$.

If $f$ is discontinuous at $x\in\overline{A}$, the closure of $A$, we can then distinguish four types of
different discontinuities as follows  \cite{hoskins, contline}:
\begin{enumerate}
\item If $f(x+)=f(x-)$, but $f(x)\neq f(x\pm)$,
then $x$ is called a \emph{removable discontinuity} of $f$. 
If we modify the value of $f$ at $x$ to $f(x)=f(x\pm)$, 
then $f$ will become continuous at $x$. 
This is clear since the modified $f$ (call it $f_0$) satisfies 
$f_0(x) = f_0(x+)=f_0(x-).$
\item If $f(x+)=f(x-)$, but $x$ is not in $A$ (so $f(x)$ is 
not defined), then $x$ is also called a \emph{removable discontinuity}. 
If we assign $f(x)=f(x\pm)$, then this modification renders $f$
continuous at $x$. 
\item If $f(x-)\neq f(x+)$, then $f$ has a \emph{jump discontinuity} at $x$
Then the number $f(x+)-f(x-)$ is then called the \emph{jump},
or \emph{saltus}, of $f$ at $x$.  
\item If either (or both) of $f(x+)$ or $f(x-)$ does not exist, then 
$f$ has an \emph{essential discontinuity} at $x$
(or a \emph{discontinuity of the second kind}).
\end{enumerate}
Note that $f$ may be continuous (continuous in all points in $A$), but still have discontinuities in $\overline{A}$
 
 \section*{ Examples}
\begin{enumerate}
\item Consider the function $f:\sR\to \sR$ given by
\[
f(x)=\begin{cases} 
 1 & \text{when }x\neq 0, \\
 0 & \text{when }x=0.  
\end{cases}
\]
Since $f(0-)=1$, $f(0)=0$, and $f(0+)=1$,
it follows that $f$ has a removable discontinuity at $x=0$. 
If we modify $f(0)$ so that $f(0)=1$, then $f$ becomes the 
continuous function $f(x)=1$. 
\item Let us consider the function defined by the formula 
\[
f(x) = \frac{\sin x }{x}
\]
where $x$ is a nonzero real number. When $x=0$, the formula is undefined, so 
$f$ is only determined for $x\neq 0$. Let us show that this point is
a removable discontinuity. Indeed, it is easy 
to see that $f$ is continuous for all $x\neq 0$, and using 
\PMlinkname{L'H\^opital's rule}{LHpitalsRule} we have $f(0+)=f(0-)=1$. 
Thus, if we assign $f(0)=1$, then $f$ becomes a continuous function
defined for all real $x$.  In fact, $f$ can be made into an analytic 
function on the whole complex plane. 
\newcommand{\signum}[0]{\mathop{\mathrm{sign}}}
\item  The signum function $\signum\colon\sR\to \sR$ is defined as
\[
 \signum (x) =\begin{cases} 
 -1 & \text{when }x<0, \\
 0 & \text{when } x=0, \text{ and}\\
 1 & \text{when } x>0. 
\end{cases}
\]
Since $\signum(0+)=1$, $\signum(0)=0$, and since $\signum(0-)=-1$,
it follows that $\signum$ has a jump discontinuity at $x=0$
with jump $\signum(0+)-\signum(0-)=2$.
\item The function $f:\sR\to\sR$ given by
\[
f(x) =\begin{cases}
 1 & \text{when }x= 0, \\
 \sin(1/x) & \text{when } x\neq 0
\end{cases}
\]
has an essential discontinuity at $x=0$.  See \cite{contline} for details.
\end{enumerate} 
 
\section*{General Definition} Let $X,Y$ be topological spaces, and let $f$ be a mapping
 $f:X\to Y$. Then $f$ is \emph{discontinuous} at $x\in X$, if $f$ is not
 \PMlinkname{continuous at}{Continuous} $x$.

In this generality, one generally does not classify discontinuities quite so closely, since they can have quite complicated behaviour.

\section*{Notes}
A jump discontinuity is also called a \emph{simple discontinuity}, or  a 
\emph{ discontinuity of the first kind}.  
An  \emph{essential discontinuity} is also called a 
\emph{discontinuity of the second kind}.


\begin{thebibliography}{9}
\bibitem{hoskins}
R.F. Hoskins, \emph{Generalised functions}, 
Ellis Horwood Series: Mathematics and its applications, 
John Wiley \& Sons, 1979.
\bibitem{contline} P. B. Laval, 
\PMlinkexternal{http://science.kennesaw.edu/~plaval/spring2003/m4400_02/Math4400/contwork.pdf}{http://science.kennesaw.edu/~plaval/spring2003/m4400_02/Math4400/contwork.pdf}.
 \end{thebibliography}
%%%%%
%%%%%
\end{document}
