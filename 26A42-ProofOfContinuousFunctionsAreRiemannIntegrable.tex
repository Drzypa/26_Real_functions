\documentclass[12pt]{article}
\usepackage{pmmeta}
\pmcanonicalname{ProofOfContinuousFunctionsAreRiemannIntegrable}
\pmcreated{2013-03-22 13:45:34}
\pmmodified{2013-03-22 13:45:34}
\pmowner{paolini}{1187}
\pmmodifier{paolini}{1187}
\pmtitle{proof of continuous functions are Riemann integrable}
\pmrecord{7}{34462}
\pmprivacy{1}
\pmauthor{paolini}{1187}
\pmtype{Proof}
\pmcomment{trigger rebuild}
\pmclassification{msc}{26A42}

\endmetadata

% this is the default PlanetMath preamble.  as your knowledge
% of TeX increases, you will probably want to edit this, but
% it should be fine as is for beginners.

% almost certainly you want these
\usepackage{amssymb}
\usepackage{amsmath}
\usepackage{amsfonts}

% used for TeXing text within eps files
%\usepackage{psfrag}
% need this for including graphics (\includegraphics)
%\usepackage{graphicx}
% for neatly defining theorems and propositions
%\usepackage{amsthm}
% making logically defined graphics
%%%\usepackage{xypic}

% there are many more packages, add them here as you need them

% define commands here
\begin{document}
Recall the definition of Riemann integral. To prove that $f$ is integrable we have to prove that $\lim_{\delta \to 0^+} S^*(\delta)-S_*(\delta) =0$.
Since $S^*(\delta)$ is decreasing and $S_*(\delta)$ is increasing it is enough to show that given $\epsilon>0$ there exists $\delta>0$ such that $S^*(\delta)-S_*(\delta)<\epsilon$.

So let $\epsilon>0$ be fixed.

By Heine-Cantor Theorem $f$ is uniformly continuous i.e.
\[
  \exists \delta>0\ 
   \vert x-y\vert < \delta \Rightarrow \vert f(x)-f(y)\vert <\frac{\epsilon}{b-a}.
\]

Let now $P$ be any partition of $[a,b]$ in $C(\delta)$ i.e. a partition $\{x_0=a, x_1,\ldots,x_N=b\}$ such that $x_{i+1}-x_i<\delta$. In any small interval $[x_i,x_{i+1}]$ the function $f$ (being continuous) has a maximum $M_i$ and minimum $m_i$. Since $f$ is uniformly continuous and $x_{i+1}-x_i<\delta$ we  have $M_i - m_i < \epsilon/(b-a)$. So the difference between upper and lower Riemann sums is 
\[
  \sum_i M_i(x_{i+1}-x_i) - \sum_i m_i(x_{i+1}-x_i)
  \le \frac{\epsilon}{b-a} \sum_i (x_{i+1}-x_i) = \epsilon.
\] 

This being true for every partition $P$ in $C(\delta)$ we conclude that $S^*(\delta)-S_*(\delta)<\epsilon$.
%%%%%
%%%%%
\end{document}
