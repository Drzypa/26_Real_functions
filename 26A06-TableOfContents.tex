\documentclass[12pt]{article}
\usepackage{pmmeta}
\pmcanonicalname{TableOfContents}
\pmcreated{2014-08-03 22:21:27}
\pmmodified{2014-08-03 22:21:27}
\pmowner{PMBookProject}{1000683}
\pmmodifier{rspuzio}{6075}
\pmtitle{Table of Contents}
\pmrecord{5}{87449}
\pmprivacy{1}
\pmauthor{PMBookProject}{6075}
\pmtype{Topic}
\pmclassification{msc}{26A06}

\endmetadata

% this is the default PlanetMath preamble.  as your knowledge
% of TeX increases, you will probably want to edit this, but
% it should be fine as is for beginners.

% almost certainly you want these
\usepackage{amssymb}
\usepackage{amsmath}
\usepackage{amsfonts}

% need this for including graphics (\includegraphics)
\usepackage{graphicx}
% for neatly defining theorems and propositions
\usepackage{amsthm}

% making logically defined graphics
%\usepackage{xypic}
% used for TeXing text within eps files
%\usepackage{psfrag}

% there are many more packages, add them here as you need them

% define commands here

\begin{document}
THE CALCULUS

by

ELLERY WILLIAMS DAVIS

Professor of Mathematics, The University of Nebraska

Assisted by

WILLIAM CHARLES BRENKE

Associate Professor of Mathematics, The University of Nebraska

Edited by

EARLE RAYMOND HEDRICK
\end{center}

CONTENTS

Preface v-viii

[Page numbers In Roman type refer to the body of the book; those in italic type refer to
pages of the {\it Tables}.]

CHAPTER I FUNCTIONS  1-5

\S 1. Dependence . 1

\S 2. Variables. Constants. Functions 1-2

Exercises I. Functions and Graphs 2-8

\S 3. The Function Notation. 3

Exercises II. Substitution. Funotion Notation S4

CHAPTER II RATES LIMITS DERIVATIVES 6-27

\S 4. Rate of Increase. Slope 6-8

\S 5. General Rules . . . 8

\S 6. Slope Negative or Zero. [Maxima and Minima.] 8-11

Exercises III. Slopes of Curves 11-12

\S 7. Speed 12-14

\S 8. Component Speeds 14

\S 9. Continuous Functions 14-16

Exercises IV. Speed 16-16

\S 10. Limits. Infinitesimals. 16-17

\S 11. Properties of Limits . 17-18

\S 12. Ratio of an Arc to its Chord. . 18-19

\S 18. Ratio of the Sine of an Angle to the Angle 19

\S 14. Infinity. . . . . 19-20

Exercises V. Limits and Infinitesimals 20-22

\S 16. Derivatives . . 22-23

\S 16. Formula for Derivatives 28-24

\S 17. Rule for Differentiation . 24-26

Exercises VI. Formal Differentiation 26-27

CHAPTER III DIFFERENTIATION OF ALGEBRAIC FUNCTIONS  28-57

\begin{center}
PART I EXPLICIT FUNCTIONS 28-43
\end{center}

\S 18. Classification of Functions . 28-29

\S 19. Differentiation of Polynomials . 29-31

% \begin{center}
%x
% \end{center}
%
% $\mathrm{x}$

Exercises VII. Differentiation of Polynomials. 31-32

\S 20. Differentiation of Rational Functions. [Quotient.] 32-23

Exercises VIII. Differentiation of Rational Functions. 34-35

\S 21. Derivative of a Product. 35-36

\S 22. Derivative of a Function of a Function .  36-37

Exercises IX. Short Methods. Rational Functions 37-38

\S 23. Differentiation of Irrational Functions. 38-40

\S 24. Collection of Formulas. 40-41

\S 26. Illustrative Examples of Irrational Functions 41-42

Exercises X. Algebraic Functions 42-43

PART II EQUATIONS NOT IN EXPLICIT FORM. DIFFERENTIALS 44-57

\S 26. Solution of Equations 44-45

\S 27. Explicit and Implicit Functions 45-46

\S 28. Inverse Functions.46-47

\S 29. Parameter Forms . 47

Exercises XL Functions not in Explicit Form 47-49

\S 30. Rates . . . . 49-50

\S 31. The Differential Notation 50-52

\S 32. Differential Formulas 52-53

Exercises XII. Differentials. 54-57

CHAPTER IV FIRST APPLICATIONS OF DIFFERENTIATION 58-90

\begin{center}
PART I APPLICATION TO CURVES---EXTREMES 58-70
\end{center}

\S 33. Tangents and Normals . . . . 58-59

\S 34. Tangents and Normals for Curves not in Explicit Form 59

\S 35. Secondary Quantities 60

\S 36. Illustrative Examples . . 60-61

Exercises XIII. Tangents and Normal 62-63

\S 37. Extremes. [Maxima and Minima.] 63

\S 38. Critical Values 63-64

\S 39. Fundamental Theorem. 64

\S 40. Final Tests . 64-65

\S 41. Illustrative Examples in Extremes 65-67

Exercises XIV. Extremes 67-70

PART II RATES 

\S 42. Time Rates 70

\S 43. Speed 71
%
%CONTENTS
%
%$\mathrm{Xj}$
%$$
%\mathrm{PA}l*l
%$$

\S 44. Tangential Acceleration 71

\S 46. Second  Derivative. Flexion .  71-72

Exercises XV. Second Derivatives. Acceleration 73-75

\S 46. Concavity. Points of Inflexion 75

\S 47. Second Test for Extremes. 75-76

\S 48. Illustrative Examples 76-77

\S 49. Derived Curves.  77-79

Exercises XVI. Flexion. Derived Curves. 79-81

\S 50. Angular Speed . 81

\S 51. Angular Acceleration 81-82

\S 52. Momentum. Force . 82-83

Exercises XVII. Time Rates 83-85

\S 53. Related Rates . $85-87$

Exercises XVIII. Related Rates 88-90

CHAPTER V REVERSAL OF RATES INTEGRATION SUMMATION. 91-129

\begin{center}
PART I INTEGRALS BY REVERSAL OF RATES 91-109
\end{center}

\S 54. Reversal of Rates . 91-92

\S 56. Principle Involved in \S 54. 92

\S 55. Illustrative Examples . 92-94

Exercises XIX. Reversal of Rates 94-95

\S 67. Integral Notation . . . 96-97

Exercises XX. Notation. Indefinite Integrals. 97-98

\S 58. Fundamental Theorem 99

\S 59. Definite Integrals. 100-101

Exercises XXI. Definite Integrals 102-103

\S 60. Area under a Curve 108-104

Exercises XXII. Area 105

\S 61. Lengths of Curves .  106-107

\S 62. Motion on a Curve. Parameter Forms 107

\S 63. Illustrative Examples . 108-109

Exercises XXIII. Length. Total Speed 109


\begin{center}
PART II INTEGRALS AS LIMITS OF SUMS 110-129
\end{center}

\S 64. Step.by-Step Process. 110-111

\S 65. Approximate Summation. 111-112

Exercises XXIV. Step-by-Step Summation. Approximate Results 112-113

% page xii

\S 66. Exact Results. Summation Formula 114-116

\S 67. Integrals as Limits of Sums . 116-117

\S 68. Water Pressure. 117-119

Exercises XXV. Integrals as Limits of Sums 119-120

\S 69. Volumes .  120-121

\S 70. Volume of Any Frustum 121-123

Exercises XXVI. Volumes of Solids. Frusta . 124-125

\S 71. Cavalieri's Theorem The Prismoid Formula 125-127

Exercises XXVII. General Exercises 128-129

CHAPTER VI TRANSCENDENTAL FUNCTIONS 130-173

\begin{center}
PART I LOGARITHMS EXPONENTIAL FUNCTIONS 130-149
\end{center}

\S 72. Necessity of Operations on Transcendental Functions 130

\S 73. Properties of Logarithms 130-131

\S 74. Graphical Representation. 131-132

Exercises XXVIII. Logarithms and Exponentials 132-133

\S 75. Slope of $y=\log_{10}x$ at $x=0$. [Modulus $M$] 133-134

\S 76. Differentiation of $\log_{10}x$ . 134

\S ?7. Differentiation of $\log_{B}x$. [Napierlan Baae] 135-136

\S 78. Illustrative Examples. 136-137

Exercises XXIX. Logarithms . 137-138

\S 79. Differentiation of Exponentials. 138-139

\S 80. Illustrative Examples 139-140

Exercises XXX. Exponentials 140-141

\S 81. Compound Interest Law 141-143
\begin{center}
Example 1. Work in Expanding Gas (142)

Example 2. Cooling in a Moving Fluid $(142-143)$

Example 3. Bacterial Growth. (143)

Example 4. Atmospheric Pressure (143)
\end{center}

\S 82. Percentage Rate of Increase. [Relative Rates] 144

Exercises XXXI. Compound Interest Law. 144-146

\S 88. Logarithmic Differentiation. Relative Increase. 146-147

\S 84. Logarithmic Methods .  147-148

Exercises XXXII. Logarithmic Differentiation 149

\begin{center}
PART II TRIGONOMETRIC FUNCTIONS. 150-173
\end{center}

\S 86. Introduction of Trigonometric Functions 150

\S 86. Differentiation of Sines and Cosines. 150 -152

% PAGE8

\S 87. Illustrative Examples .  152-153

Exercises XXXIII. Trigonometric Functions. 153-155

\S 88. Simple Harmonic Motion. 155-156

\S 89. Relative Acceleration 156

\S 90. Vibration. 157

\S 91. Waves. 157-158

Exercises XXXIV. Simple Harmonic Motion. Vibrations 158-160

\S 92. Damped Vibrations. 160-162

Exercises XXXV. Damped Vibrations 162-163

\S 93. Inverse Trigonometric Functions 163-164

\S 94. Integrals of Irrational Functions 164

\S 95.Illustrative Examples . . . . 164-165

Exercises XXXVI. Inverse Trigonometric Functions 165-166

\S 96. Polar Coordinates . 166-168

Exercises XXXVII. Polar Coordinates 168-169

\S 97. Curvature. . 169-171

Exercises XXXVIII. Curvature 171-172

\S 98. Collection of Formulas 173

CHAPTER VII TECHNIQUE. TABLES. SUCCESSIVE INTEGRATION 174-226

\begin{center}
PART I TECHNIQUE or INTEGRATION. 174-200
\end{center}

\S 99. Question of Technique. Collection of Formulas 174-175

\S 100. Polynomials. Other Simple Forms . 176

\S 101. Substitution. [Algebraic and Trigonometric]. 176-177

\S 102. Substitution in Definite Integrals . 177

Exercises XXXIX. Elementary Integration. Substitution 178-181

\S 108. Integration by Parts. [Algebraic and Transcendental] .  181

Exercises XL. Integration by Parts. .  182-183

\S 104. Rational Functions. [Partial Fractions] 184-186

Exercises XLI. Rational Functions. 186-188

\S 106. Rationalization of Linear Radicals . 188-189

\S 107. Quadratic Irrationals . . . 189-190

Exercises XLII. Integrals involving Radicals. [Trigonometric Substitutions]. 191-194

\S 107. Elliptic and Other Integrals . 195

\S 108. Binomial Differentials . 195-196

% PAGE xiv

\S 109. General Remarks. [Tables.] . 196

Exercises XLIII. General Integration. [All Methods] 197-200

\begin{center}
PART II. IMPROPER AND MULTIPLE INTEGRALS 201-226
\end{center}

\S 110. Limits Infinite. Horizontal Asymptote. 201

\S 111. Integrand Infinite. Vertical Asymptote. 202-203

\S 112. Precautions . 203

Exercises XLIV. 204-206

\S 113. Repeated Integration. 206

\S 114. Successive Integration in Two Letters 206-207

Exercises XLV. Successive Integration .208-209

\S 115. Double Integrals .210-211

\S 116. Illustrative Examples. 211-214

[A] Volumes by Double Integration (212)

[B] Area in Polar Coordinates . . $\langle$212-213)

[C] Moment of Inertia of a Thin Plate. (213-214)

[D] Moment of Inertia in Polar Co\&quot;{o}rdlnates (214)

Exercises XLVI. Double Integrals. 214-217

\S 117. Triple and Multiple Integrals. 217-218
.
Exercises XLVII. Multiple Integrals . 218-219

\S 118. Other Applications. Averages. Centers of Gravity 219-220

Exercises XLVIII. General Problems in Integration 221-226

CHAPTER VIII. METHODS OF APPROXIMATION  227-280

\begin{center}
PART I. EMPIRICAL CURVES. INCREMENTS 227-250
\end{center}

\S 119. Empirical Curves. 227

\S 120. Polynomial Approximations. 227

\S 121. Review of Elementary Methods. 227-229

\S 122. Logarithmic Plotting 229-230

\S 123. Semi-Logarithmic Plotting .  230

Exercises XLIX. Empirical Curves. Elementary Methods. 230-233

\S 124. Method ot Increments .  233-236

Exercises L. Empirical Curves by Increments. 236-238

\S 125. Approximate Integration . 239-240

\S 126. Integration from Empirical Formulas 240-241

\S 127. Derived and Integral Curves. 241-242

Exercises LI. Approximate Evaluation of Integrals 242-243

\S 128. Integrating Devices. [Planimeter. Integraph]. 243-246

% PAGE xv

\S 129. Tabulated Integral Values .  246-247

Exercises LII. Integrating Devices. Numerical Tables 248-250

\begin{center}
PART II POLYNOMIAL APPROXIMATIONS. SERIES.
TAYLOR'S THEOREM 260-280
\end{center}

\S 180. Rolle's Theorem . . 250

\S 131. The Law of the Mean. [Finite Differences] 251

\S 132. Increments. [ Small Errors]. 252-253

Exercises LIII. Increments. Law of the Mean. 253-254

\S 133. Limit of Error.  254-257

\S 134. Extended Law of the Mean. Taylor's Theorem 257-259

Exercises LIV. Extended Law of the Mean 259-263

\S 135. Application of Taylor's Theorem to Extremes 260-261

Exercises LV. Extremes. 261-263

\S 136. Indeterminate Forms. [Form $0\div 0$]. 263-265

\S 137. Infinitesimals of Higher Order. 265-266

Exercises LVI. Indeterminate Forms. Infinitesimals 266-267

\S 138. Double Law of the Mean . 267-268

\S 139. The Indeterminate Form $\infty\div\infty$. Vertical Asymptotes. 268-269

\S 140. Other Indeterminate Forms. 269-270

Exercises LVII. Secondary Indeterminate Forms 270-271

\S 141. Infinite Series. 271-273

\S 142. Taylor Series. General Convergence Test. 273-275

Exeicises LVIII. Taylor Series 275-276

\S 143. Precautions about Infinite Series 276-279

Exercises LIX. Infinite Series 279-280

CHAPTER IX SEVERAL VARIABLES. PARTIAL DERIVATIVES.
 APPLICATIONS. GEOMETRY. 281-344

\begin{center}
PART I PARTIAL DIFFERENTIATION ELEMENTARY APPLICATIONS 281-297
\end{center}

\S 144. Partial Derivatives. 281-282

\S 146. Technique. 282

\S 146. Higher Partial Derivatives. 282-283

Exeicises LX. Technique of Partial Differentiation. 283-284

\S 147. Geometric Interpretation. 284-285

\S 148. Total Derivative. 285-287

% PAGE xvi

\S 149. Elementary Use . . . 287-288

\S 150. Small Errors. Partial Differentials . 288-290

Exercises LXI. Total Derivatives and Differentials . 290-292

\S 151. Significance of Partial and Total Derivatives. 292-296
\begin{center}
Example 1. Isothermal Expansion (292-293)

Example 2. Adiabatic Expansion (293-294)

Example 3. Implicit Equations. Contour Lines (294)

Example 4. Flow of Heat in a Metal Plate (294-295)

Example 5. Flow of Water in Pipes (295-296)
\end{center}

Exercises LXII. Applications of Total Derivatives. 296-297

\begin{center}
PART II APPLICATIONS To PLANE GEOMETRY 298-315
\end{center}
\S 152. Envelopes . 298-300

\S 168. Envelope of Normals. Evolute. 300-303

Exercises LXIII. Envelopes. Evolutes. 303

\S 154. Properties of Evolutes 304-305

\S 155. Center of Curvature 305

\S 166. Rate of Change of $R$. 305-306

\S 167. Illustrative Examples. 310-311

Exercises LXIV. Properties of Evolutes. 308-309

\S 158. Singular Points. 309-310

\S 159. Illustrative Examples. 310-311

\S 160. Asymptotes. 311-313

\S 161. Curve Tracing. 313

Exercises LXV. Singular Points. Asymptotes. Curve Tracing. 313-314

PART III GEOMETRY OF SPACE. EXTREMES. 315-344

\S 162. Resum\'{e} of Formulas. 315-318

$(a)$ Distance between Two Points . (315)

$(b)$  Distance from Origin to $(x,\ y,\ z)$ . (315)

$(c)$ Direction Cosines.(315)

$(d)$ Angle between Two Directions (315)

$(e)$ The Plane . (316-317)

$(f)$ The Straight Line . (317)

$(g)$ Quadric Surfaces. (318)

\S 168. Loci of One or More Equations. 318-319

Exerclses LXVI. Resum\'{e} of Solid Geometry. 319-120

% PAGE xvii

\S 164. Tangent Plane to a Surface . . . 321-322

\S 165. Extremes on a Surface. [Least Squares] 322-325

\S 166. Final Tests. 325-326

Exercises LXVII. Tangent Planes. Extremes 327-329

\S 167. Tangent Planes. Implicit Forms 329-330

\S 168. Line Normal to a Surface 330

\S 169. Parametric Forms of Equations. 330-332

\S 170. Tangent Planes and Normals. Parameter Forms. 332-333

Exercises LXVIII. Equations not in Explicit Form. 333-334

\S 171. Area of a Curved Surface 334-335

Exercises LXIX. Area of a Surface . 336

\S 172. Tangent to a Space Curve. 337

\S 173. Length of a Space Curve. 338

Exercises LXX. Tangents to Curves. Lengths. 338

Exercises LXXI. General Review. Several Variables 339-344

CHAPTER X DIFFERENTIAL EQUATIONS 345-333
\begin{center}
PART I ORDINARY DIFFERENTIAL EQUATIONS OF
THE FIRST ORDER 345-362
\end{center}

\S 174. Reversal of Rates . 345

\S 175. Other Reversed Problems . 345-346

\S 176. Determination of the Arbitrary Constants. 346-347

\S 177. Vital Character of Inverse Problems. 347-348

\S 178. Elementary Definitions. Ordinary Differential Equations. 343

\S 179. Elimination of Constants. 348-350

\S 180. Integral Curves. 350-351

Exercises LXXII. EliminatIon. Integral Curves. 351-352

\S 181. General Statement. 352-353

\S 182. Type I. Separation of Variables 353

\S 188. Type II. Homogeneous Equations. 354-356

Exercises LXXIII. Separation of Variables 355-356

\S 184. Type III. Linear Equations 356-358

\S 185. Extended Linear Equations. 358

Bxercises LXXIV. Linear Equations. 359

\S 186. Other Methods. Non-linear Equations. 359-360

Exercises. LXXV. Miscellaneous Exercises.  360-362

% PAGE xvii
\begin{center}
PART II ORDINARY DIFFERENTIAL EQUATIONS OF
THE SECOND ORDBR. 363-374
\end{center}

\S 187. Special Types. 363

\S 188. Type I: $d^{2}s/dt^{2}=\pm k^{2}s$. 363-365

Exercises LXXVI. Type I. 365

\S 189. Type II. Homogeneous Linear --- Constant Coefficients 366-368

Exercises LXXVII. Type II. Linear Homogeneous 368-369

\S 190. Type III. Non-homogeneous Equations. 369-370

Exeroises LXXVIII. Non-homogeneous. Type III. 371

\S 191. Type IV. One of the quantities $x, y, y'$ absent. 371-374
\begin{center}
$(a)$ Type IV $(a):\phi(y'')=0$ (371)

$(b)$ Type IV $(b):\phi(x,\ y',\ y'')=0$ (371-373)

$(c)$ Type IV $(c):\phi\langle y, y', y'')=0$ (373-374)
\end{center}

Exercises LXXIX. Type IV. 374

\begin{center}
PART III GENERALIZATIONS 375-383
\end{center}

\S 192. Ordinary Equations of Higher Order. 375

\S 193. Linear Homogeneous Type. 375-377

\S 194. Non-homogeneous Type. 377-378

Exercises LXXX. Linear Equations of Higher Order 378-379

\S 195. Systems of Differential Equations 379

\S 196. Linear Systems of the First Order 379

\S 197. $dx/P=dy/Q=dz/R$. 379-381

Exercises LXXXI. Systems of Equations. 381-382

\S 198. Partial Differential Equations. 382

\S 199. Relation to Systems of Ordinary Equations 383

Exercises LXXXII. Partial Differential Equations 383

\begin{center}
TABLES

[Note page numbers in {\it italic} numerals]

TABLE I SIGNS AND ABBREVIATIONS {\it 1-3}

TABLE II STANDARD FORMULAS. {\it 3-16}
\end{center}

A. Exponents and Logarithms. {\it 3}

B. Factors. {\it 4}

C. Solution of Equations. Determinants. {\it 4-5}
% PAGE xix

D. Applications of Algebra. {\it 5-6}

E. Series. [Special. Theorems of Taylor and Fourier] {\it 7-8}

F. Geometric Magnitudes. Mensuration {\it 8-11}

G. Trigonometric Relations {\it 12-13}

H. Hyperbolic Functions {\it 13-14}

I. Analytic Geometry. {\it 14-15}

J. Differential Formulas {\it 15-16}

\begin{center}
TABLE III. STANDARD CURVES. {\it 17-32}
\end{center}

$L$ Curves $ y=x^{n}$. [Chart of Entire Family] {\it 17-18}

B. Logarithmic Paper. Curves $y=x^{n}, y=kx^{n}$. {\it 18}

C. Trigonometric Functions. {\it 19}

D. Logarithms and Exponentials. [Bases 10 and $e$]. {\it 19}

E. Exponential and Hyperbolic Functions. {\it 20}

F. Harmonic Curves. [Simple and Compound] {\it 21-22}

G. The Roulettes [Cycloid, Trochoids, etc.] {\it 23-24}

H. The Tractrlx. {\it 24}

I. Cubic and Quartic Curves. Contour Lines. {\it 25-27}

J. Error or Probability Curves . {\it 28}

K. Polynomial Approximations. [Taylor and Lagrange]. {\it 28-29}

L. Trigonometric Approximations. [Fourier] {\it 29}

M. Spirals. {\it 30-31}

N. Quadric Surfaces. {\it 31-32}

\begin{center}
TABLE IV STANDARD INTEGRALS {\it 33-48}
\end{center}

A. Fundamental General Formulas. {\it 33}

B. Integrand Rational Algebraic {\it 34-36}

C. Integrand Irrational . . {\it 37-39}
\begin{center}
$(a)$ Linear radicals $r=\sqrt{ax+b}$ {\it 37}

$(b)$ Quadratic radicals {\it 37-39}
\end{center}

D. Binomial Differentials--Reduction Formulas.{\it 39}

E. Integrand Transcendental {\it 39-44}
\begin{center}
$(a)$ Trigonometric.({\it 39-42})

$( b)$ Trigonometric --Algebraic. {(\it 42-43})

(c) Inverse Trigonometric. ({\it 43})

$(d)$ Exponential and Logarithmic ({\it 43-44})
\end{center}

F. Some Important Definite Integrals. {\it 44-45}

G. Approximation Formulas. {\it 45-46}

B Standard Applications of Integration. {\it 46-48}

% PAGE xx

\begin{center}
TABLE V. NUMERICAL TABLES. {\it 49-58}
\end{center}

A. Trigonometric functions. [Values and Logarithms] {\it 49}

B. Common Logarithms. {\it 50-51}

C. Exponential and Hyperbolic Functions. Natural Logarithms. {\it 52}

D. Elliptic Integral of the First Kind. {\it 53}

E. Elliptic Integral of the Second Kind. {\it 53}

F. Values of $\Pi(p)=\Gamma(p+1)$. Gamma Function. {\it 54}

G. Values of the Probability Integral. {\it 54}

H. Values of the Integral $\displaystyle \int_{-\infty}^{x}(e^{x}/x)\, dx$. {\it 54}

I. Reciprocals; Squares; Cubes. {\it 55}

J. Square Roots. {\it 56}

K. Radians to Degrees. {\it 56}

L. Important Constants .{\it 57}

M. Degrees to Radians. {\it 57}

N. Short Conversion Table and Other Data. {\it 58}

INDEX. {\it 59}
\end{document}
