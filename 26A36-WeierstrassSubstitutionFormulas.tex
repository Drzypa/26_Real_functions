\documentclass[12pt]{article}
\usepackage{pmmeta}
\pmcanonicalname{WeierstrassSubstitutionFormulas}
\pmcreated{2013-03-22 17:05:25}
\pmmodified{2013-03-22 17:05:25}
\pmowner{Wkbj79}{1863}
\pmmodifier{Wkbj79}{1863}
\pmtitle{Weierstrass substitution formulas}
\pmrecord{12}{39383}
\pmprivacy{1}
\pmauthor{Wkbj79}{1863}
\pmtype{Definition}
\pmcomment{trigger rebuild}
\pmclassification{msc}{26A36}
\pmclassification{msc}{33B10}
\pmsynonym{Weierstra{\ss} substitution formulas}{WeierstrassSubstitutionFormulas}
\pmrelated{GoniometricFormulae}
\pmrelated{IntegrationOfRationalFunctionOfSineAndCosine}
\pmrelated{PolynomialAnalogonForFermatsLastTheorem}
\pmdefines{Weierstrass substitution}
\pmdefines{Weiersta{\ss} substitution}
\pmdefines{universal trigonometric substitution}

\usepackage{amssymb}
\usepackage{amsmath}
\usepackage{amsfonts}

\usepackage{psfrag}
\usepackage{graphicx}
\usepackage{amsthm}
%%\usepackage{xypic}

\begin{document}
\PMlinkescapeword{formulas}
\PMlinkescapeword{restriction}

The \emph{Weierstrass substitution formulas} for $-\pi < x < \pi$ are:

\begin{center}
$\begin{array}{rl}
\sin x & =\displaystyle \frac{2t}{1+t^2} \\
& \\
\cos x & =\displaystyle \frac{1-t^2}{1+t^2}\\
& \\
dx & =\displaystyle \frac{2}{1+t^2} \, dt
\end{array}$
\end{center}

They can be obtained in the following manner:

Make the \emph{Weierstrass substitution} $\displaystyle t=\tan \left( \frac{x}{2} \right)$.  (This substitution is also known as the \emph{universal trigonometric substitution}.)  Then we have

\begin{center}
$\begin{array}{rl}
\displaystyle \cos \left( \frac{x}{2} \right) & = \displaystyle \frac{1}{\displaystyle \sec \left( \frac{x}{2} \right)} \\
& \\
& = \displaystyle \frac{1}{\displaystyle \sqrt{1+\tan^2\left( \frac{x}{2} \right)}} \\
& \\
& = \displaystyle \frac{1}{\sqrt{1+t^2}}
\end{array}$
\end{center}

and

\begin{center}
$\begin{array}{rl}
\displaystyle \sin \left( \frac{x}{2} \right) & = \displaystyle \cos \left( \frac{x}{2} \right) \cdot \tan \left( \frac{x}{2} \right) \\
& \\
& = \displaystyle \frac{t}{\sqrt{1+t^2}}.
\end{array}$
\end{center}

Note that these are just the ``formulas involving \PMlinkname{radicals}{Radical6}'' as designated in the entry goniometric formulas; however, due to the restriction on $x$, the $\pm$'s are unnecessary.

Using the above formulas along with the double angle formulas, we obtain

\begin{center}
$\begin{array}{rl}
\sin x & =\displaystyle 2\sin\left( \frac{x}{2} \right)\cdot \cos\left( \frac{x}{2} \right) \\
& \\
& =\displaystyle 2 \cdot \frac{t}{\sqrt{1+t^2}} \cdot \frac{1}{\sqrt{1+t^2}} \\
& \\
& =\displaystyle \frac{2t}{1+t^2}
\end{array}$
\end{center}

and

\begin{center}
$\begin{array}{rl}
\cos x & =\displaystyle \cos^2\left( \frac{x}{2} \right)-\sin^2\left( \frac{x}{2} \right) \\
& \\
& =\displaystyle \left( \frac{1}{\sqrt{1+t^2}} \right)^2-\left( \frac{t}{\sqrt{1+t^2}} \right)^2 \\
& \\
& =\displaystyle \frac{1}{1+t^2}-\frac{t^2}{1+t^2} \\
& \\
& =\displaystyle \frac{1-t^2}{1+t^2}.
\end{array}$
\end{center}

Finally, since $\displaystyle t=\tan\left( \frac{x}{2} \right)$, solving for $x$ yields that $x=2\arctan t$.  Thus, $\displaystyle dx=\frac{2}{1+t^2} \, dt$.

The Weierstrass substitution formulas are most useful for \PMlinkname{integrating rational functions of sine and cosine}{IntegrationOfRationalFunctionOfSineAndCosine}.
%%%%%
%%%%%
\end{document}
