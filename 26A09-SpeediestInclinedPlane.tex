\documentclass[12pt]{article}
\usepackage{pmmeta}
\pmcanonicalname{SpeediestInclinedPlane}
\pmcreated{2013-03-22 19:19:11}
\pmmodified{2013-03-22 19:19:11}
\pmowner{pahio}{2872}
\pmmodifier{pahio}{2872}
\pmtitle{speediest inclined plane}
\pmrecord{9}{42258}
\pmprivacy{1}
\pmauthor{pahio}{2872}
\pmtype{Example}
\pmcomment{trigger rebuild}
\pmclassification{msc}{26A09}
\pmclassification{msc}{26A06}
\pmrelated{Extremum}
\pmrelated{CalculusOfVariations}
\pmrelated{BrachistochroneCurve}

\endmetadata

% this is the default PlanetMath preamble.  as your knowledge
% of TeX increases, you will probably want to edit this, but
% it should be fine as is for beginners.

% almost certainly you want these
\usepackage{amssymb}
\usepackage{amsmath}
\usepackage{amsfonts}
\usepackage{pstricks}

% used for TeXing text within eps files
%\usepackage{psfrag}
% need this for including graphics (\includegraphics)
%\usepackage{graphicx}
% for neatly defining theorems and propositions
 \usepackage{amsthm}
% making logically defined graphics
%%\usepackage{xypic}

% there are many more packages, add them here as you need them

% define commands here

\theoremstyle{definition}
\newtheorem*{thmplain}{Theorem}

\begin{document}
We set the problem, how great must be the difference in altitude of the top and the bottom of an inclined plane in \PMlinkescapetext{order} that a little ball would frictionlessly roll the whole length of the plane as soon as possible 
(cf. the \PMlinkname{brachistochrone problem}{CalculusOfVariations}).\, It is assumed that the \PMlinkid{projection}{9475} of the length on a horizontal plane has a given value $b$.
\begin{center}
\begin{pspicture}(-3,-1)(3,3)
\psline[linewidth=0.05,linecolor=green](-2.5,0)(2,0)
\psline(-2.5,0)(2,0)(2,2.5)
\psline[linecolor=blue](2,2.5)(-2.5,0)
\psarc(-2.5,0){0.4}{0}{32}
\psline(1.8,0)(1.8,0.2)(2,0.2)
\rput(-1.9,0.15){$\alpha$}
\rput(0,-0.2){$b$}
\rput(2.2,1.25){$x$}
\rput(-1,1.4){$\sqrt{x^2\!+\!b^2}$}
\psdot[linewidth=0.1,linecolor=red](1.96,2.6)
\rput(-3,-1){.}
\rput(3,3){.}
\end{pspicture}
\end{center}

Using notations of mechanics, we can write
$$F \;=\; ma \;=\; mg\sin\alpha \;=\; m\frac{gx}{\sqrt{x^2\!+\!b^2}},$$
$$\sqrt{x^2\!+\!b^2} \;=\; s \;=\; \frac{1}{2}t^2a 
    \;=\; \frac{t^2}{2}\!\cdot\!\frac{gx}{\sqrt{x^2\!+\!b^2}}.$$
Thus we get the function
$$t^2 \;=\; \frac{2}{g}\!\cdot\!\frac{x^2\!+\!b^2}{x} \;=:\; f(x) \qquad(x > 0),$$
the absolute minimum point of which is to be found.\, This function is differentiable, and its derivative is
$$f'(x) \;=\; \frac{2}{g}\!\cdot\!\frac{x^2\!-\!b^2}{x^2}.$$
The only zero of $f'(x)$ is\, $x = b$,\, where the sign changes from minus to plus as $x$ increases.\, It means that\, $x = b$\, is the searched minimum point.\, The difference in altitude is thus equal to the \PMlinkid{base}{11642}, and the inclination $\alpha$ must be $45^\circ$.
%%%%%
%%%%%
\end{document}
