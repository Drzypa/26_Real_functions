\documentclass[12pt]{article}
\usepackage{pmmeta}
\pmcanonicalname{LipschitzConditionAndDifferentiabilityResult}
\pmcreated{2013-03-22 13:32:42}
\pmmodified{2013-03-22 13:32:42}
\pmowner{paolini}{1187}
\pmmodifier{paolini}{1187}
\pmtitle{Lipschitz condition and differentiability result}
\pmrecord{5}{34145}
\pmprivacy{1}
\pmauthor{paolini}{1187}
\pmtype{Result}
\pmcomment{trigger rebuild}
\pmclassification{msc}{26A16}

% this is the default PlanetMath preamble.  as your knowledge
% of TeX increases, you will probably want to edit this, but
% it should be fine as is for beginners.

% almost certainly you want these
\usepackage{amssymb}
\usepackage{amsmath}
\usepackage{amsfonts}

% used for TeXing text within eps files
%\usepackage{psfrag}
% need this for including graphics (\includegraphics)
%\usepackage{graphicx}
% for neatly defining theorems and propositions
\usepackage{amsthm}
% making logically defined graphics
%%%\usepackage{xypic}

% there are many more packages, add them here as you need them

% define commands here

\newtheorem{theorem}{Theorem}
\begin{document}
About lipschitz continuity of differentiable functions the following holds.

\begin{theorem}
Let $X,Y$ be Banach spaces and let $A$
be a convex (see convex set), open subset of $X$. 
Let $f\colon \overline{A}\to Y$ be a function which is continuous in $\overline A$ and differentiable in $A$. Then $f$ is lipschitz continuous on $\overline A$ 
if and only if the derivative $Df$ is bounded on $A$ i.e.
\[
   \sup_{x\in A} \Vert Df(x) \Vert < +\infty.
\]
\end{theorem}

\begin{proof}
Suppose that $f$ is lipschitz continuous:
\[
  \Vert f(x) - f(y)\Vert \le L \Vert x - y\Vert.
\]
Then given any $x\in A$ and any $v\in X$, for all small $h\in \mathbb R$ we have 
\[
   \Vert \frac{f(x+hv)-f(x)}{h}\Vert \le L.
\]
Hence, passing to the limit $h\to 0$ it must hold $\Vert Df(x)\Vert\le L$.

On the other hand suppose that $Df$ is bounded on $A$:
\[
  \Vert Df(x)\Vert \le L,\qquad \forall x \in A.
\]
Given any two points $x,y\in \overline A$ and given any $\alpha\in Y^*$ consider the function $G:[0,1]\to \mathbb R$
\[
  G(t) = \langle \alpha, f((1-t)x +t y)\rangle.
\]
For $t\in (0,1)$ it holds
\[
  G'(t) = \langle \alpha, Df((1-t)x+ty)[y-x]\rangle
\]
and hence
\[
   \vert G'(t)\vert \le L \Vert \alpha\Vert\, \Vert y-x\Vert.
\]
Applying Lagrange mean-value theorem to $G$ we know that there exists $\xi\in(0,1)$ such that
\[
  \vert \langle \alpha, f(y)-f(x)\rangle\vert =  \vert G(1)-G(0)\vert  = \vert G'(\xi)\vert \le \Vert \alpha\Vert L \Vert y-x\Vert 
\]
and since this is true for all $\alpha\in Y^*$ we get
\[
  \Vert f(y)-f(x) \Vert \le L \Vert y-x\Vert
\]
which is the desired claim.
\end{proof}
%%%%%
%%%%%
\end{document}
