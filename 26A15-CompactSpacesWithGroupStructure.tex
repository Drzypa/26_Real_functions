\documentclass[12pt]{article}
\usepackage{pmmeta}
\pmcanonicalname{CompactSpacesWithGroupStructure}
\pmcreated{2013-03-22 19:15:13}
\pmmodified{2013-03-22 19:15:13}
\pmowner{joking}{16130}
\pmmodifier{joking}{16130}
\pmtitle{compact spaces with group structure}
\pmrecord{6}{42179}
\pmprivacy{1}
\pmauthor{joking}{16130}
\pmtype{Corollary}
\pmcomment{trigger rebuild}
\pmclassification{msc}{26A15}
\pmclassification{msc}{54C05}

% this is the default PlanetMath preamble.  as your knowledge
% of TeX increases, you will probably want to edit this, but
% it should be fine as is for beginners.

% almost certainly you want these
\usepackage{amssymb}
\usepackage{amsmath}
\usepackage{amsfonts}

% used for TeXing text within eps files
%\usepackage{psfrag}
% need this for including graphics (\includegraphics)
%\usepackage{graphicx}
% for neatly defining theorems and propositions
%\usepackage{amsthm}
% making logically defined graphics
%%%\usepackage{xypic}

% there are many more packages, add them here as you need them

% define commands here

\begin{document}
\textbf{Proposition.} Assume that $(G,M)$ is a group (with multiplication $M:G\times G\to G$) and $G$ is also a topological space. If $G$ is compact Hausdorff and $M:G\times G\to G$ is continuous, then $(G,M)$ is a topological group.

\textit{Proof.} Indeed, all we need to show is that function $f:G\to G$ given by $f(g)=g^{-1}$ is continuous. Note, that the following holds for the graph of $f$:
$$\Gamma(f)=\{(g,f(g))\in G\times G\}=\{(g,g^{-1})\in G\times G\} = M^{-1}(e),$$
where $e$ denotes the neutral element in $G$. It follows (from continuity of $M$) that $\Gamma(f)$ is closed in $G\times G$. It is well known (see the parent object for details) that this implies that $f$ is continuous, which completes the proof. $\square$
%%%%%
%%%%%
\end{document}
