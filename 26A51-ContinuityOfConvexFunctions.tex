\documentclass[12pt]{article}
\usepackage{pmmeta}
\pmcanonicalname{ContinuityOfConvexFunctions}
\pmcreated{2013-03-22 15:28:00}
\pmmodified{2013-03-22 15:28:00}
\pmowner{pbruin}{1001}
\pmmodifier{pbruin}{1001}
\pmtitle{continuity of convex functions}
\pmrecord{5}{37319}
\pmprivacy{1}
\pmauthor{pbruin}{1001}
\pmtype{Result}
\pmcomment{trigger rebuild}
\pmclassification{msc}{26A51}
\pmclassification{msc}{26B25}
%\pmkeywords{continuity}
%\pmkeywords{convexity}

\endmetadata

% this is the default PlanetMath preamble.  as your knowledge
% of TeX increases, you will probably want to edit this, but
% it should be fine as is for beginners.

% almost certainly you want these
\usepackage{amssymb}
\usepackage{amsmath}
\usepackage{amsfonts}

% used for TeXing text within eps files
%\usepackage{psfrag}
% need this for including graphics (\includegraphics)
\usepackage{graphicx}
% for neatly defining theorems and propositions
%\usepackage{amsthm}
% making logically defined graphics
%%%\usepackage{xypic}

% there are many more packages, add them here as you need them

% define commands here
\begin{document}
\PMlinkescapeword{open}
We will prove below that every convex function on an
\PMlinkname{open}{Open} convex subset $A$ of a finite-dimensional
real vector space is continuous.  This statement becomes false if we
do not require $A$ to be open,
since we can increase the value of $f$ at any point of $A$ which is
not a convex combination of two other points without affecting the
convexity of $f$.  An example of this is shown in Figure \ref{fig2}.
\begin{figure}
\begin{center}
\includegraphics{convex.2}
\end{center}
\caption{A convex function on a non-open set need not be continuous.}
\label{fig2}
\end{figure}

Let $A$ be an open convex set in a finite-dimensional vector space $V$
over $\mathbb{R}$, and let $f\colon A\to\mathbb{R}$ be a convex
function.  Let $x\in A$ be arbitrary, and let $P$ be a parallelepiped
centered at $x$ and lying completely inside $A$.  Here ``a
parallelepiped centered at $x$'' means a subset of $V$ of the form
$$
P=\left\{x+\sum_{i=1}^n \lambda_ib_i\colon -1\le\lambda_i\le 1
\text{ for }i=1,2,\ldots,n\right\},
$$
where $\{b_1,\ldots,b_n\}$ is some basis of $V$.  Furthermore, let
$$
\partial P=\left\{x+\sum_{i=1}^n \lambda_ib_i\colon\max_{1\le i\le n}
\vert\lambda_i\vert=1\right\}
$$
denote the boundary of $P$.  We will show that $f$ is continuous at
$x$ by showing that $f$ attains a maximum on $\partial P$ and by
estimating $\vert f(y)-f(x)\vert$ in \PMlinkescapetext{terms}
of this maximum as $y\to x$.

The idea is to use the condition of convexity to `squeeze' the graph
of $f$ near $x$, as is shown in Figure \ref{fig1}.
\begin{figure}
\begin{center}
\includegraphics{convex.1}
\end{center}
\caption{Given the values of $f$ in $x$ and on
  $\partial P=\{y_1,y_2\}$, the convexity condition restricts the
  graph of $f$ to the grey area.}
\label{fig1}
\end{figure}

For $\lambda\in[0,1]$ and $y\in\partial P$, the convexity of $f$
implies
\begin{eqnarray}
\label{ineq1}
\nonumber
f\big((1-\lambda)x+\lambda y\big)&\le&(1-\lambda) f(x)+\lambda f(y) \\
&=&f(x)+\lambda\big(f(y)-f(x)\big).
\end{eqnarray}
On the other hand, for all $\mu\in[0,1/2]$ we have
\begin{eqnarray*}
f(x)&=&f\left((1-\mu)\left[\frac{(1-2\mu)x}{1-\mu}
+\frac{\mu y}{1-\mu}\right]+\mu(2x-y)\right) \\
&\le&(1-\mu)f\left(\frac{(1-2\mu)x}{1-\mu}
+\frac{\mu y}{1-\mu}\right)+\mu f(2x-y).
\end{eqnarray*}
Dividing by $1-\mu$ and setting $\lambda=\frac{\mu}{1-\mu}\in[0,1]$
gives
\begin{equation}
\label{ineq2}
(1+\lambda)f(x)\le f\big((1-\lambda)x+\lambda y\big)+\lambda f(2x-y).
\end{equation}
From the two inequalities (\ref{ineq1}) and (\ref{ineq2}) we obtain
\begin{equation}
\label{ineq3}
-\lambda\big(f(2x-y)-f(x)\big)\le f\big(x+\lambda(y-x)\big)-f(x)
\le\lambda\big(f(y)-f(x)\big).
\end{equation}
Note that both $y$ and $2x-y$ \PMlinkescapetext{lie on}
$\partial P$, and that $f$ is
bounded on $P$ (hence in particular on $\partial P$).  Indeed, the
convexity of $f$ implies that $f$ is bounded by its values at two
\PMlinkescapetext{opposite} faces of $P$, and repeatedly applying
this \PMlinkescapetext{property} shows
that $f$ attains a maximum at one of the corners of $P$.

Write $P_\lambda$ for the parallelepiped $P$ shrunk by a
\PMlinkescapetext{factor} $\lambda$ relative to $x$:
$$
P_\lambda=\{x+\lambda(y-x)\colon y\in P\}.
$$
Now the inequality (\ref{ineq3}) implies that for all
$\lambda\in[0,1]$ and all $z\in\partial
P_\lambda$, 
we have
$$
\left\vert f(z)-f(x)\right\vert\le\lambda
\left\vert\max_{y\in\partial P} f(y)-f(x)\right\vert.
$$
Consequently, the same inequality holds for all $\lambda\in(0,1]$ and
all $z$ in the open neighbourhood $P_\lambda\setminus\partial
P_\lambda$ of $x$.  The right-hand \PMlinkescapetext{side}
of this inequality goes to
zero as $\lambda\to 0$, from which we conclude that $f$ is continuous
at $x$.
%%%%%
%%%%%
\end{document}
