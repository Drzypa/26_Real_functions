\documentclass[12pt]{article}
\usepackage{pmmeta}
\pmcanonicalname{MacLaurinsInequality}
\pmcreated{2013-03-22 13:19:28}
\pmmodified{2013-03-22 13:19:28}
\pmowner{Mathprof}{13753}
\pmmodifier{Mathprof}{13753}
\pmtitle{MacLaurin's inequality}
\pmrecord{7}{33835}
\pmprivacy{1}
\pmauthor{Mathprof}{13753}
\pmtype{Definition}
\pmcomment{trigger rebuild}
\pmclassification{msc}{26D15}
%\pmkeywords{Young's Inequality}

\endmetadata

% this is the default PlanetMath preamble.  as your knowledge
% of TeX increases, you will probably want to edit this, but
% it should be fine as is for beginners.

% almost certainly you want these
\usepackage{amssymb}
\usepackage{amsmath}
\usepackage{amsfonts}

% used for TeXing text within eps files
%\usepackage{psfrag}
% need this for including graphics (\includegraphics)
%\usepackage{graphicx}
% for neatly defining theorems and propositions
%\usepackage{amsthm}
% making logically defined graphics
%%%\usepackage{xypic} 

% there are many more packages, add them here as you need them

% define commands here
\begin{document}
Let $a_1,a_2,\ldots,a_n$ be positive real numbers , and define the sums
$S_k$ as follows :
$$ S_k = \frac{\displaystyle \sum_{ 1\leq i_1 < i_2 < \cdots < i_k \leq n}a_{i_1} a_{i_2}
\cdots a_{i_k}}{\displaystyle {n \choose k}}$$
Then the following chain of
inequalities is true :
$$ S_1 \geq \sqrt{S_2} \geq \sqrt[3]{S_3} \geq \cdots \geq \sqrt[n]{S_n}$$
\textbf{Note} : $S_k$ are called the averages of the elementary symmetric sums
\\ This inequality is in fact important because it shows that the arithmetic-geometric mean inequality is nothing but a consequence of a chain of stronger inequalities
%%%%%
%%%%%
\end{document}
