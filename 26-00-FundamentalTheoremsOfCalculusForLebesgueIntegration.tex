\documentclass[12pt]{article}
\usepackage{pmmeta}
\pmcanonicalname{FundamentalTheoremsOfCalculusForLebesgueIntegration}
\pmcreated{2013-03-22 12:27:54}
\pmmodified{2013-03-22 12:27:54}
\pmowner{mathcam}{2727}
\pmmodifier{mathcam}{2727}
\pmtitle{fundamental theorems of calculus for Lebesgue integration}
\pmrecord{17}{32625}
\pmprivacy{1}
\pmauthor{mathcam}{2727}
\pmtype{Theorem}
\pmcomment{trigger rebuild}
\pmclassification{msc}{26-00}
\pmsynonym{first fundamental theorem of calculus}{FundamentalTheoremsOfCalculusForLebesgueIntegration}
\pmsynonym{second fundamental theorem of calculus}{FundamentalTheoremsOfCalculusForLebesgueIntegration}
\pmsynonym{fundamental theorem of calculus}{FundamentalTheoremsOfCalculusForLebesgueIntegration}
\pmrelated{FundamentalTheoremOfCalculusClassicalVersion}
\pmrelated{FundamentalTheoremOfCalculusForRiemannIntegration}
\pmrelated{ChangeOfVariableInDefiniteIntegral}

\endmetadata

% this is the default PlanetMath preamble.  as your knowledge
% of TeX increases, you will probably want to edit this, but
% it should be fine as is for beginners.

% almost certainly you want these
\usepackage{amssymb}
\usepackage{amsmath}
\usepackage{amsfonts}

% used for TeXing text within eps files
%\usepackage{psfrag}
% need this for including graphics (\includegraphics)
%\usepackage{graphicx}
% for neatly defining theorems and propositions
%\usepackage{amsthm}
% making logically defined graphics
%%%\usepackage{xypic} 

% there are many more packages, add them here as you need them

% define commands here
\begin{document}
Loosely, the \emph{Fundamental Theorems of Calculus} serve to demonstrate that integration and differentiation are inverse processes.  Suppose that $F(x)$ is an absolutely continuous function on an interval $[a,b]\subset\mathbb{R}$.  The two following forms of the theorem are equivalent.

{\bf First form of the Fundamental Theorem:}

There exists a function $f(t)$ Lebesgue-integrable on $[a,b]$ such that for any $x\in [a,b]$, we have $F(x)-F(a)=\int_a^x f(t) dt$.

{\bf Second form of the Fundamental Theorem:}

$F(x)$ is differentiable almost everywhere on $[a,b]$ and its derivative, denoted $F'(x)$, is Lebesgue-integrable on that interval.  In addition, we have the relation $F(x)-F(a)=\int_a^x F'(t)dt$ for any $x\in [a,b]$.
%%%%%
%%%%%
\end{document}
