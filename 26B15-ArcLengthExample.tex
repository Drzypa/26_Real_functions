\documentclass[12pt]{article}
\usepackage{pmmeta}
\pmcanonicalname{ArcLengthExample}
\pmcreated{2013-03-22 18:57:43}
\pmmodified{2013-03-22 18:57:43}
\pmowner{pahio}{2872}
\pmmodifier{pahio}{2872}
\pmtitle{arc length example}
\pmrecord{15}{41820}
\pmprivacy{1}
\pmauthor{pahio}{2872}
\pmtype{Example}
\pmcomment{trigger rebuild}
\pmclassification{msc}{26B15}
\pmsynonym{logarithm of sine function}{ArcLengthExample}
\pmrelated{SubstitutionNotation}

\endmetadata

% this is the default PlanetMath preamble.  as your knowledge
% of TeX increases, you will probably want to edit this, but
% it should be fine as is for beginners.

% almost certainly you want these
\usepackage{amssymb}
\usepackage{amsmath}
\usepackage{amsfonts}
\usepackage{pstricks}
\usepackage{pst-plot}

% used for TeXing text within eps files
%\usepackage{psfrag}
% need this for including graphics (\includegraphics)
%\usepackage{graphicx}
% for neatly defining theorems and propositions
 \usepackage{amsthm}
% making logically defined graphics
%%%\usepackage{xypic}

% there are many more packages, add them here as you need them

% define commands here

\newcommand{\sijoitus}[2]%
{\operatornamewithlimits{\Big/}_{\!\!\!#1}^{\,#2}}
\begin{document}
\PMlinkescapeword{ln}

The functions
$$x \;\mapsto\; \ln\sin{x} \quad \mbox{and} \quad x \;\mapsto\, \ln\cos{x}$$
belong to the few real functions, the arc length of which are expressible in closed form (other ones are mentioned in the entry arc length of parabola).\\

We calculate the arc length of the curve
$$y \;=\; \ln\sin{x}\qquad (0 \;<\; a \;\leqq\; x \;\leqq\; \frac{\pi}{2}).$$
By the chain rule, we have
$$y' \;=\; \frac{1}{\sin{x}}\cdot\cos{x} \;=\; \cot{x}.$$
Hence the arc length is
$$s \;=\; \int_a^{\frac{\pi}{2}}\!\sqrt{1+(\cot{x})^2}\,dx \;=\; \int_a^{\frac{\pi}{2}}\!\frac{1}{\sin{x}}\,dx
\;=\; \sijoitus{a}{\quad\frac{\pi}{2}}\!\ln|\tan\frac{x}{2}| \;=\;\ln1-\ln|\tan\frac{a}{2}| \;=\; \ln\cot\frac{a}{2}$$
(see integration of rational function of sine and cosine). \\

\begin{center}
\begin{pspicture}(-1,-6)(4,2.5)
\psaxes[Dx=10,Dy=10]{->}(0,0)(-0.5,-5.5)(3.5,2)
\rput(3.6,0.2){$x$}
\rput(0.2,2.1){$y$}
\psplot[linecolor=blue]{0.01}{1.5708}{180 x mul 3.14159 div sin ln}
\rput(2,-1.5){$y = \ln\sin{x}$}
\psdot[linecolor=blue](1.5708,0)
\rput(1.57,-0.35){$\frac{\pi}{2}$}
\psline(-0.07,1)(0.07,1)
\rput(-0.2,1){1}
\rput(-1,-6){.}
\rput(4,2.5){.}
\end{pspicture}
\end{center}
%%%%%
%%%%%
\end{document}
