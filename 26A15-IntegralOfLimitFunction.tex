\documentclass[12pt]{article}
\usepackage{pmmeta}
\pmcanonicalname{IntegralOfLimitFunction}
\pmcreated{2013-03-22 19:01:41}
\pmmodified{2013-03-22 19:01:41}
\pmowner{pahio}{2872}
\pmmodifier{pahio}{2872}
\pmtitle{integral of limit function}
\pmrecord{10}{41900}
\pmprivacy{1}
\pmauthor{pahio}{2872}
\pmtype{Theorem}
\pmcomment{trigger rebuild}
\pmclassification{msc}{26A15}
\pmclassification{msc}{40A30}
\pmrelated{TermwiseDifferentiation}

\endmetadata

% this is the default PlanetMath preamble.  as your knowledge
% of TeX increases, you will probably want to edit this, but
% it should be fine as is for beginners.

% almost certainly you want these
\usepackage{amssymb}
\usepackage{amsmath}
\usepackage{amsfonts}

% used for TeXing text within eps files
%\usepackage{psfrag}
% need this for including graphics (\includegraphics)
%\usepackage{graphicx}
% for neatly defining theorems and propositions
 \usepackage{amsthm}
% making logically defined graphics
%%%\usepackage{xypic}

% there are many more packages, add them here as you need them

% define commands here

\theoremstyle{definition}
\newtheorem*{thmplain}{Theorem}

\begin{document}
\textbf{Theorem.}\, If a sequence $f_1,\,f_2,\,\ldots$ of real functions, continuous on the interval \,$[a,\,b]$,\, converges uniformly on this interval to the limit function $f$, then
\begin{align}
\int_a^b\!f(x)\,dx \;=\; \lim_{n\to\infty}\int_a^b\!f_n(x)\,dx.
\end{align}


\emph{Proof.}\, Let\, $\varepsilon > 0$.\, The uniform continuity implies the existence of a positive integer 
$n_\varepsilon$ such that 
$$|f_n(x)\!-\!f(x)| \;<\; \frac{\varepsilon}{b\!-\!a} \quad \forall x \in [a,\,b] \qquad \mbox{when}\;\; 
n \,>\, n_\varepsilon.$$
The function $f$ is continuous (see \PMlinkid{this}{7191}) and thus \PMlinkname{Riemann integrable}{RiemannIntegral} (see \PMlinkid{this}{4461}) on the interval.\, Utilising the estimation theorem of integral, we obtain
$$\left|\int_a^b\!f_n(x)\,dx\!-\!\int_a^b\!f(x)\,dx\right| \,=\, \left|\int_a^b\!(f_n(x)\!-\!f(x))\,dx\right| 
\,\leqq\, \int_a^b\!|f_n(x)\!-\!f(x)|\,dx \,<\, \frac{\varepsilon}{b\!-\!a}(b\!-\!a) \,=\, \varepsilon$$
as soon as\, $n > n_\varepsilon$.\, Consequently, (1) is true.\\

\textbf{Remark 1.}\, The equation (1) may be written in the form
\begin{align}
\int_a^b\!\lim_{n\to\infty}f_n(x)\,dx \;=\; \lim_{n\to\infty}\int_a^b\!f_n(x)\,dx,
\end{align}
i.e. under the assumptions of the theorem, the integration and the limit process can be interchanged.\\

\textbf{Remark 2.}\, Considering the partial sums of a series $\sum_{n=1}^\infty f_n(x)$ with continuous terms and converging uniformly on\, $[a,\,b]$,\, one gets from the theorem the result analogous to (2):
\begin{align}
\int_a^b\!\sum_{n=1}^\infty f_n(x)\,dx \;=\; \sum_{n=1}^\infty\int_a^b\!f_n(x)\,dx.
\end{align}


%%%%%
%%%%%
\end{document}
