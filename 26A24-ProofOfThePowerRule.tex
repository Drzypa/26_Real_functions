\documentclass[12pt]{article}
\usepackage{pmmeta}
\pmcanonicalname{ProofOfThePowerRule}
\pmcreated{2013-03-22 12:28:06}
\pmmodified{2013-03-22 12:28:06}
\pmowner{mathcam}{2727}
\pmmodifier{mathcam}{2727}
\pmtitle{proof of the power rule}
\pmrecord{9}{32631}
\pmprivacy{1}
\pmauthor{mathcam}{2727}
\pmtype{Proof}
\pmcomment{trigger rebuild}
\pmclassification{msc}{26A24}
\pmrelated{ProductRule}
\pmrelated{Derivative2}

\endmetadata

\usepackage{amssymb}
\usepackage{amsmath}
\usepackage{amsfonts}
\newcommand{\D}[1]{\ensuremath{\mathrm{d}#1}}
\newcommand{\DDX}{\ensuremath{\frac{\D{}}{\D{x}}}}
\begin{document}
The power rule can be derived by repeated application of the product rule.

\subsubsection*{Proof for all positive integers $n$}

The power rule has been shown to hold for $n = 0$ and $n = 1$.
If the power rule is known to hold for some $k > 0$, then we have

\begin{eqnarray*} 
\DDX x^{k+1}
& = & \DDX (x\cdot x^k) \\
& = & x\left(\DDX x^k\right) + x^k \\
& = & x\cdot(kx^{k-1}) + x^k \\
& = & kx^k + x^k \\
& = & (k + 1)x^k
\end{eqnarray*}

Thus the power rule holds for all positive integers $n$.

\subsubsection*{Proof for all positive rationals $n$}

Let $y = x^{p/q}$.  We need to show

\begin{equation}
\frac{\D{y}}{\D{x}}(x^{p/q}) = \frac{p}{q}x^{p/q-1} \label{prat}
\end{equation}

The proof of this comes from implicit differentiation.

By definition, we have $y^q = x^p$.  We now take the derivative with respect to $x$ on both sides of the equality.

\begin{eqnarray*}
\DDX y^q & = & \DDX x^p \\ 
\frac{\D{}}{\D{y}}(y^q)\frac{\D{y}}{\D{x}} & = & px^{p-1} \\
qy^{q-1}\frac{\D{y}}{\D{x}} & = & px^{p-1} \\
\frac{\D{y}}{\D{x}} & = & \frac{p}{q}\frac{x^{p-1}}{y^{q-1}} \\
&=& \frac{p}{q}x^{p-1}y^{1-q} \\
&=& \frac{p}{q}x^{p-1}x^{p(1-q)/q} \\
&=& \frac{p}{q}x^{p-1+p/q-p} \\
&=& \frac{p}{q}x^{p/q-1}
\end{eqnarray*}

\subsubsection*{Proof for all positive irrationals $n$}

For positive irrationals we claim continuity due to the fact that (\ref{prat})
holds for all positive rationals, and there are positive rationals that approach any positive irrational.

\subsubsection*{Proof for negative powers $n$}

We again employ implicit differentiation.  Let $u = x$, and differentiate $u^n$ with respect to $x$ for some non-negative $n$.  We must show

\begin{equation}
\frac{\D{u^{-n}}}{\D{x}} = -nu^{-n-1} \label{neg}
\end{equation}

By definition we have $u^nu^{-n} = 1$.
We begin by taking the derivative with respect to $x$ on both sides of the equality.  By application of the product rule we get

\begin{eqnarray*}
\DDX(u^nu^{-n}) & = & 1 \\
u^n\frac{\D{u^{-n}}}{\D{x}} + u^{-n}\frac{\D{u^n}}{\D{x}} & = & 0\, \\
u^n\frac{\D{u^{-n}}}{\D{x}} + u^{-n}(nu^{n-1}) & = & 0\, \\
u^n\frac{\D{u^{-n}}}{\D{x}} & = & -nu^{-1} \\
\frac{\D{u^{-n}}}{\D{x}} & = & -nu^{-n-1}
\end{eqnarray*}
%%%%%
%%%%%
\end{document}
