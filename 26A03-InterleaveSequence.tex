\documentclass[12pt]{article}
\usepackage{pmmeta}
\pmcanonicalname{InterleaveSequence}
\pmcreated{2013-03-22 11:52:12}
\pmmodified{2013-03-22 11:52:12}
\pmowner{djao}{24}
\pmmodifier{djao}{24}
\pmtitle{interleave sequence}
\pmrecord{7}{30449}
\pmprivacy{1}
\pmauthor{djao}{24}
\pmtype{Definition}
\pmcomment{trigger rebuild}
\pmclassification{msc}{26A03}
\pmclassification{msc}{40-00}
\pmclassification{msc}{11A15}

\endmetadata

\usepackage{amssymb}
\usepackage{amsmath}
\usepackage{amsfonts}
\usepackage{graphicx}
%%%%\usepackage{xypic}
\begin{document}
Let $S$ be a set, and let $\{x_i\},\ i=0,1,2,\dots$ and $\{y_i\},\ i=0,1,2,\dots$ be two sequences in $S$. The {\em interleave sequence} is defined to be the sequence $x_0, y_0, x_1, y_1, \dots$. Formally, it is the sequence $\{z_i\},\ i=0,1,2,\dots$ given by
$$
z_i :=
\begin{cases}
x_k & \text{\ \ if } i=2k \text{ is even,}\\
y_k & \text{\ \ if } i=2k+1 \text{ is odd.}
\end{cases}
$$
%%%%%
%%%%%
%%%%%
%%%%%
\end{document}
