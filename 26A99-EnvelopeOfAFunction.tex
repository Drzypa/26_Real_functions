\documentclass[12pt]{article}
\usepackage{pmmeta}
\pmcanonicalname{EnvelopeOfAFunction}
\pmcreated{2013-03-22 15:44:22}
\pmmodified{2013-03-22 15:44:22}
\pmowner{cvalente}{11260}
\pmmodifier{cvalente}{11260}
\pmtitle{envelope of a function}
\pmrecord{5}{37690}
\pmprivacy{1}
\pmauthor{cvalente}{11260}
\pmtype{Definition}
\pmcomment{trigger rebuild}
\pmclassification{msc}{26A99}

\endmetadata

% this is the default PlanetMath preamble.  as your knowledge
% of TeX increases, you will probably want to edit this, but
% it should be fine as is for beginners.

% almost certainly you want these
\usepackage{amssymb}
\usepackage{amsmath}
\usepackage{amsfonts}

% used for TeXing text within eps files
%\usepackage{psfrag}
% need this for including graphics (\includegraphics)
%\usepackage{graphicx}
% for neatly defining theorems and propositions
%\usepackage{amsthm}
% making logically defined graphics
%%%\usepackage{xypic}

% there are many more packages, add them here as you need them

% define commands here
\DeclareMathOperator{\env}{env}
\begin{document}
Consider $f:\mathbb{R} \to \mathbb{R}$ a real function of real variable.

We call the upper envelope of $f$ to the function:

$\env_{\sup}(f)(x) = \inf_{\epsilon} \{\sup \{ f(y): \epsilon>0,\: |y-x|<\epsilon \}\}$

similarly the lower envelope of $f$ is the function:

$\env_{\inf}(f)(x) = \sup_{\epsilon} \{\inf \{ f(y): \epsilon>0,\: |y-x|<\epsilon \}\}$

The envelopes have the following properties:
(in this list $\env_\ast$ represents either the upper or lower envelope)

\begin{itemize}
\item $ \env_{\inf}(f)(x) \le f(x) \le \env_{\sup}(f)(x)$
\item $ \env_{\sup}(f) = \env_{\inf}(f) \iff f \text{is continuous} $
\item $ \env_{\sup}(f)(x) - \env_{\inf}(f)(x) = \text{oscillation of} f \text{at } x$
\item $\env_{\inf} f = - \env_{\sup}(-f)$
\end{itemize}
%%%%%
%%%%%
\end{document}
