\documentclass[12pt]{article}
\usepackage{pmmeta}
\pmcanonicalname{LandauNotation}
\pmcreated{2013-03-22 11:42:56}
\pmmodified{2013-03-22 11:42:56}
\pmowner{Mathprof}{13753}
\pmmodifier{Mathprof}{13753}
\pmtitle{Landau notation}
\pmrecord{28}{30090}
\pmprivacy{1}
\pmauthor{Mathprof}{13753}
\pmtype{Definition}
\pmcomment{trigger rebuild}
\pmclassification{msc}{26A12}
\pmclassification{msc}{20H15}
\pmclassification{msc}{20B30}
\pmsynonym{O notation}{LandauNotation}
\pmsynonym{omega notation}{LandauNotation}
\pmsynonym{theta notation}{LandauNotation}
\pmsynonym{big-O notation}{LandauNotation}
%\pmkeywords{complexity}
%\pmkeywords{estimate}
%\pmkeywords{estimation}
%\pmkeywords{runtime complexity}
%\pmkeywords{space complexity}
\pmrelated{LowerBoundForSorting}
\pmrelated{ConvergenceOfIntegrals}
\pmdefines{big-o}
\pmdefines{small-o}
\pmdefines{small-omega}

\usepackage{amssymb}
\usepackage{amsmath}
\usepackage{amsfonts}
\begin{document}
\PMlinkescapeword{group}
Given two functions $f$ and $g$ from $\mathbb{R}^+$ to $\mathbb{R}^+$,
the notation
$$f=O(g)$$
means that the ratio $\displaystyle \frac{f(x)}{g(x)}$
stays bounded as $x\to\infty$. If moreover that ratio approaches zero,
we write
$$f=o(g).$$

It is legitimate to write, say, $2x=O(x)=O(x^2)$, with the understanding
that we are using the equality sign in an unsymmetric (and informal) way,
in that we do not have, for example, $O(x^2)=O(x)$.

The notation
$$f=\Omega(g)$$
means that the ratio $\displaystyle \frac{f(x)}{g(x)}$
is bounded away from zero as $x\to\infty$, or equivalently $g=O(f)$.

If both $f=O(g)$ and $f=\Omega(g)$, we write $f=\Theta(g)$.

One more notational convention in this group is $$f(x)\sim g(x),$$
meaning $\displaystyle \lim_{x\to\infty}\frac{f(x)}{g(x)}=1$.

In analysis, such notation is useful in describing error \PMlinkname{estimates}{AsymptoticEstimate}.
For example, the Riemann hypothesis is equivalent to the conjecture
$$\pi(x)=\operatorname{li} x+O(\sqrt{x}\log x),$$ where $\operatorname{li} x$ denotes the logarithmic integral.

Landau notation is also handy in applied mathematics, e.g. in describing
the time complexity of an algorithm. It is common to say that an algorithm
requires $O(x^3)$ steps, for example, without needing to specify exactly what
is a step; for if $f=O(x^3)$, then $f=O(Ax^3)$ for any positive constant
$A$.
%%%%%
%%%%%
%%%%%
%%%%%
%%%%%
%%%%%
%%%%%
%%%%%
%%%%%
%%%%%
%%%%%
%%%%%
\end{document}
