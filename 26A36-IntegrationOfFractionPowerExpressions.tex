\documentclass[12pt]{article}
\usepackage{pmmeta}
\pmcanonicalname{IntegrationOfFractionPowerExpressions}
\pmcreated{2013-03-22 17:50:33}
\pmmodified{2013-03-22 17:50:33}
\pmowner{pahio}{2872}
\pmmodifier{pahio}{2872}
\pmtitle{integration of fraction power expressions}
\pmrecord{7}{40313}
\pmprivacy{1}
\pmauthor{pahio}{2872}
\pmtype{Application}
\pmcomment{trigger rebuild}
\pmclassification{msc}{26A36}
\pmrelated{FractionPower}
\pmrelated{RationalFunction}
\pmrelated{IntegrationBySubstitution}
\pmrelated{SubstitutionForIntegration}

% this is the default PlanetMath preamble.  as your knowledge
% of TeX increases, you will probably want to edit this, but
% it should be fine as is for beginners.

% almost certainly you want these
\usepackage{amssymb}
\usepackage{amsmath}
\usepackage{amsfonts}

% used for TeXing text within eps files
%\usepackage{psfrag}
% need this for including graphics (\includegraphics)
%\usepackage{graphicx}
% for neatly defining theorems and propositions
 \usepackage{amsthm}
% making logically defined graphics
%%%\usepackage{xypic}

% there are many more packages, add them here as you need them

% define commands here

\theoremstyle{definition}
\newtheorem*{thmplain}{Theorem}

\begin{document}
The antiderivatives of every expression containing fraction powers can not be expressed by using elementary functions. However, there are \PMlinkescapetext{types in which the integration succeeds} after making a substitution.

\begin{itemize}
\item $\displaystyle\int R(x,\,x^{r_1},\,\ldots,\,x^{r_m})\,dx$,\, where $R$ means a rational function of its arguments.
If the common denominator of the fraction power exponents $r_j$ is $n$, the substitution
$$x \;:=\; t^n, \qquad dx \;=\; nt^{n-1}dt$$
changes each exponent to an integer and the whole integrand to a rational function in the variable $t$.\\

\textbf{Example.}\, For\, $\displaystyle\int\frac{x^{\frac{1}{2}}}{x^{\frac{3}{4}}+1}\,dx$\, the least common multiple of the denominators of $\frac{1}{2}$ and $\frac{3}{4}$ is 4, whence we make the substitution\, $x = t^4$,\, $dx = 4t^3dt$.\, Then we obtain
$$\int\frac{x^{\frac{1}{2}}}{x^{\frac{3}{4}}+1}\,dx \;=\; 4\!\int\frac{t^5dt}{t^3+1} \;=\; 4\!\int\left(t^2-\frac{t^2}{t^3+1}\right)dt
\;=\; 4\left(\frac{t^3}{3}-\frac{1}{3}\ln|t^3+1|\right)+C$$ 
$$=\; \frac{4}{3}\left(x^{\frac{3}{4}}-\ln|x^{\frac{3}{4}}+1|\right)+C.\\$$


\item In $\displaystyle\int R\left(x,\,\left(\frac{ax+b}{cx+d}\right)^{r_1},\,\ldots,\,\left(\frac{ax+b}{cx+d}\right)^{r_m}\right)\,dx$,\, correspondently the substitution
$$\frac{ax+b}{cx+d} \;:=\; t^n$$
changes the integrand to a rational function.\\

\textbf{Example.}\, For\, $\displaystyle\int\frac{\sqrt{x+4}}{x}\,dx$\, we substitute\, $x+4 = t^2$,\, $dx = 2t\,dt$,\, getting
$$\int\frac{\sqrt{x+4}}{x}\,dx \;=\;  2\!\int\frac{t^2}{t^2-4}\,dt \;=\; 2\!\int\left(1+\frac{4}{t^2-4}\right)dt 
\;=\; 2t+2\ln\left|\frac{t-2}{t+2}\right|+C$$
$$=\; 2\sqrt{x+4}+2\ln\left|\frac{\sqrt{x+4}-2}{\sqrt{x+4}+2}\right|+C.$$
\end{itemize} 

\begin{thebibliography}{9}
\bibitem{NP}{\sc N. Piskunov:} {\em Diferentsiaal- ja integraalarvutus k\~{o}rgematele tehnilistele \~{o}ppeasutustele}.  Viies, t\"aiendatud tr\"ukk.\, Kirjastus ``Valgus'', Tallinn  (1965).
\end{thebibliography}
%%%%%
%%%%%
\end{document}
