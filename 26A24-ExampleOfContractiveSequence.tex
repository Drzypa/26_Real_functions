\documentclass[12pt]{article}
\usepackage{pmmeta}
\pmcanonicalname{ExampleOfContractiveSequence}
\pmcreated{2014-09-22 8:10:59}
\pmmodified{2014-09-22 8:10:59}
\pmowner{pahio}{2872}
\pmmodifier{pahio}{2872}
\pmtitle{example of contractive sequence}
\pmrecord{5}{88159}
\pmprivacy{1}
\pmauthor{pahio}{2872}
\pmtype{Example}

% this is the default PlanetMath preamble.  as your knowledge
% of TeX increases, you will probably want to edit this, but
% it should be fine as is for beginners.

% almost certainly you want these
\usepackage{amssymb}
\usepackage{amsmath}
\usepackage{amsfonts}

% need this for including graphics (\includegraphics)
\usepackage{graphicx}
% for neatly defining theorems and propositions
\usepackage{amsthm}

% making logically defined graphics
%\usepackage{xypic}
% used for TeXing text within eps files
%\usepackage{psfrag}

% there are many more packages, add them here as you need them

% define commands here

\begin{document}
Define the sequence\, $a_1,a_2,a_3,\ldots$\, by
\begin{align}
a_1 := 1, \qquad a_{n+1} := \sqrt{5-2a_n} \quad (n = 1,2,3,\ldots}).
\end{align}
We see by induction that the radicand in (1) cannot become 
negative; in fact we justify that
\begin{align}
1 \leqq a_n \leqq \sqrt{3}
\end{align}
for every $n$:\,  It's clear when $n = 1$.  If it is true for 
an $a_n$, it implies that $1 < 5-2a_n \leqq 3$, i.e. 
$1 < a_{n+1} \leqq \sqrt{3}$.

As for the convergence of the sequence, which is not monotonic, one could think to show that 
it is a Cauchy sequence.  Unfortunately, it is almost impossible to directly express and 
estimate the needed absolute value of $a_m-a_n$.  Fortunately, 
the recursive definition (1) allows quite easily to estimate 
$|a_n-a_{n+1}|$.\,  Then it turns out that it's a question of a 
contractive sequence, whence it is by the \PMlinkname{parent entry}{ContractiveSequence} a 
Cauchy sequence.

We form the difference
\begin{align*}
a_n-a_{n+1} &= 
\frac{(\sqrt{5-2a_{n-1}}-\sqrt{5-2a_n})(\sqrt{5-2a_{n-1}}+\sqrt{5-2a_n})}
{\sqrt{5-2a_{n-1}}+\sqrt{5-2a_n}}\\
&= \frac{-2(a_{n-1}-a_n)}{\sqrt{5-2a_{n-1}}+\sqrt{5-2a_n}}
\end{align*}
where $n > 1$.\, Thus we can estimate its absolute value, by 
using (2):
$$|a_n-a_{n+1}| = 
\frac{2|a_{n-1}-a_n|}{\sqrt{5-2a_{n-1}}+\sqrt{5-2a_n}} \leqq
\frac{2|a_{n-1}-a_n|}{\sqrt{5-2\sqrt{3}}+\sqrt{5-2\sqrt{3}}} =
\frac{|a_{n-1}-a_n|}{\sqrt{5-2\sqrt{3}}}$$
Since $\frac{1}{\sqrt{5-2\sqrt{3}}} < 1$, our sequence (1) is 
contractive, consequently Cauchy.\, Therefore it converges to 
a limit $A$.

We have 
$$A^2 = \left(\lim_{n \to \infty}a_{n+1}\right)^2 = 
\lim_{n \to \infty}a_{n+1}^2 = 
\lim_{n \to \infty}(5-2a_n) = 5-2A.$$
From the quadratic equation\, $A^2+2A-5 = 0$\, we get the positive 
root $A = \sqrt{6}-1$. I.e.,
\begin{align}
\lim_{n\to\infty}a_n = \sqrt{6}-1.
\end{align\

\end{document}
