\documentclass[12pt]{article}
\usepackage{pmmeta}
\pmcanonicalname{Exponential}
\pmcreated{2013-03-22 12:29:59}
\pmmodified{2013-03-22 12:29:59}
\pmowner{rmilson}{146}
\pmmodifier{rmilson}{146}
\pmtitle{exponential}
\pmrecord{17}{32730}
\pmprivacy{1}
\pmauthor{rmilson}{146}
\pmtype{Definition}
\pmcomment{trigger rebuild}
\pmclassification{msc}{26A03}
\pmsynonym{exponential operation}{Exponential}
\pmrelated{RealNumber}
\pmdefines{exponent}
\pmdefines{power}

\usepackage{amsmath}
\usepackage{amsfonts}
\usepackage{amssymb}

\newcommand{\reals}{\mathbb{R}}
\newcommand{\natnums}{\mathbb{N}}
\newcommand{\cnums}{\mathbb{C}}
\newcommand{\znums}{\mathbb{Z}}

\newcommand{\lp}{\left(}
\newcommand{\rp}{\right)}
\newcommand{\lb}{\left[}
\newcommand{\rb}{\right]}

\newcommand{\supth}{^{\text{th}}}


\newtheorem{proposition}{Proposition}
\begin{document}
\paragraph{Preamble.}  We use $\reals^+\subset\reals$ to denote the set of
positive real numbers. Our aim is to define the exponential, or
the generalized power operation,  
$$x^p,\quad x\in\reals^+,\; p\in\reals.$$
The power index $p$ in the above
expression is called the exponent. We take it as proven that $\reals$
is a complete, ordered field.  No other properties of the real numbers
are invoked.

\paragraph{Definition.}
For $x\in\reals^+$ and $n\in\znums$ we define $x^n$ in terms of
repeated multiplication.  To be more precise, we inductively
characterize natural number powers as follows:
$$x^0 = 1,\quad x^{n+1} = x\cdot x^n,\quad n\in\natnums.$$
The existence of the
reciprocal is guaranteed by the assumption that $\reals$ is a field.
Thus, for negative exponents, we can define
$$x^{-n} = (x^{-1})^n,\quad n\in\natnums,$$
where $x^{-1}$ is the reciprocal of $x$.

The case of arbitrary exponents is somewhat more complicated.  A
possible strategy is to define roots, then rational powers, and then
extend by continuity.  Our approach is different.  For $x\in\reals^+$
and $p\in \reals$, we define the set of all reals that one would want
to be smaller than $x^p$, and then define the latter as the least
upper bound of this set.  To be more precise, let $x>1$ and define
$$L(x,p)=\{ z\in\reals^+: z^n<x^m \text{ for all } m\in\znums,\; n\in
\natnums \text{ such that } m<pn\}.$$
We then define $x^p$ to be the least upper bound of $L(x,p)$.
For $x<1$ we define  
$$x^p = (x^{-1})^{-p}.$$



The exponential operation possesses a number of
important \PMlinkname{properties}{PropertiesOfTheExponential},
some of which characterize it up to uniqueness.


\paragraph{Note.} It is also possible to define the exponential operation in
terms of the exponential function and the natural logarithm. Since these concepts require
the context of
differential theory, it seems preferable to give a basic definition
that relies only on the foundational property of the reals.
%%%%%
%%%%%
\end{document}
