\documentclass[12pt]{article}
\usepackage{pmmeta}
\pmcanonicalname{DifferentialEquationsForXx}
\pmcreated{2013-03-22 17:24:37}
\pmmodified{2013-03-22 17:24:37}
\pmowner{rspuzio}{6075}
\pmmodifier{rspuzio}{6075}
\pmtitle{differential equations for $x^x$}
\pmrecord{6}{39783}
\pmprivacy{1}
\pmauthor{rspuzio}{6075}
\pmtype{Derivation}
\pmcomment{trigger rebuild}
\pmclassification{msc}{26A99}

% this is the default PlanetMath preamble.  as your knowledge
% of TeX increases, you will probably want to edit this, but
% it should be fine as is for beginners.

% almost certainly you want these
\usepackage{amssymb}
\usepackage{amsmath}
\usepackage{amsfonts}

% used for TeXing text within eps files
%\usepackage{psfrag}
% need this for including graphics (\includegraphics)
%\usepackage{graphicx}
% for neatly defining theorems and propositions
%\usepackage{amsthm}
% making logically defined graphics
%%%\usepackage{xypic}

% there are many more packages, add them here as you need them

% define commands here

\begin{document}
In this entry, we will derive differential equations satisfied by the function $x^x$.
\footnote{In this entry, we restrict $x$, and hence $x^x$ to be strictly positive
real numbers, hence it is justified to divide by these quantities.}
We begin by computing its derivative.  To do this, we write $x^x = e^{x \log x}$ and
apply the chain rule:
\[
{d \over dx} x^x =
{d \over dx} e^{x \log x} =
e^{x \log x} (1 + \log x) =
x^x (1 + \log x)
\]
Set $y = x^x$.  Then we have $y'/y = 1 + \log x$.  Taking
another derivative, we have
\[
{d \over dx} \left( {y' \over y} \right) = {1 \over x} \cdot
\]
Applying the quotient rule and simplifying, this becomes
\[
yy'' - (y')^2 - y^2/x = 0 .
\]
It is also possible to derive an equation in which $x$ does not appear.
We start by noting that, if $z = 1/x$, then $z' + z^2 = 0$.  If, as
above, $y = x^x$, we have $(d/dx) (y'/y) = z$.  Combining equations,
\[
{d^2 \over dx^2} \left( {y' \over y} \right) +
\left( {d \over dx} \left( {y' \over y} \right) \right)^2 = 0 ;
\]
applying the quotient rule and simplifying,
\[
y^3 y''' - y^2 (y'')^2 + 2 y (y')^2 y'' - 3 y^2 y' y'' - (y')^4 + 2 y (y')^3 = 0 .
\]
%%%%%
%%%%%
\end{document}
