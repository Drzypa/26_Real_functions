\documentclass[12pt]{article}
\usepackage{pmmeta}
\pmcanonicalname{ProofOfProductRule}
\pmcreated{2013-03-22 12:28:00}
\pmmodified{2013-03-22 12:28:00}
\pmowner{mathcam}{2727}
\pmmodifier{mathcam}{2727}
\pmtitle{proof of product rule}
\pmrecord{6}{32629}
\pmprivacy{1}
\pmauthor{mathcam}{2727}
\pmtype{Proof}
\pmcomment{trigger rebuild}
\pmclassification{msc}{26A24}
\pmrelated{Derivative}
\pmrelated{ProductRule}

\usepackage{amssymb}
\usepackage{amsmath}
\usepackage{amsfonts}
\newcommand{\D}[1]{\ensuremath{\mathrm{d}#1}}
\begin{document}
We begin with two differentiable functions $f(x)$ and $g(x)$ and show that their product is differentiable, and that the derivative of the product has the desired form.

By simply calculating, we have for all values of $x$ in the domain of $f$ and $g$ that 

\begin{eqnarray*}
\frac{\D{}}{\D{x}}\left[f(x)g(x)\right]
& = & \lim_{h\to0}\frac{f(x+h)g(x+h) - f(x)g(x)}{h} \\
& = & \lim_{h\to0}\frac{f(x+h)g(x+h) + f(x+h)g(x) - f(x+h)g(x) - f(x)g(x)}{h} \\
& = & \lim_{h\to0}\left[f(x+h)\frac{g(x+h)-g(x)}{h} + g(x)\frac{f(x+h)-f(x)}{h}\right] \\
& = & \lim_{h\to0}\left[f(x+h)\frac{g(x+h)-g(x)}{h}\right] + \lim_{h\to0}\left[g(x)\frac{f(x+h)-f(x)}{h}\right] \\
& = & f(x)g'(x) + f'(x)g(x).
\end{eqnarray*}

The key argument here is the next to last line, where we have used the fact that both $f$ and $g$ are differentiable, hence the limit can be distributed across the sum to give the desired equality.
%%%%%
%%%%%
\end{document}
