\documentclass[12pt]{article}
\usepackage{pmmeta}
\pmcanonicalname{LogarithmicDerivative}
\pmcreated{2013-03-22 16:47:02}
\pmmodified{2013-03-22 16:47:02}
\pmowner{rspuzio}{6075}
\pmmodifier{rspuzio}{6075}
\pmtitle{logarithmic derivative}
\pmrecord{11}{39015}
\pmprivacy{1}
\pmauthor{rspuzio}{6075}
\pmtype{Definition}
\pmcomment{trigger rebuild}
\pmclassification{msc}{26B05}
\pmclassification{msc}{46G05}
\pmclassification{msc}{26A24}
\pmrelated{ZeroesOfDerivativeOfComplexPolynomial}

\endmetadata

% this is the default PlanetMath preamble.  as your knowledge
% of TeX increases, you will probably want to edit this, but
% it should be fine as is for beginners.

% almost certainly you want these
\usepackage{amssymb}
\usepackage{amsmath}
\usepackage{amsfonts}

% used for TeXing text within eps files
%\usepackage{psfrag}
% need this for including graphics (\includegraphics)
%\usepackage{graphicx}
% for neatly defining theorems and propositions
%\usepackage{amsthm}
% making logically defined graphics
%%%\usepackage{xypic}

% there are many more packages, add them here as you need them

% define commands here

\begin{document}
Given a function $f$, the quantity $f'/f$ is known as the
\emph{logarithmic derivative} of $f$.  This name comes 
from the observation that, on account of the chain 
rule, 
\[
(\log f)' = f' \log' (f) = f'/f.
\]

The logarithmic derivative has several basic properties
which make it useful in various contexts.

The logarithmic derivative of the product of
functions is the sum of their logarithmic
derivatives.  This follows from the product rule:
$$
{(fg)' \over fg} =
{fg' + f'g \over fg} = 
{f' \over f} + {g' \over g}
$$

The logarithmic derivative of the quotient of
functions is the difference of their logarithmic
derivatives.  This follows from the quotient rule:
\[
{(f/g)' \over f/g} =
{f'g - fg' \over g^2} {g \over f} =
{f' \over f} - {g' \over g}
\]

The logarithmic derivative of the $p$-th power
of a function is $p$ times the logarithmic
derivative of the function.  This follows
from the power rule:
\[
{(f^p)' \over f^p} =
{p f^{p-1} f' \over f^p} =
p \, {f' \over f}
\]

The logarithmic derivative of the exponential 
of a function equals the derivative of a 
function.  This follows from the chain rule:
\[
{\left( e^f \right)' \over e^f} =
{e^f \, f' \over e^f} = f'
\]

Using these identities, it is rather easy to 
compute the logarithmic derivatives of expressions
which are presented in factored form.  For instance,
suppose we want to compute the logarithmic derivative
of 
\[
e^{x^2} { (x-2)^3 (x-3) \over x - 1}.
\]
Using our identities, we find that its logarithic 
derivative is
\[
2x + {3 \over x-2} + {1 \over x-3} - {1 \over x-1}.
\]
%%%%%
%%%%%
\end{document}
