\documentclass[12pt]{article}
\usepackage{pmmeta}
\pmcanonicalname{KummersAccelerationMethod}
\pmcreated{2014-12-12 10:34:19}
\pmmodified{2014-12-12 10:34:19}
\pmowner{pahio}{2872}
\pmmodifier{pahio}{2872}
\pmtitle{Kummer's acceleration method}
\pmrecord{8}{42623}
\pmprivacy{1}
\pmauthor{pahio}{2872}
\pmtype{Algorithm}
\pmclassification{msc}{26A06}
\pmclassification{msc}{40A05}
\pmrelated{ValueOfTheRiemannZetaFunctionAtS2}

\endmetadata

% this is the default PlanetMath preamble.  as your knowledge
% of TeX increases, you will probably want to edit this, but
% it should be fine as is for beginners.

% almost certainly you want these
\usepackage{amssymb}
\usepackage{amsmath}
\usepackage{amsfonts}

% used for TeXing text within eps files
%\usepackage{psfrag}
% need this for including graphics (\includegraphics)
%\usepackage{graphicx}
% for neatly defining theorems and propositions
 \usepackage{amsthm}
% making logically defined graphics
%%\usepackage{xypic}

% there are many more packages, add them here as you need them

% define commands here

\theoremstyle{definition}
\newtheorem*{thmplain}{Theorem}

\begin{document}
\PMlinkescapeword{form}

There are several methods for acceleration of the convergence of a given series 
\begin{align}
\sum_{n=1}^\infty a_n \;=\; S.
\end{align}\
One of the simplest is the following one due to Kummer (1837).\\

We suppose that the terms $a_n$ of (1) are nonzero.\, Let
$$\sum_{n=1}^\infty b_n \;=\; C$$
be a series with nonzero terms and the known sum $C$.\, We use the limit
$$\lim_{n\to\infty}\frac{a_n}{b_n} \;=\; \varrho \;\neq\; 0$$
and the identity
\begin{align}
S \;=\; \varrho C+ \sum_{n=1}^\infty\left(1-\varrho\frac{b_n}{a_n}\right)a_n.
\end{align}
Thus the original series (1) has attained a new form (2) the convergence of which is faster because of 
$$\lim_{n\to\infty}\left(1-\varrho\frac{b_n}{a_n}\right) \;=\;0.$$\\

\textbf{Example.}\, For replacing the series 
$$\sum_{n=1}^\infty\frac{1}{n^2} \;=\; S$$
by a faster converging series we may take 
$$\sum_{n=1}^\infty\frac{1}{n(n\!+\!1)} \;=:\; C,$$
which, for its part, can be expressed as the telescoping series
$$C \;=\; \sum_{n=1}^\infty\left(\frac{1}{n}-\frac{1}{n\!+\!1}\right) \;=\; 1.$$
Now we have\, $\varrho = 1$,\, and using (2) we obtain
$$S \;=\; 1+\sum_{n=1}^\infty\frac{1}{n^2(n\!+\!1)}.$$
The convergence of this series may accelerated similarly taking e.g.
$$\sum_{n=1}^\infty\frac{1}{n(n\!+\!1)(n\!+\!2)} \;=:\; C,$$
where now\, $C = \frac{1}{4}$;\, then we get
$$S \;=\; \frac{5}{4}+2\!\sum_{n=1}^\infty\frac{1}{n^2(n\!+\!1)(n\!+\!2)}.$$
The procedure may be repeated $N$ times in all, yielding the result
$$S \;=\; \sum_{n=1}^N\frac{1}{n^2}+N!\sum_{n=1}^\infty\frac{1}{n^2(n\!+\!1)(n\!+\!2)\cdots(n\!+\!N)}.$$

As for the sum of this series, see 
\PMlinkname{Riemann zeta function at $s = 2$}{valueoftheriemannzetafunctionats2}.




\begin{thebibliography}{8}
\bibitem{SG}{\sc Pascal Sebah \& Xavier Gourdon}: \PMlinkexternal{{\it Acceleration of the convergence of series} (2002)}{http://numbers.computation.free.fr/Constants/constants.html}.
\end{thebibliography}

%%%%%
\end{document}
