\documentclass[12pt]{article}
\usepackage{pmmeta}
\pmcanonicalname{ALectureOnIntegrationByParts}
\pmcreated{2013-03-22 15:38:33}
\pmmodified{2013-03-22 15:38:33}
\pmowner{alozano}{2414}
\pmmodifier{alozano}{2414}
\pmtitle{a lecture on integration by parts}
\pmrecord{6}{37574}
\pmprivacy{1}
\pmauthor{alozano}{2414}
\pmtype{Feature}
\pmcomment{trigger rebuild}
\pmclassification{msc}{26A36}
\pmrelated{ALectureOnIntegrationBySubstitution}
\pmrelated{ALectureOnTrigonometricIntegralsAndTrigonometricSubstitution}
\pmrelated{ALectureOnThePartialFractionDecompositionMethod}
\pmrelated{ExampleOfIntegrationByPartsInvolvingAlgebraicManipulation}

\endmetadata

% this is the default PlanetMath preamble.  as your knowledge
% of TeX increases, you will probably want to edit this, but
% it should be fine as is for beginners.

% almost certainly you want these
\usepackage{amssymb}
\usepackage{amsmath}
\usepackage{amsthm}
\usepackage{amsfonts}

% used for TeXing text within eps files
%\usepackage{psfrag}
% need this for including graphics (\includegraphics)
%\usepackage{graphicx}
% for neatly defining theorems and propositions
%\usepackage{amsthm}
% making logically defined graphics
%%%\usepackage{xypic}

% there are many more packages, add them here as you need them

% define commands here

\newtheorem{thm}{Theorem}[section]
\newtheorem{conj}[thm]{Conjecture}
\newtheorem{cor}[thm]{Corollary}
\newtheorem{lem}[thm]{Lemma}
\newtheorem{prop}[thm]{Proposition}
\newtheorem{defn}[thm]{Definition}
\newtheorem{exe}{Problem}
\newtheorem*{exe1}{Problem 1}
\newtheorem*{exe2}{Problem 2}
\newtheorem*{exe3}{Problem 3}
\newtheorem*{exe4}{Problem 4}

\theoremstyle{definition}
\newtheorem{exa}[thm]{Example}

\def\notdiv{\ \mathbin{\mkern-8mu|\!\!\!\smallsetminus}}
\newcommand{\Qoft}{\mathbb{Q}(T)}  %use in linux
\newcommand{\done}{\Box} %use in linux
\newcommand{\R}{\ensuremath{\mathbb{R}}}
\newcommand{\C}{\ensuremath{\mathbb{C}}}
\newcommand{\Z}{\ensuremath{\mathbb{Z}}}
\newcommand{\Q}{\mathbb{Q}}
\newcommand{\peri}{\operatorname{Perimeter}}
\newcommand{\lc}{\lim_{x\to c}}
\newcommand{\lzero}{\lim_{x\to 0}}
\newcommand{\lhzero}{\lim_{h\to 0}}
\newcommand{\linf}{\lim_{x\to \infty}}
\begin{document}
\section*{The Method of Integration by Parts}

This method is used to find indefinite integrals that look like the result of a product rule.

\begin{itemize}
\item {\it When to use it:} Use this method when the integrand is a product of two functions (and when the method of
substitution clearly does not work).

\item {\it How to use it:} The method is based in the following formula:
$$\int U\cdot V' \ dx = U \cdot V - \int U' \cdot V \ dx $$
Suppose we want to solve $\int f(x) \ dx$:
\begin{enumerate}
\item Find functions $U$ and $V'$ such that $U\cdot V' = f(x)$. There are many possible choices.
\item $U$ should be a function {\it easy} to derive and such that the derivative of $U$ is {\it easier, less complicated}
than $U$ itself. For example good choices for $U$ are $U=x, x^2, x^3, e^x,\ln x$.
\item $V'$ should be a function which is {\it easy} to integrate, such that we can find $\int V' dx$ easily and the integral
is less complicated than $V'$ itself. Good choices
for $V'$ are $V'=e^x, \sin x, \cos x$. The functions $x, x^2, x^3 , \ln x$ are {\bf bad} choices.
\item Once the functions $U$ and $V'$ are chosen, find $U' =\frac{d}{dx}U$ and $V=\int V' \ dx$.
\item Plug in the formula.
\item Solve the new integral $\int U' \cdot V \ dx$, which if the choices of $U$ and $V'$ were good, should be easy.
\item If the new integral is hard, the choices of $U$ and $V'$ might be wrong. So repeat the choice.
\end{enumerate}
\end{itemize}

Again, the method is best explained through examples:

\begin{exa}
Find $\int xe^x \ dx$. This is a product of two functions and there is no composition visible, so we will be using
integration by parts. We need two functions $U$ and $V'$ such that $U\cdot V' = xe^x$. A good choice for $U$ is $x$ because
$U'=(x)'=1$ and the derivative is simpler than $x$ itself. A good choice for $V'=e^x$ because it is easy to integrate,
$\int e^x \ dx = e^x $ (and $U \cdot V'=xe^x$). Therefore we apply the formula:
$$\int xe^x \ dx = UV-\int U'V \ dx= xe^x-\int 1\cdot e^x \ dx = xe^x -\int e^x \ dx = xe^x - e^x + C.$$
\end{exa}

\begin{exa}
Find: $$\int x\cos (x) \ dx .$$
We choose $U=x$ and $V'=\cos(x)$. Then $U'=1$ and $V=\int V' \ dx= \sin(x)$. Apply the formula:
$$\int x\cos(x) \ dx = x \sin (x) - \int \sin (x) \ dx= x\sin(x) - (-\cos (x)) +C = x\sin (x) + \cos (x) + C.$$
\end{exa}

\begin{exa}
Find $\int \ln x \ dx$. Although this is not the product of two functions, $\ln x$ is a typical example of a function to be
integrated by parts. Here is how, let $U=\ln x$ and $V'=1$ so that $U'=1/x$ and $V=x$. Look what happens when we use the formula:
$$\int \ln x \ dx = x\ln x - \int \frac{1}{x}\cdot x \ dx = x\ln x - \int 1 \ dx = x\ln x - x + C.$$
\end{exa}

\eject
\begin{exa}
In some cases we have to use the method of integration by parts twice to get an answer. For example find $\int x^2 \sin x \ dx$.
We put $U=x^2$ and $V'=\sin x$, and so $U'=2x$ and $V=\int \sin x \ dx=-\cos x$. Thus:
$$\int x^2 \sin x \ dx = -x^2 \cos x +\int 2x \cos x \ dx = -x^2 \cos x + 2\int x \cos x \ dx$$
and above, we have seen that in order to find $\int x \cos x \ dx$ we use integration by parts. Therefore the final anwser is:
$$\int x^2 \sin x \ dx = -x^2 \cos x + 2(x \sin x + \cos x) + C.$$
\end{exa}

\begin{exa}
Finally, in some other cases, after we do integration by parts twice, we get to the same integral we wanted to solve.
Although it would seem we are stuck, no no! we will be able to find a solution right away.

Find $\int e^x \sin x \ dx$. Let $I=\int e^x \sin x \ dx$. We start with integration by parts, taking
$$U=e^x, V'=\sin x, U'=e^x, V=-\cos x$$
Thus:
$$I=-e^x\cos x + \int e^x \cos x \ dx $$
Hmmm...the integral $\int e^x \cos x \ dx$ looks like the one we started with...we use integration by parts to solve this one.
Take $U=e^x$ and $V'=\cos x $ so $U'=e^x$ and $V=\sin x$. Thus:
$$\int e^x \cos x \ dx = e^x \sin x - \int e^x \sin x \ dx. $$
Therefore:
$$I=-e^x \cos x + ( e^x \sin x - \int e^x \sin x \ dx )= -e^x \cos x + e^x \sin x  - I$$
Remember that we want to find $I$, so if we solve for $I$ above, we obtain:
$$2I=-e^x \cos x + e^x \sin x$$
and so $I=1/2 ( -e^x \cos x + e^x \sin x) + C$.
\end{exa}
%%%%%
%%%%%
\end{document}
