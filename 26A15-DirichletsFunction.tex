\documentclass[12pt]{article}
\usepackage{pmmeta}
\pmcanonicalname{DirichletsFunction}
\pmcreated{2013-03-22 13:11:14}
\pmmodified{2013-03-22 13:11:14}
\pmowner{mathcam}{2727}
\pmmodifier{mathcam}{2727}
\pmtitle{Dirichlet's function}
\pmrecord{9}{33639}
\pmprivacy{1}
\pmauthor{mathcam}{2727}
\pmtype{Definition}
\pmcomment{trigger rebuild}
\pmclassification{msc}{26A15}
\pmrelated{FunctionContinuousAtOnlyOnePoint}
\pmrelated{APathologicalFunctionOfRiemann}

\endmetadata

% this is the default PlanetMath preamble.  as your knowledge
% of TeX increases, you will probably want to edit this, but
% it should be fine as is for beginners.

% almost certainly you want these
\usepackage{amssymb}
\usepackage{amsmath}
\usepackage{amsfonts}

% used for TeXing text within eps files
%\usepackage{psfrag}
% need this for including graphics (\includegraphics)
%\usepackage{graphicx}
% for neatly defining theorems and propositions
%\usepackage{amsthm}
% making logically defined graphics
%%%\usepackage{xypic}

% there are many more packages, add them here as you need them

% define commands here
\begin{document}
\emph{Dirichlet's function} $f:\mathbb{R}\to\mathbb{R}$ is defined as
\begin{displaymath}
  f\left(x\right) =
  \left\{ 
  \begin{array}{ll} 
    \frac{1}{q}  &  \textrm{if } x=\frac{p}{q}
                        \textrm{ is a rational number in lowest terms,} \\
    0              &  \textrm{if } x \textrm{ is an irrational number.} 
  \end{array}
  \right.
\end{displaymath}
This function has the property that it is continuous at every
irrational number and discontinuous at every rational one.

Another function that often goes by the same name is the function
\begin{displaymath}
  f\left(x\right) =
  \left\{ 
  \begin{array}{ll} 
    1  &  \textrm{if } x \textrm{ is an rational number.}\\
    0              &  \textrm{if } x \textrm{ is an irrational number.} 
  \end{array}
  \right.
\end{displaymath}
This nowhere-continuous function has the surprising \PMlinkescapetext{analytic} expression
\begin{align*}
f(x) = \lim_{m \to \infty} \lim_{n \to \infty} \cos^{2 n} (m! \pi x).
\end{align*}

This is often given as the (amazing!) example of a sequence of everywhere-continuous functions whose limit function is nowhere continuous.
%%%%%
%%%%%
\end{document}
