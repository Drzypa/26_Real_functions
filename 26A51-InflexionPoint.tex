\documentclass[12pt]{article}
\usepackage{pmmeta}
\pmcanonicalname{InflexionPoint}
\pmcreated{2013-03-26 15:57:49}
\pmmodified{2013-03-26 15:57:49}
\pmowner{pahio}{2872}
\pmmodifier{pahio}{2872}
\pmtitle{inflexion point}
\pmrecord{21}{39975}
\pmprivacy{1}
\pmauthor{pahio}{2872}
\pmtype{Definition}
\pmcomment{trigger rebuild}
\pmclassification{msc}{26A51}
\pmclassification{msc}{53A04}
\pmsynonym{inflection point}{InflexionPoint}
\pmsynonym{point of inflection}{InflexionPoint}
%\pmkeywords{second derivative}
%\pmkeywords{curvature}
\pmrelated{DerivativesOfSolutionOfFirstOrderODE}
\pmdefines{concave upwards}
\pmdefines{concave up}
\pmdefines{concave downwards}
\pmdefines{concave down}
\pmdefines{saddle-point}
\pmdefines{plain point (?)}

\endmetadata

% this is the default PlanetMath preamble.  as your knowledge
% of TeX increases, you will probably want to edit this, but
% it should be fine as is for beginners.

% almost certainly you want these
\usepackage{amssymb}
\usepackage{amsmath}
\usepackage{amsfonts}

% used for TeXing text within eps files
%\usepackage{psfrag}
% need this for including graphics (\includegraphics)
%\usepackage{graphicx}
% for neatly defining theorems and propositions
 \usepackage{amsthm}
% making logically defined graphics
%%%\usepackage{xypic}
\usepackage{pstricks}
\usepackage{pst-plot}

% there are many more packages, add them here as you need them

% define commands here

\theoremstyle{definition}
\newtheorem*{thmplain}{Theorem}

\begin{document}
In examining the graphs of differentiable real functions, it may be useful to \PMlinkescapetext{state} the intervals where the function is \PMlinkescapetext{convex} and the ones where it is \PMlinkescapetext{concave}.  

\begin{itemize}

\item A function $f$ is said to be \PMlinkescapetext{{\em convex} on an interval} if the \PMlinkname{restriction}{RestrictionOfAFunction} of $f$ to this interval is a (strictly) convex function; this may be characterized more illustratively by saying that the graph of $f$ is \PMlinkescapetext{{\em concave upwards}} or \PMlinkescapetext{\emph{concave up}}.  On such an interval, the tangent line of the graph is constantly turning counterclockwise, \PMlinkname{i.e.}{Ie}, the derivative $f'$ is increasing and thus the second derivative $f''$ is positive.  In the picture below, the sine curve is concave up on the interval\, $(-\pi,\,0)$.

\item The {\em concavity} of the function $f$ on an interval correspondingly:  On such an interval, the graph of $f$ is \PMlinkescapetext{{\em concave downwards}} or \PMlinkescapetext{\emph{concave down}}, the tangent line turns clockwise, $f'$ decreases, and $f''$ is negative.  In the picture below, the sine curve is concave down on the interval\, $(0,\,\pi)$.

\item The points in which a function changes from \PMlinkescapetext{concave to convex} or vice versa are the {\em inflexion points} (or \emph{inflection points}) of the graph of the function.  At an inflexion point, the tangent line crosses the curve, the second derivative vanishes and changes its sign when one passes through the point.

\item A graph may have points where the second derivative vanishes but does not change its sign when passed such a point; thus the first derivative is here changing ``very slowly''.\, Such a point is sometimes called an {\it undulation point}.\, The graph of\, $x\mapsto x^4$\, has the origin as an undulation point.
\end{itemize}

\begin{center}
\begin{pspicture}(-5,-2.5)(5,2)
\psaxes[Dx=9,Dy=1]{->}(0,0)(-4.5,-1.5)(5,2)
\rput(5,-0.2){$x$}
\rput(0.2,2){$y$}
\rput(3,-0.2){$\pi$}
\rput(-3.1,-0.2){$-\pi$}
\psplot[linecolor=blue]{-4}{4}{x 60 mul sin}
\psdot[linecolor=red](0,0)
\rput(0.2,-2.3){The origin is an inflexion point of the sinusoid \,$y = \sin{x}$.}
\end{pspicture}
\end{center}

Since the sine function is $2\pi$-periodic, the sinusoid possesses infinitely many inflexion points.  Indeed,\, 
$f(x) = \sin x$;\, $f''(x) = -\sin x = 0$\, for\, $x = 0,\,\pm\pi,\,\pm2\pi,\,\dots$;\, $f'''(x) = -\cos x$, $f'''(n\pi) = -\cos n\pi = (-1)^{n+1} \neq 0$.  Non-nullity of the third derivative at these critical points assures us the existence of those inflexion points.

\textbf{Remarks}

1.  For finding the inflexion points of the graph of $f$ it does not suffice to find the \PMlinkname{roots}{Equation} of the equation\, $f''(x) = 0$, since the sign of $f''$ does not necessarily change as one passes such a \PMlinkescapetext{root}.  If the second derivative maintains its sign when one of its zeros is passed, we can speak of a {\em plain point} (?) of the graph.  E.g. the origin is a plain point of the graph of\, $x\mapsto x^4$.

2.  Recalling that the \PMlinkname{curvature}{CurvaturePlaneCurve} $\kappa$ for a plane curve \,$y = f(x)$\, is given by
$$\kappa(x) \;=\; \frac{f''(x)}{[1+f'(x)^2]^{3/2}},$$
we can say that the inflexion points are the points of the curve where the curvature changes its sign and where the curvature equals zero.

3.  If an inflexion point\, $x = \xi$\, satisfies the additional condition \,$f'(\xi) = 0$,\, the point is said to be a \PMlinkescapetext{{\em stationary inflexion point}} or a {\em saddle-point}, while in the case\, $f'(\xi) \neq 0$\, it is a {\em non-stationary inflexion point}.
%%%%%
\end{document}
