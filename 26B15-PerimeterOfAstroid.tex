\documentclass[12pt]{article}
\usepackage{pmmeta}
\pmcanonicalname{PerimeterOfAstroid}
\pmcreated{2013-03-22 17:14:06}
\pmmodified{2013-03-22 17:14:06}
\pmowner{pahio}{2872}
\pmmodifier{pahio}{2872}
\pmtitle{perimeter of astroid}
\pmrecord{7}{39564}
\pmprivacy{1}
\pmauthor{pahio}{2872}
\pmtype{Example}
\pmcomment{trigger rebuild}
\pmclassification{msc}{26B15}
\pmsynonym{perimetre of astroid}{PerimeterOfAstroid}
\pmrelated{Parameter}
\pmrelated{ClairautsEquation}
\pmrelated{SubstitutionNotation}

% this is the default PlanetMath preamble.  as your knowledge
% of TeX increases, you will probably want to edit this, but
% it should be fine as is for beginners.

% almost certainly you want these
\usepackage{amssymb}
\usepackage{amsmath}
\usepackage{amsfonts}

% used for TeXing text within eps files
%\usepackage{psfrag}
% need this for including graphics (\includegraphics)
%\usepackage{graphicx}
% for neatly defining theorems and propositions
%\usepackage{amsthm}
% making logically defined graphics
%%%\usepackage{xypic}

% there are many more packages, add them here as you need them

% define commands here
\newcommand{\sijoitus}[2]%
{\operatornamewithlimits{\Big/}_{\!\!\!#1}^{\,#2}}
\begin{document}
The {\em astroid}
$$\sqrt[3]{x^2}+\sqrt[3]{y^2} = \sqrt[3]{a^2}$$
can be presented in the parametric form
$$x = a\cos^3\varphi,\quad y = a\sin^3\varphi,$$
where the polar angle $\varphi$ gets the values from $0$ to $2\pi$.  The curve consists of four congruent arcs, one of which is obtained letting\, $0\leqq \varphi \leqq \frac{\pi}{2}$.  Thus the whole perimeter of the astroid is
$$s = 
4\int_0^{\frac{\pi}{2}}\!\sqrt{\left(\frac{dx}{d\varphi}\right)^2+\left(\frac{dy}{d\varphi}\right)^2}\,d\varphi.$$
This expression is easily simplified to
$$s = 12a\int_0^{\frac{\pi}{2}}\sin\varphi\cos\varphi\,d\varphi$$
giving the result 
$$s = 6a\int_0^{\frac{\pi}{2}}\sin{2\varphi}\,d\varphi = 
-3a\!\sijoitus{0}{\quad \frac{\pi}{2}}\!\cos{2\varphi} = 6a.$$


%%%%%
%%%%%
\end{document}
