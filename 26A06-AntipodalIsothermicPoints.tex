\documentclass[12pt]{article}
\usepackage{pmmeta}
\pmcanonicalname{AntipodalIsothermicPoints}
\pmcreated{2013-03-22 18:32:10}
\pmmodified{2013-03-22 18:32:10}
\pmowner{pahio}{2872}
\pmmodifier{pahio}{2872}
\pmtitle{antipodal isothermic points}
\pmrecord{6}{41247}
\pmprivacy{1}
\pmauthor{pahio}{2872}
\pmtype{Application}
\pmcomment{trigger rebuild}
\pmclassification{msc}{26A06}

\endmetadata

% this is the default PlanetMath preamble.  as your knowledge
% of TeX increases, you will probably want to edit this, but
% it should be fine as is for beginners.

% almost certainly you want these
\usepackage{amssymb}
\usepackage{amsmath}
\usepackage{amsfonts}

% used for TeXing text within eps files
%\usepackage{psfrag}
% need this for including graphics (\includegraphics)
%\usepackage{graphicx}
% for neatly defining theorems and propositions
 \usepackage{amsthm}
% making logically defined graphics
%%%\usepackage{xypic}

% there are many more packages, add them here as you need them

% define commands here

\theoremstyle{definition}
\newtheorem*{thmplain}{Theorem}

\begin{document}
Assume that the momentary temperature on any great circle of a sphere varies \PMlinkname{continuously}{Continuous}.\, Then there exist two diametral points (i.e. antipodal points, end points of a certain \PMlinkname{diametre}{Diameter}) having the same temperature.\\

{\em Proof.}\, Denote by $x$ the distance of any point $P$ measured in a certain direction along the great circle from a \PMlinkescapetext{fixed point} and let $T(x)$ be the temperature in $P$.\, Then we have a continuous (and \PMlinkname{periodic}{PeriodicFunctions}) real function $T$ defined for \,$x \geqq 0$\, satisfying \,$T(x\!+\!p) = T(x)$\, where 
$p$ is the perimetre of the circle.\, Then also the function $f$ defined by
                 $$f(x) \;:=\; T\left(x\!+\!\frac{p}{2}\right)-T(x),$$
i.e. the temperature difference in two antipodic (diametral) points of the great circle, is continuous.\, We have
\begin{align}
f\left(\frac{p}{2}\right) \;=\; T(p)-T\left(\frac{p}{2}\right) = T(0)-T\left(\frac{p}{2}\right) = -f(0).
\end{align}
If $f$ happens to vanish in\, $x = 0$,\, then the temperature is the same in\, $x = \frac{p}{2}$\, and the assertion proved.\, But if\, $f(0) \neq 0$,\, then by (1), the values of $f$ in\, $x = 0$\, and in\, $x = \frac{p}{2}$\, have opposite signs.\, Therefore, by Bolzano's theorem, there exists a point $\xi$ between $0$ and $\frac{p}{2}$ such that\, $f(\xi) = 0$.\, Thus the temperatures in $\xi$ and $\xi\!+\!\frac{\pi}{2}$ are the same.\\


\textbf{Reference:}\; \PMlinkexternal{Fr{\aa}ga Lund om matematik, 6 april 2006}{http://www.maths.lth.se/query/}
%%%%%
%%%%%
\end{document}
