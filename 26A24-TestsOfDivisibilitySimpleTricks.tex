\documentclass[12pt]{article}
\usepackage{pmmeta}
\pmcanonicalname{TestsOfDivisibilitySimpleTricks}
\pmcreated{2014-08-07 17:35:51}
\pmmodified{2014-08-07 17:35:51}
\pmowner{burgess}{1001318}
\pmmodifier{burgess}{1001318}
\pmtitle{Tests of Divisibility- Simple tricks}
\pmrecord{1}{}
\pmprivacy{1}
\pmauthor{burgess}{1001318}
\pmtype{Topic}

% this is the default PlanetMath preamble.  as your knowledge
% of TeX increases, you will probably want to edit this, but
% it should be fine as is for beginners.

% almost certainly you want these
\usepackage{amssymb}
\usepackage{amsmath}
\usepackage{amsfonts}

% need this for including graphics (\includegraphics)
\usepackage{graphicx}
% for neatly defining theorems and propositions
\usepackage{amsthm}

% making logically defined graphics
%\usepackage{xypic}
% used for TeXing text within eps files
%\usepackage{psfrag}

% there are many more packages, add them here as you need them

% define commands here

\begin{document}
With this simple short cuts you can find out a number is divisible by a given number
Divisible by 2: A number is divisible by 2, if its unit’s digit is any of 0, 2, 4, 6, 8.
Example: 6798512

Divisible by 3: A number is divisible by 3, if sum of its digits divisible by 3.
Example : 123456
 1+2+3+4+5+6 = 21
21 is divisible by 3 so 123456 is also divisible by 3 

Divisible by 4: if the last two digits of a given are divisible 4, so the number can be divisible by 4.
Example : 749232
Last two digits are 32 which are divisible by 4 so the given number is also divisible by 4

Divisible by 5: If unit’s digit of a number is either ‘0’ or ‘5’ it is divisible 5.
Example : 749230

Divisible by 6: If a given number is divisible by 2 and 3 (which are factors of 6), then the number is divisible by 6.
Example : 35256
Unit’s digit is 6 so divisible by 2
3+5+2+5+6 = 21 so divisible by 3
So 35256 divisible by 6

Divisible by 8: if last 3 digits of a given number is divisible 8, then the given number is divisible 8.
Example: 953360
360 is divisible by 8, so 953360 is divisible by 8

Divisible by 9: A number is divisible by 9, if sum of its digits divisible by 9. 
Example : 50832
5+0+8+3+2 = 18 divisible by 9 so 50832 divisible by 9 

Divisible by 10: A number is divisible 10, if it ends with 0.
Example : 508320

Divisible by 11: A number is divisible by 11,if the difference of sum of its digits at odd places and sum of its digits at even places , is either 0 or a number divisible by 11.
Example : 4832718
(sum of digits at odd places ) – (sum of digits at even places)
=(8+7+3+4)-(1+2+8) = 11 which is divisible by 11.
So 4832718 is divisible by 11.

I hope this simple tricks, will be very helpful to solve math’s homework problems easily.

\end{document}
