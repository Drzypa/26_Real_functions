\documentclass[12pt]{article}
\usepackage{pmmeta}
\pmcanonicalname{RealFunction}
\pmcreated{2013-12-26 13:42:26}
\pmmodified{2013-12-26 13:42:26}
\pmowner{rspuzio}{6075}
\pmmodifier{pahio}{2872}
\pmtitle{real function}
\pmrecord{9}{36218}
\pmprivacy{1}
\pmauthor{rspuzio}{2872}
\pmtype{Topic}
\pmcomment{trigger rebuild}
\pmclassification{msc}{26A99}
\pmrelated{RationalFunction}
\pmrelated{ComplexFunction}

\endmetadata

% this is the default PlanetMath preamble.  as your knowledge
% of TeX increases, you will probably want to edit this, but
% it should be fine as is for beginners.

% almost certainly you want these
\usepackage{amssymb}
\usepackage{amsmath}
\usepackage{amsfonts}

% used for TeXing text within eps files
%\usepackage{psfrag}
% need this for including graphics (\includegraphics)
%\usepackage{graphicx}
% for neatly defining theorems and propositions
%\usepackage{amsthm}
% making logically defined graphics
%%%\usepackage{xypic}

% there are many more packages, add them here as you need them

% define commands here
\begin{document}
A {\em real function} is a function from a subset of $\mathbb{R}$ to $\mathbb{R}$.
Without imposing extra conditions, there is really not much one can say about real
functions.  Therefore, the usual procedure is to select some subclass of real 
functions according to some criterion such as differentiability or integrability
and focus attention on this subclass.  There follows a list of such criteria:

\begin{itemize}
\item elementary function
\item increasing function, decreasing function
\item positive function, negative function
\item
\end{itemize}
%%%%%
%%%%%
\end{document}
