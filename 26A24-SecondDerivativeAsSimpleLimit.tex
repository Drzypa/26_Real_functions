\documentclass[12pt]{article}
\usepackage{pmmeta}
\pmcanonicalname{SecondDerivativeAsSimpleLimit}
\pmcreated{2013-03-22 19:00:00}
\pmmodified{2013-03-22 19:00:00}
\pmowner{pahio}{2872}
\pmmodifier{pahio}{2872}
\pmtitle{second derivative as simple limit}
\pmrecord{9}{41868}
\pmprivacy{1}
\pmauthor{pahio}{2872}
\pmtype{Result}
\pmcomment{trigger rebuild}
\pmclassification{msc}{26A24}
\pmsynonym{second derivative as limit}{SecondDerivativeAsSimpleLimit}
%\pmkeywords{second derivative}
\pmrelated{DifferenceQuotient}
\pmrelated{ImproperLimits}

\endmetadata

% this is the default PlanetMath preamble.  as your knowledge
% of TeX increases, you will probably want to edit this, but
% it should be fine as is for beginners.

% almost certainly you want these
\usepackage{amssymb}
\usepackage{amsmath}
\usepackage{amsfonts}

% used for TeXing text within eps files
%\usepackage{psfrag}
% need this for including graphics (\includegraphics)
%\usepackage{graphicx}
% for neatly defining theorems and propositions
 \usepackage{amsthm}
% making logically defined graphics
%%%\usepackage{xypic}

% there are many more packages, add them here as you need them

% define commands here

\theoremstyle{definition}
\newtheorem*{thmplain}{Theorem}

\begin{document}
If the real function $f$ is  twice differentiable  in a neighbourhood of\, $x = x_0$,\, then
\begin{align}
f''(x_0) \;=\; \lim_{h \to 0}\frac{f(x_0\!+\!2h)-2f(x_0\!+\!h)+f(x_0)}{h^2}.
\end{align}


\emph{Proof.}\, The right hand side of the asserted equation is of the indeterminate form $\frac{0}{0}$.\, Using 
\PMlinkid{l'H\^opital's rule}{2657}, we obtain
\begin{align*}
\lim_{h \to 0}\frac{f(x_0\!+\!2h)-2f(x_0\!+\!h)+f(x_0)}{h^2} &\;=\; 
\lim_{h \to 0}\frac{f'(x_0\!+\!2h)\cdot2-2f'(x_0\!+\!h)}{2h}-\frac{f'(x_0)}{h}+\frac{f'(x_0)}{h}\\
&\;=\;2\lim_{2h \to 0}\frac{f'(x_0\!+\!2h)-f'(x_0)}{2h}-\lim_{h \to 0}\frac{f'(x_0\!+\!h)-f'(x_0)}{h} \\
&\;=\; 2f''(x_0)-f''(x_0) \\
&\;=\; f''(x_0).
\end{align*}
%%%%%
%%%%%
\end{document}
