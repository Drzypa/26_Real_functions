\documentclass[12pt]{article}
\usepackage{pmmeta}
\pmcanonicalname{FundamentalTheoremOfCalculus}
\pmcreated{2013-03-22 14:13:27}
\pmmodified{2013-03-22 14:13:27}
\pmowner{paolini}{1187}
\pmmodifier{paolini}{1187}
\pmtitle{fundamental theorem of calculus}
\pmrecord{13}{35660}
\pmprivacy{1}
\pmauthor{paolini}{1187}
\pmtype{Theorem}
\pmcomment{trigger rebuild}
\pmclassification{msc}{26A42}
\pmsynonym{Newton-Leibniz}{FundamentalTheoremOfCalculus}
\pmsynonym{Barrow's rule}{FundamentalTheoremOfCalculus}
\pmsynonym{Barrow's formula}{FundamentalTheoremOfCalculus}
\pmrelated{FundamentalTheoremOfCalculus}
\pmrelated{FundamentalTheoremOfCalculusForKurzweilHenstockIntegral}
\pmrelated{FundamentalTheoremOfCalculusForRiemannIntegration}
\pmrelated{LaplaceTransformOfFracftt}
\pmrelated{LimitsOfNaturalLogarithm}
\pmrelated{FundamentalTheoremOfIntegralCalculus}

\endmetadata

% this is the default PlanetMath preamble.  as your knowledge
% of TeX increases, you will probably want to edit this, but
% it should be fine as is for beginners.

% almost certainly you want these
\usepackage{amssymb}
\usepackage{amsmath}
\usepackage{amsfonts}

% used for TeXing text within eps files
%\usepackage{psfrag}
% need this for including graphics (\includegraphics)
%\usepackage{graphicx}
% for neatly defining theorems and propositions
%\usepackage{amsthm}
% making logically defined graphics
%%%\usepackage{xypic}

% there are many more packages, add them here as you need them

% define commands here
\begin{document}
Let $f\colon[a,b]\to \mathbf R$ be a continuous function, let $c\in[a,b]$ be given 
and consider the integral function $F$ defined on $[a,b]$ as
\[
  F(x)= \int_c^x f(t)\, dt.
\]

Then $F$ is an antiderivative of $f$ that is, 
$F$ is differentiable in $[a,b]$ and
\[
  F'(x)=f(x)\qquad \forall x\in [a,b].
\]


The previous relation rewritten as
\[
   \frac{d}{dx} \int_c^x f(t)\, dt = f(x)  
\]
shows that the differentiation operator $\frac{d}{dx}$ is the inverse of the integration operator $\int_c^x$. This formula is sometimes called Newton-Leibniz formula.

On the other hand if $f\colon[a,b]\to \mathbf R$ is a continuous function 
and $G\colon[a,b]\to \mathbf R$ is any antiderivative of $f$, i.e.\ $G'(x)=f(x)$ for all $x\in[a,b]$, then
\begin{equation}\label{eq:barrow}
  \int_a^b f(t) \, dt = G(b)-G(a). 
\end{equation}

This shows that up to a constant, the integration operator is the inverse of the derivative operator:
\[
  \int_a^x D G = G - G(a).
\]

\section*{Notes}
Equation~\eqref{eq:barrow} is sometimes called ``Barrow's rule'' or ``Barrow's formula''.
%%%%%
%%%%%
\end{document}
