\documentclass[12pt]{article}
\usepackage{pmmeta}
\pmcanonicalname{SincIsNotL1}
\pmcreated{2013-03-22 15:44:32}
\pmmodified{2013-03-22 15:44:32}
\pmowner{cvalente}{11260}
\pmmodifier{cvalente}{11260}
\pmtitle{sinc is not $L^1$}
\pmrecord{14}{37693}
\pmprivacy{1}
\pmauthor{cvalente}{11260}
\pmtype{Result}
\pmcomment{trigger rebuild}
\pmclassification{msc}{26A06}

% this is the default PlanetMath preamble.  as your knowledge
% of TeX increases, you will probably want to edit this, but
% it should be fine as is for beginners.

% almost certainly you want these
\usepackage{amssymb}
\usepackage{amsmath}
\usepackage{amsfonts}

% used for TeXing text within eps files
%\usepackage{psfrag}
% need this for including graphics (\includegraphics)
%\usepackage{graphicx}
% for neatly defining theorems and propositions
%\usepackage{amsthm}
% making logically defined graphics
%%%\usepackage{xypic}

% there are many more packages, add them here as you need them

% define commands here
\begin{document}
The main results used in the proof will be that $f \in L^1(A) \iff |f|\in L^1(A)$ and the dominated convergence theorem.

Proof by contradiction:

Let $f(x) = |\operatorname{sinc}(x)|$ and suppose it's Lebesgue integrable in $\mathbb{R}^+$.


Consider the intervals $I_k = [k\pi, (k+1)\pi]$ and $U_k = \bigcup_{i=0}^k I_k = [0, (k+1)\pi]$.

and the succession of functions $f_n(x) = f(x)\chi_{U_n}(x)$, where $\chi_{U_n}$ is the characteristic function of the set $U_n$.

Each $f_n$ is a continuous function of compact support and will thus be integrable in $\mathbb{R}^+$. Furthermore $f_n(x) \nearrow f(x)$ (pointwise)

in each $I_k$, $f(x)\ge \frac{|\sin(x)|}{(k+1)\pi}$.

So

$ \displaystyle \int_{\mathbb{R}^+} f_n = \sum_{k=0}^{n} \int_{k\pi}^{(k+1)\pi} \frac{|\sin(x)|}{x} dx \ge \sum_{k=0}^{n} \int_{k\pi}^{(k+1)\pi} \frac{|sin(x)|}{(k+1)\pi} = \sum_{k=0}^{n} \frac{2}{(k+1)\pi}$.

Suppose $f$ is integrable in $\mathbb{R}^+$. Then by the dominated convergence theorem $\lim_{n \to \infty} \int_{\mathbb{R}^+} f_n = \int_{\mathbb{R}^+} f$.

But $\lim_{n \to \infty} \int_{\mathbb{R}^+} f_n \ge \lim_{n \to \infty}  \sum_{k=0}^{n} \frac{2}{(k+1)\pi} = +\infty$ and we get the contradiction $\int_{\mathbb{R}^+} f \ge +\infty$.

So $f$ cannot be integrable in $\mathbb{R}^+$.
This implies that $f$ cannot be integrable in $\mathbb{R}$ and since a function is integrable in a set iff its absolute value is

$\operatorname{sinc}(x) \notin L^1(\mathbb{R})$
%%%%%
%%%%%
\end{document}
