\documentclass[12pt]{article}
\usepackage{pmmeta}
\pmcanonicalname{ProofOfImplicitFunctionTheorem}
\pmcreated{2013-03-22 13:31:23}
\pmmodified{2013-03-22 13:31:23}
\pmowner{paolini}{1187}
\pmmodifier{paolini}{1187}
\pmtitle{proof of implicit function theorem}
\pmrecord{8}{34113}
\pmprivacy{1}
\pmauthor{paolini}{1187}
\pmtype{Proof}
\pmcomment{trigger rebuild}
\pmclassification{msc}{26B10}

\endmetadata

% this is the default PlanetMath preamble.  as your knowledge
% of TeX increases, you will probably want to edit this, but
% it should be fine as is for beginners.

% almost certainly you want these
\usepackage{amssymb}
\usepackage{amsmath}
\usepackage{amsfonts}
\usepackage{amsthm}

% used for TeXing text within eps files
%\usepackage{psfrag}
% need this for including graphics (\includegraphics)
%\usepackage{graphicx}
% for neatly defining theorems and propositions
%\usepackage{amsthm}
% making logically defined graphics
%%%\usepackage{xypic}

% there are many more packages, add them here as you need them

% define commands here
\newcommand{\R}{\mathbb R}
\newcommand{\CC}{\mathcal C}
\newtheorem{theorem}{Theorem}

\begin{document}
We state the Theorem with a different notation:
\begin{theorem}
Let $\Omega$ be an open subset of $\R^n \times \R^m$ 
and let $f\in \CC^1(\Omega,\R^m)$. Let $(x_0,y_0)\in \Omega \subset \R^n\times\R^m$.
If the matrix $D_y f(x_0,y_0)$ defined by
\[
D_y f(x_0,y_0) = \left( \frac{\partial f_j}{\partial y_k}(x_0,y_0)\right)_{j,k}
\quad j=1,\ldots,m\quad k=1,\ldots,m
\]
is invertible, then there exists a neighbourhood $U\subset \R^n$ of $x_0$,
a neighbourhood $V\subset \R^m$ of $y_0$
and a function $g \in \CC^1(U,V)$ such that
\[
  f(x,y) = f(x_0,y_0) \Leftrightarrow y=g(x) \qquad \forall (x,y) \in U\times V.
\]

Moreover
\[
  Dg(x) = - (D_y f(x,g(x))) ^ {-1} D_x f(x,g(x)).
\]
\end{theorem}

\begin{proof}

Consider the function $F\in \CC^1(\Omega, \R^n \times \R^m)$ defined by
\[
  F(x,y)=(x,f(x,y)).
\]
One finds that
\[
  DF(x,y) = 
\left(\begin{array}{c|c}
I_m & 0 \\ \hline
D_x f & D_y f \\
\end{array}\right).
\]

Being $D_y f(x_0,y_0)$ invertible, $DF(x_0,y_0)$ is invertible too. 
Applying the inverse function Theorem to $F$ 
we find that there exist a neighbourhood $U$ of $x_0$ and $V$ of $y_0$ and 
a function $G\in C^1(U\times V,\R^{n+m})$ such that $F(G(x,y))=(x,y)$ 
for all $(x,y)\in U\times V$. Letting $G(x,y)=(G_1(x,y),G_2(x,y))$ 
(so that $G_1\colon V\times W\to\R^n$, $G_2\colon V\times W\to \R^m$)
we hence have
\[
  (x,y) = F(G_1(x,y),G_2(x,y)) = (G_1(x,y), f(G_1(x,y),G_2(x,y)))
\]
and hence $x=G_1(x,y)$ and $y=f(G_1(x,y),G_2(x,y))=f(x,G_2(x,y))$.
So we only have to set $g(x)=G_2(x,f(x_0,y_0))$ to obtain
\[
  f(x,g(x)) = f(x_0,y_0),\quad \forall x\in U.
\]
Differentiating with respect to $x$ we obtain 
\[
  D_x f(x,g(x)) + D_y f(x,g(x)) Dg(x) = 0
\]
which gives the desired formula for the computation of $Dg$.
\end{proof}

%%%%%
%%%%%
\end{document}
