\documentclass[12pt]{article}
\usepackage{pmmeta}
\pmcanonicalname{HigherOrderDerivatives}
\pmcreated{2013-03-22 16:46:30}
\pmmodified{2013-03-22 16:46:30}
\pmowner{PrimeFan}{13766}
\pmmodifier{PrimeFan}{13766}
\pmtitle{higher order derivatives}
\pmrecord{7}{39004}
\pmprivacy{1}
\pmauthor{PrimeFan}{13766}
\pmtype{Definition}
\pmcomment{trigger rebuild}
\pmclassification{msc}{26B05}
\pmclassification{msc}{26A24}
\pmrelated{HigherOrderDerivativesOfSineAndCosine}
\pmrelated{TaylorSeriesOfHyperbolicFunctions}
\pmdefines{derivative function}
\pmdefines{first derivative}
\pmdefines{second derivative}
\pmdefines{order of derivative}
\pmdefines{differentiation}
\pmdefines{differentiate}
\pmdefines{twice differentiable}

\endmetadata

% this is the default PlanetMath preamble.  as your knowledge
% of TeX increases, you will probably want to edit this, but
% it should be fine as is for beginners.

% almost certainly you want these
\usepackage{amssymb}
\usepackage{amsmath}
\usepackage{amsfonts}

% used for TeXing text within eps files
%\usepackage{psfrag}
% need this for including graphics (\includegraphics)
%\usepackage{graphicx}
% for neatly defining theorems and propositions
 \usepackage{amsthm}
% making logically defined graphics
%%%\usepackage{xypic}

% there are many more packages, add them here as you need them

% define commands here

\theoremstyle{definition}
\newtheorem*{thmplain}{Theorem}

\begin{document}
\PMlinkescapeword{order} 

Let the real function $f$ be defined and differentiable on the open interval $I$.\, Then for every\, $x \in I$,\, there exists the value $f'(x)$ as a certain real number.\, This means that we have a new function
\begin{align}
x \mapsto f'(x),
\end{align}
the so-called {\em derivative function} of $f$; it is denoted by 
$$f':\, I\to \mathbb{R}$$
or simply $f'$.

Forming the derivative function of a function is called {\em differentiation}, the corresponding verb is {\em differentiate}.

If the derivative function $f'$ is differentiable on $I$, then we have again a new function, the derivative function of the derivative function of $f$, which is denoted by $f''$.\, Then $f$ is said to be \emph{twice differentiable}.\, Formally,
$$f''(x) = \lim_{h\to 0}\frac{f'(x+h)-f'(x)}{h}\quad \mathrm{for\,all\,}\,x\in I.$$
The function\, $x\mapsto f''(x)$\, is called the \PMlinkescapetext{{\em second order derivative}} or {\em the second derivative} of $f$.\, Similarly, one can call (1) the \PMlinkescapetext{{\em first (order) derivative}} of $f$.

\textbf{Example.}\, The first derivative of\, $x\mapsto x^3$\, is\, $x\mapsto 3x^2$\, and the second derivative is\, $x\mapsto 6x$,\, since
$$\frac{d}{dx}(3x^2) = 2\cdot 3x^{2-1} = 6x.$$

If also $f''$ is a differentiable function, its derivative function is denoted by $f'''$ and called the \PMlinkescapetext{{\em third (order) derivative}} of $f$, and so on.

Generally, $f$ can have the derivatives of first, second, third, \ldots, $n$th order, where $n$ may be an arbitrarily big positive integer.\, If $n$ is four or greater, the $n$th derivative of $f$ is usually denoted by $f^{(n)}$.\, In \PMlinkescapetext{addition}, it's sometimes convenient to think that the $0${\em th order derivative} $f^{(0)}$ of $f$ is the function $f$ itself.

The phrase ``$f$ is infinitely differentiable'' means that $f$ has the derivatives of all \PMlinkescapetext{orders}.
%%%%%
%%%%%
\end{document}
