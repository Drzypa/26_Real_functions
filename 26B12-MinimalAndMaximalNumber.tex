\documentclass[12pt]{article}
\usepackage{pmmeta}
\pmcanonicalname{MinimalAndMaximalNumber}
\pmcreated{2014-02-15 18:33:33}
\pmmodified{2014-02-15 18:33:33}
\pmowner{pahio}{2872}
\pmmodifier{pahio}{2872}
\pmtitle{minimal and maximal number}
\pmrecord{25}{36118}
\pmprivacy{1}
\pmauthor{pahio}{2872}
\pmtype{Definition}
\pmcomment{trigger rebuild}
\pmclassification{msc}{26B12}
\pmclassification{msc}{03E04}
\pmsynonym{least and greatest number}{MinimalAndMaximalNumber}
%\pmkeywords{least}
%\pmkeywords{greatest}
\pmrelated{Infimum}
\pmrelated{Supremum}
\pmrelated{UltrametricTriangleInequality}
\pmrelated{GrowthOfExponentialFunction}
\pmrelated{EstimatingTheoremOfContourIntegral}
\pmrelated{LeastAndGreatestValueOfFunction}
\pmrelated{FuzzyLogic2}
\pmrelated{ZerosAndPolesOfRationalFunction}
\pmrelated{UniformConvergenceOnUnionInterval}
\pmrelated{Interprime}
\pmrelated{LehmerMean}
\pmrelated{Ab}
\pmdefines{least number}
\pmdefines{greatest number}
\pmdefines{minimal number}
\pmdefines{maximal number}
\pmdefines{set function}

% this is the default PlanetMath preamble.  as your knowledge
% of TeX increases, you will probably want to edit this, but
% it should be fine as is for beginners.

% almost certainly you want these
\usepackage{amssymb}
\usepackage{amsmath}
\usepackage{amsfonts}

% used for TeXing text within eps files
%\usepackage{psfrag}
% need this for including graphics (\includegraphics)
%\usepackage{graphicx}
% for neatly defining theorems and propositions
%\usepackage{amsthm}
% making logically defined graphics
%%%\usepackage{xypic}

% there are many more packages, add them here as you need them

% define commands here
\begin{document}
Let's consider a finite non-empty set\, 
$A \,=\, \{a_1,\,\ldots,\,a_n\}$\,\, of real numbers or an infinite but compact (i.e. bounded and closed) set $A$ of real numbers.\, In both cases the set has a unique least number and a unique greatest number.
\begin{itemize}
 \item The least number of the set is denoted by\, 
 $\min\{a_1,\,\ldots,\,a_n\}$\, or\, $\min{A}$.
 \item The greatest number of the set is denoted by\, $\max\{a_1,\,\ldots,\,a_n\}$\, or\, $\max{A}$.
\end{itemize}

In both cases we have
$$\min{A} \;=\; \inf{A},$$
$$\max{A} \;=\; \sup{A},$$
$$\min{A} \;\leqq\; x \;\leqq\; \max{A} \quad \forall x\in A,$$
where\, $\inf{A}$\, and\, $\sup{A}$\, are the infimum and supremum of the set $A$.

The $\min$ and $\max$ are {\em set functions}, i.e. they map 
subsets of a certain set to $\mathbb{R}$.

The $\min$ and $\max$ have the following distributive properties 
with respect to addition:
$$\min\{a_1,\,\ldots,\,a_n\}+b 
\;=\; \min\{a_1+b,\,\ldots,\,a_n+b\}$$
$$\max\{a_1,\,\ldots,\,a_n\}+b 
\;=\; \max\{a_1+b,\,\ldots,\,a_n+b\}$$\\

The minimal and maximal number of a set of two real numbers obey 
the formulae
$$\min\{a,\,b\} \;=\; \frac{a\!+\!b}{2}\!-\!\frac{|a\!-\!b|}{2},$$
$$\max\{a,\,b\} \;=\; \frac{a\!+\!b}{2}\!+\!\frac{|a\!-\!b|}{2},$$\
$$\max\{a,\,b\}-\min\{a,\,b\} \;=\; |a\!-\!b|,$$\
$$\max\{a,\,b\}+\min\{a,\,b\} \;=\; a\!+\!b,$$\
$$\max\{a,\,-a\} \;=\; |a|$$

%%%%%
%%%%%
\end{document}
