\documentclass[12pt]{article}
\usepackage{pmmeta}
\pmcanonicalname{IfDxiXi112iThenXiIsACauchySequence}
\pmcreated{2013-03-22 14:37:31}
\pmmodified{2013-03-22 14:37:31}
\pmowner{matte}{1858}
\pmmodifier{matte}{1858}
\pmtitle{if $d(x_i, x_{i+1})<1/2^i$ then $x_i$ is a Cauchy sequence}
\pmrecord{5}{36205}
\pmprivacy{1}
\pmauthor{matte}{1858}
\pmtype{Result}
\pmcomment{trigger rebuild}
\pmclassification{msc}{26A03}
\pmclassification{msc}{54E35}

\endmetadata

% this is the default PlanetMath preamble.  as your knowledge
% of TeX increases, you will probably want to edit this, but
% it should be fine as is for beginners.

% almost certainly you want these
\usepackage{amssymb}
\usepackage{amsmath}
\usepackage{amsfonts}
\usepackage{amsthm}

% used for TeXing text within eps files
%\usepackage{psfrag}
% need this for including graphics (\includegraphics)
%\usepackage{graphicx}
% for neatly defining theorems and propositions
%
% making logically defined graphics
%%%\usepackage{xypic}

% there are many more packages, add them here as you need them

% define commands here

\newcommand{\sR}[0]{\mathbb{R}}
\newcommand{\sC}[0]{\mathbb{C}}
\newcommand{\sN}[0]{\mathbb{N}}
\newcommand{\sZ}[0]{\mathbb{Z}}

 \usepackage{bbm}
 \newcommand{\Z}{\mathbbmss{Z}}
 \newcommand{\C}{\mathbbmss{C}}
 \newcommand{\R}{\mathbbmss{R}}
 \newcommand{\Q}{\mathbbmss{Q}}



\newcommand*{\norm}[1]{\lVert #1 \rVert}
\newcommand*{\abs}[1]{| #1 |}



\newtheorem{thm}{Theorem}
\newtheorem{defn}{Definition}
\newtheorem{prop}{Proposition}
\newtheorem{lemma}{Lemma}
\newtheorem{cor}{Corollary}
\begin{document}
\begin{lemma}
Suppose $x_1, x_2, \ldots$, is a sequence in a metric space. 
If for some $N\ge 1$, we have $d(a_i,a_{i+1})<1/2^i$ for all $i\ge N$, 
then $\{x_i\}$ is a Cauchy sequence. 
\end{lemma}

\begin{proof} 
Let us denote by $d$ the metric function. 
If $\varepsilon>0$, then for some $N\in \sN$ we have $1/2^N<\varepsilon$.
Thus, if $N<m<n$ we have
\begin{eqnarray*}
d(x_m, x_n) &\le& d(x_m, x_{m+1}) + \cdots + d(x_{n-1}, x_n) \\
   &=& \left(\frac{1}{2}\right)^m + \cdots + \left(\frac{1}{2}\right)^{n-1} \\
   &=&  \left(\frac{1}{2}\right)^{m-1} \ \sum_{i=1}^{n-m} \left(\frac{1}{2}\right)^i \\
   &=&  \left(\frac{1}{2}\right)^{m-1} \ \frac{1-\left(\frac{1}{2}\right)^{n-m}}{1-\frac{1}{2}} \\
   &<&  \left(\frac{1}{2}\right)^{m} \\
   &<&  \left(\frac{1}{2}\right)^{N} \\
   &<&  \varepsilon,
\end{eqnarray*}
where we have used the triangle inequality and the 
\PMlinkname{geometric sum formula}{GeometricSeries}.
\end{proof}
%%%%%
%%%%%
\end{document}
