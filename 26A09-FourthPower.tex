\documentclass[12pt]{article}
\usepackage{pmmeta}
\pmcanonicalname{FourthPower}
\pmcreated{2013-03-22 18:25:16}
\pmmodified{2013-03-22 18:25:16}
\pmowner{CompositeFan}{12809}
\pmmodifier{CompositeFan}{12809}
\pmtitle{fourth power}
\pmrecord{5}{41072}
\pmprivacy{1}
\pmauthor{CompositeFan}{12809}
\pmtype{Definition}
\pmcomment{trigger rebuild}
\pmclassification{msc}{26A09}
\pmclassification{msc}{11A05}
\pmsynonym{biquadratic number}{FourthPower}

\endmetadata

% this is the default PlanetMath preamble.  as your knowledge
% of TeX increases, you will probably want to edit this, but
% it should be fine as is for beginners.

% almost certainly you want these
\usepackage{amssymb}
\usepackage{amsmath}
\usepackage{amsfonts}

% used for TeXing text within eps files
%\usepackage{psfrag}
% need this for including graphics (\includegraphics)
%\usepackage{graphicx}
% for neatly defining theorems and propositions
%\usepackage{amsthm}
% making logically defined graphics
%%%\usepackage{xypic}

% there are many more packages, add them here as you need them

% define commands here

\begin{document}
The {\em fourth power} of a number $x$ is the number obtained multiplying $x$ by itself three times thus: $x \times x \times x \times x$. It's more commonly denoted as $x^4$. For example, $2^4 = 2 \times 2 \times 2 \times 2 = 16$. Since the square of a square number is a fourth power, $x^2 x^2 = x^{2 + 2} = x^4$, fourth powers are sometimes called {\em biquadratic numbers}. For example, $2^4 = 2^2 2^2 = 4^2 = 16$. The first few integer fourth powers are 1, 16, 81, 256, 625, 1296, 2401, etc., listed in A000290 of Sloane's OEIS.

Any integer can be represented by the sum of at most 19 integer fourth powers (see Waring's problem).

Euler's conjecture was first disproven with fifth powers, but there are also counterexamples using fourth powers. Sloane's A003828 lists the known integers $n$ having solutions to $n^4 = a^4 + b^4 + c^4$.

The fourth power of a negative number is always a positive number; the fourth root of a negative real number is a complex number $a + bi$ with $|a| = |b|$ and $a \neq 0$.
%%%%%
%%%%%
\end{document}
