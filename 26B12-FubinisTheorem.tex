\documentclass[12pt]{article}
\usepackage{pmmeta}
\pmcanonicalname{FubinisTheorem}
\pmcreated{2013-03-22 13:39:13}
\pmmodified{2013-03-22 13:39:13}
\pmowner{mathcam}{2727}
\pmmodifier{mathcam}{2727}
\pmtitle{Fubini's theorem}
\pmrecord{11}{34307}
\pmprivacy{1}
\pmauthor{mathcam}{2727}
\pmtype{Theorem}
\pmcomment{trigger rebuild}
\pmclassification{msc}{26B12}
\pmrelated{TonellisTheorem}
\pmrelated{FubinisTheoremForTheLebesgueIntegral}
\pmrelated{IntegrationUnderIntegralSign}

\endmetadata

% this is the default PlanetMath preamble.  as your knowledge
% of TeX increases, you will probably want to edit this, but
% it should be fine as is for beginners.

% almost certainly you want these
\usepackage{amssymb}
\usepackage{amsmath}
\usepackage{amsfonts}

% used for TeXing text within eps files
%\usepackage{psfrag}
% need this for including graphics (\includegraphics)
%\usepackage{graphicx}
% for neatly defining theorems and propositions
%\usepackage{amsthm}
% making logically defined graphics
%%%\usepackage{xypic}

% there are many more packages, add them here as you need them

% define commands here

\newcommand{\R}{\mathbb{R}}
\begin{document}
\PMlinkescapeword{states}
\PMlinkescapeword{order}
\PMlinkescapeword{time}

\textbf{Fubini's theorem}
Let $I \subset \R^N$ and $J \subset \R^M$ be compact intervals, and let $f : I \times J \to \R^K$ be a Riemann integrable function such that, for each $x \in I$ the integral
\[
F(x) := \int_J f(x, y)\, d\mu_J(y)
\]
exists. Then $F:I\to\R^K$ is Riemann integrable, and
\[
\int_I F = \int_{I\times J} f.
\]

This theorem effectively states that, given a function of $N$ variables, you may integrate it one variable at a time, and that the order of integration does not affect the result.

\textbf{Example} Let $I := [0, \pi/2]\times[0,\pi/2]$, and let $f : I \to \R, x \mapsto \sin(x)\cos(y)$ be a function.
Then
\begin{equation*}
\begin{split}
\int_I f &= \iint_{[0, \pi/2]\times[0,\pi/2]} \sin(x)\cos(y) \\
&=\int_0^{\pi/2} \left( \int_0^{\pi/2} \sin(x)\cos(y)\,dy\right)\,dx \\
&=\int_0^{\pi/2} \sin(x)\left(1 - 0\right)\,dx =(0 - -1) = 1.
\end{split}
\end{equation*}

Note that it is often simpler (and no less correct) to write $\idotsint_I f$ as $\int_I f$.
%%%%%
%%%%%
\end{document}
