\documentclass[12pt]{article}
\usepackage{pmmeta}
\pmcanonicalname{DiracSequence}
\pmcreated{2013-03-22 14:11:35}
\pmmodified{2013-03-22 14:11:35}
\pmowner{mathwizard}{128}
\pmmodifier{mathwizard}{128}
\pmtitle{Dirac sequence}
\pmrecord{5}{35623}
\pmprivacy{1}
\pmauthor{mathwizard}{128}
\pmtype{Definition}
\pmcomment{trigger rebuild}
\pmclassification{msc}{26A30}
\pmsynonym{delta sequence}{DiracSequence}
\pmrelated{DiracDeltaFunction}
\pmrelated{FejerKernel}

% this is the default PlanetMath preamble.  as your knowledge
% of TeX increases, you will probably want to edit this, but
% it should be fine as is for beginners.

% almost certainly you want these
\usepackage{amssymb}
\usepackage{amsmath}
\usepackage{amsfonts}

% used for TeXing text within eps files
%\usepackage{psfrag}
% need this for including graphics (\includegraphics)
%\usepackage{graphicx}
% for neatly defining theorems and propositions
%\usepackage{amsthm}
% making logically defined graphics
%%%\usepackage{xypic}

% there are many more packages, add them here as you need them

% define commands here
\begin{document}
A \emph{Dirac sequence} is a sequence $(\delta_k)$ of functions $\delta_k$, which satisfies the following conditions:
\begin{enumerate}
\item $\delta_k\geq0$ for all $k$.\\
\item $\int_{-\infty}^\infty\delta_k(t)dt=1$ for all $k$.\\
\item For every $r>0$ and $\varepsilon>0$ there is an $N\in\mathbb{N}$, such that for all $k>N$ we have
$$\int_{\mathbb{R}\backslash[-r,r]}\delta_k(t)dt<\varepsilon.$$
\end{enumerate}
These functions ``converge'' to the Dirac delta function.
%%%%%
%%%%%
\end{document}
