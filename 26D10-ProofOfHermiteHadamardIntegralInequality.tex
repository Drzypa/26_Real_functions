\documentclass[12pt]{article}
\usepackage{pmmeta}
\pmcanonicalname{ProofOfHermiteHadamardIntegralInequality}
\pmcreated{2013-03-22 16:59:22}
\pmmodified{2013-03-22 16:59:22}
\pmowner{Andrea Ambrosio}{7332}
\pmmodifier{Andrea Ambrosio}{7332}
\pmtitle{proof of Hermite-Hadamard integral inequality}
\pmrecord{7}{39269}
\pmprivacy{1}
\pmauthor{Andrea Ambrosio}{7332}
\pmtype{Proof}
\pmcomment{trigger rebuild}
\pmclassification{msc}{26D10}
\pmclassification{msc}{26D15}

% this is the default PlanetMath preamble.  as your knowledge
% of TeX increases, you will probably want to edit this, but
% it should be fine as is for beginners.

% almost certainly you want these
\usepackage{amssymb}
\usepackage{amsmath}
\usepackage{amsfonts}

% used for TeXing text within eps files
%\usepackage{psfrag}
% need this for including graphics (\includegraphics)
%\usepackage{graphicx}
% for neatly defining theorems and propositions
%\usepackage{amsthm}
% making logically defined graphics
%%%\usepackage{xypic}

% there are many more packages, add them here as you need them

% define commands here

\begin{document}
First of all, let's recall that a convex function on a open
interval $(a,b)$ is continuous on $(a,b)$ and admits left and right
derivative $f^{+}(x)$ and $f^{-}(x)$ for any $x\in (a,b)$. For this reason,
it's always possible to construct at least one \PMlinkname{supporting line}{ConvexFunctionsLieAboveTheirSupportingLines} for $f\left(
x\right) $ at any $x_{0}\in (a,b)$ : if  $f\left( x_{0}\right) $ is
differentiable in $x_{0}$, one has $r(x)=f\left( x_{0}\right) +f^{\prime
}\left( x_{0}\right) \left( x-x_{0}\right) $; if not, it's obvious that all $%
r(x)=f\left( x_{0}\right) +c\left( x-x_{0}\right) $ are supporting lines for
any $c\in \lbrack f^{-}(x_{0}),f^{+}(x_{0})]$.\\
Let now $r(x)=f\left( \frac{a+b}{2}\right) +c\left( x-\frac{a+b}{2}%
\right) $ be a supporting line of $f(x)$ in $x=\frac{a+b}{2}\in (a,b)$.
Then, $r\left( x\right) \leq f\left( x\right) $. On the other side, by
convexity definition, having defined $s\left( x\right) =f\left( a\right) +%
\frac{f(b)-f(a)}{b-a}(x-a)$ the line connecting the points $(a,f(a))$ and $%
(b,f(b))$ , one has $f(x)\leq s(x)$. Shortly,%
\[
r\left( x\right) \leq f\left( x\right) \leq s(x)
\]
Integrating both inequalities between $a$ and $b$
\[
\int_{a}^{b}r\left( x\right) dx\leq \int_{a}^{b}f\left( x\right) dx\leq
\int_{a}^{b}s(x)dx 
\]
\begin{eqnarray*}
&&\int_{a}^{b}r\left( x\right) dx \\
&=&\int_{a}^{b}\left[ f\left( \frac{a+b}{2}\right) +c\left( x-\frac{a+b}{2}%
\right) \right] dx \\
&=&f\left( \frac{a+b}{2}\right) (b-a)+c\int_{a}^{b}\left( x-\frac{a+b}{2}%
\right) dx \\
&=&f\left( \frac{a+b}{2}\right) (b-a) \\
&& \\
&&\int_{a}^{b}s(x)dx \\
&=&\int_{a}^{b}\left[ f\left( a\right) +\frac{f(b)-f(a)}{b-a}(x-a)\right] dx
\\
&=&f(a)(b-a)+\frac{f(b)-f(a)}{b-a}\int_{a}^{b}(x-a)dx \\
&=&\frac{f(a)+f(b)}{2}(b-a)
\end{eqnarray*}
and so%
\[
f\left( \frac{a+b}{2}\right) (b-a)\leq \int_{a}^{b}f\left( x\right) dx\leq 
\frac{f(a)+f(b)}{2}(b-a) 
\]
which is the thesis.
%%%%%
%%%%%
\end{document}
