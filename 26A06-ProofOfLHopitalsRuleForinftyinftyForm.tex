\documentclass[12pt]{article}
\usepackage{pmmeta}
\pmcanonicalname{ProofOfLHopitalsRuleForinftyinftyForm}
\pmcreated{2013-03-22 15:40:15}
\pmmodified{2013-03-22 15:40:15}
\pmowner{stevecheng}{10074}
\pmmodifier{stevecheng}{10074}
\pmtitle{proof of l'H\^opital's rule for $\infty/\infty$ form}
\pmrecord{7}{37611}
\pmprivacy{1}
\pmauthor{stevecheng}{10074}
\pmtype{Proof}
\pmcomment{trigger rebuild}
\pmclassification{msc}{26A06}

\endmetadata

\usepackage{amssymb}
\usepackage{amsmath}
\usepackage{amsfonts}

% used for TeXing text within eps files
%\usepackage{psfrag}
% need this for including graphics (\includegraphics)
%\usepackage{graphicx}
% making logically defined graphics
%%%\usepackage{xypic}

\providecommand{\abs}[1]{\lvert#1\rvert}
\providecommand{\absW}[1]{\left\lvert#1\right\rvert}
\providecommand{\absB}[1]{\Bigl\lvert#1\Bigr\rvert}
\begin{document}
This is the proof of \PMlinkname{L'H\^opital's Rule}{LHpitalsRule}
in the case of the indeterminate form $\pm \infty / \infty$.
Compared to \PMlinkname{the proof for the $0/0$ case}{ProofOfDeLHopitalsRule},
more complicated estimates are needed.

Assume that
\[
\lim_{x \to a} f(x) = \pm \infty \,, \quad \lim_{x \to a} g(x) = \pm \infty\,, \quad
\lim_{x \to a} \frac{f'(x)}{g'(x)} = m\,, 
\]
where $a$ and $m$ are real numbers. 
The case when $a$ or $m$ is infinite only involves
slight modifications to the arguments below.

Given $\epsilon > 0$.
there is a $\delta > 0$ such that 
\[
\absW{ \frac{f'(\xi)}{g'(\xi)}  - m} < \epsilon
\]
whenever $0 < \abs{\xi - a} < \delta$.

Let $c$ and $x$ be points such that
$a-\delta < c < x < a$ or $a < x < c < a+\delta$.
(That is, both $c$ and $x$ are within distance $\delta$ of $a$,
but $x$ is always closer.)
By Cauchy's mean value theorem, there exists some $\xi_x$ 
in between $c$ and $x$ (and hence $0 < \abs{\xi_x -a} < \delta$)
such that
\[
\frac{f(x) - f(c)}{g(x) - g(c)} = \frac{f'(\xi_x)}{g'(\xi_x)}\,.
\]
We can assume the values $f(x)$, $g(x)$, $f(x) - f(c)$, $g(x) - g(c)$
are \emph{all non-zero} when $x$ is close enough to $a$,
say, when $0 < \abs{x - a} < \delta'$ for some $0 < \delta' < \delta$.
(So there is no division by zero in our equations.)
This is because $f(x)$ and $g(x)$ were assumed to approach $\pm \infty$,
so when $x$ is close enough to $a$, 
they will exceed the fixed values $f(c)$, $g(c)$, and $0$.

We write
\begin{align*}
\frac{f(x)}{g(x)} &=
\frac{f(x)}{f(x)-f(c)} \cdot
\frac{g(x)-g(c)}{g(x)} \cdot
\frac{f(x)-f(c)}{g(x)-g(c)}  \\
&= 
\frac{1-g(c)/g(x)}{1-f(c)/f(x)}
\cdot \frac{f'(\xi_x)}{g'(\xi_x)}\,.
\end{align*}
Note that
\[
\lim_{x \to a} \frac{1-g(c)/g(x)}{1-f(c)/f(x)} = 1\,,
\]
but $\xi_x$ is not guaranteed to approach $a$ as $x$ approaches $a$,
so we cannot just take the limit $x \to a$ directly.  However:
there exists $0 < \delta'' < \delta'$ so that
\[
\absW{ \frac{1-g(c)/g(x)}{1-f(c)/f(x)}  - 1} < 
\frac{\epsilon}{\abs{m} + \epsilon}
\]
whenever $0 < \abs{x - a} < \delta''$.  
Then
\begin{align*}
\absW{\frac{f(x)}{g(x)} - m }
&= \absW{\left( \frac{f'(\xi_x)}{g'(\xi_x)} -  m\right) + \frac{f'(\xi_x)}{g'(\xi_x)}  \left( \frac{1-g(c)/g(x)}{1-f(c)/f(x) } - 1 \right) }
\\
&\leq \epsilon + (\abs{m} + \epsilon) \frac{\epsilon}{\abs{m}+\epsilon}
= 2\epsilon
\end{align*}
for $0 < \abs{x-a} < \delta''$.

This proves
\[
\lim_{x \to a} \frac{f(x)}{g(x)} = m
=
\lim_{x \to a} \frac{f'(x)}{g'(x)}\,.
\]

\begin{thebibliography}{XXXXXX}
\bibitem{Spivak}
Michael Spivak, \emph{Calculus}, 3rd ed. Publish or Perish, 1994.
\end{thebibliography}
%%%%%
%%%%%
\end{document}
