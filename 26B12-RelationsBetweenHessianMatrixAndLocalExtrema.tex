\documentclass[12pt]{article}
\usepackage{pmmeta}
\pmcanonicalname{RelationsBetweenHessianMatrixAndLocalExtrema}
\pmcreated{2013-03-22 12:59:52}
\pmmodified{2013-03-22 12:59:52}
\pmowner{bshanks}{153}
\pmmodifier{bshanks}{153}
\pmtitle{relations between Hessian matrix and local extrema}
\pmrecord{14}{33375}
\pmprivacy{1}
\pmauthor{bshanks}{153}
\pmtype{Result}
\pmcomment{trigger rebuild}
\pmclassification{msc}{26B12}
\pmrelated{Extrema}
\pmrelated{Extremum}
\pmrelated{HessianForm}
\pmrelated{TestsForLocalExtremaForLagrangeMultiplierMethod}
\pmdefines{second derivative test}

% this is the default PlanetMath preamble.  as your knowledge
% of TeX increases, you will probably want to edit this, but
% it should be fine as is for beginners.

% almost certainly you want these
\usepackage{amssymb}
\usepackage{amsmath}
\usepackage{amsfonts}

% used for TeXing text within eps files
%\usepackage{psfrag}
% need this for including graphics (\includegraphics)
%\usepackage{graphicx}
% for neatly defining theorems and propositions
%\usepackage{amsthm}
% making logically defined graphics
%%%\usepackage{xypic}

% there are many more packages, add them here as you need them

% define commands here
\begin{document}
Let $x$ be a vector, and let $H(x)$ be the Hessian for $f$ at a point $x$.  Let $f$ have continuous partial derivatives 
of first and second order in a neighborhood of $x$. Let $\nabla f (x)= 0$.

If $H(x)$ is \PMlinkname{positive definite}{PositiveDefinite}, then $x$ is a strict local minimum for $f$. 

If $x$ is a local minimum for $x$, then $H(x)$ is positive semidefinite.


If $H(x)$ is \PMlinkname{negative definite}{NegativeDefinite}, then $x$ is a strict local maximum for $f$. 

If $x$ is a local maximum for $x$, then $H(x)$ is negative semidefinite.

If $H(x)$ is indefinite, $x$ is a nondegenerate saddle point.

If the case when the dimension of $x$ is 1 (i.e. $f: \mathbb{R} \to \mathbb{R}$), this reduces to the Second Derivative Test, which is as follows:

Let the neighborhood of $x$ be in the domain for $f$, and let $f$ have continuous partial derivatives of first and second order. 
Let $f'(x) = 0$. If $f''(x) > 0$, then $x$ is a strict local minimum. If $f''(x) < 0$, then $x$ is a strict local maximum. In the case that $f''(x)=0$, being $f'''(x)\neq 0$, $x$ is said to be an inflexion point (also called turning point). A typical example is $f(x)=\sin x$, $f''(x)=-\sin x=0$, $x=n\pi$, $n=0, \pm 1, \pm 2, \dots$, 
$f'''(x)=-\cos x$, $f'''(n\pi)=-\cos n\pi=(-1)^{n+1}\neq 0$.
%%%%%
%%%%%
\end{document}
