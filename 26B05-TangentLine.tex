\documentclass[12pt]{article}
\usepackage{pmmeta}
\pmcanonicalname{TangentLine}
\pmcreated{2013-03-22 14:50:31}
\pmmodified{2013-03-22 14:50:31}
\pmowner{Mathprof}{13753}
\pmmodifier{Mathprof}{13753}
\pmtitle{tangent line}
\pmrecord{12}{36511}
\pmprivacy{1}
\pmauthor{Mathprof}{13753}
\pmtype{Definition}
\pmcomment{trigger rebuild}
\pmclassification{msc}{26B05}
\pmclassification{msc}{26A24}
\pmsynonym{tangent}{TangentLine}
\pmsynonym{tangent of the curve}{TangentLine}
\pmsynonym{tangent to the curve}{TangentLine}
\pmrelated{Curve}
\pmrelated{TangentOfConicSection}
\pmrelated{Hyperbola2}
\pmdefines{tangency point}

\endmetadata

% this is the default PlanetMath preamble.  as your knowledge
% of TeX increases, you will probably want to edit this, but
% it should be fine as is for beginners.

% almost certainly you want these
\usepackage{amssymb}
\usepackage{amsmath}
\usepackage{amsfonts}

% used for TeXing text within eps files
%\usepackage{psfrag}
% need this for including graphics (\includegraphics)
%\usepackage{graphicx}
% for neatly defining theorems and propositions
%\usepackage{amsthm}
% making logically defined graphics
%%%\usepackage{xypic}

% there are many more packages, add them here as you need them

% define commands here
\begin{document}
If the curve \, $y = f(x)$\, of $xy$-plane is sufficiently smooth in its point\, $(x_0,\,y_0)$\, and in a neighborhood of this, the curve may have a tangent line (or simply \PMlinkescapetext{{\em tangent}\footnote{The word is initially a participial form {\em tangens} (its genitive: {\em tangentis}) of the Latin verb {\em tangere} `to touch'.}}) in\, $(x_0,\,y_0)$.\, Then the {\em tangent line} of the curve\, $y = f(x)$\, in the point\, $(x_0,\,y_0)$\, is the limit position of the secant line through the two points\, $(x_0,\,y_0)$\, and\, $(x,\,f(x))$\, of the curve, when $x$ limitlessly tends to the value $x_0$ (i.e.\, $x\to x_0)$.\, Due to the smoothness, 
                   $$f(x)\to f(x_0) = y_0,$$
                 $$(x,\,f(x))\to (x_0,\,y_0),$$
and the slope $m$ of the \PMlinkname{secant}{SecantLine} tends to
          $$\lim_{x\to x_0}\frac{f(x)\!-\!f(x_0)}{x\!-\!x_0} = f'(x_0)$$
which will be the slope of the tangent line.

\textbf{Note.}\, Because the tangency is a local property on the curve, the tangent with the {\em tangency point}\, $(x_0,\,y_0)$\, may intersect the curve in another point, and then the tangent is a \PMlinkname{secant}{SecantLine}, too.\, For example, the curve\, $y = x^3\!-\!3x^2$\, has the line\, $y = 0$\, as its tangent in the point\, $(0,\,0)$\, but this line \PMlinkescapetext{cuts} the curve also in the point\, $(3,\,0)$.
%%%%%
%%%%%
\end{document}
