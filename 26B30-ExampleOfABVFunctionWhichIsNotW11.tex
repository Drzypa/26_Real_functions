\documentclass[12pt]{article}
\usepackage{pmmeta}
\pmcanonicalname{ExampleOfABVFunctionWhichIsNotW11}
\pmcreated{2013-03-22 15:12:59}
\pmmodified{2013-03-22 15:12:59}
\pmowner{paolini}{1187}
\pmmodifier{paolini}{1187}
\pmtitle{example of a $BV$ function which is not $W^{1,1}$}
\pmrecord{5}{36979}
\pmprivacy{1}
\pmauthor{paolini}{1187}
\pmtype{Example}
\pmcomment{trigger rebuild}
\pmclassification{msc}{26B30}

% this is the default PlanetMath preamble.  as your knowledge
% of TeX increases, you will probably want to edit this, but
% it should be fine as is for beginners.

% almost certainly you want these
\usepackage{amssymb}
\usepackage{amsmath}
\usepackage{amsfonts}

% used for TeXing text within eps files
%\usepackage{psfrag}
% need this for including graphics (\includegraphics)
%\usepackage{graphicx}
% for neatly defining theorems and propositions
\usepackage{amsthm}
% making logically defined graphics
%%%\usepackage{xypic}

% there are many more packages, add them here as you need them

% define commands here
\newcommand{\R}{\mathbb R}
\newtheorem{theorem}{Theorem}
\newtheorem{definition}{Definition}
\theoremstyle{remark}
\newtheorem{example}{Example}
\begin{document}
The following example presents a function $u\in BV(\Omega)\setminus W^{1,1}(\Omega)$.
\begin{example}
Let $\Omega:=(-1,1)\times(-1,1)\subset \R^2$. We will show that the function
\[
u(x,y)=\begin{cases}
1,\quad\text{if $x\ge 0$}\\
0,\quad\text{if $x< 0$}\end{cases}
\]
belongs to $BV(\Omega)$.
Given $\phi\in C_c^1(\Omega,\R^2)$, $\phi=(\phi^1,\phi^2)$, one has
\begin{align*}
\iint_\Omega u(x,y)\mathrm{div}\phi(x,y)\, dxdy
&=\int_{-1}^1 \left[\int_0^1 \phi^1_x(x,y)\, dx \right]dy
+ \int_0^1 \left[\int_{-1}^1 \phi^2_y(x,y)\, dy \right]dx\\
&=\int_{-1}^1 \phi^1(1,y)-\phi^1(0,y)\, dy + \int_0^1 \phi^2(x,1)-\phi^2(x,-1)\, dx\\
&=-\int_{-1}^1 \phi^1(0,y) + 0
=-\int \phi(x,y)\,d \mu(x,y)
\end{align*}
if we choose $\mu:=(\mu^1,\mu^2):=(\mathcal H^1\llcorner(\{0\}\times(-1,1)),0)$. So we notice that $u\in BV(\Omega)$ and $Du=\mu$ is singular with respect to the Lebesgue measure $\mathcal L$.
\end{example}
%%%%%
%%%%%
\end{document}
