\documentclass[12pt]{article}
\usepackage{pmmeta}
\pmcanonicalname{ZeroOfAFunction}
\pmcreated{2013-03-22 14:00:58}
\pmmodified{2013-03-22 14:00:58}
\pmowner{mathcam}{2727}
\pmmodifier{mathcam}{2727}
\pmtitle{zero of a function}
\pmrecord{30}{34921}
\pmprivacy{1}
\pmauthor{mathcam}{2727}
\pmtype{Definition}
\pmcomment{trigger rebuild}
\pmclassification{msc}{26E99}
\pmsynonym{zero}{ZeroOfAFunction}
\pmsynonym{vanish}{ZeroOfAFunction}
\pmsynonym{vanishes}{ZeroOfAFunction}
\pmrelated{SupportOfFunction}
\pmdefines{zero set}

\endmetadata

\usepackage{amssymb}
\usepackage{amsmath}
\usepackage{amsfonts}

\def\C{\mathbb{C}}
\def\R{\mathbb{R}}
\begin{document}
\PMlinkescapeword{closed}
\PMlinkescapephrase{closed set}
\PMlinkescapeword{root}
\PMlinkescapeword{simple}

Suppose $X$ is a set and $f$ a \PMlinkname{complex}{Complex}-valued function\, $f\colon X\to \C$.\, Then a {\em zero} of $f$ is an element\, $x\in X$\, such that\, $f(x) = 0$.\, It is also said that $f$ {\em vanishes} at $x$.

The {\em zero set} of $f$ is the set
$$Z(f) := \{ x\in X \mid f(x)=0\}.$$

{\bf Remark.} When $X$ is a ``simple'' space, such as $\R$ or $\C$ a zero is also called a {\em root}.\, However, in pure mathematics and especially if $Z(f)$ is infinite, it seems to be customary to talk of zeroes and the zero set instead of roots. 

{\bf Examples}

\begin{itemize}
\item For any $z\in \C$, define $\hat{z}:X\to \C$ by $\hat{z}(x)=z$.  Then $Z(\hat{0})=X$ and $Z(\hat{z})=\varnothing$ if $z\ne 0$.
\item Suppose $p$ is a \PMlinkname{polynomial}{Polynomial}\, $p\colon\C\to\C$\, of degree $n\ge 1$.\, Then $p$ has at most $n$ zeroes. That is, $|Z(p)|\le n$. 
\item If $f$ and $g$ are functions $f\colon X\to\C$ and  $g\colon X\to\C$, then
\begin{eqnarray*}
Z(fg)&=&Z(f)\cup Z(g),\\
Z(fg)&\supseteq& Z(f),
\end{eqnarray*}
where $fg$ is the function\, $x\mapsto f(x) g(x)$.
\item For any $f\colon X\to \R$, then $$Z(f)=Z(|f|)=Z(f^n),$$ where $f^n$ is the defined $f^n(x)=(f(x))^n$.
\item If $f$ and $g$ are both real-valued functions, then
$$Z(f)\cap Z(g)=Z(f^2+g^2)=Z(|f|+|g|).$$
\item If $X$ is a topological space and $f:X\to \C$ is a function, then the \PMlinkname{support}{SupportOfFunction} of $f$ is given by:
$$\operatorname{supp} f = \overline{Z(f)^\complement}$$
Further, if $f$ is continuous, then $Z(f)$ is \PMlinkname{closed}{ClosedSet} in $X$ (assuming that $\C$ is given the usual topology of the complex plane where
$\{0\}$ is a closed set).
\end{itemize}
%%%%%
%%%%%
\end{document}
