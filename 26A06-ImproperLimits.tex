\documentclass[12pt]{article}
\usepackage{pmmeta}
\pmcanonicalname{ImproperLimits}
\pmcreated{2013-03-22 14:40:45}
\pmmodified{2013-03-22 14:40:45}
\pmowner{pahio}{2872}
\pmmodifier{pahio}{2872}
\pmtitle{improper limits}
\pmrecord{24}{36283}
\pmprivacy{1}
\pmauthor{pahio}{2872}
\pmtype{Definition}
\pmcomment{trigger rebuild}
\pmclassification{msc}{26A06}
\pmsynonym{infinite limits}{ImproperLimits}
\pmsynonym{improper limit}{ImproperLimits}
\pmrelated{LHpitalsRule}
\pmrelated{ExtendedRealNumbers}
\pmrelated{LimitRulesOfFunctions}
\pmrelated{IntegratingTanXOver0fracpi2}
\pmrelated{IndeterminateForm}
\pmrelated{ExampleOfJumpDiscontinuity}
\pmrelated{ListOfCommonLimits}
\pmrelated{LimitsOfNaturalLogarithm}
\pmrelated{SecondDerivativeAsSimpleLimit}
\pmrelated{AngleBetweenTwoLines}
\pmdefines{limit at infinity}
\pmdefines{mnemonic of infinite}

% this is the default PlanetMath preamble.  as your knowledge
% of TeX increases, you will probably want to edit this, but
% it should be fine as is for beginners.

% almost certainly you want these
\usepackage{amssymb}
\usepackage{amsmath}
\usepackage{amsfonts}

% used for TeXing text within eps files
%\usepackage{psfrag}
% need this for including graphics (\includegraphics)
%\usepackage{graphicx}
% for neatly defining theorems and propositions
%\usepackage{amsthm}
% making logically defined graphics
%%%\usepackage{xypic}

% there are many more packages, add them here as you need them

% define commands here
\begin{document}
In calculus there is often used such expressions as ``the limit of a function is infinite'', and one may write for instance that 
$$\lim_{x \to 0}\frac{1}{x^2} \;=\; \infty.$$
Such ``limits'' are actually \PMlinkescapetext{extensions} of the limit notion, and can be defined exactly.\, They are called {\em improper limits}.


\textbf{Definition.}\, Let the real function $f$ be defined in a neighbourhood of the point $x_0$. 
            $$\lim_{x \to x_0}f(x) \;=\; \infty$$
iff for every real number $M$ there exists a number $\delta_M$ such that
                     $$f(x) \;>\; M$$
as soon as
              $$0 \;<\; |x\!-\!x_0| \;<\; \delta_M.$$\\
In a similar way we can define the improper limit $-\infty$ of a real function.\, The definition may be extended also to the cases\, $x \to \pm\infty$, when one speaks of \emph{limits at infinity}.\\


\textbf{Note 1.}\, If\, $\lim_{x \to x_0}f(x) \,=\, \infty$\, and\, 
$\lim_{x \to x_0}g(x) \,=\, a > 0$,\, then we have 
          $$\lim_{x \to x_0}f(x)g(x) \;=\; \infty.$$
Hence we can say that\, $\infty\cdot a = \infty$\, when\, $a > 0$.\, There are some other \PMlinkescapetext{comparable} ``mnemonics of infinite'' (cf. the extended real numbers):
$$\infty\cdot a \;=\; -\infty \qquad(a \;<\; 0)$$
$$\pm\infty+a \;=\; \pm\infty$$
$$\frac{a}{\pm\infty} \;=\; 0$$
$$\infty+\infty \;=\; \infty$$
$$\infty\cdot\infty \;=\; \infty$$
$$-\infty\cdot\infty \;=\; -\infty$$


On the contrary, there exist no mnemonics for the cases
   $$\infty\cdot0,\,\, \infty-\infty,\,\, \frac{\infty}{\infty},\,\, 
     \frac{0}{0},\,\, 0^0,\,\, \infty^0,\,\, 1^\infty;$$
they are \PMlinkescapetext{indefinite} and depend on the instance (cf. the indeterminate form).\\


\textbf{Note 2.}\, In the complex plane, the expression
          $$\lim_{z \to z_0}f(z) \;=\; \infty$$
means that\, $\displaystyle \lim_{z \to z_0}|f(z)| \,=\, \infty$.
%%%%%
%%%%%
\end{document}
