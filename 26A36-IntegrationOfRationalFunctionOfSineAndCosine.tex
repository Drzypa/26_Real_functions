\documentclass[12pt]{article}
\usepackage{pmmeta}
\pmcanonicalname{IntegrationOfRationalFunctionOfSineAndCosine}
\pmcreated{2013-03-22 17:05:15}
\pmmodified{2013-03-22 17:05:15}
\pmowner{pahio}{2872}
\pmmodifier{pahio}{2872}
\pmtitle{integration of rational function of sine and cosine}
\pmrecord{29}{39380}
\pmprivacy{1}
\pmauthor{pahio}{2872}
\pmtype{Topic}
\pmcomment{trigger rebuild}
\pmclassification{msc}{26A36}
\pmsynonym{universal trigonometric substitution}{IntegrationOfRationalFunctionOfSineAndCosine}
\pmrelated{GoniometricFormulae}
\pmrelated{SubstitutionForIntegration}
\pmrelated{WeierstrassSubstitutionFormulas}
\pmrelated{EulersSubstitutionsForIntegration}
\pmrelated{ErrorsCanCancelEachOtherOut}
\pmdefines{universal hyperboloc substitution}

\endmetadata

% this is the default PlanetMath preamble.  as your knowledge
% of TeX increases, you will probably want to edit this, but
% it should be fine as is for beginners.

% almost certainly you want these
\usepackage{amssymb}
\usepackage{amsmath}
\usepackage{amsfonts}

% used for TeXing text within eps files
%\usepackage{psfrag}
% need this for including graphics (\includegraphics)
%\usepackage{graphicx}
% for neatly defining theorems and propositions
% \usepackage[utf8]{inputenc}
 \usepackage{amsthm}
 \usepackage[T2A]{fontenc}
 \usepackage[russian, english]{babel}

% making logically defined graphics
%%%\usepackage{xypic}

% there are many more packages, add them here as you need them

% define commands here

\theoremstyle{definition}
\newtheorem*{thmplain}{Theorem}

\begin{document}
The integration task
\begin{align}
   \int\!R(\sin{x},\,\cos{x})\,dx,
\end{align}
where the integrand is a rational function of $\sin{x}$ and $\cos{x}$, changes via the Weierstrass substitution
\begin{align}
   \tan\frac{x}{2} \;=\; t
\end{align}
to a form having an integrand that is a rational function of $t$.\, Namely, since\, $x = 2\arctan{t}$,\, we have 
\begin{align}
dx \;=\; 2\cdot\frac{1}{1\!+\!t^2}\,dt,
\end{align}
and we can substitute
\begin{align}
   \sin{x}\;=\; \frac{2t}{1\!+\!t^2}, \quad \cos{x} \;=\; \frac{1\!-\!t^2}{1\!+\!t^2},
\end{align}
getting
$$\int\!R(\sin{x},\,\cos{x})\,dx \;=\; 
2\int\!R\!\left(\frac{2t}{1\!+\!t^2},\,\frac{1\!-\!t^2}{1\!+\!t^2}\right)\frac{dt}{1\!+\!t^2}.$$

{\em Proof} of the formulae (4):\; Using the double angle formulas of sine and cosine and then dividing the numerators and the denominators by\, $\cos^2\frac{x}{2}$\, we obtain
$$\sin{x} \;=\; 
\frac{2\sin\frac{x}{2}\cos\frac{x}{2}}{\sin^2\frac{x}{2}+\cos^2\frac{x}{2}}
 \;=\; \frac{2\tan\frac{x}{2}}{1+\tan^2\frac{x}{2}} \;=\; \frac{2t}{1+t^2},$$
$$\cos{x} \;=\; \frac{\cos^2\frac{x}{2}-\sin^2\frac{x}{2}}{\sin^2\frac{x}{2}+\cos^2\frac{x}{2}} 
\;=\; \frac{1-\tan^2\frac{x}{2}}{1+\tan^2\frac{x}{2}} \;=\; \frac{1-t^2}{1+t^2}.$$

\textbf{Example.}\; The above formulae give from\, $\displaystyle \int\frac{dx}{\sin{x}}$\, the result
$$\int\frac{dx}{\sin{x}} \;=\; \int\frac{1\!+\!t^2}{2t}\cdot2\cdot\frac{1}{1\!+\!t^2}\;dt = 
\int\frac{dt}{t} \;=\; \ln|t|+C \;=\; \ln\left|\tan\frac{x}{2}\right|+C$$
(which can also be expressed in the form $-\ln|\csc{x}+\cot{x}|+C$; see the goniometric formulas).\\

\textbf{Note 1.}\; The substitution (2) is sometimes called the \PMlinkname{``universal trigonometric substitution''}{UniversalTrigonometricSubstitution}.\, In practice, it often gives rational functions that are too complicated.\, In many cases, it is more profitable to use other substitutions:
\begin{itemize}
\item In the case\, $\int\!R(\sin{x})\cos{x}\,dx$\, the substitution\, $\sin{x} = t$\, is simpler.
\item Similarly, in the case\, $\int\!R(\cos{x})\sin{x}\,dx$\, the substitution\, $\cos{x} = t$\, is simpler. 
\item If the integrand depends only on $\tan{x}$, the substitution\, $\tan{x} = t$\, is simpler.
\item If the integrand is of the form\, $R(\sin^2{x},\, \cos^2{x})$,\, one can use the substitution\, $\tan{x} = t$; then\\
$\displaystyle \cos^2{x} = \frac{1}{1+\tan^2{x}} = \frac{1}{1+t^2}$,\;\; $\displaystyle \sin^2{x} = 1-\cos^2{x} = \frac{t^2}{1+t^2}$,\;\; $\displaystyle dx = \frac{dt}{1+t^2}.$
\end{itemize}

\textbf{Example.}\; The integration of\; $\displaystyle \int\!\frac{dx}{\cos^4{x}}\,dx$\, is of the last case:
$$\int\!\frac{dx}{\cos^4{x}}\,dx = \int\!\frac{1}{(\cos^2{x})^2}\,dx = \int\!(1+t^2)^2\cdot\frac{dt}{1+t^2} = 
\int\!(1+t^2)\,dt = \frac{t^3}{3}+t+C = \frac{1}{3}\tan^3{x}+\tan{x}+C.$$

\textbf{Example.}\; The integral \;$\displaystyle I = \int\!\frac{dx}{\cos^3{x}}\,dx = \int\! \sec^3{x}\,dx $\,
is a peculiar case in which one does not have to use the substitutions mentioned above, as integration by parts is a simpler method for evaluating this integral.  Thus,

$$u = \sec{x}\; \Rightarrow\; du = \sec{x}\;\tan{x}\,dx; \qquad dv = \sec^2{x}\,dx \; \Rightarrow \; v = \tan{x}.$$
Therefore,
\begin{center}
$\begin{array}{rl}
I & \displaystyle = \int\! \sec^3{x}\,dx \\
& \displaystyle = \sec{x}\;\tan{x} - \int\! \sec{x}\;\tan^2{x}\,dx \\
& \displaystyle = \sec{x}\;\tan{x} - \int\! \sec{x}\;(\sec^2{x}-1)\,dx \\
& \displaystyle = \sec{x}\;\tan{x} - I + \int\! \sec{x}\,dx, \end{array}$
\end{center}
and consequently
$$\int\!\frac{dx}{\cos^3{x}}\,dx \;=\; \frac{1}{2} \big(\sec{x}\;\tan{x}\;+\ln\;|\sec{x}+\tan{x}| \big)+C.$$\\

\textbf{Note 2.}\, There is also the ``universal hyperbolic substitution'' for integrating rational functions of hyperbolic sine and cosine:
$$\tanh\frac{x}{2} \;=\; t, \quad dx \;=\; \frac{2dt}{1\!-\!t^2}, \quad \sinh{x} \;=\; \frac{2t}{1\!-\!t^2}, \quad \cosh{x} \;=\; \frac{1\!+\!t^2}{1\!-\!t^2}$$

\begin{thebibliography}{7}
\bibitem{lk} \CYRL. \CYRD. \CYRK\cyr\cyrd\cyrr\cyrya\cyrch\cyre\cyrv: 
{\em \CYRM\cyra\cyrt\cyre\cyrm\cyra\cyrt\cyri\cyrch\cyre\cyrc\cyrk\cyri\cyri\, \cyra\cyrn\cyra\cyrl\cyri\cyrz}.\, \CYRI\cyrz\cyrd\cyra\cyrt\cyre\cyrl\cyrsftsn\cyrs\cyrt\cyrv\cyro \,
``\CYRV\cyry\cyrs\cyrsh\cyra\cyrya \CYRSH\cyrk\cyro\cyrl\cyra''. \CYRM\cyro\cyrs\cyrk\cyrv\cyra \,(1970).
\end{thebibliography}

%%%%%
%%%%%
\end{document}
