\documentclass[12pt]{article}
\usepackage{pmmeta}
\pmcanonicalname{HarmonicMeanInTrapezoid}
\pmcreated{2013-03-22 17:49:22}
\pmmodified{2013-03-22 17:49:22}
\pmowner{pahio}{2872}
\pmmodifier{pahio}{2872}
\pmtitle{harmonic mean in trapezoid}
\pmrecord{14}{40286}
\pmprivacy{1}
\pmauthor{pahio}{2872}
\pmtype{Theorem}
\pmcomment{trigger rebuild}
\pmclassification{msc}{26B99}
\pmclassification{msc}{51M04}
\pmclassification{msc}{51M15}
%\pmkeywords{intersection of diagonals}
%\pmkeywords{parallel to bases}
%\pmkeywords{similarity}
\pmrelated{HarmonicMean}
\pmrelated{SimilarityOfTriangles}
\pmrelated{CorrespondingAnglesInTransversalCutting}
\pmrelated{SimilarityInGeometry}
\pmrelated{MedianOfTrapezoid}
\pmrelated{ConstructionOfContraharmonicMeanOfTwoSegments}
\pmrelated{IntegerHarmonicMeans}

\endmetadata

% this is the default PlanetMath preamble.  as your knowledge
% of TeX increases, you will probably want to edit this, but
% it should be fine as is for beginners.

% almost certainly you want these
\usepackage{amssymb}
\usepackage{amsmath}
\usepackage{amsfonts}

% used for TeXing text within eps files
%\usepackage{psfrag}
% need this for including graphics (\includegraphics)
%\usepackage{graphicx}
% for neatly defining theorems and propositions
 \usepackage{amsthm}
% making logically defined graphics
%%%\usepackage{xypic}

% there are many more packages, add them here as you need them

\usepackage{pstricks}

% define commands here

\theoremstyle{definition}
\newtheorem*{thmplain}{Theorem}

\begin{document}
\textbf{Theorem.}\, If a line parallel to the bases of a trapezoid passes through the intersecting point of the diagonals, then the portion of the line inside the trapezoid is the harmonic mean of the bases.

{\em Proof.}\, Let $AB$ and $DC$ be the bases of a trapezoid $ABCD$ and $E$ the intersecting point of the diagonals of $ABCD$.  Denote the cutting point of $AD$ and the line through $E$ and parallel to the bases by $P$, and the cutting point of $BC$ and the same line by $Q$.\, Then we have
$$\Delta CDE \;\sim\; \Delta ABE$$
with line ratio \,$\displaystyle\frac{k}{h} = \frac{CD}{AB}$, where $h$ and $k$ are the heights of the triangles $ABE$ and $CDE$, respectively, when $h\!+\!k$ equals the height of the trapezoid.\, We have also
$$\Delta PED \;\sim\; \Delta ABD$$
with line ratio
$$PE:AB \;=\; \frac{k}{h\!+\!k} \;=\; \frac{\frac{k}{h}}{1\!+\!\frac{k}{h}} 
\;=\; \frac{\frac{CD}{AB}}{1+\frac{CD}{AB}}.$$
Thus we can express the length of $PE$ as
$$PE \;=\; AB\cdot\frac{\frac{CD}{AB}}{1+\frac{CD}{AB}} 
\;=\; \frac{CD}{1+\frac{CD}{AB}} \;=\; \frac{AB\!\cdot\!CD}{AB\!+\!CD}.$$
Similarly we may determine $EQ$ and \PMlinkescapetext{state} that\, $EQ = PE$.\, Consequently,
$$PQ \;=\; PE\!+\!EQ \;=\; \frac{2\!\cdot\!AB\!\cdot\!CD}{AB\!+\!CD},$$
which is the harmonic mean of the bases $AB$ and $CD$.\\

\begin{center}
\begin{pspicture}(-1,-1)(7,4)
\pspolygon(0,0)(6,0)(5,4)(2,4)
\psline[linecolor=blue, linewidth=0.06](0,0)(6,0)
\psline[linecolor=blue, linewidth=0.06](2,4)(5,4)

\psline(0,0)(5,4)
\psline(6,0)(2,4)
\psline[linecolor=red, linewidth=0.06](1.33,2.67)(5.33,2.67)
\psline(-0.5,2.67)(1.33,2.67)
\psline(5.33,2.67)(6.5,2.67)
\psdot(1.33,2.67)
\psdot(5.33,2.67)
\psline[linestyle=dotted](3.33,0)(3.33,4)
\rput[a](0,-0.3){$A$}
\rput[a](6,-0.3){$B$}
\rput[a](5,4.3){$C$}
\rput[a](2,4.3){$D$}
\rput[a](3.4,2.40){$E$}
\rput[a](1.2,2.85){$P$}
\rput[a](5.5,2.85){$Q$}
\rput[a](3.5,1.30){$h$}
\rput[a](3.5,3.45){$k$}
\psline(3.33,0.2)(3.11,0.2)
\psline(3.11,0.2)(3.11,0)
\end{pspicture}
\end{center}


%%%%%
%%%%%
\end{document}
