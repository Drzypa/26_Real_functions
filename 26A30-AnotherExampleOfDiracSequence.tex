\documentclass[12pt]{article}
\usepackage{pmmeta}
\pmcanonicalname{AnotherExampleOfDiracSequence}
\pmcreated{2013-03-22 17:19:50}
\pmmodified{2013-03-22 17:19:50}
\pmowner{Wkbj79}{1863}
\pmmodifier{Wkbj79}{1863}
\pmtitle{another example of Dirac sequence}
\pmrecord{8}{39683}
\pmprivacy{1}
\pmauthor{Wkbj79}{1863}
\pmtype{Example}
\pmcomment{trigger rebuild}
\pmclassification{msc}{26A30}

\usepackage{amssymb}
\usepackage{amsmath}
\usepackage{amsfonts}
\usepackage{pstricks}
\usepackage{psfrag}
\usepackage{graphicx}
\usepackage{amsthm}
%%\usepackage{xypic}

\begin{document}
Let $\displaystyle A_n=\left[\frac{-1}{2^n},\frac{1}{2^n}\right]$ and $\delta_n=2^{n-1}\chi_{A_n}$ for every positive integer $n$, where $\chi_S$ denotes the characteristic function of the set $S$.  Then $\{\delta_n\}$ is a Dirac sequence:

\begin{enumerate}
\item $\delta_n(t)\ge 0$ for every positive integer $n$ and every $t\in\mathbb{R}$.
\item Let $n$ be a positive integer.  Then $\displaystyle \int\limits_{-\infty}^{\infty} \delta_n(t) \, dt=\int\limits_{\frac{-1}{2^n}}^{\frac{1}{2^n}} 2^{n-1} \, dt=1$.

\item Let $r>0$.  Then there exists a positive integer $N$ such that, for every integer $k>N$, we have $\displaystyle \frac{1}{2^k}<r$.  Thus, for every integer $k>N$, we have $\displaystyle \int\limits_{\mathbb{R}\setminus\left[ -r,r\right]} d_k(t) \, dt=0$.
\end{enumerate}
%%%%%
%%%%%
\end{document}
