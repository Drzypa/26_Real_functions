\documentclass[12pt]{article}
\usepackage{pmmeta}
\pmcanonicalname{ProofOfChebyshevsInequality}
\pmcreated{2013-03-22 13:08:38}
\pmmodified{2013-03-22 13:08:38}
\pmowner{pbruin}{1001}
\pmmodifier{pbruin}{1001}
\pmtitle{proof of Chebyshev's inequality}
\pmrecord{4}{33582}
\pmprivacy{1}
\pmauthor{pbruin}{1001}
\pmtype{Proof}
\pmcomment{trigger rebuild}
\pmclassification{msc}{26D15}

\endmetadata

% this is the default PlanetMath preamble.  as your knowledge
% of TeX increases, you will probably want to edit this, but
% it should be fine as is for beginners.

% almost certainly you want these
\usepackage{amssymb}
\usepackage{amsmath}
\usepackage{amsfonts}

% used for TeXing text within eps files
%\usepackage{psfrag}
% need this for including graphics (\includegraphics)
%\usepackage{graphicx}
% for neatly defining theorems and propositions
%\usepackage{amsthm}
% making logically defined graphics
%%%\usepackage{xypic}

% there are many more packages, add them here as you need them

% define commands here
\begin{document}
Let $x_1,x_2,\ldots,x_n$ and $y_1,y_2,\ldots,y_n$ be real numbers such
that $x_1\le x_2\le\cdots\le x_n$.  Write the product
$(x_1+x_2+\cdots+x_n)(y_1+y_2+\cdots+y_n)$ as
\begin{eqnarray}
\nonumber
&&(x_1y_1+x_2y_2+\cdots+x_ny_n)\\
\nonumber
&+&(x_1y_2+x_2y_3+\cdots+x_{n-1}y_n+x_ny_1)\\
\nonumber
&+&(x_1y_3+x_2y_4+\cdots+x_{n-2}y_n+x_{n-1}y_1+x_ny_2)\\
\nonumber
&+&\cdots\\
&+&(x_1y_n+x_2y_1+x_3y_2+\cdots+x_ny_{n-1}).
\label{expanded}
\end{eqnarray}
\begin{itemize}
\item
If $y_1\le y_2\le\cdots\le y_n$, each of the $n$ terms in parentheses
is less than or equal to $x_1y_1+x_2y_2+\cdots+x_ny_n$, according to
the rearrangement inequality.  From this, it follows that
$$
(x_1+x_2+\cdots+x_n)(y_1+y_2+\cdots+y_n)\le
n(x_1y_1+x_2y_2+\cdots+x_ny_n)
$$
or (dividing by $n^2$)
$$
\left(\frac{x_1+x_2+\cdots+x_n}{n}\right)
\left(\frac{y_1+y_2+\cdots+y_n}{n}\right)\le
\frac{x_1y_1+x_2y_2+\cdots+x_ny_n}{n}.
$$
\item
If $y_1\ge y_2\ge\cdots\ge y_n$, the same reasoning gives
$$
\left(\frac{x_1+x_2+\cdots+x_n}{n}\right)
\left(\frac{y_1+y_2+\cdots+y_n}{n}\right)\ge
\frac{x_1y_1+x_2y_2+\cdots+x_ny_n}{n}.
$$
\end{itemize}
It is clear that equality holds if $x_1=x_2=\cdots=x_n$ or
$y_1=y_2=\cdots=y_n$.  To see that this condition is also necessary,
suppose that not all $y_i$'s are equal, so that $y_1\neq y_n$.
Then the second term in parentheses of (\ref{expanded}) can only be
equal to $x_1y_1+x_2y_2+\cdots+x_ny_n$ if $x_{n-1}=x_n$, the third
term only if $x_{n-2}=x_{n-1}$, and so on, until the last term which
can only be equal to $x_1y_1+x_2y_2+\cdots+x_ny_n$ if $x_1=x_2$.  This
implies that $x_1=x_2=\cdots=x_n$.  Therefore, Chebyshev's inequality
is an equality if and only if $x_1=x_2=\cdots=x_n$ or
$y_1=y_2=\cdots=y_n$.
%%%%%
%%%%%
\end{document}
