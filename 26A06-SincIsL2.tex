\documentclass[12pt]{article}
\usepackage{pmmeta}
\pmcanonicalname{SincIsL2}
\pmcreated{2013-03-22 15:44:44}
\pmmodified{2013-03-22 15:44:44}
\pmowner{cvalente}{11260}
\pmmodifier{cvalente}{11260}
\pmtitle{sinc is $L^2$}
\pmrecord{9}{37697}
\pmprivacy{1}
\pmauthor{cvalente}{11260}
\pmtype{Result}
\pmcomment{trigger rebuild}
\pmclassification{msc}{26A06}

% this is the default PlanetMath preamble.  as your knowledge
% of TeX increases, you will probably want to edit this, but
% it should be fine as is for beginners.

% almost certainly you want these
\usepackage{amssymb}
\usepackage{amsmath}
\usepackage{amsfonts}

% used for TeXing text within eps files
%\usepackage{psfrag}
% need this for including graphics (\includegraphics)
%\usepackage{graphicx}
% for neatly defining theorems and propositions
%\usepackage{amsthm}
% making logically defined graphics
%%%\usepackage{xypic}

% there are many more packages, add them here as you need them

% define commands here
\begin{document}
Our objective will be to prove the integral $\int_{\mathbb{R}} f^2(x) dx$ exists in the Lebesgue sense when $f(x) = \operatorname{sinc}(x)$.

The integrand is an even function and so we can restrict our proof to the set $\mathbb{R}^+$.

Since $f$ is a continuous function, so will $f^2$ be and thus for every $a>0$, $f \in L^2([0,a])$.

Thus, if we prove $f \in L^2([\pi,\infty[)$, the result will be proved.

Consider the intervals $I_k = [k\pi, (k+1)\pi]$ and $U_k = \bigcup_{i=1}^k I_k = [\pi, (k+1)\pi]$.

and the succession of functions $f_n(x) = f^2(x)\chi_{U_n}(x)$, where $\chi_{U_n}$ is the characteristic function of the set $U_n$.

Each $f_n$ is a continuous function of compact support and will thus be integrable in $\mathbb{R}^+$. Furthermore $f_n(x) \nearrow f^2(x)$ (pointwise) in this set.

In each $I_k$,$ 0 \le f^2(x) \le \frac{\sin^2(x)}{(k\pi)^2}$, for $k>0$.

So:

$ \displaystyle \int_{x\ge \pi} f_n(x) dx = \sum_{k=1}^{n} \int_{k\pi}^{(k+1)\pi} \frac{\sin(x)^2}{x^2} dx \le \sum_{k=1}^{n} \int_{k\pi}^{(k+1)\pi} \frac{\sin(x)^2}{(k\pi)^2} = \sum_{k=1}^{n} \frac{1}{2k^2\pi} $
\footnote{we have used the well known result $\int_{0}^{\pi}\sin^2(x) dx = \frac{\pi}{2}$}

So: $\lim_{n\to \infty} \int_{x\ge \pi}f_n(x) dx \le  \lim_{n\to \infty} \sum_{k=1}^{n} \frac{1}{2k^2\pi}$ and since the series on the right side converges\footnote{asymptotic behaviour as $k^{-2}$} and $f_n \nearrow f^2$ we can use the monotone convergence theorem to state that $f^2 \in L([\pi,\infty[)$.

So we get the result that $\operatorname{sinc} \in L^2(\mathbb{R})$
%%%%%
%%%%%
\end{document}
