\documentclass[12pt]{article}
\usepackage{pmmeta}
\pmcanonicalname{HessianForm}
\pmcreated{2013-03-22 15:00:03}
\pmmodified{2013-03-22 15:00:03}
\pmowner{PrimeFan}{13766}
\pmmodifier{PrimeFan}{13766}
\pmtitle{Hessian form}
\pmrecord{6}{36708}
\pmprivacy{1}
\pmauthor{PrimeFan}{13766}
\pmtype{Definition}
\pmcomment{trigger rebuild}
\pmclassification{msc}{26B12}
\pmrelated{HessianMatrix}
\pmrelated{RelationsBetweenHessianMatrixAndLocalExtrema}
\pmrelated{TestsForLocalExtremaForLagrangeMultiplierMethod}

\endmetadata

% this is the default PlanetMath preamble.  as your knowledge
% of TeX increases, you will probably want to edit this, but
% it should be fine as is for beginners.

% almost certainly you want these
\usepackage{amssymb}
\usepackage{amsmath}
\usepackage{amsfonts}

% used for TeXing text within eps files
%\usepackage{psfrag}
% need this for including graphics (\includegraphics)
%\usepackage{graphicx}
% for neatly defining theorems and propositions
%\usepackage{amsthm}
% making logically defined graphics
%%%\usepackage{xypic}

% there are many more packages, add them here as you need them

% define commands here
\def\sse{\subseteq}
\def\bigtimes{\mathop{\mbox{\Huge $\times$}}}
\def\impl{\Rightarrow}
\def\R{\mathbb{R}}
\def\del{\partial}
\begin{document}
Given a smooth manifold $M$ and $f\colon M\to \R$ being in $C^2(M)$, if $x$ is a critical point of $f$, that is $df = 0$ at $x$, then we can define a symmetric $2$-form
\[
  H(u_x,v_x) = u(v(f)) = v(u(f)),
\]
where $H\in T^{*\otimes2}$ and $u$ and $v$ are any vector fields that take the values $u_x$ and $v_x$, respectively, at point $x$. Equality of the two defining expressions follows from the fact that $x$ is a critical point of $f$, because then $[u,v](f) = df([u,v]) = 0$, where $[u,v]$ denotes the Lie bracket of the two vector fields. The form $H$ is called the \emph{Hessian form}.

In local coordinates, the Hessian form is given by 
\[
  H = \frac{\del^2 f}{\del x^i \del x^j}\, dx^i\otimes dx^j.
\]
Its components are those of the Hessian matrix in the same coordinates. The advantage of the above formulation is coordinate independence. However, the price is that the Hessian form is only defined at critical points. It does not define a tensor field as one would na\"ively expect.

Using the Hessian form, it is possible to analyze the critical points of $f$ (determine whether they are local minima, maxima, or saddle points) in a coordinate independent way.
%%%%%
%%%%%
\end{document}
