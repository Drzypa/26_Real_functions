\documentclass[12pt]{article}
\usepackage{pmmeta}
\pmcanonicalname{ChainRuleseveralVariables}
\pmcreated{2013-03-22 12:33:10}
\pmmodified{2013-03-22 12:33:10}
\pmowner{rmilson}{146}
\pmmodifier{rmilson}{146}
\pmtitle{chain rule (several variables)}
\pmrecord{10}{32798}
\pmprivacy{1}
\pmauthor{rmilson}{146}
\pmtype{Theorem}
\pmcomment{trigger rebuild}
\pmclassification{msc}{26B12}
\pmrelated{ChainRule}
\pmrelated{Jacobian}
\pmrelated{JacobianMatrix}

\usepackage{amsmath}
\usepackage{amsfonts}
\usepackage{amssymb}
\newcommand{\vF}{\mathbf{F}}
\newcommand{\vG}{\mathbf{G}}

\newcommand{\reals}{\mathbb{R}}
\newcommand{\natnums}{\mathbb{N}}
\newcommand{\cnums}{\mathbb{C}}
\newcommand{\znums}{\mathbb{Z}}

\newcommand{\lp}{\left(}
\newcommand{\rp}{\right)}
\newcommand{\lb}{\left[}
\newcommand{\rb}{\right]}

\newcommand{\supth}{^{\text{th}}}


\newtheorem{proposition}{Proposition}
\begin{document}
The chain rule is a theorem of analysis that governs derivatives of
composed functions.  The basic theorem is the chain rule for functions
of one variables (\PMlinkname{see here}{ChainRule}). This entry
is devoted to the more general version involving functions of several
variables and partial derivatives.  Note: the symbol $D_k$ will be
used to denote the partial derivative with respect to the $k\supth$
variable.

Let $F(x_1,\ldots,x_n)$ and
$G_1(x_1,\ldots,x_m),\ldots,G_n(x_1,\ldots,x_m)$  be differentiable
functions of several variables, and let
$$H(x_1,\ldots,x_m) =
F(G_1(x_1,\ldots,x_m),\ldots,G_n(x_1,\ldots,x_m))$$
be the function determined by the composition of $F$ with
$G_1,\ldots,G_n$
The partial derivatives of $H$ are given by
$$(D_k H)(x_1,\ldots,x_m) = \sum_{i=1}^n (D_i
F)(G_1(x_1,\ldots,x_m),\ldots) (D_k G_i)(x_1,\ldots,x_m). $$

The chain rule can be more compactly (albeit less precisely) expressed
in terms of the Jacobi-Legendre partial derivative symbols
(\PMlinkexternal{historical note}{http://members.aol.com/jeff570/calculus.html}).  Just as in
the Leibniz system, the basic idea is that of one quantity (i.e.
variable) depending on one or more other quantities.  Thus we would
speak about a variable $z$ depends differentiably on $y_1,\ldots,y_n$,
which in turn depend differentiably on variables $x_1,\ldots,x_m$.  We
would then write the chain rule as
$$\frac{\partial z}{\partial x_j} = \sum_{i=1}^n \frac{\partial
  z}{\partial y_i} \frac{\partial y_i}{\partial x_j},\qquad j=1,\ldots
m.$$


% The customary approach is to use a hybrid system, one which makes use
% of the notion of a function, but nonetheless makes use of the partial
% derivative symbols.  In this approach it is vitally important to use
% different dummy variables in different functions, and this is the
% principal weakness of the hybrid system.  Now we begin with a
% functions $F(y_1,\ldots, y_n)$ and $G_i(x_1,\ldots,x_m)$,
% $i=1\ldots,n$ and agree that the composed function $H$ is a function
% of $x_1,\ldots,x_m$. The chain rule now takes the following appearance:
% $$\frac{\partial H}{\partial x_j}=\sum_{i=1}^n \frac{\partial
%   F}{\partial y_i} \left. \frac{\partial G_i}{\partial
%     x_j}\right|_{y_i = G_i},\qquad j=1,\ldots,m.$$

The most general, and conceptually clear approach to the
multi-variable chain is based on the notion of a differentiable
mapping, with the Jacobian matrix of partial derivatives playing the
role of generalized derivative.  Let, $X\subset\reals^m$ and
$Y\subset\reals^n$ be open domains and let
$$\vF:Y\rightarrow\reals^l,\qquad \vG:X\rightarrow Y$$
be differentiable mappings.  In essence, the symbol $\vF$ represents
$l$ functions of $n$ variables each:
$$\vF = (F_1,\ldots,F_l),\qquad F_i=F_i(x_1,\ldots,x_n),$$
whereas
$\vG=(G_1,\ldots,G_n)$ represents $n$ functions of $m$ variables each.
The derivative of such mappings is no longer a function, but rather a
matrix of partial derivatives, customarily called the Jacobian matrix.
Thus
$$
D\vF = 
\begin{pmatrix}
  D_1 F_1 & \ldots & D_n F_1 \\
  \vdots & \ddots & \vdots \\
  D_1 F_l & \ldots & D_n F_l
\end{pmatrix}
\qquad
D\vG = 
\begin{pmatrix}
  D_1 G_1 & \ldots & D_m G_1 \\
  \vdots & \ddots & \vdots \\
  D_1 G_n & \ldots & D_m G_n
\end{pmatrix}
$$
The chain rule now takes the same form as it did for functions of one
variable:
$$D(\vF\circ \vG) = ((D\vF) \circ \vG)\, (D\vG),$$
albeit with matrix
multiplication taking the place of ordinary multiplication.

This form of the chain rule also generalizes quite nicely to the even
more general setting where one is interested in describing the
derivative of a composition of mappings between manifolds.
%%%%%
%%%%%
\end{document}
