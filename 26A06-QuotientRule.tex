\documentclass[12pt]{article}
\usepackage{pmmeta}
\pmcanonicalname{QuotientRule}
\pmcreated{2013-03-22 12:38:51}
\pmmodified{2013-03-22 12:38:51}
\pmowner{Luci}{289}
\pmmodifier{Luci}{289}
\pmtitle{quotient rule}
\pmrecord{13}{32913}
\pmprivacy{1}
\pmauthor{Luci}{289}
\pmtype{Theorem}
\pmcomment{trigger rebuild}
\pmclassification{msc}{26A06}
%\pmkeywords{calculus}
%\pmkeywords{derivative}
%\pmkeywords{fractions}
%\pmkeywords{derivatives}

% this is the default PlanetMath preamble.  as your knowledge
% of TeX increases, you will probably want to edit this, but
% it should be fine as is for beginners.

% almost certainly you want these
\usepackage{amssymb}
\usepackage{amsmath}
\usepackage{amsfonts}

% used for TeXing text within eps files
%\usepackage{psfrag}
% need this for including graphics (\includegraphics)
%\usepackage{graphicx}
% for neatly defining theorems and propositions
%\usepackage{amsthm}
% making logically defined graphics
%%%\usepackage{xypic}

% there are many more packages, add them here as you need them

% define commands here
\begin{document}
The \emph{quotient rule} says that the derivative of the quotient \(f/g\) of two differentiable functions \(f\) and \(g\) exists at all values of \(x\) as long as \(g(x) \not= 0\) and is given by the formula

\begin{equation*}
\frac{d}{dx}\ \left[\frac{f(x)}{g(x)}\ \right] = 
\frac{g(x)f'(x) - f(x)g'(x)}{\lbrack g(x) \rbrack ^2} 
\end{equation*}


The Quotient Rule and the other differentiation formulas allow us to compute the derivative of any rational function.
%%%%%
%%%%%
\end{document}
