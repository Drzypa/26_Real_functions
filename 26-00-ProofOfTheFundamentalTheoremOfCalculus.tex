\documentclass[12pt]{article}
\usepackage{pmmeta}
\pmcanonicalname{ProofOfTheFundamentalTheoremOfCalculus}
\pmcreated{2013-03-22 13:45:37}
\pmmodified{2013-03-22 13:45:37}
\pmowner{paolini}{1187}
\pmmodifier{paolini}{1187}
\pmtitle{proof of the fundamental theorem of calculus}
\pmrecord{10}{34463}
\pmprivacy{1}
\pmauthor{paolini}{1187}
\pmtype{Proof}
\pmcomment{trigger rebuild}
\pmclassification{msc}{26-00}

% this is the default PlanetMath preamble.  as your knowledge
% of TeX increases, you will probably want to edit this, but
% it should be fine as is for beginners.

% almost certainly you want these
\usepackage{amssymb}
\usepackage{amsmath}
\usepackage{amsfonts}

% used for TeXing text within eps files
%\usepackage{psfrag}
% need this for including graphics (\includegraphics)
%\usepackage{graphicx}
% for neatly defining theorems and propositions
%\usepackage{amsthm}
% making logically defined graphics
%%%\usepackage{xypic}

% there are many more packages, add them here as you need them

% define commands here
\begin{document}
%We prove the following result.
%
%Let $f\colon[a,b]\to \mathbf R$ be a continuous function, let $c\in [a,b]$ and %consider the integral function
%\[
%  F(x)=\int_c^x f(t) \, dt.
%\]
%Then $F$ is differentiable and
%\[
%  F'(x)=f(x)\quad \forall x\in[a,b]
%\]
%
%\emph{Proof}.

Recall that a continuous function is Riemann integrable on every interval $[c,x]$, so the integral
\[
  F(x) = \int_c^x f(t)\, dt
\]
is well defined.

Consider the increment of $F$:
\[
   F(x+h)-F(x) = \int_c^{x+h} f(t)\, dt - \int_c^x f(t)\, dt
  = \int_x^{x+h} f(t)\, dt
\] 
(we have used the linearity of the integral with respect to the function and the additivity with respect to the domain).

Since $f$ is continuous, by the mean-value theorem, there exists $\xi_h\in [x,x+h]$ such that $f(\xi_h) = \frac{F(x+h)-F(x)}{h}$ so that
\[
  F'(x)= \lim_{h\to 0} \frac{ F(x+h)-F(x)}{h} = \lim_{h\to 0} f(\xi_h) = f(x)
\]
since $\xi_h\to x$ as $h\to 0$.
This proves the first part of the theorem.

For the second part suppose that $G$ is any antiderivative of $f$, i.e.\ $G'=f$.
Let $F$ be the integral function
\[
  F(x)= \int_a^x f(t) \, dt.
\]
We have just proven that $F'=f$. So $F'(x)=G'(x)$ for all $x\in [a,b]$ or, which is the same, $(G-F)'=0$. This means that $G-F$ is 
constant on $[a,b]$ that is, there exists $k$ such that $G(x)=F(x)+k$. Since $F(a)=0$ we have $G(a)=k$ and hence $G(x)=F(x)+G(a)$ for all $x\in[a,b]$.
Thus
\[
  \int_a^b f(t)\, dt = F(b) = G(b) - G(a).
\]
%%%%%
%%%%%
\end{document}
