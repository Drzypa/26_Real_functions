\documentclass[12pt]{article}
\usepackage{pmmeta}
\pmcanonicalname{DerivativeOfXn}
\pmcreated{2013-03-22 15:50:10}
\pmmodified{2013-03-22 15:50:10}
\pmowner{Algeboy}{12884}
\pmmodifier{Algeboy}{12884}
\pmtitle{derivative of $x^n$}
\pmrecord{16}{37811}
\pmprivacy{1}
\pmauthor{Algeboy}{12884}
\pmtype{Theorem}
\pmcomment{trigger rebuild}
\pmclassification{msc}{26B05}
\pmclassification{msc}{26A24}
\pmsynonym{Power rule}{DerivativeOfXn}
\pmrelated{DerivativesByPureAlgebra}
\pmrelated{AlternativeProofOfDerivativeOfXn}

\usepackage{latexsym}
\usepackage{amssymb}
\usepackage{amsmath}
\usepackage{amsfonts}
\usepackage{amsthm}

%%\usepackage{xypic}

%-----------------------------------------------------

%       Standard theoremlike environments.

%       Stolen directly from AMSLaTeX sample

%-----------------------------------------------------

%% \theoremstyle{plain} %% This is the default

\newtheorem{thm}{Theorem}

\newtheorem{coro}[thm]{Corollary}

\newtheorem{lem}[thm]{Lemma}

\newtheorem{lemma}[thm]{Lemma}

\newtheorem{prop}[thm]{Proposition}

\newtheorem{conjecture}[thm]{Conjecture}

\newtheorem{conj}[thm]{Conjecture}

\newtheorem{defn}[thm]{Definition}

\newtheorem{remark}[thm]{Remark}

\newtheorem{ex}[thm]{Example}



%\countstyle[equation]{thm}



%--------------------------------------------------

%       Item references.

%--------------------------------------------------


\newcommand{\exref}[1]{Example-\ref{#1}}

\newcommand{\thmref}[1]{Theorem-\ref{#1}}

\newcommand{\defref}[1]{Definition-\ref{#1}}

\newcommand{\eqnref}[1]{(\ref{#1})}

\newcommand{\secref}[1]{Section-\ref{#1}}

\newcommand{\lemref}[1]{Lemma-\ref{#1}}

\newcommand{\propref}[1]{Prop\-o\-si\-tion-\ref{#1}}

\newcommand{\corref}[1]{Cor\-ol\-lary-\ref{#1}}

\newcommand{\figref}[1]{Fig\-ure-\ref{#1}}

\newcommand{\conjref}[1]{Conjecture-\ref{#1}}


% Normal subgroup or equal.

\providecommand{\normaleq}{\unlhd}

% Normal subgroup.

\providecommand{\normal}{\lhd}

\providecommand{\rnormal}{\rhd}
% Divides, does not divide.

\providecommand{\divides}{\mid}

\providecommand{\ndivides}{\nmid}


\providecommand{\union}{\cup}

\providecommand{\bigunion}{\bigcup}

\providecommand{\intersect}{\cap}

\providecommand{\bigintersect}{\bigcap}
\begin{document}
Recall the typical derivative formula is
\[ \frac{df}{dx}=\lim_{h\rightarrow 0}\frac{f(x+h)-f(x)}{h}.\]
The derivative of $x^n$ can be computed directly from this formula utilizing the binomial theorem, see for instance \PMlinkname{an alternative proof of the deriviative of $x^n$}{AlternativeProofOfDerivativeOfXn}.  

However, to avoid invoking the binomial theorem one can often make use of alternative definitions of the derivative which are justified by inspecting a diagrams and/or through the use of algebra.  Instead of using $h$, use $h=x-a$ so that $a\rightarrow x$ is the same  as $h\rightarrow 0$.  This gives the formula
\[ \frac{df}{dx}=\lim_{a\rightarrow x}\frac{f(x)-f(a)}{x-a}.\]
This is the standard slope formula between the two points $(a,f(a))$ and $(x,f(x))$ only now we let $a$ approach $x$.

From this formula the casual rule 
\[\frac{d}{dx}(x^n)=nx^{n-1}\]
for positive integer values of $n$ can be easily derived.

First notice that 
\[(x-a)(x^{n-1}+x^{n-2}a+\cdots + x a^{n-2}+ a^{n-1})=x^n - a^n.\]
Therefore
\begin{equation*}
\frac{d}{dx}(x^n)=\lim_{a\rightarrow x} \frac{x^n-a^n}{x-a}
=\lim_{a\rightarrow x} (x^{n-1}+x^{n-2}a+\cdots + xa^{n-2}+ a^{n-1})
=nx^{n-1}.
\end{equation*}

When $n$ is not a positive integer the proof typically depends on implicit differentiation as follows:
\[y=x^n;\quad \ln y=\ln x^n=n\ln x;\quad \frac{y'}{y}=n\frac{1}{x};\quad
   y'=n\frac{y}{x}=nx^{n-1}.\]
For the theoretically inclined this solution can be disappointing because it depends on a proof for the derivative of $\ln x$.  Most texts simply redefine $\ln x$ as the integral of $1/x$ or in some similar fashion delay an honest proof.  

For this reason it is often instructive to prove the power rule in stages depending on the type of exponents.  Having proven the result for $n$ a positive integer, one can extend this to $-n$ using the product rule.

To begin with observe $1=x^n x^{-n}$.  Therefore
\begin{eqnarray*}
0 & = & \frac{d}{dx}(1)=\frac{d}{dx}(x^n x^{-n})=\frac{d}{dx}(x^n) x^{-n}+x^n \frac{d}{dx}(x^{-n})\\
& = &
nx^{n-1} x^{-n}+x^n\frac{d}{dx}(x^{-n})=\frac{n}{x}+x^n\frac{d}{dx}(x^{-n}).
\end{eqnarray*}
Now solve for $\frac{d}{dx}(x^{-n})$.
\[\frac{d}{dx}(x^{-n})=-\frac{n}{x}\frac{1}{x^n}=(-n)x^{(-n)-1}.\]

Likewise fractional powers can also be accommodated without the use of $\ln x$ by beginning with the property: given $1/b$ a rational number then
\[ x= (x^{1/b})^{b}.\]
Using the chain rule we can prove the power rule for $x^{1/b}$ as follows.
\[
1 = \frac{d}{dx}(x) = \frac{d}{dx}((x^{1/b})^b)
 = b(x^{1/b})^{b-1} \frac{d}{dx}(x^{1/b}).
\]
Once again solve for $\frac{d}{dx}(x^{1/b})$
\[
\frac{d}{dx}(x^{1/b})=\frac{1}{b (x^{1/b})^{b-1}}=\frac{1}{b} x^{1/b} x^{-1}
=\frac{1}{b} x^{1/b-1}.
\]

Finally, the derivative of $x^{a/b}$ for any fraction $a/b$ is done by observing that $x^{a/b}=(x^a)^{1/b}$ so indeed the chain rule once again solves the problem.
%%%%%
%%%%%
\end{document}
