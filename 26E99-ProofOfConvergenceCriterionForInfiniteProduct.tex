\documentclass[12pt]{article}
\usepackage{pmmeta}
\pmcanonicalname{ProofOfConvergenceCriterionForInfiniteProduct}
\pmcreated{2013-03-22 15:35:36}
\pmmodified{2013-03-22 15:35:36}
\pmowner{cvalente}{11260}
\pmmodifier{cvalente}{11260}
\pmtitle{proof of convergence criterion for infinite product}
\pmrecord{10}{37506}
\pmprivacy{1}
\pmauthor{cvalente}{11260}
\pmtype{Proof}
\pmcomment{trigger rebuild}
\pmclassification{msc}{26E99}
%\pmkeywords{series}
%\pmkeywords{convergent}
%\pmkeywords{limit}
%\pmkeywords{logarithm}
%\pmkeywords{product}
%\pmkeywords{limit comparison test}

% this is the default PlanetMath preamble.  as your knowledge
% of TeX increases, you will probably want to edit this, but
% it should be fine as is for beginners.

% almost certainly you want these
\usepackage{amssymb}
\usepackage{amsmath}
\usepackage{amsfonts}

% used for TeXing text within eps files
%\usepackage{psfrag}
% need this for including graphics (\includegraphics)
%\usepackage{graphicx}
% for neatly defining theorems and propositions
%\usepackage{amsthm}
% making logically defined graphics
%%%\usepackage{xypic}

% there are many more packages, add them here as you need them

% define commands here
\begin{document}
Consider the partial product $P_n = \prod_{i=1}^n p_i$.

By definition we say that the infinite product $\prod_{n=1}^\infty p_n$ is convergent iff $P_n$ is convergent.

Suppose every $p_n>0$

$\ln$ is a continuous bijection from $\mathbb{R}^+$ to $\mathbb{R}$, therefore
$\lim_{n\to \infty} a_n = a \iff \lim_{n\to \infty} \ln(a_n) = \ln(a)$, provided $a_n>0$ and $a>0$.

so saying $P_n\to P > 0$ is equivalent to saying that $\ln(P_n)$ converges.

Since $\ln(P_n) = \ln(\prod_{i=1}^n p_i) = \sum_{i=1}^n \ln(p_i)$, the infinite product converges to a positive value iff the series $\sum_{n=1}^\infty \ln(p_n)$ is convergent.

In particular, if the infinite product converges to a positive value, then $\ln(p_n)\to 0 \implies p_n \to 1$.

$P_n \to 0$, is equivalent to saying $\sum_{n=1}^\infty \ln(p_n) = -\infty$

For the second part of the theorem:

$\prod_{n=1}^\infty (1+p_n)$ converges absolutely to a positive value iff $\sum_{n=1}^\infty p_n$ converges absolutely.

as we have seen, $1+p_n \to 1 \implies p_n \to 0$

consider: $\lim_{x\to 0} \frac{\ln(1+x)}{x} = 1$ (this is easy to prove since by Taylor's expansion $\ln(1+x) = x + O(x^2)$)

Since $p_n \to 0$ we can say that $\lim_{n \to \infty} \frac{\ln(1+p_n)}{p_n} = 1$ and by the limit comparison test, either both $\sum_{n=1}^\infty \ln(1+p_n)$ and $\sum_{i=1}^n p_i$ converge or diverge.
%%%%%
%%%%%
\end{document}
