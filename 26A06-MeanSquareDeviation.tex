\documentclass[12pt]{article}
\usepackage{pmmeta}
\pmcanonicalname{MeanSquareDeviation}
\pmcreated{2013-03-22 18:21:57}
\pmmodified{2013-03-22 18:21:57}
\pmowner{pahio}{2872}
\pmmodifier{pahio}{2872}
\pmtitle{mean square deviation}
\pmrecord{8}{41005}
\pmprivacy{1}
\pmauthor{pahio}{2872}
\pmtype{Definition}
\pmcomment{trigger rebuild}
\pmclassification{msc}{26A06}
\pmclassification{msc}{41A99}
\pmclassification{msc}{26A42}
\pmsynonym{mean squared error}{MeanSquareDeviation}
\pmrelated{Variance}
\pmrelated{RmsError}
\pmrelated{AverageValueOfFunction}

% this is the default PlanetMath preamble.  as your knowledge
% of TeX increases, you will probably want to edit this, but
% it should be fine as is for beginners.

% almost certainly you want these
\usepackage{amssymb}
\usepackage{amsmath}
\usepackage{amsfonts}

% used for TeXing text within eps files
%\usepackage{psfrag}
% need this for including graphics (\includegraphics)
%\usepackage{graphicx}
% for neatly defining theorems and propositions
 \usepackage{amsthm}
% making logically defined graphics
%%%\usepackage{xypic}

% there are many more packages, add them here as you need them

% define commands here

\theoremstyle{definition}
\newtheorem*{thmplain}{Theorem}

\begin{document}
If $f$ is a Riemann integrable real function on the interval \,$[a,\,b]$ which is wished to be approximated by another function $\varphi$ with the same property, then the \PMlinkname{mean}{MeanValueTheorem}
$$m \;=\; \frac{1}{b\!-\!a}\int_a^b[f(x)\!-\!\varphi(x)]^2\,dx$$
is called the {\em mean square deviation} of $\varphi$ from $f$.

For example, if\, $\sin{x}$\, is approximated by $x$ on\, $[0,\,\frac{\pi}{2}]$, the mean square deviation is 
$$\frac{2}{\pi}\int_0^{\frac{\pi}{2}}(\sin{x}-x)^2\,dx \,\approx\, 0.04923.$$
%%%%%
%%%%%
\end{document}
