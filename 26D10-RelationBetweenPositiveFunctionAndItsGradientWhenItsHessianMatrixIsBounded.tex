\documentclass[12pt]{article}
\usepackage{pmmeta}
\pmcanonicalname{RelationBetweenPositiveFunctionAndItsGradientWhenItsHessianMatrixIsBounded}
\pmcreated{2013-03-22 15:53:10}
\pmmodified{2013-03-22 15:53:10}
\pmowner{Andrea Ambrosio}{7332}
\pmmodifier{Andrea Ambrosio}{7332}
\pmtitle{relation between positive function and its gradient when its Hessian matrix is bounded}
\pmrecord{16}{37887}
\pmprivacy{1}
\pmauthor{Andrea Ambrosio}{7332}
\pmtype{Theorem}
\pmcomment{trigger rebuild}
\pmclassification{msc}{26D10}

% this is the default PlanetMath preamble.  as your knowledge
% of TeX increases, you will probably want to edit this, but
% it should be fine as is for beginners.

% almost certainly you want these
\usepackage{amssymb}
\usepackage{amsmath}
\usepackage{amsfonts}

% used for TeXing text within eps files
%\usepackage{psfrag}
% need this for including graphics (\includegraphics)
%\usepackage{graphicx}
% for neatly defining theorems and propositions
\usepackage{amsthm}
% making logically defined graphics
%%%\usepackage{xypic}

% there are many more packages, add them here as you need them

% define commands here
\begin{document}
Let $f:R^{n}\rightarrow R$ a positive function, twice
differentiable everywhere. Furthermore, let $\left\Vert \mathbf{H}_{f}(%
\mathbf{x})\right\Vert _{2}\leq M,M>0$ $\forall \mathbf{x}\in R^{n}$, where $%
\mathbf{H}_{f}(\mathbf{x})$ is the Hessian matrix of $f(\mathbf{x})$.
Then, for any $\mathbf{x}\in R^{n}$,
\[
\left\Vert \nabla f(\mathbf{x})\right\Vert _{2}\leq \sqrt{2Mf(\mathbf{x})}
\]

\begin{proof}
Let $\mathbf{x},$ $\mathbf{x}_{0}\in R^{n}$ be arbitrary points. By
positivity of $f(\mathbf{x})$, writing Taylor expansion of $f(\mathbf{x})$
with Lagrange error formula around $\mathbf{x}_{0}$, a point $\mathbf{c}\in
R^{n}$ exists such that:
\begin{eqnarray*}
0 &\leq &f(\mathbf{x}) \\
&=&f(\mathbf{x}_{0})+\nabla f(\mathbf{x}_{0})\cdot (\mathbf{x}-\mathbf{x}%
_{0})+\frac{1}{2}(\mathbf{x}-\mathbf{x}_{0})^{T}\cdot \mathbf{H}_{f}(\mathbf{%
c})\cdot (\mathbf{x}-\mathbf{x}_{0}) \\
&=&\left\vert f(\mathbf{x}_{0})+\nabla f(\mathbf{x}_{0})\cdot (\mathbf{x}-%
\mathbf{x}_{0})+\frac{1}{2}(\mathbf{x}-\mathbf{x}_{0})^{T}\cdot \mathbf{H}%
_{f}(\mathbf{c})\cdot (\mathbf{x}-\mathbf{x}_{0})\right\vert  \\
&\leq &f(\mathbf{x}_{0})+\left\vert \nabla f(\mathbf{x}_{0})\cdot (\mathbf{x}%
-\mathbf{x}_{0})\right\vert +\frac{1}{2}\left\vert (\mathbf{x}-\mathbf{x}%
_{0})^{T}\cdot \mathbf{H}_{f}(\mathbf{c})\cdot (\mathbf{x}-\mathbf{x}%
_{0})\right\vert  \\
&\leq &f(\mathbf{x}_{0})+\left\Vert \nabla f(\mathbf{x}_{0})\right\Vert
_{2}\left\Vert \mathbf{x}-\mathbf{x}_{0}\right\Vert _{2}+\frac{1}{2}%
\left\Vert \mathbf{H}_{f}(\mathbf{c})\right\Vert _{2}\left\Vert \mathbf{x}-%
\mathbf{x}_{0}\right\Vert _{2}^{2}\text{\ \ \ \ \ (by Cauchy-Schwartz
inequality)} \\
&\leq &f(\mathbf{x}_{0})+\left\Vert \nabla f(\mathbf{x}_{0})\right\Vert
_{2}\left\Vert \mathbf{x}-\mathbf{x}_{0}\right\Vert _{2}+\frac{1}{2}%
M\left\Vert \mathbf{x}-\mathbf{x}_{0}\right\Vert _{2}^{2}
\end{eqnarray*}
The rightest side is a second degree polynomial in variable $\left\Vert 
\mathbf{x}-\mathbf{x}_{0}\right\Vert _{2}$; for it to be positive for any
choice of $\left\Vert \mathbf{x}-\mathbf{x}_{0}\right\Vert _{2}$ (that is,
for any choice of $\mathbf{x}$), the discriminant
\[
\left\Vert \nabla f(\mathbf{x}_{0})\right\Vert _{2}^{2}-4\cdot \frac{1}{2}Mf(%
\mathbf{x}_{0})
\]
must be negative, whence the thesis.
\end{proof}

Note: The condition on the boundedness of the Hessian matrix is actually needed. In fact, in the Lagrange form remainder, the constant $\mathbf{c}$ depends upon the point $\mathbf{x}$. Thus, if we couldn't rely on the condition $\left\Vert \mathbf{H}_{f}(\mathbf{x})\right\Vert _{2}\leq M$, we could only state
$f(\mathbf{x}_{0})+\left\Vert \nabla f(\mathbf{x}_{0})\right\Vert
_{2}\left\Vert \mathbf{x}-\mathbf{x}_{0}\right\Vert _{2}+\frac{1}{2}%
\left\Vert \mathbf{H}_{f}(\mathbf{c(\mathbf{x})})\right\Vert _{2}\left\Vert \mathbf{x}-%
\mathbf{x}_{0}\right\Vert _{2}^{2}\geq 0$
which, not being a second degree polynomial, wouldn't imply any particular further condition.
Moreover, in the case $n=1$, the lemma assumes the simpler form:
Let $f:\mathbb{R}\rightarrow \mathbb{R}$ a positive function, twice differentiable everywhere.
Furthermore, let $f^{\prime \prime }(x)\leq M,M>0$ $\forall x\in \mathbb{R}$. Then,
for any $x\in \mathbb{R}$,
$\left\vert f^{\prime }(x)\right\vert \leq \sqrt{2Mf(x)}$.
%%%%%
%%%%%
\end{document}
