\documentclass[12pt]{article}
\usepackage{pmmeta}
\pmcanonicalname{EvenAndOddFunctions}
\pmcreated{2013-03-22 13:34:19}
\pmmodified{2013-03-22 13:34:19}
\pmowner{yark}{2760}
\pmmodifier{yark}{2760}
\pmtitle{even and odd functions}
\pmrecord{12}{34187}
\pmprivacy{1}
\pmauthor{yark}{2760}
\pmtype{Definition}
\pmcomment{trigger rebuild}
\pmclassification{msc}{26A06}
\pmrelated{HermitianFunction}
\pmdefines{even function}
\pmdefines{odd function}

\endmetadata

\usepackage{amssymb}
\usepackage{amsmath}
\usepackage{amsfonts}

\newcommand{\sR}[0]{\mathbb{R}}


\begin{document}
\PMlinkescapeword{even}
\PMlinkescapeword{odd}

\section*{Definition}
Let $f$ be a function from $\sR$ to $\sR$. 
If $f(-x)=f(x)$ for all $x\in \sR$, 
then $f$ is an \emph{even function}.
Similarly, 
if $f(-x)=-f(x)$ for all $x\in \sR$, 
then $f$ is an \emph{odd function}.

Although this entry is mainly concerned with functions $\sR\to\sR$,
the definition can be generalized to other types of function.

\section*{Notes}
A real function is even if and only if it is symmetric about the $y$-axis.
It is odd if and only if symmetric about the origin.

\section*{Examples}
\begin{enumerate}
\item The function $f(x)=x$ is odd.
\item The function $f(x)=|x|$ is even.
\item The sine and cosine functions are odd and even, respectively.
\end{enumerate}

\section*{Properties}
\begin{enumerate}
\item
The only function that is both even and odd
is the function defined by $f(x)=0$ for all real $x$.
\item
A sum of even functions is even, and a sum of odd functions is odd.
In fact, the even functions form a real vector space,
as do the odd functions.
\item
Every real function can be expressed in a unique way
as the sum of an odd function and an even function.
\item
From the above it follows that
the vector space of real functions is the direct sum of 
the vector space of even functions and the vector space of odd functions.
See the entry
\PMlinkname{direct sum of even/odd functions (example)}{DirectSumOfEvenoddFunctionsExample}.)
\item 
Let $f\colon\mathbb{R}\to \mathbb{R}$ be a differentiable function.
\begin{enumerate}
\item
If $f$ is an even function, then the derivative $f'$ is an odd function.
\item
If $f$ is an odd function, then the derivative $f'$ is an even function.
\end{enumerate}
(For a proof, see the entry
\PMlinkname{derivative of even/odd function (proof)}{DerivativeOfEvenoddFunctionProof}.)
\item Let $f\colon\sR\to \sR$ be a smooth function.
Then there exist smooth functions $g,h\colon\sR\to\sR$ such that
\[
  f(x) = g(x^2)+ xh(x^2)
\]
for all $x\in \sR$.
Thus, if $f$ is even, we have $f(x)=g(x^2)$,
and if $f$ is odd, we have $f(x)=xh(x^2)$.
(\cite{hormander}, Exercise 1.2)
\item
The Fourier transform of a real even function is purely real and even.
The Fourier transform of a real odd function is purely imaginary and odd.
\end{enumerate}

\begin{thebibliography}{9}
\bibitem{hormander}
L. H\"ormander, \emph{The Analysis of Linear Partial Differential Operators I,
(Distribution theory and Fourier Analysis)}, 2nd ed, Springer-Verlag, 1990.
 \end{thebibliography}
%%%%%
%%%%%
\end{document}
