\documentclass[12pt]{article}
\usepackage{pmmeta}
\pmcanonicalname{AreaFunctions}
\pmcreated{2013-03-22 14:21:18}
\pmmodified{2013-03-22 14:21:18}
\pmowner{pahio}{2872}
\pmmodifier{pahio}{2872}
\pmtitle{area functions}
\pmrecord{38}{35834}
\pmprivacy{1}
\pmauthor{pahio}{2872}
\pmtype{Definition}
\pmcomment{trigger rebuild}
\pmclassification{msc}{26A09}
\pmsynonym{inverse hyperbolic functions}{AreaFunctions}
\pmrelated{UnitHyperbola}
\pmrelated{CyclometricFunctions}
\pmrelated{HyperbolicAngle}
\pmrelated{IntegralTables}
\pmrelated{IntegrationOfSqrtx21}
\pmrelated{IntegralRelatedToArcSine}
\pmrelated{ArcLengthOfParabola}
\pmrelated{ListOfImproperIntegrals}
\pmrelated{InverseGudermannianFunction}
\pmrelated{EulersSubstitutionsForIntegration}
\pmrelated{ArcoshCurve}
\pmrelated{EqualArcLength}
\pmdefines{arsinh}
\pmdefines{arcosh}
\pmdefines{artanh}
\pmdefines{arcoth}

% this is the default PlanetMath preamble.  as your knowledge
% of TeX increases, you will probably want to edit this, but
% it should be fine as is for beginners.

% almost certainly you want these
\usepackage{amssymb}
\usepackage{amsmath}
\usepackage{amsfonts}

% used for TeXing text within eps files
%\usepackage{psfrag}
% need this for including graphics (\includegraphics)
%\usepackage{graphicx}
% for neatly defining theorems and propositions
%\usepackage{amsthm}
% making logically defined graphics
%%%\usepackage{xypic}

% there are many more packages, add them here as you need them

\DeclareMathOperator{\arsinh}{arsinh}
\DeclareMathOperator{\arcosh}{arcosh}
\DeclareMathOperator{\artanh}{artanh}
\DeclareMathOperator{\arcoth}{arcoth}
\begin{document}
The most usual {\em area functions}:

\begin{itemize}
 \item The inverse function of the hyperbolic sine (in Latin {\em sinus hyperbolicus}) is $\arsinh$ ({\em area sini hyperbolici}):
          $$\arsinh{x} := \ln{(x+\sqrt{x^2+1})}$$

 \item The inverse function of the hyperbolic cosine (in Latin {\em cosinus hyperbolicus}) is $\arcosh$ ({\em area cosini hyperbolici}):
          $$\arcosh{x} := \ln(x+\sqrt{x^2-1})$$
It is defined for\, $x \geqq 1$.

 \item The inverse function of the hyperbolic tangent (in Latin {\em tangens hyperbolica}) is $\artanh$ ({\em area tangentis hyperbolicae}):
          $$\artanh{x} := \frac{1}{2}\ln \frac{1+x}{1-x}$$
It is defined for\, $-1 < x < 1$.

\item The inverse function of the hyperbolic cotangent (in Latin {\em cotangens hyperbolica}) is $\arcoth$ ({\em area cotangentis hyperbolicae}):
          $$\arcoth{x} := \frac{1}{2}\ln \frac{x+1}{x-1}$$
It is defined for\, $|x| > 1$.

\end{itemize}

These four functions are denoted also by $\sinh^{-1}x$, $\cosh^{-1}x$, $\tanh^{-1}x$ and $\coth^{-1}x$.

Derivatives:
    $$\frac{d}{dx} \arsinh x = \frac{1}{\sqrt{x^2\!+\!1}}$$
    $$\frac{d}{dx} \arcosh x = \frac{1}{\sqrt{x^2\!-\!1}}$$
    $$\frac{d}{dx} \artanh x = \frac{1}{1\!-\!x^2}$$
    $$\frac{d}{dx} \arcoth x = \frac{1}{1\!-\!x^2}$$

The functions\, $\arsinh$\, and\, $\artanh$\, have the \PMlinkescapetext{simple} Taylor series
   $$\arsinh{x} = x-\frac{1}{2}\!\cdot\!\frac{x^3}{3}
+\frac{1\!\cdot\!3}{2\!\cdot\!4}\!\cdot\!\frac{x^5}{5}
-\frac{1\!\cdot\!3\!\cdot\!5}{2\!\cdot\!4\cdot\!6}\!\cdot\!\frac{x^7}{7}
+-\cdots\quad (|x|\leqq 1),$$
   $$\artanh x = x+\frac{x^3}{3}+\frac{x^5}{5}+\frac{x^7}{7}+\cdots 
\quad (|x| < 1).$$
Because the inverse tangent function (see the cyclometric functions) has the \PMlinkescapetext{expansion}
\, $\arctan x = x-\frac{x^3}{3}+\frac{x^5}{5}-\frac{x^7}{7}+-\cdots\,\, 
(|x|\leqq 1)$,
 we see that
   $$\artanh x = \frac{1}{i}\arctan ix;$$
similarly we get
   $$\arsinh x = \frac{1}{i}\arcsin ix.$$ 
Some other formulae which may be obtained by means of the addition formulae of the hyperbolic functions:
  $$\arsinh x\pm\arsinh y = \arsinh(x\sqrt{y^2\!+\!1}\pm y\sqrt{x^2\!+\!1})$$
  $$\arcosh x\pm\arcosh y = \arcosh(xy\pm\sqrt{x^2\!-\!1}\sqrt{y^2\!-\!1})$$
  $$\artanh x\pm\artanh y = \artanh\frac{x\pm y}{1\pm xy}$$

The classic abbreviations ``$\arsinh$'' and ``$\arcosh$'' are explained as follows:\, The unit hyperbola\, $x^2\!-\!y^2 = 1$\,(its right half) has the parametric \PMlinkescapetext{representation}
\[\begin{cases}       
        x = \cosh A,\\
        y = \sinh A;
\end{cases}\]
here $A$ means the area \PMlinkescapetext{bounded} by the hyperbola and the straight line segments $OP$ and $OQ$, where $O$ is the origin, $P$ is the point \,$(x,\,y)$\, of the hyperbola and $Q$ is the point\, $(x,\,-y)$\, of the hyperbola.\, Thus, conversely, $A$ is the area having hyperbolic cosine equal to $x$ ({\em area cosini hyperbolici x}), similarly $A$ is the area having hyperbolic sine equal to $y$ ({\em area sini hyperbolici y}).

{\bf Note.}\, In some countries the abbreviation ``ar'' in the symbols arsinh etc. is replaced by\, ``a'', ``Ar'', ``arc'' or ``arg''.
%%%%%
%%%%%
\end{document}
