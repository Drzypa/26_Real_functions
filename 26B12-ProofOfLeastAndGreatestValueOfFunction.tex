\documentclass[12pt]{article}
\usepackage{pmmeta}
\pmcanonicalname{ProofOfLeastAndGreatestValueOfFunction}
\pmcreated{2013-03-22 15:52:09}
\pmmodified{2013-03-22 15:52:09}
\pmowner{cvalente}{11260}
\pmmodifier{cvalente}{11260}
\pmtitle{proof of least and greatest value of function}
\pmrecord{5}{37865}
\pmprivacy{1}
\pmauthor{cvalente}{11260}
\pmtype{Proof}
\pmcomment{trigger rebuild}
\pmclassification{msc}{26B12}
\pmrelated{FermatsTheoremStationaryPoints}
\pmrelated{HeineBorelTheorem}
\pmrelated{CompactnessIsPreservedUnderAContinuousMap}

\endmetadata

% this is the default PlanetMath preamble.  as your knowledge
% of TeX increases, you will probably want to edit this, but
% it should be fine as is for beginners.

% almost certainly you want these
\usepackage{amssymb}
\usepackage{amsmath}
\usepackage{amsfonts}

% used for TeXing text within eps files
%\usepackage{psfrag}
% need this for including graphics (\includegraphics)
%\usepackage{graphicx}
% for neatly defining theorems and propositions
%\usepackage{amsthm}
% making logically defined graphics
%%%\usepackage{xypic}

% there are many more packages, add them here as you need them

% define commands here
\begin{document}
$f$ is continuous, so it will transform compact sets into compact sets.
Thus since $[a,b]$ is compact, $f([a,b])$ is also compact.
$f$ will thus attain on the interval $[a,b]$ a maximum and a minimum value because real compact sets are closed and bounded.

Consider the maximum and later use the same argument for $-f$ to consider the minimum.

By a known \PMlinkname{theorem}{FermatsTheoremStationaryPoints} if the maximum is attained in the interior of the domain, $c \in ]a,b[$ then $f(c) \text{is a maximum} \implies f'(c)=0$, since $f$ is differentiable.

If the maximum isn't attained in $]a,b[$ and since it must be attained in $[a,b]$ either $f(a)$ or $f(b)$ is a maximum.

For the minimum consider $-f$ and note that $-f$ will verify all conditions of the theorem and that a maximum of $-f$ corresponds to a minimum of $f$ and that
$-f'(c)=0 \iff f'(c)=0$.
%%%%%
%%%%%
\end{document}
