\documentclass[12pt]{article}
\usepackage{pmmeta}
\pmcanonicalname{LandauKernel}
\pmcreated{2013-03-22 14:11:38}
\pmmodified{2013-03-22 14:11:38}
\pmowner{mathwizard}{128}
\pmmodifier{mathwizard}{128}
\pmtitle{Landau kernel}
\pmrecord{7}{35624}
\pmprivacy{1}
\pmauthor{mathwizard}{128}
\pmtype{Definition}
\pmcomment{trigger rebuild}
\pmclassification{msc}{26A30}

\endmetadata

% this is the default PlanetMath preamble.  as your knowledge
% of TeX increases, you will probably want to edit this, but
% it should be fine as is for beginners.

% almost certainly you want these
\usepackage{amssymb}
\usepackage{amsmath}
\usepackage{amsfonts}

% used for TeXing text within eps files
%\usepackage{psfrag}
% need this for including graphics (\includegraphics)
%\usepackage{graphicx}
% for neatly defining theorems and propositions
%\usepackage{amsthm}
% making logically defined graphics
%%%\usepackage{xypic}

% there are many more packages, add them here as you need them

% define commands here
\begin{document}
For $k\in\mathbb{N}$ the \emph{Landau kernel} $L_k(t)$ is defined as
$$L_k=\left\{\begin{array}{lr}
\frac{1}{c_k}(1-t^2)^k&\text{if }t\in[-1,1]\\
0&\text{otherwise}
\end{array}\right.$$
with
$$c_k:=\int_{-1}^1(1-t^2)^kdt.$$
$L_k$ is nonnegative and continuous on $\mathbb{R}$. Due to the choice of $c_k$ we have
$$\int_{-\infty}^\infty L_k(t)dt=1.$$
Also we have for all positive, real $r$:
\begin{align*}
\int_{\mathbb{R}\backslash[-r,r]}L_k(t)dt&\leq\frac{2}{c_k}\int_r^1(1-t^2)^kdt\\
&\leq(k+1)(1-r^2)^k.
\end{align*}
Therefore $(L_k)_{k\in\mathbb{N}}$ is a Dirac sequence.
%%%%%
%%%%%
\end{document}
