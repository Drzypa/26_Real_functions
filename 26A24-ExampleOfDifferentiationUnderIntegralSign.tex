\documentclass[12pt]{article}
\usepackage{pmmeta}
\pmcanonicalname{ExampleOfDifferentiationUnderIntegralSign}
\pmcreated{2013-03-22 17:01:56}
\pmmodified{2013-03-22 17:01:56}
\pmowner{pahio}{2872}
\pmmodifier{pahio}{2872}
\pmtitle{example of differentiation under integral sign}
\pmrecord{4}{39318}
\pmprivacy{1}
\pmauthor{pahio}{2872}
\pmtype{Example}
\pmcomment{trigger rebuild}
\pmclassification{msc}{26A24}
\pmclassification{msc}{26B15}

\endmetadata

% this is the default PlanetMath preamble.  as your knowledge
% of TeX increases, you will probably want to edit this, but
% it should be fine as is for beginners.

% almost certainly you want these
\usepackage{amssymb}
\usepackage{amsmath}
\usepackage{amsfonts}

% used for TeXing text within eps files
%\usepackage{psfrag}
% need this for including graphics (\includegraphics)
%\usepackage{graphicx}
% for neatly defining theorems and propositions
 \usepackage{amsthm}
% making logically defined graphics
%%%\usepackage{xypic}

% there are many more packages, add them here as you need them

% define commands here

\theoremstyle{definition}
\newtheorem*{thmplain}{Theorem}

\begin{document}
Differentiation with respect to a parameter under the integral sign may sometimes yield useful formulae.\, One example is given here.

We know that the equation
$$\int_0^1x^m\,dx = \frac{1}{m+1}$$
is valid for all\, $m > -1$.\, If one differentiates with respect to $m$ \PMlinkname{under the integral sign}{DifferentiationUnderTheIntegralSign} in succession, one gets

$$\int_0^1\frac{\partial}{\partial m}e^{m\ln{x}}\,dx 
= \int_0^1 e^{m\ln{x}}\ln{x}\,dx = \int_0^1x^m\ln{x}\,dx 
= \frac{-1}{(m+1)^2}$$

$$\int_0^1\frac{\partial}{\partial m}x^m\ln{x}\,dx = \int_0^1x^m(\ln{x})^2\,dx = \frac{+1\cdot2}{(m+1)^3}$$

$$\int_0^1\frac{\partial}{\partial m}x^m(\ln{x})^2\,dx = \int_0^1x^m(\ln{x})^3\,dx = \frac{-1\cdot2\cdot3}{(m+1)^4}$$

$$\cdots$$

It's evident that repeating the differentiation $n$ times the final result is the \PMlinkescapetext{formula}
$$ \int_0^1x^m(\ln{x})^n\,dx = \frac{(-1)^n n!}{(m+1)^{n+1}} \qquad(m > -1).$$


%%%%%
%%%%%
\end{document}
