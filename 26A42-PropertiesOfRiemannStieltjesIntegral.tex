\documentclass[12pt]{article}
\usepackage{pmmeta}
\pmcanonicalname{PropertiesOfRiemannStieltjesIntegral}
\pmcreated{2013-03-22 18:54:59}
\pmmodified{2013-03-22 18:54:59}
\pmowner{pahio}{2872}
\pmmodifier{pahio}{2872}
\pmtitle{properties of Riemann--Stieltjes integral}
\pmrecord{12}{41766}
\pmprivacy{1}
\pmauthor{pahio}{2872}
\pmtype{Topic}
\pmcomment{trigger rebuild}
\pmclassification{msc}{26A42}
\pmsynonym{properties of Riemann-Stieltjes integral}{PropertiesOfRiemannStieltjesIntegral}
\pmrelated{ProductAndQuotientOfFunctionsSum}
\pmrelated{FactsAboutRiemannStieltjesIntegral}

\endmetadata

% this is the default PlanetMath preamble.  as your knowledge
% of TeX increases, you will probably want to edit this, but
% it should be fine as is for beginners.

% almost certainly you want these
\usepackage{amssymb}
\usepackage{amsmath}
\usepackage{amsfonts}

% used for TeXing text within eps files
%\usepackage{psfrag}
% need this for including graphics (\includegraphics)
%\usepackage{graphicx}
% for neatly defining theorems and propositions
 \usepackage{amsthm}
% making logically defined graphics
%%%\usepackage{xypic}

% there are many more packages, add them here as you need them

% define commands here

\theoremstyle{definition}
\newtheorem*{thmplain}{Theorem}

\begin{document}
Denote by $R(g)$ the set of bounded real functions which are \PMlinkid{Riemann--Stieltjes integrable}{3187} with respect to a given monotonically nondecreasing function $g$ on\, a given interval.

The \PMlinkid{Riemann--Stieltjes integral}{3187} is a generalisation of the Riemann integral,  and both have \PMlinkescapetext{similar} properties; N.B. however the items 5, 7 and 9.

\begin{enumerate}

\item If\, $f_1,\, f_2 \in R(g)$\, on\, $[a,\,b]$,\, then also\, $f_1\!+\!f_2,\,cf_1 \in R(g)$ on\, $[a,\,b]$\, and\\ 
$\int_a^b(f_1\!+\!f_2)dg = \int_a^bf_1\,dg+\int_a^bf_2\,dg,\;\; \int_a^bcf_1\,dg = c\int_a^bf_1\,dg$.

\item If\, $f_1,\, f_2 \in R(g)$\, on\, $[a,\,b]$, then also\, $f_1f_2 \in R(g)$ on\, $[a,\,b]$.

\item If\, $f_1,\,f_2 \in R(g)$\, on\, $[a,\,b]$\, and\, $\displaystyle\inf_{x\in[a,b]}|f_2(x)| > 0$,\, then also\, 
$\frac{f_1}{f_2} \in R(g)$\, on\, $[a,\,b]$.

\item If\, $f_1,\, f_2 \in R(g)$\, and\, $f_1 \le f_2$\, on\, $[a,\,b]$,\, then\\
$\int_a^bf_1\,dg \le \int_a^bf_2\,dg$.

\item If\, $f \in R(g)$\, on\, $[a,\,b]$,\, and $V_g$ is the total variation of $g$ on\, $[a,\,b]$,\, then\\
$\left|\int_a^bfdg\right| \le$ $\displaystyle\sup_{x\in[a,b]}f(x)\cdot V_g$.

\item If\, $f \in R(g)$\, on\, $[a,\,b]$,\, then also\, $|f| \in R(g)$ on\, $[a,\,b]$\, and\\
$\left|\int_a^bf\,dg\right| \le \int_a^b|f|\,dg$.

\item If\, $f \in R(g)$\, and\, $m \le f(x) \le M$\, on\, $[a,\,b]$,\, then\\
$m[g(b)-g(a)] \le \int_a^bf\,dg \le M[g(b)-g(a)]$.

\item If\, $f \in R(g)$\, on\, $[a,\,b]$\, and on\, $[b,\,c]$,\, then also \, $f\in R(g)$\, on\, $[a,\,c]$\, and\\
$\int_a^cf\,dg = \int_a^bf\,dg+\int_b^cf\,dg$.

\item If\, $f \in R(g_1),\,R(g_2)$\, on\, $[a,\,b]$,\, then\, $f \in R(g_1\!+\!g_2)$\, on the same interval and\\
$\int_a^bf\,d(g_1\!+\!g_2) = \int_a^bf\,dg_1+\int_a^bf\,dg_2$.

\item If\, $f \in R(g)$\, on\, $[a,\,b]$,\, then\, $g \in R(f)$\, on the same interval and one can integrate by parts:\\
$\int_a^bf\,dg \;=\; f(b)g(b)\!-\!f(a)g(a)\!-\int_a^bg\,df$.

\end{enumerate}
%%%%%
%%%%%
\end{document}
