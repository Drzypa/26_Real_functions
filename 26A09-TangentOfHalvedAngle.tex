\documentclass[12pt]{article}
\usepackage{pmmeta}
\pmcanonicalname{TangentOfHalvedAngle}
\pmcreated{2013-03-22 17:00:32}
\pmmodified{2013-03-22 17:00:32}
\pmowner{pahio}{2872}
\pmmodifier{pahio}{2872}
\pmtitle{tangent of halved angle}
\pmrecord{9}{39291}
\pmprivacy{1}
\pmauthor{pahio}{2872}
\pmtype{Derivation}
\pmcomment{trigger rebuild}
\pmclassification{msc}{26A09}
\pmrelated{DerivationOfHalfAngleFormulaeForTangent}
\pmrelated{GoniometricFormulae}

% this is the default PlanetMath preamble.  as your knowledge
% of TeX increases, you will probably want to edit this, but
% it should be fine as is for beginners.

% almost certainly you want these
\usepackage{amssymb}
\usepackage{amsmath}
\usepackage{amsfonts}

% used for TeXing text within eps files
%\usepackage{psfrag}
% need this for including graphics (\includegraphics)
%\usepackage{graphicx}
% for neatly defining theorems and propositions
 \usepackage{amsthm}
% making logically defined graphics
%%%\usepackage{xypic}

% there are many more packages, add them here as you need them

% define commands here

\theoremstyle{definition}
\newtheorem*{thmplain}{Theorem}

\begin{document}
\PMlinkescapeword{formula}
The formulae
$$\cos{2\alpha} = 1-2\sin^2{\alpha},$$
$$\cos{2\alpha} = 2\cos^2{\alpha}-1$$
may be solved for\, $\sin{\alpha}$\, and\, $\cos{\alpha}$, respectively.\, One gets the equations
$$\sin{\alpha} = \pm\sqrt{\frac{1-\cos{2\alpha}}{2}},\quad\cos{\alpha} = \pm\sqrt{\frac{1+\cos{2\alpha}}{2}},$$
where the signs have to be chosen according to the quadrant where the angle $\alpha$ is.\, Changing $\alpha$ to $\frac{x}{2}$ and dividing these equations gives us the formula
\begin{align}
\tan{\frac{x}{2}}\, = \,\pm\sqrt{\frac{1-\cos{x}}{1+\cos{x}}}.
\end{align}
Also here one must chose the sign according to the quadrant of\, $\displaystyle\frac{x}{2}$.

We obtain two alternative forms of (1) when we multiply both the numerator and
 the denominator of the radicand the first time by\, $1-\cos{x}$\, and the 
second time by\, $1+\cos{x}$; note that\, $1-\cos^2{x} = \sin^2{x}$:
\begin{align}
\tan{\frac{x}{2}} = \frac{1-\cos{x}}{\sin{x}},
\end{align}
\begin{align}
\tan{\frac{x}{2}} = \frac{\sin{x}}{1+\cos{x}}
\end{align}
Here,\, $\sin{x}$\, determines the sign of the \PMlinkescapetext{right} hand sides; 
it can be justified that it has always the same sign as $\tan{\frac{x}{2}}$.
%%%%%
%%%%%
\end{document}
