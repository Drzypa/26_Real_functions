\documentclass[12pt]{article}
\usepackage{pmmeta}
\pmcanonicalname{FundamentalTheoremOfCalculusForRiemannIntegration}
\pmcreated{2013-03-22 17:57:32}
\pmmodified{2013-03-22 17:57:32}
\pmowner{asteroid}{17536}
\pmmodifier{asteroid}{17536}
\pmtitle{fundamental theorem of calculus for Riemann integration}
\pmrecord{10}{40459}
\pmprivacy{1}
\pmauthor{asteroid}{17536}
\pmtype{Theorem}
\pmcomment{trigger rebuild}
\pmclassification{msc}{26A42}
\pmrelated{FundamentalTheoremOfCalculusClassicalVersion}
\pmrelated{FundamentalTheoremOfCalculus}
\pmdefines{first fundamental theorem of calculus (Riemann integral)}
\pmdefines{second fundamental theorem of calculus (Riemann integral)}

\endmetadata

% this is the default PlanetMath preamble.  as your knowledge
% of TeX increases, you will probably want to edit this, but
% it should be fine as is for beginners.

% almost certainly you want these
\usepackage{amssymb}
\usepackage{amsmath}
\usepackage{amsfonts}

% used for TeXing text within eps files
%\usepackage{psfrag}
% need this for including graphics (\includegraphics)
%\usepackage{graphicx}
% for neatly defining theorems and propositions
%\usepackage{amsthm}
% making logically defined graphics
%%%\usepackage{xypic}

% there are many more packages, add them here as you need them

% define commands here

\begin{document}
In this entry we discuss the fundamental theorems of calculus for Riemann integration.

{\bf \PMlinkescapetext{First Fundamental Theorem of Calculus} -} Let $f$ be a Riemann integrable function on an interval $[a,b]$ and $F$ defined in $[a,b]$ by $F(x)= \int_a^x f(t)\,dt + k$, where $k \in \mathbb{R}$ is a constant. Then $F$ is continuous in $[a,b]$ and $F'=f$ \PMlinkname{almost everywhere}{MeasureZeroInMathbbRn}.

$\,$

{\bf \PMlinkescapetext{Second Fundamental Theorem of Calculus} -} Let $F$ be a continuous function in an interval $[a,b]$ and $f$ a Riemann integrable function such that $F'(x)=f(x)$ except at most in a finite number of points $x$. Then $F(x)-F(a) = \int_a^xf(t)\,dt$.

$\,$

\subsection{Observations}
Notice that the second fundamental theorem is not a converse of the first. In the first we conclude that $F'=f$ except in a set of \PMlinkname{measure zero}{MeasureZeroInMathbbRn}, while in the second we assume that $F'=f$ except in a finite number of points. In fact, the two theorems can never be the converse of each other as the following example shows:

{\bf Example :} Let $F$ be the devil staircase function, defined on $[0,1]$. We have that
\begin{itemize}
\item $F$ is continuous in $[0,1]$,
\item $F'=0$ except in a set of \PMlinkescapetext{measure zero} (this set must be contained in the Cantor set),
\item $f:=0$ is clearly a Riemann integrable function and $\int_0^x 0\,dt = 0$.
\end{itemize}
Thus, $F(x) \neq \int_0^x F'(t)\,dt$.

This leads to the question: what kind functions $F$ can be expressed as $F(x) = F(a) + \int_a^x g(t)\,dt$, for some function $g$ ? The answer to this question lies in the concept of \PMlinkname{absolute continuity}{AbsolutelyContinuousFunction2} (a \PMlinkescapetext{property} which the devil staircase does not possess), but for that a more general \PMlinkescapetext{theory} of integration must be developed (the \PMlinkname{Lebesgue integration}{Integral2}).
%%%%%
%%%%%
\end{document}
