\documentclass[12pt]{article}
\usepackage{pmmeta}
\pmcanonicalname{ClairautsTheorem}
\pmcreated{2013-03-22 13:53:44}
\pmmodified{2013-03-22 13:53:44}
\pmowner{Mathprof}{13753}
\pmmodifier{Mathprof}{13753}
\pmtitle{Clairaut's theorem}
\pmrecord{18}{34642}
\pmprivacy{1}
\pmauthor{Mathprof}{13753}
\pmtype{Theorem}
\pmcomment{trigger rebuild}
\pmclassification{msc}{26B12}
\pmsynonym{equality of mixed partials}{ClairautsTheorem}

\endmetadata

\usepackage{amsmath}
\usepackage{amsfonts}
\usepackage{amssymb}

\usepackage{amsthm}

\newtheorem*{cthm*}{Clairaut's Theorem}
\begin{document}
\begin{cthm*}
If $\mathbf{f}\colon\mathbb{R}^n \to \mathbb{R}^m$ is a function whose second partial derivatives exist and are continuous on a set $S \subseteq \mathbb{R}^n$, then
\[
   \frac{\partial^2 f}{\partial x_i \partial x_j}
  =\frac{\partial^2 f}{\partial x_j \partial x_i}
\]
on $S$, where $1 \leq i,j \leq n$.
\end{cthm*}

This theorem is commonly referred to as \emph{the equality of mixed partials}.
It is usually first presented in a vector calculus course,
and is useful in this context for proving basic properties of the interrelations of gradient, divergence, and curl.
For example, if $\mathbf{F}\colon \mathbb{R}^3 \to \mathbb{R}^3$ is a function satisfying the hypothesis, then $\nabla \cdot (\nabla \times \mathbf{F}) =0$.
Or, if $f\colon\mathbb{R}^3 \to \mathbb{R}$ is a function satisfying the hypothesis, then $\nabla \times \nabla f= \mathbf{0}$.
%%%%%
%%%%%
\end{document}
