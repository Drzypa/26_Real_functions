\documentclass[12pt]{article}
\usepackage{pmmeta}
\pmcanonicalname{ChapterI}
\pmcreated{2014-08-03 22:49:42}
\pmmodified{2014-08-03 22:49:42}
\pmowner{PMBookProject}{1000683}
\pmmodifier{rspuzio}{6075}
\pmtitle{Chapter I}
\pmrecord{4}{87451}
\pmprivacy{1}
\pmauthor{PMBookProject}{6075}
\pmtype{Topic}
\pmclassification{msc}{26A06}

\endmetadata

% this is the default PlanetMath preamble.  as your knowledge
% of TeX increases, you will probably want to edit this, but
% it should be fine as is for beginners.

% almost certainly you want these
\usepackage{amssymb}
\usepackage{amsmath}
\usepackage{amsfonts}

% need this for including graphics (\includegraphics)
\usepackage{graphicx}
% for neatly defining theorems and propositions
\usepackage{amsthm}

% making logically defined graphics
%\usepackage{xypic}
% used for TeXing text within eps files
%\usepackage{psfrag}

% there are many more packages, add them here as you need them

% define commands here

\begin{document}
\begin{center}
CHAPTER I  
\end{center}
\begin{center}
FUNCTIONS
\end{center}

1. Dependence.

There are countless instances in which one
quantity depends upon another. The speed of a body falling
from rest depends upon the time it has fallen. One's income
from a given investment depends upon the amount invested
and the rate of interest realized. The crops depend upon rainfall, 
soil fertility and proper cultivation.
In mathematics we usually deal with quantities that are
definitely and completely determined by certain others. Thus
the area $A$ of a square is determined precisely when the length
$s$ of its side is given: $A=s^{2}$; the volume of a sphere is $\pi r^3/3$;
the force of attraction between two bodies is $ k\cdot m\cdot m'/ d^2$, where
$m$ and $m'$ are their masses, $d$. the distance between them, and $k$
a certain number given by experiment. The Calculus is the
study of the relations between such interdependent quantities,
with special reference to their rates of change.

2. Variables. Constants. Functions. A quantity which
may change is called a variable. The quantities mentioned in
\S 1, except $k$ and $\pi$, are examples of variables.

A quantity which has a fixed value is called a constant. Examples 
of constants are ordinary numbers: 1, $\sqrt{2}, -7,2/3, \pi$,
$30^{\mathrm{o}}, \log 6$, and the number $k$ in \S 1.

If one variable $y$ depends on another variable $x$, so that $y$ is
determined when $x$ is known, $y$ is said to be a function of $x$.
The variable $x$, thus thought of as determining the other, is
called the independent variable; the other variable $y$ is called
the dependent variable. Thus, in \S 1, the
area $A$ of a square is a function, $A=s^s$, of
the side $s$.

In Algebra we learn how to express such
relations by means of equations.


In Analytic Geometry such relations are
represented graphically. For example, if
the principal at simple interest is a fixed
sum $p$ and if the interest rate $r$ also is
fixed, then the amount $a$, of principal and
interest, varies solely with (is a function of)
the time $t$ that the principal has been at interest. In fact, if $p$
$=100$ and $ r=6\%$,
$$
a=p+ptr=100+6t.
$$
This is represented graphically in Fig. 1. In practice fractional
parts of a day are neglected.

The relation $A=s^{2}$ of \S 1 is represented in Fig. 2.

EXERCISES I.--FUNCTIONS AND GRAPHS

Represent graphically the following: --
1. $a=100+3t, a=300+4t, a=160+7t$.

2. The number of feet $f$ in terms of the
number of yards $y$ in a given length is given
by the equation $f=3y$.

3. The temperature in degrees Fahrenheit, F, is 32 more than 9/5 the
temperature in degrees Centigrade, C.

4. The distance $s$ that a body falls from rest in a time $t$ Is given by
$s=16t^{2}$. (Measure $t$ horizontally and $s$ vertically downward.)

5. (a)$y=x^{2}+3x+1$. $(b)y=2x^{2}-6x$.

(c) $y=x+2$. (d)$y=\frac{1}{x+1}\cdot$

(e)$y=\frac{x-1}{x+2}$. (f)$y=\frac{x^{2}+2x+3}{x-6}$.

6. The volume $v$ {\it of} a fixed quantity of gas at a constant temperature
varies inversely as the pressure $p$ upon the gas.

7. The amount of \$ 1.00 at compound interest at 10\% per annum for
t years i8 $a=(1+1/10)^t$.

8. The area $A$ of an equilateral triangle is a function of its side $s$.
Determine this function, and represent the relation graphioally. Express
the side in terms of the area.

9. Determine the area $a$ of a circle in terms of its radius $r$. Determine 
the radius in terms of the area.

10. The radius, surface, and volume of a sphere are functionally
related. Find the equations connecting each pair. Also express each of
the three as a function of the circumference of a great circle of the sphere.

11. The area $A$ bounded by the straight line $y=ax+b$, the ordinate
$y$, and the axes, is a function of $x$. Determine it; and also express $y$ as
a function of the area.

3. The Function Notation.
A very useful abbreviation for
functions consists in writing $f(x)$ (read $f$ of $x$) in place of the
given expression.

Thus if $f(x)=x^2+3x+1$, we may write $f(2)=2^{2}+3\cdot 2+1=11$, that is, 
the value of $x^2+3x+1$ when $x=2$ is 11.
Likewise $f(3)=19,f(-1)=-1, f(0)=1$, and so on. $f(a)= a^2+3a+1$.
$f(u+v)=(u+v)^{2}+3(u+v)+1$.

Other letters than $f$ are often used, to avoid confusion, but
$f$ is used most often, because it is the initial of the word {\it function}. 
Other letters than $x$ are often used for the variable. In
any case, given $f(x)$, to find $f(a)$, simply substitute $a$ for $x$ in
the given expression.

EXERCISES II. SUBSTITUTION~~FUNCTION NOTATION

1. If $f(x)=x^{2}-6x+2$ find $f(1), f(2), f(3), f(4), f(0), f(-1)$,
$f(-2)$. From these values (and others, if needed) draw the graph of
the curve $y=f(x)$. Mark its lowest point, and estimate the values of $x$
and $y$ there.

2. Proceed as in Ex. 1 for each of the following functions using the
function notation in calculating values; mark the highest and lowest
points if any exist, and estimate the values of $x$ and $y$ at these points.


(a)$d-2x+4$. (b)$3x^{2}-2x+1$. (c) $\displaystyle \frac{x+1}{2x-3}$. (d)$\frac{1}{x+1}+\frac{2}{x-1}$.

$(e)y=\sin x$, taking $x=\pi/6, \pi/4, \pi/2,3\pi/4, \pi, 0, -\pi/2$.

$(f)y=\log_{10}x$, taking $ x=1,2,10,1/10,1/100$.

3. If $f(x)=x^4-6x^3+3x^2-2x+3$, calculate $f(1), f(4), f(6)$.
Hence show that one solution of the equation $f(x)=0$ is $x=1$; and
that another solution lies between 4 and 5.

(This work is simplified by using the theorem that $f(a)$ is equal to the
remainder obtained by dividing $f(x)$ by $(x-a)$; and by using synthetic
division.)

4. If $f(x)=2x^{2}-3x+5$, show that $f(a)=2a^2-3a+5$, 
$f(m+n)=2(m+n)^{2}-3(m+n)+5$; find $f(a-b), f(a+2b), f(a/b)$.

5. If $f(x)=x^2+3$ and $\phi(x)=3x+1$, show that $f(1)=\phi(1)$ and
$f(2)=\phi(2)$. Show that $f(3)&gt;\phi(3)$. Draw $y=f(x)$ and $y=\phi(x)$.

6. In Ex. 5, draw the curve $y=f(x)-\phi(x)$. Mark the points
where $f(x)-\phi(x)=0$. Mark the lowest point.

7. If $f(x)=-2x^{2}+1$ and $\phi(x)=x^{2}+2x+4$, find the value for
which $f(x)=\phi(x)$ by use of $f(x)-\phi(x)$. Sketch all of the curves
$y=f(x), y=\phi(x), y=f(x)-\phi(x)$.

8. If $f(x)=\sin x$ and $\phi(x)=\cos x$, show that $[f(x)]^{2}+[\phi(x)]^{2} =1$ ;
$f(x)\div\phi(x)=\tan x;f(x+y)=f(x)\phi(y)+f(y)\phi(x);\phi(x+y)=?;
f(x)=\phi(\pi/2-x);\phi(x)=f(\pi/2-x)=-\phi(\pi-x);f(-x)=-f(+x);\phi(-x)=\phi(x)$.

9. If $f(x)=\log_{10}x$, show that
$$
f(x)+f(y)=f(x\cdot y);\ f(x^{2})=2f(x);
$$
$f(m/n)-f(n/m)=2f(m)-2f(n)$; $f(m/n)+f(n/m)=0$.

10. If $f(x)=\tan x, \phi(x)=\cos x$, draw the curves $y=f(x), y=\phi(x)$,
$y=f(x)-\phi(x)$. Mark the points where $f(x)=\phi(x)$ and estimate the
values of $x$ and $y$ there.

11. Taking $f(x)=x^{2}$, compare the graph of $y=f(x)$ with that of
$y=f(x)+1$, and with that of $y=f(x+1)$.

12. Taking any two curves $y=f(x), y=\phi(x)$, how can you most
easily draw $y=f(x)-\phi(x)? y=f(x)+\phi(x)$ ? Draw $y=x^{2}+1/x$.


13. How can you most easily draw $y=f(x)+ 5$ ? $y=f(x+5)$ ? 
assuming that $y=f(x)$ is drawn.

l4. Draw $y=x^{2}$ and show how to deduce from it the graph of
$y=2x^{2}$; the graph of $y=-x^{2}$.

Assuming that $y=f(x)$ is drawn, show how to draw the graph of
$y=2f(x)$; that of $y=-f(x)$.

15. From the graph of $y=x^{2}$, show how to draw the graph of
$y=(2x)^{2}$; that of $y=x^{2}+2$; that of $y=(x+2)^{2}$;
that of $y=(2x-3)^{2}$.

16. What change is made in a curve if $x$, in the equation, is replaced
by $-x$ ? if $y$ by $-y$ ? if both things are done ? Compare the graphs of
$y=f(x), y=f(-x), -y=f(x);y=2f(x);y=f(x)+2$.

17. What change is made in a curve if $x$ is replaced by $2x, 3x, x/2$?
Compare the graphs of $y=f(x), y=f(2x), y=f(3x), y=f(x/2)$;
$y=f(x+2)$.

18. What Is the effect upon a curve if, In the equation, $x$ and $y$ are
interchanged ? Compare the graphs of $y=f(x), x=f(y)$.

19. Plot the following curves: $(a)y+2=\sin(3x+2), (b)y=x+\sin x$,
(c) $y=2^{l}-\sin x,\ (d)y=2^x\cos x,\ (e)3x+4y=4\sin(4x-3y)$,
(f) $y=(\cos x)/(2x+3)$, (g)$\sin y=\cos 2x$, (h)$y=\log_{2}(x^{2}+1)$.

20. In polar co\:{o}rdinates $(r,\ \theta)$, what change is made in a curve if, in
the equation, $\theta$ is replaced by 2 $\theta$, if $r$ is replaced by $2r$?

21. What change in $\theta$ is equivalent to a change in the sense of $r$.

22. From the graph of $r=f(\theta)$ derive those of (a) $f=f(2\theta)$,
(b)$r=2f(\theta)$, (c)$r=f(-\theta)$, (d)$r=-f(\theta)$, (e)$f+1=f(\theta)$,
(f)$r=f(\theta+1)$, (g)$r+1=f(\theta+2)$.

Take, for example, $f(\theta)=1, f(\theta)=\theta, f(\theta)=\sin\theta, f(\theta)=2\theta, f(\theta)
= \arctan\theta$, and draw the variations from the original graph.

23. Plot the following: (a)$r=2+3\cos\theta$, (b)$r=3+2\cos\theta$, 
(c)$r=2+2\cos\theta$, (d)$r=2^{\theta}$,\ (e)$r^{2}=a\theta$,\ (f)$\theta=2^r$, (g)$\theta^2=ar$,
(h)$\theta=\sin r$, (i)$\theta=\cos r$, (j)$\theta=\tan r$, (k)$r=\sec(\theta-a)$, (l)$\theta=\sec r$.

24. Show how to obtain the graph of $y=A\sin(at+b)$ by suitable
modification of the simple sine curve $y=\sin t$.

25. Draw the graphs from the following equations: $(a)2s=e^{t}+e^{-t}$,
(b)$ 2s=e^{t}-e^{-t}$, (c) $s=(e^{t}+e^{-l})/(e^{l}-e^{-l})$,\ (d)$s=\sin t+\sin 2t$,
(e)$s=\sin t+e^{-t} \sin 2t$. Take $e=2.7$, and use logarithms in computations.
\end{document}
