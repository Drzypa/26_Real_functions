\documentclass[12pt]{article}
\usepackage{pmmeta}
\pmcanonicalname{GeneralMeansInequality}
\pmcreated{2013-03-22 12:39:49}
\pmmodified{2013-03-22 12:39:49}
\pmowner{drini}{3}
\pmmodifier{drini}{3}
\pmtitle{general means inequality}
\pmrecord{6}{32934}
\pmprivacy{1}
\pmauthor{drini}{3}
\pmtype{Theorem}
\pmcomment{trigger rebuild}
\pmclassification{msc}{26D15}
\pmsynonym{power means inequality}{GeneralMeansInequality}
\pmrelated{ArithmeticGeometricMeansInequality}
\pmrelated{ArithmeticMean}
\pmrelated{GeometricMean}
\pmrelated{HarmonicMean}
\pmrelated{PowerMean}
\pmrelated{ProofOfArithmeticGeometricHarmonicMeansI}
\pmrelated{RootMeanSquare3}
\pmrelated{DerivationOfZerothWeightedPowerMean}
\pmrelated{ProofOfArithmeticGeometricHarmonicMeansInequality}
\pmrelated{ComparisonOfPythagor}

\usepackage{graphicx}
%%%\usepackage{xypic} 
\usepackage{bbm}
\newcommand{\Z}{\mathbbmss{Z}}
\newcommand{\C}{\mathbbmss{C}}
\newcommand{\R}{\mathbbmss{R}}
\newcommand{\Q}{\mathbbmss{Q}}
\newcommand{\mathbb}[1]{\mathbbmss{#1}}
\newcommand{\figura}[1]{\begin{center}\includegraphics{#1}\end{center}}
\newcommand{\figuraex}[2]{\begin{center}\includegraphics[#2]{#1}\end{center}}
\begin{document}
The power means inequality is a generalization of arithmetic-geometric means inequality.

If $0\neq r\in\R$, the $r$-mean (or $r$-th power mean) of the nonnegative
numbers $a_1,\ldots,a_n$ is defined as
$$M^r(a_1,a_2,\ldots,a_n)= \left(\frac{1}{n}\displaystyle{\sum_{k=1}^n a_k^r}\right)^{1/r}$$

Given real numbers $x,y$ such that $xy\neq 0$
and $x<y$, we have
$$M^x \leq M^y$$
and the equality holds if and only if $a_1 = ... = a_n$.

Additionally, if we define $M^0$ to be the
geometric mean $(a_1a_2...a_n)^{1/n}$, we have
that the inequality above holds for arbitrary real numbers $x<y$.

The mentioned inequality is a special case of this one, since $M^1$ is the arithmetic mean, $M^0$ is the geometric mean and $M^{-1}$ is the harmonic mean.

This inequality can be further generalized using weighted power means.
%%%%%
%%%%%
\end{document}
