\documentclass[12pt]{article}
\usepackage{pmmeta}
\pmcanonicalname{AsymptoteOfLamesCubic}
\pmcreated{2013-03-22 17:54:34}
\pmmodified{2013-03-22 17:54:34}
\pmowner{pahio}{2872}
\pmmodifier{pahio}{2872}
\pmtitle{asymptote of Lam\'e's cubic}
\pmrecord{7}{40403}
\pmprivacy{1}
\pmauthor{pahio}{2872}
\pmtype{Example}
\pmcomment{trigger rebuild}
\pmclassification{msc}{26C05}
\pmclassification{msc}{53A04}
\pmrelated{Hyperbola}
\pmrelated{WitchOfAgnesi}
\pmrelated{ConicSection}
\pmdefines{Lam\'e's cubic}

% this is the default PlanetMath preamble.  as your knowledge
% of TeX increases, you will probably want to edit this, but
% it should be fine as is for beginners.

% almost certainly you want these
\usepackage{amssymb}
\usepackage{amsmath}
\usepackage{amsfonts}

% used for TeXing text within eps files
%\usepackage{psfrag}
% need this for including graphics (\includegraphics)
%\usepackage{graphicx}
% for neatly defining theorems and propositions
 \usepackage{amsthm}
% making logically defined graphics
%%%\usepackage{xypic}
\usepackage{pstricks}
\usepackage{pst-plot}

% there are many more packages, add them here as you need them

% define commands here

\theoremstyle{definition}
\newtheorem*{thmplain}{Theorem}

\begin{document}
We will show that the {\em Lam\'e's cubic} 
\begin{align}
x^3+y^3 = a^3,
\end{align}
where $a$ is a positive \PMlinkescapetext{constant}, has the line
$$y = \underbrace{-x}_{g(x)}$$
as its asymptote.\\

Because the equation (1) of the curve is symmetric with respect to $x$ and $y$, the curve is symmetric about the line \,$y = x$.\, From the solved form
\begin{align}
y = \underbrace{\sqrt[3]{a^3-x^3}}_{f(x)}
\end{align}
of (1) we see that every real value of $x$ gives one point of the curve. 

\begin{center}
\begin{pspicture}(-4,-4)(4,4)
\psaxes[Dx=9,Dy=9]{->}(0,0)(-3.5,-3.5)(3.5,3.5)
\rput(3.6,-0.2){$x$}
\rput(0.2,3.5){$y$}
\psdot[linecolor=blue](1,0)
\psdot[linecolor=blue](0,1)
\rput(1.13,-0.17){$a$}
\rput(-0.17,1.13){$a$}
\psline[linecolor=cyan](-3.5,3.5)(3.5,-3.5)
\psline[linestyle=dashed](-3.5,-3.5)(3.5,3.5)
\psplot[linecolor=blue]{-3.5}{1}{1 x 3 exp sub 1 3 div exp}
\psplot[linecolor=blue]{1}{3.5}{0 x 3 exp 1 sub 1 3 div exp sub}
\rput(0.5,-4){Lam\'e's cubic\, $y = \sqrt[3]{a^3-x^3}$\, (blue)}
\end{pspicture}
\end{center}

The difference \,$\Delta = f(x)\!-\!g(x)$\, \PMlinkescapetext{represents} the distance of a point \,$(x,\,y)$\, of the curve and the point of the asserted asymptote\, $y = -x$\, with the same abscissa $x$.\, We multiply the numerator and denominator with the expression \,$(\sqrt[3]{a^3-x^3})^2-x\sqrt[3]{a^3-x^3}+x^2$ for being able to utilise the polynomial formula
$$(u+v)(u^2-uv+v^2) = u^3+v^3,$$  
getting
\begin{align*}
\Delta &= f(x)\!-\!g(x)\\
       &= \frac{\sqrt[3]{a^3-x^3}+x}{1}\\
       &= \frac{(\sqrt[3]{a^3-x^3})^3+x^3}{(\sqrt[3]{a^3-x^3})^2-x\sqrt[3]{a^3-x^3}+x^2}\\
       &= \frac{a^3}{(\sqrt[3]{a^3-x^3})^2-x\sqrt[3]{a^3-x^3}+x^2}. 
\end{align*}
Thus,\, $\displaystyle \Delta \to \frac{a^3}{\infty+\infty+\infty} = 0$\; when\; $ |x| \to \infty$\; (see the improper limits).\, According to the definition of \PMlinkname{asymptote}{Asymptote}, the line\, $y = -x$\, is asymptote of Lam\'e's cubic.  




%%%%%
%%%%%
\end{document}
