\documentclass[12pt]{article}
\usepackage{pmmeta}
\pmcanonicalname{SumFunctionOfSeries}
\pmcreated{2013-03-22 14:38:15}
\pmmodified{2013-03-22 14:38:15}
\pmowner{pahio}{2872}
\pmmodifier{pahio}{2872}
\pmtitle{sum function of series}
\pmrecord{18}{36223}
\pmprivacy{1}
\pmauthor{pahio}{2872}
\pmtype{Definition}
\pmcomment{trigger rebuild}
\pmclassification{msc}{26A15}
\pmclassification{msc}{40A30}
\pmrelated{UniformConvergenceOfIntegral}
\pmrelated{SumOfSeries}
\pmrelated{OneSidedContinuityBySeries}
\pmdefines{function series}
\pmdefines{sum function}
\pmdefines{uniform convergence of series}

\endmetadata

% this is the default PlanetMath preamble.  as your knowledge
% of TeX increases, you will probably want to edit this, but
% it should be fine as is for beginners.

% almost certainly you want these
\usepackage{amssymb}
\usepackage{amsmath}
\usepackage{amsfonts}

% used for TeXing text within eps files
%\usepackage{psfrag}
% need this for including graphics (\includegraphics)
%\usepackage{graphicx}
% for neatly defining theorems and propositions
%\usepackage{amsthm}
% making logically defined graphics
%%%\usepackage{xypic}

% there are many more packages, add them here as you need them

% define commands here
\begin{document}
Let the terms of a series be real functions $f_n$ defined in a certain subset $A_0$ of $\mathbb{R}$; we can speak of a {\em function series}.\, All points $x$ where the series
 \begin{align}
                  f_1+f_2+\cdots
 \end{align}
converges form a subset $A$ of $A_0$, and we have the {\em \PMlinkescapetext{sum function}}\, $S\!:x\mapsto S(x)$\, of (1) defined in $A$.

If the sequence\, $S_1,\,S_2,\,\ldots$\, of the partial sums\, 
$S_n = f_1\!+\!f_2\!+\cdots+\!f_n$\, of the series (1) \PMlinkname{converges uniformly}{LimitFunctionOfSequence} in the interval\, $[a,\,b] \subseteq{A}$\, to a function \, $S\!:x\mapsto S(x)$,\, we say that {\em the series \PMlinkescapetext{converges uniformly}} in this interval.\, We may also set the direct

\textbf{Definition.}\, The function series (1), which converges in every point of the interval\, $[a,\,b]$\, having sum function\, $S:x\mapsto S(x)$,\, 
{\em \PMlinkescapetext{converges uniformly}} in the interval\, $[a,\,b]$,\, if for every positive number $\varepsilon$ there is an integer $n_\varepsilon$ such that each value of $x$ in the interval\, $[a,\,b]$\, \PMlinkescapetext{satisfies} the inequality
          $$|S_n(x)-S(x)| < \varepsilon$$
when\, $n \geqq n_\varepsilon$.\\

\textbf{Note.}\, One can without trouble be convinced that the term functions of a uniformly converging series converge uniformly to 0 (cf. the necessary condition of convergence).\\

The notion of \PMlinkescapetext{uniform convergence} of series can be extended to the series with complex function terms (the interval is replaced with some subset of $\mathbb{C}$).\, The significance of the \PMlinkescapetext{uniform convergence} is therein that the sum function of a series with this property and with continuous term-functions is continuous and may be integrated termwise.
%%%%%
%%%%%
\end{document}
