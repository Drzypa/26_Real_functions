\documentclass[12pt]{article}
\usepackage{pmmeta}
\pmcanonicalname{ProofOfTaylorsTheorem}
\pmcreated{2013-03-22 12:33:59}
\pmmodified{2013-03-22 12:33:59}
\pmowner{rmilson}{146}
\pmmodifier{rmilson}{146}
\pmtitle{proof of Taylor's Theorem}
\pmrecord{8}{32814}
\pmprivacy{1}
\pmauthor{rmilson}{146}
\pmtype{Proof}
\pmcomment{trigger rebuild}
\pmclassification{msc}{26A06}

\endmetadata

\usepackage{amsmath}
\usepackage{amsfonts}
\usepackage{amssymb}
\newcommand{\reals}{\mathbb{R}}
\newcommand{\natnums}{\mathbb{N}}
\newcommand{\cnums}{\mathbb{C}}
\newcommand{\znums}{\mathbb{Z}}
\newcommand{\lp}{\left(}
\newcommand{\rp}{\right)}
\newcommand{\lb}{\left[}
\newcommand{\rb}{\right]}
\newcommand{\supth}{^{\text{th}}}
\newtheorem{proposition}{Proposition}
\newtheorem{definition}[proposition]{Definition}

\newtheorem{theorem}[proposition]{Theorem}
\begin{document}

Let $f(x),\; a<x<b$ be a real-valued, $n$-times differentiable
function, and let $a<x_0<b$ be a  fixed base-point. We will show that
for all $x\neq x_0$ in the domain of the 
function, there exists a $\xi$, strictly between $x_0$ and $x$ such
that $$f(x) = \sum_{k=0}^{n-1} f^{(k)}(x_0)\, \frac{(x-x_0)^k}{k!}
+ f^{(n)}(\xi)\,\frac{(x-x_0)^n}{n!}.$$

Fix $x\neq x_0$ and let $R$ be the remainder defined by
$$f(x) = \sum_{k=0}^{n-1} f^{(k)}(x_0)\, \frac{(x-x_0)^k}{k!} +
R\,\frac{(x-x_0)^n}{n!}.$$
Next, define
$$F(\xi) = \sum_{k=0}^{n-1} f^{(k)}(\xi)\, \frac{(x-\xi)^k}{k!} +
R\,\frac{(x-\xi)^n}{n!},\quad a<\xi<b.$$
We then have
\begin{align*}
F'(\xi) &= f'(\xi)+\sum_{k=1}^{n-1} \left(f^{(k+1)}(\xi)\, \frac{(x-\xi)^k}{k!}
   - f^{(k)}(\xi)\, \frac{(x-\xi)^{k-1}}{(k-1)!}\right)
- R\,\frac{(x-\xi)^{n-1}}{(n-1)!}\\
&= f^{(n)}(\xi)\, \frac{(x-\xi)^{n-1}}{(n-1)!}-
R\,\frac{(x-\xi)^{n-1}}{(n-1)!} \\
&=  \frac{(x-\xi)^{n-1}}{(n-1)!}\, ( f^{(n)}(\xi)-R),
\end{align*}
because the sum telescopes.
Since, $F(\xi)$ is a differentiable function, and since
$F(x_0)=F(x)=f(x)$, Rolle's Theorem imples that there exists a $\xi$ lying
strictly between $x_0$ and $x$ such that $F'(\xi)=0$.  It follows that
$R=f^{(n)}(\xi)$, as was to be shown.
%%%%%
%%%%%
\end{document}
