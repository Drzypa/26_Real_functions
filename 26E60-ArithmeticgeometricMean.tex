\documentclass[12pt]{article}
\usepackage{pmmeta}
\pmcanonicalname{ArithmeticgeometricMean}
\pmcreated{2013-03-22 14:23:46}
\pmmodified{2013-03-22 14:23:46}
\pmowner{rspuzio}{6075}
\pmmodifier{rspuzio}{6075}
\pmtitle{arithmetic-geometric mean}
\pmrecord{7}{35893}
\pmprivacy{1}
\pmauthor{rspuzio}{6075}
\pmtype{Definition}
\pmcomment{trigger rebuild}
\pmclassification{msc}{26E60}
\pmclassification{msc}{33E05}
\pmsynonym{agm}{ArithmeticgeometricMean}
\pmsynonym{AGM}{ArithmeticgeometricMean}
\pmrelated{EllipticIntegralsAndJacobiEllipticFunctions}

% this is the default PlanetMath preamble.  as your knowledge
% of TeX increases, you will probably want to edit this, but
% it should be fine as is for beginners.

% almost certainly you want these
\usepackage{amssymb}
\usepackage{amsmath}
\usepackage{amsfonts}

% used for TeXing text within eps files
%\usepackage{psfrag}
% need this for including graphics (\includegraphics)
%\usepackage{graphicx}
% for neatly defining theorems and propositions
%\usepackage{amsthm}
% making logically defined graphics
%%%\usepackage{xypic}

% there are many more packages, add them here as you need them

% define commands here
\def\sse{\subseteq}
\def\bigtimes{\mathop{\mbox{\Huge $\times$}}}
\def\impl{\Rightarrow}
\def\oo{\infty}
\def\agm{\operatorname{agm}}
\def\AGM{\operatorname{AGM}}
\begin{document}
If $x$ and $y$ are non-negative real numbers, we can form their arithmetic
mean $a_0 = (x+y)/2$ as well as their geometric mean $g_0 = \sqrt{xy}$.
This procedure can be repeated to form a sequence of arithmetic and
geometic means $a_{n+1} = (a_n+g_n)/2$ and $g_{n+1} = \sqrt{a_n g_n}$.
By the arithmetic-geometric means inequality we have $a_n \ge a_{n+1} \ge g_{n+1} \ge g_n$ (with equality holding only when $a_n=g_n$),
hence these sequences converge to a number between $x$ and $y$,
with the rate of convergence being superlinear.
The \emph{arithmetic-geometric mean} $M(x,y)$ of $x$ and $y$
is defined as this limit
\begin{equation*}
  M(x,y) = \lim_{n\to\oo} a_n, g_n.
\end{equation*}
The origin of the name is obvious from the construction. Alternative notations
for $M(x,y)$ are $\agm(x,y)$ or $\AGM(x,y)$.

The AGM lies between the arithmetic and geometric
means of $x$ and $y$,
\begin{equation*}
  \frac{x+y}{2} \ge M(x,y) \ge \sqrt{xy},
\end{equation*}
with equality holding only in case of equality $x=y$. The AGM is also a
homogeneous function of degree $1$, namely $M(\alpha x, \alpha y)
= \alpha M(x,y)$ for $\alpha > 0$. It is also symmetric $M(x,y) = M(y,x)$.
These properties are obvious from the construction.

The AGM can be used to numerically evaluate elliptic integrals of the
first and second kinds. For example,
\begin{equation}
  M(x,y) = \frac{\pi}{4} \frac{x+y}{K\left(\frac{|x-y|}{x+y}\right)},
\end{equation}
where $K(k)$ is the elliptic integral of the first kind as function of
the modulus $k$.

As a numerical method, the arithmetic-geometric mean has much to recommend it.
By its nature, it automatically provides upper and lower bounds for the
answer, so one does not have to separately estimate error.  To compute
the arithmetic-geometric mean to a certain accuracy, we only need to carry
out the computation until the difference between $a_n$ and $g_n$ is smaller
than the desired accuracy.

Because convergence is superlinear, only a few iterations are necessarry to
obtain the answer.  For instance, if we compute $M(1,k)$ with $k$ less than 
a billion, we already obtain at least fifteen-place accuracy after eight 
iterations, as the following computation of $M(1,123456789)$ shows:

\begin{tabular}
{c|c|c}
$n$ & $g_n$ & $a_n$ \\
0 & 1.0 & 123456789.0 \\
1 & 11111.111060555555 & 61728395.0 \\
2 & 828173.3227017411 & 30869753.055530276 \\
3 & 5056234.365511624 & 15848963.189116009 \\
4 & 8951875.352937901 & 10452598.777313817 \\
5 & 9673177.418448625 & 9702237.06512586 \\
6 & 9687696.345716598 & 9687707.241787244 \\
7 & 9687701.793750389 & 9687701.793751922 \\
8 & 9687701.793751154 & 9687701.793751154
\end{tabular}

The fact that relatively few iterations are necessarry to obtain a highly
accurate result also means that one does not have to worry much about the 
cumulative effect of roundoff errors in the various steps of the computation.
%%%%%
%%%%%
\end{document}
