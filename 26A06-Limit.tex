\documentclass[12pt]{article}
\usepackage{pmmeta}
\pmcanonicalname{Limit}
\pmcreated{2013-03-22 12:28:25}
\pmmodified{2013-03-22 12:28:25}
\pmowner{djao}{24}
\pmmodifier{djao}{24}
\pmtitle{limit}
\pmrecord{12}{32662}
\pmprivacy{1}
\pmauthor{djao}{24}
\pmtype{Definition}
\pmcomment{trigger rebuild}
\pmclassification{msc}{26A06}
\pmclassification{msc}{26B12}
\pmclassification{msc}{54E35}
\pmrelated{Continuous}

\endmetadata

% this is the default PlanetMath preamble.  as your knowledge
% of TeX increases, you will probably want to edit this, but
% it should be fine as is for beginners.

% almost certainly you want these
\usepackage{amssymb,amsthm}
\usepackage{amsmath}
\usepackage{amsfonts}

% used for TeXing text within eps files
%\usepackage{psfrag}
% need this for including graphics (\includegraphics)
%\usepackage{graphicx}
% for neatly defining theorems and propositions
%\usepackage{amsthm}
% making logically defined graphics
%%%\usepackage{xypic} 

% there are many more packages, add them here as you need them

% define commands here
\begin{document}
Let $X$ and $Y$ be metric spaces and let $a \in X$ be a limit point of $X$. Suppose that $f\colon X \setminus \{a\} \to Y$ is a function defined everywhere except at
$a$. For $L \in Y$, we say the \emph{limit} of $f(x)$ as $x$
approaches $a$ is equal to $L$, or
$$
\lim_{x \to a} f(x) = L
$$
if, for every real number $\varepsilon > 0$, there exists a real
number $\delta > 0$ such that, whenever $x \in X$ with $0 < d_X(x,a)
< \delta$, then $d_Y(f(x), L) < \varepsilon$.

The formal definition of limit as given above has a well--deserved
reputation for being notoriously hard for inexperienced students to
master. There is no easy fix for this problem, since the concept of a
limit is inherently difficult to state precisely (and indeed wasn't
even accomplished historically until the 1800's by Cauchy, well after
the development of calculus in the 1600's by Newton and Leibniz).
However, there are number of related definitions, which, taken
together, may shed some light on the nature of the concept.

\begin{itemize}
\item The notion of a limit can be generalized to mappings between arbitrary 
topological spaces, under some mild restrictions.  In this context we say that
$\lim_{x \to a} f(x) = L$ if $a$ is a limit point of $X$ and, for every
neighborhood $V$ of $L$ (in $Y$), there is a deleted neighborhood
$U$ of $a$ (in $X$) which is mapped into $V$ by $f$. One also requires that the range $Y$ be Hausdorff (or at least $T_1$) in order to ensure that limits, when they exist, are unique.
\item Let $a_n, n\in\mathbb{N}$ be a sequence of elements in a metric
  space $X$.  We say that $L\in X$ is the limit of the sequence, if
  for every $\varepsilon>0$ there exists a natural number $N$ such
  that $d(a_n,L)< \varepsilon$ for all natural numbers $n> N$.
\item The definition of the limit of a mapping can be based on the
  limit of a sequence.  To wit, $\lim_{x \to a} f(x) = L$ if and only
  if, for every sequence of points $x_n$ in $X$ converging to $a$
  (that is, $x_n \to a$, $x_n \neq a$), the sequence of points
  $f(x_n)$ in $Y$ converges to $L$.
\end{itemize}
In calculus, $X$ and $Y$ are frequently taken to be Euclidean spaces
$\mathbb{R}^n$ and $\mathbb{R}^m$, in which case the distance
functions $d_X$ and $d_Y$ cited above are just Euclidean distance.
%%%%%
%%%%%
\end{document}
