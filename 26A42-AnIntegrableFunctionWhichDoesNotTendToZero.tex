\documentclass[12pt]{article}
\usepackage{pmmeta}
\pmcanonicalname{AnIntegrableFunctionWhichDoesNotTendToZero}
\pmcreated{2014-02-01 3:01:47}
\pmmodified{2014-02-01 3:01:47}
\pmowner{rspuzio}{6075}
\pmmodifier{rspuzio}{6075}
\pmtitle{an integrable function which does not tend to zero}
\pmrecord{15}{39233}
\pmprivacy{1}
\pmauthor{rspuzio}{6075}
\pmtype{Example}
\pmcomment{trigger rebuild}
\pmclassification{msc}{26A42}
\pmclassification{msc}{28A25}

% this is the default PlanetMath preamble.  as your knowledge
% of TeX increases, you will probably want to edit this, but
% it should be fine as is for beginners.

% almost certainly you want these
\usepackage{amssymb}
\usepackage{amsmath}
\usepackage{amsfonts}

% used for TeXing text within eps files
%\usepackage{psfrag}
% need this for including graphics (\includegraphics)
%\usepackage{graphicx}
% for neatly defining theorems and propositions
%\usepackage{amsthm}
% making logically defined graphics
%%%\usepackage{xypic}

% there are many more packages, add them here as you need them

% define commands here

\begin{document}
\PMlinkescapeword{words}

In this entry, we give an example of a function $f$ such that $f$ is Lebesgue 
integrable on $\mathbb{R}$ but $f(x)$ does not tend to zero as 
$x \rightarrow \infty$.

Set
\[
 f(x) = \sum_{k=1}^\infty {k \over k^6 (x-k)^2 + 1}.
\]
Note that every term in this series is positive, hence we may integrate
term-by-term, then make a change of variable $y = k^3 x - k^4$ and
compute the answer:
\begin{align*}
 \int_{-\infty}^{+\infty} f(x) \, dx &=
 \sum_{k=1}^\infty
 \int_{-\infty}^{+\infty}
 {k \, dx \over k^6 (x-k)^2 + 1} \\ &=
 \sum_{k=1}^\infty
 {1 \over k^2} 
 \int_{-\infty}^{+\infty}
 {dy \over y^2 + 1} \\ &=
 {\pi^2 \over 6} \cdot \pi = {\pi^3 \over 6}
\end{align*}
However, when $k$ is an integer, $f(k) > k$, so not only does $f(x)$ not 
tend to zero as $x \rightarrow \infty$, it gets arbitrarily large.
As we can see from the plot, we have a sequence of peaks which, as they get
taller, also get narrower in such a way that the total area under
the curve stays finite:
\begin{figure}
 \includegraphics{integrable-no-tend-zero.gif}
\end{figure}

By a variation of our procedure, we can produce a function which is
defined almost everywhere on the interval $(0,1)$, is Lebesgue integrable,
but is unbounded on any subinterval, no matter how small.  For instance,
define 
\[
 f(x) = \sum_{m=1}^\infty \sum_{n=1}^\infty 
 {1\over (m + n)^6 (x-m/(m+n))^2 + 1}.
\]
Making a computation similar to the one above, we find that 
\begin{align*}
 \int_{-\infty}^{+\infty} f(x) \, dx &=
 \sum_{m=1}^\infty
 \sum_{n=1}^\infty
  \int_{-\infty}^{+\infty}
   {dx \over (m+n)^6 (x-m/(m+n))^2 + 1} \\ &=
 \sum_{m=1}^\infty
 \sum_{n=1}^\infty
  {1 \over (m+n)^3} 
  \int_{-\infty}^{+\infty}
   {dy \over y^2 + 1} \\ &=
 \pi \sum_{k=1}^\infty
 {k-1 \over k^3}
\end{align*}
Hence the integral is finite.

Now, however, we find that $f$ cannot be bounded in any interval,
however small.  For, in any interval, we can find rational
numbers.  Given a rational number $r$, there are an infinite
number of ways to express it as a fraction $m/(m+n)$.  For
each of these ways, we have a term in the series which equals 1 
when $x=r$, hence $f(r)$ diverges to infinity.

To help in understanding this function, we have made a slide show
which shows partial sums of the series.  As before, the successive
peaks become narrower in such a way that the arae under the curve 
stays finite but, this time, instead of marching off to infinity,
they become dense in the interval.

\begin{figure}
 \includegraphics{integrable-everywhere-unbounded_3.gif}
\end{figure}
\end{document}
