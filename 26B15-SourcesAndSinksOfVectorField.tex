\documentclass[12pt]{article}
\usepackage{pmmeta}
\pmcanonicalname{SourcesAndSinksOfVectorField}
\pmcreated{2013-03-22 18:47:03}
\pmmodified{2013-03-22 18:47:03}
\pmowner{pahio}{2872}
\pmmodifier{pahio}{2872}
\pmtitle{sources and sinks of vector field}
\pmrecord{12}{41579}
\pmprivacy{1}
\pmauthor{pahio}{2872}
\pmtype{Definition}
\pmcomment{trigger rebuild}
\pmclassification{msc}{26B15}
\pmclassification{msc}{26B12}
%\pmkeywords{sink of vector field}
\pmrelated{Divergence}
\pmrelated{SolenoidalField}
\pmrelated{CirculationAndVorticity}
\pmdefines{source}
\pmdefines{sink}
\pmdefines{source of vector field}
\pmdefines{productivity}
\pmdefines{strength}
\pmdefines{source density}

\endmetadata

% this is the default PlanetMath preamble.  as your knowledge
% of TeX increases, you will probably want to edit this, but
% it should be fine as is for beginners.

% almost certainly you want these
\usepackage{amssymb}
\usepackage{amsmath}
\usepackage{amsfonts}

% used for TeXing text within eps files
%\usepackage{psfrag}
% need this for including graphics (\includegraphics)
%\usepackage{graphicx}
% for neatly defining theorems and propositions
 \usepackage{amsthm}
% making logically defined graphics
%%%\usepackage{xypic}

% there are many more packages, add them here as you need them

% define commands here

\theoremstyle{definition}
\newtheorem*{thmplain}{Theorem}

\begin{document}
\PMlinkescapeword{flow} \PMlinkescapeword{field} \PMlinkescapeword{closed}
\PMlinkescapeword{density}

Let the vector field $\vec{U}$ of $\mathbb{R}^3$ be interpreted, as in the remark of the \PMlinkname{parent entry}{Flux}, as the velocity \PMlinkescapetext{field of a stationary flow} of a liquid.\, Then the flux 
\[
\oint_a\vec{U}\cdot d\vec{a}
\]
of $\vec{U}$ through a closed surface $a$ expresses how much more liquid per time-unit it comes from inside of $a$ to outside than contrarily.\, Since for a usual incompressible liquid, the outwards flow and the inwards flow are equal, we must think in the case that the flux differs from 0 either that the flowing liquid is suitably compressible or that there are inside the surface some {\em sources} creating liquid and {\em sinks} annihilating liquid.\, Ordinarily, one uses the latter idea.\, Both the sources and the sinks may be called sources, when the sinks are {\em negative sources}.\, The flux of the vector $\vec{U}$ through $a$ is called the {\em productivity} or the {\em strength} of the sources inside $a$.

For example, the sources and sinks of an electric field ($\vec{E}$) are the locations containing positive and negative charges, respectively.\, The gravitational field has only sinks, which are the locations containing \PMlinkescapetext{mass}.\\

The expression
\[
\frac{1}{\Delta v}\oint_{\partial\Delta v}\vec{U}\cdot d\vec{a},
\]
where $\Delta v$ means a region in the vector field and also its volume, is the productivity of the sources in 
$\Delta v$ per a volume-unit.\, When we let $\Delta v$ to shrink towards a point $P$ in it, to an infinitesimal volume-element $dv$, we get the limiting value
\begin{align}
\varrho \;:=\; \frac{1}{dv}\oint_{\partial dv}\vec{U}\cdot d\vec{a},
\end{align}
called the {\em source density} in $P$.\, Thus the productivity of the source in $P$ is $\varrho\,dv$.\, If\, 
$\varrho = 0$, there is in $P$ neither a source, nor a sink.\\

The Gauss's theorem
\[
\int_v\nabla\cdot\vec{U}\,dv \;=\; \oint_a\vec{U}\cdot d\vec{a}
\]
applied to $dv$ says that
\begin{align}
\nabla\cdot\vec{U} \;=\; \frac{1}{dv}\oint_{\partial dv}\vec{U}\cdot d\vec{a}.
\end{align}
Accordingly,
\begin{align}
\varrho \;=\; \nabla\cdot\vec{U}
\end{align}
and
\begin{align}
\oint_{a}\vec{U}\cdot d\vec{a} \;=\; \int_v\varrho\,dv.
\end{align}
This last \PMlinkescapetext{formula} can be read that\, {\em the flux of the vector through a closed surface equals to the total productivity of the sources inside the surface.}\, For example, if $\vec{U}$ is the electric flux density 
$\vec{D}$, (4) means that the electric flux through a closed surface equals to the total charge inside.

%%%%%
%%%%%
\end{document}
