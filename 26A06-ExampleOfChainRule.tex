\documentclass[12pt]{article}
\usepackage{pmmeta}
\pmcanonicalname{ExampleOfChainRule}
\pmcreated{2013-03-22 12:35:32}
\pmmodified{2013-03-22 12:35:32}
\pmowner{rmilson}{146}
\pmmodifier{rmilson}{146}
\pmtitle{example of chain rule}
\pmrecord{4}{32843}
\pmprivacy{1}
\pmauthor{rmilson}{146}
\pmtype{Example}
\pmcomment{trigger rebuild}
\pmclassification{msc}{26A06}

\usepackage{amsmath}
\usepackage{amsfonts}
\usepackage{amssymb}

\newcommand{\reals}{\mathbb{R}}
\newcommand{\natnums}{\mathbb{N}}
\newcommand{\cnums}{\mathbb{C}}
\newcommand{\znums}{\mathbb{Z}}

\newcommand{\lp}{\left(}
\newcommand{\rp}{\right)}
\newcommand{\lb}{\left[}
\newcommand{\rb}{\right]}

\newcommand{\supth}{^{\text{th}}}


\newtheorem{proposition}{Proposition}
\newtheorem{definition}[proposition]{Definition}
\begin{document}
Suppose we wanted to differentiate 
$$h(x)=\sqrt{\sin(x)}.$$
Here, $h(x)$ is given by the composition 
$$h(x)=f(g(x)),$$
where
$$f(x)=\sqrt{x}\quad\mbox{and}\quad g(x)=\sin(x).$$
Then chain rule
says that 
$$h'(x)=f'(g(x))g'(x).$$

Since
$$f'(x)=\frac{1}{2\sqrt{x}},\quad\mbox{and}\quad
g'(x)=\cos(x),$$
we have by chain rule
$$h'(x) = \left(\frac{1}{2\sqrt{\sin x}}\right)\cos x=\frac{\cos
  x}{2\sqrt{\sin 
x}}$$

Using the Leibniz formalism, the above calculation would have the
following appearance.  First we describe the functional relation as
$$z = \sqrt{\sin(x)}.$$
Next, we introduce an auxiliary variable $y$, and write
$$z= \sqrt{y},\qquad y=\sin(x).$$
We then have
$$\frac{dz}{dy} = \frac{1}{2\sqrt{y}},\qquad \frac{dy}{dx} =
\cos(x),$$
and hence the chain rule gives
\begin{align*}
\frac{dz}{dx} &= \frac{1}{2\sqrt{y}}\, \cos(x) \\
&= \frac{1}{2}\,\frac{\cos(x)}{\sqrt{\sin(x)}}
\end{align*}
%%%%%
%%%%%
\end{document}
