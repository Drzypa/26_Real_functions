\documentclass[12pt]{article}
\usepackage{pmmeta}
\pmcanonicalname{PTAHInequalityResult}
\pmcreated{2013-03-22 18:08:33}
\pmmodified{2013-03-22 18:08:33}
\pmowner{cappymate}{20671}
\pmmodifier{cappymate}{20671}
\pmtitle{PTAH inequality result}
\pmrecord{4}{40697}
\pmprivacy{1}
\pmauthor{cappymate}{20671}
\pmtype{Result}
\pmcomment{trigger rebuild}
\pmclassification{msc}{26D15}

% this is the default PlanetMath preamble.  as your knowledge
% of TeX increases, you will probably want to edit this, but
% it should be fine as is for beginners.

% almost certainly you want these
\usepackage{amssymb}
\usepackage{amsmath}


\usepackage{amsfonts}

% used for TeXing text within eps files
%\usepackage{psfrag}
% need this for including graphics (\includegraphics)
%\usepackage{graphicx}
% for neatly defining theorems and propositions
%\usepackage{amsthm}
% making logically defined graphics
%%%\usepackage{xypic}

% there are many more packages, add them here as you need them

% define commands here

\begin{document}
There is an additional classic inequality to be added to the list (A) through (F), namely the maximum-entropy inequality (in logarithmic form)
\[   - \sum_{i=1}^n p_i \log p_i \le \log n . \]

Also, inequality (C) should be labeled ``the Kullback--Leibler inequality".

Then, the labeling could be modified: the first time the PTAH inequality appears it could be labeled (P);
then the maximum-entropy inequality could be labeled (G), followed by H\"{o}lder's inequality (already labeled (H));
and the statement could be ``the inequalities (A) through (G) are equivalent, each is a special case of (H), 
and (H) is a special case of (P).  However, it appears that none of the reverse implications hold."
%%%%%
%%%%%
\end{document}
