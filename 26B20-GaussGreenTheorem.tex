\documentclass[12pt]{article}
\usepackage{pmmeta}
\pmcanonicalname{GaussGreenTheorem}
\pmcreated{2013-03-22 15:01:51}
\pmmodified{2013-03-22 15:01:51}
\pmowner{paolini}{1187}
\pmmodifier{paolini}{1187}
\pmtitle{Gauss Green theorem}
\pmrecord{13}{36741}
\pmprivacy{1}
\pmauthor{paolini}{1187}
\pmtype{Theorem}
\pmcomment{trigger rebuild}
\pmclassification{msc}{26B20}
\pmsynonym{divergence theorem}{GaussGreenTheorem}
\pmrelated{GreensTheorem}
\pmrelated{GeneralStokesTheorem}
\pmrelated{IntegrationWithRespectToSurfaceArea}
\pmrelated{ClassicalStokesTheorem}
\pmrelated{FluxOfVectorField}

% this is the default PlanetMath preamble.  as your knowledge
% of TeX increases, you will probably want to edit this, but
% it should be fine as is for beginners.

% almost certainly you want these
\usepackage{amssymb}
\usepackage{amsmath}
\usepackage{amsfonts}

% used for TeXing text within eps files
%\usepackage{psfrag}
% need this for including graphics (\includegraphics)
%\usepackage{graphicx}
% for neatly defining theorems and propositions
%\usepackage{amsthm}
% making logically defined graphics
%%%\usepackage{xypic}

% there are many more packages, add them here as you need them

% define commands here

\newcommand{\R}{\mathbf R}
\newtheorem{theorem}{Theorem}
\begin{document}
\begin{theorem}[Gauss-Green]
Let $\Omega\subset \R^n$ be a bounded open set with $C^1$ boundary, let $\nu_\Omega\colon \partial \Omega\to \R^n$ be the exterior unit normal vector to $\Omega$ in the point $x$ and let $f\colon \overline{\Omega}\to \R^n$ be a vector function in $C^0(\overline\Omega,\R^n)\cap C^1(\Omega,\R^n)$. Then
\[
  \int_\Omega \mathrm{div} f(x)\, dx 
   =\int_{\partial \Omega} \langle f(x),\nu_\Omega(x)\rangle \, d\sigma(x).
\]
\end{theorem}

Some remarks on notation.
The operator $\mathrm{div} f$ is the divergence of the vector field $f$, which is sometimes written as $\nabla \cdot f$.
In the right-hand side we have a surface integral, $d\sigma$ is the corresponding area measure on $\partial \Omega$.
The scalar product in the second integral is sometimes written as $f_n(x)$
and represents the \emph{normal component} of $f$ with respect to $\partial \Omega$; hence the whole integral represents the \emph{flux} of the vector field $f$ through $\partial \Omega$;

This theorem can be easily extended to \emph{piecewise} regular domains. 
However the more general statement of this Theorem involves the theory of \emph{perimeters} and $BV$ functions.
\begin{theorem}[generalized Gauss-Green]
Let $E\subset \R^n$ be any measurable set.
Then 
\[
  \int_E \mathrm{div} f(x)\, dx
  = \int_{\partial^* E} \langle \nu_E(x),f(x)\rangle \,d\mathcal H^{n-1}(x)
\]
holds for every continuously differentiable function $f\colon \R^n\to\R^n$ with compact support (i.e.\ $f\in\mathcal C^1_c(\R^n,\R^n)$) where
\begin{itemize}
\item
$\partial^* E$ is the \emph{essential boundary} of $E$ which is a subset of the topological boundary $\partial E$;
\item $\nu_E(x)$ is the exterior normal vector to $E$, which is defined when $x\in\mathcal F E$;
\item $\mathcal H^{n-1}$ is the $(n-1)$-dimensional Hausdorff measure.
\end{itemize}
\end{theorem}
%%%%%
%%%%%
\end{document}
