\documentclass[12pt]{article}
\usepackage{pmmeta}
\pmcanonicalname{IntegrationUnderIntegralSign}
\pmcreated{2013-03-22 18:46:27}
\pmmodified{2013-03-22 18:46:27}
\pmowner{pahio}{2872}
\pmmodifier{pahio}{2872}
\pmtitle{integration under integral sign}
\pmrecord{5}{41567}
\pmprivacy{1}
\pmauthor{pahio}{2872}
\pmtype{Theorem}
\pmcomment{trigger rebuild}
\pmclassification{msc}{26A42}
\pmrelated{FubinisTheorem}
\pmrelated{DifferentiationUnderIntegralSign}
\pmrelated{RelativeOfExponentialIntegral}

\endmetadata

% this is the default PlanetMath preamble.  as your knowledge
% of TeX increases, you will probably want to edit this, but
% it should be fine as is for beginners.

% almost certainly you want these
\usepackage{amssymb}
\usepackage{amsmath}
\usepackage{amsfonts}

% used for TeXing text within eps files
%\usepackage{psfrag}
% need this for including graphics (\includegraphics)
%\usepackage{graphicx}
% for neatly defining theorems and propositions
%\usepackage{amsthm}
% making logically defined graphics
%%%\usepackage{xypic}

% there are many more packages, add them here as you need them

% define commands here
\newcommand{\sijoitus}[2]%
{\operatornamewithlimits{\Big/}_{\!\!\!#1}^{\,#2}}
\begin{document}
Let
$$I(\alpha) \;=\; \int_a^b\!f(x,\,\alpha)\,dx.$$
where \,$f(x,\,\alpha)$ is continuous in the rectangle
$$a \leqq x \leqq b,\, \quad \alpha_1 \leqq \alpha \leqq \alpha_2.$$
Then\, $\alpha \mapsto I(\alpha)$\, is continuous and hence \PMlinkname{integrable}{RiemannIntegrable} on the interval 
\,$\alpha_1 \leqq \alpha \leqq \alpha_2$;\, we have
$$\int_{\alpha_1}^{\alpha_2}I(\alpha)\,d\alpha \;=\; \int_{\alpha_1}^{\alpha_2}\left(\int_a^b\!f(x,\,\alpha)\,dx\right)d\alpha.$$
This is a double integral over a \PMlinkescapetext{regular domain} in the $x\alpha$-plane, whence one can change the \PMlinkname{order of integration}{FubinisTheorem} and accordingly write 
$$\int_{\alpha_1}^{\alpha_2}\left(\int_a^b\!f(x,\,\alpha)\,dx\right)d\alpha 
\;=\; \int_a^b\left(\int_{\alpha_1}^{\alpha_2}\!f(x,\,\alpha)\,d\alpha\right)dx.$$
Thus, a definite integral depending on a parametre may be integrated  with respect to this parametre by performing the integration under the integral sign.\\


\textbf{Example.}\, For being able to evaluate the improper integral
$$I \;=\; \int_0^\infty\frac{e^{-ax}-e^{-bx}}{x}\,dx \qquad (a > 0,\; b > 0),$$
we may interprete the integrand as a definite integral:
$$\frac{e^{-ax}-e^{-bx}}{x} \;=\; \sijoitus{\alpha=b}{\quad a}\!\frac{e^{-\alpha x}}{x} 
\;=\; \int_a^b\!e^{-\alpha x}\,d\alpha.$$
Accordingly, we can calculate as follows:
\begin{align*}
I & \;=\; \int_0^\infty\left(\int_a^b\!e^{-\alpha x}\,d\alpha\right)dx\\
& \;=\; \int_a^b\left(\int_0^\infty\!e^{-\alpha x}\,dx\right)d\alpha\\
& \;=\; \int_a^b\left(\sijoitus{x=0}{\quad\infty}\!-\frac{e^{-\alpha x}}{\alpha}\right)d\alpha\\
& \;=\; \int_a^b\!\frac{1}{\alpha}\,d\alpha \;=\; \sijoitus{a}{\quad b}\!\ln{\alpha}\\ 
& \;=\; \ln\frac{b}{a}
\end{align*}


%%%%%
%%%%%
\end{document}
