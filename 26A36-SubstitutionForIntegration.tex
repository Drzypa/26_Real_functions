\documentclass[12pt]{article}
\usepackage{pmmeta}
\pmcanonicalname{SubstitutionForIntegration}
\pmcreated{2013-03-22 14:33:38}
\pmmodified{2013-03-22 14:33:38}
\pmowner{pahio}{2872}
\pmmodifier{pahio}{2872}
\pmtitle{substitution for integration}
\pmrecord{21}{36114}
\pmprivacy{1}
\pmauthor{pahio}{2872}
\pmtype{Theorem}
\pmcomment{trigger rebuild}
\pmclassification{msc}{26A36}
\pmsynonym{variable changing for integration}{SubstitutionForIntegration}
\pmsynonym{integration by substitution}{SubstitutionForIntegration}
\pmsynonym{substitution rule}{SubstitutionForIntegration}
\pmrelated{IntegrationOfRationalFunctionOfSineAndCosine}
\pmrelated{IntegrationOfFractionPowerExpressions}
\pmrelated{ChangeOfVariableInDefiniteIntegral}

% this is the default PlanetMath preamble.  as your knowledge
% of TeX increases, you will probably want to edit this, but
% it should be fine as is for beginners.

% almost certainly you want these
\usepackage{amssymb}
\usepackage{amsmath}
\usepackage{amsfonts}

% used for TeXing text within eps files
%\usepackage{psfrag}
% need this for including graphics (\includegraphics)
%\usepackage{graphicx}
% for neatly defining theorems and propositions
 \usepackage{amsthm}
% making logically defined graphics
%%%\usepackage{xypic}

% there are many more packages, add them here as you need them

% define commands here
\theoremstyle{definition}
\newtheorem*{thmplain}{Theorem}
\begin{document}
For determining the antiderivative $F(x)$ of a given real function $f(x)$ in a  ``closed form'', i.e. for integrating $f(x)$, the result is often obtained by using the

\begin{thmplain}
\,\,If 
      $$\int f(x)\,dx = F(x)+C$$
and \,$x = x(t)$\, is a differentiable function,
then
\begin{align}
     F(x(t)) = \int f(x(t))\,x'(t)\,dt+c.
\end{align}
\end{thmplain}

{\em Proof.} \, By virtue of the chain rule,
   $$\frac{d}{dt}F(x(t)) = F'(x(t))\cdot x'(t),$$
and according to the supposition, $F'(x) = f(x)$. \,Thus we get the claimed equation (1).\\

\textbf{Remarks.}
\begin{itemize}
 \item The expression $x'(t)\,dt$ in (1) may be understood as the differential of $x(t)$.
 \item For returning to the original variable $x$, the inverse function \,$t = t(x)$\, of $x(t)$ must be substituted to $F(x(t))$.\\
\end{itemize}

\textbf{Example.} \, For integrating $\int \frac{x\,dx}{1+x^4}$ we take \,$x^2 = t$\, as a new variable. \,Then, \,$2x\,dx = dt$, $x\,dx = \frac{dt}{2}$, and we get
$$\int \frac{x\,dx}{1+x^4} = \frac{1}{2}\int \frac{dt}{1+t^2} = \frac{1}{2}\arctan t+ C= \frac{1}{2}\arctan x^2+C.$$
%%%%%
%%%%%
\end{document}
