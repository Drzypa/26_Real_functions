\documentclass[12pt]{article}
\usepackage{pmmeta}
\pmcanonicalname{TrigonometricIdentityInvolvingProductOfSinesOfRootsOfUnity}
\pmcreated{2013-03-22 19:00:03}
\pmmodified{2013-03-22 19:00:03}
\pmowner{rm50}{10146}
\pmmodifier{rm50}{10146}
\pmtitle{trigonometric identity involving product of sines of roots of unity}
\pmrecord{11}{41869}
\pmprivacy{1}
\pmauthor{rm50}{10146}
\pmtype{Theorem}
\pmcomment{trigger rebuild}
\pmclassification{msc}{26A09}
\pmclassification{msc}{33B10}

\endmetadata

% this is the default PlanetMath preamble.  as your knowledge
% of TeX increases, you will probably want to edit this, but
% it should be fine as is for beginners.

% almost certainly you want these
\usepackage{amssymb}
\usepackage{amsmath}
\usepackage{amsfonts}

% used for TeXing text within eps files
%\usepackage{psfrag}
% need this for including graphics (\includegraphics)
%\usepackage{graphicx}
% for neatly defining theorems and propositions
\usepackage{amsthm}
% making logically defined graphics
%%%\usepackage{xypic}

% there are many more packages, add them here as you need them

% define commands here
\newtheorem{thm}{Theorem}
\newtheorem{cor}[thm]{Corollary}
\newtheorem{lem}[thm]{Lemma}
\begin{document}
\PMlinkescapeword{simple}
\PMlinkescapeword{proof}
Let $n>1$ be a positive integer, and $\zeta_n=e^{2i\pi/n}$, a primitive $n^{\mathrm{th}}$ root of unity.

The purpose of this article is to prove
\begin{thm} \label{thm:one}Let $m=\left\lfloor\frac{n}{2}\right\rfloor$. Then
\begin{equation}
 \prod_{k=1}^m \sin^2\left(\frac{\pi k}{n}\right) = \prod_{k=1}^{n-1} \sin\left(\frac{\pi k}{n}\right) 
    = \frac{n}{2^{n-1}}
\end{equation}
\end{thm}

The theorem follows easily from the following simple lemma:
\begin{lem} Let $n>1$ be a positive integer. Then 
\[
  \prod_{k=1}^{n-1} (1-\zeta_n^k) = n
\]
\end{lem}
\begin{proof}
We have $x^n-1 = \prod_{k=1}^n (x-\zeta_n^k)$. Dividing both sides by $x-1$ gives
\[
  \frac{x^n-1}{x-1} = 1+x+x^2+\dots x^{n-1} = \prod_{k=1}^{n-1} (x-\zeta_n^k)
\]
Substitute $x=1$ to get the result.
\end{proof}

\begin{proof} [Proof of Theorem \ref{thm:one}]
Using the definition of $\zeta_n$ and the half-angle formulas, we have
\begin{align*}
  1-\zeta_n^k &= 1-\cos\left(\frac{2\pi k}{n}\right) - i\sin\left(\frac{2\pi k}{n}\right) \\
              &= 2\sin^2\left(\frac{\pi k}{n}\right) - 
                   2i\sin\left(\frac{\pi k}{n}\right)\cos\left(\frac{\pi k}{n}\right) \\
              &= 2\sin\left(\frac{\pi k}{n}\right)
                   \left(\sin\left(\frac{\pi k}{n}\right)-i\cos\left(\frac{\pi k}{n}\right)\right)
\end{align*}
Note that $\lvert \sin\theta - i\cos\theta\rvert = \sin^2\theta + \cos^2\theta=1$, so taking absolute values, we get
\[
  \left\lvert 1-\zeta_n^k\right\rvert = 2\left\lvert \sin\left(\frac{\pi k}{n}\right)\right\rvert
\]
Now, for $1\leq k\leq n-1$, $\sin\left(\frac{\pi k}{n}\right) > 0$ so is equal to its absolute value. Thus (using, for $n$ even, the fact that $\sin\frac{\pi}{2}=1$),
\begin{align*}
  \prod_{k=1}^m \sin^2\left(\frac{\pi k}{n}\right)
     &= \prod_{k=1}^m \sin\left(\frac{\pi k}{n}\right) \sin\left(\pi - \frac{\pi k}{n}\right) 
      = \prod_{k=1}^m \sin\left(\frac{\pi k}{n}\right) \sin\left(\frac{\pi(n-k)}{n}\right) \\
     &= \prod_{k=1}^{n-1} \sin\left(\frac{\pi k}{n}\right) 
      = \prod_{k=1}^{n-1} \left\lvert\sin\left(\frac{\pi k}{n}\right)\right\rvert \\
     &= \frac{1}{2^{n-1}}\left\lvert \prod_{k=1}^{n-1} (1-\zeta_n^k)\right\rvert
      = \frac{1}{2^{n-1}} \left\lvert n\right\rvert \\
     &= \frac{n}{2^{n-1}}
\end{align*}
\end{proof}

(Thanks to dh2718 for greatly simplifying the original proof.)
%%%%%
%%%%%
\end{document}
