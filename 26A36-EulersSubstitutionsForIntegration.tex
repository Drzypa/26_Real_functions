\documentclass[12pt]{article}
\usepackage{pmmeta}
\pmcanonicalname{EulersSubstitutionsForIntegration}
\pmcreated{2013-03-22 17:19:43}
\pmmodified{2013-03-22 17:19:43}
\pmowner{pahio}{2872}
\pmmodifier{pahio}{2872}
\pmtitle{Euler's substitutions for integration}
\pmrecord{15}{39681}
\pmprivacy{1}
\pmauthor{pahio}{2872}
\pmtype{Topic}
\pmcomment{trigger rebuild}
\pmclassification{msc}{26A36}
\pmsynonym{integration of expressions of square roots of quadratic polynomials}{EulersSubstitutionsForIntegration}
%\pmkeywords{square root of quadratic polynomial}
\pmrelated{IntegrationOfRationalFunctionOfSineAndCosine}
\pmrelated{Tractrix}
\pmrelated{Arsinh}
\pmrelated{Arcosh}
\pmdefines{Euler's substitutions}
\pmdefines{substitutions of Euler}

\endmetadata

% this is the default PlanetMath preamble.  as your knowledge
% of TeX increases, you will probably want to edit this, but
% it should be fine as is for beginners.

% almost certainly you want these
\usepackage{amssymb}
\usepackage{amsmath}
\usepackage{amsfonts}

% used for TeXing text within eps files
%\usepackage{psfrag}
% need this for including graphics (\includegraphics)
%\usepackage{graphicx}
% for neatly defining theorems and propositions
%\usepackage{amsthm}
% making logically defined graphics
%%%\usepackage{xypic}

% there are many more packages, add them here as you need them

\DeclareMathOperator{\arsinh}{arsinh}
\DeclareMathOperator{\arcosh}{arcosh}
\DeclareMathOperator{\artanh}{artanh}
\DeclareMathOperator{\arcoth}{arcoth}
\begin{document}
In the integration task
               $$\int\!R(x,\,\sqrt{ax^2+bx+c})\,dx,$$
where the integrand is a rational function of $x$ and $\sqrt{ax^2+bx+c}$, the integrand can be changed to a rational function of a new variable $t$ by using the following substitutions of Euler.

\begin{itemize}

\item \textbf{The first substitution of Euler.}\; If\, $a > 0$,\, we may write
\begin{align}
               \sqrt{ax^2+bx+c} \;=\; \pm x\sqrt{a}+t.
\end{align}
When we take $\sqrt{a}$ with the minus sign, then
$$ax^2+bx+c \;=\; ax^2-2xt\sqrt{a}+t^2,$$
from which we get the expression
$$x \;=\; \frac{t^2-c}{b+2t\sqrt{a}};$$
thus also $dx$ is expressible rationally via $t$.  We have
$$\sqrt{ax^2+bx+c} \;=\; -x\sqrt{a}+t \;=\; \frac{c-t^2}{b+2t\sqrt{a}}\sqrt{a}+t.$$

\item \textbf{The second substitution of Euler.}\; If\, $c > 0$,\, we take
\begin{align}
                \sqrt{ax^2+bx+c} \;=\; xt\pm\sqrt{c}.
\end{align}
With the minus sign we obtain, similarly as above,
$$x \;=\; \frac{2t\sqrt{c}+b}{t^2-a}.$$

\item \textbf{The third substitution of Euler.}\; If the polynomial $ax^2\!+\!bx\!+\!c$ has the real zeros $\alpha$ and $\beta$, we may chose
\begin{align}
\sqrt{ax^2\!+\!bx\!+\!c} \;=\; (x\!-\!\alpha)t.
\end{align}
Now
$$ax^2\!+\!bx\!+\!c \;=\; a(x\!-\!\alpha)(x\!-\!\beta) \;=\; (x\!-\!\alpha)^2t^2,$$
whence\, $a(x\!-\!\beta) = (x\!-\!\alpha)t^2$.  This gives the expression
$$x \;=\; \frac{a\beta\!-\!\alpha t^2}{a\!-\!t^2}.$$
As in the preceding cases, we can express $dx$ and $\sqrt{ax^2\!+\!bx\!+\!c}$ rationally via $t$.

\end{itemize}

\textbf{Examples.}

\textbf{1.}  In the integral $\displaystyle\int\!\frac{dx}{\sqrt{x^2+c}}$ we can use the first substitution:\, $\sqrt{x^2+c} = -x+t$;\, then\, $x^2+c = x^2-2xt+t^2$\, and thus
$$x \;=\; \frac{t^2-c}{2t},\quad dx \;=\; \frac{t^2+c}{2t^2}\,dt,\quad \sqrt{x^2+c} \;=\; -\frac{t^2-c}{2t}+t \;=\; \frac{t^2+c}{2t}.$$
Accordingly we obtain
$$\int\!\frac{dx}{\sqrt{x^2+c}} \;=\; \int\!\frac{\frac{t^2+c}{2t^2}dt}{\frac{t^2+c}{2t}} \;=\; \int\!\frac{dt}{t} 
\;=\; \ln|t|+C \;=\; \ln|x+\sqrt{x^2+c}|+C.$$
Especially the cases \,$c = \pm1$\, give the formulas
$$\int\!\frac{dx}{\sqrt{x^2+1}} \;=\; \arsinh{x}+C,  \quad \int\!\frac{dx}{\sqrt{x^2-1}} \;=\; \arcosh{x}+C \;\; 
(x > 1).$$ 

\textbf{2.}  The integral $\displaystyle\int\!\frac{\sqrt{c^2-x^2}}{x}\,dx$ is needed in deriving the equation of the tractrix.  We use for integrating the second substitution\, 
$\sqrt{c^2-x^2} = xt-c$;\, then\, $c^2-x^2 = x^2t^2-2cxt+c^2$, which implies
$$x \;=\; \frac{2ct}{t^2+1},\quad dx \;=\; \frac{2c(1-t^2)dt}{(1+t^2)^2},\quad 
\sqrt{c^2-x^2} \;=\; \frac{2ct^2}{t^2+1}-c \;=\; \frac{c(t^2-1)}{t^2+1}.$$ 
We then obtain
$$\int\!\frac{\sqrt{c^2-x^2}}{x}\,dx \;=\; -c\int\!\frac{(1-t^2)^2}{t(1+t^2)^2}\,dt \;=\; 
c\int\!\left(\frac{4t}{(1+t^2)^2}-\frac{1}{t}\right)dt \;=\; -\frac{2c}{1+t^2}-c\ln|t|+C_1.$$
The equation tying $x$ and $t$ gives\, $\frac{2c}{1+t^2} = \frac{x}{t}$\, and\, $t = \frac{c+\sqrt{c^2-x^2}}{x}$,\, whence
$$\int\!\frac{\sqrt{c^2-x^2}}{x}\,dx \;=\; -\frac{x^2}{c+\sqrt{c^2-x^2}}-c\ln\frac{c+\sqrt{c^2-x^2}}{x}+C_1 \;=\; 
-c+\sqrt{c^2-x^2}-c\ln\frac{c+\sqrt{c^2-x^2}}{x}+C_1,$$
i.e. 
$$\int\!\frac{\sqrt{c^2-x^2}}{x}\,dx \;=\; \sqrt{c^2-x^2}-c\ln\frac{c+\sqrt{c^2-x^2}}{x}+C.$$

\textbf{3.}  In the integral $\displaystyle\int\!\frac{dx}{\sqrt{x^2+3x-4}}$, the radicand is $(x+4)(x-1)$.  Using the third substitution of Euler, we take\, $\sqrt{x^2+4x-3} = (x+4)t$.  This simplifies to\, $x-1 = (x+4)t^2$.  Then we get
$$x \;=\; \frac{1+4t^2}{1-t^2},\quad dx \;=\; \frac{10t}{(1-t^2)^2}\,dt,\quad 
\sqrt{x^2+3x-4} \;=\; \left(\frac{1+4t^^2}{1-t^2}+4\right)t \;=\; \frac{5t}{1-t^2}.$$
And we obtain
$$\int\!\frac{dx}{\sqrt{x^2+3x-4}} \;=\; \int\!\frac{10t(1-t^2)}{(1-t^2)^2\cdot5t}\,dt \;=\; \int\!\frac{2}{1-t^2}\,dt
= \ln\left|\frac{1+t}{1-t}\right|+C \;=\;
 \ln\left|\frac{1+\sqrt{\frac{x-1}{x+4}}}{1-\sqrt{\frac{x-1}{x+4}}}\right|+C$$
$$=\; \ln\left|\frac{\sqrt{x+4}+\sqrt{x-1}}{\sqrt{x+4}-\sqrt{x-1}}\right|+C.$$



\begin{thebibliography}{9}
\bibitem{NP}{\sc N. Piskunov:} {\em Diferentsiaal- ja integraalarvutus k\~{o}rgematele tehnilistele \~{o}ppeasutustele}.  Viies, t\"aiendatud tr\"ukk.\, Kirjastus ``Valgus'', Tallinn  (1965).
\end{thebibliography}
%%%%%
%%%%%
\end{document}
