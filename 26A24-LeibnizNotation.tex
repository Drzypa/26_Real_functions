\documentclass[12pt]{article}
\usepackage{pmmeta}
\pmcanonicalname{LeibnizNotation}
\pmcreated{2013-03-22 12:30:47}
\pmmodified{2013-03-22 12:30:47}
\pmowner{mathcam}{2727}
\pmmodifier{mathcam}{2727}
\pmtitle{Leibniz notation}
\pmrecord{6}{32750}
\pmprivacy{1}
\pmauthor{mathcam}{2727}
\pmtype{Topic}
\pmcomment{trigger rebuild}
\pmclassification{msc}{26A24}
\pmrelated{Derivative}
\pmrelated{FixedPointsOfNormalFunctions}
\pmrelated{Differential}

\usepackage{amssymb}
\usepackage{amsmath}
\usepackage{amsfonts}
\begin{document}
\PMlinkescapeword{centers}

\emph{Leibniz notation} centers around the concept of a \emph{differential element}.
The differential element of $x$ is represented by $dx$.
You might think of $dx$ as being an infinitesimal change in $x$. It is important
to note that $d$ is an operator, not a variable. So, when you see $\frac{dy}{dx}$,
you can't automatically write as a replacement $\frac{y}{x}$.

We use $\frac{df(x)}{dx}$ or $\frac{d}{dx}f(x)$ to represent the derivative of a
function $f(x)$ with respect to $x$.
$$ \frac{df(x)}{dx} = \lim_{Dx \to 0} \frac{f(x+Dx) - f(x)}{Dx} $$
We are dividing two numbers infinitely close to 0,
and arriving at a finite answer. $D$ is another operator that can be
thought of just a change in $x$. When we take the limit of $Dx$ as $Dx$ approaches 0,
we get an infinitesimal change $dx$.

Leibniz notation shows a wonderful use in the following example:
$$ \frac{dy}{dx} = \frac{dy}{dx} \frac{du}{du} = \frac{dy}{du} \frac{du}{dx} $$
The two $du$s can be cancelled out to arrive at the original derivative.
This is the Leibniz notation for the Chain Rule.

Leibniz notation shows up in the most common way of representing an integral,
$$ F(x) = \int f(x) dx $$
The $dx$ is in fact a differential element. Let's start with a derivative that
we know (since $F(x)$ is an antiderivative of $f(x)$).
\begin{eqnarray*}
\frac{dF(x)}{dx} & = & f(x) \\
dF(x) & = & f(x)dx \\
\int dF(x) & = & \int f(x)dx \\
F(x) & = & \int f(x) dx 
\end{eqnarray*}
We can think of $dF(x)$ as the differential element of area. Since $dF(x) = f(x) dx$,
the element of area is a rectangle, with $f(x) \times dx$ as its dimensions. Integration is
the sum of all these infinitely thin elements of area along a certain interval. The result: a finite number.

(a diagram is deserved here)

One clear advantage of this notation is seen when finding the length $s$ of a curve.
The formula is often seen as the following:
$$ s = \int ds $$
The length is the sum of all the elements, $ds$, of length. If we have a function
$f(x)$, the length element is usually written as $ ds = \sqrt{1+[\frac{df(x)}{dx}]^2} dx $. If we
modify this a bit, we get $ ds = \sqrt{[dx]^2 + [df(x)]^2} $. Graphically, we
could say that the length element is the hypotenuse of a right triangle with one
leg being the $x$ element, and the other leg being the $f(x)$ element.

(another diagram would be nice!)

There are a few caveats, such as if you want to take the value of a
derivative. Compare to the prime notation.
$$ f'(a) = \left. \frac{df(x)}{dx} \right |_{x=a} $$

A second derivative is represented as follows:
$$ \frac{d}{dx} \frac{dy}{dx} = \frac{d^2y}{dx^2} $$
The other derivatives follow as can be expected: $\frac{d^3y}{dx^3}$, etc.
You might think this is a little sneaky, but it is the notation. Properly using
these terms can be interesting. For example, what is $\int \frac{d^2y}{dx} $? We
could turn it into $\int \frac{d^2y}{dx^2} dx$ or $\int d\frac{dy}{dx} $.
Either way, we get $\frac{dy}{dx}$.
%%%%%
%%%%%
\end{document}
