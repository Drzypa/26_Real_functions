\documentclass[12pt]{article}
\usepackage{pmmeta}
\pmcanonicalname{YoungsInequality}
\pmcreated{2013-03-22 13:19:25}
\pmmodified{2013-03-22 13:19:25}
\pmowner{rspuzio}{6075}
\pmmodifier{rspuzio}{6075}
\pmtitle{Young's inequality}
\pmrecord{8}{33834}
\pmprivacy{1}
\pmauthor{rspuzio}{6075}
\pmtype{Theorem}
\pmcomment{trigger rebuild}
\pmclassification{msc}{26D15}
%\pmkeywords{Young's Inequality}
\pmrelated{YoungInequality}

% this is the default PlanetMath preamble.  as your knowledge
% of TeX increases, you will probably want to edit this, but
% it should be fine as is for beginners.

% almost certainly you want these
\usepackage{amssymb}
\usepackage{amsmath}
\usepackage{amsfonts}

% used for TeXing text within eps files
%\usepackage{psfrag}
% need this for including graphics (\includegraphics)
%\usepackage{graphicx}
% for neatly defining theorems and propositions
%\usepackage{amsthm}
% making logically defined graphics
%%%\usepackage{xypic} 

% there are many more packages, add them here as you need them

% define commands here
\begin{document}
Let $\phi : \mathbb{R} \rightarrow \mathbb{R}$ be a continuous , strictly
increasing function such that $\phi(0)=0$ . Then the following inequality holds:
$$ ab \leq \int_{0}^a \phi(x) dx + \int_{0}^b \phi^{-1}(y) dy $$
Equality only holds when $b = \phi(a)$.
This inequality can be demonstrated by drawing the graph of $\phi(x)$
and by observing that the sum of the two areas represented by the integrals
above is greater than the area of a rectangle of sides $a$ and $b$, as
is illustrated in \PMlinkid{an attachment}{5575}.
%%%%%
%%%%%
\end{document}
