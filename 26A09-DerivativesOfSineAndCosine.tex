\documentclass[12pt]{article}
\usepackage{pmmeta}
\pmcanonicalname{DerivativesOfSineAndCosine}
\pmcreated{2013-03-22 16:58:58}
\pmmodified{2013-03-22 16:58:58}
\pmowner{Wkbj79}{1863}
\pmmodifier{Wkbj79}{1863}
\pmtitle{derivatives of sine and cosine}
\pmrecord{10}{39259}
\pmprivacy{1}
\pmauthor{Wkbj79}{1863}
\pmtype{Derivation}
\pmcomment{trigger rebuild}
\pmclassification{msc}{26A09}
\pmrelated{DerivativesOfSinXAndCosX}
\pmrelated{LimitOfDisplaystyleFracsinXxAsXApproaches0}
\pmrelated{DefinitionsInTrigonometry}
\pmrelated{LimitRulesOfFunctions}

\endmetadata

% this is the default PlanetMath preamble.  as your knowledge
% of TeX increases, you will probably want to edit this, but
% it should be fine as is for beginners.

% almost certainly you want these
\usepackage{amssymb}
\usepackage{amsmath}
\usepackage{amsfonts}

% used for TeXing text within eps files
%\usepackage{psfrag}
% need this for including graphics (\includegraphics)
%\usepackage{graphicx}
% for neatly defining theorems and propositions
 \usepackage{amsthm}
% making logically defined graphics
%%%\usepackage{xypic}

% there are many more packages, add them here as you need them

% define commands here

\theoremstyle{definition}
\newtheorem*{thmplain}{Theorem}

\begin{document}
The \PMlinkescapetext{derivation} of the derivatives of sine and cosine is a bit simpler by using the prosthaphaeresis formulas
\begin{align}
\sin\alpha-\sin\beta = \,2\sin \left( \frac{\alpha\!-\!\beta}{2} \right) \,\cos \left( \frac{\alpha\!+\!\beta}{2} \right),
\end{align}
\begin{align}
\cos\alpha-\cos\beta = -2\sin \left( \frac{\alpha\!+\!\beta}{2} \right) \, \sin\left( \frac{\alpha\!-\!\beta}{2} \right).
\end{align}

Let $x,\,t$ be any real numbers such that\, $t \neq x$.\, Then we obtain
$$\frac{\sin{x}-\sin{t}}{x-t} = 
\frac{2\sin \left( \frac{x-t}{2} \right) \cos \left( \frac{x+t}{2} \right) }{x-t} =
\frac{\sin \left( \frac{x-t}{2} \right) }{\left( \frac{x-t}{2} \right) }\cdot\cos \left( \frac{x\!+\!t}{2} \right) \;\;
\longrightarrow\; 1\cdot\cos \left( \frac{x\!+\!x}{2} \right) = \cos{x},$$
as\; $t\to x$.\, Here we used the known limit \;$\displaystyle\lim_{u\to0}\frac{\sin{u}}{u} = 1$\; (see \PMlinkname{this entry}{LimitOfDisplaystyleFracsinXxAsXApproaches0}).

The derivative of cosine is calculated similarly:
$$\frac{\cos{x}-\cos{t}}{x-t} = \frac{-2\sin \left( \frac{x+t}{2} \right) \sin\left( \frac{x-t}{2} \right)}{x-t} =-1 \cdot \frac{\sin\left( \frac{x-t}{2} \right) }{\left( \frac{x-t}{2} \right) }\cdot \sin \left( \frac{x\!+\!t}{2} \right) \;\; \longrightarrow\; -1 \cdot 1\cdot \sin \left( \frac{x\!+\!x}{2} \right) =-\sin{x},$$
as\; $t\to x$.
%%%%%
%%%%%
\end{document}
