\documentclass[12pt]{article}
\usepackage{pmmeta}
\pmcanonicalname{PeriodicPoint}
\pmcreated{2013-03-22 12:43:38}
\pmmodified{2013-03-22 12:43:38}
\pmowner{mathwizard}{128}
\pmmodifier{mathwizard}{128}
\pmtitle{periodic point}
\pmrecord{14}{33026}
\pmprivacy{1}
\pmauthor{mathwizard}{128}
\pmtype{Definition}
\pmcomment{trigger rebuild}
\pmclassification{msc}{26A18}
\pmdefines{hyperbolic periodic point}
\pmdefines{attractive periodic point}
\pmdefines{repelling periodic point}
\pmdefines{least period}
\pmdefines{prime period}

\endmetadata

% this is the default PlanetMath preamble.  as your knowledge
% of TeX increases, you will probably want to edit this, but
% it should be fine as is for beginners.

% almost certainly you want these
\usepackage{amssymb}
\usepackage{amsmath}
\usepackage{amsfonts}

% used for TeXing text within eps files
%\usepackage{psfrag}
% need this for including graphics (\includegraphics)
%\usepackage{graphicx}
% for neatly defining theorems and propositions
%\usepackage{amsthm}
% making logically defined graphics
%%%\usepackage{xypic}

% there are many more packages, add them here as you need them

% define commands here
\begin{document}
\PMlinkescapeword{period}
\PMlinkescapeword{prime}
Let $f:X\to X$ be a function and $f^n$ its $n$-th iteration. A point $x$ is called a \textit{periodic point} of period $n$ of $f$ if it is a fixed point of $f^n$. The least $n$ for which $x$ is a fixed point of $f^n$ is called \textit{prime period} or \textit{least period}.

If $f$ is a function \PMlinkescapetext{mapping} $\mathbb{R}$ to $\mathbb{R}$ or $\mathbb{C}$ to $\mathbb{C}$ then a periodic point $x$ of prime period $n$ is called \textit{hyperbolic} if $|(f^n)'(x)|\neq 1$, \textit{attractive} if $|(f^n)'(x)|<1$ and \textit{repelling} if $|(f^n)'(x)|>1$.
%%%%%
%%%%%
\end{document}
