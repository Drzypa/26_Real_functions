\documentclass[12pt]{article}
\usepackage{pmmeta}
\pmcanonicalname{LipschitzCondition}
\pmcreated{2013-03-22 11:57:48}
\pmmodified{2013-03-22 11:57:48}
\pmowner{paolini}{1187}
\pmmodifier{paolini}{1187}
\pmtitle{Lipschitz condition}
\pmrecord{27}{30765}
\pmprivacy{1}
\pmauthor{paolini}{1187}
\pmtype{Definition}
\pmcomment{trigger rebuild}
\pmclassification{msc}{26A16}
\pmsynonym{Lipschitz}{LipschitzCondition}
\pmsynonym{Lipschitz continuous}{LipschitzCondition}
\pmrelated{RademachersTheorem}
\pmrelated{NewtonsMethod}
\pmrelated{KantorovitchsTheorem}
\pmdefines{Holder}
\pmdefines{Holder continuous}
\pmdefines{Lipschitz constant}

\endmetadata

\usepackage{amsmath}
\usepackage{amsfonts}
\usepackage{amssymb}
\newcommand{\reals}{\mathbb{R}}
\newcommand{\natnums}{\mathbb{N}}
\newcommand{\cnums}{\mathbb{C}}
\newcommand{\znums}{\mathbb{Z}}
\newcommand{\lp}{\left(}
\newcommand{\rp}{\right)}
\newcommand{\lb}{\left[}
\newcommand{\rb}{\right]}
\newcommand{\supth}{^{\text{th}}}
\newtheorem{proposition}{Proposition}
\newtheorem{definition}[proposition]{Definition}

\newtheorem{theorem}[proposition]{Theorem}
\begin{document}
A mapping $f: X \to Y$ between metric spaces is said to satisfy the
Lipschitz condition, or to be \emph{Lipschitz continuous} or \emph{$L$-Lipschitz} if there exists a real constant $L$ such
that
$$ d_Y(f(p),f(q)) \leq L d_X(p,q),\quad \text{for all}\; p,q\in X.$$

The least constant $L$ for which the previous inequality holds, is called the \emph{Lipschitz constant} of $f$.
The space of Lipschitz continuous functions is often denoted by $\mathrm{Lip}(X,Y)$.

Clearly, every Lipschitz continuous function is continuous.

\paragraph{Notes.}
More generally, one says that a mapping satisfies
a Lipschitz condition of order $\alpha>0$ if there exists a real constant $C$ such that
$$ d_Y(f(p),f(q)) \leq C d_X(p,q)^\alpha,\quad \text{for all}\; p,q\in X.$$

Functions which satisfy this condition are also called \emph{H{\"o}lder continuous} or \emph{$\alpha$-H{\"o}lder}. The vector space of such functions is denoted by $C^{0,\alpha}(X,Y)$ and hence $\mathrm{Lip}=C^{0,1}$.
%%%%%
%%%%%
%%%%%
\end{document}
