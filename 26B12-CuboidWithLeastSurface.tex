\documentclass[12pt]{article}
\usepackage{pmmeta}
\pmcanonicalname{CuboidWithLeastSurface}
\pmcreated{2013-03-22 17:24:00}
\pmmodified{2013-03-22 17:24:00}
\pmowner{pahio}{2872}
\pmmodifier{pahio}{2872}
\pmtitle{cuboid with least surface}
\pmrecord{7}{39770}
\pmprivacy{1}
\pmauthor{pahio}{2872}
\pmtype{Example}
\pmcomment{trigger rebuild}
\pmclassification{msc}{26B12}
\pmdefines{cuboid}

% this is the default PlanetMath preamble.  as your knowledge
% of TeX increases, you will probably want to edit this, but
% it should be fine as is for beginners.

% almost certainly you want these
\usepackage{amssymb}
\usepackage{amsmath}
\usepackage{amsfonts}

% used for TeXing text within eps files
%\usepackage{psfrag}
% need this for including graphics (\includegraphics)
%\usepackage{graphicx}
% for neatly defining theorems and propositions
 \usepackage{amsthm}
% making logically defined graphics
%%%\usepackage{xypic}
\usepackage{pstricks}
\usepackage{pst-plot}

% there are many more packages, add them here as you need them

% define commands here

\theoremstyle{definition}
\newtheorem*{thmplain}{Theorem}

\begin{document}
Let us determine among all {\em cuboids} (i.e. rectangular parallelepipeds) with a given volume $k^3$ such one which has the least surface area.

Let the three edges of the cuboid beginning from a vertex be $x$, $y$ and $z$; then we must start from the condition\, $xyz = k^3$,\, whence\, $z = \frac{k^3}{xy}$.  We get the expression
\begin{align}
f(x,\,y) := 2(yz\!+\!zx\!+\!xy) = 2\!\left(\!xy+\frac{k^3}{x}+\frac{k^3}{y}\!\right)
\end{align}
for the whole area of the \PMlinkescapetext{surface} of the cuboid.  Thus we have to make\, $f(x,\,y)$ a minimum, when only the positive values of $x$ and $y$ can be taken into consideration.  

The function $f$ and its first order partial derivatives are continuous for all positive $x$ and $y$.  According to the \PMlinkescapetext{theorem} of the \PMlinkname{parent entry}{ExtremumPointsOfFunctionOfSeveralVariables}, a minimum can occur only when simultaneously
\begin{align*}
\begin{cases}
{f'_x(x,\,y) = y-\frac{k^3}{x^2} = 0},\\
{f'_y(x,\,y) = x-\frac{k^3}{y^2} = 0}.
\end{cases}
\end{align*}
These equations are true only for\, $x = y = k$,\, i.e. for the case that the cuboid is a cube.  
We can infer that a cube has the least area.  In fact, we see from (1) that\, $f(x,\,y)\to\infty$\, as\, $x\to 0$\, or $y\to 0$\, or\, $xy\to\infty$;\, therefore there exist a small positive number $m$ and a big positive number $M$ such that outside and on the boundary of the region resembling a triangle and bounded by the lines \, $x = m$\, and\, $y = m$\, and the rectangular hyperbola \, $xy = M$,\, the value of\, $f(x,\,y)$\, is always greater than in the point \, $(k,\,k)$\, inside this region.  Thus the function gets its smallest value in an interior point of the region, and this point must be\, $(k,\,k)$\, since it is the only point where $f'_x$ and $f'_y$ both vanish.\\

\begin{center}
\begin{pspicture}(-0.5,-0.5)(3.5,4)
\psaxes[Dx=9,Dy=9]{->}(0,0)(-0.5,-0.5)(5.5,4.5)
\rput(5.4,-0.2){$x$}
\rput(0.2,4.44){$y$}
\rput(-0.2,-0.2){$0$}
\psline[linecolor=blue](0.7,0)(0.7,4.4)
\psline[linecolor=blue](0,0.7)(4.9,0.7)
\psplot[linecolor=blue]{0.57}{4.9}{2.5 x div}
\rput(0.7,-0.25){$m$}
\rput(-0.25,0.7){$m$}
\rput(2,2.4){$xy = M$}
\psdot[linecolor=red](1,1)
\rput(1.4,1.15){$^{(k,\,k)}$}
\end{pspicture}
\end{center}
%%%%%
%%%%%
\end{document}
