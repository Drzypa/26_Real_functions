\documentclass[12pt]{article}
\usepackage{pmmeta}
\pmcanonicalname{AreaOfSphericalZone}
\pmcreated{2013-03-22 18:19:05}
\pmmodified{2013-03-22 18:19:05}
\pmowner{pahio}{2872}
\pmmodifier{pahio}{2872}
\pmtitle{area of spherical zone}
\pmrecord{9}{40944}
\pmprivacy{1}
\pmauthor{pahio}{2872}
\pmtype{Derivation}
\pmcomment{trigger rebuild}
\pmclassification{msc}{26B15}
\pmclassification{msc}{53A05}
\pmclassification{msc}{51M04}
\pmsynonym{area of spherical calotte}{AreaOfSphericalZone}
%\pmkeywords{area of sphere}
\pmrelated{AreaOfTheNSphere}
\pmrelated{CentreOfMassOfHalfDisc}

% this is the default PlanetMath preamble.  as your knowledge
% of TeX increases, you will probably want to edit this, but
% it should be fine as is for beginners.

% almost certainly you want these
\usepackage{amssymb}
\usepackage{amsmath}
\usepackage{amsfonts}

% used for TeXing text within eps files
%\usepackage{psfrag}
% need this for including graphics (\includegraphics)
%\usepackage{graphicx}
% for neatly defining theorems and propositions
 \usepackage{amsthm}
% making logically defined graphics
%%%\usepackage{xypic}

% there are many more packages, add them here as you need them

% define commands here

\theoremstyle{definition}
\newtheorem*{thmplain}{Theorem}

\begin{document}
\PMlinkescapeword{formula}
Let us consider the circle 
$$(x\!-\!r)^2\!+\!y^2 \;=\; r^2$$
with radius $r$ and centre \,$(r,\,0)$.\, A spherical zone may be thought to be formed when an arc of the circle rotates around the $x$-axis.\, For finding the are of the zone, we can use the formula
\begin{align}
A \;=\; 2\pi\!\int_{a}^{b}\!y\,\sqrt{1+\left(\frac{dy}{dx}\right)^2}\,dx
\end{align}
of the entry area of surface of revolution.\, Let the ends of the arc correspond the values $a$ and $b$ of the abscissa such that\, $b\!-\!a = h$\, is the \PMlinkescapetext{height} of the spherical zone.\, In the formula, we must use the solved form
$$y \;=\; (\pm)\sqrt{rx\!-\!x^2}$$
of the equation of the circle.\, The formula then yields
$$A \;=\; 2\pi\!\int_a^b\sqrt{rx\!-\!x^2}\,\sqrt{1+\left(\frac{r\!-\!x}{\sqrt{rx\!-\!x^2}}\right)^2}\,dx 
\;=\; 2\pi\!\int_a^br\,dx \;=\; 2\pi r(b\!-\!a).$$
Hence the area of a spherical zone (and also of a spherical calotte) is
\begin{align}
A \;=\; 2\pi rh.
\end{align}
From here one obtains as a special case \,$h = 2r$\, the area of the whole sphere:
\begin{align}
A \;=\; 4\pi r^2.
\end{align}


\textbf{Remark.}\, The formula (2) implies that the centre of mass of a half-sphere is at the halfway point of the axis of symmetry ($h = \frac{r}{2}$).


%%%%%
%%%%%
\end{document}
