\documentclass[12pt]{article}
\usepackage{pmmeta}
\pmcanonicalname{ExampleOfRiemannTripleIntegral}
\pmcreated{2013-03-22 19:10:59}
\pmmodified{2013-03-22 19:10:59}
\pmowner{pahio}{2872}
\pmmodifier{pahio}{2872}
\pmtitle{example of Riemann triple integral}
\pmrecord{14}{42091}
\pmprivacy{1}
\pmauthor{pahio}{2872}
\pmtype{Example}
\pmcomment{trigger rebuild}
\pmclassification{msc}{26A42}
\pmsynonym{volume as triple integral}{ExampleOfRiemannTripleIntegral}
\pmrelated{Volume2}
\pmrelated{VolumeAsIntegral}
\pmrelated{SubstitutionNotation}
\pmrelated{ChangeOfVariablesInIntegralOnMathbbRn}
\pmrelated{ExampleOfRiemannDoubleIntegral}

\endmetadata

% this is the default PlanetMath preamble.  as your knowledge
% of TeX increases, you will probably want to edit this, but
% it should be fine as is for beginners.

% almost certainly you want these
\usepackage{amssymb}
\usepackage{amsmath}
\usepackage{amsfonts}

% used for TeXing text within eps files
%\usepackage{psfrag}
% need this for including graphics (\includegraphics)
%\usepackage{graphicx}
% for neatly defining theorems and propositions
%\usepackage{amsthm}
% making logically defined graphics
%%%\usepackage{xypic}

% there are many more packages, add them here as you need them

% define commands here
\newcommand{\sijoitus}[2]%
{\operatornamewithlimits{\Big/}_{\!\!\!#1}^{\,#2}}
\begin{document}
Determine the volume of the solid \PMlinkescapetext{bounded} in $\mathbb{R}^3$ by the part of the surface
$$(x^2\!+\!y^2\!+\!z^2)^3 \;=\; 3a^3xyz$$
being in the first octant ($a > 0$).\\


Since $x^2\!+\!y^2\!+\!z^2$ is the squared distance of the point \,$(x,\,y,\,z)$\, from the origin, the solid is apparently defined by
$$D \;:=\; 
\{(x,\,y,\,z)\in\mathbb{R}^3\,\vdots\;\, x \geqq 0,\;\, y \geqq 0,\;\, z \geqq 0,\;\, 
(x^2\!+\!y^2\!+\!z^2)^3 \leqq 3a^3xyz\}.$$
By the definition
$$\mathbf{meas}(D) \;:=\: \int \chi_D(v)\, dv$$
in the \PMlinkname{parent entry}{RiemannMultipleIntegral}, the volume in the question
is
\begin{align}
V \;=\; \int_D1\,dv \;=\; \iiint_D dx\,dy\,dz.
\end{align}
For calculating the integral (1) we express it by the (geographic) spherical coordinates through
\begin{align*}
\begin{cases}
x \;=\; r\cos\varphi\cos\lambda\\
y \;=\; r\cos\varphi\sin\lambda\\
z \;=\; r\sin\varphi
\end{cases}
\end{align*}
where the latitude angle $\varphi$ of the position vector $\vec{r}$ is measured from the $xy$-plane (not as the colatitude $\phi$ from the positive $z$-axis); $\lambda$ is the longitude.\, For the change of coordinates, we need the Jacobian determinant
\begin{align*}
\frac{\partial(x,\,y,\,z)}{\partial(r,\,\varphi,\,\lambda)} \:=\;
 \left| \begin{matrix}
\frac{\partial x}{\partial r} & \frac{\partial y}{\partial r} & \frac{\partial z}{\partial r} \\
\frac{\partial x}{\partial \varphi} & \frac{\partial y}{\partial \varphi} & \frac{\partial z}{\partial \varphi} \\
\frac{\partial x}{\partial \lambda} & \frac{\partial y}{\partial \lambda} & \frac{\partial z}{\partial \lambda}
\end{matrix}\right| \;=\; 
 \left| \begin{matrix}
\cos\varphi\cos\lambda & \cos\varphi\sin\lambda & \sin\varphi \\
-r\sin\varphi\cos\lambda & -r\sin\varphi\sin\lambda & r\cos\varphi \\
-r\cos\varphi\sin\lambda & r\cos\varphi\cos\lambda & 0
\end{matrix}\right|,
\end{align*}
which is simplified to $r^2\cos\varphi$.\, The equation of the surface attains the form
$$r^6 \;=\; 3a^3r^3\cos^2\varphi\sin\varphi\cos\lambda\sin\lambda,$$
or
$$r \;=\; \sqrt[3]{3a^3\cos^2\varphi\sin\varphi\cos\lambda\sin\lambda} \;:=\; r(\varphi,\,\lambda).$$
In the solid, we have\, $0 \leqq r \leqq r(\varphi,\,\lambda)$\, and
$$r \;=\; 0 \quad \mbox{if only if}\;\; \cos^2\varphi\sin\varphi\cos\lambda\sin\lambda \;=\; 0.$$
Thus we can write
$$V \;=\; \int_0^{\frac{\pi}{2}}\int_0^{\frac{\pi}{2}}\int_0^{r(\varphi,\,\lambda)}\!r^2\cos\varphi\;d\varphi\;d\lambda\;dr
\;=\; 
\frac{1}{3}\int_0^{\frac{\pi}{2}}\int_0^{\frac{\pi}{2}}\left(\sijoitus{r=0}{\quad r(\varphi,\,\lambda)}\!r^3\right)\cos\varphi\;d\varphi\;d\lambda,$$
getting then
$$V \;=\; 
a^3\int_0^{\frac{\pi}{2}}\!(\cos^3\varphi)(-\sin\varphi)\,d\varphi
\cdot\int_0^{\frac{\pi}{2}}\!(\cos\lambda)(-\sin\lambda)\,d\lambda
\;=\; a^3\!\sijoitus{\varphi=0}{\quad\frac{\pi}{2}}\!\frac{\cos^4\varphi}{4}
\cdot\!\sijoitus{\lambda=0}{\quad\frac{\pi}{2}}\!\frac{\cos^2\lambda}{2} \;=\; \frac{a^3}{8}.$$\\


\textbf{Remark.}\, The general \PMlinkescapetext{formula} for variable changing in a triple integral is
$$\iiint_Df(x,\,y,\,z)\,dx\,dy\,dz \:=\; 
\iiint_\Delta\!f(x(\xi,\,\eta,\,\zeta),\,y(\xi,\,\eta,\,\zeta),\,z(\xi,\,\eta,\,\zeta))
\left|\frac{\partial(x,\,y,\,z)}{\partial(\xi,\,\eta,\,\zeta)}\right|\,d\xi\,d\eta\,d\zeta.$$



%%%%%
%%%%%
\end{document}
