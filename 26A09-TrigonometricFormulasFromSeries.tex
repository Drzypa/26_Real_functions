\documentclass[12pt]{article}
\usepackage{pmmeta}
\pmcanonicalname{TrigonometricFormulasFromSeries}
\pmcreated{2013-03-22 18:50:47}
\pmmodified{2013-03-22 18:50:47}
\pmowner{pahio}{2872}
\pmmodifier{pahio}{2872}
\pmtitle{trigonometric formulas from series}
\pmrecord{9}{41654}
\pmprivacy{1}
\pmauthor{pahio}{2872}
\pmtype{Derivation}
\pmcomment{trigger rebuild}
\pmclassification{msc}{26A09}
\pmsynonym{series definition of sine and cosine}{TrigonometricFormulasFromSeries}
%\pmkeywords{sine}
%\pmkeywords{cosine}
\pmrelated{RigorousDefinitionOfTrigonometricFunctions}
\pmrelated{ApplicationOfFundamentalTheoremOfIntegralCalculus}
\pmrelated{TrigonometricFormulasFromDeMoivreIdentity}
\pmrelated{GoniometricFormulae}
\pmdefines{$\pi$}

% this is the default PlanetMath preamble.  as your knowledge
% of TeX increases, you will probably want to edit this, but
% it should be fine as is for beginners.

% almost certainly you want these
\usepackage{amssymb}
\usepackage{amsmath}
\usepackage{amsfonts}

% used for TeXing text within eps files
%\usepackage{psfrag}
% need this for including graphics (\includegraphics)
%\usepackage{graphicx}
% for neatly defining theorems and propositions
 \usepackage{amsthm}
% making logically defined graphics
%%%\usepackage{xypic}

% there are many more packages, add them here as you need them

% define commands here

\theoremstyle{definition}
\newtheorem*{thmplain}{Theorem}

\begin{document}
One may define the sine and the cosine functions for real (and complex) arguments using the power series
\begin{align}
\sin{x} \;=\; x-\frac{x^3}{3!}+\frac{x^5}{5!}-+\ldots,
\end{align}
\begin{align}
\cos{x} \;=\; 1-\frac{x^2}{2!}+\frac{x^4}{4!}-+\ldots,
\end{align}
and using only the properties of power series, easily derive most of the goniometric formulas, without any geometry.\, For example, one gets instantly from (1) and (2) the values
$$\sin0 \;=\; 0, \qquad \cos0 \;=\; 1$$
and the \PMlinkname{parity relations}{EvenoddFunction}
$$\sin(-x) \;=\; -\sin{x}, \qquad \cos(-x) \;=\; \cos{x}.$$
Using the Cauchy multiplication rule for series one can obtain the addition formulas
\begin{align}
\begin{cases}
   \sin(x\!+\!y) \;=\; \sin{x}\cos{y}+\cos{x}\sin{y},\\
   \cos(x\!+\!y) \;=\; \cos{x}\cos{y}-\sin{x}\sin{y}.
\end{cases}
\end{align}
These produce straightforward many other important formulae, e.g.
\begin{align}
\sin2x \;=\; 2\sin{x}\cos{x}, \qquad \cos2x \;=\; \cos^2x-\sin^2x \qquad (y \;=:\; x)
\end{align}
and
\begin{align}
\cos^2x+\sin^2x \;=\; 1 \qquad\qquad\qquad (y \;=:\; -x).
\end{align}


The value\, $\displaystyle\cos\frac{\pi}{2} = 0$,\, as well as the formulae expressing the periodicity of sine and cosine, cannot be directly obtained from the series (1) and (2) --- in fact, one must define the number $\pi$ by using the function properties of the \PMlinkescapetext{cosine series} and its \PMlinkname{derivative series}{PowerSeries}.

The equation
$$\cos{x} \;=\; 0$$
has on the interval \,$(0,\,2)$\, exactly one \PMlinkname{root}{Equation}.\, Actually, as sum of a power series, 
$\cos{x}$ is continuous,\, $\cos0 = 1 > 0$\, and\, $\cos2 < 1-\frac{2^2}{2!}+\frac{2^4}{4!} < 0$\, (see \PMlinkname{Leibniz' estimate for alternating series}{LeibnizEstimateForAlternatingSeries}), whence there is at least one root.\, If there were more than one root, then the derivative
$$-\sin{x} \;=\; -x+\frac{x^3}{3!}-+\ldots \;=\; -x(1-\frac{x^2}{3!}+-\ldots)$$
would have at least one zero on the interval; this is impossible, since by Leibniz the series in the parentheses does not change its sign on the interval:
$$1-\frac{x^2}{3!}+-\ldots \;>\;1-\frac{2^2}{3!} \;>\; 0$$
Accordingly, we can define the number pi to be the least positive solution of the equation\, $\cos{x} = 0$, multiplied by 2.

Thus we have\, $0 < \pi < 4$\, and\, $\cos\frac{\pi}{2} = 0$.\, Furthermore, by (5),
$$\sin\frac{\pi}{2} \;=\; 1,$$
and by (4),
$$\sin\pi \;=\; 0, \qquad \cos\pi \;=\; -1, \qquad \sin2\pi \;=\; 0, \qquad \cos2\pi \;=\; 1.$$
Consequently, the addition formulas (3) yield the \PMlinkname{periodicities}{PeriodicFunctions}
$$\sin(x\!+\!2\pi) \;=\; \sin{x}, \qquad \cos(x\!+\!2\pi) \;=\; \cos{x}.$$





%%%%%
%%%%%
\end{document}
