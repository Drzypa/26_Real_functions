\documentclass[12pt]{article}
\usepackage{pmmeta}
\pmcanonicalname{LinkBetweenInfiniteProductsAndSums}
\pmcreated{2013-03-22 13:41:38}
\pmmodified{2013-03-22 13:41:38}
\pmowner{paolini}{1187}
\pmmodifier{paolini}{1187}
\pmtitle{link between infinite products and sums}
\pmrecord{7}{34368}
\pmprivacy{1}
\pmauthor{paolini}{1187}
\pmtype{Theorem}
\pmcomment{trigger rebuild}
\pmclassification{msc}{26E99}
\pmclassification{msc}{40A20}

\endmetadata

% this is the default PlanetMath preamble.  as your knowledge
% of TeX increases, you will probably want to edit this, but
% it should be fine as is for beginners.

% almost certainly you want these
\usepackage{amssymb}
\usepackage{amsmath}
\usepackage{amsfonts}

% used for TeXing text within eps files
%\usepackage{psfrag}
% need this for including graphics (\includegraphics)
%\usepackage{graphicx}
% for neatly defining theorems and propositions
\usepackage{amsthm}
% making logically defined graphics
%%%\usepackage{xypic}

% there are many more packages, add them here as you need them

% define commands here
\newcommand{\R}{\mathbb R}
\newtheorem{theorem}{Theorem}
\newtheorem{definition}{Definition}
\theoremstyle{remark}
\newtheorem{example}{Example}
\begin{document}
Let 
\[
  \prod_{k=1}^\infty p_k
\]
be an infinite product such that $p_k>0$ for all $k$.
Then the infinite product converges if and only if the infinite sum
\[
  \sum_{k=1}^\infty \log p_k
\]
converges. Moreover
\[
  \prod_{k=1}^\infty p_k = \exp \sum_{k=1}^\infty \log p_k.
\]

\emph{Proof.}

Simply notice that
\[
  \prod_{k=1}^N p_k = \exp \sum_{k=1}^N \log p_k.
\]
If the infinite sum converges then (by continuity of $\exp$ function)
\[
  \lim_{N\to \infty} \prod_{k=1}^N p_k = \lim_{N\to\infty} \exp \sum_{k=1}^N \log p_k = \exp \sum_{k=1}^\infty \log p_k
\]
and also the infinite product converges.
%%%%%
%%%%%
\end{document}
