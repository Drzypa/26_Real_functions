\documentclass[12pt]{article}
\usepackage{pmmeta}
\pmcanonicalname{FresnelFormulas}
\pmcreated{2013-03-22 18:20:46}
\pmmodified{2013-03-22 18:20:46}
\pmowner{pahio}{2872}
\pmmodifier{pahio}{2872}
\pmtitle{Fresnel formulas}
\pmrecord{11}{40980}
\pmprivacy{1}
\pmauthor{pahio}{2872}
\pmtype{Theorem}
\pmcomment{trigger rebuild}
\pmclassification{msc}{26B15}
\pmclassification{msc}{26A36}
\pmsynonym{Fresnel formulae}{FresnelFormulas}
\pmsynonym{Fresnel integrals at infinity}{FresnelFormulas}
\pmrelated{UsingResidueTheoremNearBranchPoint}
\pmrelated{AreaUnderGaussianCurve}

\endmetadata

% this is the default PlanetMath preamble.  as your knowledge
% of TeX increases, you will probably want to edit this, but
% it should be fine as is for beginners.

% almost certainly you want these
\usepackage{amssymb}
\usepackage{amsmath}
\usepackage{amsfonts}

% used for TeXing text within eps files
%\usepackage{psfrag}
% need this for including graphics (\includegraphics)
%\usepackage{graphicx}
% for neatly defining theorems and propositions
 \usepackage{amsthm}
% making logically defined graphics
%%%\usepackage{xypic}
\usepackage{pstricks}
\usepackage{pst-plot}

% there are many more packages, add them here as you need them

% define commands here

\theoremstyle{definition}
\newtheorem*{thmplain}{Theorem}

\begin{document}
$$\int_0^\infty\!\cos{x^2}\,dx \,=\, \int_0^\infty\!\sin{x^2}\,dx \,=\, \frac{\sqrt{2\pi}}{4}$$


{\em Proof.}

\begin{center}
\begin{pspicture}(-1,-1)(6.2,4.2)
\psaxes[Dx=10,Dy=10]{->}(0,0)(-1,-1)(6,4)
\rput(-0.2,4.1){$y$}
\rput(6.1,-0.2){$x$}
\rput(-0.17,-0.22){0}
\rput(5,-0.3){$R$}
\rput(0.7,0.3){$\frac{\pi}{4}$}
\rput(1.4,1.7){$s$}
\rput(4.7,2.2){$b$}
\psline[linecolor=blue,linewidth=0.05]{->}(0,0)(5,0)
\psline[linecolor=blue,linewidth=0.04]{->}(3.55,3.55)(0,0)
\psarc[linecolor=blue,linewidth=0.04]{->}(0,0){5}{-1}{45}
\psarc(0,0){0.5}{0}{45}
\end{pspicture}
\end{center}

The function \,$\displaystyle z \mapsto e^{-z^2}$\, is entire, whence by the fundamental theorem of complex analysis we have
\begin{align}
\oint_\gamma e^{-z^2}\,dz \;=\; 0
\end{align}
where $\gamma$ is the \PMlinkname{perimeter}{BasicPolygon} of the circular sector described in the picture.\, We split this contour integral to three portions:
\begin{align}
\underbrace{\int_0^R\!e^{-x^2}\,dx}_{I_1}+\underbrace{\int_b\!e^{-z^2}\,dz}_{I_2}
+\underbrace{\int_s\!e^{-z^2}\,dz}_{I_3} \,=\,0
\end{align}
By the entry concerning the Gaussian integral, we know that
$$\lim_{R\to\infty}I_1 = \frac{\sqrt{\pi}}{2}.$$

For handling $I_2$, we use the substitution
$$z \,:=\, Re^{i\varphi} = R(\cos\varphi+i\sin\varphi), \quad dz \,=\,iRe^{i\varphi}\,d\varphi \quad
(0 \leqq \varphi \leqq \frac{\pi}{4}).$$
Using also de Moivre's formula we can write
$$|I_2| = \left|iR\int_0^{\frac{\pi}{4}}e^{-R^2(\cos2\varphi+i\sin2\varphi)}e^{i\varphi}d\varphi\right| \leqq 
R\!\int_0^{\frac{\pi}{4}}\left|e^{-R^2(\cos2\varphi+i\sin2\varphi)}\right|\cdot\left|e^{i\varphi}\right|\cdot|d\varphi|
= R\!\int_0^{\frac{\pi}{4}}e^{-R^2\cos2\varphi}d\varphi.$$
Comparing the graph of the function \,$\varphi \mapsto \cos2\varphi$\, with the line through the points \,$(0,\,1)$\, and\, $(\frac{\pi}{4},\,0)$\, allows us to estimate $\cos2\varphi$ downwards:
$$\cos2\varphi \geqq 1\!-\!\frac{4\varphi}{\pi} \quad\mbox{for}\quad 0 \leqq \varphi \leqq \frac{\pi}{4}$$
Hence we obtain
$$|I_2| \leqq R\int_0^{\frac{\pi}{4}}\frac{d\varphi}{e^{R^2\cos2\varphi}} 
        \leqq R\int_0^{\frac{\pi}{4}}\frac{d\varphi}{e^{R^2(1-\frac{4\varphi}{\pi})}} 
\leqq \frac{R}{e^{R^2}} \int_0^{\frac{\pi}{4}} e^{\frac{4R^2}{\pi}\varphi} d\varphi,$$
and moreover
$$|I_2| \leqq \frac{\pi}{4Re^{R^2}}(e^{R^2}-1) < \frac{\pi e^{R^2}}{4Re^{R^2}} = \frac{\pi}{4R} \; \to 0
\quad \mbox{as} \quad R \to \infty.$$
Therefore
$$\lim_{R\to\infty}I_2 = 0.\\$$

Then make to $I_3$ the substitution
$$z \;:=\; \frac{1\!+\!i}{\sqrt{2}}t, \quad dz \,=\, \frac{1\!+\!i}{\sqrt{2}}dt \quad(R \geqq t \geqq 0).$$
It yields
\begin{align*}
I_3 &\quad = \frac{1\!+\!i}{\sqrt{2}}\int_R^0e^{-it^2}\,dt
= -\frac{1}{\sqrt{2}}\int_0^R(1+i)(\cos{t^2}-i\sin{t^2})\,dt \\
    &\quad = -\frac{1}{\sqrt{2}}\left(\int_0^R\sin{t^2}\,dt+\int_0^R\cos{t^2}\,dt\right)
+\frac{i}{\sqrt{2}}\left(\int_0^R\sin{t^2}\,dt-\int_0^R\cos{t^2}\,dt\right).
\end{align*}
Thus, letting\, $R \to \infty$,\, the equation (2) implies
\begin{align}
\frac{\sqrt{\pi}}{2}\!+\!0\!
-\frac{1}{\sqrt{2}}\left(\int_0^\infty\!\sin{t^2}\,dt+\!\int_0^\infty\!\cos{t^2}\,dt\right)\!
+\!\frac{i}{\sqrt{2}}\left(\int_0^\infty\!\sin{t^2}\,dt-\!\int_0^\infty\!\cos{t^2}\,dt\right) \;=\; 0.
\end{align}
Because the imaginary part vanishes, we infer that\, $\int_0^\infty\cos{x^2}\,dx = \int_0^\infty\sin{x^2}\,dx$,\, whence (3) reads
$$\frac{\sqrt{\pi}}{2}+0-\frac{1}{\sqrt{2}}\!\cdot\!2\!\int_0^\infty\!\sin{t^2}\,dt \,=\, 0.$$ 
So we get also the result\, $\int_0^\infty\sin{x^2}\,dx = \frac{\sqrt{2}}{2}\cdot\frac{\sqrt{\pi}}{2} = 
\frac{\sqrt{2\pi}}{4}$,\, Q.E.D.


%%%%%
%%%%%
\end{document}
