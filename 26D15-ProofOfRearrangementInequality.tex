\documentclass[12pt]{article}
\usepackage{pmmeta}
\pmcanonicalname{ProofOfRearrangementInequality}
\pmcreated{2013-03-22 13:08:41}
\pmmodified{2013-03-22 13:08:41}
\pmowner{pbruin}{1001}
\pmmodifier{pbruin}{1001}
\pmtitle{proof of rearrangement inequality}
\pmrecord{4}{33583}
\pmprivacy{1}
\pmauthor{pbruin}{1001}
\pmtype{Proof}
\pmcomment{trigger rebuild}
\pmclassification{msc}{26D15}
%\pmkeywords{inequality}
\pmrelated{ChebyshevsInequality}

\endmetadata

% this is the default PlanetMath preamble.  as your knowledge
% of TeX increases, you will probably want to edit this, but
% it should be fine as is for beginners.

% almost certainly you want these
\usepackage{amssymb}
\usepackage{amsmath}
\usepackage{amsfonts}

% used for TeXing text within eps files
%\usepackage{psfrag}
% need this for including graphics (\includegraphics)
%\usepackage{graphicx}
% for neatly defining theorems and propositions
%\usepackage{amsthm}
% making logically defined graphics
%%%\usepackage{xypic}

% there are many more packages, add them here as you need them

% define commands here
\begin{document}
We first prove the rearrangement inequality for the case $n=2$.  Let
$x_1,x_2,y_1,y_2$ be real numbers such that $x_1\le x_2$ and $y_1\le
y_2$.  Then
$$
(x_2-x_1)(y_2-y_1)\ge 0,
$$
and therefore
$$
x_1y_1+x_2y_2\ge x_1y_2+x_2y_1.
$$ Equality holds iff $x_1=x_2$ or $y_1=y_2$.\\
\\
For the general case, let $x_1,x_2,\ldots,x_n$ and
$y_1,y_2,\ldots,y_n$ be real numbers such that $x_1\le x_2\le\cdots\le
x_n$.  Suppose that $(z_1,z_2,\ldots,z_n)$ is a permutation
(rearrangement) of $\{y_1,y_2,\ldots,y_n\}$ such that the sum
$$
x_1z_1+x_2z_2+\cdots+x_nz_n
$$
is maximized.  If there exists a pair $i<j$ with $z_i>z_j$, then
$x_iz_j+x_jz_i\ge x_iz_i+x_jz_j$ (the $n=2$ case); equality holds iff
$x_i=x_j$.  Therefore, $x_1z_1+x_2z_2+\cdots+x_nz_n$ is not maximal
unless $z_1\le z_2\le\cdots\le z_n$ or $x_i=x_j$ for all pairs $i<j$
such that $z_i>z_j$.  In the latter case, we can consecutively
interchange these pairs until $z_1\le z_2\le\cdots\le z_n$ (this is
possible because the number of pairs $i<j$ with $z_i>z_j$ decreases
with each step).  So $x_1z_1+x_2z_2+\cdots+x_nz_n$ is maximized if
$$
z_1\le z_2\le\cdots\le z_n.
$$
To show that $x_1z_1+x_2z_2+\cdots+x_nz_n$ is minimal for a
permutation $(z_1,z_2,\ldots,z_n)$ of $\{y_1,y_2,\ldots,y_n\}$ if
$z_1\ge z_2\ge\cdots\ge z_n$, observe that
$-(x_1z_1+x_2z_2+\cdots+x_nz_n)=x_1(-z_1)+x_2(-z_2)+\cdots+x_n(-z_n)$
is maximized if $-z_1\le-z_2\le\cdots\le-z_n$.  This implies that
$x_1z_1+x_2z_2+\cdots+x_nz_n$ is minimized if
$$
z_1\ge z_2\ge\cdots\ge z_n.
$$
%%%%%
%%%%%
\end{document}
