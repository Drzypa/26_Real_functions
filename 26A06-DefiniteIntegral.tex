\documentclass[12pt]{article}
\usepackage{pmmeta}
\pmcanonicalname{DefiniteIntegral}
\pmcreated{2013-03-22 12:15:17}
\pmmodified{2013-03-22 12:15:17}
\pmowner{mathwizard}{128}
\pmmodifier{mathwizard}{128}
\pmtitle{definite integral}
\pmrecord{16}{31637}
\pmprivacy{1}
\pmauthor{mathwizard}{128}
\pmtype{Definition}
\pmcomment{trigger rebuild}
\pmclassification{msc}{26A06}
\pmrelated{AreaOfPlaneRegion}
\pmrelated{IntegralsOfEvenAndOddFunctions}
\pmrelated{IntegralOverAPeriodInterval}
\pmdefines{interval of integration}
\pmdefines{upper limit}
\pmdefines{lower limit}

\endmetadata

\usepackage{amssymb}
\usepackage{amsmath}
\usepackage{amsfonts}
\usepackage{graphicx}

\begin{document}
The \emph{definite integral} with respect to $x$ of some function $f(x)$ over the compact interval $[a,b]$ with $a<b$, the \emph{interval of integration}, is
defined to be the ``area under the graph of $f(x)$ with respect to $x$'' (if $f(x)$ is negative, then you have a negative area). The numbers $a$ and $b$ are called \emph{lower} and \emph{upper limit} respectively. It is written as:
$$ \int_a^bf(x) \ dx .$$
One way to find the value of the integral is to take a limit of an approximation technique
as the precision increases to infinity.

For example, use a Riemann sum which approximates
the area by dividing it into $n$ intervals of equal widths, and then calculating the area
of rectangles with the width of the interval and height dependent on the function's value in the interval.
Let $R_n$ be this approximation, which can be written as
$$ R_n = \sum_{i=1}^{n} f(x_i^*) \Delta x ,$$
where $x_i^*$ is some $x$ inside the $i^{\rm th}$ interval. This process is illustrated by figure \ref{fig:bars}.
\begin{figure}[htbp]
\begin{centering}
\includegraphics[angle=270,scale=0.5]{definite_integral.eps}
\caption{The area under the graph approximated by rectangles}\label{fig:bars}
\end{centering}
\end{figure}

Then, the integral would be
$$ \int_a^bf(x) \ dx = \lim_{n \to \infty} R_n =
   \lim_{n \to \infty} \sum_{i=1}^{n} f(x_i^*) \Delta x .$$
This limit does not necessarily exist for every function $f$ and it may depend on the particular choice of the $x_i^*$.  If all those limits coincide and are finite, then the integral exists. This is true in particular for continuous $f$.

Furthermore we define
$$\int_b^af(x)\ dx=-\int_a^bf(x)\ dx.$$

We can use this definition to arrive at some important properties of definite integrals
($a$, $b$, $c$ are constant with respect to $x$):
\begin{eqnarray*}
\int_a^b(f(x) + g(x)) \ dx & = & \int_a^bf(x)\ dx + \int_a^bg(x)\ dx; \\
\int_a^b(f(x) - g(x)) \ dx & = & \int_a^bf(x)\ dx - \int_a^bg(x)\ dx ;\\
\int_a^bf(x) \ dx & = & \int_a^cf(x)\ dx + \int_c^bf(x)\ dx ;\\
\int_a^bcf(x) \ dx & = & c\int_a^bf(x)\ dx.
\end{eqnarray*}

There are other generalizations about integrals, but many require the fundamental theorem of calculus.

%%%%%
%%%%%
%%%%%
\end{document}
