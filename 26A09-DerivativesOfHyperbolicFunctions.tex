\documentclass[12pt]{article}
\usepackage{pmmeta}
\pmcanonicalname{DerivativesOfHyperbolicFunctions}
\pmcreated{2013-03-22 14:32:18}
\pmmodified{2013-03-22 14:32:18}
\pmowner{alozano}{2414}
\pmmodifier{alozano}{2414}
\pmtitle{derivatives of hyperbolic functions}
\pmrecord{5}{36083}
\pmprivacy{1}
\pmauthor{alozano}{2414}
\pmtype{Derivation}
\pmcomment{trigger rebuild}
\pmclassification{msc}{26A09}

% this is the default PlanetMath preamble.  as your knowledge
% of TeX increases, you will probably want to edit this, but
% it should be fine as is for beginners.

% almost certainly you want these
\usepackage{amssymb}
\usepackage{amsmath}
\usepackage{amsthm}
\usepackage{amsfonts}

% used for TeXing text within eps files
%\usepackage{psfrag}
% need this for including graphics (\includegraphics)
%\usepackage{graphicx}
% for neatly defining theorems and propositions
%\usepackage{amsthm}
% making logically defined graphics
%%%\usepackage{xypic}

% there are many more packages, add them here as you need them

% define commands here

\newtheorem{thm}{Theorem}
\newtheorem{defn}{Definition}
\newtheorem{prop}{Proposition}
\newtheorem{lemma}{Lemma}
\newtheorem{cor}{Corollary}

% Some sets
\newcommand{\Nats}{\mathbb{N}}
\newcommand{\Ints}{\mathbb{Z}}
\newcommand{\Reals}{\mathbb{R}}
\newcommand{\Complex}{\mathbb{C}}
\newcommand{\Rats}{\mathbb{Q}}
\begin{document}
In this entry we compute the derivative of the hyperbolic functions $\sinh(x)$ and $\cosh(x)$.

Recall that by definition:

\begin{eqnarray*}
\sinh(x)&:=&\frac{e^x-e^{-x}}{2}\\
\cosh(x)&:=&\frac{e^x+e^{-x}}{2}.
\end{eqnarray*}

Therefore:

\begin{eqnarray*}
\frac{d}{dx}\sinh(x) &=& \frac{d}{dx}\left(\frac{e^x-e^{-x}}{2}\right)\\
&=& \frac{1}{2}\cdot\frac{d}{dx}\left(e^x-e^{-x}\right)\\
&=& \frac{1}{2}\cdot\left(e^x-(-e^{-x})\right)\\
&=& \frac{e^x+e^{-x}}{2}\\
&=& \cosh(x).
\end{eqnarray*}

Similarly $\displaystyle \frac{d}{dx}\cosh(x)=\sinh(x)$. Using the quotient rule, we compute the derivative of $\displaystyle \tanh(x)=\frac{\sinh(x)}{\cosh(x)}$:
$$\frac{d}{dx}\tanh(x)=\frac{\cosh^2(x)-\sinh^2(x)}{\cosh^2(x)}=\frac{1}{\cosh^2(x)}$$
where we have used the equality $\cosh^2(x)-\sinh^2(x)=1$.
%%%%%
%%%%%
\end{document}
