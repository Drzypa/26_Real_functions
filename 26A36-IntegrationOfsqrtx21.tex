\documentclass[12pt]{article}
\usepackage{pmmeta}
\pmcanonicalname{IntegrationOfsqrtx21}
\pmcreated{2013-03-22 18:06:58}
\pmmodified{2013-03-22 18:06:58}
\pmowner{pahio}{2872}
\pmmodifier{pahio}{2872}
\pmtitle{integration of $\sqrt{x^2+1}$}
\pmrecord{7}{40662}
\pmprivacy{1}
\pmauthor{pahio}{2872}
\pmtype{Derivation}
\pmcomment{trigger rebuild}
\pmclassification{msc}{26A36}
\pmsynonym{antiderivative of $\sqrt{x^2+1}$}{IntegrationOfsqrtx21}
\pmrelated{IntegrationByParts}
\pmrelated{AreaFunctions}
\pmrelated{DerivativeOfInverseFunction}

% this is the default PlanetMath preamble.  as your knowledge
% of TeX increases, you will probably want to edit this, but
% it should be fine as is for beginners.

% almost certainly you want these
\usepackage{amssymb}
\usepackage{amsmath}
\usepackage{amsfonts}

% used for TeXing text within eps files
%\usepackage{psfrag}
% need this for including graphics (\includegraphics)
%\usepackage{graphicx}
% for neatly defining theorems and propositions
%\usepackage{amsthm}
% making logically defined graphics
%%%\usepackage{xypic}

% there are many more packages, add them here as you need them

\DeclareMathOperator{\arsinh}{arsinh}
\DeclareMathOperator{\arcosh}{arcosh}
\DeclareMathOperator{\artanh}{artanh}
\DeclareMathOperator{\arcoth}{arcoth}
\begin{document}
The integral
$$I \,:=\, \int\!\sqrt{x^2\!+\!1}\;dx$$
can be found by using the first \PMlinkname{Euler's substitution}{EulersSubstitutionsForIntegration} 
$$\sqrt{x^2\!+\!1} \;:=\; -x\!+\!t,$$
but another possibility is to use \PMlinkname{partial integration}{ASpecialCaseOfPartialIntegration} if one knows the integral $\int\!\frac{dx}{\sqrt{x^2+1}}$.\, The corresponding may be said of the more general
$$\int\!\sqrt{x^2\!+\!c}\;dx.$$

We think that the integrand of $I$ has the other factor 1 and integrate partially:
$$I \,=\, \int\!1\cdot\sqrt{x^2\!+\!1}\;dx \,=\, x\sqrt{x^2\!+\!1}-\!\int\!x\cdot\frac{1}{2\sqrt{x^2\!+\!1}}\cdot2x\,dx+C' 
\,=\, x\sqrt{x^2\!+\!1}-\!\int\!\frac{x^2}{\sqrt{x^2\!+\!1}}\,dx+C'.$$
Writing the numerator as $(x^2\!+\!1)\!-\!1$ and dividing its minuend and subtrahend separately, we can write
$$ I \,=\, x\sqrt{x^2\!+\!1}-\!\left(\!\int\!\sqrt{x^2\!+\!1}\,dx-\!\int\!\frac{1}{\sqrt{x^2\!+\!1}}\,dx\right)+C' \,=\,
x\sqrt{x^2\!+\!1}-I+\!\int\!\frac{dx}{\sqrt{x^2\!+\!1}}+C'.$$
Having $I$ in two \PMlinkescapetext{places}, we solve it from these equalities, obtaining
$$I \,=\, \frac{x}{2}\sqrt{x^2\!+\!1}+\frac{1}{2}\!\int\frac{dx}{\sqrt{x^2\!+\!1}}+C,$$
i.e.,
$$\int\!\sqrt{x^2\!+\!1}\;dx \;=\; \frac{x}{2}\sqrt{x^2\!+\!1}+\frac{1}{2}\arsinh{x}+C$$

%%%%%
%%%%%
\end{document}
