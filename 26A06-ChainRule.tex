\documentclass[12pt]{article}
\usepackage{pmmeta}
\pmcanonicalname{ChainRule}
\pmcreated{2013-03-22 12:26:43}
\pmmodified{2013-03-22 12:26:43}
\pmowner{matte}{1858}
\pmmodifier{matte}{1858}
\pmtitle{chain rule}
\pmrecord{12}{32561}
\pmprivacy{1}
\pmauthor{matte}{1858}
\pmtype{Theorem}
\pmcomment{trigger rebuild}
\pmclassification{msc}{26A06}
\pmrelated{Derivative}
\pmrelated{ChainRuleSeveralVariables}
\pmrelated{ExampleOnSolvingAFunctionalEquation}
\pmrelated{GudermannianFunction}

\endmetadata

\usepackage{amsmath}
\usepackage{amsfonts}
\usepackage{amssymb}

\newcommand{\reals}{\mathbb{R}}
\newcommand{\natnums}{\mathbb{N}}
\newcommand{\cnums}{\mathbb{C}}
\newcommand{\znums}{\mathbb{Z}}

\newcommand{\lp}{\left(}
\newcommand{\rp}{\right)}
\newcommand{\lb}{\left[}
\newcommand{\rb}{\right]}

\newcommand{\supth}{^{\text{th}}}


\newtheorem{proposition}{Proposition}
\begin{document}
Let $f, g$ be differentiable,
real-valued functions such that $g$ is defined on an open set 
$I\subseteq \mathbb{R}$, and $f$ is defined on $g(I)$. 
Then the derivative of the composition $f\circ g$ is given by
the \emph{chain rule}, which asserts that
$$
  (f\circ g)'(x) = (f'\circ g)(x)\, g'(x), \quad x\in I.
$$

The chain rule has a particularly suggestive appearance in terms of
the Leibniz formalism.  Suppose that $z$ depends differentiably on
$y$, and that $y$ in turn depends differentiably on $x$.  Then we have
$$
  \frac{dz}{dx} = \frac{dz}{dy}\, \frac{dy}{dx}.
$$
The apparent cancellation  of the $dy$ term is at best a formal
mnemonic, and does not constitute a rigorous proof of this result.
Rather, the Leibniz format is well suited to the interpretation of the
chain rule in terms of related rates. To wit:
\begin{quote}
  \em The instantaneous rate of change of $z$ relative to $x$ is equal to the
  rate of change of $z$ relative to $y$ times the rate of change of
  $y$ relative to $x$.
\end{quote}
%%%%%
%%%%%
\end{document}
