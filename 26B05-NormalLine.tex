\documentclass[12pt]{article}
\usepackage{pmmeta}
\pmcanonicalname{NormalLine}
\pmcreated{2013-03-22 17:09:53}
\pmmodified{2013-03-22 17:09:53}
\pmowner{pahio}{2872}
\pmmodifier{pahio}{2872}
\pmtitle{normal line}
\pmrecord{17}{39476}
\pmprivacy{1}
\pmauthor{pahio}{2872}
\pmtype{Definition}
\pmcomment{trigger rebuild}
\pmclassification{msc}{26B05}
\pmclassification{msc}{26A24}
\pmclassification{msc}{53A04}
\pmsynonym{normal of curve}{NormalLine}
\pmsynonym{normal}{NormalLine}
\pmsynonym{perpendicular}{NormalLine}
\pmrelated{ConditionOfOrthogonality}
\pmrelated{ParallelCurve}
\pmrelated{SurfaceNormal}
\pmrelated{Grafix}
\pmrelated{NormalOfPlane}
\pmdefines{foot of normal}
\pmdefines{foot of perpendicular}

\endmetadata

% this is the default PlanetMath preamble.  as your knowledge
% of TeX increases, you will probably want to edit this, but
% it should be fine as is for beginners.

% almost certainly you want these
\usepackage{amssymb}
\usepackage{amsmath}
\usepackage{amsfonts}

% used for TeXing text within eps files
%\usepackage{psfrag}
% need this for including graphics (\includegraphics)
%\usepackage{graphicx}
% for neatly defining theorems and propositions
 \usepackage{amsthm}
% making logically defined graphics
%%%\usepackage{xypic}

% there are many more packages, add them here as you need them
\usepackage{pstricks}

% define commands here

\theoremstyle{definition}
\newtheorem*{thmplain}{Theorem}

\begin{document}
A {\it normal line} (or simply {\it normal} or {\it perpendicular}) of a curve at one of its points $P$ is the line passing through this point and perpendicular to the tangent line of the curve at $P$.\, The point $P$ is the {\it foot} of the normal.

If the plane curve\, $y = f(x)$\, has a skew tangent at the point\, $(x_0,\,f(x_0))$,\, then the slope of the tangent at that point is\, $f'(x_0)$\, and the slope of the normal at that point is\, $\displaystyle -\frac{1}{f'(x_0)}$.\, The equation of the normal is thus
             $$y-f(x_0) = -\frac{1}{f'(x_0)}(x-x_0).$$
In the case that the tangent is horizontal, the equation of the vertical normal is
                           $$x = x_0,$$
and in the case that the tangent is vertical, the equation of the normal is
                           $$y = f(x_0).$$  

The normal of a curve at its point $P$ always goes through the center of curvature belonging to the point $P$.

In the picture below, the black curve is a parabola, the red line is the tangent at the point $P$, and the blue line is the normal at the point $P$.

\begin{center}
\begin{pspicture}(-2,-1)(2,4)
\parabola{-}(2,4)(0,0)
\rput[b](-2,4){.}
\rput[b](2,4){.}
\rput[l](-0.05,-1){.}
\psline[linecolor=red](0,-1)(2,3)
\psline[linecolor=blue](-2,2.5)(2,0.5)
\psdot(1,1)
\rput[b](0.9,1.2){$P$}
\psline[linewidth=0.2pt](0.9,0.8)(1.1,0.7)
\psline[linewidth=0.2pt](1.1,0.7)(1.2,0.9)
\end{pspicture}
\end{center}

%%%%%
%%%%%
\end{document}
