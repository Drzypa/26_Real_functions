\documentclass[12pt]{article}
\usepackage{pmmeta}
\pmcanonicalname{ComparisonOfsinthetaAndthetaNeartheta0}
\pmcreated{2013-03-22 16:58:29}
\pmmodified{2013-03-22 16:58:29}
\pmowner{Wkbj79}{1863}
\pmmodifier{Wkbj79}{1863}
\pmtitle{comparison of $\sin \theta$ and $\theta$ near $\theta = 0$}
\pmrecord{7}{39251}
\pmprivacy{1}
\pmauthor{Wkbj79}{1863}
\pmtype{Theorem}
\pmcomment{trigger rebuild}
\pmclassification{msc}{26A03}
\pmclassification{msc}{51N20}
\pmclassification{msc}{26A06}
\pmrelated{LimitOfDisplaystyleFracsinXxAsXApproaches0}
\pmrelated{JordansInequality}

\endmetadata

\usepackage{amssymb}
\usepackage{amsmath}
\usepackage{amsfonts}

\usepackage{amsthm}
\usepackage{pstricks}

\newtheorem{thm*}{Theorem}
\newtheorem{cor*}{Corollary}

\begin{document}
\begin{thm*}
Let $\displaystyle 0< \theta < \frac{\pi}{2}$, where $\theta$ is an angle measured in radians.  Then $\sin \theta<\theta$.
\end{thm*}

\begin{proof}
Let $O=(0,0)$, $P=(1,0)$, and $Q=(\cos \theta, \sin \theta)$.  Note that the circle $x^2+y^2=1$ passes through $P$ and $Q$ and that the shortest arc along this circle from $P$ to $Q$ has length $\theta$.  Note also that the line segments $\overline{OP}$ and $\overline{OQ}$ are radii of the circle $x^2+y^2=1$ and therefore must each have length 1.

\begin{center}
\begin{pspicture}(-1,1)(-2,2)
\psdots(0,0)(2,0)(1.2,1.6)
\rput[b](0,-0.5){$O$}
\rput[b](2,-0.5){$P$}
\rput[b](1.2,1.6){$Q$}
\psline(1.2,1.6)(0,0)(2,0)
\psarc(0,0){0.3}{0}{55.5}
\rput[l](0.3,0.2){$\theta$}
\psarc(0,0){2}{0}{55.5}
\end{pspicture}
\end{center}

Draw the line segment $\overline{PQ}$.  Since this does not correspond to the arc, its length $\left| \overline{PQ} \right|<\theta$.

\begin{center}
\begin{pspicture}(-1,1)(-2,2)
\psdots(0,0)(2,0)(1.2,1.6)
\rput[b](0,-0.5){$O$}
\rput[b](2,-0.5){$P$}
\rput[b](1.2,1.6){$Q$}
\psline(2,0)(1.2,1.6)(0,0)(2,0)
\psarc(0,0){0.3}{0}{55.5}
\rput[l](0.3,0.2){$\theta$}
\psarc(0,0){2}{0}{55.5}
\end{pspicture}
\end{center}

Drop the perpendicular from $Q$ to $\overline{OP}$.  Let $R$ be the point of intersection.  Note that $\left| \overline{OR} \right|=\cos \theta$ and $\left| \overline{QR} \right|=\sin \theta$.

\begin{center}
\begin{pspicture}(-1,1)(-2,2)
\psdots(0,0)(2,0)(1.2,1.6)(1.2,0)
\rput[b](0,-0.5){$O$}
\rput[b](2,-0.5){$P$}
\rput[b](1.2,1.6){$Q$}
\psline(2,0)(1.2,1.6)(0,0)(2,0)
\psarc(0,0){0.3}{0}{55.5}
\rput[l](0.3,0.2){$\theta$}
\psarc(0,0){2}{0}{55.5}
\psline(1.2,0)(1.2,1.6)
\rput[b](1.2,-0.5){$R$}
\end{pspicture}
\end{center}

Since $\displaystyle 0< \theta <\frac{\pi}{2}$, $0< \sin \theta <1$ and $0< \cos \theta <1$.  Thus, $R \neq Q$ and $R$ lies strictly in between $O$ and $P$.  Therefore, $\left| \overline{PR} \right|>0$.

By the Pythagorean theorem, $\left| \overline{QR} \right|^2+\left| \overline{PR} \right|^2=\left| \overline{PQ} \right|^2$.  Thus, $\left| \overline{QR} \right|^2< \left| \overline{PQ} \right|^2$.  Therefore, $\sin \theta = \left| \overline{QR} \right|< \left| \overline{PQ} \right| < \theta$.
\end{proof}

The analogous result for $\theta$ slightly below 0 is:

\begin{cor*}
Let $\displaystyle \frac{-\pi}{2}< \theta <0$, where $\theta$ is an angle measured in radians.  Then $\theta < \sin \theta$.
\end{cor*}

\begin{proof}
Since $\displaystyle 0< -\theta < \frac{\pi}{2}$, the previous theorem yields $\sin (-\theta)<-\theta$.  Since $\sin$ is an odd function, $-\sin \theta < -\theta$.  It follows that $\theta < \sin \theta$.
\end{proof}
%%%%%
%%%%%
\end{document}
