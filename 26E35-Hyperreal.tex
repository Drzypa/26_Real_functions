\documentclass[12pt]{article}
\usepackage{pmmeta}
\pmcanonicalname{Hyperreal}
\pmcreated{2013-03-22 12:35:45}
\pmmodified{2013-03-22 12:35:45}
\pmowner{djao}{24}
\pmmodifier{djao}{24}
\pmtitle{hyperreal}
\pmrecord{4}{32847}
\pmprivacy{1}
\pmauthor{djao}{24}
\pmtype{Definition}
\pmcomment{trigger rebuild}
\pmclassification{msc}{26E35}
\pmsynonym{nonstandard real}{Hyperreal}
\pmsynonym{non-standard real}{Hyperreal}
\pmrelated{Infinitesimal2}
\pmdefines{nonprincipal ultrafilter}
\pmdefines{infinitesimal}
\pmdefines{hypernatural}
\pmdefines{hyperinteger}
\pmdefines{hyperrational}
\pmdefines{hyperfinite}

% this is the default PlanetMath preamble.  as your knowledge
% of TeX increases, you will probably want to edit this, but
% it should be fine as is for beginners.

% almost certainly you want these
\usepackage{amssymb}
\usepackage{amsmath}
\usepackage{amsfonts}

% used for TeXing text within eps files
%\usepackage{psfrag}
% need this for including graphics (\includegraphics)
%\usepackage{graphicx}
% for neatly defining theorems and propositions
%\usepackage{amsthm}
% making logically defined graphics
%%%\usepackage{xypic} 

% there are many more packages, add them here as you need them

% define commands here
\newcommand{\F}{\mathcal{F}}
\newcommand{\R}{\mathbb{R}}
\newcommand{\N}{\mathbb{N}}
\begin{document}
An ultrafilter $\F$ on a set $I$ is called {\em nonprincipal} if no finite subsets of $I$ are in $\F$.

Fix once and for all a nonprincipal ultrafilter $\F$ on the set $\N$ of natural numbers. Let $\sim$ be the equivalence relation on the set $\R^\N$ of sequences of real numbers given by
$$
\{a_n\} \sim \{b_n\} \iff \{n \in \N \mid a_n = b_n\} \in \F
$$
Let $^*\R$ be the set of equivalence classes of $\R^\N$ under the equivalence relation $\sim$. The set $^*\R$ is called the set of {\em hyperreals}. It is a field under coordinatewise addition and multiplication:
\begin{eqnarray*}
\{a_n\} + \{b_n\} & = & \{a_n+b_n\} \\
\{a_n\} \cdot \{b_n\} & = & \{a_n\cdot b_n\}
\end{eqnarray*}
The field $^*\R$ is an ordered field under the ordering relation
$$
\{a_n\} \leq \{b_n\} \iff \{n \in \N \mid a_n \leq b_n\} \in \F
$$
The real numbers embed into $^*\R$ by the map sending the real number $x \in \R$ to the equivalence class of the constant sequence given by $x_n := x$ for all $n$. In what follows, we adopt the convention of treating $\R$ as a subset of $^*\R$ under this embedding.

A hyperreal $x \in\,^*\R$ is:
\begin{itemize}
\item {\em limited} if $a < x < b$ for some real numbers $a,b \in \R$
\item {\em positive unlimited} if $x > a$ for all real numbers $a \in \R$
\item {\em negative unlimited} if $x < a$ for all real numbers $a \in \R$
\item {\em unlimited} if it is either positive unlimited or negative unlimited
\item {\em positive infinitesimal} if $0 < x < a$ for all positive real numbers $a \in \R^+$
\item {\em negative infinitesimal} if $a < x < 0$ for all negative real numbers $a \in \R^-$
\item {\em infinitesimal} if it is either positive infinitesimal or negative infinitesimal
\end{itemize}

For any subset $A$ of $\R$, the set $^*A$ is defined to be the subset of $^*\R$ consisting of equivalence classes of sequences $\{a_n\}$ such that
$$
\{n \in \N \mid a_n \in A\} \in \F.
$$
The sets $^*\mathbb{N}$, $^*\mathbb{Z}$, and $^*\mathbb{Q}$ are called {\em hypernaturals}, {\em hyperintegers}, and {\em hyperrationals}, respectively. An element of $^*\mathbb{N}$ is also sometimes called {\em hyperfinite}.
%%%%%
%%%%%
\end{document}
