\documentclass[12pt]{article}
\usepackage{pmmeta}
\pmcanonicalname{StructureOfFiniteHyperrealNumbers}
\pmcreated{2013-03-22 17:26:02}
\pmmodified{2013-03-22 17:26:02}
\pmowner{asteroid}{17536}
\pmmodifier{asteroid}{17536}
\pmtitle{structure of finite hyperreal numbers}
\pmrecord{4}{39810}
\pmprivacy{1}
\pmauthor{asteroid}{17536}
\pmtype{Theorem}
\pmcomment{trigger rebuild}
\pmclassification{msc}{26E35}

% this is the default PlanetMath preamble.  as your knowledge
% of TeX increases, you will probably want to edit this, but
% it should be fine as is for beginners.

% almost certainly you want these
\usepackage{amssymb}
\usepackage{amsmath}
\usepackage{amsfonts}

% used for TeXing text within eps files
%\usepackage{psfrag}
% need this for including graphics (\includegraphics)
%\usepackage{graphicx}
% for neatly defining theorems and propositions
%\usepackage{amsthm}
% making logically defined graphics
%%%\usepackage{xypic}

% there are many more packages, add them here as you need them

% define commands here

\begin{document}
{\bf Theorem -} Every \PMlinkescapetext{finite} (or \PMlinkescapetext{limited}) hyperreal number $x \in {}^*\mathbb{R}$ admits a unique \PMlinkescapetext{decomposition} of the form
\begin{displaymath}
x = a + \epsilon
\end{displaymath}
where $a \in \mathbb{R}$ and $\epsilon$ is infinitesimal.

{\bf Remark :} This theorem just says that every finite hyperreal number has a real part and an infinitesimal part (just like real and imaginary parts in complex numbers).
%%%%%
%%%%%
\end{document}
