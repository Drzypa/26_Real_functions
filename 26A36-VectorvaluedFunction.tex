\documentclass[12pt]{article}
\usepackage{pmmeta}
\pmcanonicalname{VectorvaluedFunction}
\pmcreated{2013-03-22 19:02:19}
\pmmodified{2013-03-22 19:02:19}
\pmowner{pahio}{2872}
\pmmodifier{pahio}{2872}
\pmtitle{vector-valued function}
\pmrecord{8}{41915}
\pmprivacy{1}
\pmauthor{pahio}{2872}
\pmtype{Definition}
\pmcomment{trigger rebuild}
\pmclassification{msc}{26A36}
\pmclassification{msc}{26A42}
\pmclassification{msc}{26A24}
\pmrelated{Component}
\pmrelated{DifferenceOfVectors}
\pmdefines{integrable}

% this is the default PlanetMath preamble.  as your knowledge
% of TeX increases, you will probably want to edit this, but
% it should be fine as is for beginners.

% almost certainly you want these
\usepackage{amssymb}
\usepackage{amsmath}
\usepackage{amsfonts}

% used for TeXing text within eps files
%\usepackage{psfrag}
% need this for including graphics (\includegraphics)
%\usepackage{graphicx}
% for neatly defining theorems and propositions
 \usepackage{amsthm}
% making logically defined graphics
%%%\usepackage{xypic}

% there are many more packages, add them here as you need them

% define commands here

\theoremstyle{definition}
\newtheorem*{thmplain}{Theorem}

\begin{document}
Let $n$ be a positive integer greater than 1.\, A function $F$ from a subset $T$ of $\mathbb{R}$ to the Cartesian product $\mathbb{R}^n$ is called a \emph{vector-valued function} of one real variable.\, Such a function \PMlinkescapetext{joins} to any real number $t$ of $T$ a coordinate vector
$$F(t) \;=\; (f_1(t),\,\ldots,\,f_n(t)).$$
Hence one may say that the vector-valued function $F$ is composed of $n$ real functions \,$t \mapsto f_i(t)$,\, the values of which at $t$ are the components of $F(t)$.\, Therefore the function $F$ itself may be written in the component form
\begin{align}
F \;=\; (f_1,\,\ldots,\,f_n).
\end{align}


\textbf{Example.}\, The ellipse
$$\{(a\cos{t},\,b\sin{t})\,\vdots\;\; t \in \mathbb{R}\}$$
is the value set of a vector-valued function\, $\mathbb{R} \to \mathbb{R}^2$\, ($t$ is the eccentric anomaly).\\


\emph{Limit, derivative and integral} of the function (1) are defined componentwise through the equations
\begin{itemize}
\item $\displaystyle\lim_{t\to t_0}F(t) \;:=\; \left(\lim_{t\to t_0}f_1(t),\,\ldots,\,\lim_{t\to t_0}f_n(t)\right)$
\item $\displaystyle F'(t) \;:=\; \left(f_1'(t),\,\ldots,\,f_n'(t)\right)$
\item $\displaystyle\int_a^b\!F(t)\,dt \;:=\; \left(\int_a^b\!f_1(t)\,dt,\,\ldots,\,\int_a^b\!f_n(t)\,dt\right)$
\end{itemize}
The function $F$ is said to be \emph{continuous, differentiable or integrable} on an interval \,$[a,\,b]$\, if every component of $F$ has such a property.\\

\textbf{Example.}\, If $F$ is continuous on \,$[a,\,b]$,\, the set
\begin{align}
\gamma \;:=\; \{F(t)\,\vdots\;\;\; t \in [a,\,b]\}
\end{align}
is a (continuous) curve in $\mathbb{R}^n$.\, It follows from the above definition of the derivative $F'(t)$ that $F'(t)$ is the limit of the expression
\begin{align}
\frac{1}{h}[F(t\!+\!h)-F(t)]
\end{align}
as\, $h \to 0$.\, Geometrically, the vector (3) is parallel to the line segment connecting (the end points of the position vectors of) the points $F(t\!+\!h)$ and $F(t)$.\, If $F$ is differentiable in $t$, the direction of this line segment then tends infinitely the direction of the tangent line of $\gamma$ in the point $F(t)$.\, Accordingly, the direction of the tangent line is determined by the derivative vector $F'(t)$.

%%%%%
%%%%%
\end{document}
