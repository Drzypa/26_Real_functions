\documentclass[12pt]{article}
\usepackage{pmmeta}
\pmcanonicalname{IntegralMeanValueTheorem}
\pmcreated{2013-03-22 17:15:56}
\pmmodified{2013-03-22 17:15:56}
\pmowner{me_and}{17092}
\pmmodifier{me_and}{17092}
\pmtitle{integral mean value theorem}
\pmrecord{9}{39604}
\pmprivacy{1}
\pmauthor{me_and}{17092}
\pmtype{Theorem}
\pmcomment{trigger rebuild}
\pmclassification{msc}{26A06}
\pmrelated{EstimatingTheoremOfContourIntegral}

\endmetadata

%\usepackage{amssymb}
\usepackage{amsmath} %Needed for align & align* and to render proofs properly
%\usepackage{amsfonts}
\usepackage{amsthm}

%Named sets
%\newcommand{\R}{\mathbb{R}} %Real numbers (amssymb or amsfonts)
%\newcommand{\C}{\mathbb{C}} %Complex numbers (amssymb or amsfonts)

%Functions
\newcommand{\modulus}[1]{\left|{#1}\right|} %|z|
\newcommand{\integral}[4]{\int_{#1}^{#2}\!{#3}\,\mathrm{d}{#4}}

%Numbers
%\newcommand{\I}{\mathrm{i}} %sqrt{-1}
%\newcommand{\e}{\mathrm{e}} %exponential

%Letters
%\newcommand{\ve}{\varepsilon} %nice epsilon
\begin{document}
\newtheorem*{imvt}{The Integral Mean Value Theorem}

\begin{imvt}
If $f$ and $g$ are continuous real functions on an interval $[a,b]$, and $g$ is additionally non-negative on $(a,b)$, then there exists a $\zeta\in(a,b)$ such that
\[
  \integral{a}{b}{f(x)g(x)}{x}=f(\zeta)\integral{a}{b}{g(x)}{x}
.\]
\end{imvt}

\begin{proof}
Since $f$ is continuous on a closed bounded set, $f$ is bounded and attains its bounds, say $f\left(x_0\right)\leq f(x)\leq f\left(x_1\right)$ for all $x\in[a,b]$. Thus, since $g$ is non-negative for all $x\in[a,b]$
\[
  f\left(x_0\right)g(x)\leq f(x)g(x)\leq f\left(x_1\right)g(x)
.\]
Integrating both sides gives
\[
  f\left(x_0\right)\integral{a}{b}{g(x)}{x}\leq\integral{a}{b}{f(x)g(x)}{x}\leq f\left(x_1\right)\integral{a}{b}{g(x)}{x}
.\]
If $\integral{a}{b}{g(x)}{x}=0$, then $g(x)$ is identically zero, and the result follows trivially. Otherwise,
\[
  f\left(x_0\right)\leq\frac{\integral{a}{b}{f(x)g(x)}{x}}{\integral{a}{b}{g(x)}{x}}\leq f\left(x_1\right)
,\]
and the result follows from the intermediate value theorem.
\end{proof}
%%%%%
%%%%%
\end{document}
