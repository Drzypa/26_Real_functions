\documentclass[12pt]{article}
\usepackage{pmmeta}
\pmcanonicalname{FunctionalAnalysis}
\pmcreated{2013-03-22 16:28:19}
\pmmodified{2013-03-22 16:28:19}
\pmowner{PrimeFan}{13766}
\pmmodifier{PrimeFan}{13766}
\pmtitle{functional analysis}
\pmrecord{12}{38636}
\pmprivacy{1}
\pmauthor{PrimeFan}{13766}
\pmtype{Topic}
\pmcomment{trigger rebuild}
\pmclassification{msc}{26E35}

\endmetadata

% this is the default PlanetMath preamble.  as your knowledge
% of TeX increases, you will probably want to edit this, but
% it should be fine as is for beginners.

% almost certainly you want these
\usepackage{amssymb}
\usepackage{amsmath}
\usepackage{amsfonts}

% used for TeXing text within eps files
%\usepackage{psfrag}
% need this for including graphics (\includegraphics)
%\usepackage{graphicx}
% for neatly defining theorems and propositions
%\usepackage{amsthm}
% making logically defined graphics
%%%\usepackage{xypic}

% there are many more packages, add them here as you need them

% define commands here

\begin{document}
Functional analysis is the branch of mathematics, and specifically of analysis, concerned with the study of spaces of functions. It has its historical roots in the study of transformations, such as the Fourier transform, and in the study of differential and integral equations. This usage of the word functional goes back to the calculus of variations, implying a function whose argument is a function. Its use in general has been attributed to mathematician and physicist Vito Volterra and its founding is largely attributed to mathematician Stefan Banach.

\subsection{Normed vector spaces}
In the modern view, functional analysis is seen as the study of complete normed vector spaces over the real or complex numbers. Such spaces are called Banach spaces. An important example is a Hilbert space, where the norm arises from an inner product. These spaces are of fundamental importance in many areas, including the mathematical formulation of quantum mechanics. More generally, functional analysis includes the study of Fréchet spaces and other topological vector spaces not endowed with a norm.

An important object of study in functional analysis are the continuous linear operators defined on Banach and Hilbert spaces. These lead naturally to the definition of \PMlinkname{$C^*$-algebras}{CAlgebra} and other operator algebras.

\subsection{Hilbert spaces}
Hilbert spaces can be completely classified: there is a unique Hilbert space up to isomorphism for every cardinality of the base. Since finite-dimensional Hilbert spaces are fully understood in linear algebra, and since morphisms of Hilbert spaces can always be divided into morphisms of spaces with Aleph-null ($\aleph_0$) dimensionality, functional analysis of Hilbert spaces mostly deals with the unique Hilbert space of dimensionality Aleph-null, and its morphisms. One of the open problems in functional analysis is the invariant subspace problem, which conjectures that every operator on a Hilbert space has a non-trivial invariant subspace. Many special cases have already been proven.

\subsection{Banach spaces}
General Banach spaces are more complicated. There is no clear definition of what would constitute a base, for example.

For any real number $p \ge 1$, an example of a Banach space is given by "all Lebesgue-measurable functions whose absolute value's $p$-th power has finite integral" (see $L^p$ spaces).

In Banach spaces, a large part of the study involves the dual space: the space of all continuous linear functionals. The dual of the dual is not always isomorphic to the original space, but there is always a natural monomorphism from a space into its dual's dual. This is explained in the dual space article.

Also, the notion of derivative can be extended to arbitrary functions between Banach spaces. See, for instance, the Fréchet derivative.

\subsection{Major and foundational results}
Important results of functional analysis include:

The uniform boundedness principle applies to sets of operators with tight bounds. 
One of the spectral theorems (there are indeed more than one) gives an integral formula for the normal operators on a Hilbert space. This theorem is of central importance for the mathematical formulation of quantum mechanics. 

The Hahn-Banach theorem extends functionals from a subspace to the full space, in a norm-preserving fashion. An implication is the non-triviality of dual spaces. 
The open mapping theorem and closed graph theorem. 
See also: List of functional analysis topics.

\subsection{Foundations of mathematics considerations}
Most spaces considered in functional analysis have infinite dimension. To show the existence of a vector space basis for such spaces may require Zorn's lemma. Many very important theorems require the Hahn-Banach theorem, which relies on the axiom of choice that is strictly weaker than the Boolean prime ideal theorem.

\subsection{Points of view}
Functional analysis in its present form includes the following tendencies:

Soft analysis. An approach to analysis based on topological groups, topological rings, and topological vector spaces; 
Geometry of Banach spaces. A combinatorial approach primarily due to Jean Bourgain; 
Noncommutative geometry. Developed by Alain Connes, partly building on earlier notions, such as George Mackey's approach to ergodic theory; 
Connection with quantum mechanics. Either narrowly defined as in mathematical physics, or broadly interpreted by, e.g. Israel Gelfand, to include most types of representation theory. 

{\it This entry was adapted from the Wikipedia article \PMlinkexternal{Functional analysis}{http://en.wikipedia.org/wiki/Functional_analysis} as of December 18, 2006.}
%%%%%
%%%%%
\end{document}
