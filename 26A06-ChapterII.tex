\documentclass[12pt]{article}
\usepackage{pmmeta}
\pmcanonicalname{ChapterII}
\pmcreated{2014-08-03 22:52:11}
\pmmodified{2014-08-03 22:52:11}
\pmowner{PMBookProject}{1000683}
\pmmodifier{rspuzio}{6075}
\pmtitle{Chapter II}
\pmrecord{3}{87452}
\pmprivacy{1}
\pmauthor{PMBookProject}{6075}
\pmtype{Topic}
\pmclassification{msc}{26A06}

\endmetadata

% this is the default PlanetMath preamble.  as your knowledge
% of TeX increases, you will probably want to edit this, but
% it should be fine as is for beginners.

% almost certainly you want these
\usepackage{amssymb}
\usepackage{amsmath}
\usepackage{amsfonts}

% need this for including graphics (\includegraphics)
\usepackage{graphicx}
% for neatly defining theorems and propositions
\usepackage{amsthm}

% making logically defined graphics
%\usepackage{xypic}
% used for TeXing text within eps files
%\usepackage{psfrag}

% there are many more packages, add them here as you need them

% define commands here

\begin{document}
CHAPTER II
\begin{center}
RATES LIMITS DERIVATIVES
\end{center}

4. Rate of Increase.
Slope. In the study of any quantity,
its rate of increase (or decrease), when some related quantity
changes, is very important for any complete understanding.
Thus, the rate of increase of the speed of a boat when the
power applied is increased is a fundamental consideration.
Graphically, the rate of increase of $y$ with respect to $x$ is
shown by the rate of increase of the height
of a curve. If the curve is very flat, there
is a small rate of increase; if steep, a large
rate.

The steepness, or slope, of a curve shows
the rate at which the dependent variable is
increasing with respect to the independent
variable.

When we speak of the slope of a curve
at any point $P$ we mean the slope of its tangent 
at that point. To find this, we must
start, as in Analytic Geometry, with a {\it secant}
through $P$.

Fig.3

Let the equation of the curve, Fig. 3, be
$ y=x^2$, and let the point $P$ at which the slope is to be found, be
the point (2, 4).

Let $Q$ be any other point on the curve, and let $\Delta x$ represent
the difference of the values of $x$ at the two points $P$ and {\it Q}.
\footnote{$\Delta x$ may be regarded as an abbreviation of the phrase, 
`` difference of the
$x$'s.'' The quotient of two such differences is called a difference quotient.
Notice particularly that $\Delta x$ does {\it not} mean $\Delta\times x$. 
Instead of `` difference of
the $x$'s'' the phrases ` ` change in $x$ '' and ``increment of $x$'' are often used.}

Then, in the figure, $OA=2, AB=\Delta x,$ and $OB=2+\Delta x$.
Moreover, since $y=x^{2}$ at every point, the value of $y$ at $Q$ is
$BQ=(2+\Delta x)^{2}$.

The slope $S$ of the secant $PQ$ is the quotient of the 
differences $\Delta y$ and $\Delta x$:

$$S= \tan\angle MPQ=\frac{\Delta y}{\Delta x}=\frac{MQ}{PM}=
\frac{(2+\Delta x)^{2}-4}{\Delta x}=4+\Delta x.$$

The slope $m$ of the tangent at $P$, that is $\tan\angle MPT$, is the
limit of the slope of the secant as $Q$ approaches $P$.
The slope of the secant is the average slope of the curve between the points
$P$ and Q. The slope of the curve at the single point $P$ is the limit of this
average slope as $\mathrm{Q}$ approaches $P$.

But, since $S=4+\Delta x$, it is clear that the limit of $S$ as $Q$ 
approaches $P$ is 4, since $\Delta x$ approaches zero when $Q$ approaches
$P$; hence the slope $m$ of the curve is 4 at the point $P$.

At any other point the argument would be similar. If the
co\&quot;{o}rdinates of $P$ are $(a, a^2)$, those of $Q$ would be 
$[(a+\Delta x), (a+\Delta x)^{2}]$; and the slope of the secant would be the difference
quotient $\Delta y \div \Delta x$:
$$
S=\frac{\Delta y}{\Delta x}=\frac{(a+\Delta x)^{2}-a^{2}}{\Delta x}=
\frac{2a\Delta x+{\overline \Delta x}^{2}}{\Delta x}=2a+\Delta x.
$$
Hence the slope of the curve at the point $(a,\ a^{s})$ is 
\footnote{Read ``$\Delta x \doteq 0$'' as ``$\Delta x$ approaches zero.'' 
A detailed discussion of limits is given in \S 10, p. 16.}
$$
m=\lim_{\Delta x \doteq 0}S=\lim_{\Delta x\doteq 0}\Delta y/\Delta x=
\lim_{\Delta x \doteq 0}(2a+\Delta x)=2a.
$$
On the curve $y=x^2$, the slope at any point is numerically twice
the value of $x$.

When the slope can be found, as above, the {\it equation of the
tangent} at $P$ can be written down at once, by Analytic Geometry, 
since the slope $m$ and a point $(a,\ b)$ on a line determine
its equation:
$$
(y-b)=m(x-a).
$$
Hence, in the preceding example, at the point (2, 4), where
we found $m=4$, the equation of the tangent is
\begin{center}
$(y-4)=4(x-2)$, or $4x-y=4$.
\end{center}
At the point $(a, a^{2})$ on the curve $y=x^2$, we found $m=2a$;
hence the equation of the tangent there is
\begin{center}
$(y-a^2)=2a(x-a)$, or 2 $ax- y=a^2$.
\end{center}

5. General Rules.

A part of the preceding work holds true
for any curve, and all of the work is at least similar. Thus,
for any curve, the slope is
$$
m=\lim_{\Delta x \doteq 0}S=\lim_{\Delta x \doteq 0}(\Delta y/\Delta x);
$$
{\it that is, the slope} $m$ {\it of the curve is the limit of the 
difference quotient} $\Delta y/\Delta x$.

The changes in various examples arise in the calculation of
the difference quotient, $\Delta y+\Delta x$, or $S$.

{\it This difference quotient. is always obtained, as above, by finding
the value of} $y$ {\it at} $Q$, {\it from the value of} $x$ {\it at} $Q$, 
{\it from the equation of the curve, then finding} $\Delta y$  
{\it by subtracting from this the value of} $y$ {\it at} $P$, 
{\it and finally forming the difference quotient by dividing} $\Delta y$ 
{\it by} $\Delta x$.

\$. Slope Negative or Zero. If the slope of the curve is
{\it negative}, the rate of increase in its height is negative, that is,
the height is really {\it decreasing} with respect to the independent
variable. 
\footnote{Increase or decrease in the height is always measured as we go toward
the right, i.e. as the independent variable increases.}

If the slope is {\it zero}, the tangent to the curve is {\it horizontal}.
this is what happens ordinarily at a highest point (maximum)
or at a lowest point (minimum) on a curve.
\footnote{A maximum need not be the highest point on the entire curve, but merely
the highest point in a small arc of the curve about that point See \S 37, p. 63.
Horizontal tangents sometimes occur without any maximum or any minimum.
See \S 38, p. 63.}

{\it Example} 1. Thus the curve $y=x^{2}$, as we have just seen, has, at any
point $x=a$, a slope $m=2a$. Since $m1s$ positive when $a$ is positive, the
curve is rising on the right of the origin; since $m$ is negative when $a$ is
negative, the curve is falling (that is, thee height $y$ decreases as $x$ increases)
on the left of the origin. At the origin $m=0$; the origin is the lowest
point (a minimum) on the curve, because the curve falls as we come toward
the origin and rises afterwards.

{\it Example} 2. Find the slope of the curve
\begin{equation}
y=x^{2}+3x-6
\end{equation}
at the point where $x=-2$; also in general at a point $x=a$. Use these
values to find the equation
of the tangent at $x=2$; the
tangent at any point; the
maximum or minimum
points if any exist.

When $x=-2$, we find $y=-7$, ( $P$ in Fig. 4); taking
any second point $Q$, $(-2+\Delta x, -7+\Delta y)$, its 
co\&quot;{o}rdinates must satisfy the given equation, therefore

(2) $-7+\Delta y=(-2+\Delta x)^{2}+3(-2+\Delta x)-6$,

or

(3) $\Delta y=-4 +\overline{\Delta x}^{2}+ 
\Delta x=-\Delta x+\overline{\Delta x}^{2}$,

where $\overline{\Delta}^{\eta}x$ means the square

of $\Delta x$. Hence the slope of
the secant $PQ$ i8
\begin{center}
(4)   $S=\Delta y/\Delta x=-1+\Delta x$.
\end{center}
The slope $m$ of the curve is the limit of $S$ as $\Delta x$ approaches zero; {\it i.e}.
\begin{center}
(5)   $m=\displaystyle \lim_{\Delta x \doteq 0} S=
\lim_{\Delta x \doteq 0}\frac{\Delta y}{\Delta x}=
\lim_{\Delta x \doteq 0}(-1+\Delta x)=-1$.
$$
u\underline{\wedge}0\ \mathrm{A}ae\pm 0M\ \mathrm{A}a=0
$$
\end{center}
It follows that the equation of the tangent at $(-2,\ -7)$ is
\begin{center}
(6)   $(y+7)=-1(x+2)$, or $x+y+9=0$.
\end{center}
Likewise, if we take the point $P(a,\ b)$ in any position on the curve
whatsoever, the equation (1) gives
\begin{center}
(7)   $b=a^{2}+3a-6$.
\end{center}
Any second point $Q$ has co\:{o}rdinates $(a+\Delta x,\ b+\Delta y)$ where $\Delta x$ and $\Delta y$ are the differences in $x$ and in $y$, respectively, 
between $P$ and $Q$. Since $Q$ also lies on the curve, these 
co\:{o}rdinates satisfy (1) :
%\begin{center}
%\includegraphics[width=53.47mm,height=64.90mm]{./chap2_images/image001.eps}
%\end{center}

\begin{center}
(8)   $b+\Delta y=(a+\Delta x)^{2}+3(a+\Delta x)-5$.
\end{center}
Subtracting the equation (7) from (8),
$\Delta y=2a\Delta x+{\overline\Delta x}^{2}+3\Delta x$, whence $S=\Delta y/\Delta x=(2 a + 3)+ \Delta x$,
and
\begin{center}
(9)   $m=\displaystyle \lim_{x\doteq 0}S=
\lim_{x\doteq 0}\frac{\Delta y}{\Delta x}=
\lim_{x\doteq 0}[(2a+3)+\Delta x]=2a+3$.
\end{center}
Therefore the tangent at $(a,\ b)$ is
\begin{center}
(10) $y-(a^{2}+3a-6)=(2a+3)(x-a)$, or $(2\ a+3)x-y=a^{2}+5$.
\end{center}
From (9) we observe that $m=0$, when $2a+3=0$, {\it i.e.}. when $a=-3/2$.
For all values greater than $-3/2, m=(2a+3$) is positive; for all
values less than $-3/2, m$ is negative. Hence the curve has a minimum
at $(-3/2,\ -29/4)$ In Fig. 4, since the curve falls as we come toward
this point and rises afterwards.

{\it Example} 3. Consider the curve $y=x^{2}-12x+7.$ 
If the value of $x$
at any point $P$ is $a$, the value of $y$ is $a^{2}-12a+7$. If the value of $x$ at
$Q$ is $a+\Delta x$, the value of $y$ at $Q$ is $(a+\Delta x)^{2}-12(a+\Delta x)+7$.

Hence
$$
S=\frac{\Delta y}{\Delta x}=\frac{[(a+\Delta x)^{2}-12(a+\Delta x)+7]-[a^{2}-12a+7]}{\Delta x}
$$
$$
=(3a^{2}+3a\Delta x+\overline{\Delta x}^{2})-12,
$$
and
$$
m=\lim_{\Delta \doteq 0}\frac{\Delta y}{\Delta x}=3a^{2}-12.
$$

For example, if $x=1, y=-4$; at this point $(1,\ -4)$
the slope is S. $1^{2}-12=-9$:
and the equation of the tangent is
$(y+4)=-9(x-1)$, or
\begin{center}
$ 9x+y-5=0$.
\end{center}
Since 3 $a^{2}-12$ is negative
when $a^{2}&lt;4$, the curve is falling when $a$ lies between $-2$
and $+2$. Since $3a^{2}-12$ is
positive when $a^{2}&gt;4$, the curve is rising whien $x&lt;-2$ and when $x&gt;+2$.
At $x=\pm 2$, the slope is zero. At $x=+2$ there is a minimum (see Fig.
5 $)$, since the curve i8 falling before this point and rising afterwards. At
$x=-2$ there is a maximum. At $x=+2, y=(2)^{2}-12\cdot 2+7=-9$,
which is the lowest value of $y$ near that point. At $x=-2, y=23$, the
highest value near it.

This information is quite useful in drawing an accurate figure. We
know also that the curve rises faster and faster to the right of $x=2$.

Draw an accurate figure of your own on a large scale.

\begin{center}
EXERCISES III.--SLOPES OF CURVES
\end{center}

1. Find the slope of the curve $y=x^{2}+2$ at the point where $x=1$.
Find the equation of the tangent at that point. Verify the fact that the
equation obtained i8 a straight line, that it has the correct slope, and that
it passes through the point $(1, 3)$.

2. Draw the curve $y=x^{2}+2$ on a large scale. Through the point
$(1, 3)$ draw secants which make $\Delta x=1,3,0.1,0.01$, respectively. 
Calculate the slope of each of these secants and show that the values are 
approaching the value of the slope of the curve at (1, 3).

3. Find the slope of the curve and the equation of the tangent to
each of the following curves at the point mentioned. Verify each answer
as in Ex. 1.

(a) $y=3x^{2};(1,3)$.\ (d) $y=x^{2}+4x-5;(1,0)$.

(b)$y=2x^{2}-5;(2,3)$\. (e) $y=x^3+x^{2};(1,2)$.

(c) $y=x^{3};(1,1)$.\ (f) $y=x^{3}-3x+4;(2,6)$.

4. Find the slope of the curve $y=x^{2}-3x+1$ at any point $x=a$;
from this find the highest (maximum) or lowest (minimum) point (if
any), and show in what portions the curve is rising or falling.

5. Draw the following curves, using for greater accuracy the precise
values of $x$ and $y$ at the highest (maximum) and the lowest (minimum)
points, and the knowledge of the values of $x$ for which the curve rises or
falls. The slope of the curve at the point where $x=0$ is also useful in
$(b), (c), (e), (g)$.

(a) $y=x^{2}+5x+2$.\ (d) $y=x^{4}$.\ (g) $y=2x^3-8x$.

(b) $y=x^3$.\ (e)$y=-x^{2}+3x$.\ (h) $y=x^3-6x+5$.

(c) $y=x^3-3x+4$.\ (f) $y=3+12x-x^{3}$. (i) $y=x^{3}+x^{2}$.

6. Show that the slope of the graph of $y=ax+b$ is always $m=a$,
(1) geometrically,(2) by the methods of \S 6.

7. Show that the lowest point on $y=x^{2}+px+q$ is the point where
$x=-p/2,$ (1) by Analytic Geometry, (2) by the methods of \S 6.

8. The normal to a curve at a point ls defined in Analytic Geometry
to be the perpendicular to the tangent at that point. Its slope $n$ is shown
to be the negative reciprocal of the slope $m$ of the tangent: $n=-1/m$.
Find the slope of the normal, and the equation of the normal in Ex. 1;
in each of the equations under Ex. 8.

9. The slope $m$ of the curve $y=x^{2}$ at any point where $x=a$ is
$m=2a$. Show that the slope is $+1$ at the point where $a=1/2$. Find the
points where the slope has the value $-1,2,10$. Note that if the curve
is drawn by taking different scales on the two axes, the slope no longer
means the tangent of the angle made with the horizontal axis.

10. Find the points on the following curves where the slope has the
values assigned to it;
\begin{center}
(a) $y=x^{2}-3x+6;(m=1,\ -1,2)$.
\end{center}
\begin{center}
(b) $y=x^3;(m=0.\ +1,\ +6)$.
\end{center}
\begin{center}
(c) $y=x^3-3x+4;(m=9,1)$.
\end{center}

11. Show that the curve $y=x^{3}-0.03x+2$ has a minimum at 
$(0.1, 1.998)$ and a maximum at $(-0.1,2.002)$. Draw the curve near the point
$(0,2)$ on a very large scale.

12. Draw each of the following curves on an appropriate scale; in
each case show that the peculiar twist of the curve through its maximum
and minimum would have been overlooked in ordinary plotting by
points:
\begin{center}
(a) $y=48x^{3}-x+1$.
\end{center}
[HINT. Use a very small vertical scale and a rather large horizontal
scale. The slope at $x=0$ is also useful.]
\begin{center}
(b) $y=x^3-30x^{2}+297x$.
\end{center}
[HINT. Use an exceedingly small vertical scale and a moderate 
horizontal scale. The slope at $x=10$ is also useful.]

7. Speed.

An important case of rate of change of a quantity is the
rate at which a body moves,--its speed.

Consider the motion of a body falling from rest under the
influence of gravity. During the first second it passes over
16 ft., during the next it passes over 48 ft., during the third
over 80 ft. In general, if $t$ is the number of seconds, and $s$
the entire distance it has fallen, $s=16t^{2}$ if the gravitational
constant $g$ be taken as 32. The graph of this equation (see
Fig. 6) is a parabola with its vertex at the origin.

The speed, that is the rate of increase of the space passed
over, is the slope of this curve, {\it i.e}.
$$
\lim_{\Delta t \doteq 0}\Delta s/\Delta t.
$$

This may be seen directly in another way. The {\it average}
speed for an interval of time $\Delta t$ is found by dividing the 
difference between the space passed over at the beginning and at
the end of that interval of time by the difference in time: {\it i.e}.
the average speed is the difference quotient $\Delta s\div\Delta t$. By the
speed at a given instant we mean {\it the limit of the average speed}
over an interval $\Delta t$ beginning or ending at that instant as that
interval approaches zero, {\it i.e}.
$$
\mathrm{speed}=\lim_{\Delta t \doteq 0}\Delta s/\Delta t.
$$
%\includegraphics[width=97.03mm,height=73.32mm]{./chap2_images/image002.eps}

Taking the equation $s = 16 t^2$, if $t = 1/2$, $s = 4$. 
(See point P in Fig. 6).  After a lapse of time $\Delta t$, the
new values are $t = 1/2 + \Delta t$, and $s = 16 (t = 1/2 + \Delta t)^2$
(Q in Fig. 6)

Then
$$
\Delta s = 16 (t = 1/2 + \Delta t)^2 - 4
= 16 \Delta t + 16 \overline{\Delta t}^2,
$$
$$
\frac{\Delta t}{\Delta s} = 16 + 16 \Delta t.
$$
Whence
$$
\mathrm{speed} = \lim_{\Delta t \doteq 0} \frac{\Delta t}{\Delta s} =
\lim_{\Delta t \doteq 0}(16 + 16 \Delta t) = 16 ;
$$
that is, the speed at the end of the first half second is 16 ft. per second. 

Likewise, for any value of $t$, say $t=T$, $s=16T^2$, while for
$t = T+\Delta t$, $s = 16(T+\Delta t)^2$; hence
$$
\mathrm{average\ speed}=\frac{\Delta s}{\Delta t}=\frac{16(T+\Delta t)^{2}-16T^{l}}{\Delta t}=32T+16 \Delta t
$$
and 
$$
\mathrm{speed} = \lim_{\Delta t \doteq 0} \frac{\Delta t}{\Delta s} =32T.
$$

Thus, at the end of two seconds, $T=2$, and the speed is $32 \cdot 2=64$, 
in feet per second.

% \includegraphics[width=96.60mm,height=76.83mm]{./chap2_images/image003.eps}

8. Component Speeds.  Any curve may be regarded as the
path of a moving point.  If a point $P$ does move along a curve,
both $x$ and $y$ are fixed when the time $t$ is fixed.  To specify
the motion completely, we need equations which give the values
of $x$ and $y$ in terms of $t$. 

The {\it horizontal speed} is the rate of increase of $x$ with
respect to the time.  This may be thought of as the speed of
the projection $M$ of $P$ on the $x$-axis.  As shown in \S 7,
this speed is the limit of the difference quotient
$\Delta x \div \Delta t$ as $\Delta t \doteq 0$.

Likewise, the {\it vertical speed} is the limit of the difference quotient
$\Delta y \div \Delta t$ as $\Delta t \doteq 0$.  Since the slope
$m$ of the curve is the limit of $\Delta y \div \Delta x$ as
$\Delta x \doteq 0$; and since
$$
\frac{\Delta y}{\Delta x} = 
\frac{\Delta y}{\Delta t} \div \frac{\Delta x}{\Delta t},
$$
it follows that
$$
 m=({\it vertical\ speed}) \div ({\it horizontal\ speed});
$$
that is, the slope of the curve is the ratio of the rate of increase
of $y$ to the rate of increase of $x$.

9. Continuous Functions.
In \S\S 4-8, we have supposed that
the curves used were {\it smooth}. The functions which we have
bad have all been representable by smooth curves; except
perhaps at isolated points, to a small change in the value of
one co\&quot;{o}rdinate, there has been a correspondingly small change
in the value of the other co\&quot;{o}rdinate. Throughout this text,
unless the contrary is expressly stated, the functions dealt with
will be of the same sort. Such functions are called continuous.
(See \S 10, p. 17.)

The curve $y=1/x$ is continuous except at the point $x=0$; $y=\tan x$
is continuous except at the points $x=\pm\pi/2, \pm 3\pi/2,$ etc. 
Such exceptional points occur frequently; we do not discard a curve because of them, but it is understood that any of our results may fail at such points.

\begin{center}
EXERCISES IV.--SPEED
\end{center}

1. From the formula $s=16t^{2}$, calculate the values of $s$ when $t=1,2,
1.1, 1.01, 1.001$. From these values calculate the average speed between
$t=1$ and $t=2$; between $t=1$ and $t=1.1$ ; between $t=1$ and $t=1.01$;
between $t=1$ and $t=1.001$. Show that these average speeds are 
successively nearer to the speed at the instant $t=1$.

2. Calculate as in Ex. 1 the average speed for smaller and smaller 
intervals of time after $t=2$; and show that these approach the speed at the
instant $t=2$.

3. A body thrown vertically downwards from any height with an
original velocity of 100 ft. per second, passes over in time $t$ 
(in seconds) a distance $s$ (in feet) given by the equation 
$s=100t+16t^{2}$ (if $g=32$, as in \S 7). Flnd the speed $v$ at the 
time $t=1$; at the time $t=2$; at the time $t=4$; at the time $t=T$.

4. In Ex. $3$ calculate the average speeds for smaller and smaller 
intervals of time after $t=0$; and show that they approach the original
speed $v_{0}=100$. Repeat the calculations for intervals beginning with $t=2$.

5. Calculate the speed of a body at the times indicated in the 
following possible relations between $s$ and $t$:

(a)$s=t^{2};t=1,2,10,T$. (c) $s=-16t^{2}+160t;t=0,2,5$.

(b) $s=16t^{2}-100t;t=0,2,T$. (d) $s=t^3-3t+4;t=0,1/2,1$.

6. The relation (c) in Ex. 5 holds (approximately, since $g=32$ 
approximately) for a body thrown upward with an initial speed of 160 ft.
per second, where $s$ means the distance from the starting point counted
positive upwards. Draw a graph which represents this relation between
the values of $s$ and $t$.

In this graph mark the greatest value of $s$. What is the value of $v$ at
that point? Find {\it exact} values of $s$ and $t$ for this point.

7. A body thrown horizontally with an original speed of 4 ft. per
second falls in a vertical plane curved path so that the values of its 
horizontal and its vertical distances from its original position are respectively,
$x=4t,  y=16t^2$, where $y$ is measured downwards. Show that the vertical
speed is $32 T$, and that the horizontal speed is 4, at the instant $t=T$.
Eliminate $t$ to show that the path is the curve $y=x^{2}$.

8. Show by Ex. 7 and \S 8 that the slope of the curve $y=x^{2}$ at the
point where $t=1$, {\it i.e.} $(4,16)$, is $32 \div 4$, or 8. Write the
equation of the tangent at that point.

9. Show that the slope of the curve $y=x^{2}$ (Ex. 7) at the point 
$(a,\ a^{2})$, {\it i.e.} $t=a/4$, is $2a$, from Ex. 7 and \S 8; and also directly by means of \S 6.

10. If a body moves so that its horizontal and its vertical distances
from the starting point are, respectively, $x=16t^{2}, y=4t$, show that its
path is the curve $y^{2}=x$; that its horizontal speed and its vertical speed
are, respectively, $32 T$ and 4, at the instant $t=T$.

11. From Ex. 10 and \S 8 show that the slope of the curve $y^{2}=x$ at the
point $(16, 4)$, {\it i.e.} when $t=1$, is $4\div 32=1/8$. Write the equation of the tangent at that point.

12. From Ex. 10 and \S 8 show that the slope of the curve $y^{2}=x$ at the
point where $t=T$ is $4\div(32T)=1/(8T)=1/(2k)$, where $k$ is the value
of $y$ at the point. Compare this result with that of Ex. 8.

10. Limits. Infinitesimals.

We have been led in what precedes to make use of limits. 
Thus the tangent to a curve at
the point $P$ is defined by saying that its slope is the {\it limit} of
the slope of a variable secant through $P$; the speed at a given
instant is the {\it limit} of the average speed; the difference of the
two values of $x, \Delta x$, was thought of as {\it approaching zero}; and
so on. To make these concepts clear, the following precise
statements are necessary and desirable.

{\it When the difference between the variable} $x$ {\it and a constant} $a$ {\it becomes and remains less, in absolute value, than} \footnote{
When dealing with real numbers, absolute value is the value without
regard to signs so that the absolute value of $-2$ is $2$. A convenient symbol
for it is two vertical lines; thus $|3-7|=4$.}
{\it any preassigned positive quantity, however small, then} $a$ {\it is the} limit {\it of the variable} $x$.

We also use the expression `` $x$ approaches $a$ as a limit,'' or,
more simply, `` $x$ approaches $a.$'' The symbol for limit is $\lim$;
the symbol for approaches is $\doteq$ thus we may write 
$\lim x=a$, or $x \doteq a$, or $\lim(a-x)=0$, or $a-x \doteq 0$.

When the limit of a variable is {\it zero}, the variable is called
an infinitesimal. Thus $a-x$ above is an infinitesimal. The
difference between any variable and its limit is always an 
infinitesimal. {\it When a variable} $x$ {\it approaches a limit} $a$, 
{\it any} oontinuous {\it function} $f(x)$ {\it approaches the limit} $f(a)$: thus, if $y=f(x)$ and $b=f(a)$, we may write
\begin{center}
$\displaystyle \lim_{x \doteq a}y=b$, or 
$\displaystyle \lim_{x \doteq a}f(x)=f(a)$.
\end{center}
This condition is the precise definition of continuity at the
point $x=a$. (See \S 9, p. 14.)

11. Properties of Limits.
The following properties of limits
will be assumed as self-evident; some of them have already
been used in the articles noted below.

THEOREM A. {\it The limit of the sum of two variables is the sum
of the limits of the two variables}. This is easily extended to the
case of more than two variables. (Used in \S\S 4, 6, and 7.)

THEOREM B. {\it The limit of the produot of two variables is the
product of the limits of the variables}. (Used in \S\S 4, 6, and 7.)

THEOREM C. {\it The limit of the quotient of one variable divided
by another is the quotient of the limits of the variables}, provided
{\it the limit of the divisor is not zero}. (Used in \S 8.)

The exceptional case in Theorem $\mathrm{C}$ is really the most 
interesting and important case of all. The exception arises
because when zero occurs as a denominator, the division 
cannot be performed. In finding the slope of a curve, we consider
$\displaystyle \lim(\Delta y/\Delta x)$ {\it as} $\Delta x$ {\it approaches zero}; notice that this is precisely the case ruled out in Theorem C. Again, the speed is
$\displaystyle \lim(\Delta s/\Delta t)$ {\it as} $\Delta t$ {\it approaches zero}. The limit of any such
{\it difference quotient} is one of these exceptional cases.

Now it is clear that the slope of a curve (or the speed of an
object) may have a great variety of values in different cases:
{\it no one answer is sufficient} for all examples, in the case of the
limit of a quotient when the denominator approaches zero.

THEOREM D. The {\it limit of the ratio of two infinitesimals
pends upon the law connecting them; otherwise it is quite inderminate}. 
Of this the student will see many instances; for
{\it thee Differential Calculus consists of the consideration of just
such limits}. In fact, the very reason for the existence of the 
Differential Calculus is that the exceptional case of Theorem 
$\mathrm{C}$ is important, and cannot be settled in an offhand manner.

The thing to be noted here is, that, no matter how small two
quantities may be, their ratio may be either small or large;
and that, if the two quantities are variables whose limit is
zero, the limit of their ratio may be either finite, zero, or
non-existent. In our work with such forms we shall try to
substitute an equivalent form whose limit can be found.
Obviously, to say that two variables are vanishing implies
nothing about the limit of their {\it ratio}.

12. Ratio of an Arc to its Chord.

Another important illustration of a ratio of infinitesimals is 
the ratio of the chord of
a curve to its subtended arc:
$$
R=\frac{\mathrm{chord}\ PQ}{\mathrm{arc}\ PQ}.
$$
% \includegraphics[width=39.24mm,height=33.78mm]{./chap2_images/image004.eps}
If $Q$ approaches $P$, both the arc
and the chord approach zero. At
any stage of the process the arc is
greater than the chord; but as $Q$
approaches $P$ this difference 
diminishes very rapidly, and the
ratio $R$ approaches 1:
$$
\lim_{Q \doteq P}R=
\lim_{PQ \doteq 0}\frac{\mathrm{chord}\ PQ}{\mathrm{arc}\ PQ}=1.
$$

This property is self-evident because it amounts to the same
thing as the definition of the length of the curve; we ordinarily
think of the length of an are of a curve as the limit of the
length of an inscribed broken line, as the lengths of the
segments of the broken line approach zero. Thus, the length
of circumference of a circle is defined to be the limit of the
perimeter of an inscribed polygon
as the lengths of all its sides 
approach zero. This would not be
true if the ratio of an arc to its
chord did not approach $1$.
\footnote{This point of view is fundamental. See Goursat-Hedrick,
{\it Mathematical Analysis}, Vol. I, \S 80, p. 161. At some exceptional
points the property may fail, but such points we always subject to 
special investigation.}
% \begin{center}
% \includegraphics[width=38.90mm,height=39.50mm]{./chap2_images/image005.eps}
% \end{center}

la. Ratio of the Sine of an
Angle to the Angle. In a circle,
the arc $PQ$ and the chord $PQ$ can
be expressed in terms of the angle
at the center. Let $\alpha=\angle QOP/2$;
then $\mathrm{arc}\ PQ=2\alpha\times r$ if $\alpha$ 
is measured in {\it circular measure} (see
{\it Tables}, II, F, 3); and the chord $PQ=2r\sin a$, 
since $r\sin\alpha=AP$.

It follows that
$$
\lim_{\alpha \doteq 0}\frac{\mathrm{chord}\ PQ}{\mathrm{arc}\ PQ}=
\lim_{\alpha \doteq 0}\frac{2r\sin a}{2r\alpha}=
\lim_{\alpha \doteq 0}\frac{\sin\alpha}{\alpha}=1,
$$
hence $\displaystyle \lim_{\alpha \doteq 0}
\frac{\sin \alpha}{\alpha}=1$,
for we have just seen that the limit of the ratio of an infinitesimal 
chord to its arc is 1.

This result is very important in later work; just here it
serves as a new illustration of the ratio of infinitesimals: 
{\it the ratio of the sine of an angle to the angle itself 
(measured in circular measure) approaches 1 as the angle 
approaches zero}.

14. Infinity.
Theorem D accounts for the case when the
numerator as well as the denominator in Theorem C is 
infinitesimal. There remains the case when the denominator only
is infinitesimal. {\it A variable whose reciprocal is infinitesimal is
said to} become infinite as {\it the reciprocal approaches zero}.

Thus $y=1/x$ is a variable whose reciprocal is $x$. As $x$ 
approaches zero, $y$ is said to become infinite. Notice however
that $y$ has no value whatever when $x=0$.
Likewise $y=\sec x$ is a variable whose reciprocal, $\cos x$, is
infinitesimal as $x$ approaches $\pi/2$; hence we say that $\sec x$ 
becomes infinite as $x$ approaches $\pi/2$.

In any case, it is clear that a variable which becomes infinite
becomes and remains larger in absolute value than any 
preassigned positive number, however large.

The student should carefully notice that infinity is not a
number; when we say that '` $\sec x$ becomes infinite as $x$ 
approaches $\pi/2$,
\footnote{Or, as is stated in short form in many texts, ``$\sec(\pi/2)=\infty.$''}
we do not mean that $\sec(\pi/2)$ has a value, we
merely tell what occurs when $x$ approaches $\pi/2$.

\begin{center}
EXERCISES V.--LIMITS AND INPINITESSIMALS
\end{center}

1. Imagine a point traversing a line-segment in such fashion that it
traverses half the segment in the first second, half the remainder in the
next second, and so on; always half the remainder in the next following
second. Will it ever traverse the entire line ? Show that the remainder
after $t$ seconds is $1/2^{t}$, if the total length of the segment is 1. 
Is this infinitessimal? Why?

2. Show that the distance traversed by the point in Ex. 1 in $t$ seconds
is $1/2+1/2^{n}+\cdots+1/2^{t}$. Show that this sum is equal to $1-1/2^{t}$; hence show that its limit is 1. Show that in any case the limit of the distance
traversed ls the total distance, as $t$ increases indefinitely.

3. Show that the limit of $3-x^{2}$ as $x$ approaches zero is 3. State this
result in the symbols used in \S 10. Draw the graph of $y=3-x^{2}$ and
show that $y$ approaches 3 as $x$ approaches zero.

4. Evaluate the following limits:

(a)$\displaystyle \lim_{x \doteq 0}(2-6x+3x^{2})$.
(d)$\displaystyle \lim_{x \doteq 1}\frac{3-2x^{2}}{4+2x^{2}}$. 
(g)$\displaystyle \lim_{x \doteq 7}\frac{x^{2}-3x+2}{x^{2}+2x+3}$.

(b)$\displaystyle \lim_{x \doteq 1}(2-6x+3x^{2})$.
(e)$\displaystyle \lim_{x \doteq0}\frac{x_{j}}{1-x}$.
(h)$\displaystyle \lim_{x \doteq0}\frac{a+bx}{c+dx}$.

(c) $\displaystyle \lim_{x \doteq k}(a+bx+cx^{2})$.
(f) $\displaystyle \lim_{x \doteq 1}\frac{1-x}{x}$. 
(i) $\displaystyle \lim_{x \doteq 0}\frac{a+bx+cx^{2}}{m+nx+lx^{2}}$.

5. If the numerator and denominator of a fraction contain a common
factor, that factor may be canceled in finding a limit, since the value of
the fraction which we use is not changed. Evaluate before and after
canceling a common factor:

(a) $\displaystyle \lim_{x \doteq 1}\frac{(x+2)(x+1)}{(2x+3)(x+1)}$.
(b) $\displaystyle \lim_{x \doteq 0}\frac{x(x+2)}{(x+1)(x+2)}$.

Evaluate after (not before) removing a common factor:

(c) $\displaystyle \lim_{x \doteq 0}\frac{x^{2}}{x}$.
(d) $\displaystyle \lim_{x \doteq 1}\frac{x^{2}-3x+2}{x^{2}-1}$.
(e) $\displaystyle \lim_{x \doteq 1}\frac{(x+2)(x-1)}{(2x+3)(x-1)}$.

(f) $\displaystyle \lim_{x \doteq 1}\frac{\sqrt{x-1}}{x-1}$
(g) $\lim_{x \doteq  0}\frac{x^{2}(x+1)^{2}}{x^{3}+2x^{2}}$
(h) $\displaystyle \lim_{x \doteq  0}\frac{x^{n}}{x}=\left\{\begin{array}{l}
0,n&gt;1\\ \mathrm{l},n=\mathrm{l} \end{array}\right.$

6. Show that
$$
\lim_{x \doteq \infty}\frac{2x^{2}+3}{x^{2}+4x+b}=2.
$$
[HINT. Divide numerator and denominator by $x^{2}$; then such 
terms as $3/x^2$ approach zero as $x$ becomes infinite.]

7. Evaluate:

(a) $\displaystyle \lim_{x \doteq \infty}\frac{2x+1}{3x+2}$.
(b) $\displaystyle \lim_{x \doteq \infty}\frac{2x^{2}-4}{3x^{2}+2}$.
(c) $\displaystyle \lim_{x \doteq \infty}\frac{ax+b}{mx+n}$.

(d) $\displaystyle \lim_{x \doteq \infty}\frac{x}{\sqrt{1+x^{2}}}$.
(e) $\displaystyle \lim_{x \doteq \infty}\frac{\sqrt{1+x^{2}}}{\sqrt{x^{2}-1}}$ .
(f) $\displaystyle \lim_{x \doteq \infty}\frac{\sqrt{ax^2+bx+c}}{mx+n}$.

8. Let $0$ be the center of a circle of radius $r=OB$, and let 
$a=\angle {\it COB}$ be an angle at the center. Let $BT$ be
perpendicular to $OB$, and let $BF$ be perpendicular to
$OC$. Show that $OF$ approaches $OC$ as $\alpha$ 
approaches zero; likewise arc $OB \doteq 0$, arc $DB \doteq 0$,
and $FC \doteq 0$, as $\alpha \doteq 0$.

%\begin{center}
%\includegraphics[width=28.15mm,height=20.95mm]{./chap2_images/image006.eps}
%\end{center}

9. In the figure of Ex. 8, show that the 
obvious geometric inequality $FB&lt; \mathrm{arc} CB&lt;BT$
is equivalent to $ r\sin\alpha&lt;r\cdot\alpha&lt;r\tan\alpha$, if $\alpha$ is measured in circular
measure. Hence show that $\alpha/\sin\alpha$ lies between 1 and 
$1/\cos\alpha$, and therefore that $\displaystyle \lim(\alpha/\sin\alpha)=1$ 
as $\alpha \doteq 0$. (Verification of. \S l3.)

10. In the flgure of Ex. 8, show that

$\displaystyle \lim{\alpha \doteq 0} \frac{FB}{r}=0;
\lim_{\alpha \doteq 0} \frac{OF}{r}=1;
\lim_{\alpha \doteq 0} \frac{BT}{r}=0;
\lim_{\alpha \doteq 0} \frac{FC}{r}=0;
\lim_{\alpha \doteq 0} \frac{\mathrm{arc}\ CB}{r}=0$.

11. Show that the following quantities become infinite as the 
independent variable approaches the value specified: in $(a)$ and $(b)$ draw the graph.

(a) $\displaystyle \lim_{x \doteq 0}\frac{1}{x^{2}}$. 
(b) $\displaystyle \lim_{x \doteq 1}\frac{x+2}{x-1}$

(c) $\displaystyle \lim_{x \doteq 0}\frac{r}{FC}$, (Ex. 8). 
(d) $\displaystyle \lim_{x \doteq 0}\frac{f}{BT}$, (Ex. 8). 

(e) $\displaystyle \lim_{x \doteq 0}\frac{x^n}{x}, (n&lt;1)$.
(f) $\displaystyle \lim_{x \doteq 2}\frac{2x+3}{\sqrt{x^{2}-3x+2}}$.

12. As the chord of a circle approaches zero, which of the following
ratios has a finite limit, which is infinitesimal, and which is becoming
infinite: the chord to its arc; the radius to the chord; the sector of the
arc to the triangle cut off by the chord; the area of the circle to the 
sector; the chord of twice the arc to the chord of thrice the arc; the 
radius of the circle to the chord of an arc a thousand times the given arc ?

13. Is the sum of two infinitesimals itself infinitesimal ? Is the 
difference ? Is the product Is the quotient ? Is a constant times an
infinitesimal an infinitesimal?

14. If to each of two integers an infinitesimal is added, show that the
ratio of these sums differs from that of the integers by an infinitesimal.
[See Ex. 4 $(h).$]

15. Show that the graph of $y=f(x)$ has a vertical asymptote if $f(x)$
becomes infinite as $x=a$. Illustrate this by drawing the following graphs:

(a) $y=\displaystyle \frac{3x+2}{x-2}$.
(c) $y=\displaystyle \frac{1}{1-\cos x}$.
(e) $y=\displaystyle \frac{1}{\sqrt{1-x^{2}}}$.

(b) $y=\displaystyle \frac{2x-1}{(x+1)(x-b)}$.
(d) $y=\displaystyle \frac{e^{x}+e^{-x}}{e^{x}-e^{-x}}$.
(f) $y=\displaystyle \frac{ax+b}{cx+d}$.

15. Derivatives. While such illustrations as those in \S 12
and Exercises V, above, are interesting and reasonably important, 
by far the most important cases of the ratio of two infinitesimals 
are those of the type studied in \S\S 4-8, in which each
of the infinitesimals is the difference of two values of a variable, 
such as $\Delta y/\Delta x$ or $\Delta s/\Delta t$. Such a difference q
quotient $\Delta y/\Delta x$ of $y$ with respect to $x$ evidently 
represents the {\bf average}
rate of increase of $y$ with respect to $x$ in the interval $\Delta x$; 
if $x$ represents time and $y$ distance, then $\Delta y/\Delta x$ is the 
average speed over the interval $\Delta x$ (\S 7, p. 13); 
if $y=f(x)$ is thought of as a curve, then $\Delta y/\Delta x$ is the 
slope of a secant or the average rate of rise of the curve in the 
interval $\Delta x$ (\S 4, p. 6).

The {\it limit} obtained in such cases represents the {\bf instantaneous}
{\it rate of increase} of one variable with respect to the other, --
this may be tho slope of a curve, or the speed of a moving
object, or some other {\it rate}, depending upon the nature of the
problem in which it arises.

In general, {\it the limit of the quotient} $\Delta y/\Delta x$ 
{\it of two infinitesimal differences is called} the derivative of 
$y$ with respect to $x$; {\it it is represented by the symbol} $dy/dx$:
$$
\frac{dy}{dx} \equiv \mathrm{derivative\ of\ y\ with\ respect\ to\ x} =
\lim_{\Delta x \doteq 0} \frac{\Delta y}{\Delta x}.
$$
Henceforth we shall use this new symbol $dy/dx$ or other
convenient abbreviations; 
\footnote{Often read `` {\it the} $x$ derivative of $y.$'' Other names sometimes used are
{\it differential coefficient}, and {\it derived function}. 
Other convenient notations
often used are $D_{x}y, y_{x}, y',\dot{y}$ ; the last two are not safe 
unless it is otherwise
clear what the independent variable is.} 
but the student must not forget the
real meaning: {\it slope}, in the case of curve; 
{\it speed}, in the case of
motion; some other tangible concept in any new problem
which we may undertake; {\it in every case the rate of increase of}
$y$ {\it with respect to} $x$.

Any mathematical formulas we obtain will apply in any of
these cases; we shall use the letters $x$ and $y$, the letters $s$ and
$t$, and other suggestive combinations; but the student should
remember that any formula written in $x$ and $y$ also holds true,
for example, with the letters $s$ and $t$, or for any other pair of
letters.

16. Formula for Derivatives. If we are to find the value of
a derivative, as in \S\S 4-7, we must have given one of the 
variables $y$ as a function of the other $x$:
\begin{center}
(1)   $y=f(x)$.
\end{center}
If we think of (1) as a curve, we may, as in \S 4, take any
point $P$ whose co\&quot;{o}rdinates are $x$ and $y$, and join it by a secant
$PQ$ to any other point $Q$, whose co\&quot;{o}rdinates are 
$x+\Delta x, y+\Delta y$.
Here $x$ aud $y$ represent fixed values
of $x$ and $y$; this will prove more 
convenient than to use new letters
each time, as we did in'\S\S 4-7.

Since $P$ lies on the curve (1), its
co\&quot;{o}rdinates $(x,\ y)$ satisfy the 
equation (1), $y=f(x)$. Since $Q$ lies on
(1), $x+\Delta x$ and $y+\Delta y$ satisfy the
same equation; hence we must have
\begin{center}
(2)   $y+\Delta y=f(x+\Delta x)$.
\end{center}
Subtracting (1) from (2) we get
\begin{center}
(3)   $\Delta y=f(x+\Delta x)-f(x)$;
\end{center}
whence the {\it difference quotient} is
\begin{center}
(4) $\displaystyle \frac{\Delta y}{\Delta x}=
\frac{f(x+\Delta x)-f(x)}{\Delta x}=
{\it average\ slope\ over\ PM}$,
\end{center}
and therefore the {\it derivative} is
$$
\frac{dy}{dx} =
\lim_{\Delta x \doteq 0} \frac{\delta y}{\Delta x} \equiv
\lim_{\Delta x \doteq 0} \frac{f(x+\Delta x)-f(x)}{\Delta x}=
{\it slope\ at\ P}
$$
\footnote{Instead of {\it slope}, read {\it speed} in case the problem 
deals with a motion, as in \S 7. In general, $\Delta y/\Delta x$ is the
{\it average} rate of increase, and $dy/dx$ is the
{\it instantaneous} rate.}
% \includegraphics[width=37.21mm,height=26.46mm]{./chap2_images/image007.eps}

This formula is often convenient; we shall apply it at once.

17. Rule for Differentiation.

The process of finding a derivative is called differentiation. 
To apply formula (5) of \S 16:

{\it (A) Find} $(y+\Delta y)$ {\it by substituting} $(x+\Delta x)$ 
{\it for} $x$ {\it in the given function or equation}; this gives 
$y+\Delta y=f(x+\Delta x)$.

{\it (B) Subtract} $y$ {\it from} $y+\Delta y$; this gives $\Delta y=f(x+\Delta x)-f(x)$.

{\it (C) Divide} $\Delta y$ {\it by} $\Delta x$ {\it to find the difference quotient} $\Delta|J/\Delta x$; {\it simplify this result}.

{\it (D) Find the} {\bf limit} {\it of} $\Delta y/\Delta x$ {\it as} 
$\Delta x$ {\it approaches zero}; this
result is the {\bf derivative}, $dy/dx$.

{\it Example} 1. Given $y=f(x)\equiv x^{2}$, to flnd $dy/dx$.

$(A) f(x+\Delta x)=(x+\Delta x)^{2}$.

$(B) \Delta y=f(x+\Delta x)-f(x)=(x+\Delta x)^{2}-x^{2}=2x\Delta x+\overline{\Delta x}^{2}$.

$(C) \Delta y/\Delta x=(2x\Delta x+\overline{\Delta x}^{2})+\Delta x=2x+\Delta x$.

$(D) dy/dx = \displaystyle \lim_{\Delta x \doteq 0}\Delta y/\Delta x=\lim_{\Delta x \doteq 0}(2x+\Delta x)=2x$.

Compare this work and the answer with the work of \S 4, p. 6.

{\it Example} 2. Given $y=f(x)\equiv\theta-12x+7$, to flnd $dy/k$.

$(A) f(x+\Delta x) =(x+\Delta x)^3-12(x+\Delta x)+7$.

$(B) \Delta\nu=f(x+\Delta x)-f(x)=3x^{2}\Delta x+3x\overline{\Delta x}^{2}+\overline{\Delta x}^{3}-12\Delta x$.

$(C) \Delta y/\Delta x=3x^{2}+3x\Delta x+\overline{\Delta x}^{2}-12$.

$(D) dy/dx=\displaystyle \lim_{\Delta x \doteq 0}\Delta y/\Delta x=
\lim_{\Delta x \doteq 0}(3x^{2}+3x\Delta x+\overline{\Delta x}^{2}-12)=
3x^{2}-12$

Compare thls work and the answer with the work of Example 3, \S 6.

{\it Example} 3. Given $y=f(x)\equiv 1/x^{2}$, to flnd $dy/dx$.

$(A) f(x+\Delta x)=\frac{1}{(x+\Delta x)^{2}}$ .

$(B) \displaystyle \Delta y=f(x+\Delta x)-f(x)=
\frac{1}{(x+\Delta x)^{2}}-\frac{1}{x^{2}}=
-\frac{2x\Delta x+\overline{\Delta x}^2}{x^{2}(x+\Delta x)^{2}}\cdot$

$(C) \displaystyle \Delta y/\Delta x=-\frac{2x+\Delta x}{x^{2}(x+\Delta x)^{2}}\cdot$

$(D) dy/dx=\lim_{\Delta x \doteq 0}\displaystyle \frac{\Delta y}{\Delta x}=
\lim_{\Delta x \doteq 0}\left[-\displaystyle \frac{2x+\Delta x}
{x^{2}(x+\Delta x)^{2}}\right]=
-\frac{2x}{x^{4}}=-\frac{2}{x^{3}}\cdot$

{\it Example} 4. Given $y=f(x)\equiv\sqrt{x}$, to flnd $dy/dx$, or $ df(x)/d\alpha$

$(A) f(x+\Delta x)=\sqrt{x+\Delta x}$.

$(B) \Delta y=f(x+\Delta x)-f(x)=\sqrt{x+\Delta x}-\sqrt{x}$.

(C) $\displaystyle \frac{\Delta y}{\Delta x}=
\frac{\sqrt{x+\Delta x}-\sqrt{x}}{\Delta x}
=\frac{\sqrt{x+\Delta x}-\sqrt{x}}{\Delta x} \cdot 
\displaystyle \frac{\sqrt{x+\Delta x}+\sqrt{x}}{\sqrt{x+\Delta x}+\sqrt{x}}$
$$
=\frac{1}{\sqrt{x+\Delta x}+\sqrt{x}}
$$
$(D) \displaystyle \frac{dy}{dx}=
\lim_{\Delta x \doteq 0}\frac{\Delta y}{\Delta x}=
\lim_{\Delta x \doteq 0}\frac{1}{\sqrt{x+\Delta x}+\sqrt{}\overline{x}}=
\frac{1}{2\sqrt{x}}$.

(Compare Ex. 11, p. 10.)

{\it Example} 6. Given $y=f(x)\equiv x^{7}$, to flnd $df(x)/dx$.

$(A) f(x+\Delta x)=(x+\Delta x)^{7}=x^{7}+7x^{0}\Delta x+$( $\mathrm{terms}$ with a factor $\overline{\Delta x}^{2}$).

$(B) \Delta\nu=f(x+\Delta x)-f(x)=7x^{6}\Delta x+$( $\mathrm{terms}$ with a factor $\overline{\Delta}x^{2}$).

$(C) \Delta y/\Delta x=7x^6+$( $\mathrm{t}\epsilon \mathrm{ms}$ with a factor $\overline{\Delta}x^{2}$).

$(D) dy/dx=\displaystyle \lim_{\Delta x \doteq 0}\Delta y/\Delta x=\lim_{Delta x \doteq 0}$[ $7x^6+($ terms with a factor $\Delta x)$] $=7x^6$.

\begin{center}
EXERCISES VI.--FORMAL DIFFERENTIATION
\end{center}

1. Find the derivative of $y=x^{3}$ with respect to $x$. [Compare Ex. 8
(c), p. 11.] Write the equation of the tangent at the point $(2, 8)$ to the
curve $y=x^3$.

2. Find the derivatives of the following functions with respect to $x$:

(a)$x^{2}-3x+4$.\ (b) $x^3-6x+7$. (c) $x^4+5$.

(d) $x^{4}+3x^{2}-2$.\ (e) $x^3+2x^{2}-4$.\ (f) $x^{4}-3x^3+6x$.

(g) $\displaystyle \frac{1}{x^{2}}$.\ 
(h) $\displaystyle \frac{1}{x+1}$. 
(i) $\displaystyle \frac{1}{2x-3}$.

(j) $\sqrt{x+1}$.\
(k) $\displaystyle \frac{x}{x+1}$.\ 
(l) $\displaystyle \frac{2x+3}{x-2}$

3. Find the equation of the tangent and the equation of the normal
to the curve $y=1/x$ at the point where $x=2$. (See Ex. 8, p. 11.)

4. Find the values of x for which the curve $y=x^{3}-16x+1$ rises
and those for which it falls; find the highest point (maximum) and the
lowest point (minimum). Draw the graph accurately.

5. Draw accurate graphs for the following curves:

(a)$y=x^3-18x+3$. (c) $y=x^{4}-32x$.

(b)$y=x^{3}+3x^{2}$.\ (d)$y=x^{4}-18x^{2}$.

6. Determine the speed of a body which moves so that
$$
s=16t^{2}+10t+5.
$$
[A body thrown down from a height with initial speed 10 ft. per second 
moves in this way approximately, if s is measured downward from a
mark 5 ft. above the starting point.]

7. If a body moves so that its horizontal and its vertical distances
from a point are, respectively, $x=10t, y=-16t^{2}+10t$, flnd its 
horizontal speed and its vertical speed. Show that the path is
$$
y=-16x^{2}/100+x,
$$
and that the slope of this path is the ratio of the vertioal speed to the
horizontal speed. [These equations represent, approximately, the motion
of an object thrown upward at an angle of $45^{\mathrm{o}}$ with a speed $10\sqrt{}\overline{2}.$]

8. A stone is dropped into still water. The circumference $\mathrm{c}$ of the
growing circular waves thus made, as a function of the radius $r$, is $c=2\pi r$.

Show that $d\mathrm{c}/dr=2\pi, i.e$. that the circumference changes 
$2 \pi$ times as fast as the radius.

Let $A$ be the area of the circle. Show that $dA/dr=2\pi r$; {\it i.e}. the
rate at which the area is changing compared to the radius is numerically
equal to the circumference.

9. Determine the rates of change of the following variables:
(a) The surface of a sphere compared with its radius, as the sphere
expands.
(b) The volume of a cube compaoed with its edge, as the cube enlarges.
(c) The volume of a right circular cone compared with the radius of
its base (the height being flxed), as the base spreads out.

10. If a man 6 ft. tall is at a distance $x$ from the base of an arc light
10 ft. high, and if the length of his shadow is $s$, show that $s/6=x/4$, or
$s=3x/2$. Find the rate $(ds/dx)$ at which the length $s$ of his shadow
increases as compared with hls distance $x$ from the lamp base.

11. The {\it specific heat} of a substance ({\it e.g}. water) is the amount of
heat required to raise the temperature of a unit volume of that substance $1^{\text{o}}$. (Centigrade). This amount is known to change for the same substance for different temperatures. The average speciflc heat between two 
temperatures is the ratio of the quantity of heat $\Delta H$ consumed in 
raising the temperature divided by the change $\Delta t$ in the temperature; show that the actual specific heat at a given temperature is $dH/dt$.

12. The {\it coefflcient of expansion} of a solid substance is the amount a
bar of that substance 1 ft. long will expand when the temperature changes
$1^{\mathrm{o}}$. Express the average coefflcient of expansion, and show that 
the coefficient of expansion at any given temperature is $dl/dt$, if the bar 
is precisely 1 ft. long at that temperature. (See also Ex. 12, p. 145.)
\end{document}
