\documentclass[12pt]{article}
\usepackage{pmmeta}
\pmcanonicalname{LaplaceTransformOfConvolution}
\pmcreated{2013-03-22 18:24:04}
\pmmodified{2013-03-22 18:24:04}
\pmowner{pahio}{2872}
\pmmodifier{pahio}{2872}
\pmtitle{Laplace transform of convolution}
\pmrecord{6}{41047}
\pmprivacy{1}
\pmauthor{pahio}{2872}
\pmtype{Theorem}
\pmcomment{trigger rebuild}
\pmclassification{msc}{26A42}
\pmclassification{msc}{44A10}
\pmsynonym{convolution property of Laplace transform}{LaplaceTransformOfConvolution}
\pmrelated{Convolution}

\endmetadata

% this is the default PlanetMath preamble.  as your knowledge
% of TeX increases, you will probably want to edit this, but
% it should be fine as is for beginners.

% almost certainly you want these
\usepackage{amssymb}
\usepackage{amsmath}
\usepackage{amsfonts}

% used for TeXing text within eps files
%\usepackage{psfrag}
% need this for including graphics (\includegraphics)
%\usepackage{graphicx}
% for neatly defining theorems and propositions
 \usepackage{amsthm}
% making logically defined graphics
%%%\usepackage{xypic}

% there are many more packages, add them here as you need them

% define commands here

\theoremstyle{definition}
\newtheorem*{thmplain}{Theorem}

\begin{document}
\textbf{Theorem.}\, If
$$\mathcal{L}\{f_1(t)\} \,=\, F_1(s) \quad\mbox{and}\quad \mathcal{L}\{f_2(t)\} \,=\, F_2(s),$$
then
$$\mathcal{L}\left\{\int_0^tf_1(\tau)f_2(t-\tau)\,d\tau\right\} \;=\; F_1(s)F_2(s).$$\\

{\em Proof.}\, According to the definition of Laplace transform, one has
$$\mathcal{L}\left\{\int_0^tf_1(\tau)f_2(t-\tau)\,d\tau\right\} 
 \;=\; \int_0^\infty e^{-st}\left(\int_0^t f_1(\tau)f_2(t-\tau)\,d\tau\right)dt,$$
where the right hand side is a double integral over the angular region bounded by the lines \,$\tau = 0$\, and\, $\tau = t$\, in the first quadrant of the $t\tau$-plane.\, Changing the \PMlinkescapetext{order} of integration, we write
$$\mathcal{L}\left\{\int_0^tf_1(\tau)f_2(t-\tau)\,d\tau\right\} 
 \;=\; \int_0^\infty\left(f_1(\tau)\int_\tau^\infty e^{-st}f_2(t-\tau)\,dt\right)d\tau.$$
Making in the inner integral the substitution\, $t-\tau := u$,\, we obtain
$$\int_\tau^\infty e^{-st}f_2(t-\tau)\,dt \,=\, \int_0^\infty e^{-(u+\tau)s}f_2(u)\,du 
= e^{-\tau s}\int_0^\infty e^{-su}f_2(u)\,du = e^{-\tau s}F_2(s),$$
whence
$$\mathcal{L}\left\{\int_0^tf_1(\tau)f_2(t-\tau)\,d\tau\right\} \;=\; \int_0^\infty f_1(\tau)e^{-\tau s}F_2(s)\,d\tau 
\,=\, F_2(s)\int_0^\infty f_1(\tau)e^{-\tau s}\,d\tau  = F_1(s)F_2(s),$$
Q.E.D.

%%%%%
%%%%%
\end{document}
