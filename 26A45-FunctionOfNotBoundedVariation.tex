\documentclass[12pt]{article}
\usepackage{pmmeta}
\pmcanonicalname{FunctionOfNotBoundedVariation}
\pmcreated{2013-03-22 17:56:29}
\pmmodified{2013-03-22 17:56:29}
\pmowner{pahio}{2872}
\pmmodifier{pahio}{2872}
\pmtitle{function of not bounded variation}
\pmrecord{6}{40438}
\pmprivacy{1}
\pmauthor{pahio}{2872}
\pmtype{Example}
\pmcomment{trigger rebuild}
\pmclassification{msc}{26A45}
\pmsynonym{example of unbounded variation}{FunctionOfNotBoundedVariation}
\pmsynonym{function of unbounded variation}{FunctionOfNotBoundedVariation}

% this is the default PlanetMath preamble.  as your knowledge
% of TeX increases, you will probably want to edit this, but
% it should be fine as is for beginners.

% almost certainly you want these
\usepackage{amssymb}
\usepackage{amsmath}
\usepackage{amsfonts}

% used for TeXing text within eps files
%\usepackage{psfrag}
% need this for including graphics (\includegraphics)
%\usepackage{graphicx}
% for neatly defining theorems and propositions
 \usepackage{amsthm}
% making logically defined graphics
%%%\usepackage{xypic}
\usepackage{pstricks}
\usepackage{pst-plot}

% there are many more packages, add them here as you need them

% define commands here

\theoremstyle{definition}
\newtheorem*{thmplain}{Theorem}

\begin{document}
\textbf{Example.}\, We show that the function
\begin{eqnarray*}
 f\!:\; x\mapsto\! & \left\{ \begin {array}{ll} x\cos\frac{\pi}{x} & \mbox{when}\,\,x \neq 0,\\
 0 & \mbox{when}\,\, x = 0,
 \end{array} \right.
 \end{eqnarray*}
which is continuous in the whole $\mathbb{R}$, is not of bounded variation on any interval containing the zero.

Let us take e.g. the interval \,$[0,\,a]$.\, Chose a positive integer $m$ such that \, $\frac{1}{m} < a$ and the partition of the interval with the points \, $\frac{1}{m},\, \frac{1}{m+1},\,\frac{1}{m+2},\, \ldots,\, \frac{1}{n}$\, into the subintervals
$[0,\,\frac{1}{n}],\; [\frac{1}{n},\,\frac{1}{n-1}],\;\ldots,\; [\frac{1}{m+1},\,\frac{1}{m}],\; [\frac{1}{m},\,a]$.\; For each positive integer $\nu$ we have (see \PMlinkname{this}{CosineAtMultiplesOfStraightAngle})
$$f\left(\frac{1}{\nu}\right) = \frac{1}{\nu}\cos\nu\pi = \frac{(-1)^\nu}{\nu}.$$
Thus we see that the total variation of $f$ in all partitions of \,$[0,\,a]$\, is at least
$$\frac{1}{n}\!+\!\left(\frac{1}{n}\!+\!\frac{1}{n\!-\!1}\right)\!+\ldots+\!\left(\frac{1}{m\!+\!1}+\frac{1}{m}\right)
= \frac{1}{m}+2\!\sum_{\nu=m+1}^n\frac{1}{\nu}.$$
Since the harmonic series diverges, the above sum increases to $\infty$ as\, $n\to\infty$.\, Accordingly, the total variation must be infinite, and the function $f$ is not of bounded variation on\, $[0,\,a]$.

It is not difficult to justify that $f$ is of bounded variation on any finite interval that does not contain 0.

\begin{thebibliography}{8}
\bibitem{lindelof}{\sc E. Lindel\"of}: {\em Differentiali- ja integralilasku
ja sen sovellutukset III. Toinen osa.}\, Mercatorin Kirjapaino Osakeyhti\"o, Helsinki (1940).
\end{thebibliography}
%%%%%
%%%%%
\end{document}
