\documentclass[12pt]{article}
\usepackage{pmmeta}
\pmcanonicalname{ProofOfDeLHopitalsRule}
\pmcreated{2013-03-22 13:23:31}
\pmmodified{2013-03-22 13:23:31}
\pmowner{paolini}{1187}
\pmmodifier{paolini}{1187}
\pmtitle{proof of De l'H\^opital's rule}
\pmrecord{10}{33930}
\pmprivacy{1}
\pmauthor{paolini}{1187}
\pmtype{Proof}
\pmcomment{trigger rebuild}
\pmclassification{msc}{26A24}
\pmclassification{msc}{26C15}
%\pmkeywords{Hospital}
%\pmkeywords{Hopital}

% this is the default PlanetMath preamble.  as your knowledge
% of TeX increases, you will probably want to edit this, but
% it should be fine as is for beginners.

% almost certainly you want these
\usepackage{amssymb}
\usepackage{amsmath}
\usepackage{amsfonts}

% used for TeXing text within eps files
%\usepackage{psfrag}
% need this for including graphics (\includegraphics)
%\usepackage{graphicx}
% for neatly defining theorems and propositions
%\usepackage{amsthm}
% making logically defined graphics
%%%\usepackage{xypic}

% there are many more packages, add them here as you need them

% define commands here

\newcommand{\R}{\mathbb R}
\begin{document}
Let $x_0\in \R$, $I$ be an interval containing $x_0$ and let $f$ and $g$ be two differentiable functions defined on $I\setminus\{x_0\}$ with $g'(x)\neq 0$ for all $x\in I$. Suppose that
\[
\lim_{x\to x_0} f(x) = 0, \quad \lim_{x\to x_0} g(x)=0
\]
and that
\[
   \lim_{x\to x_0} \frac{f'(x)}{g'(x)}=m.
\]

We want to prove that hence $g(x)\neq 0$ for all $x\in I\setminus\{x_0\}$ and
\[
  \lim_{x\to x_0} \frac{f(x)}{g(x)}=m.
\]

First of all (with little abuse of notation) we suppose that $f$ and $g$ are defined also in the point $x_0$ by $f(x_0)=0$ and $g(x_0)=0$. The resulting functions are continuous in $x_0$ and hence in the whole interval $I$. 

Let us first prove that $g(x)\neq 0$ for all $x\in I\setminus\{x_0\}$. If by contradiction $g(\bar x)=0$ since we also have $g(x_0)=0$, by Rolle's Theorem we get that $g'(\xi)=0$ for some $\xi\in (x_0,\bar x)$ which is against our hypotheses.

Consider now any sequence $x_n\to x_0$ with $x_n\in I\setminus\{x_0\}$.
By Cauchy's mean value Theorem there exists a sequence $x'_n$ such that
\[
  \frac{f(x_n)}{g(x_n)} = \frac{f(x_n)-f(x_0)}{g(x_n)-g(x_0)}
  = \frac{f'(x'_n)}{g'(x'_n)}.
\]
But as $x_n\to x_0$ and since $x'_n \in (x_0,x_n)$ we get that $x'_n\to x_0$ and hence
\[
  \lim_{n\to\infty} \frac{f(x_n)}{g(x_n)} 
  = \lim_{n\to\infty} \frac{f'(x_n)}{g'(x_n)}
  = \lim_{x\to x_0} \frac{f'(x)}{g'(x)} = m.
\]

Since this is true for any given sequence $x_n\to x_0$ we conclude that
\[
  \lim_{x\to x_0} \frac{f(x)}{g(x)} = m.
\]
%%%%%
%%%%%
\end{document}
