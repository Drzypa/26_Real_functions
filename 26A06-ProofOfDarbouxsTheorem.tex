\documentclass[12pt]{article}
\usepackage{pmmeta}
\pmcanonicalname{ProofOfDarbouxsTheorem}
\pmcreated{2013-03-22 12:45:04}
\pmmodified{2013-03-22 12:45:04}
\pmowner{paolini}{1187}
\pmmodifier{paolini}{1187}
\pmtitle{proof of Darboux's theorem}
\pmrecord{7}{33056}
\pmprivacy{1}
\pmauthor{paolini}{1187}
\pmtype{Proof}
\pmcomment{trigger rebuild}
\pmclassification{msc}{26A06}

\endmetadata

% this is the default PlanetMath preamble.  as your knowledge
% of TeX increases, you will probably want to edit this, but
% it should be fine as is for beginners.

% almost certainly you want these
\usepackage{amssymb}
\usepackage{amsmath}
\usepackage{amsfonts}

% used for TeXing text within eps files
%\usepackage{psfrag}
% need this for including graphics (\includegraphics)
%\usepackage{graphicx}
% for neatly defining theorems and propositions
%\usepackage{amsthm}
% making logically defined graphics
%%%\usepackage{xypic}

% there are many more packages, add them here as you need them

% define commands here

\newcommand{\Prob}[2]{\mathbb{P}_{#1}\left\{#2\right\}}
\newcommand{\norm}[1]{\left\|#1\right\|}

% Some sets
\newcommand{\Nats}{\mathbb{N}}
\newcommand{\Ints}{\mathbb{Z}}
\newcommand{\Reals}{\mathbb{R}}
\newcommand{\Complex}{\mathbb{C}}
\begin{document}
Without loss of generality we migth and shall assume $f'_{+}(a)>t>f'_{-}(b)$. Let $g(x):=f(x)-tx$.  
Then $g'(x)=f'(x)-t$, $g'_{+}(a)>0>g'_{-}(b)$, and we wish to find a zero of $g'$.

Since $g$ is a continuous function on $[a,b]$, it attains a maximum on $[a,b]$.  
Since $g'_+(a)>0$ and $g'_+(b)<0$ \PMlinkname{Fermat's theorem}{FermatsTheoremStationaryPoints} states that 
neither $a$ nor $b$ can be points where $f$ has a local maximum. 
So a maximum is attained at some $c \in (a,b)$.  But then $g'(c)=0$ again by \PMlinkname{Fermat's theorem}{FermatsTheoremStationaryPoints}.
%%%%%
%%%%%
\end{document}
