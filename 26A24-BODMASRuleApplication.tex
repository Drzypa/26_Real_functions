\documentclass[12pt]{article}
\usepackage{pmmeta}
\pmcanonicalname{BODMASRuleApplication}
\pmcreated{2014-08-08 11:22:45}
\pmmodified{2014-08-08 11:22:45}
\pmowner{burgess}{1001318}
\pmmodifier{burgess}{1001318}
\pmtitle{BODMAS Rule application}
\pmrecord{1}{}
\pmprivacy{1}
\pmauthor{burgess}{1001318}
\pmtype{Application}

% this is the default PlanetMath preamble.  as your knowledge
% of TeX increases, you will probably want to edit this, but
% it should be fine as is for beginners.

% almost certainly you want these
\usepackage{amssymb}
\usepackage{amsmath}
\usepackage{amsfonts}

% need this for including graphics (\includegraphics)
\usepackage{graphicx}
% for neatly defining theorems and propositions
\usepackage{amsthm}

% making logically defined graphics
%\usepackage{xypic}
% used for TeXing text within eps files
%\usepackage{psfrag}

% there are many more packages, add them here as you need them

% define commands here

\begin{document}
Today I want to share what Is BODMAS Rule & How to use it?

BODMAS rule decides the order of operations for add, subtract, multiply, divide, etc as below shown
B-Brackets (do all operations contained in the brackets first)
O-Orders (powers and square roots etc)
D-Division 
 M-Multiplication
A-Addition 
S- Subtraction

Let’s see an example and check how BODMASS Rule works
30-(2*6+15/3) +8*3/6

Step1: Brackets
2*6+15/3 = 12+5 = 17

Step2: Division
30-17+8*1/2

Step3: Multiplication
30-17+4

Step 4: Addition and Subtraction
30-17+4=17


20+30-5/4*2+ (1+6)
(5-6/4)+9*7
1-6*(2+9)/8


BODMAS Rule is very helpful in solving various algebra math problems.

\end{document}
