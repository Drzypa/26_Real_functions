\documentclass[12pt]{article}
\usepackage{pmmeta}
\pmcanonicalname{BinomialFormulaForNegativeIntegerPowers}
\pmcreated{2013-03-22 14:57:26}
\pmmodified{2013-03-22 14:57:26}
\pmowner{rspuzio}{6075}
\pmmodifier{rspuzio}{6075}
\pmtitle{binomial formula for negative integer powers}
\pmrecord{9}{36654}
\pmprivacy{1}
\pmauthor{rspuzio}{6075}
\pmtype{Corollary}
\pmcomment{trigger rebuild}
\pmclassification{msc}{26A06}
\pmrelated{GeneralizedBinomialCoefficients}

\endmetadata

% this is the default PlanetMath preamble.  as your knowledge
% of TeX increases, you will probably want to edit this, but
% it should be fine as is for beginners.

% almost certainly you want these
\usepackage{amssymb}
\usepackage{amsmath}
\usepackage{amsfonts}

% used for TeXing text within eps files
%\usepackage{psfrag}
% need this for including graphics (\includegraphics)
%\usepackage{graphicx}
% for neatly defining theorems and propositions
%\usepackage{amsthm}
% making logically defined graphics
%%%\usepackage{xypic}

% there are many more packages, add them here as you need them

% define commands here

\addtocounter{MaxMatrixCols}{2}
\begin{document}
For negative integer powers, the binomial formula can be written in terms of binomial coefficients like so:
 $$(1 - x)^{-n} = \sum_{m = 1}^\infty \binom{m+n-1}{n-1} x^m$$

{\it Proof: \,}  We shall prove this by induction on $n$.  First, note that, if $n=1$, then $\binom{m}{0} = 1$, so our formula reduces to
 $$(1 - x)^{-1} = \sum_{m = 1}^\infty x^m ,$$
which is the formula for the sum of an infinite geometric series.

Next, suppose that the formula is valid for a certain value of $n$.  Then we have
 $$(1 - x)^{-n-1} = (1 - x)^{-1} (1 - x)^{-n} = \left( \sum_{k = 0}^\infty x^k \right) \left( \sum_{m = 0}^\infty {m+n-1 \choose n-1} x^m \right)$$
The product of sums can be rewritten as the following double sum:
 $$\sum_{m = 0}^\infty \sum_{k = 0}^m {n+k-1 \choose n-1} x^m$$
The easiest way to see this is by rearranging the double sum as follows and adding columns
 $$\begin{matrix}
x^0 \sum_{m = 0}^\infty \binom{m+n-1}{n-1} x^m = & \binom{n-1}{n-1} & + & \binom{n}{n-1} x & + & \binom{n+1}{n-1} x^2 & + & \binom{n+2}{n-1} x^3 & + & \binom{n+3}{n-1} x^4 & + & \cdots \\
x^1 \sum_{m = 0}^\infty \binom{m+n-1}{n-1} x^m = & & & \binom{n-1}{n-1} x & + & \binom{n}{n-1} x^2 & + & \binom{n+1}{n-1} x^3 & + & \binom{n+2}{n-1} x^4 & + & \cdots \\
x^2 \sum_{m = 0}^\infty \binom{m+n-1}{n-1} x^m = & & & & & \binom{n-1}{n-1} x^2 & + & \binom{n}{n-1} x^3 & + & \binom{n+1}{n-1} x^4 & + & \cdots \\
x^3 \sum_{m = 0}^\infty \binom{m+n-1}{n-1} x^m = & & & & & & & \binom{n-1}{n-1} x^3 & + & \binom{n}{n-1} x^4 & + & \cdots \\
. & . & . & . & . & . & . & . & . & . & . & .
\end{matrix}$$
To evaluate the finite sums, we shall use the following identity for binomial coefficients.  (See the entry \PMlinkid{``binomial coefficient''}{273} for more information about this identity.)
 $$\sum_{k = 0}^m \binom{n+k-1}{n-1} = \binom{m + n}{n}$$
Inserting this result value for the finite sum back into the double sum, we obtain
 $$(1 - x)^{-n-1} = \sum_{m = 0}^\infty \binom{m + n}{n} x^m.$$
\rightline{Q.E.D.}
%%%%%
%%%%%
\end{document}
