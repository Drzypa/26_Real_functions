\documentclass[12pt]{article}
\usepackage{pmmeta}
\pmcanonicalname{FactsAboutRiemannStieltjesIntegral}
\pmcreated{2013-03-22 18:55:03}
\pmmodified{2013-03-22 18:55:03}
\pmowner{pahio}{2872}
\pmmodifier{pahio}{2872}
\pmtitle{facts about Riemann--Stieltjes integral}
\pmrecord{5}{41767}
\pmprivacy{1}
\pmauthor{pahio}{2872}
\pmtype{Topic}
\pmcomment{trigger rebuild}
\pmclassification{msc}{26A42}
\pmsynonym{properties of Riemann--Stieltjes integral}{FactsAboutRiemannStieltjesIntegral}
%\pmkeywords{Riemann Stieltjes integral}
\pmrelated{PropertiesOfRiemannStieltjesIntegral}

\endmetadata

% this is the default PlanetMath preamble.  as your knowledge
% of TeX increases, you will probably want to edit this, but
% it should be fine as is for beginners.

% almost certainly you want these
\usepackage{amssymb}
\usepackage{amsmath}
\usepackage{amsfonts}

% used for TeXing text within eps files
%\usepackage{psfrag}
% need this for including graphics (\includegraphics)
%\usepackage{graphicx}
% for neatly defining theorems and propositions
 \usepackage{amsthm}
% making logically defined graphics
%%%\usepackage{xypic}

% there are many more packages, add them here as you need them

% define commands here

\theoremstyle{definition}
\newtheorem*{thmplain}{Theorem}

\begin{document}
\begin{itemize}

\item If the integrator $g$ of the \PMlinkid{Riemann--Stieltjes integral}{3187}\, $\int_a^bf(x)\,dg(x)$ is the identity function, then the integral reduces to the Riemann integral $\int_a^bf(x)\,dx$.

\item If the integrand of the Riemann--Stieltjes integral is a constant function, one has
$$\int_a^b\!c\,dg(x) \;=\; c\!\cdot\!(g(b)\!-\!g(a)).$$

\item If the integrand $f$ is continuous and the integrator $g$ monotonically nondecreasing on the interval \,$[a,\,b]$,\, then there exists a number $\xi$ on the interval such that
$$\int_a^b\!f(x)\,dg(x) \;=\; f(\xi)(g(b)\!-\!g(a)).$$
Cf. the integral mean value theorem.

\item If $f$ is continuous, $g$ monotonically nondecreasing and differentiable on the interval \,$[a,\,b]$,\, then
$$\frac{d}{dx}\int_a^x\!f(t)\,dg(t) \;=\; f(x)g'(x) \quad \mbox{for\;\;} a < x < b.$$

\end{itemize}
%%%%%
%%%%%
\end{document}
