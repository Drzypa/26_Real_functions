\documentclass[12pt]{article}
\usepackage{pmmeta}
\pmcanonicalname{ReductionFormulasForIntegrationOfPowers}
\pmcreated{2013-03-22 18:36:53}
\pmmodified{2013-03-22 18:36:53}
\pmowner{pahio}{2872}
\pmmodifier{pahio}{2872}
\pmtitle{reduction formulas for integration of powers}
\pmrecord{10}{41349}
\pmprivacy{1}
\pmauthor{pahio}{2872}
\pmtype{Topic}
\pmcomment{trigger rebuild}
\pmclassification{msc}{26A36}
\pmclassification{msc}{26A09}
\pmsynonym{integration of powers}{ReductionFormulasForIntegrationOfPowers}
%\pmkeywords{power of function}
\pmrelated{GeneralFormulasForIntegration}
\pmrelated{IntegralTables}
\pmrelated{WallisFormulae}

\endmetadata

% this is the default PlanetMath preamble.  as your knowledge
% of TeX increases, you will probably want to edit this, but
% it should be fine as is for beginners.

% almost certainly you want these
\usepackage{amssymb}
\usepackage{amsmath}
\usepackage{amsfonts}

% used for TeXing text within eps files
%\usepackage{psfrag}
% need this for including graphics (\includegraphics)
%\usepackage{graphicx}
% for neatly defining theorems and propositions
 \usepackage{amsthm}
% making logically defined graphics
%%%\usepackage{xypic}

% there are many more packages, add them here as you need them

% define commands here

\theoremstyle{definition}
\newtheorem*{thmplain}{Theorem}

\begin{document}
The following reduction formulas, with integer $n$ and \PMlinkescapetext{derivable} via integration by parts, may be used for lowing ($n > 0$) or raising ($n < 0$) the the powers:
\begin{itemize}
\item $\displaystyle \int\sin^nx\,dx 
\;=\; -\frac{1}{n}\sin^{n-1}x\cos{x}+\frac{n\!-\!1}{n}\int\sin^{n-2}x\,dx \qquad (n \gtrless 0)$
\item $\displaystyle \int\cos^nx\,dx 
\;=\; \frac{1}{n}\cos^{n-1}x\sin{x}+\frac{n\!-\!1}{n}\int\cos^{n-2}x\,dx \qquad (n \gtrless 0)$
\item $\displaystyle\int(\ln{x})^n\,dx \;=\; x(\ln{x})^n-n\int(\ln{x})^{n-1}\,dx \qquad (n \gtrless 0)$
\item $\displaystyle \int\frac{1}{(1+x^2)^n}\,dx 
\;=\; \frac{1}{2n\!-\!2}\cdot\frac{x}{(1\!+\!x^2)^{n-1}}+\frac{2n\!-\!3}{2n\!-\!2}\int\frac{1}{(1\!+\!x^2)^{n-1}}\,dx
      \quad (n > 1)$
\end{itemize}

\textbf{Example.}\, For finding $\displaystyle\int\!\frac{dx}{\sin^3x}$, we apply the first formula with\, $n := -1$,\, getting first
$$\int\!\frac{dx}{\sin{x}} \,=\, -\frac{1}{-1}\cdot\frac{\cos{x}}{\sin^2x}+\frac{-2}{-1}\int\frac{dx}{\sin^3x}.$$
From this we solve
$$\int\!\frac{dx}{\sin^3x} \,=\, -\frac{1}{2}\frac{\cos{x}}{\sin^2x}+\int\!\frac{dx}{\sin{x}} 
\,=\, -\frac{1}{2}\frac{\cos{x}}{\sin^2x}+\ln\left|\tan\frac{x}{2}\right|+C$$
(see integration of rational function of sine and cosine).\\

\textbf{Note 1.}\, Instead of the two first formulae, it is simpler in the cases when $n$ is a positive odd or a negative even number to use the following\\
$\displaystyle\int\sin^{2m+1}x\,dx \,=\, \int\sin^{2m}x\sin{x}\,dx \,=\, -\int(1-\cos^2x)^m(-\sin{x})\,dx$,\\
$\displaystyle\int\cos^{2m+1}x\,dx \,=\, \int\cos^{2m}x\cos{x}\,dx \,=\, \int(1-\sin^2x)^m\cos{x}\,dx,$\\
$\displaystyle\int\frac{1}{\sin^{2m}x}\,dx \,=\; \int\frac{1}{\sin^{2m-2}x}\cdot\frac{1}{\sin^2x}\,dx
\,=\, -\int(1+\cot^2x)^{m-1}\,d\cot{x}$,\\
$\displaystyle\int\frac{1}{\cos^{2m}x}\,dx \,=\; \int\frac{1}{\cos^{2m-2}x}\cdot\frac{1}{\cos^2x}\,dx
\,=\, \int(1+\tan^2x)^{m-1}\,d\tan{x}$,\\
which may be found after making the powers on the right hand sides to polynomials.\\

\textbf{Note 2.}\, $\int\tan^nx\,dx$\, ($n \in \mathbb{Z}_+$)\, is obtained easily by the \PMlinkname{substitution}{IntegrationBySubstitution} \,$\tan{x} := t$,\; $dx = \frac{dt}{t^2\!+\!1}$\,  and a division; e.g.
\begin{align*}
\int\tan^5x\,dx &\,=\, \int\frac{t^5}{t^2\!+\!1}\,dt \,=\, \int\!\left(t^3-t+\frac{t}{t^2\!+\!1}\right)dt\\
&\,=\, \frac{t^4}{4}-\frac{t^2}{2}+\frac{1}{2}\ln(t^2\!+\!1)+C \\
&\,=\, \frac{\tan^4x}{4}-\frac{\tan^2x}{2}+\ln\sqrt{\tan^2x+1}+C.
\end{align*}

%%%%%
%%%%%
\end{document}
