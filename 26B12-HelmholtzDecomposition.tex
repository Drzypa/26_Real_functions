\documentclass[12pt]{article}
\usepackage{pmmeta}
\pmcanonicalname{HelmholtzDecomposition}
\pmcreated{2013-03-22 17:59:40}
\pmmodified{2013-03-22 17:59:40}
\pmowner{invisiblerhino}{19637}
\pmmodifier{invisiblerhino}{19637}
\pmtitle{Helmholtz decomposition}
\pmrecord{5}{40506}
\pmprivacy{1}
\pmauthor{invisiblerhino}{19637}
\pmtype{Definition}
\pmcomment{trigger rebuild}
\pmclassification{msc}{26B12}
\pmsynonym{fundamental theorem of vector calcululs}{HelmholtzDecomposition}

% this is the default PlanetMath preamble.  as your knowledge
% of TeX increases, you will probably want to edit this, but
% it should be fine as is for beginners.

% almost certainly you want these
\usepackage{amssymb}
\usepackage{amsmath}
\usepackage{amsfonts}

% used for TeXing text within eps files
%\usepackage{psfrag}
% need this for including graphics (\includegraphics)
%\usepackage{graphicx}
% for neatly defining theorems and propositions
%\usepackage{amsthm}
% making logically defined graphics
%%%\usepackage{xypic}

% there are many more packages, add them here as you need them

% define commands here

\begin{document}
The Helmholtz theorem states that any vector $\mathbf{F}$ may be decomposed into an irrotational (curl-free) and a solenoidal (divergence-free) part under certain conditions (given below). More precisely, it may be written in the form:
\begin{equation}
\mathbf{F} = -\nabla \varphi + \nabla \times \mathbf{A}
\end{equation}
where $\varphi$ is a scalar potential and $\mathbf{A}$ is a vector potential. By the definitions of scalar and vector potentials it follows that the first term on the right-hand side is irrotational and the second is solenoidal. The  general conditions for this to be true are:
\begin{enumerate}
\item The divergence of $\mathbf{F}$ must vanish at infinity.
\item The curl of $\mathbf{F}$ must also vanish at infinity.
\end{enumerate}

%%%%%
%%%%%
\end{document}
