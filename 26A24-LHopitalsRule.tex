\documentclass[12pt]{article}
\usepackage{pmmeta}
\pmcanonicalname{LHopitalsRule}
\pmcreated{2013-03-22 12:28:15}
\pmmodified{2013-03-22 12:28:15}
\pmowner{mathwizard}{128}
\pmmodifier{mathwizard}{128}
\pmtitle{l'H\^opital's rule}
\pmrecord{13}{32657}
\pmprivacy{1}
\pmauthor{mathwizard}{128}
\pmtype{Theorem}
\pmcomment{trigger rebuild}
\pmclassification{msc}{26A24}
\pmclassification{msc}{26C15}
\pmsynonym{l'Hospital's rule}{LHopitalsRule}
\pmrelated{IndeterminateForm}
\pmrelated{DerivationOfHarmonicMeanAsTheLimitOfThePowerMean}
\pmrelated{ImproperLimits}
\pmrelated{ExampleUsingStolzCesaroTheorem}

\endmetadata

% this is the default PlanetMath preamble.  as your knowledge
% of TeX increases, you will probably want to edit this, but
% it should be fine as is for beginners.

% almost certainly you want these
\usepackage{amssymb}
\usepackage{amsmath}
\usepackage{amsfonts}

% used for TeXing text within eps files
%\usepackage{psfrag}
% need this for including graphics (\includegraphics)
%\usepackage{graphicx}
% for neatly defining theorems and propositions
%\usepackage{amsthm}
% making logically defined graphics
%%%\usepackage{xypic} 

% there are many more packages, add them here as you need them

% define commands here
\begin{document}
L'H\^opital's rule states that given an unresolvable limit of the form $\frac{0}{0}$ or $\frac{\infty}{\infty}$, the ratio of functions $\frac{f(x)}{g(x)}$ will have the same limit at $c$ as the ratio $\frac{f'(x)}{g'(x)}$. In short, if the limit of a ratio of functions approaches an indeterminate form, then
$$\lim_{x\rightarrow c}\frac{f(x)}{g(x)} = \lim_{x\rightarrow c}\frac{f'(x)}{g'(x)}$$
provided this last limit exists. L'H\^opital's rule may be applied indefinitely as long as the conditions are satisfied. However it is important to note, that the nonexistance of $\lim\frac{f'(x)}{g'(x)}$ does not prove the nonexistance of $\lim\frac{f(x)}{g(x)}$.

\textbf{Example:}
We try to determine the value of
$$\lim_{x\to \infty}\frac{x^2}{e^x}.$$
As $x$ approaches $\infty$ the expression becomes an indeterminate form $\frac{\infty}{\infty}$. By applying L'H\^opital's rule twice we get
$$\lim_{x\to\infty}\frac{x^2}{e^x}=\lim_{x\to \infty}\frac{2x}{e^x}=\lim_{x\to \infty}\frac{2}{e^x}=0.$$
Another example of the usage of L'H\^opital's rule can be found \PMlinkid{here}{5741}.

%\textbf{Proof of L'H\^opital's Rule for} $\frac{0}{0}$ : 
%Let there be two differentiable functions $f(x)$ and $g(x)$, such that trying %to evaluate $\lim_{x\rightarrow c}\frac{f(x)}{g(x)}$ yields an indeterminate %form $\frac{0}{0}$. The linearization of $f$ at $c$ is $f(x)\approx %f(c)+f'(c)(x-c)$, and is similar for $g(x)$. Since $x$ is approaching $c$, we %can use the linearization at $c$ to replace $f(x)$ and $g(x)$. Thus 
%$$\lim_{x\rightarrow c}\frac{f(x)}{g(x)}=\lim_{x\rightarrow %c}\frac{f(c)+f'(c)(x-c)}{g(c)+g'(c)(x-c)}$$
%We know that $f(c)$ and $g(c)$ approach $0$, so we can ignore these terms. The %$x-c$ term cancels out and we're left with 
%$\frac{f'(c)}{g'(c)}$.
%$\displaystyle \lim_{x\rightarrow c}\frac{f(x)}{g(x)} = %\lim_{x\rightarrowc}\frac{f'(x)}{g'(x)}$

%Another way to prove this statement is by using Taylor's theorem, instead of %linearization. Thus
%$$f(x) = f'(c)(x-c) + R(x)(x-c)^2$$
%where $R(x)$ is a continuous function. When we use this substitution in the %limit, we see
%$$\lim_{x\rightarrow c} \frac{f'(c)(x-c) + R(x)(x-c)^2}{g'(c)(x-c) + %S(x)(x-c)^2}$$
%We can factor out an $x-c$ term leaving
%$$\lim_{x\rightarrow c} \frac{f'(c) + R(x)(x-c)}{g'(c) + S(x)(x-c)}$$
%which approaches $\frac{f'(c)}{g'(c)}$.

%[examples and proof at infinity coming]
%%%%%
%%%%%
\end{document}
