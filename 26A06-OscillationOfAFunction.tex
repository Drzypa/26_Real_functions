\documentclass[12pt]{article}
\usepackage{pmmeta}
\pmcanonicalname{OscillationOfAFunction}
\pmcreated{2013-03-22 17:45:50}
\pmmodified{2013-03-22 17:45:50}
\pmowner{perucho}{2192}
\pmmodifier{perucho}{2192}
\pmtitle{oscillation of a function}
\pmrecord{5}{40217}
\pmprivacy{1}
\pmauthor{perucho}{2192}
\pmtype{Definition}
\pmcomment{trigger rebuild}
\pmclassification{msc}{26A06}
\pmrelated{TotalVariation}

\endmetadata

% this is the default PlanetMath preamble.  as your knowledge
% of TeX increases, you will probably want to edit this, but
% it should be fine as is for beginners.

% almost certainly you want these
\usepackage{amssymb}
\usepackage{amsmath}
\usepackage{amsfonts}
\usepackage{amsthm}

% used for TeXing text within eps files
%\usepackage{psfrag}
% need this for including graphics (\includegraphics)
%\usepackage{graphicx}
% for neatly defining theorems and propositions
%\usepackage{amsthm}
% making logically defined graphics
%%%\usepackage{xypic}

% there are many more packages, add them here as you need them

% define commands here
\newtheorem{theorem}{Theorem}
\newtheorem{defn*}{Definition}
\newtheorem{prop}{Proposition}
\newtheorem{lemma}{Lemma}
\newtheorem{cor}{Corollary}
\DeclareMathOperator{\sgn}{sgn}
\begin{document}
\begin{defn*}
Let $f:X\subset\mathbb{R}\to\mathbb{R}$. The {\bf oscillation} of the function $f$ on the set $X$ is said to be
$$\omega(f,X)=\sup_{a,b\,\in\, X}|f(b)-f(a)|,$$
where $a, b$ are arbitrary points in $X$.
\end{defn*}

\subsection{Examples}
\begin{itemize}
\item $\omega(x^2, \, [-1,2])=4$
\item $\omega(x, \, [-1,2])=3$
\item $\omega(x, \, (-1,2))=3$
\item $\omega(\sgn x \, [-1,2])=2$
\item $\omega(\sgn x \, [0,2])=1$
\item $\omega(\sgn x \, (0,2])=0$
\end{itemize}
Cauchy's criterion can be expressed in terms of this concept.\cite{cite:Zorich}

\begin{thebibliography}{99}
\bibitem{cite:Zorich}
V., Zorich, {\em Mathematical Analysis I}, pp. 131, First Ed., Springer-Verlag, 2004. 
\end{thebibliography}

%%%%%
%%%%%
\end{document}
