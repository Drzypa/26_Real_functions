\documentclass[12pt]{article}
\usepackage{pmmeta}
\pmcanonicalname{ProofOfIntermediateValueTheorem}
\pmcreated{2013-03-22 12:33:56}
\pmmodified{2013-03-22 12:33:56}
\pmowner{yark}{2760}
\pmmodifier{yark}{2760}
\pmtitle{proof of intermediate value theorem}
\pmrecord{9}{32813}
\pmprivacy{1}
\pmauthor{yark}{2760}
\pmtype{Proof}
\pmcomment{trigger rebuild}
\pmclassification{msc}{26A06}

\usepackage{amssymb}
\usepackage{amsmath}
\usepackage{amsfonts}

\begin{document}
We first prove the following lemma.

If $f:[a,b] \to \mathbb{R}$ is a continuous function with $f(a) \le 0 \le f(b)$ then there exists a $c \in [a,b]$ such that $f(c) = 0$.

Define the sequences $(a_n)$ and $(b_n)$ inductively, as follows.

$$a_0 = a \quad b_0 = b$$
$$c_n = \frac{a_n + b_n}{2}$$
$$(a_n, b_n) = \begin{cases} (a_{n-1}, c_{n-1}) & f(c_{n-1}) \ge 0 \\			(c_{n-1}, b_{n-1}) & f(c_{n-1}) < 0  \end{cases}$$

We note that

$$ a_0 \le a_1 \le \cdots \le a_n \le b_n \le \cdots \le b_1 \le b_0 $$
\begin{equation} (b_n - a_n) = 2^{-n}(b_0 - a_0) \end{equation}
\begin{equation} \label{eqn} f(a_n) \le 0 \le f(b_n) \end{equation}

By the fundamental axiom of analysis $(a_n) \to \alpha$ and $(b_n) \to \beta$. But $(b_n - a_n) \to 0$ so $\alpha = \beta$.
By continuity of $f$
$$(f(a_n)) \to f(\alpha) \quad (f(b_n)) \to f(\alpha)$$
But we have $f(\alpha) \le 0$ and $f(\alpha) \ge 0$ so that $f(\alpha) = 0$. Furthermore we have $a \le \alpha \le b$, proving the assertion.

Set $g(x) = f(x) - k$ where $f(a) \le k \le f(b)$. $g$ satisfies the same conditions as before, so there exists a $c$ such that $f(c) = k$. Thus proving the more general result.
%%%%%
%%%%%
\end{document}
