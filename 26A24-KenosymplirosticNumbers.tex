\documentclass[12pt]{article}
\usepackage{pmmeta}
\pmcanonicalname{KenosymplirosticNumbers}
\pmcreated{2014-10-11 21:53:52}
\pmmodified{2014-10-11 21:53:52}
\pmowner{imaginary.i}{1001376}
\pmmodifier{imaginary.i}{1001376}
\pmtitle{Kenosymplirostic numbers}
\pmrecord{6}{88144}
\pmprivacy{1}
\pmauthor{imaginary.i}{1001376}
\pmtype{Definition}
\pmsynonym{numbers}{KenosymplirosticNumbers}
\pmsynonym{pi}{KenosymplirosticNumbers}
%\pmkeywords{types of numbers}
%\pmkeywords{numbers}
%\pmkeywords{greek numbers}
\pmrelated{pi}
\pmrelated{tau}
\pmrelated{phi}
\pmdefines{Kenosymplirostic numbers}

\endmetadata

% this is the default PlanetMath preamble.  as your knowledge
% of TeX increases, you will probably want to edit this, but
% it should be fine as is for beginners.

% almost certainly you want these
\usepackage{amssymb}
\usepackage{amsmath}
\usepackage{amsfonts}

% need this for including graphics (\includegraphics)
\usepackage{graphicx}
% for neatly defining theorems and propositions
\usepackage{amsthm}

% making logically defined graphics
%\usepackage{xypic}
% used for TeXing text within eps files
%\usepackage{psfrag}

% there are many more packages, add them here as you need them

% define commands here

\begin{document}
\begin{flushleft}
So, basically. I am now introducing a new number system called kenosymplirostic numbers.
It is a type of number that you get when dividing by 0.
So, what does kenosymplirostic mean?
The name comes from two Greek words, "\textit{keno}" meaning "gap" and "\textit{sympliroste}" meaning "fill".
These numbers fill the gap that was left open, the gap people thought will never be closed.
\end{flushleft}

Here is how it works:
\[\frac{1}{0}= 1\mathit{k}\]
The value of the numerator is the coefficient of "kenosym unit" k.

\textbf{So, what about 0/0?}
\begin{flushleft}
The thing is, it will be 0. Why? Because 0*k(the kenosym unit) is still 0 by definition.
\end{flushleft}



\begin{flushleft}
And now, we talk about its place in the complex plane, or with the addition of 
kenosyymplirostic numbers, the complex space.
It will go through the complex plane where it will meet in 0, the origin of real, imaginary and
kenosymplirostic numbers.
\end{flushleft}

ADDITION AND SUBTRACTION
\begin{flushleft}
So, let's say we have:
\end{flushleft}

\[1\mathit{k}+3\mathit{k}\]
It will be equal to:
\[\frac{1}{0}+\frac{3}{0}\]
Which will be
\[\frac{4}{0} or 4\mathit{k}\]
This means that subtraction is just the same, as we treat 0 as a normal number.

MULTIPLICATION
\begin{flushleft}
So, let's say we have:
\end{flushleft}
\[3k \cdot 5k\]
That would mean 3/0 * 5/0 which will be 15/0 or 15k.
(K IS NOT TREATED AS A VARIABLE)
Kenosymplirostic numbers cannot be divided since it will always equal to 0 per the properties
of dividing fractions.

EXPONENT
\begin{flushleft}
The kenosym unit k acts like the real number 1, where k to the power of any number n is k.
kenosym numbers with coefficients will just raise the coefficient to that power and just
copy the kenosym unit k.
\end{flushleft}

RULES OF COEFFICIENTS
\begin{flushleft}
Only integers coefficients are allowed since there cant be a fourth dimension in
the complex space and to avoiid many numbers having the same kenosym.
\end{flushleft}

Axioms:
Addition:
\mathit{a+b=b+a (Commutativity)

a+(b+c)=(a+b)+c (Associativity)

a+0=a (Identity element exists)

a+(−a)=0 (Inverse exists)

Multiplication:

ab=ba (Commutativity)

a(bc)=(ab)c (Associativity)

a*1=a (Identity element exists)


Distributive Property:

a(b+c)=ab+ac
}
\textbf{Shoutouts:
Herve Arki from the Mathematics G+ community for bringing up an issue. :)}

As of now, there are some paradoxes that might arise, but I will resolve  all once I gain 
more knowledge.
Also, if someone would help suggest to me a better kenosym unit, You can comment below
this article.
\end{document}
