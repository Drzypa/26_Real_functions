\documentclass[12pt]{article}
\usepackage{pmmeta}
\pmcanonicalname{FundamentalTheoremOfCalculusForKurzweilHenstockIntegral}
\pmcreated{2013-03-22 16:44:27}
\pmmodified{2013-03-22 16:44:27}
\pmowner{jirka}{4157}
\pmmodifier{jirka}{4157}
\pmtitle{fundamental theorem of calculus for Kurzweil-Henstock integral}
\pmrecord{4}{38964}
\pmprivacy{1}
\pmauthor{jirka}{4157}
\pmtype{Theorem}
\pmcomment{trigger rebuild}
\pmclassification{msc}{26A42}
\pmrelated{FundamentalTheoremOfCalculusClassicalVersion}

\endmetadata

% this is the default PlanetMath preamble.  as your knowledge
% of TeX increases, you will probably want to edit this, but
% it should be fine as is for beginners.

% almost certainly you want these
\usepackage{amssymb}
\usepackage{amsmath}
\usepackage{amsfonts}

% used for TeXing text within eps files
%\usepackage{psfrag}
% need this for including graphics (\includegraphics)
%\usepackage{graphicx}
% for neatly defining theorems and propositions
\usepackage{amsthm}
% making logically defined graphics
%%%\usepackage{xypic}

% there are many more packages, add them here as you need them

% define commands here
\theoremstyle{theorem}
\newtheorem*{thm}{Theorem}
\newtheorem*{lemma}{Lemma}
\newtheorem*{conj}{Conjecture}
\newtheorem*{cor}{Corollary}
\newtheorem*{example}{Example}
\newtheorem*{prop}{Proposition}
\theoremstyle{definition}
\newtheorem*{defn}{Definition}
\theoremstyle{remark}
\newtheorem*{rmk}{Remark}

\begin{document}
Let the $\int$ symbol denote the Kurzweil-Henstock integral.  We can then give the most general version of the fundamental theorem of calculus.

\begin{thm}
Let $F \colon [a,b] \to {\mathbb{R}}$ and suppose the derivative
$F'(x)$ exists for all $x \in [a,b]$.  Then
\begin{equation*}
\int_a^b F'(x) dx = F(b)-F(a) .
\end{equation*}
\end{thm}

The reader should note the subtle difference from the standard version.  Here we do not assume anything about $F'$ except that it exists.  For the standard version we usually assume that $F'$ is continuous, and if we use the Lebesgue integral we must assume that $F'$ is Lebesgue integrable.  Part of this theorem is that $F'$ is Kurzweil-Henstock integrable, hence no extra assumptions are necessary.

An example of a function where the standard version has problems is the function
\begin{equation*}
F(x) :=
\begin{cases}
x^2 \sin \frac{1}{x^2} & \text{ if $x \not= 0$} \\
0 & \text{ if $x = 0$} .
\end{cases}
\end{equation*}
$F$ is differentiable everywhere, but
\begin{equation*}
F'(x) =
\begin{cases}
2x \sin \frac{1}{x^2} - \frac{2}{x}\cos \frac{1}{x^2} & \text{ if $x \not= 0$} \\
0 & \text{ if $x = 0$} .
\end{cases}
\end{equation*}
Which is not continuous and in fact unbounded on any interval containing zero.
%%%%%
%%%%%
\end{document}
