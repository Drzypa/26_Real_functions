\documentclass[12pt]{article}
\usepackage{pmmeta}
\pmcanonicalname{Derivative}
\pmcreated{2013-03-22 12:41:41}
\pmmodified{2013-03-22 12:41:41}
\pmowner{rmilson}{146}
\pmmodifier{rmilson}{146}
\pmtitle{derivative}
\pmrecord{35}{32975}
\pmprivacy{1}
\pmauthor{rmilson}{146}
\pmtype{Definition}
\pmcomment{trigger rebuild}
\pmclassification{msc}{26B05}
\pmclassification{msc}{46G05}
\pmclassification{msc}{26A24}
\pmrelated{TableOfDerivatives}
\pmrelated{DerivativeOfInverseFunction}
\pmrelated{PartialDerivative}
\pmrelated{Gradient}
\pmrelated{RelatedRates}
\pmrelated{CalculatingLipschitzRatios}
\pmdefines{directional derivative}
\pmdefines{Fr\'echet derivative}

% this is the default PlanetMath preamble.  as your knowledge
% of TeX increases, you will probably want to edit this, but
% it should be fine as is for beginners.

% almost certainly you want these
\usepackage{amssymb}
\usepackage{amsmath}
\usepackage{amsfonts}

% used for TeXing text within eps files
%\usepackage{psfrag}
% need this for including graphics (\includegraphics)
\usepackage{graphicx}
\usepackage{color}
% for neatly defining theorems and propositions
%\usepackage{amsthm}
% making logically defined graphics
%%%\usepackage{xypic}

% there are many more packages, add them here as you need them
%\usepackage{color}
%\usepackage{graphicx}
%\usepackage{epsfig}

% define commands here
\def\R{{\mathbb R}}
\def\V{{\mathsf V}}
\def\W{{\mathsf W}}
\def\x{{\mathbf x}}
\def\y{{\mathbf y}}
\def\h{{\mathbf h}}
\def\eps{\epsilon}
\def\0{{\mathbf 0}}
\def\d{{\mathrm d}}
\def\of{\circ}
\begin{document}
\PMlinkescapeword{word}
\PMlinkescapeword{words}

	Qualitatively the {\em derivative} is a \PMlinkescapetext{measure} of the change of a
	function in a small \PMlinkescapetext{region} around a specified point.

\section*{Motivation}
	The idea behind the derivative comes from the straight line. What
	characterizes a straight line is the fact that it has constant
	``slope''. 
	\begin{figure}[h]
	\begin{center}
		\input{derivative-line.pstex_t}
		\caption{The straight line $y=mx+b$\label{line}}
	\end{center}
	\end{figure}
%%	\begin{figure}[h]
%%	\begin{center}
%%		\includegraphics{derivative-line.eps}
%%		\caption{The straight line $y=mx+b$\label{line}}
%%	\end{center}
%%	\end{figure}
	In other words, for a line given by the equation $y=mx+b$, as in Fig.
	\ref{line}, the ratio of $\Delta y$ over $\Delta x$ is always constant
	and has the value $\displaystyle \frac{\Delta y}{\Delta x} = m$.

	\begin{figure}[h]
	\begin{center}
		\input{derivative-parabola.pstex_t}
		\caption{The parabola $y=x^2$ and its tangent at $(x_0,y_0)$
			\label{parabola}}
	\end{center}
	\end{figure}
%%	\begin{figure}[h]
%%	\begin{center}
%%		\includegraphics{derivative-parabola.eps}
%%		\caption{The parabola $y=x^2$ and its tangent at $(x_0,y_0)$
%%			\label{parabola}}
%%	\end{center}
%%	\end{figure}
	For other curves we cannot define a ``slope'', like for the straight
	line, since such a quantity would not be constant. However, for
	sufficiently smooth curves, each point on a curve has a tangent line.
	For example consider the curve $y=x^2$, as in Fig. \ref{parabola}. At
	the point $(x_0,y_0)$ on the curve, we can draw a tangent of slope $m$
	given by the equation $y-y_0=m(x-x_0)$.

	Suppose we have a curve of the form $y=f(x)$, and at the point
	$(x_0,f(x_0))$ we have a tangent given by $y-y_0=m(x-x_0)$. Note that
	for values of $x$ sufficiently close to $x_0$ we can make the
	approximation $f(x)\approx m(x-x_0)+y_0$. So the slope $m$ of the
	tangent describes how much $f(x)$ changes in the vicinity of $x_0$. It
	is the slope of the tangent that will be associated with the
	derivative of the function $f(x)$.

\section*{Formal definition}
	More formally for any real function $f\colon\R\to\R$, we define the
	{\em derivative} of $f$ at the point $x$ as the following limit (if
	it exists)
	\[
		f'(x) := \lim_{h\to 0} \frac{f(x+h)-f(x)}{h}.
	\]
	This definition turns out to be \PMlinkescapetext{consistent} with
	the motivation introduced above.

	The derivatives for some elementary functions are (cf. derivative
	notation)
	\begin{enumerate}
	\item $\displaystyle \frac{d}{dx} c = 0$, ~~~ where $c$ is constant;
	\item $\displaystyle \frac{d}{dx} x^n = nx^{n-1}$;
	\item $\displaystyle \frac{d}{dx} \sin x = \cos x$;
	\item $\displaystyle \frac{d}{dx} \cos x = -\sin x$;
	\item $\displaystyle \frac{d}{dx} e^x = e^x$;
	\item $\displaystyle \frac{d}{dx} \ln x = \frac{1}{x}$.
	\end{enumerate}
	While derivatives of more complicated expressions can be
	calculated algorithmically using the following rules
%	\begin{quote}
	\begin{description}
	\item[Linearity] $\displaystyle \frac{d}{dx}\left(af(x)+bg(x)\right) = af'(x)+bg'(x)$;
	\item[Product rule] $\displaystyle \frac{d}{dx}\left(f(x)g(x)\right)
		= f'(x)g(x) + f(x)g'(x)$;
	\item[Chain rule] $\displaystyle \frac{d}{dx}g(f(x)) = g'(f(x))f'(x)$;
	\item[Quotient Rule] $\displaystyle \frac{d}{dx}\frac{f(x)}{g(x)} = 
		\frac{f'(x)g(x)-f(x)g'(x)}{g(x)^2}$.
	\end{description}
%	\end{quote}
	Note that the quotient rule, although given as much importance as the
	other rules in elementary calculus, can be derived by succesively
	applying the product rule and the chain rule to
	$\displaystyle \frac{f(x)}{g(x)}=f(x)\frac{1}{g(x)}$. Also the quotient rule does not generalize as well as the other ones.

	Since the derivative $f'(x)$ of $f(x)$ is also a function $x$,
	higher derivatives can be obtained by applying the same procedure
	to $f'(x)$ and so on.

\section*{Generalization}
\subsection*{Banach Spaces}
	Unfortunately the notion of the ``slope of the tangent'' does not
	directly generalize to more abstract situations. What we can do is
	keep in mind the facts that the tangent is a linear function and that
	it approximates the function near the point of tangency, as well as
	the formal definition above.

	Very general conditions under which we can define a derivative in
	a manner much similar to the above areas follows.
	Let $f\colon\V\to\W$, where $\V$ and $\W$ are Banach
	spaces. Let $\h\ne 0$ be an element of $\V$. We define the
	\emph{directional derivative} $(D_\h f)(\x)$ at $\x$ as the following
	limit (when it exists):
	\[
		(D_\h f)(\x) := \lim_{\eps\to0} \frac{f(\x+\eps\h)-f(\x)}{\eps},
	\]
	where $\eps$ is a scalar. Note that $f(x+\eps\h)\approx f(\x) +
	\eps(D_\h f)(\x)$, which is \PMlinkescapetext{consistent} with
	our original motivation. In certain contexts, this directional derivative is also called the \emph{G\^ateaux derivative}.

	Finally we define the {\em derivative} at
	$\x$ as the bounded linear map $(Df)(\x)\colon\V\to\W$ such that for any
	non-zero $\h\in\V$
	\[
		\lim_{\|\h\|\to0} \frac{(f(\x+\h)-f(\x))-(Df)(\x)\cdot\h}{\|\h\|}=0.
	\]
	Once again we have $f(\x+\h)\approx f(\x)+(Df)(\x)\cdot\h$. In fact,
	if the derivative $(Df)(\x)$ exists, the directional derivatives can
	be obtained as $(D_\h f)(\x) = (Df)(\x)\cdot\h$.%
		\footnote{The notation $A\cdot\h$ is used when $\h$ is a
		vector and $A$ a linear operator. This notation can be
		considered advantageous to the usual notation $A(\h)$, since
		the latter is rather bulky and the former incorporates the
		intuitive distributive properties of linear operators also
		associated with usual multiplication.} %
	However, the
	existence of $(D_\h f)$ for each non-zero $\h\in\V$ does not guarantee
	the existence of $(Df)(\x)$. This derivative is also called the
	\emph{Fr\'echet derivative}. In the more familiar case
	$f\colon\R^n\to\R^m$, the derivative $Df$ is simply the Jacobian of
	$f$.

	Under these general conditions the following properties of the
	derivative remain
	\begin{enumerate}
	\item $D\h = 0$, ~~~ where $\h$ is a constant;
	\item $D(A\cdot\x) = A$, ~~~ where $A$ is linear.
	\end{enumerate}
	\begin{description}
	\item[Linearity] $D(af(\x)+bg(\x))\cdot\h =
		a(Df)(\x)\cdot\h+b(Dg)(\x)\cdot\h$;
	\item[``Product'' rule] $D(B(f(\x),g(\x)))\cdot\h =
		B((Df)(\x)\cdot\h,g(\x)) + B(f(\x),(Dg)(\x)\cdot\h)$,
		~~~ where $B$ is bilinear;
	\item[Chain rule] $D(g(f(\x))\cdot\h =
		(Dg)(f(\x))\cdot((Df)(\x)\cdot\h)$.
	\end{description}

	Note that the derivative of $f$ can be seen as a function $Df\colon
	\V\to L(\V,\W)$ given by $Df\colon\x\mapsto(Df)(\x)$, where $L(\V,\W)$
	is the space of bounded linear maps from $\V$ to $\W$. Since $L(\V,\W)$
	can be considered a Banach space itself with the norm taken as the
	operator norm, higher derivatives can be
	obtained by applying the same procedure to $Df$ and so on.

\subsection{Partial derivatives}
A straightforward extension of the derivatives defined above is that of partial derivatives for functions
of several independent variables. Partial derivatives have numerous applications, as for example in
physics and engineering; wave equations are among such important examples of the use of partial derivatives
in physics and engineering. 

\subsection*{Manifolds}
	Let $\V$ be a Banach space (for finite dimensional manifolds $\V=\R^n$). A manifold modeled on $\V$ is a topological space that is locally homeomorphic to
	$\V$  and
	is endowed with enough structure to define derivatives. Since the
	notion of a manifold was constructed specifically to generalize the
	notion of a derivative, this seems like the end of the road for this
	entry. The following discussion is rather technical, a more
	intuitive explanation of the same concept can be found in the entry
	on related rates.

	Consider manifolds $V$ and $W$ modeled on Banach spaces $\V$ and $\W$,
	respectively. Say we have
	$y=f(x)$ for some $x\in V$ and $y\in W$, then, by definition of a
	manifold, we can find charts $(X,\x)$ and $(Y,\y)$, where $X$ and $Y$
	are neighborhoods of $x$ and $y$, respectively. These charts provide us
	with canonical isomorphisms between the Banach spaces $\V$ and $\W$,
	and the respective tangent spaces $T_x V$ and $T_y W$:
	\[
		\d\x_x \colon T_x V \to \V, \quad
		\d\y_y \colon T_y W \to \W.
	\]

	Now consider a map $f\colon V\to W$ between the manifolds. By
	composing it with the chart maps we construct the map
	\[ g_{(X,\x)}^{(Y,\y)}=\y\of f\of \x^{-1} \colon \V \to \W, \]
	defined on an appropriately \PMlinkescapetext{restricted} domain.
	Since we now have a map between Banach spaces, we can define its
	derivative at $\x(x)$ in the sense defined above, namely
	$Dg_{(X,\x)}^{(Y,\y)}(\x(x))$. If this derivative exists for every
	choice of admissible charts $(X,\x)$ and $(Y,\y)$, we can say that
	the derivative of $Df(x)$ of $f$ at $x$ is defined and given by
	\[ Df(x) = \d\y_y^{-1}\of Dg_{(X,\x)}^{(Y,\y)}(\x(x)) \of \d\x_x \]
	(it can be shown that this is well defined and independent of the
	choice of charts).

	Note that the derivative is now a map between the tangent spaces of
	the two manifolds $Df(x)\colon T_x V \to T_y W$. Because of this a
	common notation for the derivative of $f$ at $x$ is $T_x f$. Another
	alternative notation for the derivative is $f_{*,x}$ because of its
	connection to the category-theoretical pushforward.

\subsection*{Distributions}
	Derivatives can also be generalized in less ``smooth'' contexts.
	For example the derivative is one \PMlinkescapetext{type} of
	\PMlinkname{operation}{OperationsOnDistributions} that can be
	defined for distributions.

\subsection*{Standard connection of $\R^n$}
Let $\Omega$ be an open set in $\R^n$. There is an operator on vectors fields in $\Omega$ which measure how a
pair of them, $X,Y:\Omega\to \R^n$ vary, one with respect to the other: 
$$D_XY=(JY)X$$
Here $JY$ is the Jacobian of $Y$, so when we multiply, we can see that the components 
of $D_XY$ are the directional variations of the components of $Y$ in the direction $X$.

\subsection*{Additional Topic}
\begin{itemize}
\item Non-Newtonian calculus
\end{itemize}





%%%%%
%%%%%
\end{document}
