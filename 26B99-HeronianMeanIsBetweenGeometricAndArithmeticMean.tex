\documentclass[12pt]{article}
\usepackage{pmmeta}
\pmcanonicalname{HeronianMeanIsBetweenGeometricAndArithmeticMean}
\pmcreated{2013-03-22 17:49:14}
\pmmodified{2013-03-22 17:49:14}
\pmowner{pahio}{2872}
\pmmodifier{pahio}{2872}
\pmtitle{Heronian mean is between geometric and arithmetic mean}
\pmrecord{10}{40284}
\pmprivacy{1}
\pmauthor{pahio}{2872}
\pmtype{Theorem}
\pmcomment{trigger rebuild}
\pmclassification{msc}{26B99}
\pmclassification{msc}{26D07}
\pmclassification{msc}{01A20}
\pmclassification{msc}{00A05}
\pmsynonym{Heronian mean inequalities}{HeronianMeanIsBetweenGeometricAndArithmeticMean}
%\pmkeywords{mean}
\pmrelated{ArithmeticGeometricMeansInequality}
\pmrelated{ComparisonOfPythagoreanMeans}
\pmrelated{SquareOfSum}
\pmrelated{Equivalent3}
\pmrelated{HeronsPrinciple}

% this is the default PlanetMath preamble.  as your knowledge
% of TeX increases, you will probably want to edit this, but
% it should be fine as is for beginners.

% almost certainly you want these
\usepackage{amssymb}
\usepackage{amsmath}
\usepackage{amsfonts}

% used for TeXing text within eps files
%\usepackage{psfrag}
% need this for including graphics (\includegraphics)
%\usepackage{graphicx}
% for neatly defining theorems and propositions
 \usepackage{amsthm}
% making logically defined graphics
%%%\usepackage{xypic}

% there are many more packages, add them here as you need them

% define commands here

\theoremstyle{definition}
\newtheorem*{thmplain}{Theorem}

\begin{document}
\PMlinkescapeword{chains}

\textbf{Theorem.}\, For non-negative numbers $x$ and $y$, the inequalities
$$\sqrt{xy} \;\leqq\; \frac{x\!+\!\sqrt{xy}\!+\!y}{3} \;\leqq\; \frac{x\!+\!y}{2}$$
are in \PMlinkescapetext{force}, i.e. the Heronian mean is always at least equal to the geometric mean and at most equal to the arithmetic mean.\, The equality signs are true if and only if \,$x = y$.\\

{\em Proof.}\\
$1^\circ.$\; \begin{align*}
\sqrt{xy}\;\leqq\;\frac{x\!+\!\sqrt{xy}\!+\!y}{3}&\quad\Leftrightarrow\quad 3\sqrt{xy} \leqq x\!+\!\sqrt{xy}\!+\!y\\ 
& \quad \Leftrightarrow \quad 2\sqrt{xy} \leqq x\!+\!y\\ 
& \quad \Leftrightarrow \quad 4xy \leqq x^2\!+\!2xy\!+\!y^2\\
& \quad \Leftrightarrow \quad 0 \leqq x^2\!-\!2xy\!+\!y^2\\
& \quad \Leftrightarrow \quad 0 \leqq (x\!-\!y)^2
\end{align*}
$2^\circ.$\; \begin{align*}
\frac{x\!+\!\sqrt{xy}\!+\!y}{3} \leqq \frac{x\!+\!y}{2}\;\;& \quad\Leftrightarrow\quad 2x\!+\!2\sqrt{xy}\!+\!2y \leqq 3x\!+\!3y\\ 
& \quad \Leftrightarrow \quad 2\sqrt{xy} \leqq x\!+\!y\\
& \quad \Leftrightarrow \quad  4xy \leqq x^2\!+\!2xy\!+\!y^2\\
& \quad \Leftrightarrow \quad 0 \leqq (x\!-\!y)^2
\end{align*}
All \PMlinkescapetext{consecutive} inequalities of both chains are \PMlinkname{equivalent}{Equivalent3} since $x$ and $y$ are non-negative.\, As for the equalities, the chains are valid with the mere equality signs.
%%%%%
%%%%%
\end{document}
