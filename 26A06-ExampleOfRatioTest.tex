\documentclass[12pt]{article}
\usepackage{pmmeta}
\pmcanonicalname{ExampleOfRatioTest}
\pmcreated{2013-03-22 15:03:20}
\pmmodified{2013-03-22 15:03:20}
\pmowner{drini}{3}
\pmmodifier{drini}{3}
\pmtitle{example of ratio test}
\pmrecord{9}{36773}
\pmprivacy{1}
\pmauthor{drini}{3}
\pmtype{Example}
\pmcomment{trigger rebuild}
\pmclassification{msc}{26A06}
\pmclassification{msc}{40A05}

\endmetadata

\usepackage{graphicx}
%%%\usepackage{xypic} 
\usepackage{bbm}
\newcommand{\Z}{\mathbbmss{Z}}
\newcommand{\C}{\mathbbmss{C}}
\newcommand{\R}{\mathbbmss{R}}
\newcommand{\Q}{\mathbbmss{Q}}
\newcommand{\mathbb}[1]{\mathbbmss{#1}}
\newcommand{\figura}[1]{\begin{center}\includegraphics{#1}\end{center}}
\newcommand{\figuraex}[2]{\begin{center}\includegraphics[#2]{#1}\end{center}}
\newtheorem{dfn}{Definition}
\begin{document}
Consider the sequence given by $a_n=x^n$ (geometric progression) where  $|x|<1$.
Then the series
\[
\sum_{j=0}^\infty a_n 
\]
converges. \,To see this, we can use the ratio test. We need to consider the sequence  $|a_{n+1}/a_n|$. But for any \,$n \ge 0$\, we have (when \,$x \neq 0$)
\[
\left|\frac{a_{n+1}}{a_n}\right| = \left|\frac{x^{n+1}}{x^n}\right| = |x| < 1,
\]
and therefore the series converges. The ratio test and the previous argument shows that the geometric series diverges for \,$|x|>1$.
%%%%%
%%%%%
\end{document}
