\documentclass[12pt]{article}
\usepackage{pmmeta}
\pmcanonicalname{DerivationOfHalfangleFormulaeForTangent}
\pmcreated{2013-03-22 17:00:19}
\pmmodified{2013-03-22 17:00:19}
\pmowner{rspuzio}{6075}
\pmmodifier{rspuzio}{6075}
\pmtitle{derivation of half-angle formulae for tangent}
\pmrecord{9}{39287}
\pmprivacy{1}
\pmauthor{rspuzio}{6075}
\pmtype{Derivation}
\pmcomment{trigger rebuild}
\pmclassification{msc}{26A09}
\pmrelated{TangentOfHalvedAngle}

% this is the default PlanetMath preamble.  as your knowledge
% of TeX increases, you will probably want to edit this, but
% it should be fine as is for beginners.

% almost certainly you want these
\usepackage{amssymb}
\usepackage{amsmath}
\usepackage{amsfonts}

% used for TeXing text within eps files
%\usepackage{psfrag}
% need this for including graphics (\includegraphics)
%\usepackage{graphicx}
% for neatly defining theorems and propositions
%\usepackage{amsthm}
% making logically defined graphics
%%%\usepackage{xypic}

% there are many more packages, add them here as you need them

% define commands here

\begin{document}
Start with the angle duplication formula
\[
\tan (x) = {2 \tan (x/2) \over 1 - \tan^2 (x/2)}.
\]
Cross-multiply and move terms around:
\[
\tan (x) \tan^2 (x/2) + 2 \tan (x/2) = \tan (x)
\]
Divide by $\tan (x)$:
\[
\tan^2 (x/2) + {2 \tan (x/2) \over \tan x} = 1
\]
Add $1 / \tan^2 (x)$ to both sides:
\[
\tan^2 (x/2) + {2 \tan (x/2) \over \tan x} + {1 \over \tan^2 (x)} = 1 + {1 \over \tan^2 (x)}
\]
\PMlinkname{Complete the square}{CompletingTheSquare}:
\[
\left (\tan (x/2) + {1 \over \tan (x)} \right)^2 =  1 + {1 \over \tan^2 (x)}
\]
Take a square root and move a term to obtain the half-angle formula:
\[
\tan (x/2) = \sqrt{ 1 + {1 \over \tan^2 (x)} } - {1 \over \tan (x)}
\]

To derive the other forms of the formula, we start by substituting
$\sin (x) / \cos (x)$ for $\tan (x)$:
\[
\tan (x/2) = \sqrt{ 1 + {\cos^2 (x) \over \sin^2 (x)}} - 
{\cos (x) \over \sin (x)}
\]
Put the stuff inside the square root over a common denominator:
\[
\sqrt {\sin^2 (x) + \cos^2 (x) \over \sin^2 (x)} - 
{\cos (x) \over \sin (x)}
\]
Recall that $\sin^2 (x) + \cos^2 (x) = 1$.  Hence, we may get rid
of the square root:
\[
{1 \over \sin x} - {\cos (x) \over \sin (x)}
\]
Putting the terms over a common denominator, we obtain our formula:
\[
\tan (x/2) = {1 - \cos (x)  \over \sin (x)}
\]

To obtain the next formula, multiply both numerator and denominator
by $1 + \cos (x)$:
\[
{(1 - \cos (x)) (1 + \cos (x))  \over \sin (x) (1 + \cos (x))}
\]
Multiply out the numerator and simplify:
\[
{1 - \cos^2 (x) \over \sin (x) (1 + \cos (x))}
\]
Note that the numerator equals $\sin^2 (x)$:
\[
{\sin^2 (x) \over \sin (x) (1 + \cos (x))}
\]
Cancel a common factor of $\sin (x)$ to obtain the formula
\[
\tan (x/2) = {\sin (x) \over 1 + \cos (x)} .
\]

To obtain the last formula, multiply the previous two formulae:
\[
\tan^2 (x/2) = {1 - \cos (x)  \over \sin (x)}\cdot
{\sin (x) \over 1 + \cos (x)}
\]
Cancel the common factor of $\sin (x)$:
\[
\tan^2 (x/2) = {1 - \cos (x)  \over 1 + \cos (x)}
\]
Take the square root of both sides to obtain the formula
\[
\tan{\frac{x}{2}}\; = \;\pm\sqrt{1 - \cos{x}  \over 1 + \cos{x}};
\]
here the sign ($\pm$) has to be chosen according to the 
quadrant where the angle $\displaystyle\frac{x}{2}$ is.

%%%%%
%%%%%
\end{document}
