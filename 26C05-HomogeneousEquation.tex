\documentclass[12pt]{article}
\usepackage{pmmeta}
\pmcanonicalname{HomogeneousEquation}
\pmcreated{2013-03-22 15:14:41}
\pmmodified{2013-03-22 15:14:41}
\pmowner{pahio}{2872}
\pmmodifier{pahio}{2872}
\pmtitle{homogeneous equation}
\pmrecord{7}{37022}
\pmprivacy{1}
\pmauthor{pahio}{2872}
\pmtype{Topic}
\pmcomment{trigger rebuild}
\pmclassification{msc}{26C05}
\pmclassification{msc}{26B35}
\pmclassification{msc}{00A99}
%\pmkeywords{proportional}
\pmrelated{Variation}
\pmrelated{HomogeneousPolynomial}
\pmrelated{Equation}
\pmrelated{RegularDecagonInscribedInCircle}

% this is the default PlanetMath preamble.  as your knowledge
% of TeX increases, you will probably want to edit this, but
% it should be fine as is for beginners.

% almost certainly you want these
\usepackage{amssymb}
\usepackage{amsmath}
\usepackage{amsfonts}

% used for TeXing text within eps files
%\usepackage{psfrag}
% need this for including graphics (\includegraphics)
%\usepackage{graphicx}
% for neatly defining theorems and propositions
 \usepackage{amsthm}
% making logically defined graphics
%%%\usepackage{xypic}

% there are many more packages, add them here as you need them

% define commands here

\theoremstyle{definition}
\newtheorem*{thmplain}{Theorem}
\begin{document}
The {\em homogeneous equation}
   $$f(x,\,y) = 0,$$
where the left hand \PMlinkescapetext{side} is a homogeneous polynomial of degree $r$ in $x$ and $y$,\, determines the ratio $x/y$ between the indeterminates.\, One can be persuaded of this by dividing both \PMlinkescapetext{sides} of the equation by $y^r$.\, Then the left \PMlinkescapetext{side} depends only on $x/y$ (which may be denoted e.g. by $t$).

\textbf{Examples} 
\begin{itemize}
 \item The equation\, $5x+8y = 0$\, determines that\, $x/y = -\frac{8}{5}$.
 \item The equation\, $x^2-7xy+10y^2 = 0$\, determines that\, $x/y = 2$\, or\, $x/y = 5$\, (these values are obtained by first dividing both \PMlinkescapetext{sides} of the equation by $y^2$ and then solving the equation\, $(x/y)^2-7(x/y)+10 = 0$).
 \item The equation\, $2x^3-x^2y-6xy^2+3y^3 = 0$\, determines that
\, $x/y = \frac{1}{2}$\, or\, $x/y = \pm\sqrt{3}$ (first divide the equation by $y^3$ and then solve\, $2(x/y)^3-(x/y)^2-6(x/y)+3 = 0$).
\end{itemize}
%%%%%
%%%%%
\end{document}
