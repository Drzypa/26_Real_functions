\documentclass[12pt]{article}
\usepackage{pmmeta}
\pmcanonicalname{ApplicationOfCauchySchwarzInequality}
\pmcreated{2013-03-22 18:59:42}
\pmmodified{2013-03-22 18:59:42}
\pmowner{pahio}{2872}
\pmmodifier{pahio}{2872}
\pmtitle{application of Cauchy--Schwarz inequality}
\pmrecord{5}{41862}
\pmprivacy{1}
\pmauthor{pahio}{2872}
\pmtype{Application}
\pmcomment{trigger rebuild}
\pmclassification{msc}{26A42}
\pmclassification{msc}{26A06}
\pmsynonym{application of Cauchy-Schwarz inequality}{ApplicationOfCauchySchwarzInequality}

\endmetadata

% this is the default PlanetMath preamble.  as your knowledge
% of TeX increases, you will probably want to edit this, but
% it should be fine as is for beginners.

% almost certainly you want these
\usepackage{amssymb}
\usepackage{amsmath}
\usepackage{amsfonts}

% used for TeXing text within eps files
%\usepackage{psfrag}
% need this for including graphics (\includegraphics)
%\usepackage{graphicx}
% for neatly defining theorems and propositions
%\usepackage{amsthm}
% making logically defined graphics
%%%\usepackage{xypic}

% there are many more packages, add them here as you need them

% define commands here
\newcommand{\sijoitus}[2]%
{\operatornamewithlimits{\Big/}_{\!\!\!#1}^{\,#2}}
\begin{document}
In determining the perimetre of ellipse one encounters the elliptic integral
$$\int_0^{\frac{\pi}{2}}\!\!\sqrt{1-\varepsilon^2\sin^2t}\;dt,$$
where the parametre $\varepsilon$ is the eccentricity of the ellipse ($0 \leqq \varepsilon < 1$).\, A good upper bound for the integral is obtained by utilising the \PMlinkid{Cauchy--Schwarz inequality}{1628} 
$$\left|\int_a^bfg\right| \;\leqq\; \sqrt{\int_a^bf^2}\,\sqrt{\int_a^bg^2}$$
choosing in it\, $f(t) := 1$\, and\, $g(t) := \sqrt{1-\varepsilon^2\sin^2t}$.\, Then we get
\begin{align*}
0 \;<\; \int_0^{\frac{\pi}{2}}\!\!\sqrt{1-\varepsilon^2\sin^2t}\;dt
&\;\leqq\; \sqrt{\int_0^{\frac{\pi}{2}}1^2\,dt}\sqrt{\int_0^{\frac{\pi}{2}}\left(1-\varepsilon^2\sin^2t\right)\,dt}\\
&\;=\;\sqrt{\frac{\pi}{2}}\sqrt{\int_0^{\frac{\pi}{2}}\left(1-\varepsilon^2\cdot\frac{1-\cos2t}{2}\right)\,dt}\\ 
&\;=\; \frac{\pi}{2}\sqrt{1-\frac{\varepsilon^2}{2}}.
\end{align*}
Thus we have the estimation
$$\int_0^{\frac{\pi}{2}}\!\!\sqrt{1-\varepsilon^2\sin^2t}\;dt
    \;\leqq\; \frac{\pi}{2}\sqrt{1-\frac{\varepsilon^2}{2}}.$$
It is the better approximation for the perimetre of ellipse the smaller is its eccentricity, i.e. the closer the ellipse is to circle.\, The accuracy is $O(\varepsilon^4)$

%%%%%
%%%%%
\end{document}
