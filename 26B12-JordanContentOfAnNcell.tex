\documentclass[12pt]{article}
\usepackage{pmmeta}
\pmcanonicalname{JordanContentOfAnNcell}
\pmcreated{2013-03-22 13:37:37}
\pmmodified{2013-03-22 13:37:37}
\pmowner{Mathprof}{13753}
\pmmodifier{Mathprof}{13753}
\pmtitle{Jordan content of an $n$-cell}
\pmrecord{5}{34269}
\pmprivacy{1}
\pmauthor{Mathprof}{13753}
\pmtype{Definition}
\pmcomment{trigger rebuild}
\pmclassification{msc}{26B12}

\endmetadata

\usepackage{amssymb}
\usepackage{amsmath}
\usepackage{amsfonts}

\newcommand{\R}{\mathbb{R}}
\begin{document}
Let $I = [a_1, b_1]\times\cdots\times[a_n,b_n]$ be an $n$-cell in $\R^n$. Then the {\em Jordan content} (denoted $\mu(I)$) of $I$ is defined as
\[
  \mu(I) := \prod_{j=1}^n(b_j-a_j).
\]
%%%%%
%%%%%
\end{document}
