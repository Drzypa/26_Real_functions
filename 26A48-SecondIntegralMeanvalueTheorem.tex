\documentclass[12pt]{article}
\usepackage{pmmeta}
\pmcanonicalname{SecondIntegralMeanvalueTheorem}
\pmcreated{2013-03-22 18:20:21}
\pmmodified{2013-03-22 18:20:21}
\pmowner{pahio}{2872}
\pmmodifier{pahio}{2872}
\pmtitle{second integral mean-value theorem}
\pmrecord{12}{40972}
\pmprivacy{1}
\pmauthor{pahio}{2872}
\pmtype{Theorem}
\pmcomment{trigger rebuild}
\pmclassification{msc}{26A48}
\pmclassification{msc}{26A42}
\pmclassification{msc}{26A06}

% this is the default PlanetMath preamble.  as your knowledge
% of TeX increases, you will probably want to edit this, but
% it should be fine as is for beginners.

% almost certainly you want these
\usepackage{amssymb}
\usepackage{amsmath}
\usepackage{amsfonts}

% used for TeXing text within eps files
%\usepackage{psfrag}
% need this for including graphics (\includegraphics)
%\usepackage{graphicx}
% for neatly defining theorems and propositions
 \usepackage{amsthm}
% making logically defined graphics
%%%\usepackage{xypic}

% there are many more packages, add them here as you need them

% define commands here

\theoremstyle{definition}
\newtheorem*{thmplain}{Theorem}

\begin{document}
\textbf{\PMlinkescapetext{Second integral mean-value theorem.}}\, If the real functions $f$ and $g$ are continuous and $f$ monotonic on the interval \,$[a,\,b]$,\, then the equation
\begin{align}
\int_a^b\!f(x)g(x)\,dx \;=\; f(a)\!\int_a^{\,\xi}\!g(x)\,dx+f(b)\!\int_\xi^{\,b}\!g(x)\,dx
\end{align}
is true for a value $\xi$ in this interval.\\


\emph{Proof.}\, We can suppose that\, $f(a) \neq f(b)$\, since otherwise any value of $\xi$ between $a$ and $b$ would do.

Let's first prove the auxiliary result, that if a function $\varphi$ is continuous on an open interval $I$ containing\, 
$[a,\,b]$\, then
\begin{align}
\lim_{h\to0}\int_a^b\!\frac{\varphi(x\!+\!h)\!-\!\varphi(x)}{h}\,dx \;=\; \varphi(b)\!-\varphi(a).
\end{align}
In fact, when we take an antiderivative $\Phi$ of $\varphi$, then for every nonzero $h$ the function
$$x \;\mapsto\; \frac{\Phi(x\!+\!h)-\Phi(x)}{h}$$
is an antiderivative of the integrand of (2) on the interval\, $[a,\,b]$.\, Thus we have
$$\int_a^b\!\frac{\varphi(x\!+\!h)\!-\!\varphi(x)}{h}\,dx 
\;=\; \frac{\Phi(b\!+\!h)\!-\!\Phi(x)}{h}-\frac{\Phi(a\!+\!h)\!-\!\Phi(x)}{h} \;\to\; \varphi(b)\!-\!\varphi(a) 
\quad \mbox{as} \;\; h \,\to\, 0.$$

The given functions $f$ and $g$ can be extended on an open interval $I$ containing\, $[a,\,b]$\, such that they remain continuous and $f$ monotonic.\, We take an antiderivative $G$ of $g$ and a nonzero number $h$ having small absolute value.\, Then we can write the identity
\begin{align}
\int_a^b\!\frac{f(x\!+\!h)G(x\!+\!h)\!-\!f(x)G(x)}{h}\,dx \;=\; 
\int_a^b\!\frac{f(x\!+\!h) [G(x\!+\!h)\!-\!G(x)]}{h}\,dx-\int_a^b\!\frac{f(x\!+\!h)\!-\!f(x)}{h}G(x)\,dx. 
\end{align}

By (2), the left hand side of (3) may be written
\begin{align}
\int_a^b\!\frac{f(x\!+\!h)G(x\!+\!h)\!-\!f(x)G(x)}{h}\,dx \;=\; f(b)G(b)\!-\!f(a)G(a)\!+\!\varepsilon_1(h)
\end{align}
where\, $\varepsilon_1(h) \to 0$\, as\, $h \to 0$.\, Further, the function
\begin{align*}
x \;\mapsto\;
\begin{cases}
\frac{f(x+h)[G(x+h)-G(x)]}{h} \quad \mbox{for}\quad h \;\neq\; 0 \\
f(x)g(x) \qquad \mbox{for}\qquad h \;=\; 0
\end{cases}
\end{align*}
is continuous in a rectangle \;$a \le x \le b,\;\, -\delta \le h \le \delta$,\, whence we have
\begin{align}
\int_a^b\!\frac{f(x\!+\!h)[G(x\!+\!h)\!-\!G(x)]}{h}\,dx \;=\; \int_a^b\!f(x)g(x)\,dx+\varepsilon_2(h)
\end{align}
where\, $\varepsilon_2(h) \to 0$\, as\, $h \to 0$.\, Because of the monotonicity of $f$, the expression 
$\frac{f(x\!+\!h)-f(x)}{h}$ does not change its sign when\, $a \le x \le b$.\, Then the usual integral mean value theorem guarantees for every $h$ (sufficiently near 0) a number $\xi_h$ of the interval\, $[a,\,b]$\, such that
$$\int_a^b\!\frac{f(x\!+\!h)\!-\!f(x)}{h}G(x)\,dx \;=\; G(\xi_h)\!\int_a^b\!\frac{f(x\!+\!h)\!-\!f(x)}{h}\,dx,$$
and the auxiliary result (2) allows to write this as
\begin{align}
\int_a^b\!\frac{f(x\!+\!h)\!-\!f(x)}{h}G(x)\,dx \;=\; G(\xi_h)[f(b)\!-\!f(a)\!+\!\varepsilon_3(h)]
\end{align}
with\, $\varepsilon_3(h) \to 0$\, as\, $h \to 0$.\, Now the equations (4), (5) and (6) imply
\begin{align}
f(b)G(b)\!-\!f(a)G(a)+\varepsilon_1(h) \;=\; 
\int_a^b\!f(x)g(x)\,dx+\varepsilon_2(h)+G(\xi_h)[f(b)\!-\!f(a)\!+\!\varepsilon_3(h)].
\end{align}
Because\, $f(b)\!-\!f(a) \neq 0$,\, the expression $G(\xi_h)$ has a limit $L$ for\, $h \to 0$.\, By the continuity of 
$G$ there must be a number $\xi$ between $a$ and $b$ such that\, $G(\xi) = L$.\, Letting then $h$ tend to 0 we thus get the limiting equation
$$f(b)G(b)\!-\!f(a)G(a) \;=\; \int_a^b\!f(x)g(x)\,dx+G(\xi)[f(b)\!-\!f(a)],$$
which finally gives
$$\int_a^b\!f(x)g(x)\,dx \;=\; f(a)[G(\xi)\!-\!G(a)]+f(b)[G(b)\!-\!G(\xi)] 
\;=\; f(a)\!\int_a^{\,\xi}\!g(x)\,dx+f(b)\!\int_\xi^{\,b}\!g(x)\,dx$$
Q.E.D.
%%%%%
%%%%%
\end{document}
