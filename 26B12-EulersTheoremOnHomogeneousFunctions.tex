\documentclass[12pt]{article}
\usepackage{pmmeta}
\pmcanonicalname{EulersTheoremOnHomogeneousFunctions}
\pmcreated{2013-03-22 15:18:58}
\pmmodified{2013-03-22 15:18:58}
\pmowner{CWoo}{3771}
\pmmodifier{CWoo}{3771}
\pmtitle{Euler's theorem on homogeneous functions}
\pmrecord{10}{37121}
\pmprivacy{1}
\pmauthor{CWoo}{3771}
\pmtype{Theorem}
\pmcomment{trigger rebuild}
\pmclassification{msc}{26B12}
\pmclassification{msc}{26A06}
\pmclassification{msc}{15-00}
\pmdefines{Euler operator}

\endmetadata

% this is the default PlanetMath preamble.  as your knowledge
% of TeX increases, you will probably want to edit this, but
% it should be fine as is for beginners.

% almost certainly you want these
\usepackage{amssymb}
\usepackage{amsmath}
\usepackage{amsfonts}

% used for TeXing text within eps files
%\usepackage{psfrag}
% need this for including graphics (\includegraphics)
%\usepackage{graphicx}
% for neatly defining theorems and propositions
\usepackage{amsthm}
% making logically defined graphics
%%%\usepackage{xypic}

% there are many more packages, add them here as you need them

% define commands here
\def\sse{\subseteq}
\def\bigtimes{\mathop{\mbox{\Huge $\times$}}}
\def\impl{\Rightarrow}
\def\del{\partial}

\newtheorem{thm}{Theorem}
\begin{document}
\PMlinkescapeword{identity}
\PMlinkescapeword{degree}
\PMlinkescapeword{state}
\PMlinkescapeword{equivalent}
\PMlinkescapeword{side}
%
\begin{thm}[Euler] Let $f(x_1,\ldots,x_k)$ be a smooth homogeneous function of degree $n$. That is,
\begin{equation*}
  f(t x_1, \ldots, t x_k) = t^n f(x_1, \ldots, x_k).
    \label{hom-def} \tag{*}
\end{equation*}
Then the following identity holds
\[
  x_1 \frac{\del f}{\del x_1} + \cdots + x_k \frac{\del f}{\del x_k}
    = n f.
\]
\end{thm}
\begin{proof}
By homogeneity, the relation \eqref{hom-def} holds for all $t$. Taking the t-derivative of both sides, we establish that the following identity holds for all $t$:
\[
  x_1 \frac{\del f}{\del x_1}(t x_1, \ldots, t x_k) + \cdots +
  x_k \frac{\del f}{\del x_k}(t x_1, \ldots, t x_k)
    = n t^{n-1} f(x_1, \ldots, x_k).
\]
To obtain the result of the theorem, it suffices to set $t=1$ in the previous formula.
\end{proof}

Sometimes the differential operator $\displaystyle{x_1 \frac{\del}{\del x_1} + \cdots
+ x_k \frac{\del}{\del x_k}}$ is called the \emph{Euler operator}. An equivalent way to state the theorem is to say that homogeneous functions are eigenfunctions of the Euler operator, with the degree of homogeneity as the eigenvalue.
%%%%%
%%%%%
\end{document}
