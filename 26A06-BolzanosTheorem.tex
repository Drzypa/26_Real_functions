\documentclass[12pt]{article}
\usepackage{pmmeta}
\pmcanonicalname{BolzanosTheorem}
\pmcreated{2013-03-22 15:39:06}
\pmmodified{2013-03-22 15:39:06}
\pmowner{pahio}{2872}
\pmmodifier{pahio}{2872}
\pmtitle{Bolzano's theorem}
\pmrecord{5}{37584}
\pmprivacy{1}
\pmauthor{pahio}{2872}
\pmtype{Theorem}
\pmcomment{trigger rebuild}
\pmclassification{msc}{26A06}
\pmrelated{PolynomialEquationOfOddDegree}
\pmrelated{Evolute2}
\pmrelated{ExampleOfConvergingIncreasingSequence}

% this is the default PlanetMath preamble.  as your knowledge
% of TeX increases, you will probably want to edit this, but
% it should be fine as is for beginners.

% almost certainly you want these
\usepackage{amssymb}
\usepackage{amsmath}
\usepackage{amsfonts}

% used for TeXing text within eps files
%\usepackage{psfrag}
% need this for including graphics (\includegraphics)
%\usepackage{graphicx}
% for neatly defining theorems and propositions
 \usepackage{amsthm}
% making logically defined graphics
%%%\usepackage{xypic}

% there are many more packages, add them here as you need them

% define commands here

\theoremstyle{definition}
\newtheorem*{thmplain}{Theorem}
\begin{document}
{\em A continuous function can not change its \PMlinkname{sign}{SignumFunction} without going through the zero.}

This contents of Bolzano's theorem may be formulated more precisely as the
\begin{thmplain}
If a real function $f$ is continuous on a closed interval $I$ and the values of $f$ in the end points of $I$ have \PMlinkname{opposite}{Positive} signs, then there exists a zero of this function inside the interval.
\end{thmplain}

The theorem is used when using the interval halving method for getting an approximate value of a root of an equation of the form\, $f(x) = 0$.
%%%%%
%%%%%
\end{document}
