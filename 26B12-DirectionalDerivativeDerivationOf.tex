\documentclass[12pt]{article}
\usepackage{pmmeta}
\pmcanonicalname{DirectionalDerivativeDerivationOf}
\pmcreated{2013-03-22 15:25:22}
\pmmodified{2013-03-22 15:25:22}
\pmowner{apmc}{9183}
\pmmodifier{apmc}{9183}
\pmtitle{directional derivative, derivation of}
\pmrecord{7}{37267}
\pmprivacy{1}
\pmauthor{apmc}{9183}
\pmtype{Derivation}
\pmcomment{trigger rebuild}
\pmclassification{msc}{26B12}
\pmclassification{msc}{26B10}

% this is the default PlanetMath preamble.  as your knowledge
% of TeX increases, you will probably want to edit this, but
% it should be fine as is for beginners.

% almost certainly you want these
\usepackage{amssymb}
\usepackage{amsmath}
\usepackage{amsfonts}

% used for TeXing text within eps files
%\usepackage{psfrag}
% need this for including graphics (\includegraphics)
%\usepackage{graphicx}
% for neatly defining theorems and propositions
%\usepackage{amsthm}
% making logically defined graphics
%%%\usepackage{xypic}

% there are many more packages, add them here as you need them

% define commands here
\begin{document}
Let $f(x,y)$ be a function where $x=x(t)$ and $y=y(t)$.  Let $\vec{r}=a\hat{\mathbf{i}}+b\hat{\mathbf{j}}$ be the vector in the desired direction.  The line through this vector is given parametrically by:

\begin{center}$\displaystyle x=x_0+at;\quad y=y_0+bt$\end{center}

The derivative of $f$ with respect to $t$ is as follows:

\begin{center}$\displaystyle \frac{\partial f}{\partial t}=\frac{\partial f}{\partial x}\frac{dx}{dt}+\frac{\partial f}{\partial y}\frac{dy}{dt}$\end{center}

But from the equation of the line, we know that $\frac{dx}{dt}=a$ and $\frac{dy}{dt}=b$ so the derivative becomes:

\begin{center}$\displaystyle \frac{\partial f}{\partial t}=\frac{\partial f}{\partial x}a+\frac{\partial f}{\partial y}b=\nabla f\cdot\vec{r}$\end{center}
%%%%%
%%%%%
\end{document}
