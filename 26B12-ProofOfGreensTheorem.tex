\documentclass[12pt]{article}
\usepackage{pmmeta}
\pmcanonicalname{ProofOfGreensTheorem}
\pmcreated{2013-03-22 12:28:47}
\pmmodified{2013-03-22 12:28:47}
\pmowner{mathcam}{2727}
\pmmodifier{mathcam}{2727}
\pmtitle{proof of Green's theorem}
\pmrecord{11}{32690}
\pmprivacy{1}
\pmauthor{mathcam}{2727}
\pmtype{Proof}
\pmcomment{trigger rebuild}
\pmclassification{msc}{26B12}

% this is the default PlanetMath preamble.  as your knowledge
% of TeX increases, you will probably want to edit this, but
% it should be fine as is for beginners.

% almost certainly you want these
\usepackage{amssymb}
\usepackage{amsmath}
\usepackage{amsfonts}

% used for TeXing text within eps files
%\usepackage{psfrag}
% need this for including graphics (\includegraphics)
%\usepackage{graphicx}
% for neatly defining theorems and propositions
%\usepackage{amsthm}
% making logically defined graphics
%%%\usepackage{xypic} 

% there are many more packages, add them here as you need them

% define commands here
\begin{document}
Consider the region $R$ bounded by the closed curve $P$ in a simply connected space. $P$ can be given by a vector valued function $\vec{F}(x,y)=( f(x,y), g(x,y))$. 
The region $R$ can then be described by 
$$\int\!\!\!\int_R \left(\frac{\partial g}{\partial x} - \frac{\partial f}{\partial y}\right)\;dA = \int\!\!\!\int_R \frac{\partial g}{\partial x}\;dA -  \int\!\!\!\int_R \frac{\partial f}{\partial y}\;dA$$
The double integrals above can be evaluated separately. Let's look at
$$\int\!\!\!\int_R \frac{\partial g}{\partial x}\;dA = \int_a^b\int_{A(y)}^{B(y)}\frac{\partial g}{\partial x}\;dxdy$$
Evaluating the above double integral, we get
$$\int_a^b (g(A(y),y) - g(B(y),y))\;dy = \int_a^b g(A(y),y)\;dy - \int_a^b g(B(y),y)\;dy$$
According to the fundamental theorem of line integrals, the above equation is actually equivalent to the evaluation of the line integral of the function $\vec{F}_1(x,y)=( 0, g(x,y))$ over a path $P=P_1 + P_2$, where $P_1=(A(y), y)$ and $P_2=(B(y), y)$.
$$\int_a^b g(A(y), y)\;dy - \int_a^b g(B(y), y)\;dy = \int_{P_1} \vec{F_1}\cdot d\vec{t} + \int_{P_2}\vec{F_1}\cdot d\vec{t} = \oint_P \vec{F_1}\cdot d\vec{t}$$
Thus we have 
$$\int\!\!\!\int_R \frac{\partial g}{\partial x}\;dA = \oint_P \vec{F_1}\cdot d\vec{t}$$
By a similar argument, we can show that
$$\int\!\!\!\int_R \frac{\partial f}{\partial y}\;dA = -\oint_P \vec{F_2}\cdot d\vec{t}$$
where $\vec{F}_2=( f(x,y), 0)$. Putting all of the above together, we can see that
$$\int\!\!\!\int_R \left(\frac{\partial g}{\partial x} - \frac{\partial f}{\partial y}\right)\;dA = \oint_P \vec{F_1}\cdot d\vec{t} + \oint_P \vec{F_2}\cdot d\vec{t} = \oint_P (\vec{F}_1 + \vec{F}_2)\cdot d\vec{t}=\oint_P (f(x,y), g(x,y))\cdot d\vec{t}$$
which is Green's theorem.
%%%%%
%%%%%
\end{document}
