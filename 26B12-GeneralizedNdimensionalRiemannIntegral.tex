\documentclass[12pt]{article}
\usepackage{pmmeta}
\pmcanonicalname{GeneralizedNdimensionalRiemannIntegral}
\pmcreated{2013-03-22 13:37:43}
\pmmodified{2013-03-22 13:37:43}
\pmowner{vernondalhart}{2191}
\pmmodifier{vernondalhart}{2191}
\pmtitle{Generalized N-dimensional Riemann Integral}
\pmrecord{6}{34271}
\pmprivacy{1}
\pmauthor{vernondalhart}{2191}
\pmtype{Definition}
\pmcomment{trigger rebuild}
\pmclassification{msc}{26B12}

\endmetadata

% this is the default PlanetMath preamble.  as your knowledge
% of TeX increases, you will probably want to edit this, but
% it should be fine as is for beginners.

% almost certainly you want these
\usepackage{amssymb}
\usepackage{amsmath}
\usepackage{amsfonts}

% used for TeXing text within eps files
%\usepackage{psfrag}
% need this for including graphics (\includegraphics)
%\usepackage{graphicx}
% for neatly defining theorems and propositions
%\usepackage{amsthm}
% making logically defined graphics
%%%\usepackage{xypic}

% there are many more packages, add them here as you need them

% define commands here

\newcommand{\R}{\mathbb{R}}
\begin{document}
Let $I = [a_1, b_1] \times \cdots \times [a_N, b_N] \subset \R^N$ be a compact interval, and let $f:I\to\R^M$ be a function. Let $\epsilon > 0$. If there exists a $y \in \R^M$ and a partition $P_\epsilon$ of $I$ such that for each refinement $P$ of $P_\epsilon$ (and corresponding Riemann Sum $S(f, P)$),
\[
\left\|S(f,P) - y\right\| < \epsilon
\]
Then we say that $f$ is Riemann integrable over $I$, that $y$ is the Riemann integral of $f$ over $I$, and we write
\[
\int_I f := \int_I f\, d\mu := y
\]

Note also that it is possible to extend this definition to more arbitrary sets; for any bounded set $D$, one can find a compact interval $I$ such that $D \subset I$, and define a function
\[
\tilde{f}:I\to\R^M\quad x \mapsto
\begin{cases}
f(x),& x \in D \\
0, &x \notin D
\end{cases}
\]
in which case we define
\[
\int_D f := \int_I \tilde{f}
\]
%%%%%
%%%%%
\end{document}
