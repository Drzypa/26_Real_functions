\documentclass[12pt]{article}
\usepackage{pmmeta}
\pmcanonicalname{ConverseOfDarbouxsTheoremanalysisIsNotTrue}
\pmcreated{2013-03-22 17:33:51}
\pmmodified{2013-03-22 17:33:51}
\pmowner{Gorkem}{3644}
\pmmodifier{Gorkem}{3644}
\pmtitle{converse of  Darboux's theorem (analysis) is not true}
\pmrecord{6}{39973}
\pmprivacy{1}
\pmauthor{Gorkem}{3644}
\pmtype{Example}
\pmcomment{trigger rebuild}
\pmclassification{msc}{26A06}

\endmetadata

\usepackage{amssymb}
\usepackage{amsmath}
\usepackage{amsfonts}
\usepackage{mathrsfs}
\usepackage{amssymb,amsbsy}
\usepackage{graphicx,color}
\usepackage{epsfig}


% used for TeXing text within eps files
%\usepackage{psfrag}
% need this for including graphics (\includegraphics)
%\usepackage{graphicx}
% for neatly defining theorems and propositions
%\usepackage{amsthm}
% making logically defined graphics
%%%\usepackage{xypic}

% there are many more packages, add them here as you need them

% define commands here





\newtheorem{thm}{Theorem}[section]
\newtheorem{defn}[thm]{Definition}
\newtheorem{lemma}[thm]{Lemma}
\newtheorem{prop}[thm]{Proposition}
\newtheorem{rk}[thm]{Remark}
\newtheorem{crl}[thm]{Corollary}
\newtheorem{stp}{Step}

\newcommand{\disp}{\displaystyle}
\newcommand{\dintl}{\disp\int\limits}
\newcommand{\dsuml}{\disp\sum\limits}
\newcommand{\hsp}{\hspace{30pt}}
\newcommand{\ba}{\begin{array}}
\newcommand{\ea}{\end{array}}
\newcommand{\trns}{\,\widehat{}\,\,}

\newcommand{\bthm}{\begin{thm}}
\newcommand{\ethm}{\end{thm}}
\newcommand{\bstp}{\begin{stp}}
\newcommand{\estp}{\end{stp}}
\newcommand{\blemma}{\begin{lemma}}
\newcommand{\elemma}{\end{lemma}}
\newcommand{\bprop}{\begin{prop}}
\newcommand{\eprop}{\end{prop}}
\newcommand{\bpf}{\begin{pf}}
\newcommand{\epf}{\end{pf}}
\newcommand{\bdefn}{\begin{defn}}
\newcommand{\edefn}{\end{defn}}
\newcommand{\brk}{\begin{rk}}
\newcommand{\erk}{\end{rk}}
\newcommand{\bcrl}{\begin{crl}}
\newcommand{\ecrl}{\end{crl}}


\newcommand{\norm}[1]{\left\|#1\right\|}
\newcommand{\brackets}[1]{\left[#1\right]}
\newcommand{\beqn}{\begin{equation}}
\newcommand{\eeqn}{\end{equation}}
\newcommand{\supnorm}[1]{\norm{#1}_\infty}
\newcommand{\normt}[1]{\norm{#1}_2}
\newcommand{\ip}[2]{\left\langle#1 , #2\right\rangle}
\newcommand{\supp}{\operatorname{supp}}
\newcommand{\calg}[1]{\mathcal{#1}}


\newcommand{\sinc}{\operatorname{sinc}}
\newcommand{\qed}{$\ \ \ \ \ \Box$}
\newcommand{\qedin}{\ \ \ \ \ \Box}
\newcommand{\Tr}{\operatorname{\Tr}}

\newenvironment{pf}{\begin{trivlist}\item[\hskip%
\labelsep{\bf Proof.}]}
{\rm\end{trivlist}}

\begin{document}
Darboux' theorem says that, if $f\colon \mathbb{R} \rightarrow \mathbb{R}$ has an antiderivative, than $f$ has to satisfy the \emph{intermediate value property}, namely, for any $a<b$, for any number $C$ with $f(a)<C<f(b)$ or $f(b)<C<f(a)$, there exists a $c \in (a,b)$ such that $f(c) = C$.  With this theorem, we understand that if $f$ does not satisfy the intermediate value property, then no function $F$ satisfies $F' = f$ on $\mathbb{R}$.

Now, we will give an example to show that the converse is not true, i.e., a function that satisfies the intermediate value property might still have no antiderivative.

Let
$$
f(x) = \left\{
    \begin{array}{ccr}
          \disp \frac{1}{x}\cos(\ln x)& \mbox{if} & x>0  \\
        0 & \mbox{if} & x\leq 0
    \end{array}
  \right.  .
$$

First let us see that $f$ satisfies the intermediate value property. Let $a<b$.  If $0<a$ or $b\leq 0$, the property is satisfied, since $f$ is continuous on $(-\infty,0]$ and $(0,\infty)$. If $a\leq 0<b$, we have $f(a) = 0$ and $f(b) = (1/b)\cos(\ln b)$.  Let $C$ be between $f(a)$ and $(b)$.  Let $a_0 = \exp(-2\pi k_0 +\pi)$ for some $k_0$ large enough such that $a_0 < b$.  Then $f(a_0)=0 = f(a)$, and since $f$ is continuous on $(a_0,b)$, we must have a $c \in (a_0,b)$ with $f(c) = C$.

Assume, for a contradiction that there exists a differentiable function $F$ such that $F'(x) = f(x)$ on $\mathbb{R}$.  Then consider the function $G(x) = \sin(\ln x)$ which is defined on $(0,\infty)$.
We have $G'(x) = f(x)$ on $(0,\infty)$, and since it is a an open connected set, we must have $F(x) = G(x) + c$ on $(0,\infty)$ for some $c\in\mathbb R$. But then, we have
\begin{align*}
\limsup_{x\rightarrow 0^+} F(x) &= \limsup _{x\rightarrow 0^+} G(x) + c  = 1+c\\
\end{align*}
and
\begin{align*}
\liminf_{x\rightarrow 0^+} F(x) &= \liminf _{x\rightarrow 0^+} G(x) + c  = -1+c\\
\end{align*}
which contradicts the differentiability of $F$ at $0$.
%%%%%
%%%%%
\end{document}
