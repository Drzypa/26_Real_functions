\documentclass[12pt]{article}
\usepackage{pmmeta}
\pmcanonicalname{Semicontinuous}
\pmcreated{2013-03-22 12:45:41}
\pmmodified{2013-03-22 12:45:41}
\pmowner{drini}{3}
\pmmodifier{drini}{3}
\pmtitle{semi-continuous}
\pmrecord{6}{33069}
\pmprivacy{1}
\pmauthor{drini}{3}
\pmtype{Definition}
\pmcomment{trigger rebuild}
\pmclassification{msc}{26A15}
\pmclassification{msc}{54-XX}
\pmsynonym{semicontinuous}{Semicontinuous}

\endmetadata

\usepackage{graphicx}
%%%\usepackage{xypic} 
\usepackage{bbm}
\newcommand{\Z}{\mathbbmss{Z}}
\newcommand{\C}{\mathbbmss{C}}
\newcommand{\R}{\mathbbmss{R}}
\newcommand{\Q}{\mathbbmss{Q}}
\newcommand{\mathbb}[1]{\mathbbmss{#1}}
\newcommand{\figura}[1]{\begin{center}\includegraphics{#1}\end{center}}
\newcommand{\figuraex}[2]{\begin{center}\includegraphics[#2]{#1}\end{center}}
\begin{document}
A real function $f: A \rightarrow \R$, where $A\subseteq \R$ is said to be \emph{lower semi-continuous} in $x_0$ if
\[ \forall \varepsilon > 0\ \exists \delta > 0\ \forall x \in A\ |x-x_0| < \delta \Rightarrow f(x) > f(x_0) - \varepsilon, \]

and $f$ is said to be \emph{upper semi-continuous} if
\[ \forall \varepsilon > 0\ \exists \delta > 0\ \forall x \in A\ |x-x_0| < \delta \Rightarrow f(x) < f(x_0) + \varepsilon. \]

\paragraph{Remark}
A real function is continuous in $x_0$ if and only if it is both upper and lower semicontinuous in $x_0$.

We can generalize the definition to arbitrary topological spaces as follows.

Let $A$ be a topological space.
$f:A\to\R$ is lower semicontinuous at $x_0$ if, for each $\varepsilon>0$ there is a neighborhood $U$ of $x_0$ such that $x\in U$ implies $f(x)>f(x_0)-\varepsilon$.

\paragraph{Theorem}
Let $f: [a,b] \rightarrow \mathbb{R}$ be a lower (upper) semi-continuous function. Then $f$ has a minimum (maximum) in $[a,b]$.
%%%%%
%%%%%
\end{document}
