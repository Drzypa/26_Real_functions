\documentclass[12pt]{article}
\usepackage{pmmeta}
\pmcanonicalname{ProofOfSchursInequality}
\pmcreated{2013-03-22 15:35:25}
\pmmodified{2013-03-22 15:35:25}
\pmowner{Andrea Ambrosio}{7332}
\pmmodifier{Andrea Ambrosio}{7332}
\pmtitle{proof of Schur's inequality}
\pmrecord{7}{37501}
\pmprivacy{1}
\pmauthor{Andrea Ambrosio}{7332}
\pmtype{Proof}
\pmcomment{trigger rebuild}
\pmclassification{msc}{26D15}
\pmclassification{msc}{15A42}

% this is the default PlanetMath preamble.  as your knowledge
% of TeX increases, you will probably want to edit this, but
% it should be fine as is for beginners.

% almost certainly you want these
\usepackage{amssymb}
\usepackage{amsmath}
\usepackage{amsfonts}

% used for TeXing text within eps files
%\usepackage{psfrag}
% need this for including graphics (\includegraphics)
%\usepackage{graphicx}
% for neatly defining theorems and propositions
%\usepackage{amsthm}
% making logically defined graphics
%%%\usepackage{xypic}

% there are many more packages, add them here as you need them

% define commands here
\begin{document}
By Schur's theorem, a unitary matrix $U$ and an upper triangular matrix $T$ exist such that $A=UTU^H$, $T$ being diagonal if and only if $A$ is normal.
Then $A^HA=UT^HU^HUTU^H=UT^HTU^H$, which means $A^HA$ and $T^HT$ are similar; so they have the same trace. We have:

$\|A\|_F^2=\operatorname{Tr}(A^HA)=\operatorname{Tr}(T^HT)=\sum_{i=1}^n\left|\lambda_i\right|^2+\sum_{i<j}\left|t_{ij}\right|^2=$

$=\operatorname{Tr}(D^HD)+\sum_{i<j}\left|t_{ij}\right|^2\geq\operatorname{Tr}(D^HD)=\|D\|_F^2.$

If and only if $A$ is normal, $T=D$ and therefore equality holds.$\square$
%%%%%
%%%%%
\end{document}
