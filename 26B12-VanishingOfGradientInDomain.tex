\documentclass[12pt]{article}
\usepackage{pmmeta}
\pmcanonicalname{VanishingOfGradientInDomain}
\pmcreated{2013-03-22 19:11:58}
\pmmodified{2013-03-22 19:11:58}
\pmowner{pahio}{2872}
\pmmodifier{pahio}{2872}
\pmtitle{vanishing of gradient in domain}
\pmrecord{8}{42113}
\pmprivacy{1}
\pmauthor{pahio}{2872}
\pmtype{Theorem}
\pmcomment{trigger rebuild}
\pmclassification{msc}{26B12}
\pmsynonym{partial derivatives vanish}{VanishingOfGradientInDomain}
%\pmkeywords{domain}
%\pmkeywords{open}
%\pmkeywords{connected}
\pmrelated{FundamentalTheoremOfIntegralCalculus}
\pmrelated{ExtremumPointsOfFunctionOfSeveralVariables}

% this is the default PlanetMath preamble.  as your knowledge
% of TeX increases, you will probably want to edit this, but
% it should be fine as is for beginners.

% almost certainly you want these
\usepackage{amssymb}
\usepackage{amsmath}
\usepackage{amsfonts}

% used for TeXing text within eps files
%\usepackage{psfrag}
% need this for including graphics (\includegraphics)
%\usepackage{graphicx}
% for neatly defining theorems and propositions
 \usepackage{amsthm}
% making logically defined graphics
%%%\usepackage{xypic}

% there are many more packages, add them here as you need them

% define commands here

\theoremstyle{definition}
\newtheorem*{thmplain}{Theorem}

\begin{document}
\PMlinkescapeword{domain}

\textbf{Theorem.}\, If the function $f$ is defined in a \PMlinkname{domain}{Domain2} $D$ of $\mathbb{R}^n$ and all the partial derivatives of a $f$ vanish identically in $D$, i.e.
$$\nabla{f} \;\equiv\; \vec{0} \quad \mbox{in}\; D,$$
then the function has a constant value in the whole domain.\\

\emph{Proof.}\, For the sake of simpler notations, think that\, $n = 3$; thus we have
\begin{align}
f_x'(x,\,y,\,z) \;=\; f_y'(x,\,y,\,z) \;=\; f_z'(x,\,y,\,z) \;=\; 0 \quad \mbox{for all}\;\;(x,\,y,\,z) \in D.
\end{align}
Make the antithesis that there are the points \,$P_0 = (x_0,\,y_0,\,z_0)$\, and\, $P_1 = (x_1,\,y_1,\,z_1)$\, of $D$ such that\, 
$f(x_0,\,y_0,\,z_0) \neq f(x_1,\,y_1,\,z_1)$.\,
Since $D$ is connected, one can form the broken line $P_0Q_1Q_2\ldots Q_kP_1$ contained in $D$.\, When one now goes along this broken line from $P_0$ to $P_1$, one mets the first corner where the value of $f$ does not equal 
$f(x_0,\,y_0,\,z_0)$.\, Thus $D$ contains a line segment, the end points of which give unequal values to $f$.\, When necessary, we change the notations such that this line segment is $P_0P_1$.\, Now, $f_x',\,f_y',\,f_z'$ are continuous in $D$ because they vanish.\, The  mean-value theorem for several variables guarantees an interior point \,$(a,\,b,\,c)$\, of the segment such that
$$0 \;\neq\; f(x_1,\,y_1,\,z_1)-f(x_0,\,y_0,\,z_0) 
\;=\; f_x'(a,\,b,\,c)(x_1\!-\!x_0)+f_y'(a,\,b,\,c)(y_1\!-\!y_0)+f_z'(a,\,b,\,c)(z_1\!-\!z_0).$$
But by (1), the last sum must vanish.\, This contradictory result shows that the antithesis is wrong, which settles the proof.

%%%%%
%%%%%
\end{document}
