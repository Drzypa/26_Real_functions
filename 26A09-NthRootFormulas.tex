\documentclass[12pt]{article}
\usepackage{pmmeta}
\pmcanonicalname{NthRootFormulas}
\pmcreated{2014-10-25 17:59:04}
\pmmodified{2014-10-25 17:59:04}
\pmowner{pahio}{2872}
\pmmodifier{pahio}{2872}
\pmtitle{nth root formulas}
\pmrecord{12}{42031}
\pmprivacy{1}
\pmauthor{pahio}{2872}
\pmtype{Theorem}
\pmcomment{trigger rebuild}
\pmclassification{msc}{26A09}
\pmsynonym{root formulas}{NthRootFormulas}
\pmsynonym{root formulae}{NthRootFormulas}
%\pmkeywords{root}
%\pmkeywords{nth root}
\pmrelated{FractionPower}

% this is the default PlanetMath preamble.  as your knowledge
% of TeX increases, you will probably want to edit this, but
% it should be fine as is for beginners.

% almost certainly you want these
\usepackage{amssymb}
\usepackage{amsmath}
\usepackage{amsfonts}

% used for TeXing text within eps files
%\usepackage{psfrag}
% need this for including graphics (\includegraphics)
%\usepackage{graphicx}
% for neatly defining theorems and propositions
 \usepackage{amsthm}
% making logically defined graphics
%%%\usepackage{xypic}

% there are many more packages, add them here as you need them

% define commands here

\theoremstyle{definition}
\newtheorem*{thmplain}{Theorem}

\begin{document}
\PMlinkescapeword{formulas}

In the following formulas, $a$ is a nonnegative real number 
and other letters positive integers.\, For other formulas, see 
the \PMlinkname{parent entry}{NthRoot}.

\begin{enumerate}
\item $\sqrt[n]{0} \;=\; 0$, \qquad $\sqrt[n]{1} \;=\; 1$
\item $\sqrt[1]{a} \;=\; a$
\item $\sqrt[m]{\sqrt[n]{a}} \;=\; \sqrt[mn]{a} \;=\; \sqrt[n]{\sqrt[m]{a}}$
\item $\sqrt[nk]{a^{mk}} \;=\; \sqrt[n]{a^m}$
\item $\sqrt[n]{a^m} \;=\; (\sqrt[n]{a})^m$
\item $\sqrt[m]{a}\cdot\sqrt[n]{a} \;=\; \sqrt[mn]{a^{m+n}}$
\end{enumerate}

\emph{Proof.}\, For proving, one uses the definition of 
\PMlinkid{$n$th root}{754} and the \PMlinkname{power}{GeneralAssociativity} laws.

$1^\circ$.\; $0^n = 0, \quad 1^n = 1$

$2^\circ$.\; $a^1 = a$

$3^\circ$.\; $(\sqrt[m]{\sqrt[n]{a}})^{mn} \;=\; ((\sqrt[m]{\sqrt[n]{a}})^m)^n \;=\; (\sqrt[n]{a})^n \;=\; a$

$4^\circ$.\; $(\sqrt[n]{a^m})^{nk} \;=\; ((\sqrt[n]{a^m})^n)^k \;=\; (a^m)^k \;=\; a^{mk}$

$5^\circ$.\; $((\sqrt[n]{a})^m)^n \;=\; ((\sqrt[n]{a})^n)^m \;=\; a^m$

$6^\circ$.\; $(\sqrt[m]{a}\cdot\sqrt[n]{a})^{mn} \;=\; (\sqrt[m]{a})^{mn}(\sqrt[n]{a})^{mn} \;=\;
((\sqrt[m]{a})^m)^n((\sqrt[n]{a})^n)^m \;=\; a^na^m \;=\; a^{m+n}$
%%%%%
%%%%%
\end{document}
