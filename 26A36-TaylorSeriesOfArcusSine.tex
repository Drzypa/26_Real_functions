\documentclass[12pt]{article}
\usepackage{pmmeta}
\pmcanonicalname{TaylorSeriesOfArcusSine}
\pmcreated{2013-03-22 14:51:18}
\pmmodified{2013-03-22 14:51:18}
\pmowner{pahio}{2872}
\pmmodifier{pahio}{2872}
\pmtitle{Taylor series of arcus sine}
\pmrecord{12}{36527}
\pmprivacy{1}
\pmauthor{pahio}{2872}
\pmtype{Example}
\pmcomment{trigger rebuild}
\pmclassification{msc}{26A36}
\pmclassification{msc}{26A09}
\pmclassification{msc}{11B65}
\pmclassification{msc}{05A10}
\pmrelated{ExamplesOnHowToFindTaylorSeriesFromOtherKnownSeries}
\pmrelated{TaylorSeriesOfArcusTangent}
\pmrelated{CyclometricFunctions}
\pmrelated{LogarithmSeries}

\endmetadata

% this is the default PlanetMath preamble.  as your knowledge
% of TeX increases, you will probably want to edit this, but
% it should be fine as is for beginners.

% almost certainly you want these
\usepackage{amssymb}
\usepackage{amsmath}
\usepackage{amsfonts}

% used for TeXing text within eps files
%\usepackage{psfrag}
% need this for including graphics (\includegraphics)
%\usepackage{graphicx}
% for neatly defining theorems and propositions
%\usepackage{amsthm}
% making logically defined graphics
%%%\usepackage{xypic}

% there are many more packages, add them here as you need them

% define commands here
\begin{document}
We give an example of obtaining the Taylor series \PMlinkescapetext{expansion} of an elementary function by integrating the Taylor series of its derivative.

For\, $-1 < x < 1$\, we have the derivative of the principal \PMlinkescapetext{branch} of the \PMlinkname{arcus sine}{CyclometricFunctions} function:
$$\frac{d\,\arcsin{x}}{dx} = \frac{1}{\sqrt{1\!-\!x^2}} = 
(1\!-\!x^2)^{-\frac{1}{2}}.$$

Using the generalized binomial coefficients ${-\frac{1}{2} \choose r}$ we thus can form the Taylor series for it as \PMlinkname{Newton's binomial series}{BinomialFormula}: 
$$(1\!-\!x^2)^{-\frac{1}{2}} = \sum_{r = 0}^\infty{-\frac{1}{2} \choose r}(-x^2)^r = 
1\!+\!{-\frac{1}{2}\choose 1}(-x^2)\!+\!{-\frac{1}{2}\choose 2}(-x^2)^2\!+\!
{-\frac{1}{2}\choose 3}(-x^2)^3\!+\!\cdots =$$
$$ =1\!-\!\frac{-\frac{1}{2}}{1!}x^2\!+
\!\frac{-\frac{1}{2}(-\frac{1}{2}\!-\!1)}{2!}x^4\!
-\!\frac{-\frac{1}{2}(-\frac{1}{2}\!-\!1)(-\frac{1}{2}\!-\!2)}{3!}x^6\!+-\cdots =$$
$$ = 1\!+\!\frac{1}{2}x^2\!+\!\frac{1\cdot 3}{2\cdot 4}x^4\!+\!
\frac{1\cdot 3\cdot 5}{2\cdot 4\cdot 6}x^6\!+\!\cdots
\quad\quad\quad\quad \mathrm{for}\,\, -1 < x < 1$$

Because\, $\arcsin{0} = 0$\, for the principal \PMlinkname{branch}{GeneralPower} of the function, we get, by \PMlinkname{integrating the series termwise}{SumFunctionOfSeries}, the \PMlinkescapetext{expansion}
$$\arcsin{x} = \int_0^x\frac{dx}{\sqrt{1\!-\!x^2}} = 
  x\!+\!\frac{1}{2}\!\cdot\!\frac{x^3}{3}\!+
\!\frac{1\!\cdot\!3}{2\!\cdot\!4}\!\cdot\!\frac{x^5}{5}\!+\!
\frac{1\!\cdot\!3\cdot\!5}{2\!\cdot\!4\cdot\!6}\!\cdot\!\frac{x^7}{7}\!+\!\cdots,$$
the validity of which is true for\, $|x| < 1$.\, It can be proved, in addition, that it is true also when\, $x = \pm 1$.
%%%%%
%%%%%
\end{document}
