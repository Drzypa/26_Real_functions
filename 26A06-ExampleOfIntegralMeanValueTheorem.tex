\documentclass[12pt]{article}
\usepackage{pmmeta}
\pmcanonicalname{ExampleOfIntegralMeanValueTheorem}
\pmcreated{2013-03-22 18:20:24}
\pmmodified{2013-03-22 18:20:24}
\pmowner{me_and}{17092}
\pmmodifier{me_and}{17092}
\pmtitle{example of integral mean value theorem}
\pmrecord{4}{40973}
\pmprivacy{1}
\pmauthor{me_and}{17092}
\pmtype{Example}
\pmcomment{trigger rebuild}
\pmclassification{msc}{26A06}

\endmetadata

%\usepackage{amssymb}
\usepackage{amsmath} %Needed for align & align* and to render proofs properly
%\usepackage{amsfonts}
\usepackage{amsthm}

%Named sets
%\newcommand{\R}{\mathbb{R}} %Real numbers (amssymb or amsfonts)
%\newcommand{\C}{\mathbb{C}} %Complex numbers (amssymb or amsfonts)

%Functions
\newcommand{\modulus}[1]{\left|{#1}\right|} %|z|
\newcommand{\integral}[4]{\int_{#1}^{#2}\!{#3}\,\mathrm{d}{#4}}

%Numbers
%\newcommand{\I}{\mathrm{i}} %sqrt{-1}
%\newcommand{\e}{\mathrm{e}} %exponential

%Letters
%\newcommand{\ve}{\varepsilon} %nice epsilon
\begin{document}
\theoremstyle{definition}
\newtheorem*{ex}{Example}
\begin{ex}
If $f$ is a continuous real function on an interval $[a,b]$, then there exists a $\zeta\in(a,b)$ such that
\[
  \integral{a}{b}{f(x)}{x}=f(\zeta)(b-a)
.\]
\end{ex}
\begin{proof}
Let $g(x)\equiv 1$. Then by the Integral Mean Value Theorem, there exists $\zeta\in(a,b)$ such that
\begin{align*}
  \integral{a}{b}{f(x)}{x} &= \integral{a}{b}{f(x)g(x)}{x} \\
                           &= f(\zeta)\integral{a}{b}{g(x)}{x} \\
                           &= f(\zeta)\integral{a}{b}{1}{x} \\
                           &= f(\zeta)(b-a)
\end{align*}
as required.
\end{proof}
%%%%%
%%%%%
\end{document}
