\documentclass[12pt]{article}
\usepackage{pmmeta}
\pmcanonicalname{HarmonicNumber}
\pmcreated{2013-03-22 13:01:28}
\pmmodified{2013-03-22 13:01:28}
\pmowner{mathcam}{2727}
\pmmodifier{mathcam}{2727}
\pmtitle{harmonic number}
\pmrecord{10}{33421}
\pmprivacy{1}
\pmauthor{mathcam}{2727}
\pmtype{Definition}
\pmcomment{trigger rebuild}
\pmclassification{msc}{26A06}
\pmclassification{msc}{40A05}
\pmrelated{Series}
\pmrelated{AbsoluteConvergence}
\pmrelated{HarmonicSeries}
\pmrelated{PrimeHarmonicSeries}
\pmrelated{WolstenholmesTheorem}
\pmdefines{harmonic number of order}

\endmetadata

\usepackage{amssymb}
\usepackage{amsmath}
\usepackage{amsfonts}

%\usepackage{psfrag}
%\usepackage{graphicx}
%%%\usepackage{xypic}
\begin{document}
The \emph{harmonic number of order $n$ of $\theta$} is defined as

$$ H_{\theta}(n) = \sum_{i=1}^n \frac{1}{i^{\theta}} $$

Note that $n$ may be equal to $\infty$, provided $\theta > 1$.  

If $\theta \le 1$, while $n=\infty$, the harmonic series does not converge and hence the harmonic number does not exist.

If $\theta = 1$, we may just write $H_{\theta}(n)$ as $H_n$ (this is a common notation).

\textbf{\PMlinkescapetext{Properties}}

\begin{itemize}
\item If $\Re(\theta) > 1$ and $n=\infty$ then the sum is the Riemann zeta function.
\item If $\theta=1$, then we get what is known simply as``the harmonic number'', and it has many important properties. For example, it has asymptotic expansion $H_n=\ln n+\gamma+\frac{1}{2m}+\dotsc$ where $\gamma$ is Euler's constant.
\item It is possible\footnote{See ``The Art of computer programming'' vol. 2 by D. Knuth} to define harmonic numbers for non-integral $n$. This is done by means of the series  $H_n(z)=\sum_{n\geq 1}(n^{-z}-(n+x)^{-z})$.
\end{itemize}
%%%%%
%%%%%
\end{document}
