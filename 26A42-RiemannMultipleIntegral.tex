\documentclass[12pt]{article}
\usepackage{pmmeta}
\pmcanonicalname{RiemannMultipleIntegral}
\pmcreated{2013-03-22 15:03:34}
\pmmodified{2013-03-22 15:03:34}
\pmowner{paolini}{1187}
\pmmodifier{paolini}{1187}
\pmtitle{Riemann multiple integral}
\pmrecord{14}{36778}
\pmprivacy{1}
\pmauthor{paolini}{1187}
\pmtype{Definition}
\pmcomment{trigger rebuild}
\pmclassification{msc}{26A42}
\pmrelated{Polyrectangle}
\pmrelated{RiemannIntegral}
\pmrelated{Integral2}
\pmrelated{AreaOfPlaneRegion}
\pmrelated{DevelopableSurface}
\pmrelated{VolumeAsIntegral}
\pmrelated{AreaOfPolygon}
\pmrelated{MoscowMathematicalPapyrus}
\pmrelated{IntegralOverPlaneRegion}
\pmdefines{Riemann integrable}
\pmdefines{Peano Jordan}
\pmdefines{measurable}
\pmdefines{area}
\pmdefines{volume}
\pmdefines{Jordan content}

% this is the default PlanetMath preamble.  as your knowledge
% of TeX increases, you will probably want to edit this, but
% it should be fine as is for beginners.

% almost certainly you want these
\usepackage{amssymb}
\usepackage{amsmath}
\usepackage{amsfonts}

% used for TeXing text within eps files
%\usepackage{psfrag}
% need this for including graphics (\includegraphics)
%\usepackage{graphicx}
% for neatly defining theorems and propositions
\usepackage{amsthm}
% making logically defined graphics
%%%\usepackage{xypic}

% there are many more packages, add them here as you need them

% define commands here
\newcommand{\R}{\mathbb R}
\newtheorem{theorem}{Theorem}
\newtheorem{definition}{Definition}
\theoremstyle{remark}
\newtheorem{example}{Example}
\begin{document}
We are going to extend the concept of Riemann integral to functions of several variables.

Let $f\colon\mathbb R^n \to\mathbb R$ be a bounded function with compact support. 
Recalling the definitions of polyrectangle and the definitions of upper and lower Riemann sums on polyrectangles,
we define
\[
  S^*(f) := \inf\{ S^*(f,P) \colon \text{$P$ is a polyrectangle, $f(x)=0$ for every $x\in\mathbb R^n\setminus \cup P$}\},
\]
\[
  S_*(f) := \sup\{ S_*(f,P) \colon \text{$P$ is a polyrectangle, $f(x)=0$ for every $x\in\mathbb R^n\setminus \cup P$}\}.
\]
If $S^*(f)=S_*(f)$ we say that $f$ is \emph{Riemann-integrable} on $\mathbb R^n$ and we define the Riemann integral of $f$:
\[
  \int f(x)\, dx := S^*(f) = S_*(f).
\]

Clearly one has $S^*(f,P)\ge S_*(f,P)$. Also one has $S^*(f,P)\ge S_*(f,P')$ when $P$ and $P'$ are any two polyrectangles containing the support of $f$. In fact one can always find a common refinement $P''$ of both $P$ and $P'$ so that $S^*(f,P)\ge S^*(f,P'')\ge S_*(f,P'')\ge S_*(f,P')$. So, to prove that a function is Riemann-integrable it is enough to prove that for every $\epsilon>0$ there exists a polyrectangle $P$ such that $S^*(f,P)-S_*(f,P)<\epsilon$.

Next we are going to define the integral on more general domains. As a byproduct we also define the measure of sets in $\mathbb R^n$.

Let $D\subset \mathbb R^n$ be a bounded set. We say that $D$ is \emph{Riemann measurable} if
the characteristic function 
\[
\chi_D(x):=\begin{cases} 1 &\text{if $x\in D$}\\
0 &\text{otherwise}\end{cases}
\]
is Riemann measurable on $\R^n$ (as defined above). Moreover we define the \emph{Peano-Jordan measure} of $D$ as
\[
  \mathbf{meas}(D) := \int \chi_D(x)\, dx.
\]
When $n=3$ the Peano Jordan measure of $D$ is called the \emph{volume} of $D$, 
and when $n=2$ the Peano Jordan measure of $D$ is called the \emph{area} of $D$.

Let now $D\subset \mathbb R^n$ be a Riemann measurable set and let $f\colon D\to \mathbb R$ be a bounded function. We say that $f$ is \emph{Riemann measurable} if the function $\bar f\colon\mathbb R^n\to\mathbb R$
\[
  \bar f(x) :=\begin{cases} f(x)&\text{if $x\in D$}\\
0&\text{otherwise}
\end{cases}
\]
is Riemann integrable as defined before. In this case we denote with
\[
   \int_D f(x)\, dx := \int \bar f(x)\, dx
\]
the \emph{Riemann integral} of $f$ on $D$.
%%%%%
%%%%%
\end{document}
