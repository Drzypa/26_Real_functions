\documentclass[12pt]{article}
\usepackage{pmmeta}
\pmcanonicalname{AbsolutelyContinuousOn01VersusAbsolutelyContinuousOnvarepsilon1ForEveryvarepsilon0}
\pmcreated{2013-03-22 16:12:19}
\pmmodified{2013-03-22 16:12:19}
\pmowner{Wkbj79}{1863}
\pmmodifier{Wkbj79}{1863}
\pmtitle{absolutely continuous on $[0,1]$ versus absolutely continuous on $[\varepsilon, 1]$ for every $\varepsilon >0$}
\pmrecord{11}{38300}
\pmprivacy{1}
\pmauthor{Wkbj79}{1863}
\pmtype{Example}
\pmcomment{trigger rebuild}
\pmclassification{msc}{26A46}
\pmclassification{msc}{26B30}

\endmetadata

\usepackage{amssymb}
\usepackage{amsmath}
\usepackage{amsfonts}

\usepackage{psfrag}
\usepackage{graphicx}
\usepackage{amsthm}
%%\usepackage{xypic}

\newtheorem*{lem*}{Lemma}

\begin{document}
\PMlinkescapeword{absolutely continuous}

\begin{lem*}
Define $f \colon \mathbb{R} \to \mathbb{R}$ by 

$$f(x)= \begin{cases}
0 & \text{ if } x=0 \\
\displaystyle x\sin\left( \frac{1}{x} \right) & \text{ if } x \neq 0. \end{cases}$$

Then $f$ is \PMlinkname{absolutely continuous}{AbsolutelyContinuousFunction2} on $[\varepsilon, 1]$ for every $\varepsilon >0$ but is not absolutely continuous on $[0,1]$.
\end{lem*}

\begin{proof}
Note that $f$ is continuous on $[0,1]$ and differentiable on $(0,1]$ with $\displaystyle f'(x)=\sin \left( \frac{1}{x} \right)-\frac{1}{x}\cos \left( \frac{1}{x} \right)$.

Let $\varepsilon >0$.  Then for all $x \in [\varepsilon ,1]$:

\begin{center}
$\begin{array}{ll}
|f'(x)| & \displaystyle =\left| \sin \left( \frac{1}{x} \right)-\frac{1}{x}\cos \left( \frac{1}{x} \right) \right| \\
\\
& \displaystyle \le \left| \sin \left( \frac{1}{x} \right) \right|+\left| \frac{1}{x} \right| \cdot \left| \cos \left( \frac{1}{x} \right) \right| \\
\\
& \displaystyle \le 1+\frac{1}{\varepsilon} \cdot 1 \\
\\
& \displaystyle =1+\frac{1}{\varepsilon} \end{array}$
\end{center}

Since $f$ is continuous on $[\varepsilon, 1]$ and differentiable on $(\varepsilon, 1)$, the \PMlinkname{mean value theorem}{MeanValueTheorem} can be applied to $f$.  Thus, for every $x_1, x_2 \in (\varepsilon, 1)$ with $x_1 \neq x_2$, $\displaystyle \left| \frac{f(x_2)-f(x_1)}{x_2-x_1} \right| \le 1+\frac{1}{\varepsilon}$.  This yields $\displaystyle |f(x_2)-f(x_1)| \le \left( 1+\frac{1}{\varepsilon} \right) |x_2-x_1|$, which also holds when $x_1=x_2$.  Thus, $f$ is Lipschitz on $(\varepsilon, 1)$.  It follows that $f$ is absolutely continuous on $[\varepsilon, 1]$.

On the other hand, it can be verified that $f$ is not of bounded variation on $[0,1]$ and thus cannot be absolutely continuous on $[0,1]$.
\end{proof}
%%%%%
%%%%%
\end{document}
