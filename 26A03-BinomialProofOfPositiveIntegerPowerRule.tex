\documentclass[12pt]{article}
\usepackage{pmmeta}
\pmcanonicalname{BinomialProofOfPositiveIntegerPowerRule}
\pmcreated{2013-03-22 12:29:43}
\pmmodified{2013-03-22 12:29:43}
\pmowner{mathcam}{2727}
\pmmodifier{mathcam}{2727}
\pmtitle{binomial proof of positive integer power rule}
\pmrecord{8}{32721}
\pmprivacy{1}
\pmauthor{mathcam}{2727}
\pmtype{Proof}
\pmcomment{trigger rebuild}
\pmclassification{msc}{26A03}

\endmetadata

% this is the default PlanetMath preamble.  as your knowledge
% of TeX increases, you will probably want to edit this, but
% it should be fine as is for beginners.

% almost certainly you want these
\usepackage{amssymb}
\usepackage{amsmath}
\usepackage{amsfonts}
\usepackage{amsthm}

% used for TeXing text within eps files
%\usepackage{psfrag}
% need this for including graphics (\includegraphics)
%\usepackage{graphicx}
% for neatly defining theorems and propositions
%\usepackage{amsthm}
% making logically defined graphics
%%%\usepackage{xypic}

% there are many more packages, add them here as you need them

% define commands here

\newcommand{\mc}{\mathcal}
\newcommand{\mb}{\mathbb}
\newcommand{\mf}{\mathfrak}
\newcommand{\ol}{\overline}
\newcommand{\ra}{\rightarrow}
\newcommand{\la}{\leftarrow}
\newcommand{\La}{\Leftarrow}
\newcommand{\Ra}{\Rightarrow}
\newcommand{\nor}{\vartriangleleft}
\newcommand{\Gal}{\text{Gal}}
\newcommand{\GL}{\text{GL}}
\newcommand{\Z}{\mb{Z}}
\newcommand{\R}{\mb{R}}
\newcommand{\Q}{\mb{Q}}
\newcommand{\C}{\mb{C}}
\newcommand{\<}{\langle}
\renewcommand{\>}{\rangle}
\begin{document}
We will use the difference quotient in this proof of the power rule for positive integers. Let $f(x)=x^n$ for some integer $n\geq 0$. Then we have
\begin{align*}
f'(x) = \lim_{h\rightarrow 0} \frac{(x+h)^n - x^n}{h}.
\end{align*}
We can use the binomial theorem to expand the numerator
\begin{align*}
f'(x) = \lim_{h\rightarrow 0} \frac{C_0^n x^0h^n + C_1^n x^1h^{n-1} + \cdots + C_{n-1}^n x^{n-1}h^1 + C_n^n x^nh^0 - x^n}{h}
\end{align*}
where $C_k^n=\frac{n!}{k!(n-k)!}$. We can now simplify the above
\begin{align*}
f'(x)&= \lim_{h\rightarrow 0} \frac{h^n + nxh^{n-1} + \cdots + nx^{n-1}h + x^n - x^n}{h}\\
&=\lim_{h\rightarrow 0} (h^{n-1} + nxh^{n-2} + \cdots + nx^{n-1})\\
&=nx^{n-1}\\
&= nx^{n-1}.
\end{align*}
%%%%%
%%%%%
\end{document}
