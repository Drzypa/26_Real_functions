\documentclass[12pt]{article}
\usepackage{pmmeta}
\pmcanonicalname{GradientTheorem}
\pmcreated{2013-03-22 19:11:16}
\pmmodified{2013-03-22 19:11:16}
\pmowner{pahio}{2872}
\pmmodifier{pahio}{2872}
\pmtitle{gradient theorem}
\pmrecord{8}{42096}
\pmprivacy{1}
\pmauthor{pahio}{2872}
\pmtype{Theorem}
\pmcomment{trigger rebuild}
\pmclassification{msc}{26B12}
\pmsynonym{fundamental theorem of line integrals}{GradientTheorem}
\pmrelated{LaminarField}
\pmrelated{Gradient}

% this is the default PlanetMath preamble.  as your knowledge
% of TeX increases, you will probably want to edit this, but
% it should be fine as is for beginners.

% almost certainly you want these
\usepackage{amssymb}
\usepackage{amsmath}
\usepackage{amsfonts}

% used for TeXing text within eps files
%\usepackage{psfrag}
% need this for including graphics (\includegraphics)
%\usepackage{graphicx}
% for neatly defining theorems and propositions
 \usepackage{amsthm}
% making logically defined graphics
%%%\usepackage{xypic}

% there are many more packages, add them here as you need them

% define commands here

\theoremstyle{definition}
\newtheorem*{thmplain}{Theorem}

\begin{document}
\PMlinkescapeword{domain}

If\, $u = u(x,\,y,\,z)$\, is continuously differentiable function in a simply connected \PMlinkname{domain}{Domain2} $D$ of $\mathbb{R}^3$ and\, 
$P = (x_0,\,y_0,\,z_0)$\, and\, $Q = (x_1,\,y_1,\,z_1)$\, lie in this domain, then
\begin{align}
\int_P^Q\!\nabla u\!\cdot\!\vec{ds} \;=\; u(x_1,\,y_1,\,z_1)-u(x_0,\,y_0,\,z_0)
\end{align}
where the line integral of the left hand side is taken along an arbitrary path in $D$.\\


The equation (1) is illustrated by the fact that 
$$\nabla u\!\cdot\!\vec{ds} 
\;=\; \frac{\partial u}{\partial x}dx+\frac{\partial u}{\partial y}dy+\frac{\partial u}{\partial z}dz$$
is the total differential of $u$, and thus
$$\int_P^Q\!\nabla u\!\cdot\!\vec{ds} \;=\; \int_P^Q\!du.$$
%%%%%
%%%%%
\end{document}
