\documentclass[12pt]{article}
\usepackage{pmmeta}
\pmcanonicalname{FejerKernel}
\pmcreated{2013-03-22 14:11:56}
\pmmodified{2013-03-22 14:11:56}
\pmowner{mathwizard}{128}
\pmmodifier{mathwizard}{128}
\pmtitle{Fejer kernel}
\pmrecord{8}{35630}
\pmprivacy{1}
\pmauthor{mathwizard}{128}
\pmtype{Definition}
\pmcomment{trigger rebuild}
\pmclassification{msc}{26A30}
\pmrelated{DiracSequence}

% this is the default PlanetMath preamble.  as your knowledge
% of TeX increases, you will probably want to edit this, but
% it should be fine as is for beginners.

% almost certainly you want these
\usepackage{amssymb}
\usepackage{amsmath}
\usepackage{amsfonts}

% used for TeXing text within eps files
%\usepackage{psfrag}
% need this for including graphics (\includegraphics)
\usepackage{graphicx}
% for neatly defining theorems and propositions
%\usepackage{amsthm}
% making logically defined graphics
%%%\usepackage{xypic}

% there are many more packages, add them here as you need them

% define commands here
\begin{document}
The Fejer kernel $F_n$ of order $n$ is defined as
$$F_n(t)=\frac{1}{n}\sum_{k=0}^{n-1}D_k(t),$$
where $D_n$ is the Dirichlet kernel of order $n$. The Fejer kernel can be written as
\begin{equation}\label{eq:rep}
F_n(t)=\frac{1}{n}\left(\frac{\sin\frac{nt}{2}}{\sin\frac{t}{2}}\right)^2.
\end{equation}
\textbf{Proof:} Since
$$D_n(t)=\frac{\sin\left(\left(n+\frac{1}{2}\right)t\right)}{\sin\frac{t}{2}}$$
we have
$$\sin\frac{t}{2}D_n(t)=\sin\left(\left(n+\frac{1}{2}\right)t\right).$$
Therefore
\begin{align*}
n\sin^2\frac{t}{2}F_n(t)& =\sum_{k=0}^{n-1}\sin\left(\left(k+\frac{1}{2}\right)t\right)\sin\frac{t}{2}\\
&=\frac{1}{2}\sum_{k=0}^{n-1}(\cos kt-\cos((k+1)t)\\
&=\frac{1}{2}(1-\cos nt)\\
&=\sin^2\frac{nt}{2}.
\end{align*}
From this follows equation (\ref{eq:rep}).
\begin{figure}[h]
\begin{centering}
\includegraphics[scale=0.5]{fejer2.ps}
\caption{Graphs of some Fejer kernels}\label{fig:fejer}
\end{centering}
\end{figure}
%%%%%
%%%%%
\end{document}
