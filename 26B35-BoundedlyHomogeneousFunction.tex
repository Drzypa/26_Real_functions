\documentclass[12pt]{article}
\usepackage{pmmeta}
\pmcanonicalname{BoundedlyHomogeneousFunction}
\pmcreated{2013-03-22 19:13:17}
\pmmodified{2013-03-22 19:13:17}
\pmowner{pahio}{2872}
\pmmodifier{pahio}{2872}
\pmtitle{boundedly homogeneous function}
\pmrecord{9}{42142}
\pmprivacy{1}
\pmauthor{pahio}{2872}
\pmtype{Definition}
\pmcomment{trigger rebuild}
\pmclassification{msc}{26B35}
\pmclassification{msc}{15-00}
\pmsynonym{boundedly homogeneous}{BoundedlyHomogeneousFunction}
\pmdefines{set of homogeneity}
\pmdefines{degree of homogeneity}

\endmetadata

% this is the default PlanetMath preamble.  as your knowledge
% of TeX increases, you will probably want to edit this, but
% it should be fine as is for beginners.

% almost certainly you want these
\usepackage{amssymb}
\usepackage{amsmath}
\usepackage{amsfonts}

% used for TeXing text within eps files
%\usepackage{psfrag}
% need this for including graphics (\includegraphics)
%\usepackage{graphicx}
% for neatly defining theorems and propositions
 \usepackage{amsthm}
% making logically defined graphics
%%%\usepackage{xypic}

% there are many more packages, add them here as you need them

% define commands here

\theoremstyle{definition}
\newtheorem*{thmplain}{Theorem}

\begin{document}
A function \,$f\!: \mathbb{R}^n \to \mathbb{R}$,\, where $n$ is a positive integer, is called \emph{boundedly homogeneous} with respect to a set $\Lambda$ of positive reals and a real number $r$, if the equation
   $$f(\lambda \vec{x}) \;=\; \lambda^r f(\vec{x})$$
is true for all and $\vec{x} \in \mathbb{R}^n$\, and\, $\lambda \in \Lambda$.\, Then $\Lambda$ is the \emph{set of homogeneity} and $r$ the \emph{degree of homogeneity} of $f$. \\

\textbf{Example.}\, The function\, $x \mapsto x^r\sin(\ln{x})$\, is boundedly homogeneous with respect to the set 
$\Lambda = \{e^{2\pi\nu}\,\vdots\;\; \nu \in \mathbb{Z}\}$\, and\, with degree of homogeneity $r$.\\


\textbf{Theorem.}\, Let\, $f\!: \mathbb{R}_+ \to \mathbb{R}$\, be a boundedly homogeneous function with the degree of homogeneity $r$ and the set of homogeneity\, $\Lambda \supset \{1\}$.\, Then $f$ is of the form
\begin{align}
f(x) \;=\; x^rf_1(\ln{x})
\end{align}
where\, $f_1\!: \mathbb{R} \to \mathbb{R}$\, is a periodic real function depending on $f$.

\emph{Proof.}\, Defining\, $g(x) := \frac{f(x)}{x^r}$,\, we obtain
$$g(\lambda x) \;=\; \frac{f(\lambda x)}{(\lambda x)^r} \;=\; \frac{\lambda^rf(x)}{\lambda^rx^r} \;=\; \lambda^0g(x) 
\quad \forall \lambda \in \Lambda.$$
Thus $g$ is a boundedly homogeneous function with the set of homogeneity $\Lambda$ and the degree of homogeneity 0.\, Moreover, define\, $f_1(x) := g(e^x)$.\, If\, $\lambda \in \Lambda\!\smallsetminus\!\{1\}$\, and\, $p := \ln\lambda$,\, 
we see that
$$f_1(x\!+\!p) \;=\; g(e^xe^p) \;=\; g(e^x\lambda) \;=\; g(e^x) \;=\; f_1(x) \quad \forall x \in \mathbb{R}_+.$$
Therefore, $f_1$ is periodic and (1) is in \PMlinkescapetext{force}.

\begin{thebibliography}{8}
\bibitem{KS}{\sc Konrad Schlude}: ``Bemerkung zu beschr\"ankt homogenen Funktionen''.\, -- \emph{Elemente der Mathematik} \textbf{54} (1999).
\end{thebibliography}

%%%%%
%%%%%
\end{document}
