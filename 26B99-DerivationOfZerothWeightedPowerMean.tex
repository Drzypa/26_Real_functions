\documentclass[12pt]{article}
\usepackage{pmmeta}
\pmcanonicalname{DerivationOfZerothWeightedPowerMean}
\pmcreated{2013-03-22 13:10:29}
\pmmodified{2013-03-22 13:10:29}
\pmowner{pbruin}{1001}
\pmmodifier{pbruin}{1001}
\pmtitle{derivation of zeroth weighted power mean}
\pmrecord{6}{33620}
\pmprivacy{1}
\pmauthor{pbruin}{1001}
\pmtype{Derivation}
\pmcomment{trigger rebuild}
\pmclassification{msc}{26B99}
%\pmkeywords{power mean}
\pmrelated{PowerMean}
\pmrelated{GeometricMean}
\pmrelated{GeneralMeansInequality}
\pmrelated{DerivationOfHarmonicMeanAsTheLimitOfThePowerMean}

\endmetadata

% this is the default PlanetMath preamble.  as your knowledge
% of TeX increases, you will probably want to edit this, but
% it should be fine as is for beginners.

% almost certainly you want these
\usepackage{amssymb}
\usepackage{amsmath}
\usepackage{amsfonts}

% used for TeXing text within eps files
%\usepackage{psfrag}
% need this for including graphics (\includegraphics)
%\usepackage{graphicx}
% for neatly defining theorems and propositions
%\usepackage{amsthm}
% making logically defined graphics
%%%\usepackage{xypic}

% there are many more packages, add them here as you need them

% define commands here
\begin{document}
Let $x_1,x_2,\ldots,x_n$ be positive real numbers, and let
$w_1,w_2,\ldots,w_n$ be positive real numbers such that
$w_1+w_2+\cdots+w_n=1$.  For $r\neq 0$, the $r$-th weighted power mean
of $x_1,x_2,\ldots,x_n$ is
$$
M_w^r(x_1,x_2,\ldots,x_n)=(w_1x_1^r+w_2x_2^r+\cdots+w_nx_n^r)^{1/r}.
$$
Using the Taylor series expansion $e^t=1+t+{\mathcal O}(t^2)$, where ${\mathcal O}(t^2)$
is Landau notation for terms of order $t^2$ and higher, we can
write $x_i^r$ as
$$
x_i^r=e^{r\log x_i}=1+r\log x_i+{\mathcal O}(r^2).
$$
By substituting this into the definition of $M_w^r$, we get
\begin{eqnarray*}
M_w^r(x_1,x_2,\ldots,x_n)&=&\left[w_1(1+r\log x_1)
+\cdots+w_n(1+r\log x_n)+{\mathcal O}(r^2)\right]^{1/r}\\
&=&\left[1+r(w_1\log x_1+\cdots+w_n\log x_n)+{\mathcal O}(r^2)\right]^{1/r}\\
&=&\left[1+r\log(x_1^{w_1}x_2^{w_2}\cdots x_n^{w_n})+{\mathcal O}(r^2)\right]^{1/r}\\
&=&\exp\left\{\frac{1}{r}\log\left[1+r\log(x_1^{w_1}x_2^{w_2}\cdots
x_n^{w_n})+{\mathcal O}(r^2)\right]\right\}.
\end{eqnarray*}
Again using a Taylor series, this time $\log (1+t)=t+{\mathcal O}(t^2)$, we get
\begin{eqnarray*}
M_w^r(x_1,x_2,\ldots,x_n)&=&\exp\left\{\frac{1}{r}
\left[r\log(x_1^{w_1}x_2^{w_2}\cdots x_n^{w_n})+{\mathcal O}(r^2)\right]\right\}\\
&=&\exp\left[\log(x_1^{w_1}x_2^{w_2}\cdots x_n^{w_n})+{\mathcal O}(r)\right].
\end{eqnarray*}
Taking the limit $r\to 0$, we find
\begin{eqnarray*}
M_w^0(x_1,x_2,\ldots,x_n)&=&\exp\left[\log
(x_1^{w_1}x_2^{w_2}\cdots x_n^{w_n})\right]\\
&=&x_1^{w_1}x_2^{w_2}\cdots x_n^{w_n}.
\end{eqnarray*}
In particular, if we choose all the weights to be $\frac{1}{n}$,
$$
M^0(x_1,x_2,\ldots,x_n)=\sqrt[n]{x_1x_2\cdots x_n},
$$
the geometric mean of $x_1,x_2,\ldots,x_n$.
%%%%%
%%%%%
\end{document}
