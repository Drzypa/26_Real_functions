\documentclass[12pt]{article}
\usepackage{pmmeta}
\pmcanonicalname{EquivalentFormulationsForContinuity}
\pmcreated{2013-03-22 15:18:23}
\pmmodified{2013-03-22 15:18:23}
\pmowner{matte}{1858}
\pmmodifier{matte}{1858}
\pmtitle{equivalent formulations for continuity}
\pmrecord{14}{37106}
\pmprivacy{1}
\pmauthor{matte}{1858}
\pmtype{Theorem}
\pmcomment{trigger rebuild}
\pmclassification{msc}{26A15}
\pmclassification{msc}{54C05}
\pmrelated{Characterization}

\endmetadata

% this is the default PlanetMath preamble.  as your knowledge
% of TeX increases, you will probably want to edit this, but
% it should be fine as is for beginners.

% almost certainly you want these
\usepackage{amssymb}
\usepackage{amsmath}
\usepackage{amsfonts}
\usepackage{amsthm}

\usepackage{mathrsfs}

% used for TeXing text within eps files
%\usepackage{psfrag}
% need this for including graphics (\includegraphics)
%\usepackage{graphicx}
% for neatly defining theorems and propositions
%
% making logically defined graphics
%%%\usepackage{xypic}

% there are many more packages, add them here as you need them

% define commands here

\newcommand{\sR}[0]{\mathbb{R}}
\newcommand{\sC}[0]{\mathbb{C}}
\newcommand{\sN}[0]{\mathbb{N}}
\newcommand{\sZ}[0]{\mathbb{Z}}

 \usepackage{bbm}
 \newcommand{\Z}{\mathbbmss{Z}}
 \newcommand{\C}{\mathbbmss{C}}
 \newcommand{\F}{\mathbbmss{F}}
 \newcommand{\R}{\mathbbmss{R}}
 \newcommand{\Q}{\mathbbmss{Q}}



\newcommand*{\norm}[1]{\lVert #1 \rVert}
\newcommand*{\abs}[1]{| #1 |}



\newtheorem{thm}{Theorem}
\newtheorem{defn}{Definition}
\newtheorem{prop}{Proposition}
\newtheorem{lemma}{Lemma}
\newtheorem{cor}{Corollary}

\def\closure{\overline}
\begin{document}
Suppose $f\colon X\to Y$ is a function between topological spaces
$X$, $Y$. Then the following are equivalent:
\begin{enumerate}
\item $f$ is continuous.
\item If $B$ is open in $Y$, then $f^{-1}(B)$ is open in $X$.
\item If $B$ is closed in $Y$, then $f^{-1}(B)$ is closed in $X$. 
\item $f\!\left(\closure{A}\right)\subseteq\closure{f(A)}$
  for all $A\subseteq X$.
\item If $(x_i)$ is a net in $X$ converging to $x$, then 
  $(f(x_i))$ is a net in $Y$ converging to $f(x)$. The concept of net
  can be replaced by the more familiar one of sequence if the spaces
  $X$ and $Y$ are first countable.
\item Whenever two nets $S$ and $T$ in $X$ converge to the same point, then $f \circ S$ and $f \circ T$
converge to the same point in $Y$.
\item If $\mathbb{F}$ is a filter on $X$ that converges to $x$, then $f(\mathbb{F})$ is a filter on $Y$ that converges to $f(x)$.  Here, $f(\mathbb{F})$ is the filter generated by the filter base $\lbrace f(F)\mid F\in \mathbb{F}\rbrace$.
\item If $B$ is any element of a \PMlinkname{subbase}{Subbasis} $\mathcal{S}$ for the topology of $Y$,
  then $f^{-1}(B)$ is open in $X$.
\item If $B$ is any element of a basis $\mathcal{B}$ for the topology of $Y$, then $f^{-1}(B)$ is open in $X$.
\item If $x \in X$, and $N$ is any neighborhood of $f(x)$, then $f^{-1}(N)$ is a neighborhood of $x$.
\item $f$ is continuous at every point in $X$.
\end{enumerate}
%%%%%
%%%%%
\end{document}
