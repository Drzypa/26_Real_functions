\documentclass[12pt]{article}
\usepackage{pmmeta}
\pmcanonicalname{ProofOfChainRuleseveralVariables}
\pmcreated{2013-03-22 16:05:07}
\pmmodified{2013-03-22 16:05:07}
\pmowner{paolini}{1187}
\pmmodifier{paolini}{1187}
\pmtitle{proof of chain rule (several variables)}
\pmrecord{6}{38145}
\pmprivacy{1}
\pmauthor{paolini}{1187}
\pmtype{Proof}
\pmcomment{trigger rebuild}
\pmclassification{msc}{26B12}

\endmetadata

% this is the default PlanetMath preamble.  as your knowledge
% of TeX increases, you will probably want to edit this, but
% it should be fine as is for beginners.

% almost certainly you want these
\usepackage{amssymb}
\usepackage{amsmath}
\usepackage{amsfonts}

% used for TeXing text within eps files
%\usepackage{psfrag}
% need this for including graphics (\includegraphics)
%\usepackage{graphicx}
% for neatly defining theorems and propositions
\usepackage{amsthm}
% making logically defined graphics
%%%\usepackage{xypic}

% there are many more packages, add them here as you need them

% define commands here
\newcommand{\R}{\mathbb R}
\newtheorem{theorem}{Theorem}
\newtheorem{definition}{Definition}
\theoremstyle{remark}
\newtheorem{example}{Example}
\begin{document}
We first consider the case $m=1$ i.e.\ $G\colon I\to \R^n$ where $I\subset \R$ is a neighbourhood of a point $x_0\in \R$ and $F\colon U\subset \R^n\to \R$ is defined on a neighbourhood $U$ of $y_0=G(x_0)$ such that $G(I)\subset U$. We suppose that both $G$ is differentiable at the point $x_0$ and $F$ is differentiable in $y_0$. We want to compute the derivative of the compound
function $H(x)=F(G(x))$ at $x=x_0$.

By the definition of derivative (using Landau notation) we have
\[
  F(y_0+ k) = F(y_0) + DF(y_0) k + o(|k|).
\]
Choose any $h\neq 0$ such that $x_0+h\in I$ and set $k=G(x_0+h)-G(x_0)$ to obtain
\begin{align*}
\frac{H(x_0+h)-H(x_0)}{h} &=
\frac{F(G(x_0+h))-F(G(x_0))}{h}\\
&=\frac{F(G(x_0)+k)-F(G(x_0))}{h} = \frac{F(y_0+k)-F(y_0)}{h}\\
&=\frac{DF(y_0) (G(x_0+h)-G(x_0)) + o(|G(x_0+h)-G(x_0)|)}{h}\\
&=DF(y_0) \frac{G(x_0 + h) -G(x_0)}{h} +\frac{o(|G(x_0+h)-G(x_0)|)}{h}.
\end{align*}

Letting $h\to 0$ the first term of the sum converges to $DF(y_0) G'(x_0)$ hence 
we want to prove that the second term converges to $0$.
Indeed we have
\[
  \left|\frac{o(|G(x_0+h)-G(x_0)|)}{h}\right|
 =\left|\frac{o(|G(x_0+h)-G(x_0)|)}{|G(x_0+h)-G(x_0)|}\right|\cdot
  \left|\frac{G(x_0+h)-G(x_0)}{h}\right|.
\]
By the definition of $o(\cdot)$ the first fraction tends to $0$, while 
the second fraction tends to the absolute value of $G'(x_0)$. Thus the product
tends to $0$, as needed.

Consider now the general case $G\colon V\subset \R^m\to U\subset \R^n$.
Given $v\in \R^m$ we are going to compute the directional derivative
\[
\frac{\partial F\circ G}{\partial v} (x_0) = \frac{d F\circ g}{dt} (0)
\]
where $g(t)=G(x_0+tv)$ is a function of a single variable $t\in\R$. Thus
we fall back to the previous case and we find that 
\[
\frac{\partial F\circ G}{\partial v} (x_0) = DF(G(x_0)) g'(0).
= DF(G(x_0)) \frac{\partial G}{\partial v} (x_0)
\]
In particular when $v=e_k$ is the $k$-th coordinate vector, we find
\[
g'(0) = 
 D_{x_k} F \circ G (x_0) = DF(G(x_0)) D_{x_k} G(x_0)
= \sum_{i=1}^n D_{y_i} G(x_0) D_{x_k} G^i(x_0)
\]
which can be compactly written
\[
  D F \circ G(x_0) = DF(G(x_0)) DG(x_0).
\]

%%%%%
%%%%%
\end{document}
