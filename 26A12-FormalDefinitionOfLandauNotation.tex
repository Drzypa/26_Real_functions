\documentclass[12pt]{article}
\usepackage{pmmeta}
\pmcanonicalname{FormalDefinitionOfLandauNotation}
\pmcreated{2013-03-22 15:15:48}
\pmmodified{2013-03-22 15:15:48}
\pmowner{paolini}{1187}
\pmmodifier{paolini}{1187}
\pmtitle{formal definition of Landau notation}
\pmrecord{6}{37049}
\pmprivacy{1}
\pmauthor{paolini}{1187}
\pmtype{Definition}
\pmcomment{trigger rebuild}
\pmclassification{msc}{26A12}
\pmsynonym{Landau notation}{FormalDefinitionOfLandauNotation}
\pmsynonym{small o}{FormalDefinitionOfLandauNotation}
\pmsynonym{big o}{FormalDefinitionOfLandauNotation}
\pmsynonym{order of infinity}{FormalDefinitionOfLandauNotation}
\pmsynonym{order of zero}{FormalDefinitionOfLandauNotation}
\pmrelated{PropertiesOfOAndO}

\endmetadata

% this is the default PlanetMath preamble.  as your knowledge
% of TeX increases, you will probably want to edit this, but
% it should be fine as is for beginners.

% almost certainly you want these
\usepackage{amssymb}
\usepackage{amsmath}
\usepackage{amsfonts}

% used for TeXing text within eps files
%\usepackage{psfrag}
% need this for including graphics (\includegraphics)
%\usepackage{graphicx}
% for neatly defining theorems and propositions
\usepackage{amsthm}
% making logically defined graphics
%%%\usepackage{xypic}

% there are many more packages, add them here as you need them

% define commands here
\newcommand{\R}{\mathbb R}
\newtheorem{theorem}{Theorem}
\newtheorem{definition}{Definition}
\theoremstyle{remark}
\newtheorem{example}{Example}
\begin{document}
Let us consider a domain $D$ and an accumulation point $x_0\in \overline D$. Important examples are $D=\R$ and $x_0\in D$ or $D=\mathbb N$ and $x_0=+\infty$. Let $f\colon D\to \R$ be any function. We are going to define the spaces $o(f)$ and $O(f)$ which are families of real functions defined on $D$ and which depend on the point $x_0\in \overline D$. 

Suppose first that there exists a neighbourhood $U$ of $x_0$ such that $f$ restricted to $U\cap D$ is always different from zero.
We say that $g\in o(f)$  as $x\to x_0$ if
\[
  \lim_{x\to x_0} \frac{g(x)}{f(x)}=0.
\]
We say that $g \in O(f)$ as $x\to x_0$ if there exists a neighbourhood $U$ of $x_0$ such that 
\[
  \frac{g(x)}{f(x)} \text{is bounded if restricted to $D\cap U$}.
\]
In the case when $f\equiv 0$ in a neighbourhood of $x_0$, we define $o(f)=O(f)$ as the set of all functions $g$ which are null in a neighbourhood of $0$.

The families $o$ and $O$ are usually called "small-o" and "big-o" or, sometimes,
"small ordo", "big ordo".

%%%%%
%%%%%
\end{document}
