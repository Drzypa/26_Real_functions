\documentclass[12pt]{article}
\usepackage{pmmeta}
\pmcanonicalname{PropertiesOfTheLebesgueIntegralOfLebesgueIntegrableFunctions}
\pmcreated{2013-03-22 16:14:01}
\pmmodified{2013-03-22 16:14:01}
\pmowner{Wkbj79}{1863}
\pmmodifier{Wkbj79}{1863}
\pmtitle{properties of the Lebesgue integral of Lebesgue integrable functions}
\pmrecord{19}{38334}
\pmprivacy{1}
\pmauthor{Wkbj79}{1863}
\pmtype{Theorem}
\pmcomment{trigger rebuild}
\pmclassification{msc}{26A42}
\pmclassification{msc}{28A25}
\pmrelated{PropertiesOfTheLebesgueIntegralOfNonnegativeMeasurableFunctions}

\usepackage{amssymb}
\usepackage{amsmath}
\usepackage{amsfonts}

\usepackage{psfrag}
\usepackage{graphicx}
\usepackage{amsthm}
%%\usepackage{xypic}

\newtheorem*{thm*}{Theorem}

\begin{document}
\begin{thm*}
Let $(X, \mathfrak{B}, \mu)$ be a measure space, $f \colon X \to [-\infty,\infty]$ and $g \colon X \to [-\infty,\infty]$ be Lebesgue integrable functions, and $A,B \in \mathfrak{B}$. Then the following properties hold:

\begin{enumerate}
\item $\displaystyle \left| \int_A f \, d\mu \right| \le \int_A |f| \, d\mu$

\item If $f \le g$, then $\displaystyle \int_A f \, d\mu \le \int_A g \, d\mu$.

\item $\displaystyle \int_A f \, d\mu =\int_X \chi_A f \, d\mu$, where $\chi_A$ denotes the characteristic function of $A$

\item If $c \in \mathbb{R}$, then $\displaystyle \int_A cf \, d\mu =c\int_A f \, d\mu$.

\item If $\mu(A)=0$, then $\displaystyle \int_A f \, d\mu =0$.

\item $\displaystyle \int_A (f+g) \, d\mu =\int_A f \, d\mu +\int_A g \, d\mu$.

\item If $A \cap B=\emptyset$, then $\displaystyle \int_{A \cup B} f \, d\mu =\int_A f \, d\mu +\int_B f \, d\mu$.

\item If $f=g$ almost everywhere with respect to $\mu$, then $\displaystyle \int_A f \, d\mu =\int_A g \, d\mu$.

\end{enumerate}
\end{thm*}

\begin{proof}

\begin{enumerate}
\item

\vspace{1mm}

\begin{center}
\begin{tabular}{ll}
$\displaystyle \left| \int_A f \, d\mu \right|$ & $\displaystyle =\left| \int_A f^+ \, d\mu -\int_A f^- \, d\mu \right|$ by definition \\
& $\displaystyle \le \left| \int_A f^+ \, d\mu \right| +\left| \int_A f^- \, d\mu \right|$ by the triangle inequality \\
& $\displaystyle =\int_A f^+ \, d\mu +\int_A f^- \, d\mu$ by the \\
& properties of the Lebesgue integral of nonnegative measurable functions (property 1), \\
& $\displaystyle =\int_A (f^++f^-) \, d\mu$ by the \\
& \PMlinkname{properties of the Lebesgue integral of nonnegative measurable functions}{PropertiesOfTheLebesgueIntegralOfNonnegativeMeasurableFunctions} (property 7), \\
& $\displaystyle =\int_A |f| \, d\mu$ \end{tabular}
\end{center}

\item Since $f \le g$, the following must hold:

\begin{itemize}
\item $f^+=\max\{0,f\}\le\max\{0,g\}=g^+$;
\item $-f \ge -g$;
\item $f^-=\max\{0,-f\}\ge\max\{0,-g\}=g^-$.
\end{itemize}

Thus, by the \PMlinkname{properties of the Lebesgue integral of nonnegative measurable functions}{PropertiesOfTheLebesgueIntegralOfNonnegativeMeasurableFunctions} (property 2), $\displaystyle \int_A f^+ \, d\mu \le \int_A g^+ \, d\mu$ and $\displaystyle \int_A f^- \, d\mu \ge \int_A g^- \, d\mu$.  Therefore, $\displaystyle -\int_A f^- \, d\mu \le -\int_A g^- \, d\mu$.  Hence, $\displaystyle \int_A f^+ \, d\mu -\int_A f^- \, d\mu \le \int_A g^+ \, d\mu -\int_A f^- \, d\mu \le \int_A g^+ \, d\mu -\int_A g^- \, d\mu$.  It follows that $\displaystyle \int_A f \, d\mu \le \int_A g \, d\mu$.

\item

\vspace{1mm}

\begin{center}
\begin{tabular}{ll}
$\displaystyle \int_A f \, d\mu$ & $\displaystyle =\int_A f^+ \, d\mu -\int_A f^- \, d\mu$ by definition \\
& $\displaystyle =\int_X \chi_Af^+ \, d\mu -\int_X \chi_Af^- \, d\mu$ by the \\
& \PMlinkname{properties of the Lebesgue integral of nonnegative measurable functions}{PropertiesOfTheLebesgueIntegralOfNonnegativeMeasurableFunctions} (property 3), \\
& $\displaystyle =\int_X (\chi_Af)^+ \, d\mu -\int_X (\chi_Af)^- \, d\mu$ \\
& $\displaystyle =\int_X \chi_Af \, d\mu$ by definition \end{tabular}
\end{center}

\item If $c \ge 0$, then

\begin{center}
\begin{tabular}{ll}
$\displaystyle \int_A cf \, d\mu$ & $\displaystyle =\int_A (cf)^+ \, d\mu -\int_A (cf)^- \, d\mu$ by definition \\
& $\displaystyle =\int_A cf^+ \, d\mu -\int_A cf^- \, d\mu$ \\
& $\displaystyle =c\int_A f^+ \, d\mu -c\int_A f^- \, d\mu$ by the \\
& \PMlinkname{properties of the Lebesgue integral of nonnegative measurable functions}{PropertiesOfTheLebesgueIntegralOfNonnegativeMeasurableFunctions} (property 5) \\
& $\displaystyle =c\left( \int_A f^+ \, d\mu -\int_A f^- \, d\mu \right)$ \\
& $\displaystyle =c\int_A f \, d\mu$ by definition. \end{tabular}
\end{center}

If $c<0$, then

\begin{center}
\begin{tabular}{ll}
$\displaystyle \int_A cf \, d\mu$ & $\displaystyle =\int_A (cf)^+ \, d\mu -\int_A (cf)^- \, d\mu$ by definition \\
& $\displaystyle =\int_A (-c)f^- \, d\mu -\int_A (-c)f^+ \, d\mu$ \\
& $\displaystyle =-c\int_A f^- \, d\mu +c\int_A f^+ \, d\mu$ by the \\
& \PMlinkname{properties of the Lebesgue integral of nonnegative measurable functions}{PropertiesOfTheLebesgueIntegralOfNonnegativeMeasurableFunctions} (property 5) \\
& $\displaystyle =c\left( -\int_A f^- \, d\mu +\int_A f^+ \, d\mu \right)$ \\
& $\displaystyle =c\int_A f \, d\mu$ by definition. \end{tabular}
\end{center}

\item Note that $\displaystyle \int_A f^+ \, d\mu=0$ and $\displaystyle \int_A f^- \, d\mu=0$ by the \PMlinkname{properties of the Lebesgue integral of nonnegative measurable functions}{PropertiesOfTheLebesgueIntegralOfNonnegativeMeasurableFunctions} (property 6).  It follows that $\displaystyle \int_A f \, d\mu =0$.

\item Let $\{s_n\}$ be a nondecreasing sequence of nonnegative simple functions converging pointwise to $f^++g^+$ and $\{t_n\}$ be a nondecreasing sequence of nonnegative simple functions converging pointwise to $f^-+g^-$.  Note that, for every $n$, $\displaystyle \int_A s_n \, d\mu -\int_A t_n \, d\mu =\int_A (s_n-t_n) \, d\mu$.

Since $f$ and $g$ are integrable and $|f+g| \le |f|+|g|$, $f+g$ is integrable.  Thus,

\begin{center}
\begin{tabular}{ll}
$\displaystyle \int_A f \, d\mu +\int_A g \, d\mu$ & $\displaystyle =\int_A f^+ \, d\mu -\int_A f^- \, d\mu +\int_A g^+ \, d\mu -\int_A g^- \, d\mu$ by definition \\
& $\displaystyle =\int_A f^+ \, d\mu +\int_A g^+ \, d\mu -\left( \int_A f^- \, d\mu +\int_A g^- d\mu \right)$ \\
& $\displaystyle =\int_A (f^++g^+) \, d\mu -\left( \int_A (f^-+g^-) \, d\mu \right)$ by the \\
& \PMlinkname{properties of the Lebesgue integral of nonnegative measurable functions}{PropertiesOfTheLebesgueIntegralOfNonnegativeMeasurableFunctions} (property 7) \\
& $\displaystyle =\lim_{n \to \infty} \int_A s_n \, d\mu -\left( \lim_{n \to \infty} \int_A t_n \, d\mu \right)$ by Lebesgue's monotone convergence theorem \\
& $\displaystyle =\lim_{n \to \infty} \left( \int_A s_n \, d\mu -\int_A t_n \, d\mu \right)$ \\
& $\displaystyle =\lim_{n \to \infty} \int_A (s_n-t_n) \, d\mu$ \\
& $\displaystyle =\int_A (f^++g^+-(f^-+g^-)) \, d\mu$ by Lebesgue's dominated convergence theorem \\
& $\displaystyle =\int_A (f^+-f^-+g^+-g^-) \, d\mu$ \\
& $\displaystyle =\int_A (f+g) \, d\mu$ by definition. \end{tabular}
\end{center}

\item

\vspace{1mm}

\begin{center}
\begin{tabular}{ll}
$\displaystyle \int_{A \cup B} f \, d\mu$ & $\displaystyle =\int_{A \cup B} f^+ \, d\mu -\int_{A \cup B} f^- \, d\mu$ by definition \\
& $\displaystyle =\int_A f^+ \, d\mu +\int_B f^+ \, d\mu -\left( \int_A f^- \, d\mu +\int_B f^- \, d\mu \right)$ by the \\
& \PMlinkname{properties of the Lebesgue integral of nonnegative measurable functions}{PropertiesOfTheLebesgueIntegralOfNonnegativeMeasurableFunctions} (property 8), \\
& $\displaystyle =\int_A f^+ \, d\mu -\int_A f^- \, d\mu +\int_B f^+ \, d\mu -\int_B f^- \, d\mu$ \\
& $\displaystyle =\int_A f \, d\mu +\int_B f \, d\mu$ by definition \end{tabular}
\end{center}

\item Let $E=\{x \in A:f(x)=g(x)\}$.  Since $f$ and $g$ are measurable functions and $A \in \mathfrak{B}$, it must be the case that $E \in \mathfrak{B}$.  Thus, $A-E \in \mathfrak{B}$.  By hypothesis, $\mu(A \setminus E)=0$.  Note that $E \cap (A \setminus E)=\emptyset$ and $E \cup (A \setminus E)=A$. Thus, $\displaystyle \int_A f \, d\mu =\int_E f \, d\mu +\int_{A \setminus E} f \, d\mu =\int_E f \, d\mu +0=\int_E g \, d\mu +0=\int_E g \, d\mu +\int_{A \setminus E} g \, d\mu =\int_A g \, d\mu.$

\end{enumerate}
\end{proof}
%%%%%
%%%%%
\end{document}
