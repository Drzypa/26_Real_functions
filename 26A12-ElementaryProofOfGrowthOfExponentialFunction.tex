\documentclass[12pt]{article}
\usepackage{pmmeta}
\pmcanonicalname{ElementaryProofOfGrowthOfExponentialFunction}
\pmcreated{2014-03-10 17:57:26}
\pmmodified{2014-03-10 17:57:26}
\pmowner{rspuzio}{6075}
\pmmodifier{rspuzio}{6075}
\pmtitle{elementary proof of growth of exponential function}
\pmrecord{28}{88041}
\pmprivacy{1}
\pmauthor{rspuzio}{6075}
\pmtype{Definition}
\pmcomment{refactoring final proof}
\pmclassification{msc}{26A12}
\pmclassification{msc}{26A06}

\endmetadata

% this is the default PlanetMath preamble.  as your knowledge
% of TeX increases, you will probably want to edit this, but
% it should be fine as is for beginners.

% almost certainly you want these
\usepackage{amssymb}
\usepackage{amsmath}
\usepackage{amsfonts}

% need this for including graphics (\includegraphics)
\usepackage{graphicx}
% for neatly defining theorems and propositions
\usepackage{amsthm}

% making logically defined graphics
%\usepackage{xypic}
% used for TeXing text within eps files
%\usepackage{psfrag}

% there are many more packages, add them here as you need them

% define commands here

\newtheorem{prop}{Proposition}
\begin{document}
\begin{prop}
If $x$ is a non-negative real number and $n$ 
is a non-negative integer, then $(1 + x)^n \ge 1 + nx$.
\end{lem}

\begin{proof}
When $n = 0$, we have $(1 + x)^0 = 1 \ge 1 + 0 \cdot 0$.  
If, for some natural number $n$, it is the case that 
$(1 + x)^n \ge 1 + nx$ then, multiplying both sides of
the inequality by $(1 + x)$, we have
\[ (1 + x)^{n + 1} \ge (1 + x)(1 + nx) =
   1 + (n + 1) x + nx^2 \ge 1 + (n + 1) x .\]
By induction, $(1 + x)^n \ge 1 + nx$ for every 
natural number $n$.
\end{proof}

\begin{prop}
If $b$ is a real number such that $b > 1$ and $n$ and
$k$ are non-negative integers, we have
$b^n > (\frac{b-1}{bk})^k n^k$.
\end{lem}

\begin{proof}
Let $x = b - 1$. Write $n = mk - r$ where $m$ and $r$ 
are non-negative integers and $r < k$.  

By the preceding proposition, $(1 + x)^m > mx$.  Raising both 
sides of this inequality to the $k^\hbox{th}$ power, we 
have $(1 + x)^{mk} > (mx)^k$.  Since $r < k$, we also 
have $(1 + x)^{-r} > (1 + x)^{-k}$; multiplying both 
sides by this inequality and collecting terms,
\[ (1 + x)^{mk - r} > (\frac{x}{1 + x})^k m^k. \]
Multiplying the right-hand side by $k^k/k^k$ and
rearranging,
\[ (\frac{x}{1 + x})^k m^k 
 = (\frac{x}{(1 + x)k})^k (mk)^k .\]
Since $mk \ge mk - r$, we also have
\[ (\frac{x}{(1 + x)k})^k (mk)^k \ge
(\frac{x}{(1 + x)k})^k (mk - r)^k.\]
Recalling that $mk - r = n$ and $1 + x = b$, we conclude that
\[ b^n > (\frac{b - 1}{bk})^k n^k .\]
\end{proof}

\begin{prop}
If $a$, $b$, and $x$ are real numbers such that $a \ge 0$,
$b > 1$ and $x > 0$, then
\[ b^x > ( \frac{(b - 1)^a}{b^{a + 1} (a + 1)^a} ) x^a .\]\
\end{thm}

\begin{proof}
Let $k$ and $n$ be integers such that $a \le k \le < a + 1$
and $x \le n \le x + 1$.  Since $x + 1 > n$, we have $b^{x + 1} 
> b^n$.  By the preceeding proposition, we have
\[ b^n > ( \frac{b - 1}{bk} )^k n^k .\]
Since $k < a +1$, we have $1/k^k > 1/(a + 1)^k$, so
\[ ( \frac{b - 1}{bk} )^k > ( \frac{b - 1}{b (a + 1)} )^k .\]
Since $k \ge a \ge 0$, we have
\[ ( \frac{b - 1}{b (a + 1)} )^k n^k \ge 
   ( \frac{b - 1}{b (a + 1)} )^a n^a .\]
Summarrizing our progress so far, 
\[ b^{x + 1} > ( \frac{b - 1}{b (a + 1)} )^a n^a .\]
Dividing both sides by $b$ and simplifying,
\[ b^x > ( \frac{(b - 1)^a}{b^{a + 1} (a + 1)^a} ) x^a .\]\
\end{proof}

\begin{prop}
If $a$ and $b$ are real numbers and $b >1$, then
\[ \lim_{x \to \infty} \frac{x^a}{b^x} = 0 .\]
\end{prop}

\begin{proof}
Substituting $a + 1$ for $a$% in the preceding proposition:
\[ b^x > ( \frac{(b - 1)^{a + 1}}
  {b^{a + 2} (a + 2)^{a + 1}} ) x^{a + 1} .\]
Dividing by $x$ and rearranging,
\[ 0 < \frac{x^a}{b^x} <
  ( \frac{b^{a + 2} (a + 2)^{a + 1}}{(b - 1)^{a + 1}} )
  \frac{1}{x} \]
Since $\lim_{x \to \infty} 0 = 0$ and $\lim_{x \to \infty}
\frac{1}{x} = 0$, we also have $\lim_{x \to \infty}
\frac{x^a}{b^x} = 0$ by the squeeze rule.

\end{proof}


\end{document}
