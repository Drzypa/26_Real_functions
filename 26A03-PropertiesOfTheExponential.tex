\documentclass[12pt]{article}
\usepackage{pmmeta}
\pmcanonicalname{PropertiesOfTheExponential}
\pmcreated{2013-03-22 12:30:02}
\pmmodified{2013-03-22 12:30:02}
\pmowner{rmilson}{146}
\pmmodifier{rmilson}{146}
\pmtitle{properties of the exponential}
\pmrecord{15}{32731}
\pmprivacy{1}
\pmauthor{rmilson}{146}
\pmtype{Theorem}
\pmcomment{trigger rebuild}
\pmclassification{msc}{26A03}

\endmetadata

\usepackage{amsmath}
\usepackage{amsfonts}
\usepackage{amssymb}

\newcommand{\reals}{\mathbb{R}}
\newcommand{\natnums}{\mathbb{N}}
\newcommand{\cnums}{\mathbb{C}}
\newcommand{\znums}{\mathbb{Z}}

\newcommand{\lp}{\left(}
\newcommand{\rp}{\right)}
\newcommand{\lb}{\left[}
\newcommand{\rb}{\right]}

\newcommand{\supth}{^{\text{th}}}


\newtheorem{proposition}{Proposition}
\begin{document}
The exponential operation possesses the following properties.  
\begin{itemize}
\item \PMlinkescapetext{{\bf  Homogeneity.}} For $x,y\in\reals^+, p\in \reals$ we have
$$(xy)^p = x^p y^p$$
\item \PMlinkescapetext{{\bf Exponent additivity.}} For $x\in\reals^+$ we have
$$x^0=1,\quad x^1 = x,\quad x^{p+q} = x^p x^q,\quad (x^p)^q=x^{pq}\qquad p,q\in\reals.$$
\item \PMlinkname{\bf Monotonicity.}{TotalOrder}  For $x,y\in\reals^+$ with $x<y$
  and $p\in \reals^+$ we have
  $$x^p < y^p,\qquad x^{-p} > y^{-p}.$$
\item {\bf Continuity.}
The exponential operation is continuous with respect to its
arguments. To be more precise, the following function is continuous:
$$P:\reals^+\times\reals\rightarrow \reals,\qquad P(x,y)=x^y.$$
\end{itemize}
Let us also note that the exponential operation is characterized (in
the sense of existence and uniqueness) by the {\em additivity} and
{\em continuity} properties.  [{\bf Author's note}: One can probably get away with
substantially less, but I haven't given this enough thought.]
%%%%%
%%%%%
\end{document}
