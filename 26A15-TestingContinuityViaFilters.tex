\documentclass[12pt]{article}
\usepackage{pmmeta}
\pmcanonicalname{TestingContinuityViaFilters}
\pmcreated{2013-03-22 19:09:31}
\pmmodified{2013-03-22 19:09:31}
\pmowner{CWoo}{3771}
\pmmodifier{CWoo}{3771}
\pmtitle{testing continuity via filters}
\pmrecord{4}{42064}
\pmprivacy{1}
\pmauthor{CWoo}{3771}
\pmtype{Result}
\pmcomment{trigger rebuild}
\pmclassification{msc}{26A15}
\pmclassification{msc}{54C05}
\pmrelated{Filter}

\usepackage{amssymb,amscd}
\usepackage{amsmath}
\usepackage{amsfonts}
\usepackage{mathrsfs}

% used for TeXing text within eps files
%\usepackage{psfrag}
% need this for including graphics (\includegraphics)
%\usepackage{graphicx}
% for neatly defining theorems and propositions
\usepackage{amsthm}
% making logically defined graphics
%%\usepackage{xypic}
\usepackage{pst-plot}

% define commands here
\newcommand*{\abs}[1]{\left\lvert #1\right\rvert}
\newtheorem{prop}{Proposition}
\newtheorem{thm}{Theorem}
\newtheorem{ex}{Example}
\newcommand{\real}{\mathbb{R}}
\newcommand{\pdiff}[2]{\frac{\partial #1}{\partial #2}}
\newcommand{\mpdiff}[3]{\frac{\partial^#1 #2}{\partial #3^#1}}
\begin{document}
\begin{prop} Let $X,Y$ be topological spaces.  Then a function $f:X\to Y$ is continuous iff it sends converging filters to converging filters. \end{prop}

\begin{proof}  Suppose first $f$ is continuous.  Let $\mathbb{F}$ be a filter in $X$ converging to $x$.  We want to show that $f(\mathbb{F}):=\lbrace f(F)\mid F\in \mathbb{F}\rbrace$ converges to $f(x)$.  Let $N$ be a neighborhood of $f(x)$.  So there is an open set $U$ such that $f(x)\in U\subseteq N$.  So $f^{-1}(U)$ is open and contains $x$, which means that $f^{-1}(U)\in \mathbb{F}$ by assumption.  This means that $ff^{-1}(U)\in f(\mathbb{F})$.  Since $ff^{-1}(U)\subseteq U \subseteq N$, we see that $N\in f(\mathbb{F})$ as well.

Conversely, suppose $f$ preserves converging filters.  Let $V$ be an open set in $Y$ containing $f(x)$.  We want to find an open set $U$ in $X$ containing $x$, such that $f(U)\subseteq V$.  Let $\mathbb{F}$ be the neighborhood filter of $x$.  So $\mathbb{F}\to x$.  By assumption, $f(\mathbb{F}) \to f(x)$.  Since $V$ is an open neighborhood of $f(x)$, we have $V\in f(\mathbb{F})$, or $f(F)\subseteq V$ for some $F\in \mathbb{F}$.  Since $F$ is a neighborhood of $x$, it contains an open neighborhood $U$ of $x$.  Furthermore, $f(U) \subseteq f(F) \subseteq V$.  Since $x$ is arbitrary, $f$ is continuous.
\end{proof}
%%%%%
%%%%%
\end{document}
