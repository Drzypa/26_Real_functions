\documentclass[12pt]{article}
\usepackage{pmmeta}
\pmcanonicalname{ExtremeValueTheorem}
\pmcreated{2013-03-22 14:29:21}
\pmmodified{2013-03-22 14:29:21}
\pmowner{classicleft}{5752}
\pmmodifier{classicleft}{5752}
\pmtitle{extreme value theorem}
\pmrecord{7}{36023}
\pmprivacy{1}
\pmauthor{classicleft}{5752}
\pmtype{Theorem}
\pmcomment{trigger rebuild}
\pmclassification{msc}{26A06}
\pmsynonym{Weierstrass extreme value theorem}{ExtremeValueTheorem}

\endmetadata

% this is the default PlanetMath preamble.  as your knowledge
% of TeX increases, you will probably want to edit this, but
% it should be fine as is for beginners.

% almost certainly you want these
\usepackage{amssymb}
\usepackage{amsmath}
\usepackage{amsfonts}

% used for TeXing text within eps files
%\usepackage{psfrag}
% need this for including graphics (\includegraphics)
%\usepackage{graphicx}
% for neatly defining theorems and propositions
%\usepackage{amsthm}
% making logically defined graphics
%%%\usepackage{xypic}

% there are many more packages, add them here as you need them

% define commands here
\begin{document}
{\bf Extreme Value Theorem. }
{\it
Let $a$ and $b$ be real numbers with $a<b$, and let $f$ be a continuous, real valued function on $[a,b]$. Then there exists $c,d\in[a,b]$ such that $f(c)\leq f(x) \leq f(d)$ for all $x\in[a,b]$.
}

{\it Proof. }
We show only the existence of $d$. By the boundedness theorem $f([a,b])$ is bounded above; let $l$ be the least upper bound of $f([a,b])$. Suppose, for a contradiction, that there is no $d\in[a,b]$ such that $f(d)=l$. Then the function $$ g(x) = \frac{1}{l-f(x)} $$
is well defined and continuous on $[a,b]$. Since $l$ is the least upper bound of $f([a,b])$, for any positive real number $M$ we can find $\alpha\in[a,b]$ such that $f(\alpha)>l-\frac{1}{M}$, then
$$ M < \frac{1}{l-f(\alpha)} \textrm{.}$$
So $g$ is unbounded on $[a,b]$. But by the boundedness theorem $g$ is bounded on $[a,b]$. This contradiction finishes the proof.
%%%%%
%%%%%
\end{document}
