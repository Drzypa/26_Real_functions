\documentclass[12pt]{article}
\usepackage{pmmeta}
\pmcanonicalname{DerivativeOfInverseFunction}
\pmcreated{2015-02-21 16:02:46}
\pmmodified{2015-02-21 16:02:46}
\pmowner{pahio}{2872}
\pmmodifier{pahio}{2872}
\pmtitle{derivative of inverse function}
\pmrecord{12}{39359}
\pmprivacy{1}
\pmauthor{pahio}{2872}
\pmtype{Theorem}
\pmcomment{trigger rebuild}
\pmclassification{msc}{26A24}
\pmrelated{InverseFunctionTheorem}
\pmrelated{Derivative2}
\pmrelated{DerivativeOfTheNaturalLogarithmFunction}
\pmrelated{CyclometricFunctions}
\pmrelated{SquareRoot}
\pmrelated{LimitExamples}
\pmrelated{IntegrationOfSqrtx21}

\endmetadata

% this is the default PlanetMath preamble.  as your knowledge
% of TeX increases, you will probably want to edit this, but
% it should be fine as is for beginners.

% almost certainly you want these
\usepackage{amssymb}
\usepackage{amsmath}
\usepackage{amsfonts}

% used for TeXing text within eps files
%\usepackage{psfrag}
% need this for including graphics (\includegraphics)
%\usepackage{graphicx}
% for neatly defining theorems and propositions
 \usepackage{amsthm}
% making logically defined graphics
%%%\usepackage{xypic}

% there are many more packages, add them here as you need them

% define commands here

\theoremstyle{definition}
\newtheorem*{thmplain}{Theorem}

\begin{document}
\textbf{Theorem.}\, If the real function $f$ has an inverse function $f_\leftarrow$ and the derivative of $f$ at the point \, 
$x = f_\leftarrow(y)$\; is distinct from zero, then $f_\leftarrow$ is also differentiable at the point $y$ and
\begin{equation}
\label{d}     
   f_\leftarrow'(y) = \frac{1}{f'(x)}.
\end{equation}
That is, the derivatives of a function and its inverse function are inverse numbers of each other, provided that they have been taken at the points which correspond to each other.\\

\{it Proof.}
Now we have
      $$f(f_\leftarrow(y)) = f(x) =y.$$
The derivatives of both sides must be equal:
$$\frac{d}{dy}\left[f(f_\leftarrow(y))\right] = \frac{d}{dy}y$$
Using the chain rule we get
$$f'(f_\leftarrow(y))\cdot f_\leftarrow'(y) = 1,$$
whence
$$f_\leftarrow'(y) = \frac{1}{f'(f_\leftarrow(y))}.$$
This is same as the asserted \PMlinkescapetext{equation} (1).\\


\textbf{Examples.}\; For simplicity, we express here the functions by symbols $y$ and the inverse functions by $x$.
\begin{enumerate}
\item $y = \tan{x}$,\; $x = \arctan{y}$;\;
 $\frac{dx}{dy} = \frac{1}{\frac{dy}{dx}} = \frac{1}{1+\tan^2{x}} = 
 \frac{1}{1+y^2}$
\item $y = \sin{x}$,\; $x = \arcsin{y}$;\; 
 $\frac{dx}{dy} = \frac{1}{\frac{dy}{dx}} = \frac{1}{\cos{x}} = 
 \frac{1}{+\sqrt{1-\sin^2{x}}} = +\frac{1}{\sqrt{1-y^2}}$
\item $y = x^2$,\,\; $x = \pm\sqrt{y}$;\;
$\frac{dx}{dy} = \frac{1}{\frac{dy}{dx}} = \frac{1}{2x} =
\frac{1}{\pm2\sqrt{y}}$
\end{enumerate}
If the variable symbol $y$ in those results is changed to $x$, the results can be written
$$\frac{d}{dx}\arctan{x} = \frac{1}{1+x^2},\qquad
\frac{d}{dx}\arcsin{x} = \frac{1}{\sqrt{1-x^2}},\qquad
\frac{d}{dx}\sqrt{x} = \frac{1}{2\sqrt{x}}.$$

%%%%%
%%%%%
\end{document}
