\documentclass[12pt]{article}
\usepackage{pmmeta}
\pmcanonicalname{TestingForContinuityViaBasicOpenSets}
\pmcreated{2013-03-22 19:08:55}
\pmmodified{2013-03-22 19:08:55}
\pmowner{CWoo}{3771}
\pmmodifier{CWoo}{3771}
\pmtitle{testing for continuity via basic open sets}
\pmrecord{4}{42052}
\pmprivacy{1}
\pmauthor{CWoo}{3771}
\pmtype{Result}
\pmcomment{trigger rebuild}
\pmclassification{msc}{26A15}
\pmclassification{msc}{54C05}

\usepackage{amssymb,amscd}
\usepackage{amsmath}
\usepackage{amsfonts}
\usepackage{mathrsfs}

% used for TeXing text within eps files
%\usepackage{psfrag}
% need this for including graphics (\includegraphics)
%\usepackage{graphicx}
% for neatly defining theorems and propositions
\usepackage{amsthm}
% making logically defined graphics
%%\usepackage{xypic}
\usepackage{pst-plot}

% define commands here
\newcommand*{\abs}[1]{\left\lvert #1\right\rvert}
\newtheorem{prop}{Proposition}
\newtheorem{thm}{Theorem}
\newtheorem{ex}{Example}
\newcommand{\real}{\mathbb{R}}
\newcommand{\pdiff}[2]{\frac{\partial #1}{\partial #2}}
\newcommand{\mpdiff}[3]{\frac{\partial^#1 #2}{\partial #3^#1}}
\begin{document}
\begin{prop} Let $X,Y$ be topological spaces, and $f:X\to Y$ a function.  The following are equivalent:
\begin{enumerate}
\item $f$ is continuous;
\item $f^{-1}(U)$ is open for any $U$ in a basis ($U$ called a basic open set) for the topology of $Y$;
\item $f^{-1}(U)$ is open for any $U$ is a subbasis for the topology of $Y$.
\end{enumerate}
\end{prop}

\begin{proof}
First, note that $(1)\Rightarrow (2)\Rightarrow (3)$, since every basic open set is open, and every element in a subbasis is in the basis it generates.  We next prove $(3)\Rightarrow (2)\Rightarrow (1)$.
\begin{itemize}
\item $(2)\Rightarrow (1)$.  Suppose $\mathcal{B}$ is a basis for the topology of $Y$.  Let $U$ be an open set in $Y$.  Then $U$ is the union of elements in $\mathcal{B}$.  In other words, $$U=\bigcup \lbrace U_i \in \mathcal{B} \mid i\in I\rbrace,$$ for some index set $I$.  So 
\begin{eqnarray*}
f^{-1}(U) &=& f^{-1}(\bigcup \lbrace U_i \in \mathcal{B} \mid i\in I\rbrace ) \\
&=& \bigcup \lbrace f^{-1}(U_i) \mid i \in I\rbrace.
\end{eqnarray*}
By assumption, each $f^{-1}(U_i)$ is open, so is their union $f^{-1}(U)$.
\item $(3)\Rightarrow (2)$.  Suppose now that $\mathcal{S}$ is a subbasis, which generates the basis $\mathcal{B}$ for the topology of $Y$.  If $U$ is a basic open set, then $$U=\bigcap_{i=1}^n U_i,$$ where each $U_i \in \mathcal{S}$.  Then 
\begin{eqnarray*}
f^{-1}(U) &=& f^{-1}(\bigcap_{i=1}^n U_i ) \\
&=& \bigcap_{i=1}^n f^{-1}(U_i).
\end{eqnarray*}
By assumption, each $f^{-1}(U_i)$ is open, so is their (finite) intersection $f^{-1}(U)$.
\end{itemize}
\end{proof}
%%%%%
%%%%%
\end{document}
