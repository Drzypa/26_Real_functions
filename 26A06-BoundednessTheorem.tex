\documentclass[12pt]{article}
\usepackage{pmmeta}
\pmcanonicalname{BoundednessTheorem}
\pmcreated{2013-03-22 14:29:18}
\pmmodified{2013-03-22 14:29:18}
\pmowner{classicleft}{5752}
\pmmodifier{classicleft}{5752}
\pmtitle{boundedness theorem}
\pmrecord{6}{36022}
\pmprivacy{1}
\pmauthor{classicleft}{5752}
\pmtype{Theorem}
\pmcomment{trigger rebuild}
\pmclassification{msc}{26A06}

% this is the default PlanetMath preamble.  as your knowledge
% of TeX increases, you will probably want to edit this, but
% it should be fine as is for beginners.

% almost certainly you want these
\usepackage{amssymb}
\usepackage{amsmath}
\usepackage{amsfonts}

% used for TeXing text within eps files
%\usepackage{psfrag}
% need this for including graphics (\includegraphics)
%\usepackage{graphicx}
% for neatly defining theorems and propositions
%\usepackage{amsthm}
% making logically defined graphics
%%%\usepackage{xypic}

% there are many more packages, add them here as you need them

% define commands here
\begin{document}
{\bf Boundedness Theorem. }
{\it
Let $a$ and $b$ be real numbers with $a<b$, and let $f$ be a continuous, real valued function on $[a,b]$. Then $f$ is bounded above and below on $[a,b]$.
}

{\it Proof. }
Suppose not. Then for all natural numbers $n$ we can find some $x_n \in [a,b]$ such that $|f(x_n)|>n$. The sequence $(x_n)$ is bounded, so by the Bolzano-Weierstrass theorem it has a convergent sub sequence, say $(x_{n_i})$. As $[a,b]$ is closed $(x_{n_i})$ converges to a value in $[a,b]$. By the continuity of $f$ we should have that $f(x_{n_i})$ converges, but by construction it diverges. This contradiction finishes the proof.
%%%%%
%%%%%
\end{document}
