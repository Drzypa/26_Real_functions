\documentclass[12pt]{article}
\usepackage{pmmeta}
\pmcanonicalname{GeneralizedDarbouxFunction}
\pmcreated{2013-03-22 19:18:36}
\pmmodified{2013-03-22 19:18:36}
\pmowner{joking}{16130}
\pmmodifier{joking}{16130}
\pmtitle{generalized Darboux function}
\pmrecord{5}{42247}
\pmprivacy{1}
\pmauthor{joking}{16130}
\pmtype{Definition}
\pmcomment{trigger rebuild}
\pmclassification{msc}{26A06}

% this is the default PlanetMath preamble.  as your knowledge
% of TeX increases, you will probably want to edit this, but
% it should be fine as is for beginners.

% almost certainly you want these
\usepackage{amssymb}
\usepackage{amsmath}
\usepackage{amsfonts}

% used for TeXing text within eps files
%\usepackage{psfrag}
% need this for including graphics (\includegraphics)
%\usepackage{graphicx}
% for neatly defining theorems and propositions
%\usepackage{amsthm}
% making logically defined graphics
%%%\usepackage{xypic}

% there are many more packages, add them here as you need them

% define commands here

\begin{document}
Recall that a function $f:I\to\mathbb{R}$, where $I\subseteq\mathbb{R}$ is an interval, is called a \textbf{Darboux function} if it satisfies the intermediate value theorem. This means, that if $a,b\in I$ and $f(a)\leqslant d\leqslant f(b)$ for some $d\in\mathbb{R}$, then there exists $c\in I$ such that $a\leqslant c\leqslant b$ and $f(c)=d$.

Darboux proved (see parent object) that if $f:[a,b]\to\mathbb{R}$ is differentiable then $f'$ is a Darboux function. The class of Darboux functions is very wide. It can be shown that any function $f:\mathbb{R}\to\mathbb{R}$ can be written as a sum of two Darboux functions. We wish to give more general definiton of Darboux function.

\textbf{Definition.} Let $X$, $Y$ be topological spaces. Function $f:X\to Y$ is called a \textbf{(generalized) Darboux function} if and only if whenever $C\subseteq X$ is a connected subset, then so is $f(C)\subseteq Y$.

It can be easily proved that connected subsets of intervals (in $\mathbb{R}$) are exactly intervals. Thus this definition coincides with classical definiton, when $X$ is an interval and $Y=\mathbb{R}$.

Note that every continuous map is a Darboux function.

Also the composition of Darboux functions is again a Darboux function and thus the class of all topological spaces, together with Darboux functions forms a category. The category of topological spaces and continuous maps is its subcategory.
%%%%%
%%%%%
\end{document}
