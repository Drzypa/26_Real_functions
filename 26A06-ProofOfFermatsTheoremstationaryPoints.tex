\documentclass[12pt]{article}
\usepackage{pmmeta}
\pmcanonicalname{ProofOfFermatsTheoremstationaryPoints}
\pmcreated{2013-03-22 13:45:09}
\pmmodified{2013-03-22 13:45:09}
\pmowner{paolini}{1187}
\pmmodifier{paolini}{1187}
\pmtitle{proof of Fermat's Theorem (stationary points)}
\pmrecord{5}{34452}
\pmprivacy{1}
\pmauthor{paolini}{1187}
\pmtype{Proof}
\pmcomment{trigger rebuild}
\pmclassification{msc}{26A06}

% this is the default PlanetMath preamble.  as your knowledge
% of TeX increases, you will probably want to edit this, but
% it should be fine as is for beginners.

% almost certainly you want these
\usepackage{amssymb}
\usepackage{amsmath}
\usepackage{amsfonts}

% used for TeXing text within eps files
%\usepackage{psfrag}
% need this for including graphics (\includegraphics)
%\usepackage{graphicx}
% for neatly defining theorems and propositions
%\usepackage{amsthm}
% making logically defined graphics
%%%\usepackage{xypic}

% there are many more packages, add them here as you need them

% define commands here
\begin{document}
Suppose that $x_0$ is a local maximum (a similar proof applies if $x_0$ is a local minimum). Then there exists $\delta>0$ such that $(x_0-\delta,x_0+\delta)\subset (a,b)$ and such that we have $f(x_0)\ge f(x)$ 
for all $x$ with $\vert x-x_0\vert <\delta$. Hence for $h\in (0,\delta)$ we notice that it holds
\[
  \frac{f(x_0+h) - f(x_0)}{h} \le 0.
\]
Since the limit of this ratio as $h\to 0^+$ exists and is equal to $f'(x_0)$ we conclude that $f'(x_0)\le 0$. On the other hand for $h\in (-\delta,0)$ we notice that
\[
  \frac{f(x_0+h) - f(x_0)}{h} \ge 0
\]
but again the limit as $h\to 0^+$ exists and is equal to $f'(x_0)$ so we also have $f'(x_0)\ge 0$. 

Hence we conclude that $f'(x_0)=0$.

To prove the second part of the statement (when $x_0$ is equal to $a$ or $b$), just notice that in such
points we have only one of the two estimates written above.
%%%%%
%%%%%
\end{document}
