\documentclass[12pt]{article}
\usepackage{pmmeta}
\pmcanonicalname{ApplicationOfFundamentalTheoremOfIntegralCalculus}
\pmcreated{2013-03-22 18:50:52}
\pmmodified{2013-03-22 18:50:52}
\pmowner{pahio}{2872}
\pmmodifier{pahio}{2872}
\pmtitle{application of fundamental theorem of integral calculus}
\pmrecord{6}{41656}
\pmprivacy{1}
\pmauthor{pahio}{2872}
\pmtype{Example}
\pmcomment{trigger rebuild}
\pmclassification{msc}{26A06}
\pmrelated{TrigonometricFormulasFromSeries}

\endmetadata

% this is the default PlanetMath preamble.  as your knowledge
% of TeX increases, you will probably want to edit this, but
% it should be fine as is for beginners.

% almost certainly you want these
\usepackage{amssymb}
\usepackage{amsmath}
\usepackage{amsfonts}

% used for TeXing text within eps files
%\usepackage{psfrag}
% need this for including graphics (\includegraphics)
%\usepackage{graphicx}
% for neatly defining theorems and propositions
 \usepackage{amsthm}
% making logically defined graphics
%%%\usepackage{xypic}

% there are many more packages, add them here as you need them

% define commands here

\theoremstyle{definition}
\newtheorem*{thmplain}{Theorem}

\begin{document}
We will derive the addition formulas of the sine and the cosine functions supposing known only their derivatives and the chain rule.\\

Define the function \,$F:\,\mathbb{R} \to \mathbb{R}$\, through
$$F(x) \;:=\; 
[\sin{x}\cos\alpha+\cos{x}\sin\alpha-\sin(x\!+\!\alpha)]^2+[\cos{x}\cos\alpha-\sin{x}\sin\alpha-\cos(x\!+\!\alpha)]^2$$
where $\alpha$ is, for the \PMlinkescapetext{moment}, a constant.\, The derivative of $F$ is easily calculated:\\
$F'(x) \;=\;\\
\mbox{\;\;}2[\sin{x}\cos\alpha+\cos{x}\sin\alpha-\sin(x\!+\!\alpha)][\cos{x}\cos\alpha-\sin{x}\sin\alpha-\cos(x\!+\!\alpha)]\\
+2[\cos{x}\cos\alpha-\sin{x}\sin\alpha-\cos(x\!+\!\alpha)][-\sin{x}\cos\alpha-\cos{x}\sin\alpha+\sin(x\!+\!\alpha)]$


But this expression is identically 0.\, By the fundamental theorem of integral calculus, $F$ must be a constant function.\, Since\, $F(0) = 0$,\, we have
$$F(x) \;\equiv\; 0$$
for any $x$ and naturally also for any $\alpha$.\, Because $F(x)$ is a sum of two squares, the both addends of it have to vanish identically, which yields the equalities
$$\sin{x}\cos\alpha+\cos{x}\sin\alpha-\sin(x\!+\!\alpha) \;=\; 0, \qquad
  \cos{x}\cos\alpha-\sin{x}\sin\alpha-\cos(x\!+\!\alpha) \;=\; 0.$$
These \PMlinkescapetext{contain} the \PMlinkname{addition formulas}{GoniometricFormulae}
$$\sin(x\!+\!\alpha) \;=\; \sin{x}\cos\alpha+\cos{x}\sin\alpha,$$
$$\cos(x\!+\!\alpha) \;=\; \cos{x}\cos\alpha-\sin{x}\sin\alpha.$$

%%%%%
%%%%%
\end{document}
