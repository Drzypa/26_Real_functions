\documentclass[12pt]{article}
\usepackage{pmmeta}
\pmcanonicalname{TableOfIntegrals}
\pmcreated{2013-03-22 17:34:44}
\pmmodified{2013-03-22 17:34:44}
\pmowner{CWoo}{3771}
\pmmodifier{CWoo}{3771}
\pmtitle{table of integrals}
\pmrecord{42}{39991}
\pmprivacy{1}
\pmauthor{CWoo}{3771}
\pmtype{Feature}
\pmcomment{trigger rebuild}
\pmclassification{msc}{26A42}
\pmrelated{TableOfDerivatives}
\pmrelated{ASpecialCaseOfPartialIntegration}
\pmrelated{GeneralFormulasForIntegration}
\pmrelated{AreaFunctions}
\pmrelated{TableOfLaplaceTransforms}
\pmrelated{ReductionFormulasForIntegrationOfPowers}

\endmetadata

\usepackage{amssymb,amscd}
\usepackage{amsmath}
\usepackage{amsfonts}
\usepackage{mathrsfs}
\usepackage{tabls}

% used for TeXing text within eps files
%\usepackage{psfrag}
% need this for including graphics (\includegraphics)
%\usepackage{graphicx}
% for neatly defining theorems and propositions
\usepackage{amsthm}
% making logically defined graphics
%%\usepackage{xypic}
\usepackage{pst-plot}
\usepackage{psfrag}

% define commands here
\newtheorem{prop}{Proposition}
\newtheorem{thm}{Theorem}
\newtheorem{ex}{Example}

\theoremstyle{definition}
\newtheorem{rem}{Remark}

\newcommand{\real}{\mathbb{R}}
\newcommand{\pdiff}[2]{\frac{\partial #1}{\partial #2}}
\newcommand{\mpdiff}[3]{\frac{\partial^#1 #2}{\partial #3^#1}}
\newcommand{\sech}{\operatorname{sech}}
\newcommand{\csch}{\operatorname{csch}}
\newcommand{\arccot}{\operatorname{arccot}}
\newcommand{\arcsec}{\operatorname{arcsec}}
\newcommand{\arsinh}{\operatorname{arsinh}}
\newcommand{\arcosh}{\operatorname{arcosh}}
\newcommand{\Li}{\operatorname{Li}}
\begin{document}
\PMlinkescapeword{mode}
\PMlinkescapeword{open}
\PMlinkescapeword{line}
\PMlinkescapeword{lines}
\PMlinkescapeword{code}

Below are some tables of some real-valued functions and their corresponding indefinite integrals.

\subsubsection*{\PMlinkname{Polynomials}{Polynomial} and powers}

\begin{center}
\begin{tabular}{|c|c|c|}
\hline\hline
$f(x)$ & $\displaystyle{\int f(x)\, dx}$ & derivation \\
\hline\hline
$x^n$\, for $n\ne -1$ & $\displaystyle{\frac{x^{n+1}}{n\!+\!1}}+C$ & \PMlinkname{here}{DerivativeOfXn} \\
\hline
$x^{-1}$ & $\ln|x|+C$ & \\
\hline
$|x|^n$\, for $n\ne -1$ & $\displaystyle\frac{x|x|^n}{n\!+\!1}+C$ & \\
\hline
$|x|^{-1}$ & $\displaystyle\frac{x\ln|x|}{|x|}+C$ & \\
\hline
\end{tabular}
\end{center}

\subsubsection*{Exponential and logarithmic functions}

\begin{center}
\begin{tabular}{|c|c|c|}
\hline\hline
$f(x)$ & $\displaystyle{\int f(x)\, dx}$ & derivation \\
\hline\hline
$e^x$ & $e^x+C$ & \\
\hline
$e^{kx}$\, for $k\neq 0$ & $\displaystyle\frac{e^{kx}}{k}+C$ & \\
\hline
$a^x$\, for $a>0$ & $\displaystyle\frac{a^x}{\ln{a}}+C$ & \\
\hline
$\ln{x}$ & $x\ln{x}-x+C$ & \PMlinkname{here}{ASpecialCaseOfPartialIntegration}\\
\hline
$(\ln{x})^2$ & $x[(\ln{x})^2-2\ln{x}+2]+C$ & \PMlinkname{here}{ASpecialCaseOfPartialIntegration}\\
\hline
$\displaystyle\frac{1}{\ln{x}}$ & $\Li{x}+C$ & Li \\
\hline
$\ln(\ln{x})$ & $x\ln\ln{x}-\Li{x}+C$ & \PMlinkname{here}{ASpecialCaseOfPartialIntegration}\\
\hline
\end{tabular}
\end{center}

\subsubsection*{\PMlinkname{Trigonometric functions}{Trigonometry}}

\begin{center}
\begin{tabular}{|c|c|c|}
\hline\hline
$f(x)$ & $\displaystyle{\int f(x)\, dx}$ & derivation \\
\hline\hline
$\cos{x}$ & $\sin{x}+C$ & \\
\hline 
$\sin{x}$ & $-\cos{x}+C$ & \PMlinkname{here}{DerivativesOfSinXAndCosX}\\
\hline
$\cot{x}$ & $\ln|\sin{x}|+C$ & \\
\hline
$\tan{x}$ & $-\ln|\cos{x}|+C$ & \\
\hline
$\sec{x}$ & $\ln|\sec{x}+\tan{x}|+C$ & \\
\hline
$\csc{x}$ & $-\ln|\csc{x}+\cot{x}|+C$ & \PMlinkname{here}{IntegrationOfRationalFunctionOfSineAndCosine}\\
\hline
$\displaystyle\frac{1}{\sin{x}}$ & $\displaystyle\ln\left|\tan\frac{x}{2}\right|+C$ & 
\PMlinkname{here}{IntegrationOfRationalFunctionOfSineAndCosine}\\
\hline
$\sec^2{x}$ & $\tan{x}+C$ & \\
\hline
$\csc^2{x}$ & $-\cot{x}+C$ & \\
\hline
$\sec{x}\tan{x}$ & $\sec{x}+C$ & \\
\hline
$\csc{x}\cot{x}$ & $-\csc{x}+C$ & \\
\hline
$\displaystyle\frac{1}{1+x^2}$ & $\arctan{x}+C$ & \PMlinkname{here}{DerivativeOfInverseFunction}\\
\hline
$\displaystyle\frac{1}{\sqrt{1-x^2}}$ & $\arcsin{x}+C$ & \PMlinkname{here}{DerivativeOfInverseFunction}\\
\hline
\end{tabular}
\end{center}

\subsubsection*{\PMlinkname{Hyperbolic functions}{HyperbolicFunctions}}

\begin{center}
\begin{tabular}{|c|c|c|}
\hline\hline
$f(x)$ & $\displaystyle{\int f(x)\, dx}$ & derivation \\
\hline\hline
$\cosh{x}$ & $\sinh{x}+C$ & \PMlinkname{here}{DerivativesOfHyperbolicFunctions}\\
\hline
$\sinh{x}$ & $\cosh{x}+C$ & \PMlinkname{here}{DerivativesOfHyperbolicFunctions}\\
\hline
$\tanh{x}$ & $\ln(\cosh{x})+C$ & \\
\hline
$\coth{x}$ & $\ln|\sinh{x}|+C$ & \\
\hline
$\sech^2{x}$ & $\tanh{x}+C$ & \\
\hline
$\csch^2{x}$ & $-\coth{x}+C$ & \\
\hline
$\sech{x}\tanh{x}$ & $-\sech{x}+C$ & \\
\hline
$\csch{x}\coth{x}$ & $-\csch{x}+C$ & \\
\hline
% add your function here & add its integral here \\
% \hline 
\end{tabular}
\end{center}

\subsubsection*{\PMlinkname{Cyclometric functions}{CyclometricFunctions}}

\begin{center}
\begin{tabular}{|c|c|c|}
\hline\hline
$f(x)$ & $\displaystyle{\int f(x)\, dx}$ & derivation \\
\hline\hline
$\arccos{x}$ & $x\arccos{x}-\sqrt{1-x^2}+C$ & \\
\hline
$\arcsin{x}$ & $x\arcsin{x}+\sqrt{1-x^2}+C$ & \PMlinkname{here}{ASpecialCaseOfPartialIntegration}\\
\hline
$\arccot{x}$ & $x\arccot{x}+\ln\sqrt{1+x^2}+C$ & \\
\hline
$\arctan{x}$ & $x\arctan{x}-\ln\sqrt{1+x^2}+C$ & \PMlinkname{here}{ASpecialCaseOfPartialIntegration}\\
\hline
$\arcsec{x}$ & $x\arcsec{x}-\ln(x+\sqrt{x^2-1})+C$ & \\
\hline
\end{tabular}
\end{center}

\subsubsection*{Some \PMlinkname{square roots}{SquareRoot}}

\begin{center}
\begin{tabular}{|c|c|c|}
\hline\hline
$f(x)$ & $\displaystyle{\int f(x)\, dx}$ & derivation \\
\hline\hline
$\sqrt{x}$ & $\frac{2}{3}x\sqrt{x}+C$ & \PMlinkname{here}{DerivativeOfXn}\\
\hline
$\sqrt{x^2+1}$ & $\displaystyle\frac{x}{2}\sqrt{x^2+1}+\frac{1}{2}\arsinh{x}+C$ & \PMlinkname{here}{IntegrationOfSqrtx21}\\
\hline
$\sqrt{x^2-1}$ & $\displaystyle\frac{x}{2}\sqrt{x^2-1}-\frac{1}{2}\arcosh{x}+C$ & \PMlinkname{here}{IntegrationOfSqrtx21}\\
\hline
$\displaystyle\frac{1}{\sqrt{x^2+1}}$ & $\arsinh{x}+C$ & \PMlinkname{here}{EulersSubstitutionsForIntegration} \\
\hline
$\displaystyle\frac{1}{\sqrt{x^2-1}}$ & $\arcosh{x}+C\;\; (x > 1)$ & \PMlinkname{here}{EulersSubstitutionsForIntegration}\\
\hline
\end{tabular}
\end{center}



\begin{rem}
$C$ above denotes an arbitrary constant real number; $\Li$ is the logarithmic integral.
\end{rem}

\begin{rem}
The antiderivatives may be proven by differentiation; in some cases there are also given a link to a derivation.
\end{rem}

\begin{rem}
Note that the table can only be used to compute a definite integral when the integrand is continuous on the domain of integration.  For example, note the following erroneous calculation:

\[
\int\limits_{-1}^1 |x|^{-1} \, dx=\frac{x\ln|x|}{|x|}\bigg|_{-1}^1=\frac{1\ln|1|}{|1|}-\frac{-1\ln|-1|}{|-1|}=0-0=0
\]

The above calculation is incorrect since $|x|^{-1}$ is not continuous at $x=0$.
\end{rem}

\textbf{Instructions on how to add a function and its integral}.  Open the entry in edit mode.  Using the appropriate table for your function (or make a new table if applicable), make a copy of the two lines of comment (starting with \%) in the code (within the tabular environment) and paste it immediately before the comment.  Uncomment the lines (take out the \% symbols) after completing.  Preview before saving the entry.
%%%%%
%%%%%
\end{document}
