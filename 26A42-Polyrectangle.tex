\documentclass[12pt]{article}
\usepackage{pmmeta}
\pmcanonicalname{Polyrectangle}
\pmcreated{2013-03-22 15:03:31}
\pmmodified{2013-03-22 15:03:31}
\pmowner{paolini}{1187}
\pmmodifier{paolini}{1187}
\pmtitle{polyrectangle}
\pmrecord{23}{36777}
\pmprivacy{1}
\pmauthor{paolini}{1187}
\pmtype{Definition}
\pmcomment{trigger rebuild}
\pmclassification{msc}{26A42}
\pmrelated{RiemannMultipleIntegral}
\pmdefines{Riemann sums on polyrectangles}
\pmdefines{compact rectangle}

% this is the default PlanetMath preamble.  as your knowledge
% of TeX increases, you will probably want to edit this, but
% it should be fine as is for beginners.

% almost certainly you want these
\usepackage{amssymb}
\usepackage{amsmath}
\usepackage{amsfonts}

% used for TeXing text within eps files
%\usepackage{psfrag}
% need this for including graphics (\includegraphics)
%\usepackage{graphicx}
% for neatly defining theorems and propositions
%\usepackage{amsthm}
% making logically defined graphics
%%%\usepackage{xypic}

% there are many more packages, add them here as you need them

% define commands here
\newcommand{\R}{\mathbb R}
\begin{document}
A \emph{polyrectangle} $P$ in $\R^n$ is a finite collection $P=\{R_1,\ldots,R_N\}$ of  compact rectangles $R_i\subset \R^n$ with disjoint interior. 
A \emph{compact rectangle} $R_i$ is a Cartesian product of compact intervals: $R_i=[a_1^i,b_1^i]\times \cdots \times [a_n^i,b_n^i]$ where $a_j^i<b_j^i$ (these are also called \emph{$n$-dimensional intervals}). 

The union of the compact rectangles of a polyrectangle $P$ is denoted by
\[
  \cup P := \bigcup_{R\in P} R = R_1 \cup \cdots \cup R_N.
\]
It is a compact subset of $\R^n$.

We can define the ($n$-dimensional) measure of $\cup P$ in a \PMlinkescapetext{simple} way.
If $R=[a_1,b_1]\times \cdots \times [a_n,b_n]$ is a rectangle we define the measure of $R$ as
\[
   \mathrm{meas}(R) := (b_1-a_1)\cdots (b_n-a_n)
\]
and define the measure of the polyrectangle $P$ as:
\[
  \mathrm{meas}(P) := \sum_{R\in P} \mathrm{meas}(R).
\]

Moreover if we are given a bounded function $f\colon \cup P\to\mathbb R$ we can define the \emph{upper} and \emph{lower Riemann sums} of $f$ over $\cup P$ by
\[
   S^*(f,P) := \sum_{R\in P} \mathrm{meas}(R)\sup_{x\in R} f(x),\qquad
   S_*(f,P) := \sum_{R\in P} \mathrm{meas}(R)\inf_{x\in R} f(x).
\]

Polyrectangles are then used to define the Peano Jordan measure of subsets of $\mathbb R^n$ and to define Riemann multiple integrals. 
To achieve this, it is useful to introduce the so called \emph{refinements}. The family of rectangles $R_i$ which appear in the definition~\ref{defpoly} are called a \emph{partition} of $\overline{\cup P}$ in rectangles.
It is clear that the set $\cup P$ can be represented by different polyrectangles. For example any rectangle $R$ can be split in $2^n$ smaller rectangles by dividing in two parts each of the $n$ intervals defining $R$. 
We claim that given two polyrectangles $P$ and $Q$ there exists a polyrectangle $S$ such that $(\cup P)\cup (\cup Q) \subset \cup S$ and such that given any rectangle $R$ in $P$ or $Q$, $R$ is the union of rectangles in $S$.
%%%%%
%%%%%
\end{document}
