\documentclass[12pt]{article}
\usepackage{pmmeta}
\pmcanonicalname{ExistenceOfNthRoot}
\pmcreated{2013-03-22 15:52:15}
\pmmodified{2013-03-22 15:52:15}
\pmowner{Wkbj79}{1863}
\pmmodifier{Wkbj79}{1863}
\pmtitle{existence of $n$th root}
\pmrecord{21}{37867}
\pmprivacy{1}
\pmauthor{Wkbj79}{1863}
\pmtype{Theorem}
\pmcomment{trigger rebuild}
\pmclassification{msc}{26C10}
\pmclassification{msc}{26A06}
\pmclassification{msc}{12D99}
\pmrelated{ExistenceOfNthRoot}

\endmetadata

\usepackage{amssymb}
\usepackage{amsmath}
\usepackage{amsfonts}

\usepackage{psfrag}
\usepackage{graphicx}
\usepackage{amsthm}
%%\usepackage{xypic}

\newtheorem*{thm*}{Theorem}
\begin{document}
\begin{thm*}
If $a \in \mathbb{R}$ with $a>0$ and $n$ is a positive integer, then there exists a unique positive real number $u$ such that $u^n=a$.
\end{thm*}

\begin{proof}
The statement is clearly true for $n=1$ (let $u=a$).  Thus, it will be assumed that $n>1$.

Define $p \colon \mathbb{R} \to \mathbb{R}$ by $p(x)=x^n-a$.  Note that a positive real root of $p(x)$ corresponds to a positive real number $u$ such that $u^n=a$.

If $a=1$, then $p(1)=1^n-1=0$, in which case the existence of $u$ has been established.

Note that $p(x)$ is a polynomial function and thus is continuous.  If $a<1$, then $p(1)=1^n-a>1-1=0$.  If $a>1$, then $p(a)=a^n-a=a(a^{n-1}-1)>0$.  Note also that $p(0)=0^n-a=-a<0$.  Thus, if $a \neq 1$, then the intermediate value theorem can be applied to yield the existence of $u$.

For uniqueness, note that the function $p(x)$ is strictly increasing on the interval $(0, \infty)$.  It follows that $u$ as described in the statement of the theorem exists uniquely.
\end{proof}
%%%%%
%%%%%
\end{document}
