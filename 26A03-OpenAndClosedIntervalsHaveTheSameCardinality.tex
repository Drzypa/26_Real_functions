\documentclass[12pt]{article}
\usepackage{pmmeta}
\pmcanonicalname{OpenAndClosedIntervalsHaveTheSameCardinality}
\pmcreated{2013-03-22 15:43:32}
\pmmodified{2013-03-22 15:43:32}
\pmowner{mps}{409}
\pmmodifier{mps}{409}
\pmtitle{open and closed intervals have the same cardinality}
\pmrecord{8}{37675}
\pmprivacy{1}
\pmauthor{mps}{409}
\pmtype{Result}
\pmcomment{trigger rebuild}
\pmclassification{msc}{26A03}
\pmclassification{msc}{03E10}

\endmetadata

% this is the default PlanetMath preamble.  as your knowledge
% of TeX increases, you will probably want to edit this, but
% it should be fine as is for beginners.

% almost certainly you want these
\usepackage{amssymb}
\usepackage{amsmath}
\usepackage{amsfonts}

% used for TeXing text within eps files
%\usepackage{psfrag}
% need this for including graphics (\includegraphics)
%\usepackage{graphicx}
% for neatly defining theorems and propositions
\usepackage{amsthm}
% making logically defined graphics
%%%\usepackage{xypic}

% there are many more packages, add them here as you need them

% define commands here
\newtheorem*{proposition*}{Proposition}
\DeclareMathOperator{\id}{id}
\begin{document}
\PMlinkescapeword{cover}
\PMlinkescapeword{formula}
\PMlinkescapeword{argument}
\begin{proposition*}
The sets of real numbers $[0,1]$, $[0,1)$, $(0,1]$, and $(0,1)$ all have the same cardinality.
\end{proposition*}

We give two proofs of this proposition.

\begin{proof}
Define a map $f:[0,1]\to[0,1]$ by $f(x)=(x+1)/3$.  The map $f$ is strictly increasing, hence injective.  Moreover, the image of $f$ is contained in the interval $[\frac{1}{3}, \frac{2}{3}]\subsetneq (0,1)$, so the maps 
$f_r:[0,1]\to[0,1)$ and $f_o:[0,1]\to(0,1)$ obtained from $f$ by restricting the codomain are both injective.  Since the inclusions into $[0,1]$ are also injective, the \PMlinkname{Cantor-Schr\"oder-Bernstein theorem}{SchroederBernsteinTheorem} can be used to construct bijections $h_r:[0,1]\to[0,1)$ and $h_o:[0,1]\to(0,1)$.  Finally, the map $r:(0,1]\to[0,1)$ defined by $r(x)=1-x$ is a bijection.

Since having the same cardinality is an equivalence relation, all four intervals have the same cardinality.
\end{proof}

\begin{proof}
Since $[0,1]\cap\mathbb{Q}$ is countable, there is a bijection $a:\mathbb{N}\to[0,1]\cap\mathbb{Q}$.  We may select $a$ so that $a(0)=0$ and $a(1)=1$.  The map $f:[0,1]\cap\mathbb{Q}\to(0,1)\cap\mathbb{Q}$ defined by $f(x)=a(a^{-1}(x)+2)$ is a bijection because it is a composition of bijections.  A bijection $h:[0,1]\to(0,1)$ can be constructed by gluing the map $f$ to the identity map on $(0,1)\setminus\mathbb{Q}$.  The formula for $h$ is
\[
h(x)=\begin{cases}
f(x),    & x\in\mathbb{Q} \\
x,       & x\notin\mathbb{Q}.
\end{cases}
\]
The other bijections can be constructed similarly.
\end{proof}

The reasoning above can be extended to show that any two arbitrary intervals in $\mathbb{R}$ have the same cardinality.
%%%%%
%%%%%
\end{document}
