\documentclass[12pt]{article}
\usepackage{pmmeta}
\pmcanonicalname{ASpecialCaseOfPartialIntegration}
\pmcreated{2013-03-22 17:38:35}
\pmmodified{2013-03-22 17:38:35}
\pmowner{pahio}{2872}
\pmmodifier{pahio}{2872}
\pmtitle{a special case of partial integration}
\pmrecord{11}{40065}
\pmprivacy{1}
\pmauthor{pahio}{2872}
\pmtype{Feature}
\pmcomment{trigger rebuild}
\pmclassification{msc}{26A36}
%\pmkeywords{integration by parts}
\pmrelated{IntegralTables}

\endmetadata

% this is the default PlanetMath preamble.  as your knowledge
% of TeX increases, you will probably want to edit this, but
% it should be fine as is for beginners.

% almost certainly you want these
\usepackage{amssymb}
\usepackage{amsmath}
\usepackage{amsfonts}

% used for TeXing text within eps files
%\usepackage{psfrag}
% need this for including graphics (\includegraphics)
%\usepackage{graphicx}
% for neatly defining theorems and propositions
 \usepackage{amsthm}
% making logically defined graphics
%%%\usepackage{xypic}

% there are many more packages, add them here as you need them

% define commands here

\theoremstyle{definition}
\newtheorem*{thmplain}{Theorem}
\DeclareMathOperator{\arcosh}{arcosh}
\DeclareMathOperator{\Li}{Li}
\begin{document}
In determining the antiderivative of a \PMlinkname{transcendental}{AlgebraicFunction} function $U$ whose derivative $U'$ is \PMlinkname{algebraic}{AlgebraicFunction}, the result can be obtained when choosing in the formula 
             $$\int UV'\,dx = UV-\!\int VU'\,dx$$
of integration by parts \,$V' \equiv 1$;\, then one has
          $$\int U\,dx = \int U\!\cdot\!1\,dx\; =\; U\!\cdot\!x-\!\int x\!\cdot\!U'\,dx.$$ 
The functions $U$ in question are mainly the \PMlinkname{logarithm}{NaturalLogarithm2}, the cyclometric functions and the area functions.\\ 

\textbf{Examples.}    
\begin{enumerate}
\item $\displaystyle\int\!\ln{x}\,dx = x\ln{x}-\!\int x\!\cdot\!\frac{1}{x}\,dx =\; x\ln{x}-x+C$
\item $\displaystyle\int\!\arcsin{x}\,dx = x\arcsin{x}-\!\int\!x\!\cdot\!\frac{1}{\sqrt{1\!-\!x^2}}\,dx = 
x\arcsin{x}+\frac{1}{2}\int\!\frac{-2x}{\sqrt{1\!-\!x^2}}\,dx\\ = x\arcsin{x}+\sqrt{1\!-\!x^2}+C$
\item $\displaystyle\int\!\arctan{x}\,dx = x\arctan{x}-\!\int\!x\!\cdot\!\frac{1}{1\!+\!x^2}\,dx = 
x\arctan{x}-\!\frac{1}{2}\int\!\frac{2x}{1\!+\!x^2}\,dx\\ = x\arctan{x}-\frac{1}{2}\ln(1\!+\!x^2)+C
=\; x\arctan{x}-\ln\sqrt{1\!+\!x^2}+C$
\item $\displaystyle\int\!\arcosh{x}\,dx = x\arcosh{x}-\!\int\!x\!\cdot\!\frac{1}{\sqrt{x^2\!-\!1}}\,dx \\=
x\arcosh{x}-\sqrt{x^2\!-\!1}+C$
\end{enumerate}

The choice\, $V' \equiv 1$\, works as well in such cases as\, $\int(\ln{x})^2\,dx$\, and\, $\int\ln(\ln{x})\,dx$,\, giving respectively\, $x((\ln{x})^2-2\ln{x}+2)+C$\, and\, $x\ln(\ln{x})-\Li{x}+C$ (see logarithmic integral).  Also\, 
$\int(\arcsin{x})^2\,dx$\, \PMlinkescapetext{succeeds}, requiring two integrations by parts, and giving the result\, $x(\arcsin{x})^2+2\sqrt{1\!-\!x^2}\arcsin{x}-2x+C$.
%%%%%
%%%%%
\end{document}
