\documentclass[12pt]{article}
\usepackage{pmmeta}
\pmcanonicalname{ProofOfGeneralMeansInequality}
\pmcreated{2013-03-22 13:10:26}
\pmmodified{2013-03-22 13:10:26}
\pmowner{pbruin}{1001}
\pmmodifier{pbruin}{1001}
\pmtitle{proof of general means inequality}
\pmrecord{5}{33619}
\pmprivacy{1}
\pmauthor{pbruin}{1001}
\pmtype{Proof}
\pmcomment{trigger rebuild}
\pmclassification{msc}{26D15}
%\pmkeywords{power mean}
%\pmkeywords{inequality}
\pmrelated{ArithmeticMean}
\pmrelated{GeometricMean}
\pmrelated{HarmonicMean}
\pmrelated{RootMeanSquare3}
\pmrelated{PowerMean}
\pmrelated{WeightedPowerMean}
\pmrelated{ArithmeticGeometricMeansInequality}
\pmrelated{JensensInequality}

% this is the default PlanetMath preamble.  as your knowledge
% of TeX increases, you will probably want to edit this, but
% it should be fine as is for beginners.

% almost certainly you want these
\usepackage{amssymb}
\usepackage{amsmath}
\usepackage{amsfonts}

% used for TeXing text within eps files
%\usepackage{psfrag}
% need this for including graphics (\includegraphics)
%\usepackage{graphicx}
% for neatly defining theorems and propositions
%\usepackage{amsthm}
% making logically defined graphics
%%%\usepackage{xypic}

% there are many more packages, add them here as you need them

% define commands here
\begin{document}
Let $w_1$, $w_2$, \dots, $w_n$ be positive real numbers such that
$w_1+w_2+\cdots+w_n=1$.  For any real number $r\ne0$, the
weighted power mean of degree $r$ of $n$ positive real numbers $x_1$,
$x_2$, \dots, $x_n$ (with respect to the weights $w_1$, \dots,
$w_n$) is defined as
$$
M_w^r(x_1,x_2,\ldots,x_n)=(w_1x_1^r+w_2x_2^r+\cdots+w_nx_n^r)^{1/r}.
$$
The definition is extended to the case $r=0$ by taking the limit
$r\to 0$; this yields the weighted geometric mean
$$
M_w^0(x_1,x_2,\ldots,x_n)=x_1^{w_1}x_2^{w_2}\ldots x_n^{w_n}
$$
(see derivation of zeroth weighted power mean).  We will prove the
weighted power means inequality, which states that for any two real
numbers $r<s$, the weighted power means of orders $r$ and $s$ 
of $n$ positive real numbers $x_1$, $x_2$, \dots, $x_n$ satisfy the 
inequality
$$
M_w^r(x_1,x_2,\ldots,x_n)\le M_w^s(x_1,x_2,\ldots,x_n)
$$
with equality if and only if all the $x_i$ are equal.

First, let us suppose that $r$ and $s$ are nonzero.  We
distinguish three cases for the signs of $r$ and $s$: $r<s<0$,
$r<0<s$, and $0<r<s$.  Let us consider the last case, i.e.\ assume
$r$ and $s$ are both positive; the others are similar.  We write
$t=\frac{s}{r}$ and $y_i=x_i^r$ for $1\le i\le n$; this implies
$y_i^t=x_i^s$.  Consider the function
\begin{eqnarray*}
f\colon(0,\infty)&\to&(0,\infty)\\
x&\mapsto&x^t.
\end{eqnarray*}
Since $t>1$, the second derivative of $f$ satisfies
$f''(x)=t(t-1)x^{t-2}>0$ for all $x>0$, so $f$ is a strictly
convex function.  Therefore, according to Jensen's inequality,
\begin{eqnarray*}
(w_1y_1+w_2y_2+\cdots+w_ny_n)^t&=&f(w_1y_1+w_2y_2+\cdots+w_ny_n)\\
&\le&w_1f(y_1)+w_2f(y_2)+\cdots+w_nf(y_n)\\
&=&w_1y_1^t+w_2y_2^t+\cdots+w_ny_n^t,
\end{eqnarray*}
with equality if and only if $y_1=y_2=\cdots=y_n$.  By substituting
$t=\frac{s}{r}$ and $y_i=x_i^r$ back into this inequality, we get
$$
(w_1x_1^r+w_2x_2^r+\cdots+w_nx_n^r)^{s/r}\le
w_1x_1^s+w_2x_2^s+\cdots+w_nx_n^s
$$
with equality if and only if $x_1=x_2=\cdots=x_n$.  Since $s$ is
positive, the function $x\mapsto x^{1/s}$ is strictly increasing, so
raising both sides to the power $1/s$ preserves the inequality:
$$
(w_1x_1^r+w_2x_2^r+\cdots+w_nx_n^r)^{1/r}\le
(w_1x_1^s+w_2x_2^s+\cdots+w_nx_n^s)^{1/s},
$$
which is the inequality we had to prove.  Equality holds if and only
if all the $x_i$ are equal.

If $r=0$, the inequality is still correct: $M_w^0$ is defined as
$\lim_{r\to 0}M_w^r$, and since $M_w^r\le M_w^s$ for all $r<s$ with
$r\neq 0$, the same holds for the limit $r\to 0$.  The same argument
shows that the inequality also holds for $s=0$, i.e.\ that
$M_w^r\le M_w^0$ for all $r<0$.  We conclude that for all real numbers
$r$ and $s$ such that $r<s$,
$$
M_w^r(x_1,x_2,\ldots,x_n)\le M_w^s(x_1,x_2,\ldots,x_n).
$$
%%%%%
%%%%%
\end{document}
