\documentclass[12pt]{article}
\usepackage{pmmeta}
\pmcanonicalname{ConvexityOfTangentFunction}
\pmcreated{2013-03-22 17:00:12}
\pmmodified{2013-03-22 17:00:12}
\pmowner{rspuzio}{6075}
\pmmodifier{rspuzio}{6075}
\pmtitle{convexity of tangent function}
\pmrecord{14}{39285}
\pmprivacy{1}
\pmauthor{rspuzio}{6075}
\pmtype{Result}
\pmcomment{trigger rebuild}
\pmclassification{msc}{26A09}

\endmetadata

% this is the default PlanetMath preamble.  as your knowledge
% of TeX increases, you will probably want to edit this, but
% it should be fine as is for beginners.

% almost certainly you want these
\usepackage{amssymb}
\usepackage{amsmath}
\usepackage{amsfonts}

% used for TeXing text within eps files
%\usepackage{psfrag}
% need this for including graphics (\includegraphics)
%\usepackage{graphicx}
% for neatly defining theorems and propositions
%\usepackage{amsthm}
% making logically defined graphics
%%%\usepackage{xypic}

% there are many more packages, add them here as you need them

% define commands here

\begin{document}
\PMlinkescapeword{convex}
We will show that the tangent function is convex on the interval $[0, \pi/2)$.
To do this, we will use the addition formula for the tangent and the fact that
a continuous real function $f$ is \PMlinkname{convex}{ConvexFunction} if and only if $f((x + y)/2) \le (f(x) + f(y))/2$.

We start with the observation that, if $0 \le x < 1$ and $0 \le y < 1$, then by the 
\PMlinkname{arithmetic-geometric mean inequality}{ArithmeticGeometricMeansInequality}, 
\begin{align*}
-2 x y &\ge - x^2 - y^2 \\
1 - 2 x y + x^2 y^2 &\ge 1 - x^2 - y^2 + x^2 y^2 \\
(1 - xy)^2 &\ge (1 - x^2) (1 - y^2),
\end{align*}
so
\[
{(1 - xy)^2 \over (1 - x^2) (1 - y^2)} \ge 1.
\]

Let $u$ and $v$ be two numbers in the interval $[0,\pi/4)$.  Set $x = \tan u$ and $y = \tan v$.
Then $0 \le x < 1$ and $0 \le y < 1.$  By the addition formula, we have
\begin{align*}
\tan (2u) &= {2x \over 1 - x^2} \\
\tan (u+v) &= {x + y \over 1 - x y} \\
\tan (2v) &= {2y \over 1 - y^2} .
\end{align*}
Hence,
\begin{align*}
{1 \over 2} \left( \tan (2u) + \tan (2v) \right) &= 
{x + y - x^2 y - xy^2 \over (1 - x^2) (1 - y^2)} \\ &=
{(x + y) (1 - x y) \over (1 - x^2) (1 - y^2)} \\ &=
{x + y \over 1 - x y}
{(1 - xy)^2 \over (1 - x^2) (1 - y^2)} \\ &\ge
{x + y \over 1 - x y} = \tan (u + v),
\end{align*}
so the tangent function is convex.
%%%%%
%%%%%
\end{document}
