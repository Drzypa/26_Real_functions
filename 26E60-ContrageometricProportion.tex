\documentclass[12pt]{article}
\usepackage{pmmeta}
\pmcanonicalname{ContrageometricProportion}
\pmcreated{2013-04-19 7:14:46}
\pmmodified{2013-04-19 7:14:46}
\pmowner{pahio}{2872}
\pmmodifier{pahio}{2872}
\pmtitle{contrageometric proportion}
\pmrecord{17}{42624}
\pmprivacy{1}
\pmauthor{pahio}{2872}
\pmtype{Definition}
\pmcomment{a blemish }
\pmclassification{msc}{26E60}
\pmclassification{msc}{11-00}
\pmclassification{msc}{01A20}
\pmclassification{msc}{01A17}
\pmrelated{ContraharmonicProportion}
\pmdefines{contrageometric mean}

% this is the default PlanetMath preamble.  as your knowledge
% of TeX increases, you will probably want to edit this, but
% it should be fine as is for beginners.

% almost certainly you want these
\usepackage{amssymb}
\usepackage{amsmath}
\usepackage{amsfonts}

% used for TeXing text within eps files
%\usepackage{psfrag}
% need this for including graphics (\includegraphics)
%\usepackage{graphicx}
% for neatly defining theorems and propositions
 \usepackage{amsthm}
% making logically defined graphics
%%\usepackage{xypic}

% there are many more packages, add them here as you need them

% define commands here

\theoremstyle{definition}
\newtheorem*{thmplain}{Theorem}

\begin{document}
Just as one converts the proportion equation 
$$\frac{m\!-\!x}{y\!-\!m} \;=\; \frac{x}{y}$$
defining the harmonic mean of $x$ and $y$ into the proportion equation 
$$\frac{m\!-\!x}{y\!-\!m} \;=\; \frac{y}{x}$$
defining their \PMlinkname{contraharmonic 
mean}{ContraharmonicProportion}, one also may convert the proportion 
equation 
$$\frac{m\!-\!x}{y\!-\!m} \;=\; \frac{m}{y}$$
defining the geometric mean into a new equation 
\begin{align}
\frac{m\!-\!x}{y\!-\!m} \;=\; \frac{y}{m}
\end{align}
defining the {\it contrageometric mean} $m$ of $x$ and $y$.\, Thus, the 
three positive numbers $x$, $m$, $y$ satisfying (1) are in {\it 
contrageometric proportion}.\, One integer example is 
$1,\,4,\,6$.\\
 
Solving $m$ from (1) one gets the expression
\begin{align}
m \;=\; \frac{x\!-\!y+\sqrt{(x\!-\!y)^2+4y^2}}{2} \;=:\; f(x,\,y).
\end{align}

Suppose now that\, $0 \le x \le y$.\, Using (2) we see that
$$m \;\ge\; \frac{x\!-\!y+\sqrt{0^2\!+\!4y^2}}{2} \;=\; \frac{x\!+\!y}{2} 
\;\ge\; x,$$
$$y^2\!-\!m^2 \;=\; 
\frac{-(y\!-\!x)^2+(y\!-\!x)\sqrt{(x\!-\!y)^2+4y^2}}{2} \;=\; 
\frac{(y\!-\!x)[\sqrt{(y\!-\!x)^2+4y^2}-(y\!-\!x)]}{2} \;\ge\; 0,$$
accordingly
\begin{align}
x \;\le\; f(x,\,y) \;\le\; y.
\end{align}
Thus the contrageometric mean of $x$ and $y$ also is at least equal to 
their arithmetic mean.\, We can also compare $m$ with their quadratic mean by watching the difference
$$\left(\sqrt{\frac{x^2\!+\!y^2}{2}}\right)^2-m^2 \;=\; (y\!-\!x)\left(\frac{1}{2}\sqrt{(y\!-\!x)^2+4y^2}-y\right)
\;\ge\; 0.$$
So we have
\begin{align}
\frac{x\!+\!y}{2}\;\le\; f(x,\,y) \;\le\; \sqrt{\frac{x^2\!+\!y^2}{2}}.
\end{align}
Cf. this result with the \PMlinkname{comparison of Pythagorean means}{ComparisonOfPythagoreanMeans}; there the brown curve is the graph of\, $f(x,\,1)$.

It's clear that the contrageometric mean (2) is not symmetric with 
respect to the variables $x$ and $y$, contrary to the other types of 
means in general.\, On the other hand, the contrageometric mean is, as other types of means, a first-degree homogeneous function its arguments:
\begin{align}
f(tx,\,ty) \;=\; tf(x,\,y).
\end{align}

\begin{thebibliography}{8}
\bibitem{F}{\sc Mabrouk K. Faradj}: \PMlinkexternal{{\em What mean do you mean? An exposition on means}}{http://etd.lsu.edu/docs/available/etd-07082004-091436/unrestricted/Faradj_thesis.pdf}.\, Louisiana State University (2004).
\bibitem{D}{\sc Georghe Toader \& Silvia Toader}: \PMlinkexternal{Greek means and the arithmetic-geometric mean}{http://rgmia.org/papers/monographs/Grec.pdf}.\, RGMIA (2010).
\end{thebibliography}

%%%%%
\end{document}
