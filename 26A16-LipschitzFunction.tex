\documentclass[12pt]{article}
\usepackage{pmmeta}
\pmcanonicalname{LipschitzFunction}
\pmcreated{2013-03-22 14:01:42}
\pmmodified{2013-03-22 14:01:42}
\pmowner{bwebste}{988}
\pmmodifier{bwebste}{988}
\pmtitle{Lipschitz function}
\pmrecord{12}{35054}
\pmprivacy{1}
\pmauthor{bwebste}{988}
\pmtype{Definition}
\pmcomment{trigger rebuild}
\pmclassification{msc}{26A16}
\pmdefines{Lipschitz}

\endmetadata

\usepackage{amssymb}
\usepackage{amsmath}
\usepackage{amsfonts}
\newcommand{\C}{\mathbb{C}}
\newcommand{\R}{\mathbb{R}}
\begin{document}
\PMlinkescapeword{Lipschitz}
Let $W \subseteq X \subseteq \C$ and $f\colon X\to\C$.  Then $f$ is \emph{\PMlinkescapetext{Lipschitz}} on $W$ if there exists an $M\in\R$ such that, for all $x,y\in W$, $x \neq y$
$$|f(x)-f(y)|\leq M|x-y|$$

If $a,b\in\R$ with $a<b$ and $f\colon [a,b]\to\R$ is Lipschitz on $(a,b)$, then $f$ is absolutely continuous on $[a,b]$.

\textbf{Example:} Is   

$$f(x) = \frac{1}{\sqrt{x}},~~~x \in [0,1]$$

a Lipschitz function.

We need to estimate the constant $M$. 

$$|f(x) - f(y)| = \left|\frac{1}{\sqrt{x}} - \frac{1}{\sqrt{y}} \right| = \left|\frac{\sqrt{x} - \sqrt{y}}{\sqrt{xy}} \right| = \left|\frac{x - y}{\sqrt{xy} (\sqrt{x} + \sqrt{y})} \right| = \frac{1}{|\sqrt{xy} (\sqrt{x} + \sqrt{y})|} |x-y|.$$

 
It follows that

$$ M = \frac{1}{|\sqrt{xy} (\sqrt{x} + \sqrt{y})|} $$

and $f(x)$ is not Lipschitz at $x=0$.

%%%%%
%%%%%
\end{document}
