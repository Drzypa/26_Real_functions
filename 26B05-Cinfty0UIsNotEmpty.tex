\documentclass[12pt]{article}
\usepackage{pmmeta}
\pmcanonicalname{Cinfty0UIsNotEmpty}
\pmcreated{2013-03-22 13:43:57}
\pmmodified{2013-03-22 13:43:57}
\pmowner{matte}{1858}
\pmmodifier{matte}{1858}
\pmtitle{$C^\infty_0(U)$ is not empty}
\pmrecord{17}{34422}
\pmprivacy{1}
\pmauthor{matte}{1858}
\pmtype{Theorem}
\pmcomment{trigger rebuild}
\pmclassification{msc}{26B05}

\endmetadata

% this is the default PlanetMath preamble.  as your knowledge
% of TeX increases, you will probably want to edit this, but
% it should be fine as is for beginners.

% almost certainly you want these
\usepackage{amssymb}
\usepackage{amsmath}
\usepackage{amsfonts}

% used for TeXing text within eps files
%\usepackage{psfrag}
% need this for including graphics (\includegraphics)
%\usepackage{graphicx}
% for neatly defining theorems and propositions
%\usepackage{amsthm}
% making logically defined graphics
%%%\usepackage{xypic}

% there are many more packages, add them here as you need them

% define commands here

\newcommand{\sR}[0]{\mathbb{R}}
\newcommand{\sC}[0]{\mathbb{C}}
\newcommand{\sN}[0]{\mathbb{N}}
\newcommand{\sZ}[0]{\mathbb{Z}}

% The below lines should work as the command
% \renewcommand{\bibname}{References}
% without creating havoc when rendering an entry in 
% the page-image mode.
\makeatletter
\@ifundefined{bibname}{}{\renewcommand{\bibname}{References}}
\makeatother

\newcommand*{\norm}[1]{\lVert #1 \rVert}
\newcommand*{\abs}[1]{| #1 |}
\begin{document}
\PMlinkescapeword{basis}
\PMlinkescapeword{contains}
\PMlinkescapeword{support}

{\bf Theorem.} If $U$ is a non-empty open set in $\sR^n$, then the set of
smooth functions with compact support $C^\infty_0(U)$ is non-trivial (that is, it contains functions other than the zero function). 

\emph{Remark.} This theorem may seem to be obvious at first sight. 
A way to notice that it is not so obvious, is to formulate it for
analytic functions with compact support: in that case, the result
does not hold; in fact, there are no nonconstant analytic
functions with compact support at all. 
One important consequence of this theorem is the existence of partitions
of unity.

\emph{Proof of the theorem:}
Let us first prove this for $n=1$:
If $a<b$ be real numbers, then there exists a 
smooth non-negative function $f:\sR \to \sR$, whose \PMlinkname{support}{SupportOfFunction} is the 
compact set $[a,b]$. 

To see this, let $\phi\colon \sR \to \sR$ be the function
 defined on \PMlinkname{this page}{InfinitelyDifferentiableFunctionThatIsNotAnalytic},
and let
$$ 
  f(x) = \phi(x-a) \phi(b-x).
$$
Since $\phi$ is smooth, it follows that $f$ is smooth. Also, from the
definition of $\phi$, we see that
$\phi(x-a)=0$ precisely when $x\le a$, and 
$\phi(b-x)=0$ precisely when $x\ge b$. 
Thus the support of $f$ is indeed $[a,b]$. 

Since $U$ is non-empty and
open there exists an $x\in U$ and $\varepsilon>0$ such that
$B_\varepsilon(x)\subseteq U$. Let $f$ be smooth function
such that $\operatorname{supp} f =[-\varepsilon/2,\varepsilon/2]$, and
let
$$
 h(z) = f(\Vert x-z \Vert^2).
$$
Since $\lVert\cdot \rVert^2$ (Euclidean norm) is smooth, the claim follows. 
$\Box$
%%%%%
%%%%%
\end{document}
