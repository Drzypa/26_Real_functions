\documentclass[12pt]{article}
\usepackage{pmmeta}
\pmcanonicalname{ExampleOfChangingVariable}
\pmcreated{2013-03-22 18:45:49}
\pmmodified{2013-03-22 18:45:49}
\pmowner{pahio}{2872}
\pmmodifier{pahio}{2872}
\pmtitle{example of changing variable}
\pmrecord{5}{41544}
\pmprivacy{1}
\pmauthor{pahio}{2872}
\pmtype{Example}
\pmcomment{trigger rebuild}
\pmclassification{msc}{26A06}
%\pmkeywords{change of variable}
\pmrelated{UsingResidueTheoremNearBranchPoint}
\pmrelated{MethodsOfEvaluatingImproperIntegrals}

\endmetadata

% this is the default PlanetMath preamble.  as your knowledge
% of TeX increases, you will probably want to edit this, but
% it should be fine as is for beginners.

% almost certainly you want these
\usepackage{amssymb}
\usepackage{amsmath}
\usepackage{amsfonts}

% used for TeXing text within eps files
%\usepackage{psfrag}
% need this for including graphics (\includegraphics)
%\usepackage{graphicx}
% for neatly defining theorems and propositions
 \usepackage{amsthm}
% making logically defined graphics
%%%\usepackage{xypic}

% there are many more packages, add them here as you need them

% define commands here

\theoremstyle{definition}
\newtheorem*{thmplain}{Theorem}

\begin{document}
If one performs in the improper integral
\begin{align}
I \;:=\; \int_{-\infty}^\infty\frac{e^{kx}}{1\!+\!e^x}\,dx \qquad (0 < k < 1)
\end{align}
the \PMlinkname{change of variable}{ChangeOfVariableInDefiniteIntegral}
$$x \;=\; -\ln{t}, \quad dx = -\frac{dt}{t},$$
the new lower limit becomes $\infty$ and the new upper limit 0; hence one obtains
$$I \;=\; -\int_\infty^0\frac{e^{-k\ln{t}}dt}{(1\!+\!e^{-\ln{t}})t} \;=\; \int_0^\infty\frac{t^{-k}}{t\!+\!1}\,dt.$$
Thus one has recurred $I$ to the integral
\begin{align}
\int_0^\infty\frac{x^{-k}}{x\!+\!1}\,dx,
\end{align}
the value of which has been determined in the entry using residue theorem near branch point.\, Accordingly, we may write the result
$$\int_{-\infty}^\infty\frac{e^{kx}}{1\!+\!e^x}\,dx \;=\; \frac{\pi}{\sin{\pi k}}.$$\\


Calculating the integral (1) directly is quite laborious:\, one has to use Cauchy residue theorem to the integral
$$\oint_c\frac{e^{kz}}{1\!+\!e^z}\,dz$$
about the perimetre $c$ of the rectangle 
$$-a \,\leqq\, \mbox{Re}\,z \,\leqq\, a, \quad 0 \,\leqq\, \mbox{Im}\,z \,\leqq\, 2\pi$$
and then to let\, $a \to \infty$ (one cannot use the same half-disk as in determining the integral (2)).\, As for using the \PMlinkname{method}{MethodsOfEvaluatingImproperIntegrals} of differentiation under the integral sign or taking Laplace transform with respect to $k$ yields a more complicated integral.


%%%%%
%%%%%
\end{document}
