\documentclass[12pt]{article}
\usepackage{pmmeta}
\pmcanonicalname{LogarithmicallyConvexFunction}
\pmcreated{2013-03-22 14:13:33}
\pmmodified{2013-03-22 14:13:33}
\pmowner{jirka}{4157}
\pmmodifier{jirka}{4157}
\pmtitle{logarithmically convex function}
\pmrecord{7}{35664}
\pmprivacy{1}
\pmauthor{jirka}{4157}
\pmtype{Definition}
\pmcomment{trigger rebuild}
\pmclassification{msc}{26A51}
\pmsynonym{logarithmically convex}{LogarithmicallyConvexFunction}
\pmsynonym{log-convex function}{LogarithmicallyConvexFunction}
\pmsynonym{log-convex}{LogarithmicallyConvexFunction}
\pmsynonym{log convex function}{LogarithmicallyConvexFunction}
\pmsynonym{log convex}{LogarithmicallyConvexFunction}
\pmrelated{ConvexFunction}
\pmrelated{BohrMollerupTheorem}
\pmrelated{HadamardThreeCircleTheorem}

% this is the default PlanetMath preamble.  as your knowledge
% of TeX increases, you will probably want to edit this, but
% it should be fine as is for beginners.

% almost certainly you want these
\usepackage{amssymb}
\usepackage{amsmath}
\usepackage{amsfonts}

% used for TeXing text within eps files
%\usepackage{psfrag}
% need this for including graphics (\includegraphics)
%\usepackage{graphicx}
% for neatly defining theorems and propositions
\usepackage{amsthm}
% making logically defined graphics
%%%\usepackage{xypic}

% there are many more packages, add them here as you need them

% define commands here
\theoremstyle{theorem}
\newtheorem*{thm}{Theorem}
\newtheorem*{lemma}{Lemma}
\newtheorem*{conj}{Conjecture}
\newtheorem*{cor}{Corollary}
\newtheorem*{example}{Example}
\theoremstyle{definition}
\newtheorem*{defn}{Definition}
\begin{document}
\begin{defn}
A function $f\colon [a,b] \to {\mathbb{R}}$ such that $f(x) > 0$ for all $x$ is said
to be {\em logarithmically convex} if $\log f(x)$ is a convex function.
\end{defn}

It is easy to see that a logarithmically convex function is a convex function, but the converse is not true.  For example $f(x) = x^2$ is a convex function, but $\log f(x) = \log x^2 = 2 \log x$ is not a convex function and thus $f(x) = x^2$ is not logarithmically convex.  On the other hand $e^{x^2}$ is logarithmically convex since $\log e^{x^2} = x^2$ is convex.  A less trivial example of a logarithmically convex function is the gamma function, if restricted to the positive reals.

The definition is easily extended to functions $f\colon U \subset {\mathbb{R}} \to {\mathbb{R}}$, for any connected set $U$ (where still we have $f > 0$), in the obvious way.  Such a function is logarithmically convex if it is logarithmically convex on all intervals
$[a,b] \subset U$.

\begin{thebibliography}{9}
\bibitem{Conway:complexI}
John~B. Conway.
{\em \PMlinkescapetext{Functions of One Complex Variable I}}.
Springer-Verlag, New York, New York, 1978.
\end{thebibliography}
%%%%%
%%%%%
\end{document}
