\documentclass[12pt]{article}
\usepackage{pmmeta}
\pmcanonicalname{SingularFunction}
\pmcreated{2013-03-22 14:08:05}
\pmmodified{2013-03-22 14:08:05}
\pmowner{jirka}{4157}
\pmmodifier{jirka}{4157}
\pmtitle{singular function}
\pmrecord{11}{35548}
\pmprivacy{1}
\pmauthor{jirka}{4157}
\pmtype{Definition}
\pmcomment{trigger rebuild}
\pmclassification{msc}{26A30}
\pmsynonym{purely singular function}{SingularFunction}
\pmrelated{AbsolutelyContinuousFunction2}
\pmrelated{CantorFunction}
\pmrelated{CantorSet}
\pmrelated{AbsolutelyContinuousFunction2}
\pmdefines{singular function}

% this is the default PlanetMath preamble.  as your knowledge
% of TeX increases, you will probably want to edit this, but
% it should be fine as is for beginners.

% almost certainly you want these
\usepackage{amssymb}
\usepackage{amsmath}
\usepackage{amsfonts}

% used for TeXing text within eps files
%\usepackage{psfrag}
% need this for including graphics (\includegraphics)
%\usepackage{graphicx}
% for neatly defining theorems and propositions
\usepackage{amsthm}
% making logically defined graphics
%%%\usepackage{xypic}

% there are many more packages, add them here as you need them

% define commands here
\theoremstyle{theorem}
\newtheorem*{thm}{Theorem}
\newtheorem*{lemma}{Lemma}
\newtheorem*{conj}{Conjecture}
\newtheorem*{cor}{Corollary}
\newtheorem*{example}{Example}
\newtheorem*{prop}{Proposition}
\theoremstyle{definition}
\newtheorem*{defn}{Definition}
\begin{document}
\begin{defn}
A monotone, non-constant, function $f\colon [a,b] \to {\mathbb{R}}$ is said to be a
{\em singular function} (or a {\em purely singular function}) if $f'(x) = 0$ almost everywhere.
\end{defn}

It is easy to see that a singular function cannot be 
\PMlinkname{absolutely continuous}{AbsolutelyContinuousFunction2}:
If an absolutely continuous function $f \colon [a,b] \to \mathbb{R}$ satisfies $f'(x)=0$ almost everywhere, then it must be constant. 

An example of such a function is the famous Cantor function.
While this is not a strictly increasing function, there also do exist singular functions which are in fact strictly increasing, and even more amazingly functions that are quasisymmetric (see attached example).

\begin{thm}
Any monotone increasing function can be written as a sum of an absolutely continuous function and a singular function.
\end{thm}

\begin{thebibliography}{9}
\bibitem{royden}
H.\@ L.\@ Royden.  \emph{\PMlinkescapetext{Real Analysis}}.  Prentice-Hall, Englewood Cliffs, New Jersey, 1988
\end{thebibliography}
%%%%%
%%%%%
\end{document}
