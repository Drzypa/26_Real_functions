\documentclass[12pt]{article}
\usepackage{pmmeta}
\pmcanonicalname{IntegrationTechniques}
\pmcreated{2013-03-22 17:50:13}
\pmmodified{2013-03-22 17:50:13}
\pmowner{Wkbj79}{1863}
\pmmodifier{Wkbj79}{1863}
\pmtitle{integration techniques}
\pmrecord{7}{40308}
\pmprivacy{1}
\pmauthor{Wkbj79}{1863}
\pmtype{Topic}
\pmcomment{trigger rebuild}
\pmclassification{msc}{26A36}
\pmsynonym{techniques of integration}{IntegrationTechniques}
\pmrelated{AntiderivativeOfRationalFunction}

\usepackage{amssymb}
\usepackage{amsmath}
\usepackage{amsfonts}
\usepackage{pstricks}
\usepackage{psfrag}
\usepackage{graphicx}
\usepackage{amsthm}
%%\usepackage{xypic}

\begin{document}
The following is an \PMlinkescapetext{index} of techniques of integration:

\begin{itemize}
\item general formulas for integration
\item integration by substitution
\item integration by parts
\item \PMlinkname{partial fraction decomposition}{ALectureOnThePartialFractionDecompositionMethod}
\item integration of rational function of sine and cosine
\item integration of differential binomial
\item Euler's substitutions for integration
\item integration of fraction power expressions
\item a special case of partial integration
\end{itemize}
%%%%%
%%%%%
\end{document}
