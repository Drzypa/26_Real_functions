\documentclass[12pt]{article}
\usepackage{pmmeta}
\pmcanonicalname{AreaUnderGaussianCurve}
\pmcreated{2013-03-22 15:16:36}
\pmmodified{2013-03-22 15:16:36}
\pmowner{pahio}{2872}
\pmmodifier{pahio}{2872}
\pmtitle{area under Gaussian curve}
\pmrecord{22}{37065}
\pmprivacy{1}
\pmauthor{pahio}{2872}
\pmtype{Theorem}
\pmcomment{trigger rebuild}
\pmclassification{msc}{26B15}
\pmclassification{msc}{26A36}
\pmsynonym{Gaussian integral}{AreaUnderGaussianCurve}
\pmsynonym{area under the bell curve}{AreaUnderGaussianCurve}
%\pmkeywords{double integral}
%\pmkeywords{polar coordinates}
\pmrelated{SubstitutionNotation}
\pmrelated{ProofThatNormalDistributionIsADistribution}
\pmrelated{Distribution}
\pmrelated{ErrorFunction}
\pmrelated{EvaluatingTheGammaFunctionAt12}
\pmrelated{NormalRandomVariable}
\pmrelated{TableOfProbabilitiesOfStandardNormalDistribution}
\pmrelated{ApplyingGeneratingFunction}
\pmrelated{FresnelFormulas}

\endmetadata

% this is the default PlanetMath preamble.  as your knowledge
% of TeX increases, you will probably want to edit this, but
% it should be fine as is for beginners.

% almost certainly you want these
\usepackage{amssymb}
\usepackage{amsmath}
\usepackage{amsfonts}

% used for TeXing text within eps files
%\usepackage{psfrag}
% need this for including graphics (\includegraphics)
\usepackage{graphicx}
% for neatly defining theorems and propositions
 \usepackage{amsthm}
% making logically defined graphics
%%%\usepackage{xypic}

% there are many more packages, add them here as you need them

% define commands here

\newcommand{\sijoitus}[2]%
{\operatornamewithlimits{\Big/}_{\!\!\!#1}^{\,#2}}

\theoremstyle{definition}
\newtheorem*{thmplain}{Theorem}
\begin{document}
\begin{thmplain}
The area between the curve\,\, $y = e^{-x^2}$\, and the $x$-axis equals $\sqrt{\pi}$,\, i.e.
    $$\int_{-\infty}^\infty e^{-x^2}\,dx = \sqrt{\pi}.$$
\end{thmplain}

\begin{figure}[!htb]
\begin{center}
\includegraphics{gaussian-curve.eps}
\end{center}
\end{figure}

{\em Proof.}\, The square of the area is

\begin{align*}
\bigg(\int_{-\infty}^\infty e^{-x^2}\,dx\bigg)^2 
& = \lim_{a\to\infty}\bigg(\int_{-a}^a e^{-x^2}\,dx\bigg)^2\\ 
& = \lim_{a\to\infty}\int_{-a}^a e^{-x^2}\,dx\,\cdot\int_{-a}^a e^{-y^2}\,dy\\
& = \lim_{a\to\infty}\int_{-a}^a \int_{-a}^a e^{-(x^2+y^2)}\,dx\,dy\\
& = \lim_{R\to\infty}\int_0^R\!\int_0^{2\pi}e^{-r^2}r\,dr\,d\varphi\\
& = \lim_{R\to\infty}2\pi\!\int_0^R e^{-r^2}r\,dr\\
& = -\pi\!\lim_{R\to\infty}\!\sijoitus{0}{\quad\,\,R}e^{-r^2}\\
& = \pi\!\lim_{R\to\infty}(1-e^{-R^2}) \;=\; \pi.
\end{align*}

Here, the limit of the double integral over a square has been replaced by the limit of the double integral over a disc, because both limits are equal.\, That both limits are equal can be demonstrated by the elementary \PMlinkescapetext{estimate}
$$0 \leq \int_{-a}^a \int_{-a}^a\!e^{-(x^2 + y^2)}\,dx\,dy
- \int_0^a\!\int_0^{2\pi}\!e^{-r^2} r\,dr\,d\varphi
\leq \underbrace{e^{-a^2}}_{greatest\,value}\!\cdot\,\underbrace{(4a^2\!-\!\pi a^2)}_{area}
= (4\!-\!\pi)\!\cdot\!\frac{a^2}{e^{a^2}},$$
and\, $\frac{a^2}{e^{a^2}} \to 0$\, when\, $a \to \infty$\, (see growth of exponential function).\\

\textbf{Remark.}\, Since $e^{-x^2}$ is an even function,
\begin{align*}
\int_0^\infty e^{-x^2}\,dx=\frac{\sqrt{\pi}}{2}\:\cdot
\end{align*}
%%%%%
%%%%%
\end{document}
