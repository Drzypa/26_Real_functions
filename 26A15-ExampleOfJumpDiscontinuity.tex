\documentclass[12pt]{article}
\usepackage{pmmeta}
\pmcanonicalname{ExampleOfJumpDiscontinuity}
\pmcreated{2013-03-22 16:25:02}
\pmmodified{2013-03-22 16:25:02}
\pmowner{pahio}{2872}
\pmmodifier{pahio}{2872}
\pmtitle{example of jump discontinuity}
\pmrecord{16}{38567}
\pmprivacy{1}
\pmauthor{pahio}{2872}
\pmtype{Example}
\pmcomment{trigger rebuild}
\pmclassification{msc}{26A15}
\pmclassification{msc}{54C05}
\pmrelated{ExponentialFunction}
\pmrelated{ImproperLimits}

% this is the default PlanetMath preamble.  as your knowledge
% of TeX increases, you will probably want to edit this, but
% it should be fine as is for beginners.

% almost certainly you want these
\usepackage{amssymb}
\usepackage{amsmath}
\usepackage{amsfonts}

% used for TeXing text within eps files
%\usepackage{psfrag}
% need this for including graphics (\includegraphics)
\usepackage{graphicx}
% for neatly defining theorems and propositions
 \usepackage{amsthm}
% making logically defined graphics
%%%\usepackage{xypic}

% there are many more packages, add them here as you need them

% define commands here

\theoremstyle{definition}
\newtheorem*{thmplain}{Theorem}

\begin{document}
The \PMlinkname{elementary}{ElementaryFunction} real function
$$f\colon\,x \mapsto \frac{1}{1+e^\frac{1}{x}}$$
has a jump discontinuity at the origin, since
$$\lim_{x\to 0-}f(x) = 1\quad \mathrm{and}\quad \lim_{x\to 0+}f(x) =0.$$
Indeed, 
\begin{itemize}
\item if\, $x \to 0-$,\, then\, $\displaystyle \frac{1}{x} \to -\infty$,\; 
$\displaystyle e^\frac{1}{x} \to 0$,\; 
$\displaystyle \frac{1}{1+e^\frac{1}{x}} \to 1$; 
\item if\, $x \to 0+$,\, then\, $\displaystyle \frac{1}{x} \to \infty$,\; 
$\displaystyle e^\frac{1}{x} \to \infty$,\; 
$\displaystyle \frac{1}{1+e^\frac{1}{x}} \to 0$.
\end{itemize}
These results can be seen also from the series \PMlinkescapetext{expansions} of the function gotten by performing the divisions:\, for\, $x < 0$\, we obtain the \PMlinkname{converging}{Converge} \PMlinkname{alternating series}{LeibnizEstimateForAlternatingSeries}
\begin{align*}
1:(1+e^{\frac{1}{x}}) = \sum_{k=0}^\infty(-1)^ke^{\frac{k}{x}}
= 1-e^{\frac{1}{x}}+e^{\frac{2}{x}}-e^{\frac{3}{x}}+-\ldots
\end{align*}
and for\, $x > 0$\, the series
\begin{align*}
1:(e^{\frac{1}{x}}+1) = \sum_{k=1}^\infty(-1)^{k+1}e^{-\frac{k}{x}}
= e^{-\frac{1}{x}}-e^{-\frac{2}{x}}+e^{-\frac{3}{x}}-+\ldots
\end{align*}

\textbf{Note.}\, The derivative of the function may be written as 
$$f'(x) = \frac{1}{x^2(e^{-\frac{1}{x}}+1)(1+e^\frac{1}{x})},$$ 
and thus we have the one-sided limits\, $\displaystyle \lim_{x\to 0\pm}f'(x) = 0$ (see growth of exponential function).

\begin{figure}[!tb]
\begin{center}
\includegraphics{e1xjump.eps}
\end{center}
\caption{Graph of the function $f$ with jump discontinuity}
\end{figure}

%%%%%
%%%%%
\end{document}
