\documentclass[12pt]{article}
\usepackage{pmmeta}
\pmcanonicalname{00IsPossibleAndHasAnAnswer}
\pmcreated{2014-08-02 19:59:17}
\pmmodified{2014-08-02 19:59:17}
\pmowner{imaginary.i}{1001376}
\pmmodifier{imaginary.i}{1001376}
\pmtitle{0/0 is possible and has an answer}
\pmrecord{4}{88135}
\pmprivacy{1}
\pmauthor{imaginary.i}{1001376}
\pmtype{Result}

\endmetadata

% this is the default PlanetMath preamble.  as your knowledge
% of TeX increases, you will probably want to edit this, but
% it should be fine as is for beginners.

% almost certainly you want these
\usepackage{amssymb}
\usepackage{amsmath}
\usepackage{amsfonts}

% need this for including graphics (\includegraphics)
\usepackage{graphicx}
% for neatly defining theorems and propositions
\usepackage{amsthm}

% making logically defined graphics
%\usepackage{xypic}
% used for TeXing text within eps files
%\usepackage{psfrag}

% there are many more packages, add them here as you need them

% define commands here

\begin{document}
So, I am happy to tell you that I have solved 0/0 by means of utilizing set theory.
Set Theory is a part of math dealing with sets of objects.
We all know that 
1*0=0
2*0=0
...
30000*0=0
So that means that
Since P(R) is the whole set of real numbers
P(R)*0=0
Meaning to say, 0 divided by 0 is R, or the entire set of real numbers.
That is my solution to this mysterious expression that was yet to be solved.
\end{document}
