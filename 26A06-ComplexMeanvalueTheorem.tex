\documentclass[12pt]{article}
\usepackage{pmmeta}
\pmcanonicalname{ComplexMeanvalueTheorem}
\pmcreated{2013-03-22 13:49:02}
\pmmodified{2013-03-22 13:49:02}
\pmowner{matte}{1858}
\pmmodifier{matte}{1858}
\pmtitle{complex mean-value theorem}
\pmrecord{7}{34543}
\pmprivacy{1}
\pmauthor{matte}{1858}
\pmtype{Theorem}
\pmcomment{trigger rebuild}
\pmclassification{msc}{26A06}

\endmetadata

% this is the default PlanetMath preamble.  as your knowledge
% of TeX increases, you will probably want to edit this, but
% it should be fine as is for beginners.

% almost certainly you want these
\usepackage{amssymb}
\usepackage{amsmath}
\usepackage{amsfonts}

% used for TeXing text within eps files
%\usepackage{psfrag}
% need this for including graphics (\includegraphics)
%\usepackage{graphicx}
% for neatly defining theorems and propositions
%\usepackage{amsthm}
% making logically defined graphics
%%%\usepackage{xypic}

% there are many more packages, add them here as you need them

% define commands here

\newcommand{\sR}[0]{\mathbb{R}}
\newcommand{\sC}[0]{\mathbb{C}}
\newcommand{\sN}[0]{\mathbb{N}}
\newcommand{\sZ}[0]{\mathbb{Z}}

% The below lines should work as the command
% \renewcommand{\bibname}{References}
% without creating havoc when rendering an entry in 
% the page-image mode.
\makeatletter
\@ifundefined{bibname}{}{\renewcommand{\bibname}{References}}
\makeatother

\newcommand*{\norm}[1]{\lVert #1 \rVert}
\newcommand*{\abs}[1]{| #1 |}
\begin{document}
{\bf Theorem} \cite{evard}
 Suppose $\Omega$ is an open convex set in $\sC$,
 suppose $f$ is a holomorphic function $f:\Omega\to \sC$, and
 suppose $a,b$ are distinct points in $\Omega$.
 Then there exist points $u,v$ on $L_{ab}$ (the straight line
 connecting $a$ and $b$ not containing the endpoints), such that
\begin{eqnarray*}
\Re\{ \frac{f(b)-f(a)}{b-a} \} = \Re\{ f'(u) \}, \\
\Im\{ \frac{f(b)-f(a)}{b-a} \} = \Im\{ f'(v) \}, 
\end{eqnarray*}
where $\Re$ and $\Im$ are the \PMlinkname{real}{RealPart} and imaginary parts of
a complex number, respectively.  
 
 \begin{thebibliography}{9}
 \bibitem{evard} J.-Cl. Evard, F. Jafari,
 \emph{A Complex Rolle's Theorem},
 American Mathematical Monthly, Vol. 99, Issue 9, (Nov. 1992), pp. 858-861.
 \end{thebibliography}
%%%%%
%%%%%
\end{document}
