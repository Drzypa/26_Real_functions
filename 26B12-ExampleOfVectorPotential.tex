\documentclass[12pt]{article}
\usepackage{pmmeta}
\pmcanonicalname{ExampleOfVectorPotential}
\pmcreated{2013-03-22 15:42:56}
\pmmodified{2013-03-22 15:42:56}
\pmowner{pahio}{2872}
\pmmodifier{pahio}{2872}
\pmtitle{example of vector potential}
\pmrecord{7}{37663}
\pmprivacy{1}
\pmauthor{pahio}{2872}
\pmtype{Example}
\pmcomment{trigger rebuild}
\pmclassification{msc}{26B12}

\endmetadata

% this is the default PlanetMath preamble.  as your knowledge
% of TeX increases, you will probably want to edit this, but
% it should be fine as is for beginners.

% almost certainly you want these
\usepackage{amssymb}
\usepackage{amsmath}
\usepackage{amsfonts}

% used for TeXing text within eps files
%\usepackage{psfrag}
% need this for including graphics (\includegraphics)
%\usepackage{graphicx}
% for neatly defining theorems and propositions
 \usepackage{amsthm}
% making logically defined graphics
%%%\usepackage{xypic}

% there are many more packages, add them here as you need them

% define commands here

\theoremstyle{definition}
\newtheorem*{thmplain}{Theorem}
\begin{document}
If the solenoidal vector \,$\vec{U} = \vec{U}(x,\,y,\,z)$\, is a homogeneous function of degree $\lambda$ ($\neq -2$),\, then it has the vector potential
\begin{align}
\vec{A} = \frac{1}{\lambda\!+\!2}\vec{U}\!\times\!\vec{r},
\end{align}
where\, $\vec{r} = x\vec{i}\!+\!y\vec{j}\!+\!z\vec{k}$\, is the position vector.

{\em Proof.}\, Using the entry nabla acting on products, we first may write
$$\nabla\times(\frac{1}{\lambda\!+\!2}\vec{U}\!\times\!\vec{r}) =
\frac{1}{\lambda\!+\!2}[(\vec{r}\cdot\nabla)\vec{U}
-(\vec{U}\cdot\nabla)\vec{r}-(\nabla\cdot\vec{U})\vec{r}
+(\nabla\cdot\vec{r})\vec{U}].$$
In the brackets the first product is, according to Euler's theorem on homogeneous functions, equal to $\lambda\vec{U}$.\, The second product can be written as\, $U_x\frac{\partial\vec{r}}{\partial x}+
U_y\frac{\partial\vec{r}}{\partial y}+U_z\frac{\partial\vec{r}}{\partial z}$, which is $U_x\vec{i}+U_y\vec{j}+U_z\vec{k}$, i.e. $\vec{U}$.\, The third product is, due to the sodenoidalness, equal to\, $0\vec{r} = \vec{0}$.\, The last product equals to $3\vec{U}$ (see the \PMlinkname{first formula}{PositionVector} for position vector).\, Thus we get the result
$$\nabla\times(\frac{1}{\lambda\!+\!2}\vec{U}\!\times\!\vec{r}) =
\frac{1}{\lambda\!+\!2}[\lambda\vec{U}-\vec{U}-\vec{0}+3\vec{U}] = \vec{U}.$$
This means that $\vec{U}$ has the vector potential (1).
%%%%%
%%%%%
\end{document}
