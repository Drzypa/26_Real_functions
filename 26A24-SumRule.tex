\documentclass[12pt]{article}
\usepackage{pmmeta}
\pmcanonicalname{SumRule}
\pmcreated{2013-03-22 12:28:09}
\pmmodified{2013-03-22 12:28:09}
\pmowner{mathcam}{2727}
\pmmodifier{mathcam}{2727}
\pmtitle{sum rule}
\pmrecord{8}{32637}
\pmprivacy{1}
\pmauthor{mathcam}{2727}
\pmtype{Theorem}
\pmcomment{trigger rebuild}
\pmclassification{msc}{26A24}
\pmrelated{Derivative}
\pmrelated{ProductRule}
\pmrelated{FixedPointsOfNormalFunctions}

\endmetadata

\usepackage{amssymb}
\usepackage{amsmath}
\usepackage{amsfonts}
\newcommand{\D}[1]{\ensuremath{\mathrm{d}#1}}
\newcommand{\DDX}{\ensuremath{\frac{\D{}}{\D{x}}}}
\begin{document}
\PMlinkescapeword{states}

The \emph{sum rule} states that

\begin{equation*}
\DDX\left[f(x)+g(x)\right] = f'(x) + g'(x)
\end{equation*}

\subsubsection*{Proof}

See the \PMlinkname{proof of the sum rule}{ProofOfSumRule}.

\subsubsection*{Examples}

\begin{eqnarray*}
\DDX(x + 1) & = & \DDX x + \DDX 1 = 1 \\
\DDX(x^2 - 3x + 2) & = & \DDX x^2 + \DDX(-3x) + \DDX(2) = 2x-3 \\
\DDX(\sin x + \cos x) & = & \DDX\sin x + \DDX\cos x = \cos x - \sin x
\end{eqnarray*}
%%%%%
%%%%%
\end{document}
