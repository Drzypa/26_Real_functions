\documentclass[12pt]{article}
\usepackage{pmmeta}
\pmcanonicalname{SmoothFunctionsWithCompactSupport}
\pmcreated{2013-03-22 13:44:00}
\pmmodified{2013-03-22 13:44:00}
\pmowner{matte}{1858}
\pmmodifier{matte}{1858}
\pmtitle{smooth functions with compact support}
\pmrecord{10}{34423}
\pmprivacy{1}
\pmauthor{matte}{1858}
\pmtype{Definition}
\pmcomment{trigger rebuild}
\pmclassification{msc}{26B05}
\pmrelated{Cn}

\endmetadata

% this is the default PlanetMath preamble.  as your knowledge
% of TeX increases, you will probably want to edit this, but
% it should be fine as is for beginners.

% almost certainly you want these
\usepackage{amssymb}
\usepackage{amsmath}
\usepackage{amsfonts}

% used for TeXing text within eps files
%\usepackage{psfrag}
% need this for including graphics (\includegraphics)
%\usepackage{graphicx}
% for neatly defining theorems and propositions
%\usepackage{amsthm}
% making logically defined graphics
%%%\usepackage{xypic}

% there are many more packages, add them here as you need them

% define commands here

\newcommand{\sR}[0]{\mathbb{R}}
\newcommand{\sC}[0]{\mathbb{C}}
\newcommand{\sN}[0]{\mathbb{N}}
\newcommand{\sZ}[0]{\mathbb{Z}}

% The below lines should work as the command
% \renewcommand{\bibname}{References}
% without creating havoc when rendering an entry in 
% the page-image mode.
\makeatletter
\@ifundefined{bibname}{}{\renewcommand{\bibname}{References}}
\makeatother

\newcommand*{\norm}[1]{\lVert #1 \rVert}
\newcommand*{\abs}[1]{| #1 |}
\begin{document}
\newcommand{\cD}[0]{\mathcal{D}}
\newcommand{\scomp}[0]{C^\infty_0}

{\bf Definition} 
Let $U$ be an open set in $\sR^n$. Then the set of 
{\bf smooth functions with compact support} (in $U$) is the set
of functions $f:\sR^n \to \sC$ which are smooth 
(i.e., $\partial^\alpha f:\sR^n\to\sC$ 
is a continuous function for all multi-indices $\alpha$)
and $\operatorname{supp} f$ is compact and contained in $U$.
This function space is denoted by $C^\infty_0(U)$.

\subsubsection{Remarks}
\begin{enumerate}
\item  A proof that $C^\infty_0(U)$ is non-trivial (that is, it contains other functions
than the zero function) can be found 
\PMlinkname{here}{Cinfty_0UIsNotEmpty}. 
\item With the usual point-wise addition and point-wise multiplication 
by a scalar, $C^\infty_0(U)$ is a vector space over the field $\sC$.
\item Suppose $U$ and $V$ are open subsets in $\sR^n$ and $U\subset V$. 
Then $C^\infty_0(U)$ is a vector subspace of $C^\infty_0(V)$. 
In particular, $C^\infty_0(U)\subset C^\infty_0(V)$. 
\end{enumerate}

It is possible to equip $\scomp(U)$ with a topology, which makes 
$\scomp(U)$ into a locally convex topological vector space. The idea is
to exhaust $U$ with compact sets. Then, for each compact set $K\subset U$, 
one defines  a topology of smooth functions on $U$ with 
support on $K$. The topology for $C_0^\infty(U)$ is the inductive
limit topology of these topologies. See e.g. \cite{rudin_fap}. 


\begin{thebibliography}{9}
 \bibitem{rudin_fap}
 W. Rudin, \emph{Functional Analysis},
 McGraw-Hill Book Company, 1973.
 \end{thebibliography}
%%%%%
%%%%%
\end{document}
