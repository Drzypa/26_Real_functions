\documentclass[12pt]{article}
\usepackage{pmmeta}
\pmcanonicalname{DifferentiableFunction}
\pmcreated{2013-03-22 12:39:10}
\pmmodified{2013-03-22 12:39:10}
\pmowner{Koro}{127}
\pmmodifier{Koro}{127}
\pmtitle{differentiable function}
\pmrecord{24}{32919}
\pmprivacy{1}
\pmauthor{Koro}{127}
\pmtype{Definition}
\pmcomment{trigger rebuild}
\pmclassification{msc}{26A24}
\pmclassification{msc}{57R35}
\pmsynonym{smooth function}{DifferentiableFunction}
\pmsynonym{differentiable mapping}{DifferentiableFunction}
\pmsynonym{differentiable map}{DifferentiableFunction}
\pmsynonym{smooth mapping}{DifferentiableFunction}
\pmsynonym{smooth map}{DifferentiableFunction}
\pmsynonym{continuously differentiable}{DifferentiableFunction}
%\pmkeywords{differentiable}
%\pmkeywords{smooth}
\pmrelated{OneSidedDerivatives}
\pmrelated{RoundFunction}
\pmrelated{ConverseTheorem}
\pmrelated{WeierstrassFunction}
\pmdefines{differentiable}
\pmdefines{smooth}

\endmetadata

% this is the default PlanetMath preamble.  as your knowledge
% of TeX increases, you will probably want to edit this, but
% it should be fine as is for beginners.

% almost certainly you want these
\usepackage{amssymb}
\usepackage{amsmath}
\usepackage{amsfonts}
\usepackage{mathrsfs}

% used for TeXing text within eps files
%\usepackage{psfrag}
% need this for including graphics (\includegraphics)
%\usepackage{graphicx}
% for neatly defining theorems and propositions
%\usepackage{amsthm}
% making logically defined graphics
%%%\usepackage{xypic}

% there are many more packages, add them here as you need them

% define commands here
\newcommand{\C}{\mathbb{C}}
\newcommand{\R}{\mathbb{R}}
\newcommand{\N}{\mathbb{N}}
\newcommand{\Z}{\mathbb{Z}}
\newcommand{\Per}{\operatorname{Per}}
\begin{document}
Let $f\colon V\to W$ be a function, where $V$ and $W$ are Banach spaces.
For $x\in V$, the function $f$ is said to be \emph{differentiable}
at $x$ if its derivative exists at that point. Differentiability at
$x\in V$ implies continuity at $x$. If $S\subset V$, then $f$ is said to
be differentiable on $S$ if $f$ is differentiable at every point $x\in S$.

For the most common example, a real function $f\colon\R\to\R$ is differentiable
if its derivative $\frac{df}{dx}$ exists for every point in the region of
interest. For another common case of a real function of $n$ variables
$f(x_1,x_2,\ldots,x_n)$ (more formally $f\colon\R^n\to\R$),
it is not sufficient that the partial derivatives
$\frac{\partial f}{\partial x_i}$ exist for $f$ to be differentiable. The
derivative of $f$ must exist in the original sense
at every point in the region of interest,
where $\R^n$ is treated as a Banach space under the usual Euclidean vector
norm.

If the derivative of $f$ is continuous, then $f$ is said to be $C^1$. If
the $k$th derivative of $f$ is continuous, then $f$ is said to be $C^k$. By convention, if $f$
is only continuous but does not have a continuous derivative, then $f$ is said to
be $C^0$. Note the inclusion property $C^{k+1} \subset C^k$.
And if the $k$-th derivative of $f$ is continuous for all $k$,
then $f$ is said to be $C^\infty$. In other words $C^\infty$ is the
intersection $C^\infty = \bigcap_{k=0}^\infty C^k$.

Differentiable functions are often referred to as {\em smooth}. If $f$ is
$C^k$, then $f$ is said to be $k$-smooth. Most often a function is called
smooth (without qualifiers) if $f$ is $C^\infty$ or $C^1$, depending on the
context.
%%%%%
%%%%%
\end{document}
