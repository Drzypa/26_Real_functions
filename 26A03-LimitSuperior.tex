\documentclass[12pt]{article}
\usepackage{pmmeta}
\pmcanonicalname{LimitSuperior}
\pmcreated{2013-03-22 12:21:58}
\pmmodified{2013-03-22 12:21:58}
\pmowner{rmilson}{146}
\pmmodifier{rmilson}{146}
\pmtitle{limit superior}
\pmrecord{12}{32104}
\pmprivacy{1}
\pmauthor{rmilson}{146}
\pmtype{Definition}
\pmcomment{trigger rebuild}
\pmclassification{msc}{26A03}
\pmsynonym{limsup}{LimitSuperior}
\pmsynonym{supremum limit}{LimitSuperior}
\pmrelated{LimitInferior}

\endmetadata

\usepackage{amsmath}
\usepackage{amsfonts}
\usepackage{amssymb}

\newcommand{\reals}{\mathbb{R}}
\newcommand{\natnums}{\mathbb{N}}
\newcommand{\cnums}{\mathbb{C}}

\newcommand{\lp}{\left(}
\newcommand{\rp}{\right)}
\newcommand{\lb}{\left[}
\newcommand{\rb}{\right]}

\newcommand{\supth}{^{\text{th}}}

\newtheorem{proposition}{Proposition}
\begin{document}
Let $S\subset\reals$ be a set of real numbers.  Recall that a limit
point of $S$ is a real number $x\in\reals$ such that for all
$\epsilon>0$ there exist infinitely many $y\in S$ such that
$$\vert x-y\vert <\epsilon.$$
We define $\limsup S=\overline{\lim}$, pronounced the
{\em limit superior} of $S$, to be the supremum of all the limit
points of $S$.  If there are no limit points, we define the limit
superior to be $-\infty$.  

We can generalize the above definition to the case of a 
 mapping $f:X\to\reals$. Now, we define a limit point of 
 $f$ to be an $x\in \reals$ such that for all
$\epsilon>0$ there exist infinitely many $y\in X$ such that
$$\vert x-f(y)\vert <\epsilon.$$
We then define $\limsup f$, to be the
supremum of all the limit points of $f$, or $-\infty$ if there are no
limit points.  We recover the previous definition as a special case by
considering the limit superior of the inclusion mapping $\iota: S\to
\reals$.


Since a sequence of real numbers $x_0, x_1, x_2, ,\ldots$ is just a
mapping from $\natnums$ to $\reals$, we may adapt the above definition
to arrive at the notion of the limit superior of a sequence.  However
for the case of sequences, an alternative, but equivalent definition
is available.  For each $k\in\natnums$, let $y_k$ be the supremum of
the $k\supth$ tail,
$$y_k = \sup_{j\geq k} x_j .$$
This construction produces a
non-increasing sequence
$$y_0 \geq y_1 \geq y_2 \geq \ldots,$$
which either converges to its infimum, or diverges to $-\infty$.
We define the limit superior of the original sequence to be this limit;
$$\limsup_{k} x_k = \lim_k y_k.$$
%%%%%
%%%%%
\end{document}
