\documentclass[12pt]{article}
\usepackage{pmmeta}
\pmcanonicalname{WeierstrassProductInequality}
\pmcreated{2013-03-22 13:58:23}
\pmmodified{2013-03-22 13:58:23}
\pmowner{Daume}{40}
\pmmodifier{Daume}{40}
\pmtitle{Weierstrass product inequality}
\pmrecord{5}{34744}
\pmprivacy{1}
\pmauthor{Daume}{40}
\pmtype{Theorem}
\pmcomment{trigger rebuild}
\pmclassification{msc}{26D05}

\endmetadata

\usepackage{amssymb}
\usepackage{amsmath}
\usepackage{amsfonts}
\begin{document}
%26D05
\PMlinkescapeword{fixed}
For any finite family $(a_i)_{i\in I}$ of real numbers in the interval
$[0,1]$, we have
$$\prod_i(1-a_i)\ge 1-\sum_ia_i\;.$$
\textbf{Proof: } Write
$$f=\prod_i(1-a_i)+\sum_ia_i\;.$$
For any $k\in I$, and any fixed values of the $a_i$ for $i\ne k$,
$f$ is a polynomial of the first degree in $a_k$.
Consequently $f$ is minimal either at $a_k=0$ or $a_k=1$.
That brings us down to two cases: all the $a_i$ are zero, or at least
one of them is $1$. But in both cases it is clear that $f\ge 1$, QED.
%%%%%
%%%%%
\end{document}
