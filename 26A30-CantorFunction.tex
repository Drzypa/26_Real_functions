\documentclass[12pt]{article}
\usepackage{pmmeta}
\pmcanonicalname{CantorFunction}
\pmcreated{2013-03-22 14:08:23}
\pmmodified{2013-03-22 14:08:23}
\pmowner{jirka}{4157}
\pmmodifier{jirka}{4157}
\pmtitle{Cantor function}
\pmrecord{9}{35554}
\pmprivacy{1}
\pmauthor{jirka}{4157}
\pmtype{Definition}
\pmcomment{trigger rebuild}
\pmclassification{msc}{26A30}
\pmsynonym{Cantor ternary function}{CantorFunction}
\pmsynonym{Cantor-Lebesgue function}{CantorFunction}
\pmsynonym{Devil's staircase}{CantorFunction}
\pmrelated{CantorSet}
\pmrelated{SingularFunction}
\pmdefines{Cantor function}
\pmdefines{Cantor ternary function}

\endmetadata

% this is the default PlanetMath preamble.  as your knowledge
% of TeX increases, you will probably want to edit this, but
% it should be fine as is for beginners.

% almost certainly you want these
\usepackage{amssymb}
\usepackage{amsmath}
\usepackage{amsfonts}

% used for TeXing text within eps files
%\usepackage{psfrag}
% need this for including graphics (\includegraphics)
\usepackage{graphicx}
% for neatly defining theorems and propositions
%\usepackage{amsthm}
% making logically defined graphics
%%%\usepackage{xypic}

% there are many more packages, add them here as you need them

% define commands here
\begin{document}
The {\em Cantor function} is a canonical example of a singular function.  It is based on the Cantor set, and is usually defined as follows.  Let $x$ be a real number in $[0,1]$ with the ternary expansion $0.a_1 a_2 a_3 \ldots$, then let $N$ be $\infty$ if no $a_n = 1$ and
otherwise let $N$ be the smallest value such that $a_n = 1$.  Next let $b_n = \frac{1}{2}a_n$ for all $n < N$ and let $b_N = 1$.  We define the Cantor function (or the Cantor ternary function) as
\begin{equation*}
f(x) = \sum_{n=1}^N \frac{b_n}{2^n}.
\end{equation*}

This function can be easily checked to be continuous and monotonic on $[0,1]$ and also $f'(x) = 0$ almost everywhere (it is constant on the complement of the Cantor set), with $f(0) = 0$ and $f(1) = 1$.  Another
interesting fact about this function is that the arclength of the graph is 2, hence the calculus arclength formula does not work in this case.

\begin{center}
% done in Genius, http://www.jirka.org/genius.html
% see attached source code for function Cantor
\includegraphics[width=5.19in]{cantorfunction.eps}
\vspace*{0.1in}

{\tiny Figure 1: Graph of the cantor function using 20 iterations.}
\end{center}

This function, and functions similar to it are frequently called the Devil's staircase.  Such functions sometimes occur naturally in various areas of mathematics and mathematical physics and are not just a pathological oddity.

\begin{thebibliography}{9}
\bibitem{royden}
H.\@ L.\@ Royden. \emph{\PMlinkescapetext{Real Analysis}}. Prentice-Hall, Englewood Cliffs, New Jersey, 1988
\end{thebibliography}
%%%%%
%%%%%
\end{document}
