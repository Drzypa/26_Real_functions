\documentclass[12pt]{article}
\usepackage{pmmeta}
\pmcanonicalname{ProofOfRollesTheorem}
\pmcreated{2013-03-22 12:40:19}
\pmmodified{2013-03-22 12:40:19}
\pmowner{rmilson}{146}
\pmmodifier{rmilson}{146}
\pmtitle{proof of Rolle's theorem}
\pmrecord{5}{32947}
\pmprivacy{1}
\pmauthor{rmilson}{146}
\pmtype{Proof}
\pmcomment{trigger rebuild}
\pmclassification{msc}{26A06}

\usepackage{amssymb, amsmath, amsthm, alltt, setspace}
\newtheorem{thm}{Theorem}

\theoremstyle{definition}
\newtheorem*{defn}{Definition}
\theoremstyle{definition}
\newtheorem*{rem}{Remark}

\theoremstyle{definition}
\newtheorem*{nott}{Notation}

\newtheorem{lemma}{Lemma}
\newtheorem{cor}{Corollary}
\newtheorem*{eg}{Example}
\newtheorem*{ex}{Exercise}
\newtheorem*{prop}{Proposition}


\newcommand{\RR}{\mathbb{R}}
\newcommand{\QQ}{\mathbb{Q}}
\newcommand{\ZZ}{\mathbb{Z}}
\newcommand{\NN}{\mathbb{N}}
\newcommand{\leftbb}{[ \! [}
\newcommand{\rightbb}{] \! ]}
\newcommand{\bt}{\begin{thm}}
\newcommand{\et}{\end{thm}}
\newcommand{\Rel}{\mathbf{R}}
\newcommand{\er}{\thicksim}
\newcommand{\sqle}{\sqsubseteq}
\newcommand{\floor}[1]{\lfloor{#1}\rfloor}
\newcommand{\ceil}[1]{\lceil{#1}\rceil}
\begin{document}
Because $f$ is continuous on a compact (closed and bounded) interval $I = [a,b]$, it attains its 
maximum and minimum values.  In case $f(a)=f(b)$ is both the maximum and 
the minimum, then there is nothing more to say, for then $f$ is a constant function and
$f' \equiv 0$ on the whole interval $I$.  So suppose otherwise, and $f$ attains an extremum 
in the open interval
$(a,b)$, and without loss of generality, let this extremum be a maximum, considering $-f$ in
lieu of $f$ as necessary.  We claim that at this extremum $f(c)$ we have $f'(c) = 0$, with $a < c < b$.

To show this, note that $f(x) - f(c) \leq 0$ for all 
$x \in I$, because $f(c)$ is the maximum.  By definition of the derivative, we have that
\[
f'(c) = \lim_{x \to c} \frac{f(x) - f(c)}{x - c}.
\]
Looking at the one-sided limits, we note that 
\[
R = \lim_{x \to c^+} \frac{f(x) - f(c)}{x - c} \leq 0
\]
because the numerator in the limit is nonpositive in the interval $I$, 
yet $x - c > 0$, as $x$ approaches $c$ from the right.  Similarly,
\[
L = \lim_{x \to c^-} \frac{f(x) - f(c)}{x - c} \geq 0.
\]
Since $f$ is differentiable at $c$, the left and right limits must coincide, so 
$0 \leq L = R \leq 0$, that is to say, $f'(c) = 0$.
%%%%%
%%%%%
\end{document}
