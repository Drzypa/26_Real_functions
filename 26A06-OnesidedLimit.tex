\documentclass[12pt]{article}
\usepackage{pmmeta}
\pmcanonicalname{OnesidedLimit}
\pmcreated{2013-03-22 12:40:28}
\pmmodified{2013-03-22 12:40:28}
\pmowner{CWoo}{3771}
\pmmodifier{CWoo}{3771}
\pmtitle{one-sided limit}
\pmrecord{11}{32950}
\pmprivacy{1}
\pmauthor{CWoo}{3771}
\pmtype{Definition}
\pmcomment{trigger rebuild}
\pmclassification{msc}{26A06}
\pmsynonym{limit from below}{OnesidedLimit}
\pmsynonym{limit from above}{OnesidedLimit}
\pmsynonym{left-sided limit}{OnesidedLimit}
\pmsynonym{left-handed limit}{OnesidedLimit}
\pmsynonym{right-sided limit}{OnesidedLimit}
\pmsynonym{right-handed limit}{OnesidedLimit}
%\pmkeywords{"unit step"}
\pmrelated{Limit}
\pmrelated{OneSidedDerivatives}
\pmrelated{IntegratingTanXOver0fracpi2}
\pmrelated{OneSidedContinuity}
\pmdefines{Heaviside unit step function}

\endmetadata

% this is the default PlanetMath preamble.  as your knowledge
% of TeX increases, you will probably want to edit this, but
% it should be fine as is for beginners.

% almost certainly you want these
\usepackage{amssymb}
\usepackage{amsmath}
\usepackage{amsfonts}

% used for TeXing text within eps files
%\usepackage{psfrag}
% need this for including graphics (\includegraphics)
%\usepackage{graphicx}
% for neatly defining theorems and propositions
%\usepackage{amsthm}
% making logically defined graphics
%%%\usepackage{xypic}

% there are many more packages, add them here as you need them

% define commands here

\newcommand{\sR}[0]{\mathbb{R}}
\newcommand{\sC}[0]{\mathbb{C}}
\newcommand{\sN}[0]{\mathbb{N}}
\newcommand{\sZ}[0]{\mathbb{Z}}

% The below lines should work as the command
% \renewcommand{\bibname}{References}
% without creating havoc when rendering an entry in 
% the page-image mode.

\newcommand*{\norm}[1]{\lVert #1 \rVert}
\newcommand*{\abs}[1]{| #1 |}
\begin{document}
{\bf Definition}
Let $f$ be a real-valued function defined on $S \subseteq \sR$.  The \emph{left-hand one-sided 
limit} at $a\in \sR$ is defined to be the real number $L^-$ such that for every $\epsilon > 0$ there 
exists a $\delta > 0$ such that $|f(x) - L^-| < \epsilon$ whenever $0 < a - x < \delta$.

Analogously, the \emph{right-hand one-sided limit} at $a\in \sR$ is the 
real number $L^+$ such that 
for every $\epsilon > 0$ there exists a $\delta > 0$ such that $|f(x) - L^+| < \epsilon$ whenever
$0 < x - a < \delta$.  

Common notations for the one-sided limits are
\begin{eqnarray*}
L^+ &=& f(x+) = \lim_{x \to a^+} f(x) =\lim_{x \searrow a} f(x), \\
L^-  &=&f(x-) = \lim_{x \to a^-} f(x) =\lim_{x \nearrow a} f(x).
\end{eqnarray*}
%\[
%L^+ := \lim_{x \to a^+} f(x) \mbox{ or } L^+ := \lim_{x \searrow a} f(x)
%\]
%\[
%L^- := \lim_{x \to a^-} f(x) \mbox{ or } L^- := \lim_{x \nearrow a} f(x)
%\]
Sometimes, left-handed limits are referred to as limits \emph{from below} while 
right-handed limits are \emph{from above}.  

{\bf Theorem} The ordinary limit of a function exists at a point if and only 
if both one-sided limits exist at this point and are equal (to the ordinary limit).

{\bf Example}
The Heaviside unit step function, sometimes colloquially referred to as the diving board function,
defined by
\[
H(x) = 
 \begin{cases}
  0& \mbox{ if } ~x < 0 \\
  1& \mbox{ if } ~x \geq 0
 \end{cases}
\]
has the simplest kind of discontinuity at $x = 0$, a jump discontinuity.   Its ordinary limit does not exist at this point, but the one-sided limits do exist, and are
\[
\lim_{x \to 0^-} H(x) = 0 \mbox{ and } \lim_{x \to 0^+} H(x) = 1.
\]
%%%%%
%%%%%
\end{document}
