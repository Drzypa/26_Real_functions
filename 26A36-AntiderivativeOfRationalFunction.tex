\documentclass[12pt]{article}
\usepackage{pmmeta}
\pmcanonicalname{AntiderivativeOfRationalFunction}
\pmcreated{2013-03-22 19:21:38}
\pmmodified{2013-03-22 19:21:38}
\pmowner{pahio}{2872}
\pmmodifier{pahio}{2872}
\pmtitle{antiderivative of rational function}
\pmrecord{19}{42313}
\pmprivacy{1}
\pmauthor{pahio}{2872}
\pmtype{Theorem}
\pmcomment{trigger rebuild}
\pmclassification{msc}{26A36}
\pmsynonym{integration of rational functions}{AntiderivativeOfRationalFunction}
\pmrelated{IntegrationTechniques}

% this is the default PlanetMath preamble.  as your knowledge
% of TeX increases, you will probably want to edit this, but
% it should be fine as is for beginners.

% almost certainly you want these
\usepackage{amssymb}
\usepackage{amsmath}
\usepackage{amsfonts}

% used for TeXing text within eps files
%\usepackage{psfrag}
% need this for including graphics (\includegraphics)
%\usepackage{graphicx}
% for neatly defining theorems and propositions
 \usepackage{amsthm}
% making logically defined graphics
%%%\usepackage{xypic}

% there are many more packages, add them here as you need them

% define commands here

\theoremstyle{definition}
\newtheorem*{thmplain}{Theorem}

\begin{document}
The most notable real functions, which can be integrated in a closed form, are the rational functions:

\textbf{Theorem.}\, The antiderivative of a rational function is always expressible in a closed form, which only can comprise, except a rational expression summand, summands of logarithms and arcustangents of rational functions.

One can justify the theorem by using the general form of the (unique) partial fraction decomposition

\begin{align*}R(x) \;=\quad H(x)+&
\sum_{i=1}^m\left(\frac{A_{i1}}{x\!-\!a_i}+\frac{A_{i2}}{(x\!-\!a_i)^2}+\ldots
+\frac{A_{i\mu_i}}{(x\!-\!a_i)^{\mu_i}}\right)\\
+&\sum_{j=1}^n\left(\frac{B_{j1}x\!+\!C_{j1}}{x^2\!+\!2p_jx\!+\!q_j}+
\frac{B_{j2}x\!+\!C_{j2}}{( x^2\!+\!2p_jx\!+\!q_j)^2}+\ldots
+\frac{B_{j\nu_j}x\!+\!C_{j\nu_j}}{( x^2\!+\!2p_jx\!+\!q_j)^{\nu_j}}\right)\!,
\end{align*}

of the rational function $R(x)$ ; here, $H(x)$ is a polynomial, the first sum expression is determined by the real zeroes $a_i$ of the denominator of $R(x)$, the second sum is determined by the real quadratic prime factors $x^2\!+\!2p_jx\!+\!q_j$ of the denominator (which have no real zeroes).

The addends of the form $\displaystyle\frac{A}{(x\!-\!a)^r}$ in the first sum are integrated directly, giving
\begin{align}
\int\!\frac{A}{x\!-\!a}\,dx \;=\; A\ln|x\!-\!a|+\mbox{constant} \qquad (r = 1)
\end{align}
and
\begin{align}
\int\!\frac{A}{(x\!-\!a)^r}\,dx \;=\; -\frac{A}{r\!-\!1}\!\cdot\!\frac{1}{(x\!-\!a)^{r-1}}+\mbox{constant} \qquad (r > 1).
\end{align}

The remaining partial fractions are of the form $\displaystyle\frac{Bx\!+\!C}{(x^2\!+\!2px\!+\!q)^s}$ where\, $p^2 < q$\, and $s$ is a positive integer.\, Now we may write 
$$x^2\!+\!2px\!+\!q \;=\; (x\!+\!p)^2\!+\!q\!-\!p^2 \;=\; (q\!-\!p^2)\!\left[1+\left(\!\frac{x\!+\!p}{\sqrt{q\!-\!p^2}}\!\right)^{\!2}\right]$$ and make the substitution
\begin{align}
\frac{x\!+\!p}{\sqrt{q\!-\!p^2}} \;=\; t,
\end{align}
i.e.\, $x \,=\, t\sqrt{q\!-\!p^2}\!-\!p$,\, getting
\begin{align}
\int\!\frac{Bx\!+\!C}{(x^2\!+\!2px\!+\!q)^s}\,dx \;=\; \int\!\frac{Et\!+\!F}{(1\!+\!t^2)^s}\,dt \;=\; 
E\!\int\!\frac{t\,dt}{(1\!+\!t^2)^s}+F\!\int\!\frac{dt}{(1\!+\!t^2)^s}
\end{align}
where $E$ and $F$ are certain constants.\, In the case \,$s = 1$\, we have
\begin{align}
\int\!\frac{t\,dt}{1\!+\!t^2} \;=\; \frac{1}{2}\ln(1\!+\!t^2)+\mbox{constant}
\end{align}
and in the case \,$s >1$
\begin{align}
\int\!\frac{t\,dt}{(1\!+\!t^2)^s} \;=\; -\frac{1}{2(s\!-\!1)}\!\cdot\!\frac{1}{(1\!+\!t^2)^{s-1}}+\mbox{constant}.
\end{align}
The latter addend of the right hand side of (4) is for\, $s = 1$\, got from
\begin{align}
\int\!\frac{dt}{1\!+\!t^2} \;=\; \arctan{t}+\mbox{constant}
\end{align}
and for the cases $s > 1$ on may first write 
$$\int\!\frac{dt}{(1\!+\!t^2)^s} \;=\; \int\!\frac{(1\!+\!t^2)\!-\!t^2}{(1\!+\!t^2)^s}\,dt \;=\;
\int\!\frac{dt}{(1\!+\!t^2)^{s-1}}-\int\!t\!\cdot\!\frac{t\,dt}{(1\!+\!t^2)^s}.$$ 
Using integration by parts in the last integral, this equation can be converted into the reduction formula
\begin{align}
\int\!\frac{dt}{(1\!+\!t^2)^s} \;=\; 
\frac{1}{2s\!-\!2}\!\cdot\!\frac{t}{(1\!+\!t^2)^{s-1}}+\frac{2n\!-\!3}{2n\!-\!2}\int\!\frac{dt}{(1\!+\!t^2)^{s-1}}.
\end{align}
The assertion of the theorem follows from (1), \ldots, (8).\\

\textbf{Example.}\, 
$$\int\!\frac{dx}{1\!+\!x^4} \;=\; \frac{1}{4\sqrt{2}}\ln\frac{1\!+\!x\sqrt{2}\!+\!x^2}{1\!-\!x\sqrt{2}\!+\!x^2}
+\frac{1}{2\sqrt{2}}\arctan\frac{x\sqrt{2}}{1\!-\!x^2}+C$$

%%%%%
%%%%%
\end{document}
