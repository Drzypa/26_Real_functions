\documentclass[12pt]{article}
\usepackage{pmmeta}
\pmcanonicalname{MeanvalueTheoremForSeveralVariables}
\pmcreated{2013-03-22 19:11:36}
\pmmodified{2013-03-22 19:11:36}
\pmowner{pahio}{2872}
\pmmodifier{pahio}{2872}
\pmtitle{mean-value theorem for several variables}
\pmrecord{7}{42105}
\pmprivacy{1}
\pmauthor{pahio}{2872}
\pmtype{Theorem}
\pmcomment{trigger rebuild}
\pmclassification{msc}{26A06}
\pmclassification{msc}{26B05}

\endmetadata

% this is the default PlanetMath preamble.  as your knowledge
% of TeX increases, you will probably want to edit this, but
% it should be fine as is for beginners.

% almost certainly you want these
\usepackage{amssymb}
\usepackage{amsmath}
\usepackage{amsfonts}

% used for TeXing text within eps files
%\usepackage{psfrag}
% need this for including graphics (\includegraphics)
%\usepackage{graphicx}
% for neatly defining theorems and propositions
 \usepackage{amsthm}
% making logically defined graphics
%%%\usepackage{xypic}

% there are many more packages, add them here as you need them

% define commands here

\theoremstyle{definition}
\newtheorem*{thmplain}{Theorem}

\begin{document}
The mean-value theorem for a function of one real variable may be generalised for functions of arbitrarily many real variables; for the sake of concreteness, we here formulate it for the case of three variables:\\


\textbf{Theorem.}\, If a function \,$f(x,\,y,\,z)$\, is continuously differentiable in an open set of 
$\mathbb{R}^3$ containing the points \,$(x_1,\,y_1,\,z_1)$\, and \,$(x_2,\,y_2,\,z_2)$\, and the line segment connecting them, then an equation
$$f(x_2,\,y_2,\,z_2)-f(x_1,\,y_1,\,z_1) \;=\; 
f'_x(a,\,b,\,c)(x_2\!-\!x_1)+f'_y(a,\,b,\,c)(y_2\!-\!y_1)+f'_z(a,\,b,\,c)(z_2\!-\!z_1),$$
where $(a,\,b,\,c)$ an interior point of the line segment, is \PMlinkescapetext{valid}.\\



%%%%%
%%%%%
\end{document}
