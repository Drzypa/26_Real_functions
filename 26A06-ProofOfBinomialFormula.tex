\documentclass[12pt]{article}
\usepackage{pmmeta}
\pmcanonicalname{ProofOfBinomialFormula}
\pmcreated{2013-03-22 12:24:00}
\pmmodified{2013-03-22 12:24:00}
\pmowner{rmilson}{146}
\pmmodifier{rmilson}{146}
\pmtitle{proof of binomial formula}
\pmrecord{6}{32214}
\pmprivacy{1}
\pmauthor{rmilson}{146}
\pmtype{Proof}
\pmcomment{trigger rebuild}
\pmclassification{msc}{26A06}

\usepackage{amsmath}
\usepackage{amsfonts}
\usepackage{amssymb}

\newcommand{\reals}{\mathbb{R}}
\newcommand{\natnums}{\mathbb{N}}
\newcommand{\cnums}{\mathbb{C}}

\newcommand{\lp}{\left(}
\newcommand{\rp}{\right)}
\newcommand{\lb}{\left[}
\newcommand{\rb}{\right]}

\newcommand{\supth}{^{\text{th}}}


\newtheorem{proposition}{Proposition}
\begin{document}
Let  $p\in \reals$ and $x\in\reals,\;|x|<1$ be given.  We wish to show
that 
$$(1+x)^p = \sum_{n=0}^\infty p^{\underline{n}}\; \frac{x^n}{n!},$$
where $p^{\underline{n}}$ denotes the $n\supth$ falling factorial of $p$.  

The convergence of the series in the right-hand side of the above
equation is a straight-forward consequence of the ratio test. Set 
$$f(x) = (1+x)^p.$$
and note that
$$f^{(n)}(x) = p^{\underline{n}}\, (1+x)^{p-n}.$$
The desired equality now follows from Taylor's Theorem. Q.E.D.
%%%%%
%%%%%
\end{document}
