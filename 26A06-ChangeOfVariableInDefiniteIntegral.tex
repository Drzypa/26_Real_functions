\documentclass[12pt]{article}
\usepackage{pmmeta}
\pmcanonicalname{ChangeOfVariableInDefiniteIntegral}
\pmcreated{2014-05-27 13:13:22}
\pmmodified{2014-05-27 13:13:22}
\pmowner{pahio}{2872}
\pmmodifier{pahio}{2872}
\pmtitle{change of variable in definite integral}
\pmrecord{10}{41373}
\pmprivacy{1}
\pmauthor{pahio}{2872}
\pmtype{Theorem}
\pmcomment{trigger rebuild}
\pmclassification{msc}{26A06}
\pmsynonym{change of variable in Riemann integral}{ChangeOfVariableInDefiniteIntegral}
\pmrelated{RiemannIntegral}
\pmrelated{SubstitutionForIntegration}
\pmrelated{FundamentalTheoremOfCalculus}
\pmrelated{IntegralsOfEvenAndOddFunctions}
\pmrelated{OrthogonalityOfChebyshevPolynomials}

\endmetadata

% this is the default PlanetMath preamble.  as your knowledge
% of TeX increases, you will probably want to edit this, but
% it should be fine as is for beginners.

% almost certainly you want these
\usepackage{amssymb}
\usepackage{amsmath}
\usepackage{amsfonts}

% used for TeXing text within eps files
%\usepackage{psfrag}
% need this for including graphics (\includegraphics)
%\usepackage{graphicx}
% for neatly defining theorems and propositions
 \usepackage{amsthm}
% making logically defined graphics
%%%\usepackage{xypic}

% there are many more packages, add them here as you need them

% define commands here

\theoremstyle{definition}
\newtheorem*{thmplain}{Theorem}

\begin{document}
\textbf{Theorem.}\, Let the real function \,$x \mapsto f(x)$\, be continuous on the interval \,$[a,\,b]$.\, We introduce via the the equation
$$x \;=\; \varphi(t)$$
a new variable $t$ satisfying
\begin{itemize}
\item $\varphi(\alpha) \,=\, a, \quad \varphi(\beta) \,=\, b$,
\item $\varphi$ and $\varphi'$ are continuous on the interval with endpoints $\alpha$ and $\beta$.
\end{itemize}
Then
$$\int_a^b\!f(x)\,dx \;=\; \int_\alpha^\beta\!f(\varphi(t))\,\varphi'(t)\,dt.$$\\


{\em Proof.}\, As a continuous function, $f$ has an antiderivative $F$.\, Then the composite function $F\circ\varphi$ is an antiderivative of $(f\circ\varphi)\cdot\varphi'$, since by the chain rule we have
$$\frac{d}{dt}F(\varphi(t)) \;=\; F'(\varphi(t))\,\varphi'(t) \;=\; f(\varphi(t))\,\varphi'(t).$$
Using the \PMlinkname{Newton--Leibniz formula}{node/40459} we obtain
$$\int_a^b\!f(x)\,dx \;=\; F(b)-F(a) \;=\; F(\varphi(\beta))-F(\varphi(\alpha)) 
\;=\; \int_\alpha^\beta\!f(\varphi(t))\,\varphi'(t)\,dt,$$
Q.E.D.
%%%%%
%%%%%
\end{document}
