\documentclass[12pt]{article}
\usepackage{pmmeta}
\pmcanonicalname{ProofOfBernoullisInequalityEmployingTheMeanValueTheorem}
\pmcreated{2013-03-22 15:49:53}
\pmmodified{2013-03-22 15:49:53}
\pmowner{rspuzio}{6075}
\pmmodifier{rspuzio}{6075}
\pmtitle{proof of Bernoulli's inequality employing the mean value theorem}
\pmrecord{10}{37803}
\pmprivacy{1}
\pmauthor{rspuzio}{6075}
\pmtype{Proof}
\pmcomment{trigger rebuild}
\pmclassification{msc}{26D99}

\endmetadata

% this is the default PlanetMath preamble.  as your knowledge
% of TeX increases, you will probably want to edit this, but
% it should be fine as is for beginners.

% almost certainly you want these
\usepackage{amssymb}
\usepackage{amsmath}
\usepackage{amsfonts}

% used for TeXing text within eps files
%\usepackage{psfrag}
% need this for including graphics (\includegraphics)
%\usepackage{graphicx}
% for neatly defining theorems and propositions
%\usepackage{amsthm}
% making logically defined graphics
%%%\usepackage{xypic}

% there are many more packages, add them here as you need them

% define commands here
\begin{document}
Let us take as our assumption that $x \in I = \left(-1,
\infty\right)$ and that $r \in J = \left(0, \infty\right)$. Observe
that if $x = 0$ the inequality holds quite obviously. Let us now
consider the case where $x \neq 0$. \linebreak
Consider now the
function $f: I\text{x}J\rightarrow\mathbb{R}$ given by
$$f(x,r) = (1 + x)^r - 1 - rx$$ Observe that for all $r$ in $J$
fixed, $f$ is, indeed, differentiable on $I$. In particular,
$$\frac{\partial}{{\partial}x}f(x,r) = r(1+x)^{r-1}-r$$ Consider
two points $a \neq 0$ in $I$ and $0$ in $I$. Then clearly by the
mean value theorem, for any arbitrary, fixed $\alpha$ in $J$, there
exists a $c$ in $I$ such that,
$$f'_x(c, \alpha) = \frac{f(a, \alpha) - f(0, \alpha)}{a}$$
\begin{equation}
\Leftrightarrow f'_xf(c,
\alpha)=\frac{(1+a)^\alpha-1-{\alpha}a}{a}\label{eq:last}
\end{equation}
Since $\alpha$ is in $J$, it is clear that if $a < 0$, then
$$f'_x(a,\alpha) < 0$$
and, accordingly, if $a > 0$ then
$$f'_x(a,\alpha) > 0$$
Thus, in either case, from {\ref{eq:last}} we deduce that
$$\frac{(1+a)^\alpha-1-{\alpha}a}{a} < 0$$ if $a < 0$ and
$$\frac{(1+a)^\alpha-1-{\alpha}a}{a} > 0$$ if $a > 0$. From this we
conclude that, in either case,$(1+a)^\alpha-1-{\alpha}a > 0$. That
is, $$(1 + a)^\alpha > 1+{\alpha}a$$ for all choices of $a$ in $I -
\left\{0\right\}$ and all choices of $\alpha$ in $J$. If $a = 0$ in
$I$, we have$$(1 + a)^\alpha = 1+{\alpha}a$$ for all choices of
$\alpha$ in $J$. Generally, for all $x$ in $I$ and all $r$ in $J$ we
have:
$$(1+x)^r \geq 1 + rx$$
This completes the proof.

Notice that if $r$ is in $\left(-1, 0\right)$ then the inequality would be reversed. That is: $$(1+x)^r \leq 1 + rx$$. This can be proved using exactly the same method, by fixing $\alpha$ in the proof above in $\left(-1, 0\right)$.
%%%%%
%%%%%
\end{document}
