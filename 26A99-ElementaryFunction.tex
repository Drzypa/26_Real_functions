\documentclass[12pt]{article}
\usepackage{pmmeta}
\pmcanonicalname{ElementaryFunction}
\pmcreated{2013-03-22 14:46:29}
\pmmodified{2013-03-22 14:46:29}
\pmowner{pahio}{2872}
\pmmodifier{pahio}{2872}
\pmtitle{elementary function}
\pmrecord{18}{36420}
\pmprivacy{1}
\pmauthor{pahio}{2872}
\pmtype{Definition}
\pmcomment{trigger rebuild}
\pmclassification{msc}{26A99}
\pmrelated{RiemannZetaFunction}
\pmrelated{LogarithmicIntegral}
\pmrelated{AlgebraicFunction}
\pmrelated{TableOfMittagLefflerPartialFractionExpansions}

% this is the default PlanetMath preamble.  as your knowledge
% of TeX increases, you will probably want to edit this, but
% it should be fine as is for beginners.

% almost certainly you want these
\usepackage{amssymb}
\usepackage{amsmath}
\usepackage{amsfonts}

% used for TeXing text within eps files
%\usepackage{psfrag}
% need this for including graphics (\includegraphics)
%\usepackage{graphicx}
% for neatly defining theorems and propositions
%\usepackage{amsthm}
% making logically defined graphics
%%%\usepackage{xypic}

% there are many more packages, add them here as you need them

% define commands here
\DeclareMathOperator{\Li}{Li}
\begin{document}
An {\em elementary function} is a real function (of one variable) that can be constructed by a finite number of elementary operations (addition, subtraction, multiplication and division) and compositions from constant functions, the identity function ($x \mapsto x$), algebraic functions, exponential functions, logarithm functions, trigonometric functions and cyclometric functions.

\textbf{Examples}
\begin{itemize}
 \item Consequently, the polynomial functions, the absolute value\, $|x| = \sqrt{x^2}$,\, the triangular-wave function\, $\arcsin(\sin{x})$, the power function\, $x^{\pi} = e^{\pi\ln{x}}$\, and the function\, $x^x = e^{x\ln{x}}$\, are elementary functions (N.B., the real power functions entail that\, $x > 0$).
 \item $\displaystyle\zeta(x) := \sum_{n = 1}^{\infty}\frac{1}{n^x}$\, and\, $\displaystyle\Li{x} := \int_2^{x}\frac{dt}{\ln{t}}$\, are not elementary functions --- it may be shown that they can not be expressed is such a way which is required in the definition.
\end{itemize}
%%%%%
%%%%%
\end{document}
