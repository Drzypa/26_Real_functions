\documentclass[12pt]{article}
\usepackage{pmmeta}
\pmcanonicalname{MonotoneConvergenceTheorem}
\pmcreated{2013-03-22 12:47:27}
\pmmodified{2013-03-22 12:47:27}
\pmowner{Koro}{127}
\pmmodifier{Koro}{127}
\pmtitle{monotone convergence theorem}
\pmrecord{9}{33106}
\pmprivacy{1}
\pmauthor{Koro}{127}
\pmtype{Theorem}
\pmcomment{trigger rebuild}
\pmclassification{msc}{26A42}
\pmclassification{msc}{28A20}
\pmsynonym{Lebesgue's monotone convergence theorem}{MonotoneConvergenceTheorem}
\pmsynonym{Beppo Levi's theorem}{MonotoneConvergenceTheorem}
\pmrelated{DominatedConvergenceTheorem}
\pmrelated{FatousLemma}

\endmetadata

% this is the default PlanetMath preamble.  as your knowledge
% of TeX increases, you will probably want to edit this, but
% it should be fine as is for beginners.

% almost certainly you want these
\usepackage{amssymb}
\usepackage{amsmath}
\usepackage{amsfonts}

% used for TeXing text within eps files
%\usepackage{psfrag}
% need this for including graphics (\includegraphics)
%\usepackage{graphicx}
% for neatly defining theorems and propositions
%\usepackage{amsthm}
% making logically defined graphics
%%%\usepackage{xypic}

% there are many more packages, add them here as you need them

% define commands here

\newcommand{\Prob}[2]{\mathbb{P}_{#1}\left\{#2\right\}}
\newcommand{\Expect}{\mathbb{E}}
\newcommand{\norm}[1]{\left\|#1\right\|}

% Some sets
\newcommand{\Nats}{\mathbb{N}}
\newcommand{\Ints}{\mathbb{Z}}
\newcommand{\Reals}{\mathbb{R}}
\newcommand{\Complex}{\mathbb{C}}



%%%%%% END OF SAVED PREAMBLE %%%%%%
\begin{document}
Let $X$ be a measure space, and let $0\leq f_1\leq f_2\leq\cdots$ be a monotone increasing sequence of nonnegative measurable functions. Let $f\colon X \to \mathbb{R}\cup \{\infty\}$ be the
function defined by $f(x) = \lim_{n\rightarrow\infty} f_n(x)$.
Then $f$ is measurable, and 
$$\lim_{n\rightarrow\infty} \int_X f_n = \int_X f.$$

\textbf{Remark.} This theorem is the first of several theorems which allow us to ``exchange integration and limits''.  It requires the use of the Lebesgue integral: with the Riemann integral, we cannot even formulate the theorem, lacking, as we do, the concept of ``almost everywhere''.  For instance, the characteristic function of the rational numbers in $[0,1]$ is not Riemann integrable, despite being the limit of an increasing sequence of Riemann integrable functions.
%%%%%
%%%%%
\end{document}
