\documentclass[12pt]{article}
\usepackage{pmmeta}
\pmcanonicalname{AlternativeDefinitionOfLebesgueIntegralAn}
\pmcreated{2013-03-22 17:32:46}
\pmmodified{2013-03-22 17:32:46}
\pmowner{Gorkem}{3644}
\pmmodifier{Gorkem}{3644}
\pmtitle{alternative definition of Lebesgue integral, an}
\pmrecord{6}{39948}
\pmprivacy{1}
\pmauthor{Gorkem}{3644}
\pmtype{Definition}
\pmcomment{trigger rebuild}
\pmclassification{msc}{26A42}
\pmclassification{msc}{28A25}

\endmetadata

\usepackage{amssymb}
\usepackage{amsmath}
\usepackage{amsfonts}
\usepackage{mathrsfs}
\usepackage{amssymb,amsbsy}
\usepackage{graphicx,color}
\usepackage{epsfig}


% used for TeXing text within eps files
%\usepackage{psfrag}
% need this for including graphics (\includegraphics)
%\usepackage{graphicx}
% for neatly defining theorems and propositions
%\usepackage{amsthm}
% making logically defined graphics
%%%\usepackage{xypic}

% there are many more packages, add them here as you need them

% define commands here





\newtheorem{thm}{Theorem}[section]
\newtheorem{defn}[thm]{Definition}
\newtheorem{lemma}[thm]{Lemma}
\newtheorem{prop}[thm]{Proposition}
\newtheorem{rk}[thm]{Remark}
\newtheorem{crl}[thm]{Corollary}
\newtheorem{stp}{Step}

\newcommand{\disp}{\displaystyle}
\newcommand{\dintl}{\disp\int\limits}
\newcommand{\dsuml}{\disp\sum\limits}
\newcommand{\hsp}{\hspace{30pt}}
\newcommand{\ba}{\begin{array}}
\newcommand{\ea}{\end{array}}
\newcommand{\trns}{\,\widehat{}\,\,}

\newcommand{\bthm}{\begin{thm}}
\newcommand{\ethm}{\end{thm}}
\newcommand{\bstp}{\begin{stp}}
\newcommand{\estp}{\end{stp}}
\newcommand{\blemma}{\begin{lemma}}
\newcommand{\elemma}{\end{lemma}}
\newcommand{\bprop}{\begin{prop}}
\newcommand{\eprop}{\end{prop}}
\newcommand{\bpf}{\begin{pf}}
\newcommand{\epf}{\end{pf}}
\newcommand{\bdefn}{\begin{defn}}
\newcommand{\edefn}{\end{defn}}
\newcommand{\brk}{\begin{rk}}
\newcommand{\erk}{\end{rk}}
\newcommand{\bcrl}{\begin{crl}}
\newcommand{\ecrl}{\end{crl}}


\newcommand{\norm}[1]{\left\|#1\right\|}
\newcommand{\brackets}[1]{\left[#1\right]}
\newcommand{\beqn}{\begin{equation}}
\newcommand{\eeqn}{\end{equation}}
\newcommand{\supnorm}[1]{\norm{#1}_\infty}
\newcommand{\normt}[1]{\norm{#1}_2}
\newcommand{\ip}[2]{\left\langle#1 , #2\right\rangle}
\newcommand{\supp}{\operatorname{supp}}
\newcommand{\calg}[1]{\mathcal{#1}}


\newcommand{\sinc}{\operatorname{sinc}}
\newcommand{\qed}{$\ \ \ \ \ \Box$}
\newcommand{\qedin}{\ \ \ \ \ \Box}
\newcommand{\Tr}{\operatorname{\Tr}}

\newenvironment{pf}{\begin{trivlist}\item[\hskip%
\labelsep{\bf Proof.}]}
{\rm\end{trivlist}}

\begin{document}
The standard way of defining Lebesgue integral is first to define it for  simple functions, and then to take limits for arbitrary positive measurable functions.  

There is also another way which uses the Riemann integral \cite{lieb}.

Let $(X,\calg{M},\mu)$ be a measure space. Let $f\colon x\rightarrow \mathbb{R}^+\cup \{0\}$ be a nonnegative measurable function.  We will define $\int f d\mu$ in $[0,\infty]$ and will call it as the \emph{Lebesgue integral} of $f$.

If there exists a $t>0$ such that $\mu\left(\left\{x\colon f(x)>t \right\}\right)=\infty$, then we define $\int f d\mu = \infty.$

Otherwise, assume $\mu\left(\left\{x\colon f(x)>t \right\}\right)<\infty$ for all $t\in(0,\infty)$ and let $F_f(t) = \mu\left(\left\{x\colon f(x)>t \right\}\right)$.  $F_f(t)$ is a monotonically non-increasing function on $(0,\infty)$, therefore its Riemann integral is well defined on any interval $[a,b]\subset (0,\infty)$, so it exists as an improper Riemann integral on $(0,\infty)$.  We define
$$
\int f d\mu = \int_0^\infty F_f(t) dt.
$$

The definition can be extended first to real-valued functions, then complex valued functions as usual.





\begin{thebibliography}{9}
\bibitem{lieb} Lieb, E. H., Loss, M., \textsl{Analysis},  AMS, 2001.

\end{thebibliography}

%%%%%
%%%%%
\end{document}
