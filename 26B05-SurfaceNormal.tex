\documentclass[12pt]{article}
\usepackage{pmmeta}
\pmcanonicalname{SurfaceNormal}
\pmcreated{2013-03-22 17:23:10}
\pmmodified{2013-03-22 17:23:10}
\pmowner{pahio}{2872}
\pmmodifier{pahio}{2872}
\pmtitle{surface normal}
\pmrecord{16}{39753}
\pmprivacy{1}
\pmauthor{pahio}{2872}
\pmtype{Definition}
\pmcomment{trigger rebuild}
\pmclassification{msc}{26B05}
\pmclassification{msc}{26A24}
\pmclassification{msc}{53A04}
\pmclassification{msc}{53A05}
\pmsynonym{surface normal line}{SurfaceNormal}
\pmsynonym{normal of surface}{SurfaceNormal}
\pmrelated{NormalLine}
\pmrelated{EquationOfPlane}
\pmrelated{Parameter}
\pmdefines{parametre plane}
\pmdefines{parameter plane}
\pmdefines{parametre curve}
\pmdefines{parameter curve}
\pmdefines{Gaussian coordinates}

\endmetadata

% this is the default PlanetMath preamble.  as your knowledge
% of TeX increases, you will probably want to edit this, but
% it should be fine as is for beginners.

% almost certainly you want these
\usepackage{amssymb}
\usepackage{amsmath}
\usepackage{amsfonts}

% used for TeXing text within eps files
%\usepackage{psfrag}
% need this for including graphics (\includegraphics)
%\usepackage{graphicx}
% for neatly defining theorems and propositions
 \usepackage{amsthm}
% making logically defined graphics
%%%\usepackage{xypic}

% there are many more packages, add them here as you need them

% define commands here

\theoremstyle{definition}
\newtheorem*{thmplain}{Theorem}

\begin{document}
\PMlinkescapeword{curves} \PMlinkescapeword{components} \PMlinkescapeword{length}
Let $S$ be a smooth surface in $\mathbb{R}^3$.  The {\em surface normal} of $S$ at a point $P$ of $S$ is the line passing through $P$ and perpendicular to the tangent plane $\tau$ of $S$ at the point $P$, i.e. perpendicular to all lines in $\tau$.

If the surface $S$ is given in a parametric form
$$x = x(u,\,v),\quad y = y(u,\,v),\quad z = z(u,\,v),$$
it is useful to interpret the parameters $u$ and $v$ as the rectangular coordinates of a point in a plane, the so-called {\em parameter plane}.  We can consider on $S$ the so-called {\em parameter curves}, namely the $u$-{\em curves} which correspond the lines parallel to the $u$-axis and the $v$-{\em curves} which correspond the lines parallel to the $v$-axis in the parameter plane.   One $u$-curve and one $v$-curve passes through every point on the surface (the values of $u$ and $v$ in a point of $S$ are the {\em Gaussian coordinates} of this point).  The surface normal at any point of $S$ is perpendicular to both parameter curves, and thus its direction cosines $a$, $b$, $c$ satisfy the equations
\begin{align*}
\begin{cases}
\displaystyle{a\frac{\partial x}{\partial u}+b\frac{\partial y}{\partial u}+c\frac{\partial z}{\partial u}= 0},\\
\\
\displaystyle{a\frac{\partial x}{\partial v}+b\frac{\partial y}{\partial v}+c\frac{\partial z}{\partial v}= 0.}
\end{cases}
\end{align*}
This homogeneous pair of linear equations determines the ratio of the direction cosines 
$$a:b:c = \frac{\partial(y,\,z)}{\partial(u,\,v)}:\frac{\partial(z,\,x)}{\partial(u,\,v)}:\frac{\partial(x,\,y)}{\partial(u,\,v)}$$
via the Jacobians.\\

\textbf{Example.}  Determine the direction cosines of the normal of the helicoid
$$x = u\cos{v},\quad y = u\sin{v},\quad z = cv.$$
We have the Jacobians
$$\left| \begin{matrix}
\frac{\partial y}{\partial u} & \frac{\partial z}{\partial u}\\
\frac{\partial y}{\partial v} & \frac{\partial z}{\partial v}
\end{matrix}\right| =  
\left|\begin{matrix}
\sin{v} & 0 \\
u\cos{v} & c
\end{matrix}\right| = c\sin{v},\;\;
\left|\begin{matrix}
\frac{\partial z}{\partial u} & \frac{\partial x}{\partial u}\\
\frac{\partial z}{\partial v} & \frac{\partial x}{\partial v}
\end{matrix}\right| = 
\left|\begin{matrix}0 & \cos{v} \\
c & -u\sin{v}\end{matrix}\right| = -c\cos{v},\;\;
\left|\begin{matrix}
\frac{\partial x}{\partial u} & \frac{\partial y}{\partial u}\\
\frac{\partial x}{\partial v} & \frac{\partial y}{\partial v}
\end{matrix}\right| =
\left|\begin{matrix}\cos{v} & \sin{v}\\
-u\sin{v} & u\cos{v}\end{matrix}\right| = u.$$
These are the components of the normal vector of the helicoid surface in the point with the Gaussian coordinates $u$ and $v$.\, The length of the vector is\, $\sqrt{(c\sin{v})^2+(-c\cos{v})^2+u^2} = \sqrt{u^2+c^2}$.\, If we \PMlinkname{divide}{Division} the vector by its length, we obtain a unit vector, the components of which are the direction cosines of the surface normal:
$$\frac{c\sin{v}}{\sqrt{u^2+c^2}},\;\;-\frac{c\cos{v}}{\sqrt{u^2+c^2}},\;\;\frac{u}{\sqrt{u^2+c^2}}.$$
%%%%%
%%%%%
\end{document}
