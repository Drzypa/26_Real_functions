\documentclass[12pt]{article}
\usepackage{pmmeta}
\pmcanonicalname{UniformlyContinuous}
\pmcreated{2013-03-22 12:45:38}
\pmmodified{2013-03-22 12:45:38}
\pmowner{n3o}{216}
\pmmodifier{n3o}{216}
\pmtitle{uniformly continuous}
\pmrecord{14}{33068}
\pmprivacy{1}
\pmauthor{n3o}{216}
\pmtype{Definition}
\pmcomment{trigger rebuild}
\pmclassification{msc}{26A15}
\pmrelated{UniformContinuity}
\pmdefines{uniformly continuous function}

\endmetadata

% this is the default PlanetMath preamble.  as your knowledge
% of TeX increases, you will probably want to edit this, but
% it should be fine as is for beginners.

% almost certainly you want these
\usepackage{amssymb}
\usepackage{amsmath}
\usepackage{amsfonts}

% used for TeXing text within eps files
%\usepackage{psfrag}
% need this for including graphics (\includegraphics)
%\usepackage{graphicx}
% for neatly defining theorems and propositions
%\usepackage{amsthm}
% making logically defined graphics
%%%\usepackage{xypic}

% there are many more packages, add them here as you need them

% define commands here
\begin{document}
Let $f: A \rightarrow \mathbb{R}$ be a real function defined on a subset $A$ of the real line. We say that $f$ is \emph{uniformly continuous} if, given an arbitrary small positive $\varepsilon$, there exists a positive $\delta$ such that whenever two points in $A$ differ by less than $\delta$, they are mapped by $f$ into points which differ by less than $\varepsilon$. In symbols:
\[ \forall \varepsilon > 0\ \exists \delta > 0\ \forall x,y \in A\ |x-y| < \delta \Rightarrow |f(x)-f(y)| < \varepsilon. \]

Every uniformly continuous function is also continuous, while the converse does not always hold. For instance, the function $f: ]0,+\infty[ \rightarrow \mathbb{R}$ defined by $f(x) = 1/x$ is continuous in its domain, but not uniformly.

A more general definition of uniform continuity applies to functions between metric spaces (there are even more general environments for uniformly continuous functions, i.e. uniform spaces).
Given a function $f: X \rightarrow Y$, where $X$ and $Y$ are metric spaces with distances $d_X$ and $d_Y$, we say that $f$ is uniformly continuous if
\[ \forall \varepsilon > 0\ \exists \delta > 0\ \forall x,y \in X\ d_X(x,y) < \delta \Rightarrow d_Y(f(x),f(y)) < \varepsilon. \]

Uniformly continuous functions have the property that they map Cauchy sequences to Cauchy sequences and that they preserve uniform convergence of sequences of functions.

Any continuous function defined on a compact space is uniformly continuous (see Heine-Cantor theorem).
%%%%%
%%%%%
\end{document}
