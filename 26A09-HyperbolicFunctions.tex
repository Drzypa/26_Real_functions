\documentclass[12pt]{article}
\usepackage{pmmeta}
\pmcanonicalname{HyperbolicFunctions}
\pmcreated{2013-03-22 12:38:27}
\pmmodified{2013-03-22 12:38:27}
\pmowner{mathwizard}{128}
\pmmodifier{mathwizard}{128}
\pmtitle{hyperbolic functions}
\pmrecord{13}{32905}
\pmprivacy{1}
\pmauthor{mathwizard}{128}
\pmtype{Definition}
\pmcomment{trigger rebuild}
\pmclassification{msc}{26A09}
\pmrelated{UnitHyperbola}
\pmrelated{ComplexTangentAndCotangent}
\pmrelated{ParallelCurve}
\pmrelated{HyperbolicAngle}
\pmrelated{ExampleOfCauchyMultiplicationRule}
\pmrelated{DerivationOfFormulasForHyperbolicFunctionsFromDefinitionOfHyperbolicAngle}
\pmrelated{HeavisideFormula}
\pmrelated{Catenary}
\pmrelated{HyperbolicSineIntegral}
\pmrelated{InverseGudermannia}
\pmdefines{sinh}
\pmdefines{cosh}
\pmdefines{tanh}
\pmdefines{coth}
\pmdefines{sech}
\pmdefines{csch}
\pmdefines{hyperbolic sine}
\pmdefines{hyperbolic cosine}
\pmdefines{hyperbolic tangent}
\pmdefines{hyperbolic cotangent}
\pmdefines{hyperbolic secant}
\pmdefines{hyperbolic cosecant}

% this is the default PlanetMath preamble.  as your knowledge
% of TeX increases, you will probably want to edit this, but
% it should be fine as is for beginners.

% almost certainly you want these
\usepackage{amssymb}
\usepackage{amsmath}
\usepackage{amsfonts}

% used for TeXing text within eps files
%\usepackage{psfrag}
% need this for including graphics (\includegraphics)
\usepackage{graphicx}
% for neatly defining theorems and propositions
%\usepackage{amsthm}
% making logically defined graphics
%%%\usepackage{xypic}

% there are many more packages, add them here as you need them

% define commands here
\newcommand{\sech}{\operatorname{sech}}
\newcommand{\csch}{\operatorname{csch}}
\begin{document}
The hyperbolic functions $\sinh$ ({\em sinus hyperbolicus}) and $\cosh$ ({\em cosinus hyperbolicus}) with arbitrary complex argument $x$ are defined as follows:
\begin{eqnarray*}
\sinh x&:=&\frac{e^x-e^{-x}}{2},\\
\cosh x&:=&\frac{e^x+e^{-x}}{2}.
\end{eqnarray*}
One can then also also define the functions $\tanh$ ({\em tangens hyperbolica}) and $\coth$ ({\em cotangens hyperbolica}) in analogy to the definitions of $\tan$ and $\cot$:
\begin{eqnarray*}
\tanh x&:=&\frac{\sinh x}{\cosh x}=\frac{e^x-e^{-x}}{e^x+e^{-x}},\\
\coth x&:=&\frac{\cosh x}{\sinh x}=\frac{e^x+e^{-x}}{e^x-e^{-x}}.
\end{eqnarray*}
We further define the $\sech$ and $\csch$:
\begin{eqnarray*}
\sech x&:=&\frac{1}{\cosh x}=\frac{2}{e^x+e^{-x}},\\
\csch x&:=&\frac{1}{\sinh x}=\frac{2}{e^x-e^{-x}},
\end{eqnarray*}
where $\cosh x$ resp. $\sinh x$ is not $0$.

\begin{figure}[h]
\begin{centering}
\includegraphics[angle=270, scale=0.6]{hyperbolic.ps}
\caption{Graphs of the hyperbolic functions.}\label{fig:graphs}
\end{centering}
\end{figure}

The hyperbolic functions are named in that way because the hyperbola
$$\frac{x^2}{a^2}-\frac{y^2}{b^2}=1$$
can be written in parametrical form with the equations:
$$x=a\cosh t,\quad y=b\sinh t.$$
This is because of the equation
$$\cosh^2 x-\sinh^2 x=1.$$
There are also addition formulas which are like the ones for trigonometric functions:
\begin{eqnarray*}
\sinh (x\pm y)&=&\sinh x\cosh y\pm\cosh x\sinh y\\
\cosh (x\pm y)&=&\cosh x\cosh y\pm\sinh x\sinh y.
\end{eqnarray*}
The Taylor series for the hyperbolic functions are:
\begin{eqnarray*}
\sinh x&=&\sum_{n=0}^{\infty}\frac{x^{2n+1}}{(2n+1)!}\\
\cosh x&=&\sum_{n=0}^{\infty}\frac{x^{2n}}{(2n)!}.
\end{eqnarray*}
There are  the following \PMlinkescapetext{connections} between the hyperbolic and the trigonometric functions:
\begin{eqnarray*}
\sin x&=&\frac{\sinh (ix)}{i}\\
\cos x&=&\cosh (ix).
\end{eqnarray*}

%%%%%
%%%%%
\end{document}
