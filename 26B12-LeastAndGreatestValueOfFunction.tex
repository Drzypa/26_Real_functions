\documentclass[12pt]{article}
\usepackage{pmmeta}
\pmcanonicalname{LeastAndGreatestValueOfFunction}
\pmcreated{2013-03-22 15:38:57}
\pmmodified{2013-03-22 15:38:57}
\pmowner{pahio}{2872}
\pmmodifier{pahio}{2872}
\pmtitle{least and greatest value of function}
\pmrecord{11}{37581}
\pmprivacy{1}
\pmauthor{pahio}{2872}
\pmtype{Theorem}
\pmcomment{trigger rebuild}
\pmclassification{msc}{26B12}
\pmsynonym{global extrema of real function}{LeastAndGreatestValueOfFunction}
%\pmkeywords{least value}
%\pmkeywords{greatest value}
\pmrelated{Extremum}
\pmrelated{LeastAndGreatestNumber}
\pmrelated{FermatsTheoremStationaryPoints}
\pmrelated{MinimalAndMaximalNumber}
\pmdefines{absolute minimum}
\pmdefines{absolute maximum}

% this is the default PlanetMath preamble.  as your knowledge
% of TeX increases, you will probably want to edit this, but
% it should be fine as is for beginners.

% almost certainly you want these
\usepackage{amssymb}
\usepackage{amsmath}
\usepackage{amsfonts}

% used for TeXing text within eps files
%\usepackage{psfrag}
% need this for including graphics (\includegraphics)
%\usepackage{graphicx}
% for neatly defining theorems and propositions
 \usepackage{amsthm}
% making logically defined graphics
%%%\usepackage{xypic}

% there are many more packages, add them here as you need them

% define commands here

\theoremstyle{definition}
\newtheorem*{thmplain}{Theorem}
\begin{document}
\begin{thmplain}
\, If the real function $f$ is
\begin{enumerate}
\item continuous on the closed interval\, $[a,\,b]$\, and
\item differentiable on the open interval\, $(a,\,b)$,
\end{enumerate}
then the function has on the interval\, $[a,\,b]$\, a least value and a greatest value.\, These are always got in the end of the interval or in the zero of the derivative.
\end{thmplain}

\textbf{Remark 1.}\, If the preconditions of the theorem are fulfilled by a function $f$, then one needs only to determine the values of $f$ in the end points $a$ and $b$ of the interval and in the zeros of the derivative $f'$ inside the interval; then the least and the greatest value are found among those values.\\

\textbf{Remark 2.}\, Note that the theorem does not require anything of the derivative $f'$ in the points $a$ and $b$; one needs not even the right-sided derivative in $a$ or the left-sided derivative in $b$.\, Thus e.g. the function\, $f:\,x \mapsto \sqrt{1-x^2}$,\, fulfilling the conditions of the theorem on the interval\, $[-1,\,1]$\, but not having such one-sided derivatives, gains its least value in the end-point\, $x = -1$\, and its greatest value in the zero\, $x = 0$\, of the derivative.\\

\textbf{Remark 3.}\, The least value of a function is also called the \emph{absolute minimum} and the greatest value the \emph{absolute maximum} of the function.
%%%%%
%%%%%
\end{document}
