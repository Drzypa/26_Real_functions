\documentclass[12pt]{article}
\usepackage{pmmeta}
\pmcanonicalname{BibliographyForRealAnalysis}
\pmcreated{2013-03-22 15:39:55}
\pmmodified{2013-03-22 15:39:55}
\pmowner{rspuzio}{6075}
\pmmodifier{rspuzio}{6075}
\pmtitle{bibliography for real analysis}
\pmrecord{14}{37599}
\pmprivacy{1}
\pmauthor{rspuzio}{6075}
\pmtype{Topic}
\pmcomment{trigger rebuild}
\pmclassification{msc}{26-00}

\endmetadata

% this is the default PlanetMath preamble.  as your knowledge
% of TeX increases, you will probably want to edit this, but
% it should be fine as is for beginners.

% almost certainly you want these
\usepackage{amssymb}
\usepackage{amsmath}
\usepackage{amsfonts}

% used for TeXing text within eps files
%\usepackage{psfrag}
% need this for including graphics (\includegraphics)
%\usepackage{graphicx}
% for neatly defining theorems and propositions
%\usepackage{amsthm}
% making logically defined graphics
%%%\usepackage{xypic}

% there are many more packages, add them here as you need them

% define commands here
\begin{document}
\section{Introductory}

Ralph P. Boas, {\em A Primer of Real Functions}, MAA, 1972.

Edward Gaughan, {\em Introduction to Analysis}, Brooks/Cole, 1968.

Serge Lang, {\em Undergraduate Analysis}, Springer-Verlag, 1968.

Walter Rudin, {\em Principles of Mathematical Analysis}, McGraw-Hill, 1964.

Georgi A. Shilov, {\em Elementary Real and Complex Analysis}, MIT, 1973.




\section{Advanced}

Sterling K. Berberian, {\em Measure and Integration}, Chelsea, 1970.

Edwin Hewitt and Karl Stromberg, {\em Real and Abstract Analysis}, Springer-Verlag, 1969.

H. L. Royden, {\em Real Analysis}, MacMillan, 1968.

Walter Rudin, {\em Real \& Complex Analysis}, McGraw-Hill, 1987.

Ernst Lindel\"of, {\em Differentiali- ja integralilasku ja sen sovellutukset} I -- IV (Differential and integral calculus and its applications), WSOY and Mercatorin Kirjapaino, Helsinki, 1928--1946.

G. E. Shilov and B. L. Gurevich, {\em Integral, Measure and Derivative: A Unified Approach}, Prentice-Hall, 1966.



\section{Additional Topic}

non-Newtonian calculus


%%%%%
%%%%%
\end{document}
