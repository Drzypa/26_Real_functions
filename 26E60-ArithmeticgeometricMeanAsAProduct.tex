\documentclass[12pt]{article}
\usepackage{pmmeta}
\pmcanonicalname{ArithmeticgeometricMeanAsAProduct}
\pmcreated{2013-03-22 17:09:59}
\pmmodified{2013-03-22 17:09:59}
\pmowner{rspuzio}{6075}
\pmmodifier{rspuzio}{6075}
\pmtitle{arithmetic-geometric mean as a product}
\pmrecord{6}{39478}
\pmprivacy{1}
\pmauthor{rspuzio}{6075}
\pmtype{Derivation}
\pmcomment{trigger rebuild}
\pmclassification{msc}{26E60}
\pmclassification{msc}{33E05}

% this is the default PlanetMath preamble.  as your knowledge
% of TeX increases, you will probably want to edit this, but
% it should be fine as is for beginners.

% almost certainly you want these
\usepackage{amssymb}
\usepackage{amsmath}
\usepackage{amsfonts}

% used for TeXing text within eps files
%\usepackage{psfrag}
% need this for including graphics (\includegraphics)
%\usepackage{graphicx}
% for neatly defining theorems and propositions
%\usepackage{amsthm}
% making logically defined graphics
%%%\usepackage{xypic}

% there are many more packages, add them here as you need them

% define commands here

\begin{document}
Recall that, given two real numbers $0 < x \le y$, their arithmetic-geometric
mean may be defined as $M(x,y) = \lim_{n \to \infty} g_n$, where
\begin{align*}
g_0 &= x \\
a_0 &= y \\
g_{n+1} &= \sqrt{a_n g_n} \\
a_{n+1} &= {a_n + g_n \over 2} .
\end{align*}
In this entry, we will re-express this quantity as an infinite product.
We begin by rewriting the recursion for $g_n$:
\[
g_{n+1} = \sqrt{a_n g_n} =
\sqrt{ {a_n \over g_n} \cdot g_n^2 } =
g_n \sqrt{a_n \over g_n}
\]
From this, it follows that
\[
g_n = g_0 \prod_{m=0}^{n-1} h_m
\]
where $h_n = \sqrt{a_n / g_n}$.

As it stands, this is not so interesting because no way has been given
to determine the factors $h_n$ other than first computing $a_n$ and 
$g_n$.  We shall now correct this defect by deriving a recursion which
may be used to compute the $h_n$'s directly:
\begin{align*}
h_{n+1} &= \sqrt{a_{n+1} \over g_{n+1}} \\
&= \sqrt{a_n + g_n \over 2 \sqrt{a_n g_n}} \\
&= \sqrt{ {1 \over 2} 
\left(
\sqrt{a_n \over g_n} +
\sqrt{g_n \over a_n}
\right)} \\
&= \sqrt{ {1 \over 2}
\left(
h_n + {1 \over h_n}
\right)} \\
&= \sqrt{h_n^2 + 1 \over 2 h_n}
\end{align*}

Taking the limit $n \to \infty$, we then have the formula
\[
M(x,y) = x \prod_{m=0}^\infty h_n
\]
where
\[
h_0 = {y \over x}
\]
and
\[
h_{n+1} = \sqrt{h_n^2 + 1 \over 2 h_n} .
\]
%%%%%
%%%%%
\end{document}
