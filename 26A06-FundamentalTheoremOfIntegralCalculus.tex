\documentclass[12pt]{article}
\usepackage{pmmeta}
\pmcanonicalname{FundamentalTheoremOfIntegralCalculus}
\pmcreated{2013-03-22 18:50:49}
\pmmodified{2013-03-22 18:50:49}
\pmowner{pahio}{2872}
\pmmodifier{pahio}{2872}
\pmtitle{fundamental theorem of integral calculus}
\pmrecord{7}{41655}
\pmprivacy{1}
\pmauthor{pahio}{2872}
\pmtype{Theorem}
\pmcomment{trigger rebuild}
\pmclassification{msc}{26A06}
\pmrelated{FundamentalTheoremOfCalculusClassicalVersion}
\pmrelated{VanishingOfGradientInDomain}

\endmetadata

% this is the default PlanetMath preamble.  as your knowledge
% of TeX increases, you will probably want to edit this, but
% it should be fine as is for beginners.

% almost certainly you want these
\usepackage{amssymb}
\usepackage{amsmath}
\usepackage{amsfonts}

% used for TeXing text within eps files
%\usepackage{psfrag}
% need this for including graphics (\includegraphics)
%\usepackage{graphicx}
% for neatly defining theorems and propositions
 \usepackage{amsthm}
% making logically defined graphics
%%%\usepackage{xypic}

% there are many more packages, add them here as you need them

% define commands here

\theoremstyle{definition}
\newtheorem*{thmplain}{Theorem}

\begin{document}
The derivative of a real function, which has on a whole interval a \PMlinkname{constant}{ConstantFunction} value $c$, vanishes in every point of this interval:
$$\frac{d}{dx}c \;=\; 0$$\\

The converse theorem of this is also true.\, Ernst Lindel\"of calls it the \emph{fundamental theorem of integral calculus} (in Finnish \emph{integraalilaskun peruslause}).\, It can be formulated as

\textbf{Theorem.}\, If a real function in continuous and its derivative vanishes in all points of an interval, the value of this function does not change on this interval.

\emph{Proof.}\, We make the antithesis that there were on the interval two distinct points $x_1$ and $x_2$ with\, $f(x_1) \neq f(x_2)$.\, Then the mean-value theorem guarantees a point $\xi$ between $x_1$ and $x_2$ such that
$$f'(\xi) \;=\; \frac{f(x_1)\!-\!f(x_2)}{x_1\!-\!x_2},$$
which value is distinct from zero.\, This is, however, impossible by the assumption of the theorem.\, So the antithesis is wrong and the theorem \PMlinkescapetext{right}.\\

The contents of the theorem may be expressed also such that if two functions have the same derivative on a whole interval, then the difference of the functions is constant on this interval.\, Accordingly, if $F$ is an antiderivative of a function $f$, then any other antiderivative of $f$ has the form $x \mapsto F(x)\!+\!C$, where $C$ is a constant.






%%%%%
%%%%%
\end{document}
