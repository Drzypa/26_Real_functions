\documentclass[12pt]{article}
\usepackage{pmmeta}
\pmcanonicalname{RationalSineAndCosine}
\pmcreated{2013-03-22 17:54:50}
\pmmodified{2013-03-22 17:54:50}
\pmowner{pahio}{2872}
\pmmodifier{pahio}{2872}
\pmtitle{rational sine and cosine}
\pmrecord{8}{40408}
\pmprivacy{1}
\pmauthor{pahio}{2872}
\pmtype{Theorem}
\pmcomment{trigger rebuild}
\pmclassification{msc}{26A09}
\pmclassification{msc}{11D09}
\pmclassification{msc}{11A67}
\pmrelated{RationalPointsOnTwoDimensionalSphere}
\pmrelated{GreatestCommonDivisor}
\pmrelated{GeometricProofOfPythagoreanTriplet}
\pmrelated{RationalBriggsianLogarithmsOfIntegers}
\pmrelated{AlgebraicSinesAndCosines}

\endmetadata

% this is the default PlanetMath preamble.  as your knowledge
% of TeX increases, you will probably want to edit this, but
% it should be fine as is for beginners.

% almost certainly you want these
\usepackage{amssymb}
\usepackage{amsmath}
\usepackage{amsfonts}

% used for TeXing text within eps files
%\usepackage{psfrag}
% need this for including graphics (\includegraphics)
%\usepackage{graphicx}
% for neatly defining theorems and propositions
 \usepackage{amsthm}
% making logically defined graphics
%%%\usepackage{xypic}

% there are many more packages, add them here as you need them

% define commands here

\theoremstyle{definition}
\newtheorem*{thmplain}{Theorem}

\begin{document}
\textbf{Theorem.}\, The only acute angles, whose sine and cosine are rational, are those determined by the Pythagorean triplets \,$(a,\,b,\,c)$.

{\em Proof.} $1^{\underline{o}}$. When the catheti $a$, $b$ and the hypotenuse $c$ of a right triangle are integers, i.e. they form a Pythagorean triplet, then the sine $\frac{a}{c}$ and the cosine $\frac{b}{c}$ of one of the acute angles of the triangle are rational numbers.

$2^{\underline{o}}$. Let the sine and the cosine of an acute angle $\omega$ be rational numbers 
$$\sin\omega \;=\; \frac{a}{c}, \quad \cos\omega = \frac{b}{d},$$
where the integers $a$, $b$, $c$, $d$ satisfy
\begin{align}
\gcd(a,\,c) \;=\; \gcd(b,\,d) \;=\; 1.
\end{align}
Since the square sum of sine and cosine is always 1, we have\, 
\begin{align}
\frac{a^2}{c^2}\!+\!\frac{b^2}{d^2} \;=\; 1.
\end{align}
\, By removing the denominators we get the Diophantine equation
$$a^2d^2\!+\!b^2c^2 \;=\; c^2d^2.$$
Since two of its terms are divisible by $c^2$, also the third term $a^2d^2$ is divisible by $c^2$.\, But because by (1), the integers $a^2$ and $c^2$ are coprime, we must have \,$c^2 \mid d^2$ (see the corollary of B\'ezout's lemma).\, Similarly, we also must have \,$d^2 \mid c^2$.\, The last divisibility relations mean that\, $c^2 = d^2$,\, whence (2) may be written
$$a^2+b^2 \;=\; c^2,$$
and accordingly the sides $a,\,b,\,c$ of a corresponding right triangle are integers.



%%%%%
%%%%%
\end{document}
