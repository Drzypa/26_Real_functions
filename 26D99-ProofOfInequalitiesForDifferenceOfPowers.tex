\documentclass[12pt]{article}
\usepackage{pmmeta}
\pmcanonicalname{ProofOfInequalitiesForDifferenceOfPowers}
\pmcreated{2013-03-22 15:48:45}
\pmmodified{2013-03-22 15:48:45}
\pmowner{Mathprof}{13753}
\pmmodifier{Mathprof}{13753}
\pmtitle{proof of inequalities for difference of powers}
\pmrecord{14}{37777}
\pmprivacy{1}
\pmauthor{Mathprof}{13753}
\pmtype{Proof}
\pmcomment{trigger rebuild}
\pmclassification{msc}{26D99}

% this is the default PlanetMath preamble.  as your knowledge
% of TeX increases, you will probably want to edit this, but
% it should be fine as is for beginners.

% almost certainly you want these
\usepackage{amssymb}
\usepackage{amsmath}
\usepackage{amsfonts}

% used for TeXing text within eps files
%\usepackage{psfrag}
% need this for including graphics (\includegraphics)
%\usepackage{graphicx}
% for neatly defining theorems and propositions
%\usepackage{amsthm}
% making logically defined graphics
%%%\usepackage{xypic}

% there are many more packages, add them here as you need them

% define commands here
\begin{document}
\section{First Inequality}

We have the factorization
\[ u^n - v^n = (u - v) \sum_{k=0}^{n-1} u^k v^{n-k-1}. \]
Since the largest term in the sum is is $u^{n-1}$ and the smallest is
$v^{n-1}$, and there are $n$ terms in the sum, we deduce the following
inequalities: 
\[ n (u-v) v^{n-1} < u^n - v^n < n (u-v) u^{n-1} \]

\section{Second Inequality}

This inequality is trivial when $x = 0$.  We split the rest of 
the proof into two cases.

\subsection{$-1 < x < 0$}

In this case, we set $u = 1$ and $v = 1 + x$ in the second 
inequality above:
\[ 1 - (1 + x)^n < n (-x) \]
Reversing the signs of both sides yields
\[ nx < (1 + x)^n-1  \]

\subsection{$0 < x$}

In this case, we set $u = 1 + x$ and $v = 1$ in the first
inequality above:
\[ nx <  (1 + x)^n -1\]
 
\section{Third Inequality}

This inequality is trivial when $x = 0$.  We split the rest of 
the proof into two cases.

\subsection{$-1 < x < 0$}

Start with the first inequality for differences of powers, expand
the left-hand side,
\[ n u v^{n-1} - n v^n < u^n - v^n, \]
move the $v^n$ to the other side of the inequality,
\[ n u v^{n-1} - (n-1) v^n < u^n, \]
and divide by $v^n$ to obtain
\[ n {u \over v} - n + 1 < \left( {u \over v} \right)^n. \]
Taking the reciprocal, we obtain
\[ \left( {v \over u} \right)^n < {v \over v + n(u - v)} =
1 - {n (u - v) \over v + n(u-v)} \]
Setting $u = 1$ and $v = 1 + x$, and moving a term from one side
to the other, this becomes
\[ (1 + x)^n - 1 < { n x \over 1 - (n - 1) x}. \]

\subsection{$0 < x < 1/(n-1)$}

Start with the second inequality for differences of powers, expand 
the right-hand side,
\[ u^n - v^n < n u^n - n u^{n-1} v \]
move terms from one side of the inequality to the other,
\[ n u^{n-1} v - (n - 1) u^n < v^n \]
and divide by $u^n$ to obtain
\[ n {v \over u} - n + 1 < \left( {v \over u} \right)^n \]
When the left-hand side is positive, (i.e. $n v > (n - 1) u$)
we can take the reciprocal:
\[ \left( {u \over v} \right)^n < {u \over u - n(u - v)} = 
1 + {n (u - v) \over u  - n (u - v)} \]
Setting $u = 1 + x$ and $v = 1$, and moving a term from one side 
to the other, this becomes
\[ (1 + x)^n - 1 < { n x \over 1 - (n - 1) x} \]
and the positivity condition mentioned above becomes $(n - 1) x < 1$.
%%%%%
%%%%%
\end{document}
