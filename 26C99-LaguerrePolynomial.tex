\documentclass[12pt]{article}
\usepackage{pmmeta}
\pmcanonicalname{LaguerrePolynomial}
\pmcreated{2013-03-22 12:30:56}
\pmmodified{2013-03-22 12:30:56}
\pmowner{rspuzio}{6075}
\pmmodifier{rspuzio}{6075}
\pmtitle{Laguerre polynomial}
\pmrecord{22}{32753}
\pmprivacy{1}
\pmauthor{rspuzio}{6075}
\pmtype{Definition}
\pmcomment{trigger rebuild}
\pmclassification{msc}{26C99}
\pmrelated{HermitePolynomials}
\pmrelated{OrthogonalityOfChebyshevPolynomials}
\pmdefines{associated Laguerre polynomial}
\pmdefines{Laguerre's equation}

% this is the default PlanetMath preamble.  as your knowledge
% of TeX increases, you will probably want to edit this, but
% it should be fine as is for beginners.

% almost certainly you want these
\usepackage{amssymb}
\usepackage{amsmath}
\usepackage{amsfonts}

% used for TeXing text within eps files
%\usepackage{psfrag}
% need this for including graphics (\includegraphics)
\usepackage{graphicx}
% for neatly defining theorems and propositions
%\usepackage{amsthm}
% making logically defined graphics
%%%\usepackage{xypic}

% there are many more packages, add them here as you need them

% define commands here
\begin{document}
\section{Definition}

The \emph{Laguerre polynomials} are orthogonal polynomials with respect to the 
weighting function $e^{-x}$ on the half-line $[0,\infty)$.  They are denoted 
by the letter ``$L$'' with the order as subscript and are normalized by the condition that the coefficient of the highest order term of $L_n$ is 
$(-1)^n / n!$. 

The first few Laguerre poynomials are as follows:
\begin{align*}
L_0 (x) &= 1 \\
L_1 (x) &= -x + 1 \\
L_2 (x) &= \frac{1}{2} x^2 - 2x + 1 \\
L_3 (x) &= -\frac{1}{6} x^3 + \frac{3}{2} x^2 - 3 x + 1 
\end{align*}

A generalization is given by the \emph{associated Laguerre polynomials}
which depends on a parameter (traditionally denoted ``$\alpha$'').  As it 
turns out, they are polynomials of the argument $\alpha$ as as well, so 
they are polynomials of two variables.  They  are defined over the same 
interval with the same normalization condition, but the weight function 
is generalized to $x^\alpha e^{-x}$.  They are notated by including the 
parameter as a parenthesized superscript (not all authors use the
parentheses).

The ordinary Laguerre polynomials are the special case of the generalized
Laguerre polynomials when the parameter goes to zero.  When some result
holds for generalized Laguerre polynomials which is not more complicated
than that for ordinary Laguerre polynomials, we shall only provide the more
general result and leave it to the reader to send the parameter to zero
to recover the more specific result. 

The first few generalized Laguerre polynomials are as follows:

\begin{align*}
L_0^{(\alpha)} (x) &= 1 \\
L_0^{(\alpha)} (x) &= -x + \alpha + 1 \\
L_0^{(\alpha)} (x) &= \frac{1}{2} x^2 - (\alpha + 2) x + \frac{1}{2}
(\alpha + 2) (\alpha + 1) \\
L_0^{(\alpha)} (x) &= - \frac{1}{6} x^3 + \frac{1}{2} (\alpha + 3) x^2 -
\frac{1}{2} (\alpha + 2) (\alpha + 3) x + \frac{1}{6} (\alpha + 1)
(\alpha + 2) (\alpha + 3)
\end{align*}

\section{Formulae for these polynomials}

The Laguerre polynomials may be exhibited explicitly as a sum in terms
of factorials, which may also be written using binomial coefficients:
 \[L_n (x) = \sum_{k=0}^n {n! \over (k!)^2 (n-k)!} (-x)^k =
\sum_{k=0}^n {n \choose k} {(-x)^k \over k!}\]
The generalization may be expressed in terms of gamma functions or
falling factorials:
 \[L_n^{(\alpha)} = \sum_{k=0}^n {\Gamma (n + \alpha + 1) \over
\Gamma (k + \alpha + 1)} {(-x)^k \over k! (n-k)!} = \sum_{k=0}^n
{(n + \alpha)^{\underline{n-k}} \over k! (n-k)!} (-x)^k \]

They can be computed from a Rodrigues formula:
\[L_n^{(\alpha)} (x)= \frac{1}{n!} x^{-\alpha} e^x 
\frac{d^n}{dx^n} \left( e^{-x} x^{n + \alpha} \right)\]

They have several integral representations.  They can be expressed
in terms of a countour integral
\[L_n(x) = \frac{1}{2\pi i} \oint \frac{e^{-\frac{xt} {1-t}}}
{(1-t)t^{n+1}} \, \mathit{dt},\]
where the origin is enclosed by the contour, but not $z=1$.


\section{Equations they satisfy}

The Laguerre polynomials satisfy the orthogonality relation

\[
\int_0^{\infty}e^{-x} x^\alpha L_n^{(\alpha)} (x) 
L_m^{(\alpha)} (x) \, dx = \frac {(n+\alpha)!} {n!} 
\delta_{nm}.
\]

The Laguerre polynomials satisfy the differential equation
\[
x \frac{d^2}{dx^2} L_n^{(\alpha)} (x) + 
(\alpha+1-x) \frac{d}{dx} L_n^{(\alpha)} (x) + 
(n-\alpha) L_n^{(\alpha)} (x) = 0
\]
This equation arises in many contexts such as in the quantum 
mechanical treatment of the hydrogen atom.

%\begin{figure}[htbp]
%\begin{centering}
%\includegraphics[angle=270,scale=0.5]{laguerre.ps}
%\caption{Graphs for the first five Laguerre polynomials}\label{fig:graph}
%\end{centering}
%\end{figure}
%%%%%
%%%%%
\end{document}
