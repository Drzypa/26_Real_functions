\documentclass[12pt]{article}
\usepackage{pmmeta}
\pmcanonicalname{LimitOf1SnnIsOneWhenLimitOfNSnIsZero}
\pmcreated{2013-03-22 15:48:55}
\pmmodified{2013-03-22 15:48:55}
\pmowner{rspuzio}{6075}
\pmmodifier{rspuzio}{6075}
\pmtitle{limit of $(1 + s_n)^n$ is one when limit of $n s_n$ is zero}
\pmrecord{7}{37780}
\pmprivacy{1}
\pmauthor{rspuzio}{6075}
\pmtype{Proof}
\pmcomment{trigger rebuild}
\pmclassification{msc}{26D99}

\endmetadata

% this is the default PlanetMath preamble.  as your knowledge
% of TeX increases, you will probably want to edit this, but
% it should be fine as is for beginners.

% almost certainly you want these
\usepackage{amssymb}
\usepackage{amsmath}
\usepackage{amsfonts}

% used for TeXing text within eps files
%\usepackage{psfrag}
% need this for including graphics (\includegraphics)
%\usepackage{graphicx}
% for neatly defining theorems and propositions
%\usepackage{amsthm}
% making logically defined graphics
%%%\usepackage{xypic}

% there are many more packages, add them here as you need them

% define commands here
\begin{document}
The inequalities for differences of powers may be used to show that
$\lim_{n \to \infty} (1 + s_n)^n = 1$ when $\lim_{n \to \infty} n s_n = 0$.
This fact plays an important role in the development of the theory of the 
exponential function as a limit of powers.

To derive this limit, we bound $1 + s_n$ using the inequalities for differences
of powers.  
\[ n s_n \le (1 + s_n)^n - 1 \le {n s_n \over 1 - (n-1) s_n} \]
Since $\lim_{n \to \infty} n s_n = 0$, there must exist $N$ such that $n s_n < 
1/2$ when $n > N$.  Hence, when $n > N$,
\[ | (1 + s_n)^n - 1 | < 2 |n s_n| \]
so, as $n \to \infty$, we have $(1 + s_n)^n \to 1$.
%%%%%
%%%%%
\end{document}
