\documentclass[12pt]{article}
\usepackage{pmmeta}
\pmcanonicalname{BarbualatsLemma}
\pmcreated{2013-03-22 14:52:31}
\pmmodified{2013-03-22 14:52:31}
\pmowner{jirka}{4157}
\pmmodifier{jirka}{4157}
\pmtitle{Barb\u{a}lat's lemma}
\pmrecord{7}{36552}
\pmprivacy{1}
\pmauthor{jirka}{4157}
\pmtype{Theorem}
\pmcomment{trigger rebuild}
\pmclassification{msc}{26A06}
\pmsynonym{Barbalat's lemma}{BarbualatsLemma}

\endmetadata

% this is the default PlanetMath preamble.  as your knowledge
% of TeX increases, you will probably want to edit this, but
% it should be fine as is for beginners.

% almost certainly you want these
\usepackage{amssymb}
\usepackage{amsmath}
\usepackage{amsfonts}

% used for TeXing text within eps files
%\usepackage{psfrag}
% need this for including graphics (\includegraphics)
%\usepackage{graphicx}
% for neatly defining theorems and propositions
\usepackage{amsthm}
% making logically defined graphics
%%%\usepackage{xypic}

% there are many more packages, add them here as you need them

% define commands here
\theoremstyle{theorem}
\newtheorem*{thm}{Theorem}
\newtheorem*{lemma}{Lemma}
\newtheorem*{conj}{Conjecture}
\newtheorem*{cor}{Corollary}
\theoremstyle{definition}
\newtheorem*{defn}{Definition}
\begin{document}
\begin{lemma}[Barb\u{a}lat]
Let $f \colon (0,\infty) \to {\mathbb{R}}$ be Riemann integrable and uniformly continuous then
\begin{equation*}
\lim_{t \to \infty} f(t) = 0 .
\end{equation*}
\end{lemma}

Note that if $f$ is non-negative, then Riemann integrability is the same as being $L^1$ in the sense of Lebesgue, but if $f$ oscillates then the Lebesgue integral may not exist.

Further note that the uniform continuity is required to prevent sharp ``spikes'' that might prevent the limit from existing.  For example suppose we add a spike of height 1 and area $2^{-n}$ at every integer.  Then the function is continuous and $L^1$ (and thus Riemann integrable), but
$f(t)$ would not have a limit at infinity.

\begin{thebibliography}{9}
\bibitem{LoRy}
Hartmut Logemann, Eugene P.\@ Ryan.
\PMlinkescapetext{Asymptotic behaviour of nonlinear systems}.
\emph{The American Mathematical Monthly}, 111(10):864--889,
2004.
\end{thebibliography}
%%%%%
%%%%%
\end{document}
