\documentclass[12pt]{article}
\usepackage{pmmeta}
\pmcanonicalname{ProofOfArithmeticgeometricharmonicMeansInequality}
\pmcreated{2013-03-22 15:09:37}
\pmmodified{2013-03-22 15:09:37}
\pmowner{Mathprof}{13753}
\pmmodifier{Mathprof}{13753}
\pmtitle{proof of arithmetic-geometric-harmonic means inequality}
\pmrecord{10}{36909}
\pmprivacy{1}
\pmauthor{Mathprof}{13753}
\pmtype{Proof}
\pmcomment{trigger rebuild}
\pmclassification{msc}{26D15}

\endmetadata

% this is the default PlanetMath preamble.  as your knowledge
% of TeX increases, you will probably want to edit this, but
% it should be fine as is for beginners.

% almost certainly you want these
\usepackage{amssymb}
\usepackage{amsmath}
\usepackage{amsfonts}

% used for TeXing text within eps files
%\usepackage{psfrag}
% need this for including graphics (\includegraphics)
%\usepackage{graphicx}
% for neatly defining theorems and propositions
%\usepackage{amsthm}
% making logically defined graphics
%%%\usepackage{xypic}

% there are many more packages, add them here as you need them

% define commands here
\begin{document}
For the Arithmetic Geometric Inequality, I claim it is enough to prove that if
$\prod_{i=1}^n x_i = 1$ with $x_i \geq 0$ then $\sum_{i=1}^n x_i \geq n$. The arithmetic geometric inequality for $y_1,\ldots,y_n$ will follow by taking
$x_i = \frac{y_i}{\sqrt[n]{\prod_{k=1}^n y_k}}$. The geometric harmonic inequality follows from the arithmetic geometric by taking $x_i = \frac{1}{y_i}$.

So, we show that if $\prod_{i=1}^n x_i = 1$ with $x_i \geq 0$ then $\sum_{i=1}^n x_i \geq n$ by induction on $n$.

Clear for $n=1$.

Induction Step: By reordering indices we may assume the $x_i$ are increasing, so $x_{n} \geq 1 \geq x_1$. Assuming the statement is true for $n-1$, we have
$x_2 + \cdots + x_{n-1} + x_1x_{n} \geq n-1$. Then,
\begin{equation*} 
\sum_{i=1}^n x_i \geq n-1 + x_n + x_1-x_1x_n
\end{equation*}
 by adding $x_1+x_n$ to both sides and subtracting $x_1x_n$. And so,
\begin{align*}
\sum_{i=1}^n x_i 
&\geq n+( x_n -1)+ (x_1 -x_1 x_n) \\
&= n + (x_n - 1) - x_1 (x_n -1)\\
&= n + (x_n - 1)(1 - x_1) \\
&\geq n
\end{align*}
The last line follows since $x_n\geq 1 \geq x_1$.
%%%%%
%%%%%
\end{document}
