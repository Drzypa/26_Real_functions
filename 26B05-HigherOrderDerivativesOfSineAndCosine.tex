\documentclass[12pt]{article}
\usepackage{pmmeta}
\pmcanonicalname{HigherOrderDerivativesOfSineAndCosine}
\pmcreated{2013-03-22 14:45:16}
\pmmodified{2013-03-22 14:45:16}
\pmowner{pahio}{2872}
\pmmodifier{pahio}{2872}
\pmtitle{higher order derivatives of sine and cosine}
\pmrecord{13}{36395}
\pmprivacy{1}
\pmauthor{pahio}{2872}
\pmtype{Derivation}
\pmcomment{trigger rebuild}
\pmclassification{msc}{26B05}
\pmclassification{msc}{46G05}
\pmclassification{msc}{26A24}
\pmrelated{FractionalDifferentiation}
\pmrelated{HigherOrderDerivatives}
\pmrelated{ExampleOfTaylorPolynomialsForSinX}
\pmrelated{CosineAtMultiplesOfStraightAngle}

% this is the default PlanetMath preamble.  as your knowledge
% of TeX increases, you will probably want to edit this, but
% it should be fine as is for beginners.

% almost certainly you want these
\usepackage{amssymb}
\usepackage{amsmath}
\usepackage{amsfonts}

% used for TeXing text within eps files
%\usepackage{psfrag}
% need this for including graphics (\includegraphics)
%\usepackage{graphicx}
% for neatly defining theorems and propositions
%\usepackage{amsthm}
% making logically defined graphics
%%%\usepackage{xypic}

% there are many more packages, add them here as you need them

% define commands here
\begin{document}
\PMlinkescapeword{order}

One may consider the sine and cosine either as \PMlinkname{real}{RealFunction} or complex functions.\, In both  cases they are everywhere smooth, having the derivatives of all \PMlinkname{orders}{OrderOfDerivative} in every point.\, The formulae
$$\frac{d^n}{dx^n}\sin{x} \;=\; \sin{(x+n\!\cdot\!\frac{\pi}{2})}$$
and
$$\frac{d^n}{dx^n}\cos{x} \;=\; \cos{(x+n\!\cdot\!\frac{\pi}{2})},$$
where\, $n = 0,\,1,\,2,\,\ldots$ (the derivative of the $0^\mathrm{th}$ order means the function itself), can be proven by induction on $n$.\, Another possibility is to utilize Euler's formula, obtaining
$$\frac{d^n}{dx^n}\cos{x}+i\frac{d^n}{dx^n}\sin{x} \;=\; \frac{d^n}{dx^n}e^{ix} \;=\; 
e^{ix}i^n \;=\; e^{ix+in\frac{\pi}{2}} \;=\; 
\cos{(x+n\!\cdot\!\frac{\pi}{2})}+i\sin{(x+n\!\cdot\!\frac{\pi}{2})};$$
here one has to compare the \PMlinkname{real}{ComplexFunction} and imaginary parts -- supposing that $x$ is real.
%%%%%
%%%%%
\end{document}
