\documentclass[12pt]{article}
\usepackage{pmmeta}
\pmcanonicalname{ContinuityOfNaturalPower}
\pmcreated{2013-03-22 15:39:25}
\pmmodified{2013-03-22 15:39:25}
\pmowner{pahio}{2872}
\pmmodifier{pahio}{2872}
\pmtitle{continuity of natural power}
\pmrecord{7}{37590}
\pmprivacy{1}
\pmauthor{pahio}{2872}
\pmtype{Theorem}
\pmcomment{trigger rebuild}
\pmclassification{msc}{26C05}
\pmclassification{msc}{26A15}
\pmrelated{Exponentiation2}

\endmetadata

% this is the default PlanetMath preamble.  as your knowledge
% of TeX increases, you will probably want to edit this, but
% it should be fine as is for beginners.

% almost certainly you want these
\usepackage{amssymb}
\usepackage{amsmath}
\usepackage{amsfonts}

% used for TeXing text within eps files
%\usepackage{psfrag}
% need this for including graphics (\includegraphics)
%\usepackage{graphicx}
% for neatly defining theorems and propositions
 \usepackage{amsthm}
% making logically defined graphics
%%%\usepackage{xypic}

% there are many more packages, add them here as you need them

% define commands here

\theoremstyle{definition}
\newtheorem*{thmplain}{Theorem}
\begin{document}
\begin{thmplain}
Let $n$ be arbitrary positive integer.\, The power function\, $x\mapsto x^n$\, from $\mathbb{R}$ to $\mathbb{R}$ (or $\mathbb{C}$ to $\mathbb{C}$) is continuous at each point $x_0$.
\end{thmplain}

{\em Proof.}\, Let $\varepsilon$ be any positive number.\, Denote\, $x_0+h = x$\, and\, $x^n-x_0^n = \Delta$.\, Then identically
       $$\Delta \;=\; (x-x_0)(x^{n-1}+x^{n-2}x_0+...+x_0^{n-1}).$$
Taking the absolute value and using the triangle inequality give
  $$|\Delta| \;=\; |h|\cdot|x^{n-1}+x^{n-2}x_0+...+x_0^{n-1}| \;\leqq\;
    |h|\cdot(|x^{n-1}|+|x^{n-2}x_0|+...+|x_0^{n-1}|).$$
But since\, $|x| = |x_0+h| \leqq |x_0|+|h|$\, and also\, $|x_0| \leqq |x_0|+|h|$,\, so each summand in the parentheses is at most equal to\, $(|x_0|+|h|)^{n-1}$,\, and since there are $n$ summands, the sum is at most equal to $n(|x_0|+|h|)^{n-1}$.\, Thus we get
          $$|\Delta| \;\leqq\; n|h|(|x_0|+|h|)^{n-1}.$$
We may choose\, $|h| < 1$;\, this implies
            $$|\Delta| \;\leqq\; n|h|(|x_0|+1)^{n-1}.$$
The right hand side of this inequality is less than $\varepsilon$ as soon as we still require
                $$|h| \;<\; \frac{\varepsilon}{n(|x_0|+1)^{n-1}}.$$
This means that the power function\, $x\mapsto x^n$ is continuous at the point $x_0$.\\

\textbf{Note.}\, Another way to prove the theorem is to use induction on $n$ and the rule 2 in limit rules of functions.
%%%%%
%%%%%
\end{document}
