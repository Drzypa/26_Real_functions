\documentclass[12pt]{article}
\usepackage{pmmeta}
\pmcanonicalname{ExampleOfAStrictlyIncreasingQuasisymmetricSingularFunction}
\pmcreated{2013-03-22 14:10:37}
\pmmodified{2013-03-22 14:10:37}
\pmowner{jirka}{4157}
\pmmodifier{jirka}{4157}
\pmtitle{example of a strictly increasing quasisymmetric singular function}
\pmrecord{5}{35603}
\pmprivacy{1}
\pmauthor{jirka}{4157}
\pmtype{Example}
\pmcomment{trigger rebuild}
\pmclassification{msc}{26A30}

% this is the default PlanetMath preamble.  as your knowledge
% of TeX increases, you will probably want to edit this, but
% it should be fine as is for beginners.

% almost certainly you want these
\usepackage{amssymb}
\usepackage{amsmath}
\usepackage{amsfonts}

% used for TeXing text within eps files
%\usepackage{psfrag}
% need this for including graphics (\includegraphics)
%\usepackage{graphicx}
% for neatly defining theorems and propositions
%\usepackage{amsthm}
% making logically defined graphics
%%%\usepackage{xypic}

% there are many more packages, add them here as you need them

% define commands here
\begin{document}
An example of a strictly increasing quasisymmetric function that also a purely singular function can be defined as:
\begin{equation*}
f(x) = \lim_{k \rightarrow \infty} \int_0^x \prod_{i=1}^k (1 + \lambda \cos n_i s) ds ,
\end{equation*}
where $0< \lambda < 1$ and carefully picked $n_i$.
We can pick the $n_i$ such that $n_{i+1}$ is strictly
greater then $\sum_{j=1}^i n_j$.  However if we pick the $\lambda$
and $n_i$ more carefully, we can construct functions with the quasisymmetricity
constant as close to 1 as we want.  That is, we can construct functions such that
\begin{equation*}
\frac{1}{M} 
\leq
\frac{f(x+t)-f(x)}{f(x)-f(x-t)}
\leq
M
\end{equation*}
for all $x$ and $t$ where $M$ is as close to 1 as we want.  If $M=1$ note that
the function must be a straight line.

It is also possible from this to construct a quasiconformal mapping of the upper half plane to itself by extending this function to the whole real line and then using the Beurling-Ahlfors quasiconformal extension.  Then we'd have a quasiconformal mapping such that its boundary correspondence would be a purely singular function.

For more detailed explanation, \PMlinkescapetext{graphs} and
proof (it is too long to reproduce here) see bibliography.

{\bf Bibliography}

\begin{itemize}
\item
A.\@ Beurling, L\@. V.\@ Ahlfors.  \PMlinkescapetext{The boundary
correspondence under quasiconformal mappings}.  \emph{Acta Math.}, 96:125-142,
1956.
\item
J.\@ Lebl.  \emph{\PMlinkescapetext{Quasiconformal Extensions of Quasisymmetric
Mappings}}.  \PMlinkescapetext{Masters thesis, San Diego State University, San
Diego, CA, May 2003}.  Also available at
\PMlinkexternal{http://www.jirka.org/thesis.pdf}{http://www.jirka.org/thesis.pdf}
\end{itemize}
%%%%%
%%%%%
\end{document}
