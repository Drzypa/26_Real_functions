\documentclass[12pt]{article}
\usepackage{pmmeta}
\pmcanonicalname{InequalitiesForDifferencesOfPowers}
\pmcreated{2013-03-22 15:48:42}
\pmmodified{2013-03-22 15:48:42}
\pmowner{rspuzio}{6075}
\pmmodifier{rspuzio}{6075}
\pmtitle{inequalities for differences of powers}
\pmrecord{11}{37776}
\pmprivacy{1}
\pmauthor{rspuzio}{6075}
\pmtype{Theorem}
\pmcomment{trigger rebuild}
\pmclassification{msc}{26D99}
\pmrelated{BernoullisInequality}

% this is the default PlanetMath preamble.  as your knowledge
% of TeX increases, you will probably want to edit this, but
% it should be fine as is for beginners.

% almost certainly you want these
\usepackage{amssymb}
\usepackage{amsmath}
\usepackage{amsfonts}

% used for TeXing text within eps files
%\usepackage{psfrag}
% need this for including graphics (\includegraphics)
%\usepackage{graphicx}
% for neatly defining theorems and propositions
%\usepackage{amsthm}
% making logically defined graphics
%%%\usepackage{xypic}

% there are many more packages, add them here as you need them

% define commands here
\begin{document}
Oftentimes, one needs to estimate differences of powers of real numbers.
The following inequalities are useful for this purpose:
\[ n (u-v) v^{n-1} < u^n - v^n < n (u-v) u^{n-1} \]
\[ n x \le (1 + x)^n - 1 \]
\[ (1 + x)^n - 1 \le { n x \over 1 - (n - 1) x} \]

Here $n$ is an integer greater than 1.
The first inequality holds when $0 < v < u$, the second inequality holds
when $-1 < x $, and the third inequality holds when $-1 < x < 1/(n-1)$.
Equality can only occur in the latter two inequalities when $x = 0$.
%%%%%
%%%%%
\end{document}
