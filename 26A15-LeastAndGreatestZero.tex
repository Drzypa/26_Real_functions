\documentclass[12pt]{article}
\usepackage{pmmeta}
\pmcanonicalname{LeastAndGreatestZero}
\pmcreated{2013-03-22 16:33:22}
\pmmodified{2013-03-22 16:33:22}
\pmowner{pahio}{2872}
\pmmodifier{pahio}{2872}
\pmtitle{least and greatest zero}
\pmrecord{6}{38742}
\pmprivacy{1}
\pmauthor{pahio}{2872}
\pmtype{Theorem}
\pmcomment{trigger rebuild}
\pmclassification{msc}{26A15}
\pmsynonym{zeroes of continuous function}{LeastAndGreatestZero}
%\pmkeywords{continuous real function}
\pmrelated{ZeroesOfAnalyticFunctionsAreIsolated}

% this is the default PlanetMath preamble.  as your knowledge
% of TeX increases, you will probably want to edit this, but
% it should be fine as is for beginners.

% almost certainly you want these
\usepackage{amssymb}
\usepackage{amsmath}
\usepackage{amsfonts}

% used for TeXing text within eps files
%\usepackage{psfrag}
% need this for including graphics (\includegraphics)
%\usepackage{graphicx}
% for neatly defining theorems and propositions
 \usepackage{amsthm}
% making logically defined graphics
%%%\usepackage{xypic}

% there are many more packages, add them here as you need them

% define commands here

\theoremstyle{definition}
\newtheorem*{thmplain}{Theorem}

\begin{document}
\textbf{Theorem.}\, If a real function $f$ is continuous on the interval\, $[a,\,b]$\, and has zeroes on this interval, then $f$ has a least zero and a greatest zero.

{\em Proof.}\, If\, $f(a) = 0$\, then the assertion concerning the least zero is true.\, Let's assume therefore, that\, $f(a) \neq 0$.\, 

The set\, $A = \{x\in [a,\,b]\vdots\,\, f(x) = 0\}$\, is bounded from below since all numbers of $A$ are greater than $a$.\, Let the \PMlinkname{infimum}{InfimumAndSupremumForRealNumbers} of $A$ be $\xi$.\, Let us make the antithesis, that\, $f(\xi) \neq 0$.\, Then, by the continuity of $f$, there is a positive number $\delta$ such that
$$f(x) \neq 0\quad \mathrm{always\,when}\,\,|x-\xi| < \delta.$$
Chose a number $x_1$ between $\xi$ and $\xi\!+\!\delta$; then\, $f(x_1) \neq 0$,\, but this number $x_1$ is not a lower bound of $A$.\, Therefore there exists a member $a_1$ of $A$ which is less than $x_1$ ($\xi < a_1 < x_1$).\, Now\, $|a_1-\xi| < |x_1-\xi| < \delta$,\, whence this member of $A$ ought to satisfy that\, $f(a_1) = 0$.\, This \PMlinkescapetext{contains} a contradiction.\, Thus the antithesis is wrong, and\, $f(\xi) = 0$.

This \PMlinkescapetext{means} that\, $\xi\in A$\, and $\xi$ is the least number of $A$.

Analogically one shows that the supremum of $A$ is the greatest zero of $f$ on the interval.
%%%%%
%%%%%
\end{document}
