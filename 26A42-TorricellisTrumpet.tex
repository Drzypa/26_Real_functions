\documentclass[12pt]{article}
\usepackage{pmmeta}
\pmcanonicalname{TorricellisTrumpet}
\pmcreated{2013-03-22 17:17:53}
\pmmodified{2013-03-22 17:17:53}
\pmowner{pahio}{2872}
\pmmodifier{pahio}{2872}
\pmtitle{Torricelli's trumpet}
\pmrecord{14}{39643}
\pmprivacy{1}
\pmauthor{pahio}{2872}
\pmtype{Definition}
\pmcomment{trigger rebuild}
\pmclassification{msc}{26A42}
\pmclassification{msc}{26A36}
\pmclassification{msc}{57M20}
\pmclassification{msc}{51M04}
\pmsynonym{Gabriel's horn}{TorricellisTrumpet}

\endmetadata

% this is the default PlanetMath preamble.  as your knowledge
% of TeX increases, you will probably want to edit this, but
% it should be fine as is for beginners.

% almost certainly you want these
\usepackage{amssymb}
\usepackage{amsmath}
\usepackage{amsfonts}

% used for TeXing text within eps files
%\usepackage{psfrag}
% need this for including graphics (\includegraphics)
%\usepackage{graphicx}
% for neatly defining theorems and propositions
 \usepackage{amsthm}
% making logically defined graphics
%%%\usepackage{xypic}

% there are many more packages, add them here as you need them

% define commands here

\theoremstyle{definition}
\newtheorem*{thmplain}{Theorem}

\begin{document}
\PMlinkescapeword{even}

\emph{Torricelli's trumpet} is a fictional infinitely long solid of revolution formed when the closed domain
\[
  A := \{(x,\,y)\in\mathbb{R}^2\,\vdots\;\; x \ge 1,\; 0 \le y \le \frac{1}{x}\}
\]
rotates about the $x$-axis.  It has a finite volume, $\pi$ volume
\PMlinkescapetext{units}, but the area of its surface is infinite; in
fact even the area of $A$ is infinite, i.e., the improper integral
$\displaystyle\int_1^\infty\frac{1}{x}\,dx$ is not convergent.

Torricelli's trumpet is surprising since it can be filled by a finite
amount of paint, but this paint can never suffice for painting its
surface, no matter how \PMlinkescapetext{thin} a coat of paint is
used!

%%%%%
%%%%%
\end{document}
