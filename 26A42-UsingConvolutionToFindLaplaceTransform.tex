\documentclass[12pt]{article}
\usepackage{pmmeta}
\pmcanonicalname{UsingConvolutionToFindLaplaceTransform}
\pmcreated{2013-03-22 18:44:05}
\pmmodified{2013-03-22 18:44:05}
\pmowner{pahio}{2872}
\pmmodifier{pahio}{2872}
\pmtitle{using convolution to find Laplace transform}
\pmrecord{12}{41504}
\pmprivacy{1}
\pmauthor{pahio}{2872}
\pmtype{Example}
\pmcomment{trigger rebuild}
\pmclassification{msc}{26A42}
\pmclassification{msc}{44A10}
\pmrelated{ErrorFunction}
\pmrelated{SubstitutionNotation}
\pmrelated{IntegrationOfLaplaceTransformWithRespectToParameter}

\endmetadata

% this is the default PlanetMath preamble.  as your knowledge
% of TeX increases, you will probably want to edit this, but
% it should be fine as is for beginners.

% almost certainly you want these
\usepackage{amssymb}
\usepackage{amsmath}
\usepackage{amsfonts}

% used for TeXing text within eps files
%\usepackage{psfrag}
% need this for including graphics (\includegraphics)
%\usepackage{graphicx}
% for neatly defining theorems and propositions
%\usepackage{amsthm}
% making logically defined graphics
%%%\usepackage{xypic}

% there are many more packages, add them here as you need them

% define commands here
\newcommand{\sijoitus}[2]%
{\operatornamewithlimits{\Big/}_{\!\!\!#1}^{\,#2}}
\begin{document}
\PMlinkescapeword{formula}

We start from the \PMlinkescapetext{relations} (see the table of Laplace transforms)
\begin{align}
e^{\alpha t} \;\curvearrowleft\; \frac{1}{s\!-\!\alpha}, 
\quad \frac{1}{\sqrt{t}} \;\curvearrowleft\; \sqrt{\frac{\pi}{s}} \qquad (s > \alpha)
\end{align}
where the curved \PMlinkescapetext{arrows point} from the Laplace-transformed functions to the original functions.\, Setting\, $\alpha = a^2$\, and dividing by $\sqrt{\pi}$ in (1), the convolution property of Laplace transform yields
$$\frac{1}{(s\!-\!a^2)\sqrt{s}} \;\;\curvearrowright\;\; 
e^{a^2t}*\frac{1}{\sqrt{\pi t}} \;=\; \int_0^t\!e^{a^2(t-u)}\frac{1}{\sqrt{\pi u}}\,du.$$
The \PMlinkname{substitution}{ChangeOfVariableInDefiniteIntegral} \,$a^2u = x^2$\, then gives
$$\frac{1}{(s\!-\!a^2)\sqrt{s}} \;\curvearrowright\;
\frac{e^{a^2t}}{\sqrt{pi}}\int_0^{a\sqrt{t}}\!e^{-x^2}\!\cdot\!\frac{a}{x}\!\cdot\!\frac{2x}{a^2}\,dx
\;=\; \frac{e^{a^2t}}{a}\!\cdot\!\frac{2}{\sqrt{\pi}}\int_0^{a\sqrt{t}}\!e^{-x^2}\,dx 
\;=\; \frac{e^{a^2t}}{a}\,{\rm erf}\,a\sqrt{t}.$$
Thus we may write the formula
\begin{align}
\mathcal{L}\{e^{a^2t}\,{\rm erf}\,a\sqrt{t}\} \;=\; \frac{a}{(s\!-\!a^2)\sqrt{s}} \qquad (s > a^2).
\end{align}

Moreover, we obtain
$$\frac{1}{(\sqrt{s}\!+\!a)\sqrt{s}} \;=\; \frac{\sqrt{s}\!-\!a}{(s\!-\!a^2)\sqrt{s}} \;=\, 
\frac{1}{s-a^2}-\frac{a}{(s-a^2)\sqrt{s}} \;\curvearrowright\; 
e^{a^2t}-e^{a^2t}\,{\rm erf}\,a\sqrt{t} \;=\; e^{a^2t}(1-{\rm erf}\,a\sqrt{t}),$$
whence we have the other formula
\begin{align}
\mathcal{L}\{e^{a^2t}\,{\rm erfc}\,a\sqrt{t}\} \;=\; \frac{1}{(a\!+\!\sqrt{s})\sqrt{s}}.
\end{align}

\subsection{An improper integral}

One can utilise the formula (3) for evaluating the improper integral
$$\int_0^\infty\frac{e^{-x^2}}{a^2\!+\!x^2}\,dx.$$
We have
$$e^{-tx^2} \;\curvearrowleft\; \frac{1}{s\!+\!x^2}$$
(see the \PMlinkname{table of Laplace transforms}{TableOfLaplaceTransforms}).\, Dividing this by $a^2\!+\!x^2$ and integrating from 0 to $\infty$, we can continue as follows:
\begin{align*}
\int_0^\infty\frac{e^{-tx^2}}{a^2\!+\!x^2}\,dx & \;\curvearrowleft\; \int_0^\infty\frac{dx}{(a^2\!+\!x^2)(s\!+\!x^2)}
\;=\; \frac{1}{s\!-\!a^2}\int_0^\infty\left(\frac{1}{a^2\!+\!x^2}-\frac{1}{s\!+\!x^2}\right)dx\\
& \;=\; \frac{1}{s\!-\!a^2}\sijoitus{x=0}{\quad\infty}\left(\frac{1}{a}\arctan\frac{x}{a}-\frac{1}{\sqrt{s}}\arctan\frac{x}{\sqrt{s}}\right)\\
& \;=\; \frac{1}{s\!-\!a^2}\!\cdot\!\frac{\pi}{2}\left(\frac{1}{a}-\frac{1}{\sqrt{s}}\right)
 \;=\; \frac{\pi}{2a}\!\cdot\!\frac{1}{(a\!+\!\sqrt{s})\sqrt{s}}\\
& \;\curvearrowright\; \frac{\pi}{2a}e^{a^2t}\,{\rm erfc}\,a\sqrt{t}
\end{align*}
Consequently, 
$$\int_0^\infty\frac{e^{-tx^2}}{a^2\!+\!x^2}\,dx \;=\; \frac{\pi}{2a}e^{a^2t}\,{\rm erfc}\,a\sqrt{t},$$
and especially
$$\int_0^\infty\frac{e^{-x^2}}{a^2\!+\!x^2}\,dx \;=\; \frac{\pi}{2a}e^{a^2}\,{\rm erfc}\,a.$$

%%%%%
%%%%%
\end{document}
