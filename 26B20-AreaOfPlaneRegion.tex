\documentclass[12pt]{article}
\usepackage{pmmeta}
\pmcanonicalname{AreaOfPlaneRegion}
\pmcreated{2013-03-22 15:17:46}
\pmmodified{2013-03-22 15:17:46}
\pmowner{pahio}{2872}
\pmmodifier{pahio}{2872}
\pmtitle{area of plane region}
\pmrecord{14}{37094}
\pmprivacy{1}
\pmauthor{pahio}{2872}
\pmtype{Topic}
\pmcomment{trigger rebuild}
\pmclassification{msc}{26B20}
\pmclassification{msc}{26A42}
\pmsynonym{planar area}{AreaOfPlaneRegion}
%\pmkeywords{area of ellipse}
\pmrelated{Area2}
\pmrelated{DefiniteIntegral}
\pmrelated{PolarCurve}
\pmrelated{RiemannMultipleIntegral}
\pmrelated{PropertiesOfEllipse}
\pmrelated{AreaBoundedByArcAndTwoLines}

% this is the default PlanetMath preamble.  as your knowledge
% of TeX increases, you will probably want to edit this, but
% it should be fine as is for beginners.

% almost certainly you want these
\usepackage{amssymb}
\usepackage{amsmath}
\usepackage{amsfonts}

% used for TeXing text within eps files
%\usepackage{psfrag}
% need this for including graphics (\includegraphics)
%\usepackage{graphicx}
% for neatly defining theorems and propositions
 \usepackage{amsthm}
% making logically defined graphics
%%%\usepackage{xypic}

% there are many more packages, add them here as you need them

% define commands here

\theoremstyle{definition}
\newtheorem*{thmplain}{Theorem}
\begin{document}
Let the contour of the region in the $xy$-plane be a closed curve $P$.\, Then the area of the region equals to the path integral
\begin{align}       
           A \;=\; \frac{1}{2}\oint_P (x\,dy-y\,dx)
\end{align}
taken in the positive (i.e. anticlockwise) circling direction.

\textbf{Remarks}
\begin{enumerate}
 \item The \PMlinkescapetext{formula} (1) can be gotten as a special case of Green's theorem by setting\, $\vec{F} := \frac{1}{2}(-y,\,x)$.
 \item Because \,$x\,dy+y\,dx = d(xy)$, \,we have
      $$0 \;=\; \frac{1}{2}\oint_P (x\,dy+y\,dx).$$
This equation may be added to or subtracted from (1), giving the alternative forms
\begin{align}
     A \;=\; \oint_P x\,dy \;=\; -\oint_P y\,dx.
\end{align}
 \item The formulae (1) and (2) \PMlinkescapetext{contain} all other formulae concerning the planar area computing, e.g.
$$A \;=\; \int_a^b f(x)\,dx,$$
$$A \;=\; \frac{1}{2}\int_{\varphi_1}^{\varphi_2}[r(\varphi)]^2\,d\varphi,$$
the former of which is factually same as the latter form of (2).
\end{enumerate}

\textbf{Example.}\, The ellipse \,$\frac{x^2}{a^2}+\frac{y^2}{b^2} = 1$\, has the parametric \PMlinkescapetext{presentation}\, $x = a\cos{t}$,\, $y = b\sin{t}$\, ($0 \leqq t < 2\pi$).\, We have
   $$x\,dy-y\,dx \;=\; [a\cos{t}\cdot b\cos{t}+b\sin{t}\cdot a\sin{t}]\,dt \;=\; ab\,dt,$$
and hence (1) gives for the area of the ellipse
   $$A \;=\; \frac{1}{2}ab\!\int_0^{2\pi}\!dt \;=\; \pi ab.$$
%%%%%
%%%%%
\end{document}
