\documentclass[12pt]{article}
\usepackage{pmmeta}
\pmcanonicalname{PropertiesOfHyperrealsUnderFieldOperations}
\pmcreated{2013-03-22 17:26:19}
\pmmodified{2013-03-22 17:26:19}
\pmowner{asteroid}{17536}
\pmmodifier{asteroid}{17536}
\pmtitle{properties of  hyperreals under field operations}
\pmrecord{4}{39818}
\pmprivacy{1}
\pmauthor{asteroid}{17536}
\pmtype{Result}
\pmcomment{trigger rebuild}
\pmclassification{msc}{26E35}

\endmetadata

% this is the default PlanetMath preamble.  as your knowledge
% of TeX increases, you will probably want to edit this, but
% it should be fine as is for beginners.

% almost certainly you want these
\usepackage{amssymb}
\usepackage{amsmath}
\usepackage{amsfonts}

% used for TeXing text within eps files
%\usepackage{psfrag}
% need this for including graphics (\includegraphics)
%\usepackage{graphicx}
% for neatly defining theorems and propositions
%\usepackage{amsthm}
% making logically defined graphics
%%%\usepackage{xypic}

% there are many more packages, add them here as you need them

% define commands here

\begin{document}
Let ${}^*\mathbb{R}_b$ denote the set of finite (or limited) hyperreal numbers and ${}^*\mathbb{R}_0$ the set of infinitesimal hyperreal numbers.

{\bf \PMlinkescapetext{Proposition} -} We have that
\begin{enumerate}
\item ${}^*\mathbb{R}_b$ and ${}^*\mathbb{R}_0$ are subrings of ${}^*\mathbb{R}$.
\item ${}^*\mathbb{R}_0$ is an ideal of ${}^*\mathbb{R}_b$.
\item the sum of an infinite hyperreal with a finite hyperreal is infinite.
\item the inverse of a non-zero infinitesimal hyperreal is infinite.
\item the inverse of an infinite hyperreal is infinitesimal.
\end{enumerate}

The above properties can be described more informally like:
\begin{enumerate}
\item \emph{finite} $+$ \emph{finite} $=$ \emph{finite}
\item \emph{infinitesimal} $+$ \emph{infinitesimal} $=$ \emph{infinitesimal}
\item \emph{infinite} $+$ \emph{finite} $=$ \emph{infinite}
\item \emph{finite} $\times$ \emph{finite} $=$ \emph{finite}
\item \emph{infinitesimal} $\times$ \emph{finite} $=$ \emph{infinitesimal}
\item \emph{infinitesimal}$^{-1}$ $=$  \emph{infinite}
\item \emph{infinite}$^{-1}$ $=$ \emph{infinitesimal}
\end{enumerate}
%%%%%
%%%%%
\end{document}
