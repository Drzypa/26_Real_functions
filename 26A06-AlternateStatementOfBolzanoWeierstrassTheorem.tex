\documentclass[12pt]{article}
\usepackage{pmmeta}
\pmcanonicalname{AlternateStatementOfBolzanoWeierstrassTheorem}
\pmcreated{2013-03-22 16:40:13}
\pmmodified{2013-03-22 16:40:13}
\pmowner{mathcam}{2727}
\pmmodifier{mathcam}{2727}
\pmtitle{alternate statement of Bolzano-Weierstrass theorem}
\pmrecord{7}{38877}
\pmprivacy{1}
\pmauthor{mathcam}{2727}
\pmtype{Theorem}
\pmcomment{trigger rebuild}
\pmclassification{msc}{26A06}
%\pmkeywords{limit point}
%\pmkeywords{cluster point}
%\pmkeywords{accumulation point}
%\pmkeywords{supremum}
%\pmkeywords{complete}
%\pmkeywords{least upper bound}
%\pmkeywords{completeness}
\pmrelated{BolzanoWeierstrassTheorem}
\pmrelated{LimitPoint}
\pmrelated{Bounded}
\pmrelated{Infinite}

% this is the default PlanetMath preamble.  as your knowledge
% of TeX increases, you will probably want to edit this, but
% it should be fine as is for beginners.

% almost certainly you want these
\usepackage{amssymb}
\usepackage{amsmath}
\usepackage{amsfonts}
\usepackage{amsthm}

% used for TeXing text within eps files
%\usepackage{psfrag}
% need this for including graphics (\includegraphics)
%\usepackage{graphicx}
% for neatly defining theorems and propositions
%\usepackage{amsthm}
% making logically defined graphics
%%%\usepackage{xypic}

% there are many more packages, add them here as you need them

% define commands here
\theoremstyle{plain}
\newtheorem*{thm}{Theorem}
\newtheorem*{lem}{Lemma}
\newtheorem*{cor}{Corollary}


\begin{document}
\begin{thm}
Every bounded, infinite set of real numbers has a limit point. 
\end{thm}
\begin{proof}
Let $S\subset\mathbb{R}$ be bounded and infinite. Since $S$ is bounded there exist $a,b\in\mathbb{R}$, with $a<b$, such that $S\subset[a,b]$. Let $b-a=l$ and denote the midpoint of the interval $[a,b]$ by $m$. Note that at least one of $[a,m],[m,b]$ must contain infinitely many points of $S$; select an interval satisfying this condition, denoting its left endpoint by $a_1$ and its right endpoint by $b_1$. Continuing this process inductively, for each $n\in\mathbb{N}$, we have an interval $[a_n,b_n]$ satisfying 
\begin{equation}
[a_n,b_n]\subset[a_{n-1},b_{n-1}]\subset\cdots\subset[a_1,b_1]\subset[a,b]\text{,}
\end{equation}
where, for each $i\in\mathbb{N}$ such that $1\leq i\leq n$, the interval $[a_i,b_i]$ contains infinitely many points of $S$ and is of length $l/2^i$. Next we note that the set $A=\{a_1,a_2\ldots,a_n\}$ is contained in $[a,b]$, hence is bounded, and as such, has a supremum which we denote by $x$. Now, given $\epsilon>0$, there exists $N\in\mathbb{N}$ such that $x-\epsilon<a_N\leq x$. Furthermore, for every $m\geq N$, we have $x-\epsilon<a_N\leq a_m\leq x$. In particular, if we select $m\geq N$ such that $l/2^m<\epsilon$, then we have
\begin{equation}
x-\epsilon<a_n\leq a_m\leq x\leq b_m=a_m+\dfrac{l}{2^m}<x+\epsilon\text{.}
\end{equation}
Since $[a_m,b_m]\subset(x-\epsilon,x+\epsilon)$ contains infinitely many points of $S$, we may conclude that $x$ is a limit point of $S$. 
\end{proof}
%%%%%
%%%%%
\end{document}
