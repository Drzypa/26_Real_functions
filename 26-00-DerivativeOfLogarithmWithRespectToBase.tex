\documentclass[12pt]{article}
\usepackage{pmmeta}
\pmcanonicalname{DerivativeOfLogarithmWithRespectToBase}
\pmcreated{2013-03-22 19:11:28}
\pmmodified{2013-03-22 19:11:28}
\pmowner{pahio}{2872}
\pmmodifier{pahio}{2872}
\pmtitle{derivative of logarithm with respect to base}
\pmrecord{4}{42102}
\pmprivacy{1}
\pmauthor{pahio}{2872}
\pmtype{Definition}
\pmcomment{trigger rebuild}
\pmclassification{msc}{26-00}
\pmclassification{msc}{26A09}
\pmclassification{msc}{26A06}
\pmrelated{Derivative}

% this is the default PlanetMath preamble.  as your knowledge
% of TeX increases, you will probably want to edit this, but
% it should be fine as is for beginners.

% almost certainly you want these
\usepackage{amssymb}
\usepackage{amsmath}
\usepackage{amsfonts}

% used for TeXing text within eps files
%\usepackage{psfrag}
% need this for including graphics (\includegraphics)
%\usepackage{graphicx}
% for neatly defining theorems and propositions
 \usepackage{amsthm}
% making logically defined graphics
%%%\usepackage{xypic}

% there are many more packages, add them here as you need them

% define commands here

\theoremstyle{definition}
\newtheorem*{thmplain}{Theorem}

\begin{document}
The \PMlinkescapetext{formula}
\begin{align}
\frac{\partial}{\partial a}\log_ax \;=\; -\frac{\ln{x}}{(\ln{a})^2a}
\end{align}
for the partial derivative of logarithm expression with respect to the base $a$ may be derived by denoting first
$$\log_ax \;=\; y.$$
By the definition of logarithm, this equation means the same as
$$a^y \;=\; x,$$
where we can take the natural logarithms
$$y\ln{a} \;=\; \ln{x}$$
solving then
$$y \;=\; \frac{\ln{x}}{\ln{a}}.$$
Then, the differentiation is easy:
$$\frac{\partial y}{\partial a} \;=\; \frac{0\ln{a}-\frac{1}{a}\ln{x}}{(\ln{a})^2} \;=\; -\frac{\ln{x}}{(\ln{a})^2a}.$$
%%%%%
%%%%%
\end{document}
