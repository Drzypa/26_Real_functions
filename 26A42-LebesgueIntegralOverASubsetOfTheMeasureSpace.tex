\documentclass[12pt]{article}
\usepackage{pmmeta}
\pmcanonicalname{LebesgueIntegralOverASubsetOfTheMeasureSpace}
\pmcreated{2013-03-22 16:13:54}
\pmmodified{2013-03-22 16:13:54}
\pmowner{Wkbj79}{1863}
\pmmodifier{Wkbj79}{1863}
\pmtitle{Lebesgue integral over a subset of the measure space}
\pmrecord{7}{38332}
\pmprivacy{1}
\pmauthor{Wkbj79}{1863}
\pmtype{Definition}
\pmcomment{trigger rebuild}
\pmclassification{msc}{26A42}
\pmclassification{msc}{28A25}

\endmetadata

\usepackage{amssymb}
\usepackage{amsmath}
\usepackage{amsfonts}

\usepackage{psfrag}
\usepackage{graphicx}
\usepackage{amsthm}
%%\usepackage{xypic}

\begin{document}
\PMlinkescapeword{property}

Let $(X,\mathfrak{B},\mu)$ be a measure space and $A \in \mathfrak{B}$.

Let $s \colon X \to [0,\infty]$ be a simple function.  Then $\displaystyle \int_A s \, d\mu$ is defined as $\displaystyle \int_A s \, d\mu := \int_X \chi_As \, d\mu$, where $\chi_A$ denotes the characteristic function of $A$.

Let $f \colon X \to [0,\infty]$ be a measurable function and \\ $S=\{s \colon X \to [0,\infty]~~|~~s \text{ is a simple function and } s \le f\}$.  Then $\displaystyle \int_A f \, d\mu$ is defined as $\displaystyle \int_A f \, d\mu := \sup_{s \in S} \int_A s \, d\mu$.

By the properties of the Lebesgue integral of nonnegative measurable functions (property 3), we have that $\displaystyle \int_A f \, d\mu=\int_X \chi_A f \, d\mu$.

Let $f \colon X \to [-\infty, \infty]$ be a measurable function such that not both of $\displaystyle \int_A f^+ \, d\mu$ and $\displaystyle \int_A f^- \, d\mu$ are infinite.  (Note that $f^+$ and $f^-$ are defined in the entry Lebesgue integral.)  Then $\displaystyle \int_A f \, d\mu$ is defined as $\displaystyle \int_A f \, d\mu := \int_A f^+ \, d\mu -\int_A f^- \, d\mu$.

By the properties of the Lebesgue integral of Lebesgue integrable functions (property 3), we have that $\displaystyle \int_A f \, d\mu=\int_X \chi_A f \, d\mu$.
%%%%%
%%%%%
\end{document}
