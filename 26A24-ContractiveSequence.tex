\documentclass[12pt]{article}
\usepackage{pmmeta}
\pmcanonicalname{ContractiveSequence}
\pmcreated{2014-11-30 16:45:13}
\pmmodified{2014-11-30 16:45:13}
\pmowner{pahio}{2872}
\pmmodifier{pahio}{2872}
\pmtitle{contractive sequence}
\pmrecord{13}{88151}
\pmprivacy{1}
\pmauthor{pahio}{2872}
\pmtype{Theorem}

% this is the default PlanetMath preamble.  as your knowledge
% of TeX increases, you will probably want to edit this, but
% it should be fine as is for beginners.

% almost certainly you want these
\usepackage{amssymb}
\usepackage{amsmath}
\usepackage{amsfonts}

% need this for including graphics (\includegraphics)
\usepackage{graphicx}
% for neatly defining theorems and propositions
\usepackage{amsthm}

% making logically defined graphics
%\usepackage{xypic}
% used for TeXing text within eps files
%\usepackage{psfrag}

% there are many more packages, add them here as you need them

% define commands here

\begin{document}
The sequence 
\begin{align}
a_0,a_1,a_2, \ldots
\end{align}
in a metric space $(X, d)$ is called {\it contractive}, iff there is a real number 
$r \in (0,1)$ such that for any positive integer $n$ the inequality
\begin{align}
d(a_n, a_{n+1}) \leqq r\!\cdot\!d(a_{n-1}, a_n)
\end{align}
is true.    \\

We will prove the 

\textbf{Theorem.}\, If the sequence (1) is contractive, it is 
a Cauchy sequence.

{\it Proof.}\, Suppose that the sequence (1) is 
contractive.  Let $\varepsilon$ be an arbitrary positive number and 
$m, n$ some positive integers from which e.g. $n$ is greater than
$m,$ $n = m+\delta$.

Using repeatedly the triangle inequality we get
\begin{align*}
d(a_m, a_n) &\leqq d(a_m, a_{m+1})+d(a_{m+1}, a_{m+\delta})\\
            &\leqq d(a_m, a_{m+1})+d(a_{m+1}, a_{m+2})+d(a_{m+2}, a_{m+\delta})\\
            & \ldots\\
            &\leqq d(a_m, a_{m+1})+d(a_{m+1}, a_{m+2})+
            d(a_{m+2}, a_{m+3})+\ldots+d(a_{n-1}, a_n).           

\end{align*}

Now the contractiveness gives the inequalities
$$d(a_1, a_2) \leqq rd(a_0, a_1),$$
$$d(a_2, a_3) \leqq rd(a_1, a_2) \leqq r^2d(a_0, a_1),$$
$$d(a_3, a_4) \leqq rd(a_2, a_3) \leqq r^3d(a_0, a_1),$$
$$\ldots$$
$$d(a_m, a_{m+1}) \leqq r^md(a_0, a_1),$$
$$\ldots$$
$$d(a_{n-1}, a_n}) \leqq r^{n-1}d(a_0, a_1),$$
by which we obtain the estimation
\begin{align*}
d(a_m, a_n) &\leqq d(a_0, a_1)(r^m+r^{m+1}+\ldots+r^{m+\delta-1})\\
  & = d(a_0, a_1)r^m(1+r+r^2+\ldots+r^{\delta-1})\\
  & = d(a_0, a_1)r^m\frac{1-r^\delta}{1-r}\\
  & < d(a_0, a_1)\frac{r^m}{1-r}.
\end{align*}
The last expression tends to zero as $m \to \infty$.\, Thus 
there exists a positive number $M$ such that  
$$d(a_m, a_n) < \varepsilon \mbox{  for each } m > M$$
when $n > m$.\, Consequently, (1) is a Cauchy sequence.\\

\textbf{Remark.}\; The assertion of the Theorem cannot be 
reversed.  E.g. in the usual metric of $\mathbb{R}$, the 
sequence\, $1,\frac{1}{2},\frac{1}{3},\ldots$\, converges to 0 
and hence is Cauchy, but for it the ratio
$$|a_n-a_{n+1}|:|a_{n-1}-a_n}| \;=\; 1-\frac{2}{n+1}$$
tends to 1 as\, $n \to \infty$.\\

Cf. \PMlinkname{sequences of bounded variation}{SequenceOfBoundedVariation}.


\begin{thebibliography}{8}
\bibitem{loya}{\sc Paul Loya}: {\it Amazing and Aesthetic
Aspects of Analysis: On the incredible infinite}.\, A Course in Undergraduate Analysis, Fall 2006.\; 
Available in http://www.math.binghamton.edu/dennis/478.f07/EleAna.pdf
\end{thebibliography} 

\end{document}
